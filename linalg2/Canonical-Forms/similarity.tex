\begin{defn*}

    Two matrices $M_1,M_2\in M_n(\bF)$ are \ppemph{similar} if they are representations of the same linear operator.
    In other words, $M_1$ and $M_2$ are similar if there exists some linear operator $T$, and bases $B$ and $C$ such that
    \[ M_1 = [T]_B \text{ and } M_2 = [T]_C \]
    Similarity is denoted by $M_1\sim M_2$.

\end{defn*}

Similarity is an equivalence relation, as it is obviously reflexive and symmetric.
But it is not immediate that it is transitive, as if $M_1\sim M_2$ and $M_2\sim M_3$, then all we know is that there are linear operators $T$ and $S$, and bases $B$, $C$, $D$, and $E$ such that
\[ M_1 = [T]_B,\quad M_2 = [T]_C,\quad M_2=[S]_D,\quad M_3=[S]_E \]
(By our definition, $T$ and $S$ need not even be linear operators over the same vector space, but since all the vector spaces have the same dimension, they are all isomorphic.)
We will prove transitivity another way, by finding an equivalent condition for similarity.

\begin{lemm*}

    If $A\in M_n(\bF)$ is invertible and $B$ is a basis of some $n$-dimensional vector space $V$, then there exists some bases $C$ and $D$ such that $A=[I]^C_B$ and $A=[I]^B_D$.

\end{lemm*}

\begin{proof}

    Suppose $B=(v_1,\dots,v_n)$, then let us denote $C=(u_1,\dots,u_n)$.
    We must have that $[u_i]_B=C_i(A)$, and since $[\cdot]_B$ is an isomorphism, there exists such a $u_i$ (defined by $(v_1,\dots,v_n)\cdot C_i(A)$).
    And since the $A$ is invertible, $\set{C_1(A),\dots,C_n(A)}$ are linearly independent and so the $u_i$ are also linearly independent, and therefore form a basis (since they define a linearly independent
    set whose size is the dimension of $V$).

    Now, we know that this must also be true for $A^{-1}$ and so there exists a basis $D$ such that $A^{-1}=[I]^D_B$ and so $A=[I]^B_D$ as required.
    \qed

\end{proof}

\begin{thrm*}

    $M_1$ and $M_2$ are similar if and only if there exists an invertible matrix $P$ such that $M_1=P^{-1}M_2P$.

\end{thrm*}

\begin{proof}

    Suppose $M_1$ and $M_2$ are similar, and so there exists a linear operator $T$ and bases $B$ and $C$ such that $M_1=[T]_B$ and $M_2=[T]_C$.
    But then
    \[ M_1 = [I]^C_B\cdot[T]_C\cdot[I]^B_C \]
    so if we defined $P=[I]^B_C$, then $P$ is invertible and $P^{-1}=[I]^C_B$ so
    \[ M_1 = P^{-1}M_2P \]
    as required.

    And if $M_1=P^{-1}M_2P$, then since $M_2$ represents a linear operator, suppose $M_2=[T]^B_B$.
    Then by the previous lemma, we know that there exists some basis $C$ such that $P=[I]^C_B$, and so
    \[ M_1 = [I]^B_C\cdot[T]^B_B\cdot[I]^C_B = [T]^C_C \]
    and therefore $M_1$ and $M_2$ both represent the same linear operator, and are therefore similar.
    \qed

\end{proof}

Thus similarity is transitive, since if $M_1\sim M_2$ and $M_2\sim M_3$ then there exist invertible matrices $P$ and $Q$ such that
\[ M_1 = P^{-1}M_2P,\quad M_2 = Q^{-1}M_3Q \implies M_1 = P^{-1}Q^{-1}M_3QP = (QP)^{-1}M_3(QP) \]
so $M_1\sim M_3$ as required.
So similarity is indeed an equivalence relation.

\begin{prop*}

    Similar matrices have the same characteristic polynomial.

\end{prop*}

\begin{proof}

    Suppose $A\sim B$, then $A=P^{-1}BP$ for some invertible matrix $P$.
    Then
    \[ p_A(x) = \detof{xI-A} = \detof{xI-P^{-1}BP} = \detof{P^{-1}(xI-B)P} = \detof P\cdot\detof{xI-B}\cdot\detof{P^{-1}} \]
    And since $\detof P\cdot\detof{P^{-1}}=1$, this is equal to
    \[ = \detof{xI-B} = p_B(x) \qed \]

\end{proof}

This should make sense, as two similar matrices should have the same eigenvalues as they represent the same linear transformation, which form the roots of their characteristic polynomials.
And indeed two similar matrices have the same eigenvalues, as their characteristic polynomials are equivalent.

\begin{prop*}

    Two similar matrices have the same determinants and traces.

\end{prop*}

\begin{proof}

    We can show this two ways.
    Firstly, we showed that the traces and determinant are some of the coefficients of the characteristic polynomial of a matrix.
    And since two similar matrices have the same characteristic polynomial, their determinants and traces are the same.

    We can also prove this more directly.
    Suppose $A$ and $B$ are similar, then $A=P^{-1}BP$ and so
    \[ \detof A = \detof{P^{-1}BP} = \detof{P^{-1}}\detof B\detof P = \detof B \]
    And recall that $\traceof{AB}=\traceof{BA}$ and so
    \[ \traceof A = \traceof{P^{-1}BP} = \traceof{P^{-1}(BP)} = \traceof{BPP^{-1}} = \traceof B \]
    as required.
    \qed

\end{proof}

Now, since similar matrices have the same characteristic polynomial, determinant, and trace, we can define the characteristic polynomial of a linear operator.

\begin{defn*}

    If $T$ is a linear operator, then we define the \ppemph{characteristic polynomial} of $T$ to be the characteristic polynomial of any of its matrix representations.
    And we similarly define its \ppemph{determinant} and \ppemph{trace} to be the determinant and trace of any of its matrix representations, respectively.

\end{defn*}

Since by definition all of $T$'s matrix representations are similar, they all have the same characteristic polynomial, determinant, and trace and so this definition is well-defined.

