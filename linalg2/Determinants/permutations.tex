\begin{defn*}

    A \ppemph{permutation} of $n$ elements is a bijection $\sigma\colon\set{1,\dots,n}\longto\set{1,\dots,n}$.
    $S_n$ is defined to be the set of all permutations of $n$ elements.

\end{defn*}

Recall that there are $n!$ bijections between two sets of cardinality $n$, thus $\abs{S_n}=n!$.

One common way of writing permutations is in table form, for example
\[ \begin{pmatrix} 1 & 2 & 3 & 4 & 5 \\ 4 & 3 & 1 & 5 & 2 \end{pmatrix} \]
corresponds to a permutation
\[ \begin{tikzcd}[column sep=small]
    1 & 2 & 3 & 4 & 5 \\ 4 & 3 & 1 & 5 & 2
    \arrow[maps to, from=1-1, to=2-1]
    \arrow[maps to, from=1-2, to=2-2]
    \arrow[maps to, from=1-3, to=2-3]
    \arrow[maps to, from=1-4, to=2-4]
    \arrow[maps to, from=1-5, to=2-5]
\end{tikzcd} \]

Now, let us chase elements, starting from $1$.
$1$ is mapped to $4$ which is mapped to $5$ which is mapped to $2$ which is mapped $3$ which is mapped to $1$.
This creates a cycle:

\begin{defn*}

    A \ppemph{cycle} is a permutation $\sigma$ such that there exist $n_1,\dots,n_k$ where $\sigma(n_i)=n_{i+1}$ for $i<k$ and $\sigma(n_k)=n_1$.
    And for every number $m$ which is not equal to some $n_k$, $\sigma(m)=m$.
    Cycles are denoted
    \[ \sigma = \pmat{n_1 & n_2 & \dots & n_k} \]

    A cycle whose length is $2$ is called a \ppemph{transposition}.

\end{defn*}

Thus our $\sigma$ can be written as the cycle
\[ \sigma = \pmat{1 & 4 & 5 & 2 & 3} \]

Notice that if $\sigma=\pmat{n_1&\dots&n_k}$ is a cycle, then so is $\sigma^{-1}$:
\[ \sigma^{-1} = \pmat{n_k&n_{k-1}&\dots&n_1} \]
This is since $\sigma^{-1}(n_i)=n_{i-1}$ for $i>1$ and $\sigma^{-1}(n_1)=n_k$.

\begin{note}

    Given two permutations $\sigma$ and $\tau$, we will denote the composition by the product $\sigma\tau$, neglecting the composition operator $\circ$.

\end{note}

\begin{defn*}

    The \ppemph{support} of a permutation $\sigma$ is the set of all non-invariant numbers, ie.
    \[ \supportof\sigma = \set{k}[\sigma(k)\neq k] \]
    So for example, for cycles
    \[ \supportof{\pmat{n_1 & n_2 & \dots & n_k}} = \set{n_1,\,\dots,\,n_k} \]
    Two permutations are called \ppemph{disjoint} if their supports are.

\end{defn*}

Note that a permutation $\sigma\in S_n$ can be thought of as a permutation of the elements in its support.
In other words, there is a natural equivalence between $\sigma$ and a permutation in $S_{\abs{\supportof\sigma}}$.

\begin{prop*}

    If $\sigma$ is a permutation then $\supportof\sigma=\supportof{\sigma^{-1}}$.

\end{prop*}

\begin{proof}

    If $k\in\supportof\sigma$ then $\sigma(k)\neq k$, which means $\sigma^{-1}(\sigma(k))\neq\sigma^{-1}(k)$, and so $\sigma^{-1}(k)\neq k$.
    Therefore $k\in\supportof{\sigma^{-1}}$, and so $\supportof\sigma\subseteq\supportof{\sigma^{-1}}$.
    By symmetry, we have that $\supportof\sigma\supseteq\supportof{\sigma^{-1}}$, and therefore $\supportof\sigma=\supportof{\sigma^{-1}}$ as required.
    \qed

\end{proof}

\begin{prop*}

    If $\sigma$ and $\tau$ are two disjoint permutations, then $\sigma\tau=\tau\sigma$.

\end{prop*}

\begin{proof}

    Suppose $\sigma$ and $\tau$ are disjoint, then let $k$ be some number.
    It is either in the support of $\sigma$ or in the support of $\tau$, or in neither support.
    If it is in neither support then $\sigma(k)=\tau(k)=k$, and so
    \[ \sigma\tau(k) = \tau\sigma(k) = k \]
    Otherwise, suppose without loss of generality that $k\in\supportof\sigma$.
    Then $\sigma(k)\neq k$ so $\sigma(\sigma(k))\neq\sigma(k)$ since $\sigma$ is injective, and therefore $\sigma(k)\in\supportof\sigma$.
    Since $\sigma$ and $\tau$ are disjoint, this means $\sigma(k)\notin\supportof\tau$ and so $\tau(\sigma(k))=\sigma(k)$.
    So
    \[ \tau\sigma(k) = \tau(\sigma(k)) = \sigma(k) \]
    and on the other hand, since $k\notin\supportof\tau$,
    \[ \sigma\tau(k) = \sigma(k) \]
    so $\tau\sigma(k)=\sigma\tau(k)$.

    Since this is true for every $k$, we have that $\sigma\tau=\tau\sigma$, as required.
    \qed

\end{proof}

\begin{prop*}

    If $\sigma$ and $\tau$ are permutations then $\supportof{\sigma\tau}\subseteq\supportof\sigma\cup\supportof\tau$.
    If $\sigma$ and $\tau$ are disjoint, this is an equality.

\end{prop*}

\begin{proof}

    Suppose $i\in\supportof{\sigma\tau}$, and suppose that $i\notin\supportof\sigma\cup\supportof\tau$.
    Thus $i\notin\supportof\sigma$ and $i\notin\supportof\tau$ so $\sigma(i)=\tau(i)=i$, so $\sigma\tau(i)=i$ which is a contradiction.

    If $\sigma$ and $\tau$ are disjoint then if $i\in\supportof\sigma\cup\supportof\tau$, then suppose $i\in\supportof\sigma$.
    Then $i\notin\supportof\tau$, so $\sigma\tau(i)=\sigma(i)\neq i$, so $i\in\supportof{\sigma\tau}$ as required.
    \qed

\end{proof}

Let us look at another permutation
\[ \sigma=\begin{pmatrix} 1 & 2 & 3 & 4 & 5 \\ 4 & 1 & 5 & 2 & 3 \end{pmatrix} \]
Doing some chasing gives us
\[ 1\varmapsto4\varmapsto2\varmapsto1 \]
which is a cycle, but does not fully cover all of $\sigma$'s support, as we are missing $3$ and $5$.
Chasing these numbers gives the cycle
\[ 3\varmapsto5\varmapsto3 \]
Therefore
\[ \sigma=\pmat{1&4&2}\pmat{3&5} \]
As you can see, we can write $\sigma$ as the product of two disjoint cycles.
This turns out to be true in general.

\begin{thrm*}

    Every permutation $\sigma\in S_n$ can be factorized as the product of disjoint cycles in $S_n$, and this factorization is unique (up to order).

\end{thrm*}

\begin{proof}

    Firstly, let us show that the product of disjoint cycles is unique.
    Suppose $\sigma_1,\dots,\sigma_k$ are all disjoint and $\tau_1,\dots,\tau_\ell$ are also all disjoint such that
    \[ \sigma_1\cdots\sigma_k = \tau_1\cdots\tau_\ell \]

    Suppose $\sigma_1=\pmat{n_1&\dots&n_t}$, then $n_1$ must be in the support of some $\tau_i$, so without loss of generality suppose $n_1\in\supportof{\tau_1}$ (no generality is lost because the product
    of disjoint permutations is commutative).
    Then $n_2=\sigma_1(n_1)=\tau_1(n_1)$, and so $n_2\in\supportof{\tau_1^{-1}}=\supportof\tau$, meaning $n_3=\sigma_1(n_2)=\tau_1(n_2)$ as well, and so on.
    Therefore
    \[ n_{i+1} = \sigma_1(n_i) = \tau_1(n_i) \]
    And $\tau_1(n_t)=\sigma_1(n_t)=n_1$, therefore $\tau_1=\pmat{n_1&\dots&n_t}=\sigma_1$.
    Thus we have that
    \[ \sigma_1\cdots\sigma_k=\sigma_1\cdot\tau_2\cdots\tau_\ell \]
    therefore
    \[ \sigma_2\cdots\sigma_k = \tau_2\cdots\tau_\ell \]
    We can continue inductively, and showing that $\sigma_2=\tau_2$ and so on.
    Thus the factorization of a permutation by disjoint cycles is unique, if it exists.

    Now we will show that such a factorization exists.
    Suppose $\sigma$ is a permutation, then if $\supportof\sigma=\varnothing$, $\sigma=\id$ and we have finished (since $\id$ is equal to the empty product of cycles).
    Otherwise, let $i\in\supportof\sigma$, then there exists some $n>0$ such that $\sigma^n(i)=i$.
    This is because otherwise the map $n\varmapsto\sigma^n(i)$ would be injective: if $\sigma^n(i)=\sigma^m(i)$ for $n>m$ then $\sigma^{n-m}(i)=i$ in contradiction, so the map is injective.
    But this is an injection from $\bN$ to $\set{1,\dots,n}$ which is a contradiction.

    So let $k$ be the minimum positive value such that $\sigma^k(i)=i$, since $i\in\supportof\sigma$, $i>1$.
    Now, let
    \[ \tau = \pmat{i&\sigma(i)&\dots&\sigma^{k-1}(i)} \]
    then $\tau$ is a cycle.
    Notice that for $j<k$, $\tau^{-1}\sigma(\sigma^j(i))=\sigma^j(i)$, and so $\sigma^j(i)\notin\supportof{\tau^{-1}\sigma}$.
    Since
    \[ \supportof{\tau^{-1}\sigma}\subseteq\supportof\tau\cup\supportof\sigma=\supportof\sigma \]
    So we have that
    \[ \supportof{\tau^{-1}\sigma}\subseteq\supportof\sigma\setminus\set{i,\sigma(i),\dots,\sigma^{k-1}(i)} \]
    So $\tau^{-1}\sigma$ and $\tau$ are disjoint.

    So let us induct over $n$ (in $S_n$).
    For $n=1$, $\sigma=\id$ which is trivial.
    Otherwise, we showed that there exists a cycle $\tau$ such that $\tau^{-1}\sigma$ has a strictly smaller support than $\sigma$.
    Thus $\tau^{-1}\sigma$ is essentially a permutation in $S_k$ for $k<n$, and by our inductive hypothesis, we have that
    \[ \tau^{-1}\sigma = \sigma_1\cdots\sigma_k \]
    where $\sigma_i$ are all disjoint cycles.
    Now since $\tau^{-1}\sigma$ and $\tau$ are disjoint, we must have that $\sigma_i$ and $\tau$ are disjoint, and so
    \[ \sigma = \tau\sigma_1\cdots\sigma_k \]
    and this is a product of disjoint cycles.
    \qed

\end{proof}

This is a very important theorem, as it greatly reduces the amount of work we need to do in order to work with permutations.

\begin{defn*}

    If $\sigma$ is a permutation, an \ppemph{inversion} is a pair of indexes $(i,j)$ such that $i<j$ but $\sigma(i)>\sigma(j)$ (ie. an inversion is when $\sigma$ flips the order of two numbers).
    Let $N(\sigma)$ equal the number of inversions of $\sigma$,
    \[ N(\sigma) = \abs{\set{(i,j)}[i<j,\,\sigma(i)>\sigma(j)]} \]
    Let the \ppemph{sign} of the permutation be
    \[ \signof\sigma = (-1)^{N(\sigma)} \]

    If $\signof\sigma=1$, then $\sigma$ is caled \ppemph{even} (since the number of inversions is even).
    And if $\signof\sigma=-1$, then $\sigma$ is called \ppemph{odd}.

\end{defn*}

So for example, given an transposition $\sigma=\pmat{i&j}$, then we can suppose $i<j$ (since $\pmat{i&j}=\pmat{j&i}$), and so $N(\sigma)=1$ since $i$ is mapped to $j$ and $j$ is mapped to $i$, so
$i<j$ but $\sigma(i)>\sigma(j)$.

\begin{prop*}

    If $\sigma$ and $\tau$ are two permutations (not necessarily disjoint) then $\signof{\sigma\tau}=\signof\sigma\cdot\signof\tau$.

\end{prop*}

\begin{proof}

    Let $i<j$, then suppose $\set{i,j}$ is an inversion of $\sigma\tau$.
    If $\set{i,j}$ is an inversion of $\tau$ then $\set{\tau(i),\tau(j)}$ cannot be an inversion of $\sigma$.
    And otherwise if $\set{i,j}$ is not an inversion of $\tau$ then $\set{\tau(i),\tau(j)}$ must be an inversion of $\sigma$.
    So we have that $\set{i,j}$ is an inversion of $\sigma\tau$ if and only if
    \blist
        \item $\set{i,j}$ is an inversion of $\tau$ and $\set{\tau(i),\tau(j)}$ is not an inversion of $\sigma$, or
        \item $\set{i,j}$ is not an inversion of $\tau$ and $\set{\tau(i),\tau(j)}$ is an inversion of $\sigma$.
    \elist

    Let $N$ be the number of inversions $\set{i,j}$ of $\tau$ such that $\set{\tau(i),\tau(j)}$ is also an inversion of $\sigma$.
    We claim that
    \[ N(\sigma\tau) = N(\sigma) + N(\tau) - 2N \]
    This is because $N(\sigma)$ is equal to the number of pairs $\set{i,j}$ where
    \blist
        \item $\set{i,j}$ is not an inversion of $\tau$ and $\set{\tau(i),\tau(j)}$ is an inversion of $\sigma$, or
        \item $\set{i,j}$ is an inversion of $\tau$ and $\set{\tau(i),\tau(j)}$ is an inversion of $\sigma$.
    \elist
    And similarly $N(\tau)$ is equal to the number of pairs $\set{i,j}$ where
    \blist
        \item $\set{i,j}$ is an inversion of $\tau$ and $\set{\tau(i),\tau(j)}$ is not an inversion of $\sigma$, or
        \item $\set{i,j}$ is an inversion of $\tau$ and $\set{\tau(i),\tau(j)}$ is an inversion of $\sigma$.
    \elist

    Thus when adding $N(\sigma)+N(\tau)$ we double count the number of pairs where $\set{i,j}$ is an inversion of $\tau$ and $\set{\tau(i),\tau(j)}$ is an inversion of $\sigma$, of which there are $N$.
    The rest of the cases are counted by $N(\sigma\tau)$, so
    \[ N(\sigma) + N(\tau) = N(\sigma\tau) + 2N \]
    Thus
    \[ \signof{\sigma\tau} = (-1)^{N(\sigma\tau)} = (-1)^{N(\sigma)+N(\tau)-2N} = (-1)^{N(\sigma)}(-1)^{N(\tau)} = \signof\sigma\cdot\signof\tau \]
    as required.
    \qed

\end{proof}

\begin{prop*}

    If $\sigma$ is a cycle of length $k$, then
    \[ \signof\sigma = (-1)^{k-1} \]

\end{prop*}

\begin{proof}

    Suppose $\sigma=\pmat{n_1&\dots&n_k}$, then let
    \[ \tau = \pmat{n_1&n_2}\cdot\pmat{n_2&n_3}\cdots\pmat{n_{k-1}&n_k} \]
    Notice that for $i<k$, $n_i$ is transposed with $n_{i+1}$ and then $n_{i+1}$ is not transposed again (since the only other transpositions with $n_{i+1}$ is to the right of $\pmat{n_i&n_{i+1}}$).
    So $\tau(n_i)=n_{i+1}$.
    And for $i=k$, then $n_k$ is transposed with $n_{k-1}$ which is transposed with $n_{k-2}$ and so on until $n_1$, so $\tau(n_k)=n_1$.
    Thus $\tau=\sigma$.

    Since $\sign$ is multiplicative, we have that
    \[ \signof\sigma = \prod_{i=1}^{k-1}\signof{\pmat{n_i&n_{i+1}}} = (-1)^{k-1} \]
    because the sign of a transposition is $-1$.
    \qed

\end{proof}

This provides a much easier method of computing the sign of a permutation.
First, decompose it into cycles, then find the sign of each of the cycles (which is simple, by the above proposition), and multiply them together.
For example, let us take a look at
\[ \sigma = \pmat{1&2&3&4&5&6&7&8&9&10\\6&5&9&10&1&2&3&5&7&4} \]
Chasing numbers we get
\[ 1\mapsto6\mapsto2\mapsto5\mapsto1,\quad 3\mapsto9\mapsto7\mapsto3,\quad 4\mapsto10\mapsto4 \]
And thus we get that
\[ \sigma=\pmat{1&6&2&5}\pmat{3&9&7}\pmat{4&10} \]
and so
\[ \signof\sigma=(-1)^3(-1)^2(-1)^1=1 \]
In other words, $\sigma$ is even.

\begin{prop*}

    For $n>1$, there are as many even permutations as there are odd permutations in $S_n$.

\end{prop*}

\begin{proof}

    Since $n>1$, we have the transposition $\pmat{1&2}$ in $S_n$.
    If $\sigma$ is even, then $\sigma\cdot\pmat{1&2}$ is odd (since $\signof{\sigma\cdot\pmat{1&2}}=\signof\sigma\cdot\signof{\pmat{1&2}}=-1$).
    And if $\sigma$ is odd, then $\sigma\cdot\pmat{1&2}$ is even.
    And so the map $\sigma\varmapsto\sigma\cdot\pmat{1&2}$ maps even permutations to odd permutations, and odd permutations to even permutations.
    Since the map is injective, this means that there must be at least as many even permutations as odd permutatations, and vice versa, so there is the same amount.
    \qed

\end{proof}


