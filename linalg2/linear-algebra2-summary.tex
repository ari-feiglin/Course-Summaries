\documentclass[10pt]{article}

\usepackage{amsmath, amssymb, mathtools}
\usepackage{mathdots}
\usepackage[margin=1.5cm]{geometry}
\usepackage{tikz-cd}

\input pdfmsym
\input prettyprint
\input preamble

\pdfmsymsetscalefactor{10}
\initpps
\@Arrow@def{varLeftRightarrow}\@Larrow\@Rarrow{1}
\def\implies{\,\longvarRightarrow\,}
\def\iff{\,\longvarLeftRightarrow\,}
\let\eiff=\varLeftRightarrow
\let\to=\varrightarrow
\let\longto=\longvarrightarrow
\let\oto=\varleftrightarrow
\let\injection=\longvaruphookrightarrow
\let\mapsto=\varmapsto

\def\pmat#1{\begin{pmatrix} #1 \end{pmatrix}}

\let\divides=\mid
\newfunc{image}{{\rm Img}}({})
\newfunc{deg}{{\rm deg}}({})
\newfunc{diag}{{\rm diag}}({})
\newfunc{ker}{{\rm Ker}}({})
\newfunc{metric}\rho({})
\newfunc{metricc}\sigma({})
\newfunc{spa}{{\rm span}}(\vert)
\newfunc{diam}{{\rm diam}}(\vert)
\newfunc{proj}\pi({})
\newfunc{iproj}{\pi^{-1}}({})
\newfunc{cis}{{\rm cis}}({})
\newfunc{Re}{{\rm Re}}({})
\newfunc{Im}{{\rm Im}}({})
\newfunc{sup}{{\rm sup}}\{\vert\}
\newfunc{Res}{{\rm Res}}({})
\newfunc{lspan}{{\rm span}}({})
\newfunc{support}{{\rm supp}}({})
\newfunc{sign}{{\rm sgn}}({})
\newfunc{det}{{\rm det}}({})
\newfunc{rref}{{\rm RREF}}({})
\newfunc{adj}{{\rm adj}}({})
\newfunc{trace}{{\rm trace}}({})
\newfunc{spec}{{\rm spec}}({})
\newfunc{dim}{{\rm dim}}({})

\def\id{\mathrm{id}}
\def\hort{\vcenter{\hrule width15pt height.3pt}}

\font\bigbf = cmbx12 scaled 2000
\@undervecc@def{underbar}\@linecap\@linecap

\def\pmat#1{\begin{pmatrix}#1\end{pmatrix}}

\def\mO{{\cal O}}
\def\mU{{\cal U}}
\let\lineseg=\overleftrightvecc
\let\to=\varrightarrow
\let\longto=\longvarrightarrow
\let\ds=\displaystyle

\def\pdv#1#2{\frac{\partial #1}{\partial #2}}

\def\differ#1#2{\left.d#1\strut\right|_{#2}}

\def\qed{%
    \ifmmode%
        \eqno\blacksquare%
    \else%
        \hskip1cm\allowbreak\hbox{}\nobreak\hfill$\blacksquare$%
    \fi%
}

\begin{document}

\barcolorbox{255, 200, 140}{200, 80, 0}{250, 100, 0}{
    \leftskip=0pt plus 1fill \rightskip=\leftskip
    {\bigbf Linear Algebra 2}

    \medskip
    \textit{Ari Feiglin}
}

\bigskip

For other summaries, check out \textcolor{blue}{\url{https://github.com/ari-feiglin/Course-Summaries.git}}.

\tableofcontents

\newpage

Before we begin, let me define some common terminology.
You should be familiar with these concepts, but you may not be familiar with their english names.

\begin{defn*}

    A function $f\colon A\longto B$ is called
    \blist
        \item \ppemph{injective} if $f(a)=f(b)$ means $a=b$.
        This is also called \ppemph{one-to-one}.
        \item \ppemph{surjective} if $f(A)=B$.
        This is also called \ppemph{onto}.
        \item \ppemph{bijective} if $f$ is both injective and surjective.
    \elist

\end{defn*}

\section{Determinants}

\subsection{Permutations}

\input Determinants/permutations

\subsection{The Determinant}

\input Determinants/the-determinant

\newpage
\section{Eigenvectors and Eigenvalues}

\input Eigenvectors-and-Eigenvalues/eigenvectors-and-eigenvalues

\newpage
\section{Canonical Forms}

\subsection{Similarity}

\input Canonical-Forms/similarity

\subsection{Diagonalization}

\input Canonical-Forms/diagonalization

\subsection{Triangularization}

\input Canonical-Forms/triangularization

\subsection{Zeroing Polynomials}

\input Canonical-Forms/zeroing-polynomials

\subsection{Invariant Subspaces}

\input Canonical-Forms/invariant-subspaces

\subsection{The Jordan Normal Form}

\input Canonical-Forms/jordan-normal-form

\end{document}

