\documentclass[10pt]{article}

\usepackage{amsmath, amssymb, mathtools}
\usepackage[margin=1.5cm]{geometry}

\font\tensy=cmsy10
\newfam\scrfam
\textfont\scrfam=\tensy
\def\mathscr#1{{\fam=\scrfam#1}}

\input pdfmsym
\input prettyprint
\input preamble

\pdfmsymsetscalefactor{10}
\initpps
\newmathpp{axiom}{Axiom}{255,255,200}{150,100,0}{200,150,100}			% yellow

\@Arrow@def{varLeftRightarrow}\@Larrow\@Rarrow{1}
\def\implies{\,\longvarRightarrow\,}
\def\iff{\,\longvarLeftRightarrow\,}
\let\eiff=\varLeftRightarrow
\let\to=\varrightarrow
\let\longto=\longvarrightarrow
\let\oto=\varleftrightarrow
\let\injection=\longvaruphookrightarrow

\let\nor=\downarrow
\let\nand=\uparrow

\def\aone#1#2{#2\to(#1\to#2)} \def\Aone{\textbf{A1}}
\def\atwo#1#2#3{\bigl(#1\to(#2\to#3)\bigr)\to\bigl((#1\to#2)\to(#1\to#3)\bigr)} \def\Atwo{\textbf{A2}}
\def\athree#1#2{(\neg#2\to\neg#1)\to\bigl((\neg#2\to#1)\to#2\bigr)} \def\Athree{\textbf{A3}}
\def\afour#1#2#3{(\forall #2#1)\to#1\frac{#3}{#2}}  \def\Afour{\textbf{A4}}
\def\afive#1#2#3{\bigl(\forall #3(#1\to#2)\bigr)\to\bigl(#1\to(\forall#3#2)\bigr)} \def\Afive{\textbf{A5}}
\def\MP{\textit{MP}} \def\Gen{\textit{Gen}}

\def\beqq{\mathrel{\vbox{\hbox{$=$}\kern-\baselineskip\kern.3pt\hbox{$=$}}}}

\def\bB{\mathbb{B}}
\def\mA{\mathcal{A}}
\def\mL{\mathcal{L}}
\def\mT{\mathcal{T}}
\def\mM{\mathcal{M}}
\def\mD{\mathcal{D}}
\def\fM{\mathfrak{M}}
\def\fC{\mathfrak{C}}
\def\fm{\mathfrak{m}}
\def\fn{\mathfrak{n}}

\def\Var{\mathrm{Var}}

\newfunc{var}{{\rm var}}({})
\newfunc{bound}{{\rm bound}}({})
\newfunc{free}{{\rm free}}({})

\def\qed{\hskip.5cm\hbox{}\hfill$\blacksquare$}

\def\pmat#1{\begin{pmatrix} #1 \end{pmatrix}}

\def\mB{\mathbb{B}}
\def\true{\mathsf{true}}
\def\false{\mathsf{false}}

%\def\Aone#1#2{#1\to(#2\to#1)}
%\def\Atwo#1#2#3{\bigl(#1\to(#2\to#3)\bigr)\to\bigl((#1\to#2)\to(#1\to#3)\bigr)}
%\def\Athree#1#2{(\neg#2\to\neg#1)\to\bigl((\neg#2\to#1)\to#2\bigr)}

\let\divides=\mid

\font\bigbf = cmbx12 scaled 2000

\parskip=5pt plus 1pt minus 3pt
\pp@headerskip=\z@

\begin{document}

\c@section=9

\barcolorbox{220, 255, 255}{0, 130, 130}{80, 200, 200}{
    \leftskip=0pt plus 1fill \rightskip=\leftskip
    {\bigbf Mathematical Logic}

    \medskip
    \textit{Lecture \thesection, Monday June 5, 2023}

    \textit{Ari Feiglin}
}

\bigskip

\begin{defn*}

    Let $\mM$ be some $\mL$-model whose domain is $D$, then if $\phi$ is a formula where $\free\phi\subseteq\set{x_1,\dots,x_n}$ then $\phi$ induces a function $D^n\to\bB$.
    Explicitly if $\mM\vDash\phi(a_1,\dots,a_n)$ then the value is $\true$ and otherwise is $\false$.
    Thus $\phi$ defines an $n$-ary predicate on $D$.
    Any such predicate is called \ppemph{definable}.
    The set of all values in $D^k$ for which $\mM\vDash\phi(a_1,\dots,a_n)$ are \ppemph{definable sets}.

\end{defn*}

Note that the union, intersection, and difference of two definable sets is also definable: if $A$ is defined by $\phi$ and $B$ by $\psi$ then $A\cup B$ is defined by $\phi\lor\psi$, $A\cap B$ by
$\phi\land\psi$, $A\setminus B$ by $\phi\land\neg\psi$.

\begin{exam*}

    If $\mL=\set{=,S}$ where $S$ is a unary function, let $\mM$ be a model whose domain is $\bN$ and whose interpretation of $S$ is incrementing by $1$.
    So we can define the predicate ``is equal to $0$'' by $\forall y\neg(x=S(y))$ ($0$ is the only constant not the successor of another number), and ``is greater than by $2$'' by $y=S(S(x))$.

\end{exam*}

Note that if a predicate $P$ is definable in $\mM$ then we can act as though $P$ is in the signature $\mL$, for $\mM$.
This is because we can simply replace its usage with its defining formula.

\begin{exam*}

    Formulas in the same signature but with different models may define predicates with very different meanings.
    Take for instance the signature $\mL=\set{+}$ and the model $\mM=\bN$ we can define $\leq$ by $\phi(x,y)=\exists z(x+z=y)$, but if $\mM=\bZ$ then $\phi$ simply defines $\bZ^2$ (every $x,y$ satisfies
    $\phi$).

    Thus when talking about definable predicates, it is important to know in which model you are.
    It is actually sufficient to know the structure (model without the valuation function) since the valuation function is only relevant for free variables in a formula, but for definable predicates/sets
    we are substituting all the free variables in a formula.

\end{exam*}

\begin{defn*}

    We focus for now on models of the signature $\mL=\set{+,\cdot,=}$ and the $\mL$-model $\bN$.
    Predicates definable by this model are called \ppemph{arithmetic}, and definable sets are \ppemph{arithmetic sets}.

\end{defn*}

As we showed above, $\leq$ is definable.
So is $x=0$ and $x=1$ ($\forall y(x\leq y)$ and $\forall y(y=0\lor x\leq y)$ respectively) and inductively for any fixed $n\in\bN$, $x=n$ is definable by
\[ \phi(x) = \forall y(y=0\lor y=1\lor\cdots\lor y=n-1\lor x\leq y) \]
Obviously so is $x\divides y$:
\[ \phi(x,y) = \exists z(x\cdot z=y) \]
and therefore so is ``$x$ is prime'':
\[ \phi(x) = \forall y(y\divides x\to(y=x\lor y=1)) \]
``$x$ is a power of two'' is arithmetic, since $x$ is a power of two if and only if every divisor is either $1$ or divisible by $2$:
\[ \phi(x) = \forall y(y\divides x\to(y=1\lor 2\divides x)) \]

\begin{defn*}

    Let $\mA$ be an $\mL$-structure (a model but without a valuation function, this can also be viewed as a model) whose domain is $A$.
    A bijection $\Phi\colon A\longto A$ is called a \ppemph{(structure/model) automorphism} if all predicates and functions in $\mA$'s interpretation are stable for $\Phi$.
    A $n$-ary predicate $P$ is \ppemph{stable} for $\Phi$ if
    \[ P(\Phi(x_1),\dots,\Phi(x_n)) \iff P(x_1,\dots,x_n) \]
    for all $x_1,\dots,x_n\in A$ and a $n$-ary function $f$ is stable if
    \[ f(\Phi(x_1),\dots,\Phi(x_n)) = \Phi\bigl(f(x_1),\dots,f(x_n)\bigr) \]

\end{defn*}

This generalizes the concept of group/ring/field automorphisms (this generalizes easily to isomorphisms, but the definition for homomorphisms is \emph{slightly} more complicated).
For $\vec a=(a_1,\dots,a_n)\in A^n$ we use the notation $\Phi(\vec a)$ to mean $(\Phi(a_1),\dots,\Phi(a_n))$.
So we can write that $P$ and $f$ are stable with respect to $\Phi$ if and only if
\[ P(\Phi(\vec x)) \iff P(\vec x)\quad f(\Phi(\vec x)) = \Phi(f(\vec x)) \]
respectively.

\begin{thrm*}

    Any predicate definable in a structure $\mA$ is stable with respect to all structure automorphisms.

\end{thrm*}

\begin{proof}

    We must show that $\mA\vDash\phi(\vec a)$ if and only if $\mA\vDash\phi(\Phi(\vec a))$.
    We begin by showing that for any term $t$ with $n$ variables and $\vec s\in A^n$ then
    \[ t(\Phi(\vec s)) = \Phi(t(\vec s)) \]
    We do this by term induction:
    \benum
        \item If $t=x$ ($t$ is a variable) then $t(\Phi(s))=\Phi(s)=\Phi(t(s))$ as required.
        \item If $t=f(t_1,\dots,t_n)$ then
        \[ t(\Phi(\vec s)) = f\bigl(t_1(\Phi(\vec s)),\dots,t_n(\Phi(\vec s))\bigr) = f\bigl(\Phi(t_1(\vec s)),\dots,\Phi(t_n(\vec s))\bigr) = \Phi\bigl(f(t_1(\vec s),\dots,t_n(\vec s))\bigr) =
        \Phi(t(\vec s)) \]
    \eenum

    Next we show $\mA\vDash\phi(\vec a)$ if and only if $\mA\vDash\phi(\Phi(\vec a))$ by formula induction:
    \benum
        \item If $\phi=P(t_1,\dots,t_n)$ then
        \begin{multline*}
            \mA\vDash\phi(\Phi(\vec a)) \iff P\bigl(t_1(\Phi(\vec a)),\dots,t_n(\Phi(\vec a))\bigr) \iff P\bigl(\Phi(t_1(\vec a)),\dots,\Phi(t_n(\vec a))\bigr) \\
            \iff P(t_1(\vec a),\dots,t_n(\vec a)) \iff \mA\vDash\phi(\vec a)
        \end{multline*}
        as required.

        \item If $\phi=\psi\land\mu$ then $\mA\vDash\phi(\Phi(\vec a))$ if and only if $\mA\vDash\psi(\Phi(\vec a))$ and $\mA\vDash\mu(\Phi(\vec a))$ if and only if $\mA\vDash\psi(\vec a)$ and
        $\mA\vDash\mu(\vec a)$.

        \item If $\phi=\neg\psi$ then $\mA\vDash\phi(\Phi(\vec a))$ if and only if $\mA\nvDash\psi(\Phi(\vec a))$ if and only if $\mA\nvDash\psi(\vec a)$ if and only if $\mA\vDash\phi(\vec a)$.

        \item If $\phi=\forall x\psi$ then $\mA\vDash\phi(\Phi(\vec a))$ then if $x$ is not one of the variables being substituted in $\phi$ then this is if and only if for every $b\in A$,
        $\mA\vDash\psi(\Phi(\vec a))\frac bx$ if and only if $\mA\vDash\psi\frac bx(\Phi(\vec a))$ which is if and only if $\mA\vDash\psi\frac bx(\vec a)$ by induction, and if and only if
        $\mA\vDash\psi(\vec a)\frac bx$ if and only if $\mA\vDash\phi(\vec a)$ as required.

        If $x$ is one of the variables being replaces then $\psi(\Phi(\vec a))\frac bx=\psi(\Phi(\vec a))$ so we get that $\mA\vDash\phi(\Phi(\vec a))$ if and only if $\mA\vDash\psi(\Phi(\vec a))$, and the
        rest continues as normal.
        \qed
    \eenum

\end{proof}

\begin{exam*}

    If we have the structure $(\bZ,=,<)$ then the predicate $x=0$ is not definable.
    This is because $x\varmapsto x+1$ is an automorphism, but $x=0$ is not stable under it (since if $x=0$ then $x+1\neq0$).

\end{exam*}

\begin{exam*}

    If we have the structure $(\bR,=,+,0,1)$, then the predicate $x=\gamma$ is definable for rational $\gamma$ but not for irrational $\gamma$.
    We know that $x=\frac ab$ if and only if $bx=a$ and since for $n\in\bN$ we can create the predicate $x=n$ by $x=1+1+\cdots+1$, and if $n\in\bZ$ is negative then $x=n$ by $x+1+\cdots+1=0$.
    So we can form predicates for $x=n$ for $n\in\bZ$.
    And for $b\in\bN$ and $a\in\bZ$ we can define the formula
    \[ \phi_{a/b}(x) = (x + x + \cdots + x) = a \]
    where $x$ is added with itself $b$ times.
    This predicate defines $x=\frac ab$.

    Notice that this doesn't contradict our above theorem since all group automorphisms of $(R,+)$ which map $1$ to $1$ must have rational numbers as fixed points.
    This is obvious for integers, and $b\cdot\phi\parens{\frac ab}=\phi(a)=a$ so $\phi\parens{\frac ab}=\frac ab$.

    And if $\gamma$ is irrational, we can take another irrational number $\gamma'$ then there exists an automorphism which maps $\gamma\varmapsto\gamma'$.
    Any predicate $x=\gamma$ is thus not stable under this automorphism and is therefore not definable.

\end{exam*}

\end{document}

