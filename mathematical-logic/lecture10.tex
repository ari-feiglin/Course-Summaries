\documentclass[10pt]{article}

\usepackage{amsmath, amssymb, mathtools}
\usepackage[margin=1.5cm]{geometry}

\font\tensy=cmsy10
\newfam\scrfam
\textfont\scrfam=\tensy
\def\mathscr#1{{\fam=\scrfam#1}}

\input pdfmsym
\input prettyprint
\input preamble

\pdfmsymsetscalefactor{10}
\initpps
\newmathpp{axiom}{Axiom}{255,255,200}{150,100,0}{200,150,100}			% yellow

\@Arrow@def{varLeftRightarrow}\@Larrow\@Rarrow{1}
\def\implies{\,\longvarRightarrow\,}
\def\iff{\,\longvarLeftRightarrow\,}
\let\eiff=\varLeftRightarrow
\let\to=\varrightarrow
\let\longto=\longvarrightarrow
\let\oto=\varleftrightarrow
\let\injection=\longvaruphookrightarrow

\let\nor=\downarrow
\let\nand=\uparrow

\def\aone#1#2{#2\to(#1\to#2)} \def\Aone{\textbf{A1}}
\def\atwo#1#2#3{\bigl(#1\to(#2\to#3)\bigr)\to\bigl((#1\to#2)\to(#1\to#3)\bigr)} \def\Atwo{\textbf{A2}}
\def\athree#1#2{(\neg#2\to\neg#1)\to\bigl((\neg#2\to#1)\to#2\bigr)} \def\Athree{\textbf{A3}}
\def\afour#1#2#3{(\forall #2#1)\to#1\frac{#3}{#2}}  \def\Afour{\textbf{A4}}
\def\afive#1#2#3{\bigl(\forall #3(#1\to#2)\bigr)\to\bigl(#1\to(\forall#3#2)\bigr)} \def\Afive{\textbf{A5}}
\def\MP{\textit{MP}} \def\Gen{\textit{Gen}}

\def\beqq{\mathrel{\vbox{\hbox{$=$}\kern-\baselineskip\kern.3pt\hbox{$=$}}}}

\def\bB{\mathbb{B}}
\def\mA{\mathcal{A}}
\def\mL{\mathcal{L}}
\def\mT{\mathcal{T}}
\def\mM{\mathcal{M}}
\def\mD{\mathcal{D}}
\def\fM{\mathfrak{M}}
\def\fC{\mathfrak{C}}
\def\fm{\mathfrak{m}}
\def\fn{\mathfrak{n}}

\def\Var{\mathrm{Var}}

\newfunc{var}{{\rm var}}({})
\newfunc{bound}{{\rm bound}}({})
\newfunc{free}{{\rm free}}({})
\newfunc{theory}{{\rm Th}}({})

\def\qed{\hskip.5cm\hbox{}\hfill$\blacksquare$}

\def\pmat#1{\begin{pmatrix} #1 \end{pmatrix}}

\def\mB{\mathbb{B}}
\def\true{\mathsf{true}}
\def\false{\mathsf{false}}

%\def\Aone#1#2{#1\to(#2\to#1)}
%\def\Atwo#1#2#3{\bigl(#1\to(#2\to#3)\bigr)\to\bigl((#1\to#2)\to(#1\to#3)\bigr)}
%\def\Athree#1#2{(\neg#2\to\neg#1)\to\bigl((\neg#2\to#1)\to#2\bigr)}

\let\divides=\mid

\font\bigbf = cmbx12 scaled 2000

\parskip=5pt plus 1pt minus 3pt
\pp@headerskip=\z@

\begin{document}

\c@section=10

\barcolorbox{220, 255, 255}{0, 130, 130}{80, 200, 200}{
    \leftskip=0pt plus 1fill \rightskip=\leftskip
    {\bigbf Mathematical Logic}

    \medskip
    \textit{Lecture \thesection, Friday June 9, 2023}

    \textit{Ari Feiglin}
}

\bigskip

We now generalize so that our signature can be uncountable, and therefore so can $\mL$.
We also consider only theories with equality in their signature.
Thus we have to modify some theorems from lecture 8: extending a theory to a scapegoat theory may result in a model of cardinality $\abs\mL$.

\begin{defn*}

    A language $\mL$ is \ppemph{recursive} if there exists an algorithm \emph{decides} if $\phi$ is an $\mL$-formula.
    And an $\mL$-theory $\mT$ is \ppemph{recursive} if when given an $\mL$-sentence (closed formula) $\mL$ decides whether $\phi\in\mT$.

\end{defn*}

\begin{prop*}

    If $\mL$ is a recursive language and $\mT$ a recursive $\mL$-theory, then $\set{\phi}[T\vdash\phi]$ is recursively enumerable (there exists an algorithm which halts and accepts all $\phi$ such that
    $T\vdash\phi$ and reject or not halt for $T\nvdash\phi$).

\end{prop*}

\begin{proof}

    We take the sequence of all finite sequences of $\mL$-fromulas, $\sigma_1,\sigma_2,\dots$.
    This is computable as we can iterate over all strings and check if they are formulas sicne $\mL$ is recursive.
    At stage $i$ in the algroithm, we check if $\sigma_i$ proves $\phi$.
    This requires checking whether the formulas are in $\mT$ or follow from previous formulae in the proof.
    If $\sigma_i$ proves $\phi$ then halt and accept, if not then continue.
    \qed

\end{proof}

\begin{defn*}

    An $\mL$-theory is \ppemph{complete} if for every $\mL$-sentence $\phi$ either $T\vdash\phi$ or $T\vdash\neg\phi$.

\end{defn*}

\begin{defn*}

    If $\mM$ is an $\mL$-model (or structure) then we define
    \[ \theoryof\mM = \set{\phi}[\phi\text{ is an } \mL-\text{sentence and } \mM\vDash\phi] \]

\end{defn*}

$\theoryof\mM$ is complete since for every sentence $\phi$, by definition either $\mM\vDash\phi$ or $\mM\vDash\neg\phi$ so either $\phi$ or $\neg\phi$ is in $\theoryof\mM$.
Also note that $\theoryof\mM$ is consistent.

\begin{prop*}[infModelProp]

    Let $T$ be an $\mL$-theory with infinite models.
    If $\varkappa$ is a cardinal such that $\varkappa\geq\abs\mL$ then there exists a model of $T$ of cardinality $\varkappa$.

\end{prop*}

\begin{proof}

    Let $\mL^*=\mL\cup\set{c_\alpha}[\alpha<\varkappa]$ where $c_\alpha$ is a new constant symbol for every ordinal $\alpha<\varkappa$.
    Then let $T^*$ be the $\mL^*$-theory
    \[ T^* = T\cup\set{c_\alpha\neq c_\beta}[\alpha\neq\beta<\varkappa] \]
    Notice that if $\mM\vDash T^*$ then if $\mM\vDash T$ and $\mM$ has a cardinality of at least $\varkappa$ since $\mM$'s domain must have $\varkappa$ distinct elements which are the interpretations of
    $c_\alpha$.
    So all we must do is to show that $T^*$ is consistent, and thus has a model.
    To show consistency, we will use the compactness theorem.
    If $\Delta\subseteq T^*$ is finite then there exists a finite subset $I\subseteq\varkappa$ such that
    \[ \Delta \subseteq T\cup\set{c_\alpha\neq c_\beta}[\alpha\neq\beta\in I] \]
    since $T$ has infinite models, let $\mM$ be such an infinite model.
    We can add the valuations of $c_\alpha$s as $\abs I$ distinct elements in $\mM$, and thus we have a model of $\Delta$, meaning $\Delta$ is consistent.

    Thus $T^*$ is consistent by the compactness theorem, and has a model of cardinality $\abs{\mL^*}=\varkappa$.
    This model also models $T$, as required.
    \qed

\end{proof}

\begin{defn*}

    Let $\varkappa$ be an infinite cardinal and let $T$ be a theory with models of cardinality $\varkappa$.
    Then we say that $T$ is \ppemph{$\varkappa$-categorical} if any two models of $T$ of cardinality $\varkappa$ are isomorphic.

\end{defn*}

Let $\mL=\set{+,0}$ be the language of groups, and let $T$ be the $\mL$-theory of torsion-free divisible abelian groups.
Then the axioms of $T$ are the axioms of abelian groups as well as the schemas
\begin{align*}
    \forall x(x\neq 0\to nx\neq0) & \quad\text{(Torsion-Free)}\\
    \forall y\exists x(nx=y) & \quad\text{(Divisible)}
\end{align*}

For $n=1,2,\dots$ and $nx$ denotes $x+x+\cdots+x$, $n$ times.

\begin{prop*}

    The theory of torsion-free divisible abelian groups is $\varkappa$-categorical for all $\varkappa>\aleph_0$.

\end{prop*}

\begin{proof}

    We claim that a $G$ is a model of $T$ if and only if it induces a $\bQ$-vector space.
    First, if $V$ is a vector space over $\bQ$ then it is a torsion-free divisible group under its additive operation (if $nx=0$ then since $\frac1n\in\bQ$ then $x=0$, and $x=\frac1ny$ satisfies
    divisibility).
    For the converse, suppose $G\vDash T$.
    Notice then that if $g\in G$ and $n>0$ then there exists an $h$ such that $nh=g$ by divisibility.
    If $nk=g$ then $n(h-k)=0$ and so by torsion-freeness, $h=k$ meaning that for every $g\in G$ and $n>0$ there exists a unique $h\in G$ such that $nh=g$.
    We denote $h$ by $\frac gn$.
    Thus we can view $G$ as a vector space over $\bQ$ by $\frac mng=m\parens{\frac gn}$.

    Two $\bQ$-vector spaces are isomorphic if and only if they have the same dimension.
    Therefore models of $T$ are isomorphic if they share a dimension.
    If $G$ has a basis $B$, then every element of $G$ can be written as a linear combination of a finite number of vectors in $B$.
    If $\abs B=\lambda$ then if $\lambda$ is infinite there are $\lambda$ finite subsets of $B$ and each subset generates $\aleph_0$ vectors (the span of the finite subset is isomorphic to $\bQ^n$) and thus
    $\abs G=\aleph_0\cdot\lambda$.
    If $\lambda$ is finite then there are $2^\lambda$ finite subsets of $B$, and we get $\abs G=\aleph_0\cdot2^\lambda=\aleph_0\cdot\lambda$.

    Thus if $\varkappa>\alpha_0$ and $\abs G=\varkappa$ then the dimension of $G$ must be $\varkappa$.
    Thus all models of $T$ of cardinality $\varkappa$ have the same dimension and are therefore all isomorphic.
    \qed

\end{proof}

Note that this is not true if $\varkappa\leq\aleph_0$.
For example we can take $\bQ$ and $\bQ^2$.

\def\ACF{\mathrm{ACF}_p}
Similarly, let $\ACF$ be the theory of algebraically closed fields of characteristic $p$, where $p$ is either $0$ or prime.

\begin{prop*}

    $\ACF$ is $\varkappa$-categorical for all uncountable cardinals $\varkappa$.

\end{prop*}

\begin{proof}

    We know that two fields are isomorphic if and only if they have the same characteristic and transsendence degree.
    An algebraically closed field with transcendence degree $\lambda$ has cardinality $\lambda+\aleph_0$.
    If $\varkappa>\aleph_0$ that means $\lambda=\varkappa$, and so any algebraically closed field of cardinality $\varkappa$ has transcendence degree $\varkappa$.
    Thus any two algebraically closed fields of characteristic $p$ and uncountable cardinality $\varkappa$ are isomorphic.
    \qed

\end{proof}

\begin{exam*}

    Let $\mL=\set E$ where $E$ is a binary relation.
    Let $T$ be the theory of an equivalence relation with two classes which are both infinite.
    Thus $T$ is $\aleph_0$-categorical as if we have two models $\mM_1$ and $\mM_2$ of cardinality $\aleph_0$, then we can create an isomorphism between them by mapping elements from one class in $\mM_1$ to
    one class in $\mM_2$ and from the other class in $\mM_1$ to the other in $\mM_2$.
    Such a bijection exists because all the classes are infinite, and since the models are countable, all the classes are countably infinite and thus have the same cardinality.
    This bijection is obviously an isomorphism.

    But if $\varkappa$ is uncountable we can define an equivalence relation with one countable class and one class of cardinality $\varkappa$, and we can define another equivalence relation with two classes
    of cardinality $\varkappa$.
    These two models are not isomorphic.
    So $T$ is not $\varkappa$-categorical for uncountable cardinalities $\varkappa$.

\end{exam*}

\begin{thrm*}[vaughtTest,Vaught's\ Test]

    Let $T$ be a satisfiable theory with no finite models, and is also $\varkappa$-categorical for some infinite cardinality $\varkappa\geq\abs\mL$.
    Then $T$ is complete.

\end{thrm*}

\begin{proof}

    Suppose that $T$ is not complete, meaning there exists a sentence $\phi$ such that $T\nvdash\phi$ and $T\nvdash\neg\phi$.
    This means that $T_0=T\cup\set{\phi}$ and $T_1=T\cup\set{\neg\phi}$ are consistent, and thus satisfiable.
    Since $T$ has no finite models, the models of $T_0$ and $T_1$ cannot be finite either, and so both theories must have infinite models.
    Thus by \ppref[proposition]{infModelProp}, we can find models $\mM_0$ and $\mM_1$ for each theory respectively of cardinality $\varkappa$.
    But since $\mM_0\vDash\phi$ and $\mM_1\vDash\neg\phi$, these models cannot be isomorphic (since isomorphic models satisfy the same sentences), which contradicts the $\varkappa$-categoricalness of $T$, as
    both of these models also model $T$.
    \qed

\end{proof}

\begin{exam*}

    Let $T$ be the $\set{=,+,0}$-theory of groups where every element has order $2$.
    It can be shown that $T$ is $\varkappa$-categorical for every infinite cardinality $\varkappa$.
    But $T$ is not complete since the sentence
    \[ \exists x,y,z(x\neq y\land y\neq z\land z\neq x) \]
    is false in the two element group, but true in every other non-trivial group.
    Thus $T$ cannot prove this sentence, or its negation.

\end{exam*}

Vaught's Test shows that $\ACF$ is complete, as finite fields cannot be algebraically closed: if $F$ is a finite field then let
\[ f(x) = 1 + \prod_{a\in F}(x-a) \]
$f$ has no roots in $F$ since for any $a\in F$, $f(a)=a$ (yet $f$ is not a constant polynomial).

\begin{defn*}

    An $\mL$-theory $\mT$ is \ppemph{decidable} if there exists an algorithm which \emph{decides} whether an $\mL$-sentence $\phi$ can be proven by $\mT$.

\end{defn*}

\begin{lemm*}

    Let $T$ be a recursive, complete, satisfiable theory in a recursive language $\mL$.
    Then $T$ is decidable.

\end{lemm*}

\begin{proof}

    Our algorithm searches through all possible proofs (since $\mL$ and $T$ are recursive) until it finds a proof of $\phi$ or $\neg\phi$.
    Since $T$ is complete, one must exist, and thus the algorithm always halts.
    Since $T$ is satisfiable, only one of these proofs can exist, and thus this algorithm will always decide if $\phi$ can be proven by $T$.
    \qed

\end{proof}

\begin{coro*}

    For $p=0$ or $p$ prime, $\ACF$ is decidable.
    In particular, $\theoryof\bC$ is decidable.

\end{coro*}

Since we know $\ACF$ is complete and satisfiable, all we must show is that it is recursive.
This is simple as we can write down the axioms, and thus the axioms (and hence the theory) is recursive.

\begin{note}

    There is some disagreement on the meaning of a theory, some take it as any set of formulas (``axioms'') and some take it as the deductive closure of such a set of formulas (all formulas which can
    be deduced from a set of formulas).
    In any case the definition is immaterial since we almost always only care about the deductive closure of a theory.

    We use the definition that a theory is a set of axioms, and we denote its deductive closure as $T^\vdash$.

\end{note}

\begin{lemm*}

    Then the following are equivalent:
    \benum
        \item $T$ is consistent/satisfiable and complete.
        \item There exists a model of $T$, and for every model $\mM$ of $T$, $T^\vdash=\theoryof\mM$.
        \item There exists a model $\mM$ such that $T^\vdash=\theoryof\mM$.
    \eenum

\end{lemm*}

\begin{proof}

    Since $T$ is satisfiable, there exists a model of $T$.
    Let $\mM$ be a model of $T$.
    Then if $T\vdash\phi$, we know $\mM\vDash\phi$ so $T^\vdash\subseteq\theoryof\mM$.
    And if $\mM\vDash\phi$ then since $T$ is complete if $T\nvdash\phi$ then $T\vdash\neg\phi$ and so $\mM\vDash\neg\phi$, which is a contradiction.

    Obviously since $T$ is satisfiable, $2$ implies $3$ (since there exists a model of $T$).

    Now suppose $T^\vdash=\theoryof\mM$, then since $T\subseteq T^\vdash$, we have that $\mM\vDash T$ so $T$ is satisfiable.
    And if $\phi$ is a sentence, if $T\nvdash\phi$ then $\phi\notin\theoryof\mM$, meaning $\neg\phi\in\theoryof\mM$ (since it is complete), so $T\vdash\neg\phi$.
    Thus $T$ is complete and satisfiable.
    \qed

\end{proof}

\begin{coro*}

    Let $\phi$ be a sentence in the language of rings.
    The following are equivalent:
    \benum
        \item $\phi$ is true in $\bC$.
        \item $\phi$ is true in every algebraically closed field of characteristic zero.
        \item $\phi$ is true in some algebraically closed field of characteristic zero.
        \item There are arbitrarily large primes $p$ such that $\phi$ is true in some algebraically closed field of characteristic $p$.
        \item There is an $m$ such that for all $p>m$, $\phi$ is true in all algebraically closed fields of characteristic $p$.
    \eenum

\end{coro*}

\def\ACFz{\mathrm{ACF}_0}
\begin{proof}

    Since $\mathrm{ACF}_0$ is complete and satisfiable, if $\mM$ is a model of $\mathrm{ACF}_0$, then $\mathrm{ACF}_0^\vdash=\theoryof\mM$.
    Thus $\phi$ is true in $\bC$ if and only if it is true in some model of $\mathrm{ACF}_0$ if and only if it is true in the theory itself if and only if it is true in every model of the theory
    (every algebraically closed field of characteristic $0$).
    $5$ also obviously implies $4$.

    We will show $2$ implies $5$.
    Suppose $\ACFz\vdash\phi$, since we only need a finite subset of $\ACFz$ (say $\Delta$) such that $\Delta\vdash\phi$, and by the compactness theorem, $\Delta$ is satisfiable.
    Since $\Delta$ is finite and there must be infinite axioms determining the zero-characteristic of the field, there exists some $p$ prime large enough such that $\ACF\vdash\phi$.
    Thus $\ACF\vdash\phi$ for all primes $p$ larger than some $m$ (eg. $m=p$ above).

    To show $4$ implies $2$, suppose that $\ACFz\nvdash\phi$.
    Because $\ACFz$ is complete, this means $\ACFz\vdash\neg\phi$, and so by above for sufficiently large primes, $\ACF\vdash\neg\phi$ since $2\implies5$.
    But this contradicts $4$.
    \qed

\end{proof}

\begin{thrm*}

    Every injective polynomial map from $\bC^n$ to $\bC^n$ is surjective.

\end{thrm*}

\begin{proof}

    \def\falg{\bF_p^{\mathrm{alg}}}
    We will show that for every injective polynomial map
    \[ f\colon(\falg)^n\longto(\falg)^n \]
    where $\falg$ is the algebraic closure of the field with $p$ elements, is surjective.

    Let $\vec a$ be the coefficients of $f$ and let $\vec b\in(\falg)^n$ be outside of the image of $f$.
    Let $k$ be the subfield of $\falg$ generated by the coefficients of $\vec a$ and $\vec b$.
    Then $f\bigl|_{k^n}$ is an injective map from $k^n$ into itself (the image of a polynomial over a subfield is contained within that field), but not surjective as $\vec b\in k^n$.
    Since
    \[ \falg = \bigcup_{n=1}^\infty\bF_{p^n} \]
    $\falg$ is locally finite, and therefore $k$ is finite.
    But every injection from a finite set to a finite set is surjective, and therefore $f$ must be surjective, in contradiction.

    Let $f$ be a multivariate polynomial, then it can be written as $f=(f_1,\dots,f_n)$, and we can bound the degree of these polynomials by $d$ ($f_i$ are called coordinate functions).
    There is a sentence $\Phi_{n,d}$ such that for every field $\bF$, $\bF\vDash\Phi_{n,d}$ if and only if every injective polynomial map from $\bF^n\injection\bF^n$ with coordinate functions of degree
    $\leq d$ is surjective.
    We can write $\Phi_{n,d}$ like so: we take $a^m_{i_1,\dots,i_n}$ for $0\leq i_k\leq d$ as the coefficients of $f_m$ and we say that $f_m$ is injective by stating that for all inputs
    $x_1,\dots,x_n$ and $y_1,\dots,y_n$ if $f_m(x_1,\dots,x_n)=f_m(y_1,\dots,y_n)$ for all $m$ then $x_i=y_i$ for all $i$.
    We then say that $f$ is surjective by saying that for every $u_1,\dots,u_n$ there exist $x_1,\dots,x_n$ such that $f_m(x_1,\dots,x_n)=u_m$.
    All in all we can write $\Phi_{n,d}$ as
    \begin{multline*}
        \forall a^m_{i_1,\dots,i_n}\Bigl(\forall x_1,\dots,x_n,y_1,\dots,y_n\Bigl(\bigwedge_{m=1}^n \sum a^m_{i_1,\dots,i_n}x_1^{i_1}\cdots x_n^{i_n}=
        \sum a^m_{i_1,\dots,i_n}y_1^{i_1}\cdots y_n^{i_n}\Bigr)\to\bigl(x_1=y_1\land\cdots\land x_n=y_n\bigr)\Bigr)\\
        \to\Bigl(\forall u_1,\dots,u_n\exists x_1,\dots,x_n\Bigl(\bigwedge_{m=1}^n \sum a^m_{i_1,\dots,i_n}x_1^{i_1}\cdots x_n^{i_n}=u_m\Bigr)\Bigr)\biggr)
    \end{multline*}

    Since we have shown that $\falg\vDash\Phi_{n,d}$ for all primes $p$, by the above corollary we must have $\bC\vDash\Phi_{n,d}$ for every $n,d$ as required.
    \qed

\end{proof}

\end{document}

