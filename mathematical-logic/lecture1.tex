\documentclass[10pt]{article}

\usepackage{amsmath, amssymb, mathtools}
\usepackage[margin=1.5cm]{geometry}

\input pdfmsym
\input prettyprint
\input preamble

\pdfmsymsetscalefactor{10}
\initpps

\@Arrow@def{varLeftRightarrow}\@Larrow\@Rarrow{1}
\def\implies{\,\longvarRightarrow\,}
\def\iff{\,\longvarLeftRightarrow\,}
\let\eiff=\varLeftRightarrow

\def\pmat#1{\begin{pmatrix} #1 \end{pmatrix}}

\def\@blist[#1]{%
    \bgroup\bgroup\par\vskip-\medskipamount%
    \gdef\item{%
        \par\egroup\bgroup\medskip\setbox0=\hbox{#1\quad}%
        \advance\leftskip by \wd0\leavevmode\kern-\wd0\box0%
    }%
}
\def\blist{\@ifnextchar[ \@blist {\@blist[$\bullet$]}}
\def\elist{\par\egroup\egroup\medskip}

\def\mB{\mathbb{B}}
\def\true{\mathsf{true}}
\def\false{\mathsf{false}}

\let\divides=\mid

\font\bigbf = cmbx12 scaled 2000

\begin{document}

\c@section=1

\barcolorbox{220, 255, 255}{0, 130, 130}{80, 200, 200}{
    \leftskip=0pt plus 1fill \rightskip=\leftskip
    {\bigbf Mathematical Logic}

    \medskip
    \textit{Lecture 1, Monday March 20, 2023}

    \textit{Ari Feiglin}
}

\bigskip

Lectures are given by Professor Elena Bunina, email {\tt helenbunina@gmail\!.\!com} (because I couldn't find her email elsewhere).

Mathematical logic has its origins in the mid-nineteenth century with the formalization of modern mathematics.
The main motivation of formalizing mathematics is to answer some apparent paradoxes, for example:

\benum
    \item \textbf{Russell's Paradox} ($1902$): since the notion of a set was not formalized, it was believed that anything could be a set (since why not? The definition wasn't formalized/axiomized).
    So let $S$ be the set of all which do not contain themselves as elements, ie. $S=\set{A}[A\notin A]$.
    Is $S\in S$?
    If so then by definition $S$ cannot be in $S$, in contradiction.
    So $S\notin S$, but then by definition $S\in S$, in contradiction.
    Thus resulting in an apparent paradox.

    This paradox is resolved by axiomizing set theory (ZF, ZFC, etc).

    \item \textbf{Liar Paradox}: If a man says ``I am lying'', is he lying?
    If he is lying, then he is not lying in contradiction.
    If he is not lying, then he is lying in contradiction.
    While this paradox may seem stupid, it raises an interesting point that not every possible statement may have a truth value.

    \item \textbf{Richard's Paradox} ($1905$): This paradox is similar to Cantor's diagonal argument.
    English (and many other languages) give ways to unambiguously define a real number (eg. ``the number whose integer part is $10$ and the $n$th decimal is
    $n\bmod9$'').
    Thus we can take all the English expressions which define real numbers and order them lexicographically to give $r_1,r_2,\dots$.
    Then we can define a new real number $r$ whose integer part is $0$ and $n$th decimal is $1$ if the $n$th decimal of $r_n$ is not $1$ and $2$ if it is.
    Thus $r$ is unambiguously expressed in English, but $r\neq r_n$ for all $n$ since their $n$th decimal places differ.
    This is a contradiction since $r_n$ is the list of all such real numbers.

    \item \textbf{Berry's Paradox} ($1906$): Notice that there is only a finite number of phrases of length less than $200$.
    Thus there is a finite number of natural numbers definable in fewer than $200$ words, and therefore an infinite number of natural numbers not definable in fewer than $200$ words.
    Let $A$ be the set of all natural numbers not definable in fewer than $200$ words.
    Since the naturals are well-ordered, there is a minimum element in $A$.
    Thus we can \textit{define} a number to be the ``minimum element not definable in fewer than $200$ words''.
    But this is a definition which uses fewer than $200$ words, and so this number \textit{can} be defined in fewer than $200$ words, which is a contradiction since it is in $A$.

    \item \textbf{L\"ob Paradox} ($1955$): Let $A$ be any formal sentence.
    Let $B$ be the sentence ``if this sentence is true, then $A$ is true.''
    If $B$ is true, then $A$ is true.
    If $B$ is false, suppose $A$ is false then $B$ is of the form $\mathsf{false}\varrightarrow\mathsf{false}$ so $B$ is true, in contradiction.
    So $B$ is true and therefore $A$ is true.
    Thus we have shown that every formal sentence is true, which obviously is a contradiction.
\eenum

Without the foundation of mathematical logic, we are susceptible to these paradoxes.
But if we formalize what exactly we mean, these paradoxes are no longer an issue.

\subsection{Propositional Calculus}

The following are logical connectives which we covered in discrete mathematics:
\blist
    \item $\neg A$ (also written as $\overline A$) called \emph{negation}.
    \item $A\land B$ (also written as $A\&B$ or $A\text{ and }B$) called \emph{conjunction}.
    \item $A\lor B$ (also written as $A\text{ or }B$) called \emph{disjunction}.
    \item $A\varrightarrow B$ (also written as $A\varRightarrow B$, $A$ implies $B$, $B$ follows from $A$, if $A$ then $B$) called \emph{implication}.
\elist

\begin{defn*}

    A \ppemph{statement form} is defined recursively:
    \benum
        \item Every variable (a basic statement) is a statement form.
        \item If $A$ is a statement form, so is $\neg A$.
        \item If $A$ and $B$ are statement forms, so is $A\land B$, $A\lor B$, and $A\varrightarrow B$.
    \eenum

\end{defn*}

\begin{defn*}

    The \ppemph{boolean set}, denoted $\mB$ is the set $\set{\false,\true}$ or $\set{0,1}$.
    A \ppemph{boolean function} is a function $f\colon\mB^n\longvarrightarrow\mB$ for some number $n\in\bN_{\geq1}$.

\end{defn*}

All statement forms are boolean functions, this is true recursively since a basic statement is by definition a boolean function.
The converse is also true:

\begin{prop*}

    All boolean functions are statement forms.

\end{prop*}

\begin{proof}

    Let $S=f^{-1}\set\true$, then for every $v\in S$ let $a_v=\bigwedge_{i=1}^n a_{v,i}$ where $a_{v,i}=x_i$ if $v_i=\true$ and $a_{v,i}=\neg x_i$ otherwise.
    Notice that $a_v$ is true only under $v$, since $a_v$ is true under $u$ only if every $a_{v,i}$ is true under $u$ so if $v_i=\true$, then $x_i(u)=u_i$ must be true, and if $v_i=\false$,
    $\neg x_i(u)=\neg u_i$ is true, and so $u_i=\false$.
    So $u_i=v_i$ which means $u=v$.
    And for every $a_{v,i}$, $a_{v,i}(v)$ is by definition true, so $a_v(v)$ is true.
    So $a_v(u)$ is true if and only if $v=u$.
    Then we can define $A=\bigvee_{v\in S} a_v$

    We claim that $f=A$.
    Notice that $A(u)$ is true if and only if there exists a $v\in S$ such that $a_v(u)$ is true which is if and only if $v=u$.
    So $A(u)$ is true if and only if $u\in S$, ie. $f(u)=\true$.
    So $f=A$ as required.

    \hfill$\blacksquare$

\end{proof}

This representation of a statement form is called \emph{conjunctive normal form}.
Thus this shows that every statement form (equivalently every boolean table) can be formed using just the conjunction, disjunction, and negation connectives.

\begin{defn*}

    A \ppemph{tautology} is a boolean function (equivalently a statement form) $f\colon\mB^n\longvarrightarrow\mB$ such that $f(v)=\true$ for all $v\in\mB$.

    Two statements $\psi$ and $\phi$ are \ppemph{equivalent} if $\psi\varleftrightarrow\phi$ is a tautology, where $\psi\varleftrightarrow\phi$ is $(\psi\varrightarrow\phi)\land(\phi\varrightarrow\psi)$.
    This is denoted as $\psi\eiff\phi$ or $\psi\equiv\phi$.

\end{defn*}

The following are equivalent (these are all easily verified):
\benum
    \item $(A\land B)\eiff(B\land A)$.
    \item $\bigl((A\land B)\land C\bigr)\eiff\bigl(A\land(B\land C)\bigr)$.
    \item $(A\lor B)\eiff(B\lor A)$.
    \item $\bigl((A\lor B)\lor C\bigr)\eiff\bigl(A\lor(B\lor C)\bigr)$.
    \item $\bigl(A\land(B\lor C)\bigr)\eiff\bigl((A\land B)\lor(A\land C)\bigr)$.
    \item $\bigl(A\lor(B\land C)\bigr)\eiff\bigl((A\lor B)\land(A\lor C)\bigr)$.
    \item $\neg(A\land B)\eiff(\neg A\lor\neg B)$.
    \item $\neg(A\lor B)\eiff(\neg A\land\neg B)$.
    \item $\bigl(A\lor(A\land B)\bigr)\eiff A$.
    \item $\bigl(A\land(A\lor B)\bigr)\eiff A$.
\eenum

Note that $7$ and $8$ mean every statement form can be written using just conjunction and negation or disjunction and negation.

\end{document}

