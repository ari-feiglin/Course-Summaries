\documentclass[10pt]{article}

\usepackage{amsmath, amssymb, mathtools}
\usepackage[margin=1.5cm]{geometry}
\usepackage{mathrsfs}

\font\tensy=cmsy10
\newfam\scrfam
\textfont\scrfam=\tensy
\def\mathscr#1{{\fam=\scrfam#1}}

\input pdfmsym
\input prettyprint
\input preamble

\pdfmsymsetscalefactor{10}
\initpps
\newmathpp{axiom}{Axiom}{255,255,200}{150,100,0}{200,150,100}			% yellow

\@Arrow@def{varLeftRightarrow}\@Larrow\@Rarrow{1}
\def\implies{\,\longvarRightarrow\,}
\def\iff{\,\longvarLeftRightarrow\,}
\let\eiff=\varLeftRightarrow
\let\to=\varrightarrow
\let\longto=\longvarrightarrow
\let\oto=\varleftrightarrow
\let\injection=\longvaruphookrightarrow

\let\nor=\downarrow
\let\nand=\uparrow

\def\aone#1#2{#2\to(#1\to#2)} \def\Aone{\textbf{A1}}
\def\atwo#1#2#3{\bigl(#1\to(#2\to#3)\bigr)\to\bigl((#1\to#2)\to(#1\to#3)\bigr)} \def\Atwo{\textbf{A2}}
\def\athree#1#2{(\neg#2\to\neg#1)\to\bigl((\neg#2\to#1)\to#2\bigr)} \def\Athree{\textbf{A3}}
\def\afour#1#2#3{(\forall #2#1)\to#1\frac{#3}{#2}}  \def\Afour{\textbf{A4}}
\def\afive#1#2#3{\bigl(\forall #3(#1\to#2)\bigr)\to\bigl(#1\to(\forall#3#2)\bigr)} \def\Afive{\textbf{A5}}
\def\MP{\textit{MP}} \def\Gen{\textit{Gen}}

\def\beqq{\mathrel{\vbox{\hbox{$=$}\kern-\baselineskip\kern.3pt\hbox{$=$}}}}

\def\bB{\mathbb{B}}
\def\mA{\mathcal{A}}
\def\mB{\mathcal{B}}
\def\mL{\mathcal{L}}
\def\mT{\mathcal{T}}
\def\mM{\mathcal{M}}
\def\mN{\mathcal{N}}
\def\mD{\mathcal{D}}
\def\mJ{\mathcal{J}}
\def\mK{\mathcal{K}}
\def\fM{\mathfrak{M}}
\def\fC{\mathfrak{C}}
\def\fm{\mathfrak{m}}
\def\fn{\mathfrak{n}}
\def\sS{\mathscr{S}}

\def\Var{\mathrm{Var}}

\newfunc{var}{{\rm var}}({})
\newfunc{bound}{{\rm bound}}({})
\newfunc{free}{{\rm free}}({})
\newfunc{theory}{{\rm Th}}({})
\newfunc{Diag}{{\rm Diag}}({})
\newfunc{elDiag}{{\rm Diag}_{\rm el}}({})
\newfunc{depth}{{\rm depth}}({})

\def\qed{\hskip.5cm\hbox{}\hfill$\blacksquare$}

\def\pmat#1{\begin{pmatrix} #1 \end{pmatrix}}

\def\true{\mathsf{true}}
\def\false{\mathsf{false}}

%\def\Aone#1#2{#1\to(#2\to#1)}
%\def\Atwo#1#2#3{\bigl(#1\to(#2\to#3)\bigr)\to\bigl((#1\to#2)\to(#1\to#3)\bigr)}
%\def\Athree#1#2{(\neg#2\to\neg#1)\to\bigl((\neg#2\to#1)\to#2\bigr)}

\let\divides=\mid

\font\bigbf = cmbx12 scaled 2000

\parskip=5pt plus 1pt minus 3pt
\pp@headerskip=\z@

\begin{document}

\c@section=13

\barcolorbox{220, 255, 255}{0, 130, 130}{80, 200, 200}{
    \leftskip=0pt plus 1fill \rightskip=\leftskip
    {\bigbf Mathematical Logic}

    \medskip
    \textit{Lecture \thesection, Tuesday June 27, 2023}

    \textit{Ari Feiglin}
}

\bigskip

\begin{defn*}

    Let $I$ be a non-empty set, a \ppemph{filter} over $I$ is a set $D\subseteq\powsetof I$ such that
    \benum
        \item $I\in D$
        \item If $X,Y\in D$ then $X\cap Y\in D$
        \item If $X\in D$ and $X\subseteq Z\subseteq I$, then $Z\in D$ ($D$ is upwards closed)
    \eenum

\end{defn*}

\begin{exam*}

    \benum
        \item The filter $\set I$ is the \ppemph{trivial filter}.
        \item The filter $\powsetof I$ is the \ppemph{improper filter}.
        Filters which are not the improper filter are called \ppemph{proper filters}.
        \item For each $Y\subseteq I$, $D=\set{X\subseteq I}[Y\subseteq X]$ is the \ppemph{principal filter generated by $Y$}.
        \item The \ppemph{Frech\'et} filter $D=\set{X\subseteq I}[I\setminus X\text{ is finite}]$.
    \eenum

\end{exam*}

Notice that if $\set{D_\lambda}_{\lambda\in\Lambda}$ is a family of filters over $I$, then
\[ D = \bigcap_{\lambda\in\Lambda} D_\lambda \]
is also a filter.
Obviously $I\in D$, and if $X,Y\in D$ then $X,Y\in D_\lambda$ for every $\lambda\in\Lambda$ and so $X\cap Y\in D_\lambda$ for every $\lambda\in\Lambda$ and so $X\cap Y$.
And if $X\in D$ and $X\subseteq Z\subseteq I$ then $Z\in D_\lambda$ for every $\lambda\in\Lambda$ and so $Z\in D$.

\begin{defn*}

    If $E\subseteq\powsetof I$, then the \ppemph{filter generated by $E$} is the smallest filter over $I$ which contains $E$.
    Since the intersection of arbitrary non-empty families of filters is also a filter, the filter generated by $E$ is equal to
    \[ \bigcap_{\substack{F\text{ is a filter}\\E\subseteq F}} F \]
    since this intersection is non-empty as the improper filter is in it.

\end{defn*}

\begin{defn*}

    If $E\subseteq\powsetof I$, it is said to have the \ppemph{finite intersection property} if the intersection of any finite number of sets in $E$ is non-empty.

\end{defn*}

\begin{prop*}

    Let $E\subseteq\powsetof I$, and let $D$ be the filter generated by $E$, then
    \benum
        \item $D$ is a filter over $I$
        \item $D$ is the set of all $X\in\powsetof I$ such that $X=I$ or for some $Y_1,\dots,Y_n\in E$
        \[ Y_1\cap\cdots\cap Y_n\subseteq X \]
        \item $D$ is a proper filter if and only if $E$ has the finite intersection property
    \eenum

\end{prop*}

\begin{proof}

    \benum
        \item We have shown this, as it is the intersection of a non-empty family of filters which is itself a filter.

        \item Let $D'$ be the set of all $X$ such that there exist $Y_1,\dots,Y_n\in E$ such that $Y_1\cap\cdots\cap Y_n\subseteq X$, or $X=I$, ie
        \[ D' = \set{X\in\powsetof I}[X=I\text{ or }\exists Y_1,\dots,Y_n\in E\colon Y_1\cap\cdots\cap Y_n\subseteq X] \]
        we will show $D=D'$.

        Firstly we will show that $D'$ is a filter containing $E$.
        Obviously $I\in D'$.
        If $X,X'\in D'$ then let $Y_1,\dots,Y_n,Y_1',\dots,Y'_m\in E$ such that
        \[ Y_1\cap\cdots\cap Y_n\subseteq X,\quad Y_1'\cap\cdots\cap Y_m'\subseteq X' \]
        then
        \[ Y_1\cap\cdots\cap Y_n\cap Y_1'\cap\cdots\cap Y_m' \subseteq X\cap X' \]
        and so $X\cap X'\in D'$.
        And if $X\in D$ and $X\subseteq Z\subseteq I$ then if $Y_1\cap\cdots\cap Y_n\subseteq X$, $Y_1\cap\cdots\cap Y_n\subseteq Z$ and so $Z\in D'$.
        Therefore $D'$ is a filter.
        And if $Y\in E$, then $Y\subseteq Y$ and so $E\subseteq D'$, meaning $D'$ is a filter containing $E$ as required.

        Since $D$ is the smallest filter containing $E$, $D\subseteq D'$.

        Now let $F$ be any filter over $I$ which includes $E$, then if $Y_1,\dots,Y_n\in E$ we must have that $Y_1\cap\cdots\cap Y_n\in F$.
        Moreso, since filters are upwards-closed, we must have that for every $X\in\powsetof I$ such that $Y_1\cap\cdots\cap Y_n\subseteq X$, $X\in F$.
        Meaning that $D'\subseteq F$ and in particular $D'\subseteq D$.
        Thus $D=D'$ as required.

        \item Note that a filter $F$ is a proper filter if and only if $\varnothing\notin F$.
        If $\varnothing\in F$ then for every $\varnothing\subseteq X$, $X\in F$ meaning $F=\powsetof I$ so it is the improper filter.
        And if $\varnothing\notin F$ then it is obviously is a proper filter.

        So $D$ is a proper filter if and only if $\varnothing\notin D$, which is if and only if for every $Y_1,\dots,Y_n\in E$, $Y_1\cap\cdots\cap Y_n\neq\varnothing$, which is precisely what it means for
        $E$ to have the finite intersection property.
        \qed
    \eenum

\end{proof}

Let us give an example of a particularly important filter.
Let $J$ be an infinite set and let $I=\powsetof[\omega]J$ be the set of all finite subsets of $J$.
For each $j\in J$ let
\[ \mJ_j = \set{i\in I}[j\in i] \subseteq I \]
be the set of all finite subsets of $J$ which contain $j$.
And let
\[ E = \set{\mJ_j}[j\in J] \subseteq \powsetof I \]
Then let $D$ be the filter over $I$ generated by $E$.
$E$ has the finite intersection property since if $j_1,\dots,j_n\in J$ then $\set{j_1,\dots,j_n}\in\mJ_{j_k}$ for every $1\leq k\leq n$ and so $\set{j_1,\dots,j_n}\in\mJ_{j_1}\cap\cdots\cap\mJ_{j_n}$.

\begin{defn*}

    $D$ is said to be an \ppemph{ultrafilter} over $I$ if $D$ is a filter over $I$ and for every $X\subseteq I$, $X\in D$ if and only if $I\setminus X\notin D$.
    Meaning that for every $X\in\powsetof I$, $D$ contains either $X$ or $I\setminus X$.

\end{defn*}

Notice that if $D$ is a filter over $I$ then $\bigcup_{X\in D}X=I$ since $I\in D$.
Thus if we say $D$ is a filter, we do not need to state over what.

\begin{prop*}

    The following are equivalent:
    \benum
        \item $D$ is an ultrafilter over $I$
        \item $D$ is a maximal proper filter over $I$ (if $F$ is a proper filter over $I$ such that $D\subseteq F$ then $F=D$)
    \eenum

\end{prop*}

\begin{proof}

    Suppose $D$ is an ultrafilter over $I$, then $D$ is a proper filter since $I\in D$ so $I\setminus I=\varnothing\notin D$.
    Let $F$ be a proper filter which includes $D$.
    If $X\in F$ and $X\notin D$ then $I\setminus X\in D$ which means $I\setminus X\in F$ but then $\varnothing=X\cap(I\setminus X)$ and so $\varnothing\in F$ which contradicts $F$ being proper.
    And so $F\subseteq D$, meaning $F=D$ as required.

    Now suppose $D$ is a maximal proper filter over $I$, then let $X\in\powsetof I$.
    We cannot have both $X\in D$ and $I\setminus X\in D$ since then $\varnothing\in D$ but $D$ is proper.
    So we will show that if $I\setminus X\notin D$ then $X\in D$.
    Let $E=D\cup\set X$, then let $F$ be the filter generated by $E$.
    Let $Y_1,\dots,Y_n\in E$ and let $Z=Y_1\cap\cdots\cap Y_n$, then since $D$ is closed under finite intersections, $Z=Y$ or $Z=Y\cap X$ for $Y\in D$.
    In the first case $Z\in D$ and so $Z\neq\varnothing$.
    For the second case, if $Z=\varnothing$ then $Y\cap X=\varnothing$ meaning that $Y\subseteq I\setminus X$ and so $I\setminus X\in D$, which is a contradiction.
    So we have in both cases that $Z\neq\varnothing$ and so $E$ has the finite intersection property, and therefore $F$ is a proper filter.
    Since $D$ is maximal, this means $F=D$ and so $X\in E\subseteq F=D$ as required.
    \qed

\end{proof}

\begin{lemm*}

    If $C$ is a chain of proper filters over $I$, then $D=\bigcup_{F\in C}F$ is a proper filter over $I$.

\end{lemm*}

\begin{proof}

    Obviously $I\in D$, and if $X,Y\in D$ then there exist $F_1,F_2\in C$ such that $X\in F_1$ and $Y\in F_2$, we can assume that $F_1\subseteq F_2$ in which case $X,Y\in F_2$ and so
    $X\cap Y\in F_2\subseteq D$.
    And finally if $X\in D$ and $X\subseteq Z$, then $X\in F\in C$, and so $Z\in F$ meaning $F\in D$, so $D$ is indeed a filter.
    If $\varnothing\in D$, then $\varnothing\in F$ for some $F\in C$, but $C$ is a chain of proper filters so this cannot be.
    Thus $\varnothing\notin D$ meaning $D$ is proper.
    \qed

\end{proof}

\begin{thrm*}

    If $E\subseteq\powsetof I$ and $E$ has the finite intersection property, then there exists an ultrafilter $D$ over $I$ such that $E\subseteq D$.

\end{thrm*}

\begin{proof}

    Let $F$ be the filter generated by $E$, it is proper since $E$ has the finite intersection property.
    Let
    \[ \sS = \set{F\subset\powsetof I}[F\text{ is a proper filter and }E\subseteq F] \]
    then let $C$ be a chain in $\sS$, then by the above lemma $\bigcup_{F\in C}F$ is a proper filter over $I$, and it obviously contains $E$.
    Thus every chain in $\sS$ has an upper bound in $\sS$ and so by Zorn's Lemma $\sS$ has a maximal element, $D$.
    Therefore $E\subseteq D$ and $D$ is a maximal proper filter over $I$ meaning $D$ is an ultrafilter containing $E$.
    \qed

\end{proof}

\begin{coro*}

    Any proper filter over $I$ can be extended to an ultrafilter over $I$.

\end{coro*}

This is because every proper filter has the finite intersection property.

\begin{defn*}

    If $I$ is a non-empty set and $\set{A_i}_{i\in I}$ is a family of sets, recall that
    \[ C = \prod_{i\in I}A_i \]
    is the set of all function $f\colon I\longto\bigcup_{i\in I}A_i$ such that for every $i\in I$, $f(i)\in A_i$.

    Now suppose $D$ is a proper filter over $I$, we say that $f,g\in C$ are \ppemph{$D$-equivalent} if the set of all $i\in I$ such that $f(i)=g(i)$ is an element of $D$, ie
    \[ \set{i\in I}[f(i)=g(i)] \in D] \]
    and we denote this by $f\equiv_D g$.

\end{defn*}

\begin{prop*}

    The relation $\equiv_D$ is an equivalence relation over $C$.

\end{prop*}

\begin{proof}

    Since $\set{i\in I}[f(i)=f(i)] = I\in D$ since $D$ is a filter, we have that $f\equiv_Df$, so the relation is reflexive.
    Obviously the relation is symmetric.
    Now suppose $f\equiv_Dg$ and $g\equiv_Dh$, then
    \[ \set{i\in I}[f(i) = h(i)] \supseteq \set{i\in I}[f(i) = g(i)]\cap\set{i\in I}[g(i)=h(i)] \]
    and since both of the sets on the right hand side are in $D$, so is their intersection and so $\set{i\in I}[f(i)=h(i)]$ contains a set in $D$ and thus is itself contained in $D$ since $D$ is a filter.
    So $f\equiv_Dh$, so $\equiv_D$ is transitive as required.
    \qed

\end{proof}

\begin{defn*}

    If $f\in C$, let $f_D$ be the equivalence class of $f$ under $\equiv_D$:
    \[ f_D = \set{g\in C}[f\equiv_Dg] \]
    We then define the \ppemph{reduced product of $A_i$ modulo $D$} to be the set of all equivalence classes of $\equiv_D$, it is denoted by $\prod_DA_i$:
    \[ \prod_DA_i = \set{f_D}[f\in\prod_{i\in I}A_i] \]
    or in other words, $\prod_DA_i$ is the partition of $\prod_{i\in I}A_i$ under $\equiv_D$.

    If $D$ is an ultrafilter $\prod_DA_i$ is called the \ppemph{ultraproduct of $A_i$ modulo $D$}.

    If all of the sets $A_i$ are equal to $A_i=A$, then $\prod_DA_i$ is called the \ppemph{reduced power of $A$ modulo $D$} and is written $\prod_DA$.
    If $D$ is an untrafilter then $\prod_DA$ is called the \ppemph{ultrapower of $A$ modulo $D$}.

\end{defn*}

\begin{defn*}

    Suppose $I$ is a non-empty set and $\mL$ a signature and for every $i\in I$ let $\mA_i$ be an $\mL$-structure, we use the convention that the domain of $\mA_i$ is understood to be $A_i$.
    Let $D$ be a proper filter over $I$, we define the \ppemph{reduced (filtered) product} $\mA=\prod_D\mA_i$ to be an $\mL$-interpretation whose domain is $\prod_DA_i$ and
    \blist
        \item If $P$ is an $n$-ary relation in $\mL$ then
        \[ P^\mA(f_D^1,\dots,f_D^n) \text{ if and only if } \set{i\in I}[P^{\mA_i}(f^1(i),\dots,f^n(i))] \in D \]
        this is well defined since if $f^k\equiv_D g^k$ for each $k$ then if $\set{i\in I}[P^{\mA_i}(f^1(i),\dots,f^n(i))]\in D$ then 
        \[ \set{i\in I}[P^{\mA_i}(f^1(i),\dots,f^n(i))]\cap\set{i\in I}[\forall 1\leq k\leq n\colon f^k(i)=g^k(i)] \in D \]
        as the intersection of sets in $D$.
        And this is a subset of $\set{i\in I}[P^{\mA_i}(g^1(i),\dots,g^n(i))]$, meaning that\hfil\break $\set{i\in I}[P^{\mA_i}(g^1(i),\dots,g^n(i))]\in D$ as required.

        \item Using the notation that $(a_i)_{i\in I}$ is the function $f\in\prod_{i\in I}A_i$ where $f(i)=a_i$, if $F$ is an $n$-ary function in $\mL$ then
        \[ F^\mA(f^1_D,\dots,f^n_D) = \Bigr[\bigl(F^{\mA_i}(f^1(i),\dots,f^n(i))\bigr)_{i\in I}\Bigr]_D \]
        the equivalence class under the relation $\equiv_D$.

        This is well defined since if $f^k\equiv_D g^k$ for all $1\leq k\leq n$ then we must show that
        \[ \bigl(F^{\mA_i}(f^1(i),\dots,f^n(i))\bigr)_{i\in I} \equiv_D \bigl(F^{\mA_i}(g^1(i),\dots,g^n(i))\bigr)_{i\in I} \]
        this is true because the set where these two functions are equivalent is a superset of\hfil\break
        $\set{i\in I}[\forall 1\leq k\leq n\colon f^k(i)=g^k(i)]$ which is in $D$ as the intersection of sets in $D$.
        So this definition is also well-defined.

        \item Since constants are just $0$-ary functions, the interpretation of constants inherits from the definition above:
        \[ c^\mA = \bigl(c^{\mA_i}\bigr)_{i\in I} \]
    \elist

\end{defn*}

\begin{thrm*}[expansionTheorem,The\ Expansion\ Theorem]

    Let $\mL'$ be an extension of the signature $\mL$.
    Let $I$ be a non-empty set, and let $\mA_i$ be an $\mL$-structure for each $i\in I$ and $\mB_i$ be an extension of $\mA_i$ to an $\mL'$-structure.
    Let $D$ be a proper filter over $I$ then $\prod_D\mB_i$ is an extension of $\prod_D\mA_i$.

\end{thrm*}

\begin{proof}

    Since the domain of $\mA_i$ and $\mB_i$ are the same, $A_i=B_i$ the domain of the reduced products are the same.
    Since $\mB_i$ is an extension of $\mA_i$, each symbol in $\mL$ has the same interpretation in $\mA_i$ as it does in $\mB_i$.
    Since the interpretations of symbols in $\mL$ by $\prod_D\mB_i$ depends only on the interpretations of those symbols by the $\mB_i$s, which is the same as the interpretations of the symbols in the
    $\mA_i$s, it follows that the interpretations of the symbols in $\prod_D\mB_i$ is the same as the interpretations in $\prod_D\mA_i$.
    \qed

\end{proof}

\begin{thrm*}[FTOU,The\ Fundamental\ Theorem\ of\ Ultraproducts]

    Let $\mA$ be the ultraproduct $\prod_D\mA_i$ and let $I$ be the indexing set.
    Then
    \benum
        \item For any $\mL$-term $t(x_1,\dots,x_n)$ and $f_D^1,\dots,f_D^n\in\mA$,
        \[ t^\mA(f^1_D,\dots,f^n_D) = \Bigl[\bigl(t^{\mA_i}(f^1(i),\dots,f^n(i))\bigr)_{i\in I}\Bigr]_D \]
        \item For any $\mL$-formula $\phi(x_1,\dots,x_n)$, and $f_D^1,\dots,f_D^n\in\mA$,
        \[ \mA\vDash\phi(f_D^1,\dots,f_D^n) \text{ if and only if } \set{i\in I}[\mA_i\vDash\phi(f^1_D(i),\dots,f^n(i)) \in D] \]
        \item For any $\mL$-sentence $\phi$,
        \[ \mA\vDash\phi\iff\set{i\in I}[\mA_i\vDash\phi]\in D \]
    \eenum

\end{thrm*}

\begin{proof}

    \benum
        \item We will do this by term induction.
        If $t=x$ is a variable then all we must show is that
        \[ f_D = \Bigl[\bigl(f^1(i)\bigr)_{i\in I}\Bigr]_D \]
        which is true because $(f^1(i))_{i\in I}$ is precisely $f$.
        And if $t=c$ is a constant then this is a direct result of the definition of $c^\mA$.

        Now if
        \[ t(x_1,\dots,x_n) = F(t_1(x_1,\dots,x_n),\dots,x_m(x_1,\dots,x_n)) \]
        then
        \[ t^\mA(f^1_D,\dots,f^n_D) = F^\mA(t_1^\mA(f_D^1,\dots,f_D^n),\dots,t_m^\mA(f_D^1,\dots,f_D^n)) \]
        By our inductive assumption
        \[ t_k^\mA(f_D^1,\dots,f_D^n) = g^k_D \]
        where
        \[ g^k = \bigl(t_k^{\mA_i}(f^1(i),\dots,f^n(i))\bigr)_{i\in I} \]
        and so
        \[ t^\mA(f^1_D,\dots,f^n_D) = F^\mA(g_D^1,\dots,g_D^n) \]
        And by definition
        \[ F^\mA(g_D^1,\dots,g_D^n) = \Bigl[\bigl(F^{\mA_i}(g^1(i),\dots,g^n(i))\bigr)_{i\in I}\Bigr]_D \]
        And again by definition
        \[ t^{\mA_i}(f_D^1(i),\dots,f_D^n(i)) = F^{\mA_i}(t_1^{\mA_i}(f_D^1(i),\dots,f_D^n(i)),\dots,t_m^{\mA_i}(f_D^1(i),\dots,f_D^n(i))) = F^{\mA_i}(g^1(i),\dots,g^n(i)) \]
        And so we get that
        \[ t^\mA(f^1_D,\dots,f^n_D) = \Bigl[\bigl(t^{\mA_i}(f_D^1(i),\dots,f_D^n(i))\bigr)_{i\in I}\Bigr]_D \]
        as required.

        \item The proof for atomic formulas is similar to the proof for $(1)$.
        We proceed inductively, suppose $\phi=\neg\psi$ then
        \[ \mA\vDash\phi(f^1_D,\dots,f^n_D) \iff \mA\nvDash\psi(f^1_D,\dots,f^n_D) \iff \set{i\in I}[\mA_i\vDash\psi(f^1(i),\dots,f^n(i))] \notin D \]
        and since $D$ is an ultrafilter this is if and only if
        \[ \iff \set{i\in I}[\mA_i\nvDash\psi(f^1(i),\dots,f^n(i))] \in D \iff \set{i\in I}[\mA_i\vDash\phi(f^1(i),\dots,f^n(i))] \]
        as required.

        The step for formulas of the form $\phi\land\psi$ is simple, knowing that $X\cap Y\in D$ if and only if $X\in D$ and $Y\in D$ (this is true for filters in general).

        Now suppose $\phi(x_1,\dots,x_n)=\exists x_0\psi(x_0,x_1,\dots,x_n)$, then $\mA\vDash\phi(f^1_D,\dots,f^n_D)$ if and only if there exists an $f^0_D\in\mA$ such that
        $\mA\vDash\psi(f^0_D,\dots,f^n_D)$ which inductively is if and only if $\set{i\in I}[\mA_i\vDash\psi(f^0(i),\dots,f^n(i))]\in D$.
        And so if this holds then we get that since $\mA_i\vDash\psi(f^0(i),\dots,f^n(i)$, we have $\mA_i\vDash\exists x_0\psi(x_0,f^1(i),\dots,f^n(i))$ and so $\mA_i\vDash\phi(f^1(i),\dots,f^n(i))$ and so
        \[ \set{i\in I}[\mA_i\vDash\phi(f^1(i),\dots,f^n(i))] \in D \]
        as required.

        And if
        \[ \set{i\in I}[\mA_i\vDash\phi(f^1(i),\dots,f^n(i))] \in D \]
        then for each $i\in I$ we can choose $a_i\in\mA_i$ such that $\mA_i\vDash\psi(a_i,f^1(i),\dots,f^n(i))$ and define $f^0(i)=a)_i$ and so we have
        \[ \set{i\in I}[\mA_i\vDash\psi(f^0(i),\dots,f^n(i))] \]
        is a superset of the above set and is therefore also in $D$.
        Thus as shown above this means $\mA\vDash\phi(f^1_D,\dots,f^n_D)$ as required.

        \item This is a particular result of the previous part.
        \qed
    \eenum

\end{proof}

\begin{coro*}

    For any structure $\mA$ and ultrafilter $D$, $\prod_D\mA\equiv\mA$.

\end{coro*}

This is because if $\phi$ is an $\mL$-sentence then
\[ \prod_D\mA\vDash\phi \iff \set{i\in I}[\mA\vDash\phi] \in D \]
which is if and only if $\mA\vDash\phi$ (since if the set is in $D$ it cannot be empty so $\mA\vDash\phi$ and if $\mA\vDash\phi$ then the set above is equal to $I$).

We can use the \ppref{FTOU} to provide an alternative proof of the compactness theorem:

\begin{coro*}

    Let $\Sigma$ be a set of sentences of $\mL$, and let $I=\powsetof[\omega]\Sigma$, the set of all finite subsets of $\Sigma$.
    Then if every for every $i\in I$, $i$ is satisfiable by $\mA_i$ then there exists an ultrafilter $D$ over $I$ such that $\prod_D\mA_i$ models $\Sigma$.

\end{coro*}

\begin{proof}

    For each $\sigma\in\Sigma$ let $\hat\sigma$ be the set of all $i\in I$ such that $\sigma\in I$:
    \[ \hat\sigma = \set{i\in I}[\sigma\in i] \]
    ie. $\hat\sigma$ is the set of all finite subsets of $\Sigma$ for which $\sigma$ is an element of.
    Then let
    \[ E = \set{\hat\sigma}[\sigma\in\Sigma] \]
    $E$ has the finite intersection property since if $\sigma_1,\dots,\sigma_n\in\Sigma$ then $\set{\sigma_1,\dots,\sigma_n}\in\hat\sigma_k$ for each $1\leq k\leq n$.

    Thus $E$ can be extended to an ultrafilter $D$ (since it generates a proper filter which can be extended to an ultrafilter).
    If $i\in\hat\sigma$ then $\sigma\in i$ meaning that $\mA_i\vDash\sigma$.
    Thus
    \[ \hat\sigma \subseteq \set{i\in I}[\mA_i\vDash\sigma] \]
    and since $\hat\sigma\in E\subseteq D$, we have that $\set{i\in I}[\mA_i\vDash\sigma]\in D$.
    By \ppref{FTOU}, this means that $\prod_D\mA_i\vDash\sigma$ for all $\sigma\in\Sigma$, and thus $\prod_D\mA_i\vDash\Sigma$ as required.
    \qed

\end{proof}

We now discuss classes of structures, these are many times proper classes.

\begin{defn*}

    Suppose $\mK$ is a class of $\mL$-structures, then
    \blist
        \item $\mK$ is an \ppemph{elementary class} if there exists an $\mL$-theory $T$ such that $\mK$ is precisely all the models of $T$.
        \item $\mK$ is a \ppemph{basic elementary class} if there exists an $\mL$-sentence $\phi$ such that $\mK$ is precisely all the models which satisfy $\phi$.
        \item $\mK$ is \ppemph{closed under elementary equivalence} if $\mA\in\mK$ and $\mA\equiv\mB$ then $\mB\in\mK$.
        \item $\mK$ is \ppemph{closed under ultraproducts} if for every family of structures in $\mK$, $\set{\mA_i}_{i\in I}$, and ultrafilter $D$ over $I$, $\prod_D\mA_i\in\mK$.
    \elist

\end{defn*}

\begin{thrm*}

    Let $\mK$ be a class of $\mL$-structures.
    Then
    \benum
        \item $\mK$ is an elementary class if and only if $\mK$ is closed under ultraproducts and elementary equivalence.
        \item $\mK$ is a basic elementary class if and only if both $\mK$ and the complement of $\mK$ are closed under ultraproducts and elementary equivalence.
    \eenum

\end{thrm*}

\begin{proof}

    \benum
        \item If $\mK$ is an elementary class, then it is obviously closed under elementary equivalence.
        And if $\prod_D\mA_i$ is an ultraproduct of structures in $\mK$, then since $\set{i\in I}[\mA_i\vDash\phi]=I\in D$, $\prod_D\mA_i\vDash\phi$.
        And so if $\mA_i\vDash T$ for all $i\in I$ then $\prod_D\mA_i\vDash T$ and so since $\mK$ is an elementary class this means that the ultraproduct is in $\mK$.

        Now suppose $\mK$ is closed under elementary equivalence and ultraproducts.
        Let $T$ be the theorey of all $\mL$-sentences which hold in every $\mA\in\mK$.
        Then $\mK$ is a class of models of $T$.
        Now suppose $\mB$ models $T$, let $\Sigma$ be the set of $\mL$-sentences true in $\mB$ and let $I=\powsetof[\omega]\Sigma$.
        Then for every $i=\set{\sigma_1,\dots,\sigma_n}\in I$, there exists a structure $\mA_i\in\mK$ which models $i$, as otherwise every $\mA\in\mK$ satisfies $\phi=\neg(\sigma_1\land\cdots\land\sigma_n)$.
        And thus by definition $\phi\in T$ and so $\mB\vDash\phi$, which is a contradiction since $\phi$ is false in $\mB$.

        And so by above, there exists an ultrafilter $D$ such that $\prod_D\mA_i\vDash\Sigma$, and since $\mK$ is closed under ultraproducts, $\prod_D\mA_i\in\Sigma$.
        And since the ultraproduct models $\Sigma$, it is elementarily equivalent to $\mB$, and since $\mK$ is closed under elementary equivalence, $\mB\in\mK$.
        So $\mK$ is the class of all models of $T$, and is therefore an elementary class as required.

        \item If $\mK$ is a basic elementary class then $\mK$ and $\mK^c$ are elementary classes ($\mK^c$ is all the models which satisfy $\neg\phi$), and so by above they are both closed under
        ultraproducts and elementary equivalence.

        Suppose $T_1$ is the theory of $\mK$ and $T_2$ that of $\mK^c$.
        Then let $T=T_1\cup T_2$, if $\mA\vDash T$ then $\mA\vDash T_1$ and $\mA\vDash T_2$ which means $\mA\in\mK$ and $\mA\in\mK^c$ which is a contradiction.
        Thus $T$ is unsatisfiable and therefore there exists a $\phi\in T_1$ and $\neg\phi\in T_2$.
        Let us define $T'=\set\phi$, and so if $\mA\in\mK$ then it obviously satisfies $\phi\in T_1$, and thus $T'$.
        And if $\mA\vDash T'$ then $\mA\nvDash\neg\phi$ which means $\mA\notin\mK^c$, so $\mA\in\mK$ meaning $T'=\set\phi$ is the theory of $\mK$, so $\mK$ is a basic elementary class.
        \qed
    \eenum

\end{proof}

Let $\mA$ be a structure and $\prod_D\mA$ an ultrapower where $D$ is an ultrafilter over $I$.
The \emph{natural embedding} of $\mA$ into $\prod_D\mA$ is defined by
\[ d(a) = \bigl[(a)_{i\in I}\bigr]_D \]

\begin{coro*}

    The natural embedding is an elementary embedding.

\end{coro*}

\begin{proof}

    Let $\phi(x_1,\dots,x_n)$ be an $\mL$-formula and $a_1,\dots,a_n\in\mA$ then
    \[ \prod_D\mA\vDash \phi(d(a_1),\dots,d(a_n)) \iff \set{i\in I}[\mA\vDash\phi(a_1,\dots,a_n)] \in D \iff \mA\vDash\phi(a_1,\dots,a_n) \]
    where the last equivalence is because if not then the set is empty, and if so then the set is equal to $I\in D$.
    \qed

\end{proof}

\begin{thrm*}[keislerShelahIsomorphismTheorem,Keisler-Shelah\ Isomorphism\ Theorem]

    If $\mA$ and $\mB$ are $\mL$-structures then $\mA$ and $\mB$ are elementarily equivalent if and only if there exists an ultrafilter $D$ such that
    \[ \prod_D\mA \cong \prod_D\mB \]

\end{thrm*}

\end{document}

