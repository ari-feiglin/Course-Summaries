\documentclass[10pt]{article}

\usepackage{amsmath, amssymb, mathtools}
\usepackage[margin=1.5cm]{geometry}

\font\tensy=cmsy10
\newfam\scrfam
\textfont\scrfam=\tensy
\def\mathscr#1{{\fam=\scrfam#1}}

\input pdfmsym
\input prettyprint
\input preamble

\pdfmsymsetscalefactor{10}
\initpps

\@Arrow@def{varLeftRightarrow}\@Larrow\@Rarrow{1}
\def\implies{\,\longvarRightarrow\,}
\def\iff{\,\longvarLeftRightarrow\,}
\let\eiff=\varLeftRightarrow
\let\to=\varrightarrow
\let\longto=\longvarrightarrow
\let\oto=\varleftrightarrow

\let\nor=\downarrow
\let\nand=\uparrow

\def\Language{\mathcal{L}}
\def\theory{\mathcal{T}}

\def\qed{\hskip.5cm\hbox{}\hfill$\blacksquare$}

\def\pmat#1{\begin{pmatrix} #1 \end{pmatrix}}

\def\mB{\mathbb{B}}
\def\true{\mathsf{true}}
\def\false{\mathsf{false}}

\def\Aone#1#2{#1\to(#2\to#1)}
\def\Atwo#1#2#3{\bigl(#1\to(#2\to#3)\bigr)\to\bigl((#1\to#2)\to(#1\to#3)\bigr)}
\def\Athree#1#2{(\neg#2\to\neg#1)\to\bigl((\neg#2\to#1)\to#2\bigr)}

\let\divides=\mid

\font\bigbf = cmbx12 scaled 2000

\parskip=5pt plus 1pt minus 3pt

\begin{document}

\c@section=3

\barcolorbox{220, 255, 255}{0, 130, 130}{80, 200, 200}{
    \leftskip=0pt plus 1fill \rightskip=\leftskip
    {\bigbf Mathematical Logic}

    \medskip
    \textit{Lecture \thesection, Thursday April 20, 2023}

    \textit{Ari Feiglin}
}

\bigskip

\subsection{Soundness and Completeness}

We prove some corollaries of the deduction theorem:

\begin{coro*}

    For any well-formed formulas $\phi$ and $\psi$ of $\Language$:

    \benum
        \item $\phi\to\psi, \phi\to\mu\vdash\phi\to\mu$
        \item $\phi\to(\psi\to\mu), \psi\vdash\phi\to\mu$
    \eenum

\end{coro*}

\begin{proof}

    \benum
        \item By the deduction theorem, this is equivalent to proving
        \[ \phi\to\psi, \phi\to\mu, \phi \vdash \mu \]
        Since we have $\phi$ and $\phi\to\psi$, by modus ponens we have $\psi$ and since $\psi\to\mu$, we have $\mu$ as required.
        \item Again, this is equivalent to proving
        \[ \phi\to(\psi\to\mu), \psi, \phi \vdash \mu \]
        And by modus ponens we have $\psi\to\mu$ and again by modus ponens we have $\mu$ as required.\qed
    \eenum

\end{proof}

\begin{lemm*}[wfProofsLemma]

    For any well-formed formulas $\phi$ and $\psi$ of $\Language$, the following are theorems:

    \benum
        \item $\neg\neg\phi\to\phi$
        \item $\phi\to\neg\neg\phi$
        \item $\neg\phi\to(\phi\to\psi)$
        \item $(\neg\psi\to\neg\phi)\to(\phi\to\psi)$
        \item $(\phi\to\psi)\to(\neg\psi\to\neg\phi)$
        \item $\phi\to\bigl(\neg\psi\to\neg(\phi\to\psi)\bigr)$
        \item $(\phi\to\psi)\to\bigl((\neg\phi\to\psi)\to\psi)$
    \eenum

\end{lemm*}

\begin{proof}

    \benum
        \item $\neg\neg\phi\to\phi$:
        \benum
            \item $(\neg\phi\to\neg\neg\phi)\to\bigl((\neg\phi\to\neg\phi)\to\phi$; Axiom $3$ (\textbf{A3}) for $\phi$ and $\neg\phi$.
            \item $\neg\phi\to\neg\phi$; Lemma 2.2.3.
            \item $(\neg\phi\to\neg\neg\phi)\to\phi$; \enumstyle1, \enumstyle2, Corollary 3.1.1 (2).
            \item $\neg\neg\phi\to(\neg\phi\to\neg\neg\phi)$; \textbf{A1} for $\neg\phi$ and $\neg\neg\phi$.
            \item $\neg\neg\phi\to\phi$; \enumstyle3, \enumstyle4, Corollary 3.1.1 (1).
        \eenum
        \item $\phi\to\neg\neg\phi$
        \benum
            \item $(\neg\neg\neg\phi\to\neg\phi)\to\bigl((\neg\neg\neg\phi\to\phi)\to\neg\neg\phi)$; \textbf{A3} for $\phi$ and $\neg\neg\phi$.
            \item $\neg\neg\neg\phi\to\neg\phi$; Part (1).
            \item $(\neg\neg\neg\phi\to\phi)\to\neg\neg\phi$; \enumstyle1, \enumstyle2, modus ponens (\textit{MP}).
            \item $\phi\to(\neg\neg\neg\phi\to\phi)$; \textbf{A3} for $\phi$ and $\neg\neg\phi$.
            \item $\phi\to\neg\neg\phi$; \enumstyle3, \enumstyle4, Corollary 3.1.1 (1).
        \eenum
        \item $\neg\phi\to(\phi\to\psi)$
        \benum
            \item $\neg\phi$; hypothesis (meaning we are showing $\neg\phi\vdash\phi\to\psi$).
            \item $\phi$; hypothesis.
            \item $\phi\to(\neg\psi\to\phi)$; \textbf{A1}.
            \item $\neg\phi\to(\neg\psi\to\neg\phi)$; \textbf{A1}.
            \item $\neg\phi\to\psi$; \enumstyle2, \enumstyle3, \textit{MP}.
            \item $\neg\psi\to\neg\phi$; \enumstyle1, \enumstyle4, \textit{MP}.
            \item $(\neg\psi\to\neg\phi)\to\bigl((\neg\psi\to\phi)\to\psi\bigr)$; \textbf{A3}.
            \item $(\neg\psi\to\phi)\to\psi$; \enumstyle6, \enumstyle7, \textit{MP}.
            \item $\psi$; \enumstyle5, \enumstyle8, \textit{MP}.
            \item $\phi,\neg\phi\vdash\psi$; \enumstyle1--\enumstyle9.
            \item $\neg\phi\vdash\phi\to\psi$; \enumstyle{10}, deduction theorem.
            \item $\neg\phi\to(\phi\to\psi)$; \enumstyle{11}, dedcution theorem.
        \eenum
        \item $(\neg\psi\to\neg\phi)\to(\phi\to\psi)$
        \benum
            \item $\neg\psi\to\neg\phi$; hypothesis.
            \item $\Athree\phi\psi$; \textbf{A3}.
            \item $\Aone\phi{\neg\psi}$; \textbf{A1}.
            \item $(\neg\psi\to\phi)\to\psi$; \enumstyle1, \enumstyle2, \textit{MP}.
            \item $\phi\to\psi$; \enumstyle3, \enumstyle4, Corollary 3.1.1 (1).
            \item $\neg\psi\to\neg\phi\vdash\phi\to\psi$; \enumstyle1--\enumstyle5.
            \item $(\neg\psi\to\neg\phi)\to(\phi\to\psi)$; \enumstyle6, deduction theorem.
        \eenum
        \item $(\phi\to\psi)\to(\neg\psi\to\neg\phi)$
        \benum
            \item $\phi\to\psi$; hypothesis.
            \item $\neg\neg\phi\to\phi$; part (1).
            \item $\neg\neg\phi\to\psi$; \enumstyle1, \enumstyle2, Corollary 3.1.1 (1).
            \item $\psi\to\neg\neg\psi$; part (4).
            \item $\neg\neg\phi\to\neg\neg\psi$; \enumstyle3, \enumstyle4, Corollary 3.1.1 (1).
            \item $(\neg\neg\phi\to\neg\neg\psi)\to(\neg\psi\to\neg\phi)$; part (4).
            \item $\neg\psi\to\neg\phi$; \enumstyle5, \enumstyle6, \textit{MP}.
            \item $\phi\to\psi\vdash\neg\psi\to\neg\phi$; \enumstyle1--\enumstyle7.
            \item $(\phi\to\psi)\to(\neg\psi\to\neg\phi)$; \enumstyle8, deduction theorem.
        \eenum
        \item $\phi\to(\neg\psi\to\neg\bigl(\phi\to\psi)\bigr)$
        \benum
            \item $\phi\to\psi, \phi\vdash\psi$; this is clear by \textit{MP}.
            \item $\phi\to\bigl((\phi\to\psi)\to\psi\bigr)$; \enumstyle1, deduction theorem (twice).
            \item $\bigl((\phi\to\psi)\to\psi\bigr)\to\bigl(\neg\psi\to\neg(\phi\to\psi)\bigr)$; \enumstyle2, part (5).
            \item $\phi\to(\neg\psi\to\neg\bigl(\phi\to\psi)\bigr)$; \enumstyle2, \enumstyle3, Corollary 3.1.1 (1).
        \eenum
        \item $(\phi\to\psi)\to\bigl((\neg\phi\to\psi)\to\psi\bigr)$
        \benum
            \item $\phi\to\psi$; hypothesis.
            \item $\neg\phi\to\psi$; hypothesis.
            \item $(\phi\to\psi)\to(\neg\psi\to\neg\phi)$; part (5).
            \item $\neg\psi\to\neg\phi$; \enumstyle1, \enumstyle3, \textit{MP}.
            \item $(\neg\phi\to\psi)\to(\neg\psi\to\neg\neg\phi)$; part (5).
            \item $\neg\psi\to\neg\neg\phi$; \enumstyle2, \enumstyle5, \textit{MP}.
            \item $\Athree{\neg\phi}\psi$; \textbf{A3}.
            \item $(\neg\psi\to\neg\phi)\to\psi$; \enumstyle6, \enumstyle7, \textit{MP}.
            \item $\psi$; \enumstyle4, \enumstyle8, \textit{MP}.
            \item $\phi\to\psi, \neg\phi\to\psi \vdash\psi$; \enumstyle1--\enumstyle9.
            \item $(\phi\to\psi)\to\bigl((\neg\phi\to\psi)\to\psi\bigr)$; \enumstyle{10}, deduction theorem (twice).\qed
        \eenum
    \eenum

\end{proof}

\begin{defn*}

    We define the following connectives as shorthands:
    \benum
        \item $(\phi\land\psi)$ for $\neg(\phi\to\neg\psi)$.
        \item $(\phi\lor\psi)$ for $(\neg\phi)\to\psi$.
        \item $(\phi\oto\psi)$ for $(\phi\to\psi)\land(\psi\to\phi)$, meaning $\neg\bigl((\phi\to\psi)\to\neg(\psi\to\phi)\bigr)$.
    \eenum

\end{defn*}

\begin{exercise*}

    Show the following:
    \benum
        \item $\phi\to(\phi\lor\psi)$
        \item $\phi\to(\psi\lor\phi)$
        \item $(\phi\lor\psi)\to(\psi\lor\phi)$
        \item $(\phi\land\psi)\to\phi$
        \item $(\phi\land\psi)\to\psi$
        \item $(\phi\to\mu)\to\Bigl((\psi\to\mu)\to\bigl(\phi\lor\psi)\to\mu\bigr)\Bigr)$
        \item $\bigl((\phi\to\psi)\to\phi\bigr)\to\phi$
        \item $\phi\to\bigl(\psi\to(\phi\land\psi)\bigr)$
    \eenum

\end{exercise*}

Our goal now is to show that a well-formed formula of $\Language$ is a theorem if and only if it is a tautology in the sense of statement forms.
The first part of this is simple.

\begin{thrm*}[soundnessTheorem,Soundness\ Theorem]

    Every theorem of $\Language$ is a tautology.

\end{thrm*}

\begin{proof}

    It can be shown with relative ease that every axiom of $\Language$ is a tautology.
    Given some theorem $\phi$, we must have a proof of length $n$, which we induct on.
    For $n=1$, $\phi$ is an axiom.
    Otherwise, either $\phi$ is an we must have that $\psi\to\phi$ and $\psi$ are well-formed formulas in the proof.
    Thus they can both be proven in fewer than $n$ steps and by our inductive hypothesis are thus tautologies.
    Therefore since $\psi$ is always true and $\psi\to\phi$ is always true, we can infer that $\phi$ is always true (a tautology).
    This last step takes place entirely in the world of statement forms/boolean functions.
    \qed

\end{proof}

What this means is that propositional calculus is \emph{sound}: everything that can be proven is true.
We know continue with the other half.

\begin{lemm*}

    Let $\phi$ be a well-formed formula and $B_1,\dots,B_k$ the statement letters which occur in $\phi$.
    For some assignment of truth values to these statement letters, define $B'_j$ to be $B_j$ if it is true, and $\neg B_j$ if it is false.
    Then let $\phi'$ be $\phi$ if it is true under this assignment, and $\neg\phi$ if $\phi$ is false.
    Then
    \[ B'_1,\dots,B'_k\vdash \phi' \]

\end{lemm*}

\begin{proof}

    We induct on $n$, the number of occurrences of $\neg$ or $\to$ in $\phi$ (we assume that $\phi$ is written without shorthands).
    If $n=0$, then $\phi$ is just a single statement letter $\phi=B_1$ then this reduces to $B_1\vdash B_1$ and $\neg B_1\vdash\neg B_1$ which are both trivial.

    Otherwise, we split into two cases:

    For the first case, $\phi=\neg\psi$.
    Then $\psi$ has fewer than $n$ occurrences of $\neg$ and $\to$.
    Under the given truth value assignments, we again have two possibilities: if $\psi$ is true then $\phi$ is false.
    Thus $\psi'$ is $\psi$ and $\phi'=\neg\phi=\neg\neg\psi$.
    By our inductive hypothesis
    \[ B'_1,\dots,B'_n \vdash \psi' = \psi \]
    By \ppref[lemma]{wfProofsLemma} (2), $\psi\to\neg\neg\psi$ so
    \[ B'_1,\dots,B'_n \vdash \neg\neg\psi = \phi' \]
    as required.
    And if $\psi$ is true, then $\phi$ is true and $\psi'$ is $\neg\psi$ and $\phi'$ is $\phi$, so $\psi'=\phi$.
    And by our inductive hypothesis:
    \[ B'_1,\dots,B'_n \vdash \psi' = \phi \]
    as required.

    For the second case, $\phi=\psi\to\mu$.
    We have three possibilities here: if $\psi$ is false then $\phi$ takes on the value true.
    So $\phi'=\neg\phi$ and $\phi'=\phi$, so by our inductive hypothesis:
    \[ B'_1,\dots,B'_n \vdash \neg\psi \]
    and by \ppref[lemma]{wfProofsLemma} (3), $\neg\psi\to(\psi\to\mu)$, so
    \[ B'_1,\dots,B'_n \vdash \psi\to\mu = \phi \]
    as required.
    And if $\mu$ is true then again $\phi$ takes the value true.
    So $\mu'=\mu$ and $\phi'=\phi$ and so by our inductive hypothesis
    \[ B'_1,\dots,B'_n \vdash \mu \]
    and by \textbf{A1} $\mu\to(\psi\to\mu)$, so $\mu\to\phi$ so by modus ponens
    \[ B'_1,\dots,B'_n \vdash \phi \]
    as required.
    The final possibility is that $\psi$ is true and $\mu$ is false, then $\phi$ is false.
    So $\psi'=\psi$ and $\mu'=\neg\mu$ and $\phi'=\neg\phi=\neg(\psi\to\mu)$, thus by our inductive hypothesis
    \[ B'_1,\dots,B'_n \vdash \psi, \neg\mu \]
    By \ppref[lemma]{wfProofsLemma} (6) $\psi\to\bigl(\neg\mu\to\neg(\psi\to\mu)\bigr)$, thus by applying modus ponens twice we have $\neg(\psi\to\mu)=\neg\phi=\psi'$ as required.
    \qed

\end{proof}

\begin{thrm*}[completenessTheorem,Completeness\ Theorem]

    If a well-formed formula $\phi$ of $\Language$ is a tautology, then it is a theorem of $\Language$.

\end{thrm*}

\begin{proof}

    Let $B_1,\dots,B_n$ be the statement letters in $\phi$.
    For any truth value assignment to these letters, we have $B'_1,\dots,B'_n\vdash\phi$ since $\phi'=\phi$ as $\phi$ is always true.
    So when $B_n=\true$ we have $B'_1,\dots,B_n,B_n\vdash\phi$ and when $B_n=\false$ we have $B'_1,\dots,B'_{n-1},\neg B_n\vdash\phi$, so by the deduction theorem
    \[ B'_1,\dots,B'_{n-1}\vdash (B_n\to\phi),\, (\neg B_n\to\phi) \]
    for any truth value assignment to $B_1,\dots,B_{n-1}$.
    Thus by \ppref[lemma]{wfProofsLemma} (7) we have $(B_n\to\phi)\to\bigl((\neg B_n\to\phi)\to\phi\bigr)$ and so applying modus ponens twice gives
    \[ B'_1,\dots,B'_n \vdash \phi \]
    We can continue inductively and after $n$ steps we have
    \[ \vdash \phi \]
    as required.
    \qed

\end{proof}

\begin{coro*}

    If $\psi$ is an expression involving the signs $\neg$, $\to$, $\lor$, $\land$, and $\oto$ which is a shorthand for a well-formed formula $\phi$ of $\Language$, then $\psi$ is a tautology if and only if
    $\phi$ is a theorem of $\Language$.

\end{coro*}

\begin{proof}

    Since $\psi$ is a tautology if and only if $\phi$ is (it remains an exercise to show that the definitions of the shorthands are logically equivalent to the connectives) by the \ppref{soundnessTheorem},
    if $\phi$ is a theorem then it is a tautology and therefore so is $\psi$.
    And if $\psi$ is a tautology then so is $\phi$ and therefore $\phi$ is a theorem by the \ppref{completenessTheorem}.
    \qed

\end{proof}

\end{document}

