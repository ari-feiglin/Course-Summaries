\documentclass[10pt]{article}

\usepackage{amsmath, amssymb, mathtools}
\usepackage[margin=1.5cm]{geometry}

\font\tensy=cmsy10
\newfam\scrfam
\textfont\scrfam=\tensy
\def\mathscr#1{{\fam=\scrfam#1}}

\input pdfmsym
\input prettyprint
\input preamble

\pdfmsymsetscalefactor{10}
\initpps
\newmathpp{axiom}{Axiom}{255,255,200}{150,100,0}{200,150,100}			% yellow

\@Arrow@def{varLeftRightarrow}\@Larrow\@Rarrow{1}
\def\implies{\,\longvarRightarrow\,}
\def\iff{\,\longvarLeftRightarrow\,}
\let\eiff=\varLeftRightarrow
\let\to=\varrightarrow
\let\longto=\longvarrightarrow
\let\oto=\varleftrightarrow
\let\injection=\longvaruphookrightarrow

\let\nor=\downarrow
\let\nand=\uparrow

\def\aone#1#2{#2\to(#1\to#2)} \def\Aone{\textbf{A1}}
\def\atwo#1#2#3{\bigl(#1\to(#2\to#3)\bigr)\to\bigl((#1\to#2)\to(#1\to#3)\bigr)} \def\Atwo{\textbf{A2}}
\def\athree#1#2{(\neg#2\to\neg#1)\to\bigl((\neg#2\to#1)\to#2\bigr)} \def\Athree{\textbf{A3}}
\def\afour#1#2#3{(\forall #2#1)\to#1\frac{#3}{#2}}  \def\Afour{\textbf{A4}}
\def\afive#1#2#3{\bigl(\forall #3(#1\to#2)\bigr)\to\bigl(#1\to(\forall#3#2)\bigr)} \def\Afive{\textbf{A5}}
\def\MP{\textit{MP}} \def\Gen{\textit{Gen}}

\def\beqq{\mathrel{\vbox{\hbox{$=$}\kern-\baselineskip\kern.3pt\hbox{$=$}}}}

\def\mL{\mathcal{L}}
\def\mT{\mathcal{T}}
\def\mM{\mathcal{M}}
\def\mD{\mathcal{D}}
\def\fM{\mathfrak{M}}
\def\fC{\mathfrak{C}}
\def\fm{\mathfrak{m}}
\def\fn{\mathfrak{n}}

\def\Var{\mathrm{Var}}

\newfunc{var}{{\rm var}}({})
\newfunc{bound}{{\rm bound}}({})
\newfunc{free}{{\rm free}}({})

\def\qed{\hskip.5cm\hbox{}\hfill$\blacksquare$}

\def\pmat#1{\begin{pmatrix} #1 \end{pmatrix}}

\def\mB{\mathbb{B}}
\def\true{\mathsf{true}}
\def\false{\mathsf{false}}

%\def\Aone#1#2{#1\to(#2\to#1)}
%\def\Atwo#1#2#3{\bigl(#1\to(#2\to#3)\bigr)\to\bigl((#1\to#2)\to(#1\to#3)\bigr)}
%\def\Athree#1#2{(\neg#2\to\neg#1)\to\bigl((\neg#2\to#1)\to#2\bigr)}

\let\divides=\mid

\font\bigbf = cmbx12 scaled 2000

\parskip=5pt plus 1pt minus 3pt
\pp@headerskip=\z@

\begin{document}

\c@section=8

\barcolorbox{220, 255, 255}{0, 130, 130}{80, 200, 200}{
    \leftskip=0pt plus 1fill \rightskip=\leftskip
    {\bigbf Mathematical Logic}

    \medskip
    \textit{Lecture \thesection, Monday May 29, 2023}

    \textit{Ari Feiglin}
}

\bigskip

Recall the axioms of a first order theory:
\blist
    \item First order theories inherit axioms from predicate calculus.
    \item (\textbf{A4}): If $\phi\frac tx$ is collision-free then  $(\forall x\phi)\to\phi\parens{\frac tx}$.
    \item (\textbf{A5}): If $x$ is not free in $\phi$ then $(\forall x)(\phi\to\psi)\to(\phi\to(\forall x)\psi)$.
    \item Other axioms are unique to the first order theory.
        A first order theory with no other axioms is called a \emph{predicate calculus}.
\elist

And the rules of inference are
\blist
    \item Modus ponens (\textit{MP}).
    \item Generalization (\textit{Gen}): if $\phi$ then $\forall x\phi$.
\elist

\begin{defn*}

    Let $\Gamma$ be an $\mL$-theory.
    $\Gamma$ is \ppemph{contradictory} if there exists a well-formed formula $\phi$ such that $\Gamma\vdash\phi$ and $\Gamma\vdash\neg\phi$.
    A non-contradictory theory is also known as \ppemph{consistent}.

    A set of formulas $\Gamma$ is \ppemph{satisfiable} if it has a model.

\end{defn*}

\begin{prop*}

    Every satisfiable set of closed formulas is consistent.

\end{prop*}

\begin{proof}

    Since the rules of inference are valid under satisfiability, if $\mM$ is a model of $\Gamma$, any formula which can be proven using formulas from $\Gamma$ must be modeled by $\mM$.
    Formally, this is shown by induction on the length of the proof.
    For a length of $0$, this is simply an element of $\Gamma$ which is by definition satisfiable.
    If the proof of $\phi$ is $n$ steps, $\mM$ models all the formulas used previously in the proof by induction, and since we must now use a rule of inference using only these formulas and formulas from
    $\Gamma$ which are all satisfiable, so must $\phi$.
    Again, this is due to the rules of inference being valid for satisfiability.
 
    So if $\Gamma\vdash\phi,\neg\phi$ then $\mM\models\phi,\neg\phi$ which is a contradiction to the definition of satisfiability.
    \qed

\end{proof}

\begin{defn*}

    If $x\neq y\in\Var$ and $\phi$ is a formula, then $\phi$ and $\phi\frac yx$ are \ppemph{similar} if the substitution is collision-free and $\phi$ has no free instances of $y$.

\end{defn*}

So for example if $\phi=\forall x_3\bigl(A_1(x_3,x_1)\lor A_2(x_1)\bigr)$ then $\phi\frac{x_2}{x_1}=\forall x_3\bigl(A_1(x_3,x_2)\lor A_2(x_2)\bigr)$ and these are similar since the substitution is
collision-free and $\phi$ contains no (free) instances of $x_2$.

We will write $\phi(x)$ and $\phi(y)$ for $\phi$ and $\phi\frac yx$.

\begin{lemm*}

    If $\phi(x)$ and $\phi(y)$ are similar then $\vdash(\forall x\phi(x))\oto(\forall y\phi(y))$.

\end{lemm*}

\begin{proof}

    Recall that this is just $\vdash(\forall x\phi)\oto(\forall y\phi\frac yx)$.

    $$ \blproof
        \afour\phi xy & \Afour\cr
        \forall y\bigl(\afour\phi xy\bigr) & \Gen\cr\noalign{\kern3pt}
        \afive{\forall x\phi}{\phi\frac yx}y & \Afive\cr
        (\forall x\phi)\to(\forall y\phi\frac yx) & \MP
    \elproof $$

    We similarly have $(\forall y\phi\frac yx)\to(\forall x\phi)$ (take $\psi=\phi\frac yx$ then $\phi=\psi\frac xy$ and $\phi$ and $\psi$ are similar).
    Thus as we showed for propositional calculus (which this has boiled down to): $\vdash(\forall x\phi)\oto(\forall y\phi\frac yx)$.
    \qed

\end{proof}

\begin{lemm*}

    Suppose $K$ is a consistent.
    If $\neg\phi$ is a closed formula and is not provable in $K$, then $K'$, the theory obtained by adding $\phi$ as a new axiom to $K$, is also consistent.

\end{lemm*}

\begin{proof}

    Suppose $K'$ is contradictory, then $\vdash_{K'}\psi,\neg\psi$.
    Again, this is simply propositional calculus, so $\vdash_{K'}\neg\phi$.
    Thus we have $\phi\vdash_K\neg\phi$ as we can regard $\phi$ as a hypothesis instead of an axiom in the proof of $\neg\phi$ in $K'$.
    Since $\phi$ is closed, $\vdash_K\phi\to\neg\phi$ and since $\vdash_K(\phi\to\neg\phi)\to\neg\phi$, and so by modus ponens $\vdash_K\neg\phi$ in contradiction.
    \qed

\end{proof}

\begin{defn*}

    A theory $K$ is \ppemph{complete} if for every formula $\phi$ either $\vdash_K\phi$ or $\vdash_K\neg\phi$.

\end{defn*}

\begin{defn*}

    An \ppemph{extension} of a theory $K$ is a theory $K'$ where if $\vdash_K\phi$ then $\vdash_{K'}\phi$ (everything provable in $K$ is provable in $K'$).
    In other words, $K\subseteq K'$ (as theories are seen as the set of their theorems).

\end{defn*}

\begin{prop*}

    If $K$ is inconsistent, then for any formula $\psi$, $\vdash_K\psi$.

\end{prop*}

\begin{proof}

    Since $\vdash_K\phi$ and $\vdash_K\neg\phi$ for some formula $\phi$, by \Aone{} we have
    \[ \aone{\neg\psi}\phi,\quad\aone{\neg\psi}{\neg\phi} \]
    Since we can prove $\phi$ and $\neg\phi$, by \MP{} we have $\neg\psi\to\phi$ and $\neg\psi\to\neg\phi$.
    and by \Athree{} we have
    \[ \athree\phi\psi \]
    so by applying \MP{} twice we get $\psi$, as required.
    \qed

\end{proof}

\begin{prop*}

    If $K$ is a consistent theory, then there is a consistent and complete extension of $K$.

\end{prop*}

\begin{proof}

    Let $\set{\phi_n}_{n=1}^\infty$ be an enumeration of all the closed well-formed formulas of the language of the theory.
    We define a sequence of theories $\set{J_n}_{n=0}^\infty$ recursively, where $J_0=K$ and $J_{n+1}$ is defined where if $\vdash_{J_n}\neg\phi_{n+1}$ then $J_{n+1}=J_n$.
    Otherwise let $J_{n+1}$ be obtained from $J_n$ by adding $\phi_{n+1}$ as an axiom to $J_n$.

    Let $J$ be the theory obtained by taking all the axioms in any $J_n$.
    If $\vdash_J\phi,\neg\phi$ then since these proofs must be finite, all the theorems must be contained within a theory $J_n$, and so $J_n$ is inconsistent.
    Thus it is sufficient to show that every $J_n$ is consistent.
    We prove the consistency of $J_n$ via induction: since $J_0=K$ is consistent by assumption, and either $J_{n+1}=J_n$ (in which case $J_{n+1}$ is consistent) or $J_{n+1}$ is obtained by adding
    $\phi_{n+1}$ as an axiom to $J_n$ where $\nvdash_{J_n}\neg\phi_{n+1}$ so by the lemma above, $J_{n+1}$ is consistent.

    $J$ is clearly complete as $\vdash_{J_n}\phi_n$ for every $n$, and it is obviously an extension of $K$.
    \qed

\end{proof}

\begin{defn*}

    A \ppemph{closed term} is a term without variables (a constant or the function of constants).
    A theory $K$ is called a \ppemph{scapegoat theory} if for every formula $\phi$ where $\free\phi=\set x$ there is a closed term $t$ such that
    \[ \vdash_K(\exists x\phi)\to\phi\frac tx \]
    $t$ is called the \ppemph{scapegoat} of $\phi$.

\end{defn*}

Meaning if $K$ is a scapegoat theory, then if $\phi$ is true for some $x$ then it is true if we substitute any constant value for $x$.

\begin{lemm*}

    Any consistent theory $K$ has a consistent extension $K'$ which is a scapegoat theory.

\end{lemm*}

\begin{proof}

    Let $\set{\phi_n}_{n=1}^\infty$ be an enumeration of the formulas in $K$ with exactly one free variable.
    We recursively define a sequence of theories starting with $K_0=K$.
    Let $x_n$ be the free variable of $\phi_n$.
    If $\phi_n$ has a scapegoat, then $K_n=K_{n-1}$.
    Otherwise we can extend $K_{n-1}$ by defining a new constant $b_n$ and add it to the signature of $K_{n-1}$, and we add the additional axiom $(\exists x_n\neg\phi_n)\to\phi_n\frac{b_n}{x_n}$, we will
    denote this axiom as $X(b_n)$.
    This gives a theory $K_n$.

    We now define $K'$ by taking the union of all the signatures of $K_n$ and all the axioms in any $K_n$.
    We claim that $K'$ is a consistent scapegoat theory.
    As before, it is sufficient to show that $K_n$ is consistent for each $n$ in order to show $K'$ is consistent.

    We prove the consistency of $K_n$ inductively, the base case is true since the assumption of the lemma is that $K=K_0$ is consistent.
    So if $K_n$ is consistent, let us assume that $K_{n+1}$ is not, so then $\vdash_{K_{n+1}}\neg X(b_{n+1})$.
    So $\neg X(b_{n+1})$ is deducible from $X(b_{n+1})$.
    If we replace all instance of $b_{n+1}$ in this proof with a variable $y$ not present in the proof, then we get a valid deduction from $X(y)$ to $\neg X(y)$, where
    $X(y)=(\exists y\phi_n)\to\phi_n\frac y{x_n}$.
    If we let $U$ be $\exists x_n\neg\phi_n$ and $V$ be $\neg\phi_n\frac y{x_n}$, so $X(y)=U\to V$ and so we have that
    \[ U\to V\vdash_{K_n}\neg(U\to V) \]
    We recall that this means that $U\to V\vdash U$ and $U\to V\vdash\neg V$ ($\vdash$ meaning $\vdash_{K_n}$).
    Since $y$ is the only free variable in $U$ and is never generalized (since $b_{n+1}$ was not a variable), the deduction theorem holds and we have
    \[ \vdash (U\to V)\to V, \vdash (U\to V)\to\neg V \]
    but note that $\neg U\to\neg\bigl((U\to V)\to V)$ and so $\bigl((U\to V)\to U\bigr)\to U$ is a tautology, so we have $\vdash U$ by \MP.
    But as this includes a predicate calculus, $V$ is true or false, but the formula $(U\to V)\to\neg V$ is false if $V$ is true, so $V$ is false.
    Meaning $\vdash_{K_n}\neg V$ so $\vdash_{K_n}\phi_n\frac y{x_n}$, and by generalizing $\forall y(\phi_n\frac y{x_n})$.
    But we know $\vdash_{K_n}U=\exists x_n\neg\phi_n$ which means that $K_n$ is inconsistent, in contradiction.

    And of course $K$ is a scapegoat theory.
    \qed

\end{proof}

\begin{lemm*}

    Let $K$ be a consistent, complete scapegoat theory.
    Then $K$ has a model $\mM$ whose domain is the set $D$ of closed terms in $K$.

\end{lemm*}

\begin{proof}

    If $c$ is a constant symbol in $K$, then let $c^\mM=c$.
    This is well-defined as $c$ is a closed term, so is in $D$.
    If $f$ is a function symbol then define $f^\mM$ by setting $f^\mM(t_1,\dots,t_n)=f(t_1,\dots,t_n)$ for $t_1,\dots,t_n$ closed terms of $K$.
    And if $P$ is a predicate symbol, then $P^\mM(t_1,\dots,t_n)$ if and only if $\vdash_K P(t_1,\dots,t_n)$.

    We will show that for any closed formula $\phi$, $\mM\vDash\phi$ if and only if $\vdash_K\phi$.
    This is sufficient since if $\psi$ is an axiom of $K$, then let $\phi$ be a closure of $\psi$ (prepending $\forall x$s before $\psi$), then by \Gen{} we have $\vdash_K\phi$ and so $\mM\vDash\phi$.
    And since $\phi$ is the closure of $\psi$, this means that $\mM\vDash\psi$ as required.

    We prove this by formula induction.
    If $\phi=P(t_1,\dots,t_n)$ for closed $t_i$ (these must be closed terms as $\phi$ is closed), then this is a direct result of the definition of $\phi^\mM$.

    If $\phi=\neg\psi$, since $\phi$ is closed so too must $\psi$ be.
    Then if $\mM\vDash\psi$ we have that $\mM\neg\vDash\psi$ and so by our inductive hypothesis, $\nvdash\psi$.
    Since $K$ is complete, this means that $\vdash\neg\psi$ as required.
    And if $\vdash\neg\psi$, since $K$ is consistent this means that $\nvdash\psi$ so $\mM\nvDash\psi$ by our inductive hypothesis, so $\mM\vDash\neg\psi$ as required.

    If $\phi=\psi\to\mu$, similar as before $\psi$ and $\mu$ must be closed.
    So $\mM\nvDash\phi$ if and only if $\mM\vDash\psi$ and $\mM\nvDash\mu$, by our inductive hypothesis this is if and only if $\vdash\psi$ and $\nvdash\mu$.
    Since $K$ is complete and consistent, this is if and only if $\vdash\psi$ and $\nvdash\mu$.
    By applying the tautology $A\to(\neg B\to\neg(A\to B))$ and using \MP{} twice we get $\vdash\neg(\psi\to\mu)=\vdash\neg\phi$, and by the consistency of $K$ this means $\nvdash\phi$ as required.
    And if $\nvdash\phi$ then $\vdash\neg(\psi\to\mu)$ which means $\vdash\phi$ and $\nvdash\mu$ which we showed means $\mM\nvDash\phi$ as required.

    If $\phi=\forall x\psi$, then we split into two more cases.
    If $\psi$ is closed, then $\mM\vDash\forall x\psi$ if and only if $\mM\vDash\psi$, so $\mM\vDash\phi$ if and only if $\mM\vDash\psi$ if and only if $\vdash\psi$.
    And since $\psi$ is closed, $\vdash\psi$ if and only if $\vdash\forall x\psi$ (one direction is due to \Gen{} and the other is due to \Afour{} with $t=x$).
    So $\mM\vDash\forall x\psi$ if and only if $\vdash\psi$ if and only if $\vdash\forall x\psi$ as required.

    If $\psi$ is not closed, since $\phi$ is closed this means that $\free(\psi)=\set x$.
    Suppose $\mM\vDash\phi$ but not $\vdash\phi$, so $\vdash\neg\phi=\neg(\forall x\psi)$ so $\vdash\exists x\neg\psi$.
    Since $K$ is a scapegoat theory this means $\vdash\neg\psi\frac tx$ for some closed term $t$.
    But since $\mM\vDash\phi$ we have $\mM\vDash\forall x\psi$, and $\forall x\psi\to\psi\frac tx$ is true for $\mM$ so we have $\mM\vDash\psi\frac tx$, so by induction $\vdash\psi\frac tx$.
    But this contradicts the consistency of $K$, so $\vdash\phi$ as required.

    And for the converse, suppose $\vdash\phi$ but not $\mM\vDash\phi$.
    So there must be some valuation $w$ of $D$ such that $w$ does not satisfy $\psi$, let $t=w(x)$.
    Then $\psi\frac tx$ is closed (since $\free\psi=\set x$) and so by induction $\mM\vDash\psi\frac tx$ if and only if $\vdash\psi\frac tx$.
    But by \Afour{} we know that $\vdash\psi\frac tx$, so $\mM\vDash\psi\frac tx$.
    This means that $w$ satisfies $\psi$, since this means $\mM^t_x\vDash\psi$ and all variables other than $x$ are not free in $\psi$.
    But this is a contradiction to $w$ not satisfying $\psi$.
    \qed

\end{proof}

\begin{prop*}

    Every consistent theory $K$ has a countable model.

\end{prop*}

\begin{proof}

    We know $K$ has an extension $K'$ which is a consistent scapegoat theory.
    $K'$ also has an extension to $J$ which is a complete consistent theory with the same symbols as $K'$, and is therefore a complete, consistent, scapegoat theory.
    By above this means that $J$ has a model whose domain is the set of all closed terms, which is countable, so $J$ has a countable model.
    Since $J$ is an extension of $K$, this model models $K$ as well as required.
    \qed

\end{proof}

\begin{coro*}

    Any logically valid formula $\phi$ of a theory $K$ is a theorem of $K$.

\end{coro*}

\begin{proof}

    We can assume $\phi$ is closed since $\phi$ is logically valid if and only if its closure is, and $\phi$ is provable if and only if its closure is.
    If $\nvdash_K\phi$ then we can obtain a consistent theory $K'$ by adding $\neg\phi$ as an axiom.
    So $K'$ has a model $\mM$, so $\mM\vDash\neg\phi$ since $\neg\phi$ is an axiom and $\mM\vDash\phi$ since $\mM$ is a model of $K$ as well and $\phi$ is logically valid.
    But this is a contradiction to the definition of the satisfiability of $\neg\phi$.
    \qed

\end{proof}

Of course the converse is not true: you can have an axiom of $K$ which is not logically valid (not true in every model).
But if there are no additional axioms, then it is true:

\begin{thrm*}[godelTheorem,Godel\ Completeness\ Theorem\ (1930)]

    In any predicate calculus, the theorems are precisely the logically valid formulas.

\end{thrm*}

The first direction (theorem implies logically valid) is obvious as the rules of inference and axioms are all logically valid, and the second direction was proven right above this theorem.

\begin{defn*}

    If $K$ is a theory and $\Gamma$ is a set of formulas, we denote $K+\Gamma$ as the theory obtained by adding $\Gamma$ as axioms to the existing axioms of $K$.
    In the case of a single formula, we also write $K+\phi$.

\end{defn*}

\begin{coro*}

    Let $K$ be a theory, then
    \benum
        \item A formula $\phi$ is true in every countable model of $K$ if and only if $\vdash_K\phi$.
        \item If every model of a set of formulas $\Gamma$ satisfies $\phi$, then $\Gamma\vdash_K\phi$.
    \eenum

\end{coro*}

\begin{proof}

    \benum
        \item We may assume that $\phi$ is closed, in which case if $\nvdash_K\phi$ then $K'=K+\neg\phi$ is consistent.
            Thus $K'$ has a countable model $\mM$, and so $\mM\vDash\neg\phi$.
            But $\mM$ is a model of $K$ as well, and since $\phi$ is true in any countable model so $\mM\vDash\phi$ in contradiction.
            The converse is immediate since if $\vdash_K\phi$, then since the rules of inference are valid for models, any model of $K$ must also model $\phi$.

        \item Let $K'=K+\Gamma$, then for every model of this theory $\mM$, we have $\mM\vDash\phi$ and so by above this means $\vdash_{K+\Gamma}\phi$, ie. $\Gamma\vdash_K\phi$.
            \qed
    \eenum

\end{proof}

\begin{coro*}[lowenheimSkolemThrm,Lowenheim-Skolem\ Theorem]

    Any theory that has a model has a countable model.

\end{coro*}

\begin{proof}

    If $K$ has a model, it must be consistent and therefore must have a countable model.
    \qed

\end{proof}

\begin{lemm*}

    If $\fm$ and $\fn$ are two cardinal numbers where $\fm\geq\fn$, and a theory $K$ has a model of cardinality $\fm$ then $K$ has a model of cardinality $\fn$.

\end{lemm*}

\begin{proof}

    Let $\mM$ be a model of $K$ with domain $D$ of cardinality $\fm$.
    Let $D\subseteq D'$ have cardinality $\fn$, and extend $\mM$ to $\mM'$ with domain $D'$ like so: fix an element $c\in D$ and have every $d'\in D'\setminus D$ behave like $c$ in the context of
    functions and predicates.
    Formally, define $W\colon D'\to D$ where $W(d)=d$ for $d\in D$ and $W(d')=c$ for $d'\in D'\setminus D$, then $f^{\mM'}(d_1,\dots,d_n)=f^\mM(W(d_1),\dots,W(d_n))$ and $P^{\mM'}(d_1,\dots,d_n)$ if and
    only if $P^\mM(W(d_1),\dots,W(d_n))$.
    The valuation function of $\mM'$ is equal to that of $\mM$'s.

    We claim that $\mM'\vDash K$.
    Since the valuation of $\mM'$ is equal to that of $\mM$ this is obvious.
    \qed

\end{proof}

\begin{coro*}

    For any infinite cardinal number $\fm\geq\aleph_0$, any consistent theory $K$ has a model of cardinality $\fm$.

\end{coro*}

\begin{proof}

    Since $K$ is consistent, it has a countable model and so by the lemma above $K$ has a model of cardinality $\fm$ as required.
    \qed

\end{proof}

\begin{thrm*}[compactThrm,Compactness\ Theorem]

    A theory $K$ is satisfiable if and only if every finite subset of $K$ is satisfiable.

\end{thrm*}

\begin{proof}

    If $K$ is satisfiable, every subset of $K$ is satisfiable.
    To show the converse, suppose $K$ is not satisfiable, then $K$ is inconsistent (since every consistent theory has a model).
    Let $\sigma$ be a proof of a contradiction in $K$.
    Because $\sigma$ is finite, it uses only a finite number of axioms from $K$, and if we take the set of all these axioms as $K_0$ then $K_0$ is inconsistent and therefore is not satisfiable.
    So if $K$ is not satisfiable then there exists a finite subset which is also not satisfiable.
    \qed

\end{proof}

\begin{prop*}

    Let $\mL$ be a language containing $\set{\cdot,e}$ the language of groups.
    Let $T$ be an $\mL$-theory extending the theory of groups, and let $\phi(v)$ be an $\mL$-formula.
    Suppose that for every $n$ there is a model (group) $G_n$ such that $G_n\vDash T$ and $g_n\in G_n$ with finite order greater than $n$ such that $G_n\vDash\phi(g_n)$.
    Then there is $G\vDash T$ such that $G\vDash\phi(g)$ and $g$ has infinite order.

\end{prop*}

In particular, there is no formula that defines the torsion points in every model of $G$.
This is because if there was a formula $\phi(v)$ which did, we can take $G_n=\bZ_{n+1}$ and $g_n=1$.
But then $G\vDash\phi(g)$ where $g$ has infinite order.

\begin{proof}

    Let $\mL^*=\mL\cup\set c$ where $c$ is a new constant symbol.
    Let $T^*$ be the $\mL^*$-theory
    \[ T\cup\set{\phi(c)}\cup\set{c\cdot c\cdots c\neq e\quad (c^n\neq e)}[n=1,2,\dots] \]
    If $G$ is a model of $T^*$ and $g$ is the valuation of $c$ in $G$ then $G\vDash\phi(g)$ and $g$ has infinite order.
    So it is sufficient to show that $T^*$ is consistent, and therefore has a model.

    Let $\Delta\subseteq T^*$ be finite.
    Then
    \[ \Delta \subseteq T\cup\set{\phi(c)}\cup\set{c\cdots c\neq e}[n=1,\dots,m] \]
    for some $m$.
    View $G_m$ as an $\mL^*$-structure by interpreting $c$ as $g_m$.
    Since $G_m\vDash T\cup\set{\phi(g_m)}$ and $g_m$ has order greater than $m$, we have $G_m\vDash\Delta$ and so $\Delta$ is consistent.

    So by the compactness theorem, $T^*$ is consistent as required.
    \qed

\end{proof}

\newfunc{Th}{{\it Th}}({})

\begin{prop*}

    Let $\mL=\set{\cdot,+,<,0,1}$ and let $\Thof\bN$ be the full $\mL$-theory of the naturals (ie. all first order sentences which are true for $\bN$).
    Then there is a model $\mM$ of $\Thof\bN$ such that there exists an $a\in\mM$ which is larger than every natural number (a number which is a finite sum of $1$s).

\end{prop*}

\begin{proof}

    Let $\mL^*=\mL\cup\set c$ where $c$ is a new constant symbol and let
    \[ T = \Thof\bN\cup\set{1+\cdots+1<c\quad (n\text{ times})}[n=1,2,\dots] \]
    We claim that $T$ is consistent.
    Let $\Delta\subseteq T$ be finite, then there exists an $m$ such that
    \[ \Delta\subseteq\Thof\bN\cup\set{1+\cdots+1<c\quad (n\text{ times})}[n=1,\dots,m] \]
    Then $\bN$ is a model of $\Delta$ by valuating $c$ as $m+1$ for example.
    So $\Delta$ is consistent and therefore by the compactness theorem, so is $T$.
    Therefore $\mM$ has a model, and if $a\in\mM$ is the valuation of $c$ then $a$ is larger than every natural number, as required.
    \qed

\end{proof}

\end{document}

