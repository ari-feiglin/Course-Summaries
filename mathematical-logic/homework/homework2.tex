\documentclass[10pt]{article}

\usepackage{amsmath, amssymb, mathtools}
\usepackage[margin=1.5cm]{geometry}

\font\tensy=cmsy10
\newfam\scrfam
\textfont\scrfam=\tensy
\def\mathscr#1{{\fam=\scrfam#1}}

\input pdfmsym
\input prettyprint
\input ../preamble

\pdfmsymsetscalefactor{10}
\initpps

\newcount\lproofline
\def\blproof{%
    \lproofline=0\relax%
    \vtop\bgroup\ialign\bgroup\global\advance\lproofline by 1\relax\the\lproofline.\quad$##$\hfil\tabskip=10pt&##\hfil\crcr
}
\def\elproof{\crcr\omit\leaders\hrule\hfill\cr\egroup\egroup}


\@Arrow@def{varLeftRightarrow}\@Larrow\@Rarrow{1}
\def\implies{\,\longvarRightarrow\,}
\def\iff{\,\longvarLeftRightarrow\,}
\let\eiff=\varLeftRightarrow
\let\to=\varrightarrow
\let\oto=\varleftrightarrow
\let\longto=\longvarrightarrow

\let\nor=\downarrow
\let\nand=\uparrow

\def\mL{{\cal L}}

\def\qed{\hskip.5cm\hbox{}\hfill$\blacksquare$}

\def\pmat#1{\begin{pmatrix} #1 \end{pmatrix}}

\def\bB{\mathbb{B}}
\def\true{\mathsf{true}}
\def\false{\mathsf{false}}

\let\divides=\mid

\font\bigbf = cmbx12 scaled 2000

\parskip=5pt plus 1pt minus 3pt

\def\@ppmathcount{\thesection.\thepp@mathcount}

\def\bexerc{\begin{exercise*}}
\def\eexerc{\end{exercise*}}
\def\bblank{\begin{blankpp}}
\def\eblank{\end{blankpp}}

\begin{document}

\c@section=2

\barcolorbox{220, 255, 255}{0, 130, 130}{80, 200, 200}{
    \leftskip=0pt plus 1fill \rightskip=\leftskip
    {\bigbf Mathematical Logic}

    \medskip
    \textit{Assignment \thesection}

    \textit{Ari Feiglin}
}

\bigskip

\bexerc

    Prove the independence of Axiom $3$ in propositional calculus by constructing truth tables for $\neg$ and $\to$.

\eexerc

\bblank

    The idea is to treat $\neg$ as if it did nothing, and since \textbf{A3} is the only axiom relying on negation, this should show its independence.
    The truth tables are as follows:

    \centerline{%
        \vtop{\halign{\strut\hfil\;$#$\;\hfil\vrule&\hfil\;$#$\;\hfil\cr
            A & \neg A \cr\noalign{\hrule}
            0 & 0 \cr
            1 & 1 \cr
        }}%
        \qquad
        \vtop{\halign{\strut\hfil\;$#$\;\hfil&&\vrule\hfil\;$#$\;\hfil\cr
            A & B & A\to B\cr\noalign{\hrule}
            0 & 0 & 1 \cr
            0 & 1 & 1 \cr
            1 & 0 & 0 \cr
            1 & 1 & 1 \cr
        }}%
    }

    Since \textbf{A1} and \textbf{A2} rely only on implication, and we have not changed the  truth table of implication, \textbf{A1} and \textbf{A2} still hold true in this model.
    And so does modus ponens since it is also only reliant on implication (if $\phi=1$ and $\phi\to\psi$ is true, then $\psi=1$ as well since if $\psi=0$ then $\phi\to\psi$ is false as it is $1\to0=0$).

    But \textbf{A3} does not hold.
    For instance let $\psi=0$ and $\phi=1$, then \textbf{A3} says
    \[ (\neg\psi\to\neg\phi)\to\bigl((\neg\psi\to\phi)\to\psi\bigr) \]
    and plugging in the values for $\psi$ and $\phi$, and recalling that $\neg x=x$ we get
    \[ (0\to1)\to\bigl((0\to1)\to0\bigr) \]
    since $0\to1$ is $1$ and $1\to0=0$
    \[ 1\to(1\to0) = 1\to0 = 0 \]
    thus \textbf{A3} is false in this model.

\eblank

\bexerc

    Of the axioms for linear spaces (listed in the original PDF), which is dependent of the other axioms?

\eexerc

\bblank

    The axiom of commutativity of addition is dependent, since for any $x,y\in V$ we have
    \[ x + y + x + y = (x+y) + (x+y) = (1 + 1)(x + y) \]
    by distributivity.
    But again by distributivity this is equal to
    \[ (1 + 1)x + (1 + 1)y = x + x + y + y \]
    so
    \[ x + y + x + y = x + x + y + y \]
    and since additive inverses exist, adding $-x$ to the left and $-y$ to the right sides results in
    \[ y + x = x + y \]
    as required.

\eblank

\def\Aone#1{\textbf{A1}: $#1$}
\def\Atwo#1#2{\textbf{A2}: $#1$, $#2$}
\def\Athree#1#2{\textbf{A3}: $#1$, $#2$}
\def\Afour#1#2#3{\textbf{A4}: $#1$, $#2$, $#3$}

\def\aone#1{(#1\lor#1)\to#1}
\def\atwo#1#2{#1\to(#1\lor#2)}
\def\athree#1#2{(#1\lor#2)\to(#2\lor#1)}
\def\afour#1#2#3{(#2\to#3)\to\bigl((#1\lor#2)\to(#1\lor#3)\bigr)}

\bexerc

    The formal language $\mL_1$ has connectives $\lor$ and $\neg$, and $\phi\to\psi$ is used as an abbreviation for $\neg\phi\lor\psi$.
    It has the following axioms

    \benum
        \gdef\enumstyle#1{$\mathbf{(A\the#1)}$}
        \gdef\enumindent{1cm}
        \item $\aone\phi$
        \item $\atwo\phi\psi$
        \item $\athree\phi\psi$
        \item $\afour\phi\psi\mu$
    \eenum

    The only rule of inference is modus ponens.

    Show the following are theorems of this theory.

    \benum
        \item $\phi\to\psi\vdash (\mu\lor\phi)\to(\mu\lor\psi)$
        \item $\vdash (\phi\to\psi)\to\bigl((\mu\to\phi)\to(\mu\to\psi)\bigr)$
        \item $\mu\to\phi,\phi\to\psi\vdash \mu\to\psi$
        \item $\vdash \phi\to\phi$
        \item $\vdash \phi\lor\neg\phi$
        \item $\phi\to\neg\neg\phi$
    \eenum

\eexerc

A quick note: I use phrases like ``\textbf{A4}: $\mu$, $\phi$, $\psi$'' to mean axiom four where I've replaced $\phi$ in \textbf{A4} with $\mu$, $\psi$ with $\phi$, and $\mu$ with $\psi$.

\bblank

    \benum
        \item%
        \blproof
            \afour\mu\phi\psi & \Afour\mu\phi\psi\cr
            \phi\to\psi & Hypothesis\cr
            (\mu\lor\phi)\to(\mu\lor\psi) & \textit{MP}: $1$ and $2$\cr
        \elproof

        \item%
        \blproof
            \afour{\neg\mu}\phi\psi & \Afour{\neg\mu}\phi\psi\cr
            (\phi\to\psi)\to\bigl((\mu\to\phi)\to(\mu\to\psi)\bigr) & Abbreviation for $\to$\cr
        \elproof

        \item%
        \blproof
            (\phi\to\psi)\to\bigl((\mu\to\phi)\to(\mu\to\psi)\bigr) & $(2)$\cr
            \phi\to\psi & Hypothesis\cr
            (\mu\to\phi)\to(\mu\to\psi) & \textit{MP}: $1$ and $2$\cr
            \mu\to\phi & Hypothesis\cr
            \mu\to\psi & \textit{MP}: $3$ and $4$\cr
        \elproof

        \item%
        \blproof
            \bigl((\phi\lor\phi)\to\phi\bigr)\to\Bigl(\bigl(\phi\to(\phi\lor\phi)\bigr)\to(\phi\to\phi)\Bigr) & $(2)$: $\phi\lor\phi$, $\phi$, $\phi$\cr
            \aone\phi & \Aone\phi\cr
            \bigl(\phi\to(\phi\lor\phi)\bigr)\to(\phi\to\phi) & \textit{MP}: $1$ and $2$\cr
            \phi\to(\phi\lor\phi) & \Atwo\phi\phi\cr
            \phi\to\phi & \textit{MP}: $3$ and $4$\cr
        \elproof

        \item%
        \blproof
            \phi\to\phi & $(4)$\cr
            \neg\phi\lor\phi & Abbreviation for $\to$\cr
            \athree{\neg\phi}\phi & \Athree{\neg\phi}\phi\cr
            \phi\lor\neg\phi & \textit{MP}: $2$ and $3$\cr
        \elproof

        \item%
        \blproof
            \neg\phi\lor\neg\neg\phi & $(5)$: $\neg\phi$\cr
            \phi\to\neg\neg\phi & Abbreviation for $\to$\cr
        \elproof
    \eenum

\eblank

\def\Aone#1{\textbf{A1}: $#1$}
\def\Atwo#1#2{\textbf{A2}: $#1$, $#2$}
\def\Athree#1#2#3{\textbf{A3}: $#1$, $#2$, $#3$}

\def\aone#1{#1\to(#1\land#1)}
\def\atwo#1#2{(#1\land#2)\to#1}
\def\athree#1#2#3{(#1\to#2)\to\bigl(\neg(#2\land#3)\to\neg(#3\land#1)\bigr)}

\bexerc

    Let $\mL_2$ be the language with connectives $\neg$ and $\land$.
    We use $\phi\to\psi$ as an abbreviation for $\neg(\phi\land\neg\psi)$.
    The axioms are as follows

    \benum
        \gdef\enumstyle#1{$\mathbf{(A\the#1)}$}
        \gdef\enumindent{1cm}
        \item $\aone\phi$
        \item $\atwo\phi\psi$
        \item $\athree\phi\psi\mu$
    \eenum

    The only rule of inference is modus ponens.

    Prove the following:

    \benum
        \item $\phi\to\psi,\psi\to\mu\vdash \neg(\neg\mu\land\phi)$
        \item $\vdash \neg(\neg\phi\land\phi)$
        \item $\vdash \neg\neg\phi\to\phi$
        \item $\vdash \neg(\phi\land\psi)\to(\psi\to\neg\phi)$
        \item $\vdash \phi\to\neg\neg\phi$
        \item $\vdash (\phi\to\psi)\to(\neg\psi\to\neg\phi)$
    \eenum

\eexerc

\bblank

    \benum
        \item%
        \blproof
            \athree\phi\psi{\neg\mu} & \Athree\phi\psi{\neg\mu}\cr
            \phi\to\psi & Hypothesis\cr
            \neg(\psi\land\neg\mu)\to\neg(\neg\mu\land\phi) & \textit{MP}: $1$ and $2$\cr
            \psi\to\mu & Hypothesis\cr
            \neg(\psi\land\neg\mu) & Abbreviation for $\to$\cr
            \neg(\neg\mu\land\phi) & \textit{MP}: $3$ and $5$\cr
        \elproof

        \item%
        \blproof
            \aone\phi & \Aone\phi\cr
            \atwo\phi\phi & \Atwo\phi\phi\cr
            \neg(\neg\phi\land\phi) & $(1)$: $\phi$, $\phi\land\phi$, $\phi$ with hypotheses given in $1$ and $2$\cr
        \elproof

        \item%
        \blproof
            \neg(\neg\neg\phi\land\neg\phi) & $(2)$: $\neg\phi$\cr
            \neg\neg\phi\to\phi & Abbreviation for $\to$\cr
        \elproof

        \item%
        \blproof
            \athree{\neg\neg\phi}\phi\psi & \Athree{\neg\neg\phi}\phi\psi\cr
            \neg\neg\phi\to\phi & $(2)$\cr
            \neg(\phi\land\psi)\to\neg(\psi\land\neg\neg\phi) & \textit{MP}: $1$ and $2$\cr
            \neg(\phi\land\psi)\to(\psi\to\neg\phi) & Abbreviation for $\to$\cr
        \elproof

        \item%
        \blproof
            \neg(\neg\phi\land\phi)\to(\phi\to\neg\neg\phi) & $(4)$: $\neg\phi$, $\phi$\cr
            \neg(\neg\phi\land\phi) & $(2)$\cr
            \phi\to\neg\neg\phi & \textit{MP}: $1$ and $2$\cr
        \elproof

        \item%
        \blproof
            \neg(\phi\land\neg\psi)\to(\neg\psi\to\neg\phi) & $(4)$: $\phi$, $\neg\psi$\cr
            (\phi\to\psi)\to(\neg\psi\to\neg\phi) & Abbreviation for $\to$\cr
        \elproof
    \eenum

\eblank

\bexerc

    For each of the following sentences, translate it into a well-formed formula of a first order language.
    \benum
        \item Not all birds can fly.
        \item If anyone can solve the problem, Mike can.
        \item Nobody in the statistics class is smarter than everyone in the logic class.
        \item Everyone loves somebody and no one loves everybody, or somebody loves everybody and somebody loves nobody.
        \item Everyone who knows Julia loves her.
        \item There is no barber who shaves precisely those men who do not shave themselves.
    \eenum

\eexerc

\bblank

    \benum
        \item The logical negation of this is ``all birds can fly'', which is easy enough to translate.
        Let $B(x)$ be a predicate meaning ``$x$ is a bird'', and $F(x)$ meaning ``$x$ can fly''.
        ``All birds can fly'' can thus be translated into $\forall x\bigl(B(x)\to F(x)\bigr)$, and so ``Not all birds can fly'' is simply:
        \[ \neg\forall x\bigl(B(x)\to F(x)\bigr) \]

        \item Let $S(x)$ mean ``$x$ can solve the problem'' and $m$ be the constant Mike.
        This sentence can be reconstructed as ``if there is anyone who can solve the problem, then Mike can solve the problem'' which can be translated as
        \[ \bigl(\exists x\, S(x)\bigr)\to S(m) \]

        \item This can be reconstructed as ``there does not exist someone who is in the statistics class and is smarter than everyone in the logic class''.
        Let $S(x)$ mean ``$x$ is in the statistics class'', $L(x)$ mean ``$x$ is in the logic class'', and $B(x,y)$ mean ``$x$ is smarter than $y$''.
        This sentence can then be translated to
        \[ \neg\exists x\bigl(S(x)\land\forall y(L(y)\to B(x,y))\bigr) \]
        as this is the negation of ``there exists an $x$ who is in the statistics class and for everyone, if they are in the logic class, $x$ is smarter than them.''

        \item Let $L(x,y)$ mean ``$x$ loves $y$''.
        Then ``everyone loves somebody'' can be translated to $\forall x\exists y(L(x,y))$; ``nobody loves everybody'' can be translated to $\neg\exists x\forall y(L(x,y))$;
        ``somebody loves everybody'' as $\exists x\forall y(L(x,y))$; and ``somebody loves nobody'' as $\exists x\forall y(\neg L(x,y))$.
        Putting this together we get
        \[ \Bigl(\bigl(\forall x\exists y(L(x,y))\bigr)\land\bigl(\forall x\exists y(\neg L(x,y))\bigr)\Bigr)\lor
           \Bigl(\bigl(\exists x\forall y(L(x,y))\bigr)\land\bigl(\exists x\forall y(\neg L(x,y))\bigr)\Bigr) \]

        \item Let $K(x)$ mean ``$x$ knows Julia'' and $L(x)$ mean ``$x$ loves Julia''.
        We can reconstruct the sentence as ``for everyone, if they know Julia, they love Julia'', which can be translated as
        \[ \forall x\bigl(K(x)\to L(x)\bigr) \]

        \item Let $B(x)$ mean ``$x$ is a barber'', $M(x)$ mean ``$x$ is a man'', and $S(x,y)$ mean ``$x$ shaves $y$''.
        Then this can be translated as
        \[ \neg\exists x\Bigl(B(x)\land \forall y\bigl(S(x,y)\oto M(y)\land\neg S(y,y)\bigr)\Bigr) \]
        Which is ``there does not exist an $x$ which is a barber and for every $y$, $x$ shaves $y$ if and only if $y$ is a man and $y$ doesn't shave himself''.
    \eenum

\eblank

\bexerc

    Translate the following into plain English.
    \benum
        \item $\neg(\exists y)\bigl(I(y)\land(\forall x)(I(x)\to L(x,y))\bigr)$ where $I(x)$ means \textit{$x$ is an integer} and $L(x,y)$ means $x\leq y$.
        \item In the following formulas, $A_1(x)$ means \textit{$x$ is a person} and $A_2(x,y)$ means \textit{$x$ hates $y$}.
        \benum
            \item $(\exists x)\bigl(A_1(x)\land(\forall y)(A_1(y)\to A_2(x,y))\bigr)$
            \item $(\forall x)\bigl(A_1(x)\to(\forall y)(A_1(y)\to A_2(x,y))\bigr)$
            \item $(\exists x)\Bigl(A_1(x)\land(\forall y)\bigl(A_1(y)\to(A_2(x,y)\oto A_2(y,y))\bigr)\Bigr)$
        \eenum
        \item $(\forall x)\Bigl(H(x)\to(\exists y)(\exists z)\Bigl(\neg A(y,z)\land(\forall u)\bigl(P(u,x)\oto(A(u,y)\lor A(u,z))\bigr)\Bigr)\Bigr)$ where $H(x)$ means \textit{$x$ is a person},
        $A(u,v)$ means $u=v$, and $P(u,x)$ means \textit{$u$ is a parent of $x$}.
    \eenum

\eexerc

\bblank

    \benum
        \item Translating this literally into English gives ``there does not exist a $y$ such that $y$ is an integer and for every $x$, if $x$ is an integer then $x\leq y$''.
        Simplifying this, ``there does not exist an integer $y$ such that for every integer $x$, $x\leq y$''.
        This just says ``there does not exist a maximum integer''.

        \item
        \par\vskip-\dimexpr\parskip + \baselineskip\relax
        \benum
            \item Translating literally gives ``there exists an $x$ such that $x$ is a person and for every $y$, if $y$ is a person, $x$ hates $y$''.
            Simplifying this, ``there exists a person $x$ such that for every person $y$, $x$ hates $y$''.
            Meaning, ``there exists a person who hates everyone'' (or ``some person hates every person'').

            \item Literally, this means ``for every $x$, if $x$ is a person then for every $y$, if $y$ is a person then $x$ hates $y$''.
            Simplifying, ``for every person $x$, for every person $y$, $x$ hates $y$''.
            Meaning, ``every person hates every person''.

            \item Literally, this means ``there exists an $x$ such that $x$ is a person and for every $y$, if $y$ is a person, $x$ hates $y$ if and only if $y$ hates $y$''.
            Meaning, ``there exists a person $x$ such that for every person $y$, $x$ hates $y$ if and only if $y$ hates $y$''.
            Meaning, ``there exists a person who hates precisely all people who hate themselves''.
        \eenum

        \item Literally, ``for every $x$, if $x$ is a person, then there exists a $y$ and there exists a $z$ such that $y\neq z$ and for all $u$, $u$ is a parent of $x$ if and only if $u=y$ or $u=z$''.
        Simplifying, ``for every person $x$, there exist $y\neq z$ such that for every $u$, $u$ is a parent of $x$ if and only if $u=x$ or $u=z$''.
        This means that ``for every person $x$, there exist $y\neq z$ such that $y$ and $z$ are and are the only parents of $x$''.
        This just means that ``every person has exactly two parents''.
    \eenum

\eblank

\end{document}
