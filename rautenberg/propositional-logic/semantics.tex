Propositional logic is the study of logic removed from interpretation of individual variables and context.
I will assume that the reader already has experience with propositional logic, as this is something an undergraduate will cover in one of their first courses.
While this subsection will focus mainly on the semantics of propositional logic, we will begin by defining its {\it syntax},

\def\PV{{\it PV}}
\def\mF{{\cal F}}
\bdefn

    Let $\PV$ be an arbitrary set of {\emphcolor propositional variables} (which are regarded as arbitrary symbols).
    {\emphcolor Propositional formulas} are formulas defined recursively by the following rules,
    \benum
        \item Propositional variables in $\PV$ are formulas, called {\emphcolor prime}\addtoindex{prime formula} or {\emphcolor atomic} formulas\addtoindex{atomic formula}%
        \alsosee{atomic formula}{Prime formula}{prime formula:}.
        \item If $\alpha$ and $\beta$ are formulas, then so are $(\alpha\land\beta)$, $(\alpha\lor\beta)$, and $\neg\alpha$.
        $(\alpha\land\beta)$ is called the {\emphcolor conjunction}\addtoindex{conjunction} of $\alpha$ and $\beta$, $(\alpha\lor\beta)$ their {\emphcolor disjunction}\addtoindex{disjunction}, and
        $\neg\alpha$ the {\emphcolor negation}\addtoindex{negation} of $\alpha$.
    \eenum
    The set of all the formulas constructed in this matter is denoted $\mF$.

\edefn

We can generalize this definition; instead of utilizing only the symbols $\land$ and $\lor$, we can take a general {\it logical signature}\addtoindex{signature}[logical] $\sigma$ consisting of logical
connectives of differing arities.
We then recursively define $\sigma$-formulas as following: if ${\tt c}$ is an $n$-ary logical connective in $\sigma$, and $\alpha_1,\dots,\alpha_n$ are formulas, then so is
$$ ({\tt c}\alpha_1,\dots,\alpha_n) $$
Aleternatively, if we only consider binary and unary connectives, then if ${\tt c}$ is a unary connective, we define ${\tt c}\alpha$ to be a formula, and if $\circ$ is a binary connective, then
$(\alpha\circ\beta)$ is a formula.
But we don't have much need for such generalizations, as $\set{\land, \lor, \neg}$ is complete, in the sense that all connectives can be defined using them.
This is a fact we will discuss soon.

We can define other connectives, for example $\to$ and $\oto$ are used as shorthands:
$$ (\alpha\to\beta) \coloneqq \neg(\alpha\land\neg\beta),\qquad (\alpha\oto\beta) \coloneqq \bigl((\alpha\to\beta)\land(\beta\to\alpha)\bigr) $$
We similarly define symbols for false and true:
$$ \bot \coloneqq (p_1\land\neg p_1),\qquad \top = \neg\bot $$

For readability, we will use the following conventions when writing formulas (this is not a change to the definition of a formula, rather conventions for writing them in order to enhance readability)
\benum
    \item We will omit the outermost parentheses when writing formulas, if there are any.
    \item The order of operations for logical connectives is as follows, from first to last: $\neg,\land,\lor,\to,\oto$.
    \item We associate $\to$ from the right, meaning $\alpha\to\beta\to\gamma$ is to be read as $\alpha\to(\beta\to\gamma)$.
    All other connectives associate from the left, for example $\alpha\land\beta\land\gamma$ is to be read as $(\alpha\land\beta)\land\gamma$.
    \item Instead of writing $\alpha_0\land\alpha_1\land\cdots\alpha_n$, we write $\bigwedge_{i=0}^n\alpha_i$, similar for $\lor$.
\eenum

Since formulas are constructed in a recursive manner, most of our proofs about them are inductive.

\bprin[title=Principle of Formula Induction, name=predforminduct]

    Let ${\cal E}$ be a property of strings which satisfies the following conditions:
    \benum
        \item ${\cal E}\pi$ for all prime formulas $\pi$,
        \item If ${\cal E}\alpha$ and ${\cal E}\beta$, then ${\cal E}(\alpha\land\beta)$, ${\cal E}(\alpha\lor\beta)$, and ${\cal E}\neg\alpha$ for all formulas $\alpha,\beta\in{\cal F}$.
    \eenum
    Then ${\cal E}\phi$ is true for all formulas $\phi$.

\eprin

An example of this is that every formula $\phi\in{\cal F}$ is either prime, or of one of the following forms
$$ \phi=\neg\alpha,\quad\phi=(\alpha\land\beta),\quad\phi=(\alpha\lor\beta) $$
The proof of this is straightforward: let ${\cal E}$ be this property.
Then trivially, ${\cal E}\pi$ for all prime formulas $\pi$.
And if ${\cal E}\alpha$ and ${\cal E}\beta$, then of course we have
$$ {\cal E}\neg\alpha,\quad{\cal E}(\alpha\land\beta),\quad{\cal E}(\alpha\lor\beta) $$
This is the first step in showing the {\it unique formula reconstruction property}.
Let us prove a lemma before proving the property itself,

\blemm

    Proper initial segments of formulas are not formulas.
    Equivalently (by contrapositive), if $\alpha$ and $\beta$ are formulas and $\alpha\xi=\beta\eta$ for arbitrary strings $\xi$ and $\eta$, then $\alpha=\beta$.

\elemm

Let us prove this by induction on $\alpha$.
If $\alpha$ is a prime formula, suppose that $\beta$ is not a prime formula, then its first character is either $($ or $\neg$, but then $\alpha=($ or $\alpha=\neg$, in contradiction.
Thus $\beta$ is a prime formula and so $\alpha=\beta$ as they are both a single character.
Now if $\alpha=(\alpha_1\circ\alpha_2)$, then the first character of $\beta$ must too be $($, so $\beta$ is of the form $(\beta_1*\beta_2)$.
Thus
$$ \alpha_1\circ\alpha_2)\xi = \beta_1*\beta_2)\eta $$
and so by our inductive assumption, $\alpha_1=\beta_1$, and so $\circ=*$, and thus $\alpha_2=\beta_2$ by our inductive assumption again.
And so $\alpha=\beta$ as required.
The proof for the case that $\alpha=\neg\alpha'$ is similar.
\qed

\bprop[title=Unique Formula Reconstruction Property, name=uniqueformrecon]

    Every compound formula $\phi\in{\cal F}$ is of one of the following forms:
    $$ \phi=\neg\alpha,\quad\phi=(\alpha\land\beta),\quad\phi=(\alpha\lor\beta) $$
    For some formulas $\alpha,\beta\in{\cal F}$.

\eprop

We have already shown existence.
We will now show that this is unique, meaning that $\phi$ can be written uniquely as one of these strings.
Using the lemma proven above, the proof for uniqueness of the reconstruction property is immediate.
For example, if $\phi=(\alpha_1\land\beta_1)$ then obviously $\phi$ cannot be written as $\neg\alpha_2$ since $(\neq\neg$, and if $\phi=(\alpha_2\lor\beta_2)$ then by the lemma $\alpha_1=\alpha_2$, and
so we get that $\land=\lor$ in contradiction.
And finally if $\phi=(\alpha_2\land\beta_2)$, then again by the lemma, $\alpha_1=\alpha_2$ and $\beta_1=\beta_2$ as required.
The proof for $\neg$ and $\lor$ are similar.
\qed

Utilizing formula recursion, we can define functions on formulas.
For example,

\def\Sf{{\rm Sf}}
\def\rank{{\rm rank}}
\def\Var{{\rm Var}}

\bdefn

    For a formula $\phi$, we define $\Sf\phi$ to be the set of all subformulas of $\phi$.
    This is done recursively:
    $$ \displaylines{\Sf\pi=\set\pi\hbox{ for prime formulas $\pi$},\cr
    \Sf\neg\alpha=\Sf\alpha\cup\set\alpha,\quad \Sf(\alpha\circ\beta)=\Sf\alpha\cup\Sf\beta\cup\set{(\alpha\circ\beta)}\hbox{\ \ for a binary logical connective $\circ$}} $$

    Similarly, we can define the {\emphcolor rank}\addtoindex{rank} of a formula $\phi$,
    $$ \displaylines{\rank\pi=0\hbox{ for prime formulas $\pi$},\cr
    \rank\neg\alpha=\rank\alpha+1,\quad \rank(\alpha\circ\beta)=\maxof{\rank\alpha,\,\rank\beta}+1\hbox{\ \ for a binary logical connective $\circ$}} $$

    And we can also define the set of variables in $\phi$,
    $$ \displaylines{\Var\pi=\set\pi\hbox{ for prime formulas $\pi$},\cr
    \Var\neg\alpha=\Var\alpha,\quad \Var(\alpha\circ\beta)=\Var\alpha\cup\Var\beta\hbox{\ \ for a binary logical connective $\circ$}} $$
    In all definitions $\circ$ is either $\land$ or $\lor$.

\edefn

\def\truthtable#1#2{%
    \vtop{\ialign{\hfil\strut\ $##$\ \hfil&&\vrule\ \hfil\strut$##$\ \hfil\crcr%
        #1\crcr\noalign{\hrule}%
        #2\crcr
    }}%
}
So now that we have discussed the syntax of propositional logic, it is time to discuss its semantics; how we assign to formulas truth values.
Recall the truth tables for $\land$, $\lor$, and $\neg$:
$$ \truthtable{\alpha&\beta&\alpha\land\beta}{1&1&1\cr1&0&0\cr0&1&0\cr0&0&0}\qquad
\truthtable{\alpha&\beta&\alpha\lor\beta}{1&1&1\cr1&0&1\cr0&1&1\cr0&0&0}\qquad
\truthtable{\alpha&\neg\alpha}{1&0\cr0&1} $$
These define how the logical connectives function as functions on $\set{0,1}$.

\bdefn

    A {\emphcolor propositional valuation}\addtoindex{valuation}, or a {\emphcolor propositional model}\addtoindex{model}\alsosee{model}{Valuation}{valuation:}, is a function
    $$ w\colon\PV\longto\set{0,1} $$
    We can extend it to a function $w\colon\PV\longto{\cal F}$ as follows:
    $$ w(\alpha\land\beta)=w\alpha\land w\beta,\quad w(\alpha\lor\beta)=w\alpha\lor w\beta,\quad w\neg\alpha=\neg w\alpha $$

\edefn

Notice that we would need to define, for example, $w(\alpha\to\beta)=w\alpha\to w\beta$ had $\to$ been an element of our logical signature.
But since $\to$ is defined using $\land$ and $\neg$, we must prove this identity:
$$ w(\alpha\to\beta) = w\neg(\alpha\land\neg\beta) = \neg w(\alpha\land\neg\beta) = \neg(w\alpha\land\neg w\beta) = w\alpha\to w\beta $$
This is of course not a coincidence, but a result of the fact that $\alpha\to\beta=\neg(\alpha\land\neg\beta)$ (where $\alpha,\beta\in\set{0,1}$).
Notice that furthermore,
$$ w\top=1,\quad w\bot=0 $$

\bprop

    The valuation of a formula is dependent only on its variables.
    Meaning if $\phi$ is a formula and $w$ and $w'$ are two valuations where $w\pi=w'\pi$ for all $\pi\in\Var\phi$, then $w\phi=w'\phi$.

\eprop

We will prove this by induction on $\phi$.
For prime formulas, this is obvious as $\Var\phi=\set\phi$ and then $w\phi=w'\phi$ by the proposition's assumption.
For $\phi=\alpha\land\beta$, we have that
$$ w\phi = w\alpha\land w\beta = w'\alpha\land w'\beta = w'\phi $$
where the second equality is our inductive assumption.
The proof for $\phi=\alpha\lor\beta$ and $\phi=\neg\alpha$ is similar.
\qed

Let us suppose that $\PV=\set{p_1,p_2,\dots,p_n,\dots}$, then we define ${\cal F}_n$ to be the set of formulas $\phi$ such that $\Var\phi\subseteq\set{p_1,\dots,p_n}$.

\bdefn

    A {\emphcolor boolean function}\addtoindex{boolean function} is a function
    $$ f\colon\set{0,1}^n\longto\set{0,1} $$
    for some $n\geq0$.
    The set of boolean functions of arity $n$ is denoted ${\bf B}_n$.
    A formula $\phi\in{\cal F}_n$ {\emphcolor represents} a boolean function $f\in{\bf B}_n$ (similarly, $f$ is represented by $\phi$), if for all valuations $w$,
    $$ w\phi = f(w\vec p\,) \qquad (w\vec p=(wp_1,\dots,wp_n)) $$

\edefn

So for example, $\alpha=p_1\land p_2$ represents the function $f(p,q)=p\land q$.
This is since
$$ f(wp_1,wp_2) = wp_1\land wp_2 = w(p_1\land p_2) = w\alpha $$

Since valuations of $\phi\in{\cal F}_n$ are defined by their values on $p_1,\dots,p_n$, $\phi$ represents at most a single function $f$.
In fact, it represents the function
$$ \phi^{(n)}(x_1,\dots,x_n) = w\phi $$
where $w$ is any valuation such that $wp_i=x_i$ (all of these valuations valuate $\phi$ the same).
Now, notice that ${\cal F}_n\subset{\cal F}_{n+1}$ and ${\bf B}_n\subset{\bf B}_{n+1}$ and so $\phi\in{\cal F}_n$ represents a function in ${\cal B}_{n+1}$ as well.
But this function is not essentially in ${\cal B}_n$ in the sense that its last argument does not impact its value.
Formally we say that given a function $f\colon M^n\longto M$, we call its $i$th argument {\it fictional} if for all $x_1,\dots,x_i,\dots,x_n\in M$ and $x_i'\in M$:
$$ f(x_1,\dots,x_i,\dots,x_n) = f(x_1,\dots,x'_i,\dots,x_n) $$
An {\it essentially $n$-ary}\addtoindex{essentially $n$-ary function} function is a function with no fictional arguments.

\bdefn

    Two formulas $\alpha$ and $\beta$ are {\emphcolor equivalent}\addtoindex{equivalence}[of propositional formulas] if for every valuation $w$, $w\alpha=w\beta$.
    This is denoted $\alpha\equiv\beta$.

\edefn

It is immediate that $\alpha$ and $\beta$ are equivalent if and only if they represent the same function.
A simple example of equivalence is $\alpha\equiv\neg\neg\alpha$.
The following equivalences are easy to verify and the reader should already be familiar with them ($\alpha$, $\beta$, and $\gamma$ are formulas):

\medskip
{\tabskip=0pt plus 1fil
\halign to \hsize{\hfil$#$\tabskip=0pt&${}\equiv#$\hfil\tabskip=.5cm&\hfil$#$\tabskip=0pt&${}\equiv#$\hfil\tabskip=.5cm&(#)\hfil\tabskip=0pt plus 1fil\cr
    \alpha\land(\beta\land\gamma)&\alpha\land\beta\land\gamma & \alpha\lor(\beta\lor\gamma) & \alpha\lor\beta\gamma & associativity\cr
    \alpha\land\beta&\beta\land\alpha & \alpha\lor\beta&\beta\lor\alpha & commutativity\cr
    \alpha\land\alpha&\alpha & \alpha\lor\alpha&\alpha & idempotency\cr
    \alpha\land(\alpha\lor\beta)&\alpha & \alpha\lor\alpha\land\beta&\alpha & absorption\cr
    \multispan4\hfil$\alpha\land(\beta\lor\gamma)\equiv\alpha\land\beta\lor\alpha\land\gamma$\hfil & $\land$-distributivity\cr
    \multispan4\hfil$\alpha\lor\beta\land\gamma\equiv(\alpha\lor\beta)\land(\alpha\lor\gamma)$\hfil & $\lor$-distributivity\cr
    \neg(\alpha\land\beta)&\neg\alpha\lor\neg\beta & \neg(\alpha\lor\beta)&\neg\alpha\land\neg\beta & de Morgan rules\cr
}}

Furthermore,
$$ \alpha\lor\neg\alpha \equiv \top,\quad \alpha\land\neg\alpha\equiv\bot,\quad \alpha\land\top\equiv\alpha\lor\bot\equiv\alpha $$
Since $\alpha\to\beta=\neg(\alpha\land\neg\beta)$, by de Morgan rules, this is equivalent to 
$$ \equiv\neg\alpha\lor\neg\neg\beta \equiv \neg\alpha\lor\beta $$
Notice that
$$ \alpha\to\beta\to\gamma \equiv \neg\alpha\lor(\beta\to\gamma) \equiv \neg\alpha\lor\neg\beta\lor\gamma \equiv \neg(\alpha\land\beta)\lor\gamma \equiv \alpha\land\beta\to\gamma $$
Inductively, we see that
$$ \alpha_1\to\cdots\to\alpha_n\to\gamma \equiv \alpha_1\land\cdots\land\alpha_n\to\gamma $$
We could go on, but I assume you get the point.

$\equiv$ is obviously reflexive, symmetric, and transitive: therefore it is an equivalence relation on ${\cal F}$.
But moreso it is a {\it congruence relation}\addtoindex{congruence relation}, meaning it respects connectives.
Explicitly, for all formulas $\alpha,\beta,\alpha',\beta'\in{\cal F}$:
$$ \alpha\equiv\alpha',\,\beta\equiv\beta' \implies \alpha\land\beta\equiv\alpha'\land\beta',\,\alpha\lor\beta\equiv\alpha'\lor\beta',\,\neg\alpha\equiv\neg\alpha' $$
Congruence relations will be discussed in more generality in later sections.
Inductively, we can prove the following result:

\bthrm[title=The Replacement Theorem]

    Suppose $\alpha$ and $\alpha'$ are equivalent formulas.
    Let $\phi$ be some other formula, and define $\phi'$ to be the result of replacing all occurences of $\alpha$ within $\phi$ by $\alpha'$.
    Then $\phi\equiv\phi'$.

\ethrm

This will be proven more generally later.

\bdefn

    Prime formulas and their negations are called {\emphcolor literals}\addtoindex{literal}.
    A formula of the form $\alpha_1\lor\cdots\lor\alpha_n$ where each $\alpha_i$ is a conjunction of literals is called a {\emphcolor disjunctive normal form}\addtoindex{disjunctive normal form}.
    And similarly a formula of the form $\alpha_1\land\cdots\land\alpha_n$ where each $\alpha_i$ is a disjunction of literals is called a {\emphcolor conjunctive normal form}\addtoindex{conjunctive normal
    form}.
    We will use the abreviations DNF and CNF for disjunctive and conjunctive normal forms, respectively.

\edefn

So a DNF is a formula of the form
$$ \bigvee_{i=1}^n\bigwedge_{j=1}^{m_i}\ell_{i,j} $$
where for every $i,j$, $\ell_{i,j}$ is a literal: a formula of the form $p_{i,j}$ or $\neg p_{i,j}$ for some prime formula $p_{i,j}$.
Similarly a CNF is a formula of the form
$$ \bigwedge_{i=1}^n\bigvee_{j=1}^{m_i}\ell_{i,j} $$

Let us temporarily introduce the following notation: for a prime formula $p$, let
$$ p^1 \coloneqq p,\qquad p^0 \coloneqq \neg p $$
This allows us to more concisely state and prove the following theorem:

\bthrm

    Every boolean function $f\in{\bf B}_n$ for $n>0$ is representable by the DNF
    $$ \alpha_f \coloneqq \bigvee_{f(\vec x)=1}p_1^{x_1}\land\cdots\land p_n^{x_n} $$
    and a CNF
    $$ \beta_f \coloneqq \bigwedge_{f(\vec x)=0}p_1^{\neg x_1}\land\cdots\land p_n^{\neg x_n} $$

\ethrm

Let $w$ be a valuation and $\vec p=(p_1,\dots,p_n)$ then
$$ w\alpha_f = \bigvee_{f(\vec x)=1}wp_1^{x_1}\land\cdots\land wp_n^{x_n} $$
Notice that $wp^x$ is equal to $1$ if and only if $wp=x$: suppose $x=0$ then $wp^x=\neg wp$, which is equal to $1$ if and only if $wp=0=x$, and similar for $x=1$.
Thus $w\alpha_f=1$ if and only if there exists a $\vec x$ such that $f(\vec x)=1$ and $wp_1^{x_1}\land\cdots\land wp_n^{x_n}=1$, meaning that for each $i$, $wp_i=x_i$.
This means that $\vec x=w\vec p$, and so $f(w\vec p)=f(\vec x)=1$.

Similarly if $f(w\vec p)=1$ then let $\vec x=w\vec p$, and then $wp_1^{x_1}\land\cdots\land wp_n^{x_n}=1$ and $f(\vec x)=1$, so $w\alpha_f=1$.
So $w\alpha_f=f(w\vec p)$ for all valuations $w$, which means that $f$ is represented by $\alpha_f$, as required.
The proof for $\beta_f$ is similar.
\qed

Notice that since every formula represents a boolean function, which by above can be represented by a DNF and a CNF, we get that every formula is equivalent to a DNF and a CNF.

\bcoro[name=dnfandcnf]

    Every formula is equivalent to a DNF and a CNF.

\ecoro

\bdefn

    A logical signature $\sigma$ is {\emphcolor functional complete} if every boolean function is representable by a formula in this signature.

\edefn

By \refmath[corollary]{dnfandcnf}, $\set{\neg,\land,\lor}$ is functional complete.
Since
$$ \alpha\lor\beta \equiv \neg(\neg\alpha\land\neg\beta),\quad \alpha\land\beta \equiv \neg(\neg\alpha\lor\neg\beta) $$
$\set{\neg,\land}$ and $\set{\neg,\lor}$ are both functional complete.
Thus in order to show that a logical signature $\sigma$ is functional complete, it is sufficient to show that $\neg$ and $\land$ or $\neg$ and $\lor$ can be represented by $\sigma$.

\bnote

    If $f$ is a function, instead of writing $f(x)$ or $fx$, many times we will instead write $x^f$.
    This is more concise and may reduce confusion in the case that $x$ itself is a string wrapped in parentheses.

\enote

Let us define the function $\delta\colon{\cal F}\longto{\cal F}$ on formulas recursively by $p^\delta=p$ for prime formulas $p$ and
$$ (\neg\alpha)^\delta = \neg\alpha^\delta,\quad (\alpha\land\beta)^\delta = \alpha^\delta\lor\beta^\delta,\quad (\alpha\lor\beta)^\delta=\alpha^\delta\land\beta^\delta $$
Alternatively, $\alpha^\delta$ is simply the result of swapping all occurences of $\land$ with $\lor$, and all occurences of $\lor$ with $\land$.
$\alpha^\delta$ is called the {\it dual formula}\addtoindex{dual formula} of $\alpha$.
Notice that the dual formula of a DNF is a CNF, and vice versa.

Now, suppose $f\in{\bf B}_n$, then let us define the {\it dual} of $f$,
$$ f^\delta(\vec x) \coloneqq \neg f(\neg\vec x) $$
where $\neg\vec x=(\neg x_1,\dots,\neg x_n)$.
Notice that $\delta$ is idempotent:
$$ f^{\delta^2}(\vec x) = \neg f^\delta(\neg\vec x) = \neg\neg f(\neg\neg\vec x) = f(\vec x) $$

\bthrm[title=The Duality Principle for Two-Valued Logic, name=dualityprinciple]

    If $\alpha$ represents the function $f$, then $\alpha^\delta$ represents $f^\delta$.

\ethrm

We will prove this by induction on $\alpha$.
If $\alpha=p$ is prime, then this is trivial.
Now suppose that $\alpha$ and $\beta$ represent $f_1$ and $f_2$ respectively.
Then $\alpha\land\beta$ represents $f(\vec x)=f_1(\vec x)\land f_2(\vec x)$, and $(\alpha\land\beta)^\delta=\alpha^\delta\lor\beta^\delta$ represents $g(\vec x)=f_1^\delta(\vec x)\lor f_2^\delta(\vec x)$
by the induction hypothesis.
Now,
$$ f^\delta(\vec x) = \neg f(\neg\vec x) = \neg(f_1(\neg\vec x)\land f_2(\neg\vec x)) = \neg f_1(\neg\vec x)\lor\neg f_2(\neg\vec x) = f_1^\delta(\vec x)\lor f_2^\delta(\vec x) = g(\vec x) $$
So $f^\delta=g$, meaning that $(\alpha\land\beta)^\delta$ does indeed represent $f^\delta$.
The proof for $\alpha\lor\beta$ is similar.
Now suppose $\alpha$ represents $f$, then $\neg\alpha$ represents $\neg f$, and $\alpha^\delta$ represents $f^\delta$ by the induction hypothesis.
And so $(\neg\alpha)^\delta=\neg\alpha^\delta$ represents $\neg f^\delta$, which is equal to $(\neg f)^\delta$ since
$$ (\neg f)^\delta(\vec x) = (\neg\neg f)(\neg\vec x) = \neg(\neg f(\neg\vec x)) = \neg f^\delta(\vec x) $$
And so $(\neg\alpha)^\delta$ represents $(\neg f)^\delta$, as required.
\qed

\bdefn

    Suppose $\alpha$ is a formula and $w$ is a valuation.
    Instead of writing $w\alpha=1$, we now write $w\vDash\alpha$, and this is read as ``{\it $w$ satisfies $\alpha$}''.
    If $X$ is a set of formulas, we write $w\vDash X$ if $w\vDash\alpha$ for all $\alpha\in X$, and $w$ is called a {\emphcolor propositional model}\addtoindex{model}[propositional] for $X$.
    A formula $\alpha$ (respectively a set of formulas $X$) is {\emphcolor satisfiable}\addtoindex{satisfiable} if there is some valuation $w$ such that $w\vDash\alpha$ (respectively $w\vDash X$).
    $\vDash$ is called the {\emphcolor satisfiability relation}\addtoindex{satisfiability relation}.

\edefn

$\vDash$ has the following immediate properties:

\medskip
\tabskip=0pt plus 1fil
\halign to \hsize{\hfil$#$\tabskip=0pt&${}\iff#$\hfil\tabskip=.5cm&\hfil$#$\tabskip=0pt&${}\iff#$\hfil\tabskip=0pt plus 1fil\cr
    w\vDash p & wp=1\ (p\in\PV) & w\vDash\alpha & w\nvDash\alpha\cr
    w\vDash\alpha\land\beta & w\vDash\alpha\hbox{ and }w\vDash\beta & w\vDash\alpha\lor\beta & w\vDash\alpha\hbox{ or }w\vDash\beta\cr
}
\medskip
These properties uniquely define $\vDash$, meaning we could have defined $\vDash$ recursively by these properties.

Notice that
$$ w\vDash\alpha\to\beta \iff \hbox{if }w\vDash\alpha\hbox{ then }w\vDash\beta $$
This is due to the definition of $\to$ coinciding with our common usage of implication.
Had we not defined $\to$, but instead added it to our logical signature, this above equivalence would have to be taken in the definition of the satisfiability relation (when axiomized by the above
properties).

\bdefn

    $\alpha$ is {\emphcolor logically valid}, or a {\emphcolor tautology}\addtoindex{tautology}, if $w\vDash\alpha$ for all valuations $w$.
    This is abbreviated by $\vDash\alpha$.
    A formula which cannot be satisfied; ie. for all valuations $w$, $w\nvDash\alpha$; is called a {\emphcolor contradiction}\addtoindex{contradiction}\alsosee{contradiction}{Tautology}{tautology:}.

\edefn

For example, $\alpha\lor\neg\alpha$ is a tautology, while $\alpha\land\neg\alpha$ and $\alpha\oto\neg\alpha$ are contradictions for all formulas $\alpha$.
Notice that the negation of a tautology is a contradiction and vice versa.
$\top$ is a tautology and $\bot$ is a contradiction.

The following are important tautologies of implication (keep in mind how $\to$ associates from the right):

\medskip
\tabskip=0pt plus 1fil
\halign to \hsize{$#$\hfil\tabskip=.5cm&(#)\hfil\tabskip=0pt plus 1fil\cr
    p\to p & self-implication\cr
    (p\to q)\to(q\to r)\to(p\to r) & chain rule\cr
    (p\to q\to r)\to(q\to p\to r) & exchange of premises\cr
    p\to q\to p & premise change\cr
    (p\to q\to r)\to(p\to q)\to(p\to r) & Frege's formula\cr
    ((p\to q)\to p)\to p & Peirce's formula\cr
}
\medskip

\bdefn

    Suppose $X$ is a set of formulas and $\alpha$ a formula, we say that $\alpha$ is a {\emphcolor logical consequence}\addtoindex{consequence relation}%
    \alsosee{consequence relation}{Satisfiability relation}{satisfiability relation:} if $w\vDash\alpha$ for every model $w$ of $X$.
    In other words,
    $$ w\vDash X\implies w\vDash\alpha $$
    This is denoted $X\vDash\alpha$.

\edefn

Notice that $\vDash$ here is used for logical consequence (the consequence relation), and we used it before as the symbol for the satisfiability relation.
Context will make it clear as to its meaning.
We use the notation $\alpha_1,\dots,\alpha_n\vDash\beta$ to mean $\set{\alpha_1,\dots,\alpha_n}\vDash\beta$.
This justifies the notation for tautologies: $\alpha$ is a tautology if and only if $\varnothing\vDash\alpha$ (since every valuation models $\varnothing$), which is shortened by the above notation to
$\vDash\alpha$.

And we also use $X\vDash\alpha,\beta$ to mean $X\vDash\alpha$ and $X\vDash\beta$.
And $X,\alpha\vDash\beta$ to mean $X\cup\set\alpha\vDash\beta$.

The following are examples of logical consequences
$$ \displaylines{
    \alpha,\beta\vDash\alpha\land\beta,\quad\alpha\land\beta\vDash\alpha,\beta\cr
    \alpha,\alpha\to\beta\vDash\beta\cr
    X\vDash\bot\implies X\vDash\alpha\quad\hbox{for all formulas $\alpha$}\cr
    X,\alpha\vDash\beta,\,X,\neg\alpha\vDash\beta\implies X\vDash\beta\cr
} $$
The final example is true because if $w\vDash X$ then either $w\vDash\alpha$ or $w\vDash\neg\alpha$, and in either case $w\vDash\beta$.

Let us now state some obvious properties of the consequence relation:

\medskip
\tabskip=0pt plus 1fil
\halign to \hsize{$#$\hfil\tabskip=.5cm&({\it#})\hfil\tabskip=0pt plus 1fil\cr
    \alpha\in X \implies X\vDash\alpha & reflexivity\cr
    X\vDash\alpha\hbox{ and }X\subseteq X'\implies X'\vDash\alpha & monotonicity\cr
    X\vDash Y\hbox{ and }Y\vDash\alpha\implies X\vDash\alpha & transitivity\cr
}

\bdefn

    A {\emphcolor propositional substitution}\addtoindex{substitution}[propositional] is a mapping from prime formulas to formulas, $\sigma\colon\PV\longto{\cal F}$, which is extended to a mapping between
    formulas $\sigma\colon{\cal F}\longto{\cal F}$ recursively:
    $$ (\alpha\land\beta)^\sigma = \alpha^\sigma\land\beta^\sigma,\quad (\alpha\lor\beta)^\sigma = \alpha^\sigma\lor\beta^\sigma,\quad (\neg\alpha)^\sigma = \neg\alpha^\sigma $$

\edefn

