In this subsection we will define another form of sequent calculus.

\bdefn

    A define $\Lambda\subseteq{\cal F}$ to be a set of axioms, called the {\emphcolor logical axiom scheme}.
    Now, let $\Gamma$ be a set of {\emphcolor rules of inference}, predicates of the form $R\in\Lambda^n\times\Lambda$ for $n>0$, where $R((\phi_1,\dots,\phi_n),\phi)$ which is to be understood as
    ``{\it if $\phi_1,\dots,\phi_n$ then $\phi$}''.

    If $X\subseteq{\cal F}$ is a set of formulas, then a {\emphcolor proof}\addtoindex{proof} is a sequence $\Phi=(\phi_0,\dots,\phi_n)$ where for every $i$, $\phi_i$ is either in $X\cup\Lambda$ or there
    exists a rule of inference $R\in\Gamma$ and indexes $i_1,\dots,i_n<i$ such that $R((\phi_{i_1},\dots,\phi_{i_n}),\phi)$.
    In such a case, $\phi_n$ is termed {\emphcolor derivable} (or {\emphcolor provable}) from $X$, and is written $X\vsim\phi_n$ ($\vsim$ to differentiate it from the derivability relation $\vdash$ from
    the previous subsection).

\edefn

Hilbert-style calculi will use the following axiom scheme $\Lambda$:

\medskip
\tabskip=0pt plus 1fil
\halign to \hsize{$#$\hfil\tabskip=2cm&$#$\hfil\tabskip=0pt plus 1fil\cr
    \Lambda1\quad (\alpha\to\beta\to\gamma)\to(\alpha\to\beta)\to\alpha\to\gamma & \Lambda2\quad \alpha\to\beta\to\alpha\land\beta\cr
    \Lambda3\quad \alpha\land\beta\to\alpha,\quad\alpha\land\beta\to\beta & \Lambda4\quad (\alpha\to\neg\beta)\to\beta\to\neg\alpha\cr
}

And there is only a single rule of inference: $R((\alpha,\alpha\to\beta),\beta)$ called {\it modus ponens}, abbreviated MP.
Essentially if $\alpha$ and $\alpha\to\beta$ then $\beta$.

The finiteness theorem for $\vsim$ is immediate, since $X\vsim\alpha$ requires a {\it finite} proof from $X$.
And notice that
$$ X\vsim\alpha,\alpha\to\beta \implies X\vsim\beta $$
Since if $\Phi_1=(\phi_0,\dots,\phi_n)$ is a proof of $\alpha$, and $\Phi_2=(\phi'_0,\dots,\phi'_m)$ is a proof of $\alpha\to\beta$, then
$$ \Phi=(\phi_0,\dots,\phi_n,\phi'_0,\dots,\phi'_m,\beta) $$
is a proof of $\alpha\to\beta$.

\bprop[title=Principle of Induction for $\vsim$, name=vsiminduct]

    Let $X$ be a set of formulas and ${\cal E}$ a property of formulas.
    Then if
    \benum
        \item ${\cal E}\alpha$ is true for all $\alpha\in X\cup\Lambda$, and
        \item ${\cal E}\alpha$ and ${\cal E}\alpha\to\beta$ implies ${\cal E}\beta$ for all formulas $\alpha$, $\beta$.
    \eenum
    Then $X\vsim\alpha$ implies ${\cal E}\alpha$.

\eprop

We will prove this by induction on $n$, the length of the proof of $\alpha$.
If $n=1$ then $\alpha$ is in $X\cup\Lambda$ and so by assumption ${\cal E}\alpha$.
Now suppose $\Phi=(\phi_0,\dots,\phi_n)$ is a proof of $\alpha=\phi_n$.
If $\alpha\in X\cup\Lambda$ then by assumption ${\cal E}\alpha$.
Otherwise $\Phi$ must contain formulas of the form $\alpha_i$ and $\alpha_i\to\alpha$.
Since initial segments of proofs are themselves proofs, by our inductive hypothesis ${\cal E}\phi_i$ for $i<n$.
And thus ${\cal E}\alpha_i$ and ${\cal E}\alpha_i\to\alpha$ and so ${\cal E}\alpha$ as required.
\qed

This can obviously be generalized to a principle of induction for general rules of inferences, where the second condition is replaced with a general notion of closure under rules of inference.

Now we can show that ${\vsim}\subseteq{\vDash}$, meaning if $X\vsim\alpha$ then $X\vDash\alpha$ by defining the property ${\cal E}\alpha\coloneqq X\vDash\alpha$.
Since $\Lambda$ contains only tautologies, for every $\alpha\in X\cup\Lambda$, $X\vDash\alpha$ meaning ${\cal E}\alpha$ for all $\alpha\in X\cup\Lambda$.
And if $X\vDash\alpha$ and $X\vDash\alpha\to\beta$ then we know $X\vDash\beta$.
So ${\cal E}$ satisfies the inductive properties stated above, meaning $X\vsim\alpha$ implies $X\vDash\alpha$ as required.

Now, obviously $\vsim$ is reflexive, monotonic, and transitive.
Reflexivity follows directly from its definition.
Monotonicity follows because a proof in $X$ is also a proof in $X\subseteq X'$.
And transitivity follows because if $X\vsim Y$ and $Y\vsim\alpha$, then by concatenating the proofs of $\phi\in Y$ in $X$ together with the proof of $\alpha$ in $Y$ gives a proof of $\alpha$ in $X$.

Our goal for the remainder of this subsection is showing that ${\vsim}={\vDash}$, we will do this by showing that ${\vdash}\subseteq{\vsim}$.
As explained above, $\vsim$ is reflexive and monotonic, meaning it satisfies (IS) and (MR) of the Gentzen-style calculus $\vdash$.

\blemm

    \benum
        \item If $X\vsim\alpha\to\neg\beta$ then $X\vsim\beta\to\neg\alpha$
        \item $\vsim\alpha\to\beta\to\alpha$
        \item $\vsim\alpha\to\alpha$
        \item $\vsim\alpha\to\neg\neg\alpha$
        \item $\vsim\beta\to\neg\beta\to\alpha$
    \eenum

\elemm

\benum
    \item By $\Lambda4$, we have $X\vsim(\alpha\to\neg\beta)\to\beta\to\neg\alpha$, and since $X\vsim\alpha\to\neg\beta$ by modus ponens we get $X\vsim\beta\to\neg\alpha$.
    \item By $\Lambda3$, $\vsim\beta\land\neg\alpha\to\neg\alpha$, and so by (1) we have $\vsim\alpha\to\neg(\beta\land\neg\alpha)=\alpha\to\beta\to\alpha$.
    \item Let $\gamma=\alpha$ and $\beta=\alpha\to\alpha$, then $\Lambda1$ gives
    $$ \vsim(\alpha\to(\alpha\to\alpha)\to\alpha)\to(\alpha\to\alpha\to\alpha)\to\alpha\to\alpha $$
    We know by (2), $\vsim\alpha\to(\alpha\to\alpha)\to\alpha$ and $\vsim\alpha\to\alpha\to\alpha$, so by applying modus ponens twice we get $\vsim\alpha\to\alpha$.
    \item Since $\vsim\neg\alpha\to\neg\alpha$ by (3), and applying (1) gives $\vsim\alpha\to\neg\neg\alpha$.
    \item By $\Lambda3$, $\vsim\neg\beta\land\neg\alpha\to\neg\beta$, applying (1) gives $\vsim\beta\to\neg(\neg\beta\to\neg\alpha)=\beta\to\neg\beta\to\alpha$.
    \qed
\eenum

Since $\Lambda3$ gives $\alpha\land\beta\to\alpha,\beta$, $\vsim$ satisfies $(\land2)$ of $\vdash$.
$\Lambda2$ gives $\alpha\to\beta\to(\alpha\land\beta)$ and so by applying MP twice, we get $\alpha,\beta\vsim\alpha\land\beta$ and so $\vsim$ satisfies $(\land1)$ of $\vdash$.
Now by (5) of the above lemma, since $\vsim\alpha\to\neg\alpha\to\beta$, by applying MP twice we get that $X,\alpha,\neg\alpha\vsim\beta$ for all formulas $\beta$.
By transitivity, this means that $X\vsim\alpha,\neg\alpha$ implies $X\vsim\beta$.
Thus $\vsim$ satisfies (IS), (MR), $(\land1)$, $(\land2)$, and $(\neg1)$ of $\vdash$.
We will now do a bit more work to show that it also satisfies $(\neg2)$.

\blemm[title=The Deduction Theorem, name=deductthrm]

    $X,\alpha\vsim\gamma$ implies $X\vsim\alpha\to\gamma$.

\elemm

We will prove this using the principle of induction for $\vsim$.
Let ${\cal E}\gamma$ mean $X\vsim\alpha\to\gamma$, we will show that $X,\alpha\vsim\gamma$ implies ${\cal E}\gamma$ by showing ${\cal E}$ is closed under the inductive properties stated in the
\refmath{vsiminduct}.
If $\gamma\in\Lambda\cup X\cup\set\alpha$, if $\gamma=\alpha$ then we showed above that $X\vsim\alpha\to\alpha$.
Otherwise if $\gamma\in X\cup\Lambda$ then $X\vsim\gamma$ and $X\vsim\gamma\to\alpha\to\gamma$, so by MP $X\vsim\alpha\to\gamma$ meaning ${\cal E}\gamma$ as required.

Now, if ${\cal E}\beta$ and ${\cal E}\beta\to\gamma$, meaning $X\vsim\alpha\to\beta$ and $X\vsim\alpha\to\beta\to\gamma$.
Then by $\Lambda1$, applying MP twice gives $X\vsim\alpha\to\gamma$ as required.
\qed

\blemm

    $\vsim\neg\neg\alpha\to\alpha$

\elemm

By $\Lambda3$ and MP, we have $\neg\neg\alpha\land\neg\alpha\vsim\neg\alpha,\neg\neg\alpha$.
Let $\tau$ be any formula where $\vsim\tau$, then since we have already verified rule $(\neg1)$, $\neg\neg\alpha\land\neg\alpha\vsim\neg\tau$.
And so by the deduction theorem, $\vsim\neg\neg\alpha\land\neg\alpha\to\neg\tau$.
We showed above that this means $\vsim\tau\to\neg(\neg\neg\alpha\land\neg\alpha)$, and since $\vsim\tau$ by MP we get $\vsim\neg(\neg\neg\alpha\land\neg\alpha)=\neg\neg\alpha\to\alpha$ as required.
\qed

\blemm

    $\vsim$ also satisfies rule $(\neg2)$ of the Gentzen-style calculus $\vdash$.

\elemm

This is the rule that $X,\alpha\vsim\beta$ and $X,\neg\alpha\vsim\beta$ implies $X\vsim\beta$.
If $X,\alpha\vsim\beta$ and $X,\neg\alpha\vsim\beta$ then $X,\alpha\vsim\neg\neg\beta$ and $X,\neg\alpha\vsim\neg\neg\beta$.
By the deduction theorem, this means $X\vsim\alpha\to\neg\neg\beta,\neg\alpha\to\neg\neg\beta$.
And thus $X\vsim\neg\beta\to\neg\alpha,\neg\beta\to\neg\neg\alpha$.
Thus MP yields $X,\neg\beta\vsim\neg\alpha,\neg\neg\alpha$.
So let $\vsim\tau$ and so by $(\neg1)$, we get $X,\neg\beta\vsim\neg\tau$, and so again by the deduction theorem, $X\vsim\neg\beta\to\neg\tau$, meaning $X\vsim\tau\to\neg\neg\beta$.
Since $\vsim\tau$ by MP we get $X\vsim\neg\neg\beta$ and since $X\vsim\neg\neg\beta\to\beta$, by MP we get $X\vsim\beta$ as required.
\qed

\bthrm[title=The Completeness Theorem, name=completenessthrm]

    $X\vsim\alpha$ if and only if $X\vDash\alpha$.
    More suggestively,
    $$ {\vsim} = {\vDash} $$

\ethrm

We have already shown ${\vsim}\subseteq{\vDash}$.
Since $\vsim$ satisfies all the basic rules of $\vdash$, ${\vdash}\subseteq{\vsim}$ (by $\vdash$'s principle of induction).
Now since ${\vdash}={\vDash}$, we get that ${\vDash}\subseteq{\vsim}\subseteq{\vDash}$, and so ${\vsim}={\vDash}$.
\qed

It is important to note that $\Lambda$ is sufficient to obtain all tautologies only because $\to$ was defined via $\neg$ and $\land$.
Had it been taken as just another connective, we would've needed to add axioms to $\Lambda$ stating the relation between $\to$ and $\neg$ and $\land$.

