The goal of this section is to utilize G\"odel numbers to encode formulas and related concepts in the natural numbers, and then prove related properties (recursiveness, primitive recursiveness, etc) about
them.
In this section, we will set ${\cal L}={\cal L}_{\it ar}$ with the symbols $0,{\tt S},+,\cdot$.
For symbols in the alphabet of ${\cal L}$ we define their {\it symbol code}
$$ \vcenter{\vbox{\ialign{\strut\kern.25cm\hfil$#$\hfil\kern.25cm\vrule&&\kern.5cm\hfil$#$\hfil\cr
    {\sf s}&\eq&\neg&\land&\forall&(&)&0&{\tt S}&+&\cdot&\v_0&\v_1\cr\noalign{\hrule}
    \#{\sf s}&1&3&5&7&9&11&13&15&17&19&21&23\cr
}}}\cdots $$
Let ${\cal S}_{\cal L}$ denote the set of strings over the language ${\cal L}$.
For a string $\xi=\s_0\cdots\s_n\in{\cal S}_{\cal L}$ we encode it by its {\it G\"odel number}, $\dot\xi\coloneqq\gen{\#\s_0,\dots,\#\s_1}=p_0^{1+\#\s_0}\cdots p_n^{1+\#\s_n}$.
The empty string gets G\"odel number $1$.

Obviously $(\xi\eta)^\cdot=\dot\xi*\dot\eta$ for strings $\xi,\eta$.
And so $\dot{\cal S}_{\cal L}=\set{\dot\xi}[\xi\in{\cal S}_{\cal L}]$ is a pr subset of ${\it GN}$ since $n\in\dot{\cal S}_{\cal L}\iff n\in{\it GN}\land(\forall k<\ell n)2\ndivides\bparens{n}_k$ since
symbol codes are odd.

Notice that while $\#\s$ is the symbol code of the {\it symbol} $\s$, $\dot\s=2^{1+\#\s}$ is the G\"odel number of the atomic {\it string} $\s$.
So for example, the symbol $0$ has a symbol code of $13$, but the prime term $0$ has a G\"odel number of $\dot0=2^{1+\#0}=2^{14}$.
Thus we must distinguish between the set of prime terms $\v_0,\v_1,\dots$ cannot simply be identified with $\Var$.

Let $W\subseteq{\cal S}_{\cal L}$, then we denote $\dot W\coloneqq\set{\dot\xi}[\xi\in W]$.
If $P\subseteq{\cal S}_{\cal L}^n$ is a many-placed word-predicate then define $\dot P=\set{(\dot\xi_1,\dots,\dot\xi_n)}[(\xi_1,\dots,\xi_n)\in P]$.
We say that $P$ is pr (resp. recursive) if $\dot P$ is pr (resp. recursive).
So for example, if we say that $X\subseteq{\cal L}$ is a recursive axiom system, we mean that $\dot X$ is recursive.

We need not restrict ourselves to ${\cal L}={\cal L}_{\it ar}$ for these definitions, we can define this for any language with countably many basic symbols.

Our goal now is the arithmetize the formal proof method, $\vdash$ which we use to denote the Hilbert (not Gentzen) calculus of first-order logic.
Recall that it includes the axiom system $\Lambda$ consisting of axioms $\Lambda1-\Lambda10$ and the rule of inference ${\it MP}$.
Similar as for strings, for a non-empty finite sequence $\Phi=(\phi_0,\dots,\phi_n)$ of ${\cal L}$-formulas, define its {\it G\"odel number} $\dot\Phi=\gen{\dot\phi_0,\dots,\dot\phi_n}$.
This includes the particular case where $\Phi$ is a proof.
Notice that $\dot\Phi\neq\dot\xi$ for every $\xi\in{\cal S}_{\cal L}$ since $\bparens{\dot\Phi}_0=\dot\phi_0$ which is divisible by $2$, but $\bparens{\dot\xi}_0=\#\s_0$ is not.

Let $X\subseteq{\cal L}$ form an axiom system for an ${\cal L}$-theory $T$.
A proof $\Phi$ from $X$ is also called a {\it proof in $T$}.
Here and onward, $X$ is tacitly assumed to be an essential part of $T$, interchangeable with it.
Let us now define some pr functions: $\tilde\neg,\tilde\land,\tilde{\to}$ by
$$ \tilde\neg a\coloneqq\dot\neg*a,\qquad a\tilde\land b\coloneqq\dot(*a*\dot\land*b*\dot),\qquad a\tilde{\to}b\coloneqq\tilde\neg(a\tilde\land\tilde\neg b) $$
(for $\tilde{\to}$ the parentheses are argument parentheses, not numbers).

We define the predicate $\proof_T$ to mean ``$\Phi$ is a proof in $T$'', $\bew_T$ to mean ``$\Phi$ is a proof of $\phi$ in $T$'', and $\bwb_T$ to mean ``there exists a proof of $\phi$ in $T$''.
Definitions of these predicates are as follows:
\benum
    \item $\proof_T(b)\iff b\in\GN\land b\neq1\land(\forall k<\ell b)\bigl(\bparens b_k\in\dot X\cup\dot\Lambda\lor(\exists i,j<k)\parens b_i=\bparens b_j\tto\bparens b_k\bigr)$
    Which is just a direct translation of the definition of a proof from $X$.
    \item $\bew_T(b,a)\iff\proof_T(b)\land a=\bparens b_{\it last}$.
    \item $\bwb_Ta\iff\exists b\,\bew_T(b,a)$.
\eenum
From these definitions we can obtain immediately
\benum
    \item $\vdash_T\alpha$ if and only if $\bwb_T\dot\alpha$,
    \item $\bew_T(c,a)\land\bew_T(d,a\tto b)$ implies $\bew_T(c*d*\gen b,b)$,
    \item $\bwb_Ta\land\bwb_T(a\tto b)$ implies $\bwb_Tb$,
    \item $\bwb_T\dot\alpha\land\bwb_T(\alpha\to\beta)^\cdot$ implies $\bwb_T\dot\beta$.
\eenum
$(1)$ is clear since $\vdash_T\alpha$ if and only if there exists a proof of $\alpha$ in $T$, $\Phi$.
Then $\exists n\,\bew_T(n,\dot\alpha)$ for $n=\dot\Phi$.
$(2)$ just tells us we can concatenate proofs for $\alpha$ and $\alpha\to\beta$ to obtain a proof for $\beta$.
$(3)$ results from $(2)$ by particularization, and $(3)$ implies $(4)$ since $(\alpha\to\beta)^\cdot=\dot\alpha\tto\dot\beta$.

All of the notions we defined up to $\bew_T$ are pr ($\bwb_T$ is not).
We will now spend time proving this.
Let us define
$$ \displaylines{
    a\teq b\coloneqq a*\dot{\eq}*b,\quad \tforall(i,a)\coloneqq\dot\forall*i*a,\quad \texists(i,a)\coloneqq\dot\exists*i*a,\quad \tS a\coloneqq\dot{\tt S}*a,\cr
    a\tplus b\coloneqq\dot(*a*\dot+*b*\dot),\quad a\tcdot b\coloneqq\dot(*a*\dot\cdot*b*\dot)
} $$
All these functions are obviously pr.

For strings $\xi,\eta$ let $\xi\leq\eta$ mean $\dot\xi\leq\dot\eta$ (similar for $<$).
This holds in particular if $\xi$ is a substring of $\eta$, since $a,b\leq a*b$ for G\"odel numbers.

\blemm

    The set of terms ${\cal T}$ is primitive recursive.

\elemm

The set ${\cal V}$ of the prime terms $\v_i$ is pr since $n\in\dot{\cal V}\iff(\exists k\leq n)n=2^{21+2k}$.
And so the set of all prime terms ${\cal T}_{\it prim}={\cal V}\cup\set0$ is pr as well.
The recursive definition of ${\cal T}$ gives that $t\in{\cal T}$ if and only if
$$ t\in{\cal T}_{\it prim}\lor(\exists t_1,t_2<t)\bigl(t_1,t_2\in{\cal T}\land(t={\tt S}t_1\lor t=(t_1+t_2)\lor t=(t_1\cdot t_2))\bigr) $$
And so
$$ n\in\dot{\cal T}\iff n\in{\cal T}_{\it prim}\lor(\exists i,k<n)\bigl(i,k\in\dot{\cal T}\land Q(n,i,k)\bigr) $$
where $Q(n,i,k)$ is the pr predicate $(n=\tilde Si\lor n=i\tplus k\lor n=i\tcdot k)$.
We now convert this definition of $\dot{\cal T}$ to a recursive definition of $\chi_{\dot{\cal T}}$ using $\boldsymbol{Oq}$.
Define
$$ P(a,n) \iff n\in{\cal T}_{\it prim}\lor(\exists i,k<n)\bigl(\bparens a_i=\bparens a_k=1\land Q(n,i,k)\bigr) $$
We claim then that $f\coloneqq\chi_{\dot{\cal T}}$ satisfies $\boldsymbol{Oq}$: $fn=\chi_P(\bar fn,n)$, where $\bar fn=\gen{f0,\dots,f(n-1)}$, and so $f$ is then pr.
Indeed, $fi=fk=1$ if and only if $i,k\in\dot{\cal T}$ by definition, so
$$ \eqalign{
    n\in\dot{\cal T}&\iff n\in{\cal T}_{\it prim}\lor(\exists i,k<n)\bigl(fi=fk=1\land Q(n,i,k)\bigr)\cr
    &\iff P(\bar fn,n)\qquad \hbox{(since $\bparens{\bar fn}_i=fi$ and $\bparens{\bar fn}_k=fk$)}
} $$
Which means that $fn=1\iff\chi_P(\bar fn,n)=1$ and so $fn=\chi_P(\bar fn,n)$ since they are both indicator functions.
\qed

In a similar manner we prove that ${\cal L}$ is pr.
Here
$$ n\in\dot{\cal L}_{\it prim} \iff (\exists i,k<n)\bigl(i,k\in\dot{\cal T}\land n=i\teq k\bigr) $$
which is obviously pr.
Then $\phi\in{\cal L}$ is defined by
$$ \phi\in{\cal L}_{\it prim}\lor(\exists\alpha,\beta,x<\phi)\bigl(\alpha,\beta\in{\cal L}\land x\in{\cal V}\land\bigl(\phi=\neg\alpha\lor\phi=(\alpha\land\beta)\lor\phi=\forall x\alpha\bigr)\bigr) $$
Define $Q(n,i,k,x)$ to mean $(n=\tneg i\lor n=i\tand k\lor n=\tforall(x,i))$ which is pr.
Then define
$$ P(a,n) \iff n\in\dot{\cal L}_{\it prim}\lor(\exists i,k,x<n)\bigl(\bparens a_i=\bparens a_k=1\land x\in\dot{\cal V}\land Q(n,i,k,x)\bigr) $$
Then $f\coloneqq\chi_{\dot{\cal L}}$ satisfies $fn=\chi_P(\bar fn,n)$, as before.
So we have shown

\blemm

    The set ${\cal L}={\cal L}_{\it ar}$ of all formulas is primitive recursive.

\elemm

