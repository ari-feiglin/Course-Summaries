Vaught's test, while simple and useful in specific circumstances, is quite limited in application as its conditions are quite restrictive: many complete theories are not categorical for any infinite
cardinality.
One example of this is the theory ${\sf SO}$ of {\it discretely ordered sets}, where every element which is not the right edge element has an immediate successor and every element which is not the
left edge element has an immediate predecessor.
Then we define ${\sf SO}_{ij}$ for $i,j\in\set{0,1}$ analogously to ${\sf DO}_{ij}$.
Clearly in ${\sf SO}_{10}$ and ${\sf SO}_{00}$ we can define ${\tt S}$ the successor function, and so $({\bb N},<)$ is a prime model of ${\sf SO}_{10}$.
Models of ${\sf SO}_{10}$ can be viewed as looking something like ${\bb N}\ {\bb Z}\ {\bb Z}\ \cdots$, ie. the initial segment of such a model is $({\bb N},<)$ then followed by an arbitrary amount of
segments $({\bb Z},<)$.
This demonstrates that ${\sf SO}_{10}$ cannot be categorical for any cardinality, but it is indeed complete as we will demonstrate.

We will prove this using an {\it Ehrenfeucht-Fra\"\i ss\'e game}\addtoindex{ehrenfeucht-fraisse game} (for short an EF-game).
Let ${\cal L}$ be a first-order relational language, ${\cal A}$ and ${\cal B}$ be two ${\cal L}$-structures and $k\geq0$.
Then the EF-game played in $k$ rounds on ${\cal A}$ and ${\cal B}$, denoted $\Gamma_k({\cal A},{\cal B})$ is a game played by two players: player I and player II.
The game proceeds as follows: player I starts by picking any element in either ${\cal A}$ or ${\cal B}$, suppose he choose $a\in A$, then player II must respond with an element $b\in B$.
Conversely if player I chooses an element $b\in B$, player II must respond with an element $a\in A$.
After $k$ such rounds suppose the elements chosen are $a_1,\dots,a_k\in A$ and $b_1,\dots,b_k\in B$ (where $a_i$ and $b_i$ are the elements chosen in the $i$th round), then player II wins if
$a_i\varmapsto b_i$ is a partial isomorphism, meaning that the substructures $\set{a_1,\dots,a_k}\subseteq{\cal A}$ and $\set{b_1,\dots,b_k}\subseteq{\cal B}$ are isomorphic.
Player I wins otherwise.

We say that player II has a winning strategy in the game $\Gamma_k({\cal A},{\cal B})$ if no matter what moves player I plays, player II can always find a way to win (this will be formalized more later on
this section).
We write this as ${\cal A}\sim_k{\cal B}$.
For the zero-round game, we just define ${\cal A}\sim_0{\cal B}$.

\bexam
    \newfunc{dist}{d}({})

    We will show that for every ${\cal B}\vDash{\sf SO}_{00}$, ${\cal A}=({\bb Z},<)\sim_k{\cal B}$ for every $k\geq0$.
    Let us first define the distance function $d(x,y)$ on ${\cal B}$: if $x=y$ then $d(x,y)=0$ and otherwise it is one plus the number of elements between $x$ and $y$ if finite, otherwise $d(x,y)=\infty$.
    Since we can embed ${\cal A}$ into ${\cal B}$, we will just assume that ${\cal A}\subseteq{\cal B}$.

    Our goal is to try and ensure that after $m$ rounds of the game, if $a_1<a_2<\cdots<a_m$ are the elements of ${\cal A}$ which have been played, and $b_1<b_2<\cdots<b_m$ are the elements in ${\cal B}$
    which have been played, then the mapping $a_i\varmapsto b_i$ is the partial embedding corresponding to the play of the game.
    Furthermore, if $\distof{a_i,a_{i+1}}>3^{n-m}$ then $\distof{b_i,b_{i+1}}>3^{n-m}$, and if $\distof{a_i,a_{i+1}}\leq3^{n-m}$ then $\distof{b_i,b_{i+1}}=\distof{a_i,a_{i+1}}$, for $i=1,\dots,m-1$.

    Obviously since $a_i<a_{i+1}$ and $b_i<b_{i+1}$, the function will preserve the relations of the theory and thus be a partial embedding.

    We claim that player II can always make a move to preserve this condition.
    In round $1$, player II can choose any arbitrary element and the condition will hold.
    Now suppose we have played $m$ rounds and $a_1<\cdots<a_m$ and $b_1<\cdots<b_m$ be defined as above.
    Now suppose player I plays $b\in B$, then there are several cases
    \benum
        \item If $b<b_1$ then if $\distof{b,b_1}=k<\infty$ then player II plays $a_1-k$.
            If $\distof{b,b_1}=\infty$ then player II plays $a_1-3^n$, but in any case the condition holds.
        \item If $b_i<b<b_{i+1}$ and $\distof{b_i,b_{i+1}}\leq3^{n-m}$ then $\distof{a_i,a_{i+1}}=\distof{b_i,b_{i+1}}$.
            Play $a=a_i+\distof{b,b_i}$, then $\distof{a,a_{i+1}}=\distof{b,b_{i+1}}$ as required.
        \item If $b_i<b<b_{i+1}$ and $\distof{b_i,b_{i+1}}>3^{n-m}$ and $\distof{b,b_i}<3^{n-m-1}$ then $\distof{a_i,a_{i+1}}>3^{n-m}$.
            Play $a=a_i+\distof{b,b_i}$, then $\distof{a,a_{i+1}}$ and $\distof{a_i,a}$ are greater than $3^{n-m-1}$ as required.
        \item If $b_i<b<b_{i+1}$ and $\distof{b_i,b_{i+1}}>3^{n-m}$ and $\distof{b,b_{i+1}}<3^{n-m-1}$, play $a=a_{i+1}-\distof{b,b_{i+1}}$.
        \item If $b_i<b<b_{i+1}$ and $\distof{b_i,b_{i+1}}>3^{n-m}$, $\distof{b,b_i}>3^{n-m-1}$, and $\distof{b,b_{i+1}}<3^{n-m-1}$.
            Then $\distof{a_i,a_{i+1}}>3^{n-m}$ and so choose an $a$ such that $a_i<a<a_{i+1}$ and the distance of $a$ between them both is greater than $3^{n-m-1}$.
            Playing $a$ satisfies the condition.
        \item If $b>b_m$, this is similar to the first condition.
    \eenum

\eexam

Let us write ${\cal A}\equiv_k{\cal B}$ to mean that ${\cal A}\vDash\alpha\iff{\cal B}\vDash\alpha$ for all $\alpha\in{\cal L}^0$ of quantifier rank $\leq k$.
In a relational language there are no sentences of quantifier rank zero so ${\cal A}\equiv_0{\cal B}$ vacuously.
It is our goal this section to prove the remarkable

\bthrm[name=efgametheorem]

    ${\cal A}\sim_k{\cal B}$ implies ${\cal A}\equiv_k{\cal B}$, thus ${\cal A}\equiv{\cal B}$ provided ${\cal A}\sim_k{\cal B}$ for all $k$.

\ethrm

In finite signatures the converse is also true, but we will not prove this.
Notice that this proves that ${\sf SO}_{00}$ is complete, since every ${\cal B}\vDash{\sf SO}_{00}$ has that ${\cal B}\sim_k({\bb Z},<)$ for all $k$ and thus ${\cal B}\equiv({\bb Z},<)$, meaning
all models of ${\sf SO}_{00}$ have the same theory and therefore ${\sf SO}_{00}$ is complete.
We can similarly show that ${\sf SO}_{10}$ is complete (as all its models are equivalent to $({\bb N},<)$) and so is ${\sf SO}_{01}$ (using $({\bb N},>)$).
${\sf SO}_{11}$ is not complete as it has the finite model property.

Let us generalize our concept of EF-games as follows: we define the game $\Gamma_k({\cal A},{\cal B},\vec a,\vec b)$ to be the EF-game with prior moves $\vec a\in A^n$ and $\vec b\in B^n$.
In this game, in the first round player I plays $a_{n+1}\in A$ or $b_{n+1}\in B$ and player II responds with $b_{n+1}\in B$ or $a_{n+1}\in A$, etc.
Then at the end we have $(a_1,\dots,a_n,a_{n+1},\dots,a_{n+k})\in A^{n+k}$ and $(b_1,\dots,b_n,b_{n+1},\dots,b_{n+k})\in B^{n+k}$ and player II wins if $a_i\varmapsto b_i$ is a partial isomorphism.
For $n=0$ this just coincides with our previous definition of an EF-game.

Now we can formalize the concept of a winning strategy:

\bdefn

    Player II has a {\emphcolor winning strategy} in $\Gamma_0({\cal A},{\cal B},\vec a,\vec b)$ if $a_i\mapsto b_i$ for $1\leq i\leq n$ is a partial isomorphism.
    Inductively, player II has a winning strategy in $\Gamma_{k+1}({\cal A},{\cal B},\vec a,\vec b)$ provided that for any $a\in A$ there exists a $b\in B$ and for any $b\in B$ there exists an $a\in A$
    such that player II has a winning strategy in $\Gamma_k({\cal A},{\cal B},\vec a\#(a),\vec b\#(b))$ where $\vec v\#\vec w$ is the concatenation of two vectors.

\edefn

Now we write $({\cal A},\vec a)\sim_k({\cal B},\vec b)$ if player II has a winning strategy in $\Gamma_k({\cal A},{\cal B},\vec a,\vec b)$.
In particular ${\cal A}\sim_k{\cal B}$ if this holds for $\vec a,\vec b=\varnothing$.

\blemm

    Let $({\cal A},\vec a)\sim_k({\cal B},\vec b)$ where $\vec a\in A^n$ and $\vec b\in B^n$, then for every $\phi=\phi(\vec x)$ with quantifier rank $\leq k$,
    ${\cal A}\vDash\phi(\vec a)\iff{\cal B}\vDash\phi(\vec b)$.

\elemm

We prove this by induction on $k$: for $k=0$ since $a_i\mapsto b_i$ is a partial isomorphism we have that $\set{a_1,\dots,a_n}\vDash\phi(\vec a)\iff\set{b_1,\dots,b_n}\vDash\phi(\vec b)$ for all
quantifier-free $\phi$ (ie. formulas with quantifier rank zero).
We have that by \refmath{substructuretheorem} this is equivalent to ${\cal A}\vDash\phi(\vec a)\iff{\cal B}\vDash\phi(\vec b)$.
Now suppose that $({\cal A},\vec a)\sim_{k+1}({\cal B},\vec b)$ then let $\phi=\forall y\alpha(\vec x,y)$ where ${\sl qrank}\alpha\leq k$.
Let ${\cal A}\vDash\forall y\alpha(\vec a,y)$ and $b\in B$, then player II can choose an $a\in A$ such that $({\cal A},\vec a\#(a))\sim_k({\cal B},\vec b\#(b))$ and so by our inductive hypothesis we have
that since ${\cal A}\vDash\alpha(\vec a,a)$ we get ${\cal B}\vDash\alpha(\vec b,b)$.
Since $b\in B$ is arbitrary we have ${\cal B}\vDash\forall y\alpha(\vec b,x)=\phi(\vec b)$.
Arguing similarly we obtain the converse.
All formulas of quantifier rank $\leq k+1$ are boolean combinations of formulas of the form $\forall y\alpha(\vec x,y)$ and so we obtain the result for arbitrary formulas by inducting on $\land$ and $\neg$
which is simple.
\qed

\refmath[Theorem]{efgametheorem} results simply in applying this lemma for all $k$.
This is actually quite a practical method proving the completeness of certain theories, like ${\sf SO}_{10},{\sf SO}_{01},{\sf SO}_{00}$.
${\sf SO}_{11}$ is obviously not complete as each $\exists_n$ is independent of it.
Since two finite discrete orders are isomorphic, and it can be shown by use of an EF-game that infinite models of ${\sf SO}_{11}$ are elementarily equivalent we get that
$X=\set{{\sf L},{\sf R},\exists_1,\exists_2,\dots}$ is a boolean basis of ${\cal L}^0$ in ${\sf SO}$ (by the \refmath{basistheorem}).

