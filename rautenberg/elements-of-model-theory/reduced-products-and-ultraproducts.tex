\bdefn

    A {\emphcolor filter}\addtoindex{filter} on a set $I$ is a non-empty collection ${\cal F}\subseteq\powsetof I$ which is closed under intersections and also upward closed (meaning if $A\in{\cal F}$ and
    $A\subseteq B$, then $B\in{\cal F}$).
    A filter is {\it proper} if ${\cal F}\neq\powsetof I$, which is equivalent to $\varnothing\notin{\cal F}$.

\edefn

By upward closure, obviously $I\in{\cal F}$.
And ${\cal F}$ being a filter is equivalent to
$$ A\cap B\in{\cal F} \iff A\in{\cal F}\hbox{ and }B\in{\cal F} $$
every filter satisfies this, and a collection which satisfies this is a filter: such a collection is obviously closed under intersections, and if $A\in{\cal F}$ and $A\subseteq{\cal F}$ then
$A\cap B=A\in{\cal F}$, so $B\in{\cal F}$.

For a set $K\subseteq I$, $\set{J\subseteq I}[K\subseteq K]$ is a filter, the {\it principal filter generated by $K$}, and it is proper if and only if $K\neq\varnothing$.
If $I$ is infinite, the set of all cofinite subsets of $I$ ($\set{J\subseteq I}[I\setminus J\hbox{ is finite}]$) is also a proper filter, called the {\it Frech\'et filter} of $I$.
This is a filter since $A\cap B$ is cofinite if and only if $A$ and $B$ are, since $I\setminus(A\cap B)=(I\cap A)\cup(I\cap B)$.

\bdefn

    A filter ${\cal F}\subseteq\powsetof I$ is an {\emphcolor ultrafilter}\addtoindex{filter}[ultrafilter] if it also satisfies $I\setminus A\in{\cal F}\iff A\notin{\cal F}$.
    An ultrafilter is obviously proper, and if it contains the Frech\'et filter (ie. contains all cofinite sets), it is {\it nontrivial}.

\edefn

Suppose ${\cal F}$ is a proper filter.
Then it is an ultrafilter if and only if $A\cup B\in{\cal F}$ if and only if $A\in{\cal F}$ or $B\in{\cal F}$.
An ultrafilter satisfies this property: if $A\cup B\in{\cal F}$ and suppose $A,B\notin{\cal F}$ then $A^c,B^c\in{\cal F}$ so $(A\cup B)\cap A^c\cap B^c=\varnothing\in{\cal F}$ in contradiction.
And suppose $A\in{\cal F}$ or $B\in{\cal F}$ then $A,B\subseteq A\cup B$ so by upward closure $A\cup B\in{\cal F}$.
And if ${\cal F}$ satisfies this property, then suppose $A^c\in{\cal F}$ then since the filter is proper $A\notin{\cal F}$.
And if $A\notin{\cal F}$, since $A\cup A^c\in{\cal F}$ either $A$ or $A^c$ is in the filter, so $A^c\in{\cal F}$.

This property can be extended inductively to ${\cal F}$ is an ultrafilter if and only if $A_1\cup\cdots\cup A_n\in{\cal F}$ if and only if $A_i\in{\cal F}$ for some $1\leq i\leq n$.

Ultrafilters are maximal proper filters (the converse also holds).
If ${\cal F}$ is an ultrafilter and ${\cal F}\subseteq{\cal F}'$, then let $J\in{\cal F}'$, if $J\notin{\cal F}$ then $J^c\in{\cal F}\subseteq{\cal F}'$ which contradicts ${\cal F}'$ being proper.

\bprop

    Every trivial ultrafilter over an infinite set $I$ is of the form $\set{J\subseteq I}[i\in J]$ for some $i\in I$.
    Such a filter is called the {\emphcolor principal ultrafilter} generated by $i$.

\eprop

Firstly this is indeed a trivial ultrafilter: $i\in A\cap B$ if and only if $i\in A$ and $i\in B$ so it is a filter, $i\in I\setminus A$ if and only if $i\notin A$ so it is an ultrafilter, and
$I\setminus\set i$ is cofinite yet does not exist in the filter so it is trivial.

Now suppose ${\cal F}$ is a trivial ultrafilter over $I$, then let $J$ be cofinite not in ${\cal F}$.
Suppose that $J=I\setminus\set{i_1,\dots,i_n}$ and so $\set{i_1,\dots,i_n}\in U$, and by above this means there exists some $i\in\set{i_1,\dots,i_n}$ such that $\set i\in{\cal F}$.
So by upward closure we have $\set{J\subseteq I}[i\in J]\subseteq{\cal F}$ and since the left is an ultrafilter and thus maximal, we have equality.
\qed

This proof showed that if $\set i\in{\cal F}$ for an ultrafilter, then ${\cal F}$ is the principal ultrafilter generated by $i$.
In particular if $I$ is finite, all of its ultrafilters are trivial (equivalently principal) since $I=\set{i_1,\dots,i_n}\in{\cal F}$ and so some $\set{i_j}\in{\cal F}$.

We utilize the propositional compactness theorem to prove the following

\bthrm[title=The Ultrafilter Theorem, name=ultrafiltertheorem]

    Every subset $F\subseteq\powsetof I$ can be extended to an ultrafilter ${\cal U}$ if it has the {\emphcolor finite intersection property}: for every $M_1,\dots,M_n\in F$, their intersection is nonempty.

\ethrm

For every $J\subseteq I$ define a propositional variable $p_J$.
Then we define the axiom system
$$ X\colon\qquad p_{M\cap N}\oto p_M\land p_N,\quad p_{M^c}\oto\neg p_M,\quad p_J\qquad(M,N\subseteq I,J\in F) $$
if $w\vDash X$ then define ${\cal U}\coloneqq\set{J\subseteq I}[w\vDash p_J]$.
This is obviously an ultrafilter containing $F$.
So all we must do is prove that every finite subset of $X$ is satisfiable, for which it is sufficient to prove the ultrafilter theorem for finite $I$.
But this is easy, let $F=\set{M_1,\dots,M_n}$, $\varnothing\neq D=M_1\cap\cdots\cap M_n$ and $i\in D$, then the principal ultrafiler generated by $i$ contains $F$.
\qed

We know define reduced and ultraproducts, this is a lengthier definition so stay vigilant.
Let $({\cal A}_i)_{i\in I}$ be a family of ${\cal L}$-structures, and $F$ a proper filter on the indexing set $I$.
We define an equivalence relation $\approx_F$ on the domain of ${\cal B}=\prod_{i\in I}{\cal A}_i$ by
$$ (a_i)_{i\in I}\approx_F(b_i)_{i\in I}\iff\set{i\in I}[a_i=b_i]\in F $$
In order to make notation more readable, we will use single letters like $a$ or $b$ to denote elements of $B$, like $(a_i)_{i\in I}$.
This is indeed an equivalence relation: define $I_{a=b}\coloneqq\set{i\in I}[a_i=b_i]$, obviously $I_{a=a}=I\in F$ so $\approx_F$ is reflexive, and $I_{a=b}=I_{b=a}$ so $\approx_F$ is symmetric, and
$I_{a=b}\cap I_{b=c}\subseteq I_{a=c}$ so $\approx_F$ is transitive (since $I_{a=b},I_{b=c}\in F$ implies $I_{a=b}\cap I_{b=c}\in F$ which implies in turn that $I_{a=c}\in F$).

Furthermore $\approx_F$ is a congruence on the algebraic reduct of ${\cal B}$: let $f$ be an $n$-ary function symbol in ${\cal L}$ and suppose $\vec a\approx_F\vec b$ (where $\vec a=(a^1,\dots,a^n)$ and
$\vec b=(b^1,\dots,b^n)$).
Define $I_{\vec a=\vec b}=\bigcap_{i=1}^n I_{a^i=b^i}$, and so $I_{\vec a=\vec b}\in F$ since $I_{a^i=b^i}\in F$ for every $i$.
And surely $I_{\vec a=\vec b}\subseteq I_{f^{\cal B}\vec a=f^{\cal B}\vec b}$, so $f^{\cal B}\vec a\approx_Ff^{\cal B}\vec b$.

So let $C=\set{a/F}[a\in B]$ be the partition of $B$ with respect to $\approx_F$ where $a/F$ denotes the equivalence class of $a$ with respect to $\approx_F$, meaning $a/F=b/F$ if and only if $a\approx_Fb$
or equivalently $I_{a=b}\in F$.
This is the domain of an ${\cal L}$-structure ${\cal C}$ where for function symbols $f$ we define $f^{\cal C}(\vec a/F)\coloneqq(f^{\cal B}\vec a)/F$.
This definition is well-defined since $f$ is a congruence.
And of course for constant symbols $c$, $c^{\cal C}\coloneqq c^{\cal B}/F$.

For a relation symbol $r$, define $I_{r\vec a}\coloneqq\set{i\in I}[r^{{\cal A}_i}\vec a_i]$ where $\vec a_i\coloneqq(a_i^1,\dots,a_i^n)$.
Then define $r^{\cal C}\vec a/F\iff I_{r\vec a}\in F$.
This is well defined: if $\vec a\approx_F\vec b$ then $I_{\vec a=\vec b}\in F$, and $I_{r\vec a}\cap I_{\vec a=\vec b}\subseteq I_{r\vec b}$, so since $F$ is a filter this implies $I_{r\vec b}\in F$ as well.

This structure ${\cal C}$ is called the {\it reduced product} of $\set{{\cal A}_i}_{i\in I}$ by $F$\addtoindex{reduced product}, and is denoted $\prod_{i\in I}^F{\cal A}_i$ or $\prod_{i\in I}{\cal A}_i/F$.

Let ${\cal C}=\prod_{i\in I}^F{\cal A}_i$, and let $w\colon\Var\longto B=\prod_{i\in I}A_i$ be a valuation, define the valuation $w_i\colon\Var\longto A_i$ defined by $x\varmapsto(x^w)_i$, so
$x^w=(x^{w_i})_{i\in I}$.
Term induction gives $t^w=(t^{w_i})_{i\in I}$ for all terms $t$.
Then define the valuation $w/F\colon\Var\longto C$ by $x^{w/F}=x^w/F$, and by term induction we have $t^{w/F}=t^w/F$.
Indeed: $(f\vec t)^{w/F}=f^{\cal C}(\vec t^{w/F})=f^{\cal C}(\vec t^w/F)=(f^{\cal B}\vec t^w)/F=(f\vec t)^w/F$.
Let $w'\colon\Var\longto C$ be a valuation, then define $w\colon\Var\longto B$ by choosing $x^w\in x^{w'}$, and so $x^{w'}=x^w/F$, meaning $w'=w/F$.
So every valuation on $C$ is of the form $w/F$ for some valuation $w$ on $B$.

Let $w\colon\Var\longto B$ and $\alpha\in{\cal L}$ then define $I^w_\alpha\coloneqq\set{i\in I}[{\cal A}_i\vDash\alpha{[w_i]}]$.
Then we have that $I^w_{\exists x\beta}\subseteq I^{w'}_\beta$ where $w'=w^a_x$ for some $a\in B$.
Indeed let $i\in I^w_{\exists x\beta}$ so ${\cal A}_i\vDash\exists x\beta[w_i]$, then there exists some $a_i\in A_i$ such that ${\cal A}_i\vDash\beta[w_i{}^{a_i}_x]$.
For $i\notin I^w_{\exists x\beta}$ choose any $a_i$.
Then define $a=(a_i)_{i\in I}$ and so for every $i\in I^w_{\exists x\beta}$, ${\cal A}_i\vDash\beta[w_i{}^{a_i}_x]=\beta[w'_i]$, so $i\in I^{w'}_\beta$.

The case of particular interest is when $F$ is an ultrafilter, in such a case $\prod_{i\in I}^F{\cal A}_i$ is called an {\it ultraproduct}\addtoindex{ultraproduct} of $\set{{\cal A}_i}_{i\in I}$.
If ${\cal A}_i={\cal A}$ for all $i\in I$ then we write ${\cal A}^I/F$ in place of $\prod_{i\in I}^F{\cal A}$ and this is called an {\it ultrapower}\addtoindex{ultraproduct}[ultrapower] of ${\cal A}$.
An important theorem regarding ultrapowers but not proven here is that ${\cal A}\equiv{\cal B}$ if and only if ${\cal A}$ and ${\cal B}$ have isomorphic unltrapowers.

\bthrm[title=\L o\'s's Ultraproduct Theorem, name=losultraproducttheorem]

    Let ${\cal C}=\prod_{i\in I}^F{\cal A}_i$ be an ultraproduct of the ${\cal L}$-structures ${\cal A}_i$.
    Then for all formulas $\alpha\in{\cal L}$ and all valuations $w\colon\Var\longto\prod_{i\in I}A_i$,
    $$ {\cal C}\vDash\alpha[w/F] \iff I^w_\alpha\in F \quad (\coloneqq\set{i\in I}[{\cal A}_i\vDash\alpha{[w_i]}]\in F) $$

\ethrm

We prove this by induction on $\alpha$.
For equations $t_1\eq t_2$,

\medskip
{\tabskip=0pt plus 1fil
\halign to\hsize{$#$\hfil\tabskip=.25cm&$#$\hfil\tabskip=.5cm&(#)\hfil\tabskip=0pt plus 1fil\cr
    {\cal C}\vDash t_1\eq t_2[w/F] &\iff t_1^{w/F} = t_2^{w/F} \iff t_1^w/F = t_2^w/F & since $t^{w/F}=t^w/F$\cr
        &\iff \set{i\in I}[t_1^{w_i}=t_2^{w_i}]\in F & since $t^w=(t^{w_i})_{i\in I}$\cr
        &\iff \set{i\in I}[{\cal A}_i\vDash t_1\eq t_2{[w_i]}] \iff I^w_{t_1\eq t_2}\in F\cr
}}
\medskip

For prime formulas $r\vec t$:

\medskip
{\tabskip=0pt plus 1fil
\halign to\hsize{$#$\hfil\tabskip=.25cm&$#$\hfil\tabskip=.5cm&(#)\hfil\tabskip=0pt plus 1fil\cr
    {\cal C}\vDash r\vec t[w/F] &\iff r^{\cal C}\vec t^{w/F}\iff r^{\cal C}\vec t^w/F & since $t^{w/F}=t^w/F$\cr
        &\iff I_{r\vec t^w}\in F \iff \set{i\in I}[r^{{\cal A}_i}\vec t^w_i]=\set{i\in I}[r^{{\cal A}_i}\vec t^{w_i}] & since $t^w_i=t^{w_i}$\cr
        &\iff \set{i\in I}[{\cal A}_i\vDash r\vec t{[w_i]}]\in F\iff I^w_{r\vec t}\in F\cr
}}
\medskip

For conjunctions:

\medskip
{\tabskip=0pt plus 1fil
\halign to\hsize{$#$\hfil\tabskip=.25cm&$#$\hfil\tabskip=.5cm&(#)\hfil\tabskip=0pt plus 1fil\cr
    {\cal C}\vDash \alpha\land\beta[w/F] &\iff {\cal C}\vDash\alpha,\beta[w/F]\cr
        &\iff I^w_\alpha,I^w_\beta\in F& induction hypothesis\cr
        &\iff I^w_\alpha\cap I^w_\beta\in F& filter property\cr
        &\iff I^w_{\alpha\land\beta}\in F & since $I^w_\alpha\land I^w_\beta=I^w_{\alpha\land\beta}$\cr
}}
\medskip

For negations: ${\cal C}\vDash\neg\alpha[w/F]\iff{\cal C}\nvDash\alpha[w/F]\iff I^w_\alpha\notin F$, and since $F$ is an ultrafilter this is equivalent to $\iff I\setminus I^w_\alpha\in F$.
And $I\setminus I^w_\alpha=\set{i\in I}[{\cal A}_i\nvDash\alpha{[w_i]}]=\set{i\in I}[{\cal A}_i\vDash\neg\alpha{[w_i]}]=I^w_{\neg\alpha}$, so this is equivalent to $I^w_{\neg\alpha}\in F$ as required.

For $\forall x\alpha$ we first show that $I^w_{\forall x\alpha}\in F$ implies ${\cal C}\vDash\forall x\alpha$.
Let $a\in\prod_{i\in I}A_i$ and $w'\coloneqq w^a_x$.
Since $I^w_{\forall x\alpha}\subseteq I^{w'}_\alpha$, we have that $I^{w'}_\alpha\in F$ so ${\cal C}\vDash\alpha[w^a_x/F]$ by the induction hypothesis.
But since $a$ is arbitrary, ${\cal C}\vDash\forall x\alpha[w/F]$.

To prove the converse, this is equivalent to $I^w_{\forall x\alpha}\notin F\implies{\cal C}\nvDash\forall x\alpha$.
Since $F$ is an ultrafilter this is equivalent to $I^w_{\exists x\beta}\in F\implies{\cal C}\vDash\exists x\beta[w/F]$ with $\beta\coloneqq\neg\alpha$.
If $I^w_{\exists x\beta}\in F$ then since $I^w_{\exists x\beta}\subseteq I^{w'}_\beta$ for $w'=w^a_x$ for some $a\in B$, $I^{w'}_\beta\in F$ and so ${\cal C}\vDash\beta[w'/F]$, and so
${\cal C}\vDash\exists x\beta[w/F]$ as required.
\qed

For sentences $\alpha$ since the valuation does not affect its satisfiability in a structure,
$$ \prod_{i\in I}^F{\cal A}_i\vDash\alpha \iff \set{i\in I}[{\cal A}_i\vDash\alpha]\in F $$
for an ultrafilter $F$.

Notice that we can define the embedding $\iota\colon a\varmapsto(a)_{i\in I}/F$ from ${\cal A}$ to ${\cal A}^I/F$.
This is an elementary embedding: firstly it is injective since $(a)_{i\in I}\approx_F(b)_{i\in I}$ if and only if $\set{i\in I}[a=b]\in F$, and this set is either $I$ or $\varnothing$.
Since $F$ is proper, this is if and only if $a=b$.
And it is an embedding since $f\iota\vec a=f(a_1)/F\cdots(a_n)/F=(f\vec a)/F=\iota f\vec a$, and $r\iota\vec a=r(a_1)/F\cdots(a_n)/F\iff\set{i\in I}[r\vec a]\in F$ which is either $I$ or $\varnothing$
so this is if and only if $r\vec a$.
Notice that if $w/F$ maps $x_i\varmapsto\iota a_i$ then all $w_j$ are equal and map $x_i\mapsto a_i$, since then $(x_i^{w_j})_{j\in I}/F=(a_i)_{j\in I}/F=\iota a_i=x_i^{w/F}$.
Thus
$$ {\cal A}^I/F\vDash\alpha(\iota a_1,\dots,\iota a_n) \iff \set{i\in I}[{\cal A}\vDash\alpha(a_1,\dots,a_n)] \in F $$
and this set is either $I$ or $\varnothing$ so ${\cal A}\vDash\alpha(a_1,\dots,a_n)$.
Therefore ${\cal A}\preceq{\cal A}^I/F$.

This also gives us a purely model-theoretic proof of the compactness theorem:

\bthrm[title=The Compactness Theorem, name=mtcompact]

    Let $X\subseteq{\cal L}$ and let $I$ be the set of all finite subsets of $X$.
    If every $i\in I$ has a model $({\cal A}_i,w_i)$ then there exists an ultrafilter $F$ on $I$ such that $\prod_{i\in I}^F{\cal A}_i\vDash X[w/F]$ where $x^w=(x^{w_i})_{i\in I}$.
    Meaning that if every finite subset of $X$ has a model, then so too does $X$.

\ethrm

For every $\alpha\in X$, define $J_\alpha\coloneqq\set{i\in I}[\alpha\in i]$, and then define $E\coloneqq\set{J_\alpha}[\alpha\in X]$.
$E$ has the finite intersection property, since $\set{\alpha_1,\dots,\alpha_n}\in J_{\alpha_1}\cap\cdots\cap J_{\alpha_n}$, and so by \refmath{ultrafiltertheorem} there exists an ultrafilter $F$ on $I$ such
that $E\subseteq F$.
If $\alpha\in X$ and $i\in J_\alpha$ (meaning $\alpha\in i$), then ${\cal A}_i\vDash\alpha[w_i]$ and so $J_\alpha\subseteq I^w_\alpha$, so $I^w_\alpha\in F$.
So by \refmath{losultraproducttheorem} this means $\prod_{i\in I}^F{\cal A}_i\vDash\alpha[w/F]$ as required.
\qed

Let us define $\boldsymbol K_{\cal L}$ to be the class of all ${\cal L}$-structures.
Recall that a class of structures is {\it $\Delta$-elementary} if it is the class of models of some first order theory $T$, $\Md T$.
And it is {\it elementary} if it is the class of some finitely axiomatizable first order theory.\addtoindex{elementary classes}

\def\K{\boldsymbol K}
\bthrm

    Let $\K\subseteq\K_{\cal L}$, then
    \benum
        \item $\K$ is $\Delta$-elementary if and only if $\K$ is closed under elementary equivalence and ultraproducts,
        \item $\K$ is elementary if and only if $\K$ is closed under elementary equivalence and ultraproducts and $\K^c$ is closed under ultraproducts.
    \eenum

\ethrm

\benum
    \item Obviously if $\K$ is $\Delta$-elementary then it is closed under elementary equivalence, since if ${\cal A}\equiv{\cal B}$ and ${\cal A}\vDash T$ then ${\cal B}\vDash T$.
    And if ${\cal A}_i\vDash T$ for all $i\in I$ then $\prod_{i\in I}^F{\cal A}_i\vDash T$ if and only if for every $\alpha\in T$, $\set{i\in I}[{\cal A}_i\vDash\alpha]\in F$.
    But this is just $I$, so we have the required.

    Let us define $T\coloneqq\Th\K$, and so we claim that $\K=\Md T$.
    Obviously $\K\subseteq\Md T$, so let ${\cal A}\vDash T$, and let $I$ be the set of all finite subsets of $\Th{\cal A}$.
    For every $i=\set{\alpha_1,\dots,\alpha_n}\in I$ there exists some ${\cal A}_i\in\K$ such that ${\cal A}_i\vDash i$.
    Otherwise $\bigvee_{i=1}^n\neg\alpha_i\in T$, which contradicts $i\subseteq\Th{\cal A}$ (since ${\cal A}\vDash\alpha_i,\neg\alpha_i$ for some $i$).
    So every finite subset of $\Th{\cal A}$ is satisfied by some ${\cal A}_i$ and thus by the above theorem there exists an ultrafilter $F$ such that $\prod_{i\in I}^F{\cal A}_i\vDash\Th{\cal A}$, and so
    $\prod_{i\in I}^F{\cal A}_i\equiv{\cal A}$.
    But $\K$ is closed under ultraproducts so $\prod_{i\in I}^F{\cal A}_i\in\K$, and it is closed under equivalences so ${\cal A}\in\K$.
    So ${\cal A}\vDash T\iff{\cal A}\in\K$, meaning $\K$ is $\Delta$-elementary.

    \item If $\K$ is elementary then both $\K$ and $\K^c$ are $\Delta$-elementary (as $\K=\Md\alpha\implies\K^c=\Md\neg\alpha$) and so both are closed under ultraproducts and equivalence.
    By above, $\K=\Md S$ for some $S\subseteq{\cal L}^0$.
    Let $I$ be the set of all finite nonempty subsets of $S$, then there exists some $i=\set{\alpha_1,\dots,\alpha_n}\in I$ such that $\Md i\subseteq\K$.
    Otherwise let ${\cal A}_i\vDash i$ but ${\cal A}_i\in\K^c$ for every $i\in I$.
    Again by the above compactness theorem there exists an ultraproduct ${\cal C}$ such that ${\cal C}\vDash i$ for every $i\in I$ and since $\K^c$ is closed under ultraproducts, ${\cal C}\in\K^c$.
    And so ${\cal C}\vDash S$, meaning ${\cal C}\in\K$ in contradiction.
    But since $\K=\Md S\subseteq\Md i$, we have that $\K=\Md i=\Md\bigwedge_{i=1}^n\alpha_i$, meaning it is elementary.
    \qed
\eenum

\bexam

    Let $\K$ be the $\Delta$-elementary class of fields of characteristic $0$.
    We claim that $\K$ is not elementary, and we will prove it by showing that $\K^c$ is not closed under ultraproducts.
    Let ${\cal P}_i$ be the prime field of characteristic $p_i$, and let $F$ be a nontrivial ultrafilter on ${\bb N}$.
    Then $\prod_{i\in I}^F{\cal P}_i$ has characteristic $0$, since $\set{i\in I}[{\cal P}_i\vDash\neg{\tt char}_p]$ for any prime $p$ is cofinite and thus belongs to $F$.
    So $\prod_{i\in I}^F{\cal P}_i\vDash\neg{\tt char}_p$, meaning it is a field of characteristic $0$ (it is a field since the class of fields is $\Delta$-elementary and therefore closed under
    ultraproducts).

\eexam

Notice that for the minimal filter $F=\set I$, $a\approx_F b\iff\set{i\in I}[a_i=b_i]\in F\iff a_i=b_i$ for all $i\in I$, thus $\prod_{i\in I}^{\set I}{\cal A}_i\cong\prod_{i\in I}{\cal A}_i$.
So whatever we can prove on reduced products holds for direct products too.
From here on, all filters are assumed to be proper.

\bthrm

    Let ${\cal C}=\prod_{i\in I}^F{\cal A}_i$ be a reduced product, $w\colon\Var\longto\prod_{i\in I}{\cal A}_i$, and $\alpha$ a Horn formula.
    Then $I^w_\alpha\in F\implies{\cal C}\vDash\alpha[w/F]$.
    In particular if $\alpha$ is a Horn sentence, $\set{i\in I}[{\cal A}_i\vDash\alpha]\implies{\cal C}\vDash\alpha$.

\ethrm

We prove this by induction on Horn formulas.
For prime formulas both directions of the conditional hold, since in our proof of \refmath{losultraproducttheorem} for the step on prime formulas, no ultraproduct property was used.
Since $F$ is proper $I^w_{\neg\alpha}\in F\implies I^w_\alpha\notin F\implies{\cal C}\nvDash\alpha[w/F]\implies{\cal C}\vDash\neg\alpha[w/F]$ for $\alpha$ prime, and so the conditional holds for all
literals.
Now suppose the condition holds for prime $\alpha$ and basic Horn formula $\beta$ and suppose $I^w_{\alpha\to\beta}$, then if ${\cal C}\vDash\alpha[w/F]$ then $I^w_\alpha\in F$ since $\alpha$ is prime.
Since $I^w_\alpha\cap I^w_{\alpha\to\beta}\subseteq I^w_\beta$, we get that $I^w_\beta\in F$ so ${\cal C}\vDash\beta[w/F]$ by induction, meaning ${\cal C}\vDash\alpha\to\beta[w/F]$.
Induction on $\land$ and $\forall$ proceed similar to \L o\'s, and $\exists$ follows from $I^w_{\exists x\beta}\subseteq I^{w^a_x}_\beta$ for some $a\in\prod_{i\in I}A_i$.
\qed

This means that the model classes of Horn theories are closed under reduced products, and in particular direct products.
The converse holds as well: every class of models closed under reduced products is a Horn theory, but this is significantly more challenging to prove.

