We say that a theory $T$ {\it allows quantifier elimination}\addtoindex{quantifier elimination} if for every $\phi\in{\cal L}$ there exists a quantifier-free $\phi'$ such that $\phi\equiv_T\phi'$.
By \refmath[theorem]{modelcompleteequiv}, a theory which allows quantifier elimination is model complete since quantifier-free formulas are also $\forall$-formulas.
Therefore a theory which allows quantifier elimination must be a $\forall\exists$-theory.

Since $\set{\exists,\neg,\land}$ is a complete bundle in the sense that every formula generated by this bundle is equivalent to one generated by $\set{\forall,\neg,\land}$ and vice versa, to show that a
theory allows quantifier elimination we need only show that $\exists x\alpha$ is equivalent to a quantifier-free formula for $\alpha$ quantifier-free.
This is as we can induct over the construction over formulas and convert them to quantifier-free equivalents.

We can put $\alpha$ into DNF, and since $\exists$ distributes over disjunctions, we can assume that $\alpha$ is a conjunction of literals.
If any of these literals do not contain $x$, then its prefix $\exists x$ can be disregarded as $\exists x(\alpha\land\beta)\equiv\exists x\alpha\land\beta$ for $x\notin\var\beta$, so we can further assume
that $x$ occurs in each literal.
Furthermore we can assume that no literal is of the form $x\eq t$ for $x\notin\var t$ as $\exists x(x\eq t\land\alpha)\equiv\alpha\frac tx$, and thus the quantifier has been eliminated.
Using bound renaming we can assume that $x\neq\v_0$ and so neither $x\eq x$ nor $x\neqb x$ are literals occurring in any conjunction, as they can be replaced with $\top$ and $\bot$ respectively (the reason
for $x\neq\v_0$ is as $\top\coloneqq\v_0\eq\v_0$ and $\bot\coloneqq\v_0\neqb\v_0$.
Alternatively we could simply define $\top$ and $\bot$ as new prime formulas).

Call an $\exists$-formula {\it simple} if it is of the form $\exists x\bigwedge_i\alpha_i$ where $\alpha_i$ is a conjunction of literals, and for every $i$, $x\in\var\alpha_i$.
Then we have proven the following

\bthrm

    $T$ allows quantifier elimination if and only if every simple $\exists$-formula $\exists x\bigwedge_i\alpha_i$ is equivalent in $T$ to some quantifier-free formula.
    We can further assume that none of the literals $\alpha_i$ are $x\eq x$, $x\neqb x$, or $x\eq t$ with $x\in\var t$.

\ethrm

\bexam

    $T={\sf DO}_{00}$ allows quantifier elimination.
    Notice that
    $$ y\nless z\equiv_T z<y\lor z\eq y,\quad z\neqb y\equiv_T z<y\lor y<z $$
    and since $(\alpha\lor\beta)\land\gamma\equiv(\alpha\land\gamma)\lor(\beta\land\gamma)$, we may assume that the conjunctions $\alpha_i$ contain no negations.
    Thus we deal only with formulas of the form
    $$ \exists x(y_1<x\land\cdots\land y_n<x\land x<z_1\land\cdots\land x<z_m) $$
    This is equivalent to $\bot$ if $x=y_i$ or $x=z_i$ for any $i$.
    If $n=0$ or $m=0$ this is equivalent to $\top$.
    Otherwise it is equivalent to $\bigwedge_{i,j}y_i<z_j$ which is quantifier-free.

\eexam

Notice that ${\sf DO}$ itself does not allow quantifier elimination as $\alpha(y)\coloneqq\exists x\,x<y$ has no quantifier-free equivalent.
If it were then for any ${\cal A},{\cal B}\vDash{\sf DO}$ where ${\cal A}\subseteq{\cal B}$ and $a\in A$, ${\cal B}\vDash\alpha(a)\implies{\cal A}\vDash\alpha(a)$.
But let ${\cal B}\vDash{\sf DO}_{00}$ and ${\cal A}\vDash{\sf DO}_{10}$ and let $a$ be ${\cal A}$'s right edge element (eg. $A=[1,\infty)\cap{\bb Q}$ and $B={\bb Q}$, $a=1$).

And ${\sf SO}$ does not allow quantifier elimination as it is not a $\forall\exists$-theory.
And the same holds for ${\sf SO}_{ij}$.

\bexam

    Let ${\sf ZGE}$ be the theory in ${\cal L}\set{0,1,+,-,<,2\bdivs,3\bdivs,\dots}$, whose axioms include the axioms for ordered abelian groups as well as
    $$ \forall x(0<x\oto 1\leq x),\quad\forall x(m\bdivs x\oto\exists y\,my\eq x),\quad \vartheta_m\coloneqq\forall x\bigvee_{k=0}^{m-1}m\bdivs x+k \qquad (m=2,3,\dots) $$
    The reducts of ${\sf ZGE}$-models to ${\cal L}\coloneqq{\cal L}\set{0,1,+,-,<}$ are called {\it ${\bb Z}$-groups}.
    $\vartheta_m$ states that for any ${\bb Z}$-group $G$, $G/mG$ is cyclic of order $m$.
    Let ${\sf ZG}$ be the ${\cal L}$-reduct theory of ${\sf ZGE}$ (which is well-defined as $m\bdivs$ is explicitly defined in ${\sf ZGE}$), its models are precisely all ${\bb Z}$-groups since ${\sf ZGE}$
    is a definitorial and therefore conservative extension of ${\sf ZG}$.
    Notice that $\vdash_{\sf ZG}\forall x\eta_n$ for every $n$ where $\eta_n\coloneqq 0\leq x<n\to\bigvee_{k=0}^{n-1}x\eq k$.

    We will now show that ${\sf ZGE}$ allows quantifier-elimination.
    Notice that
    $$ t\neqb s\equiv_{\sf ZGE}s<t\lor t<s,\quad m\bndivs t\equiv_{\sf ZGE}\bigvee_{i=1}^{m-1}m\bdivs t+i,\quad m\bdivs t\equiv_{\sf ZGE}m\bdivs-t $$
    And so we can assume that the kernel of a simple $\exists x$-formula is a conjunction of formulas of the form $n_ix\eq t_i^0,n_i'x<t_i^1,t_i^2<n_i''x,m_i\bdivs n_i'''x+t_i^3$ where $x\notin\var t_i^j$.
    Since we have that $t<s\equiv_{\sf ZGE}nt<ns$ and $m\bdivs t\equiv_{\sf ZGE}nm\bdivs nt$ for $n\neq0$ we can assume that $n_i=n_i'=n_i''=n_i'''=n>1$ for some $n$ by multiplying both sides of each
    literal by a suitable value.
    This means that we can asssume that a simple $\exists$-formula is of the form
    $$ \exists x\parens{\bigwedge_{i=1}^{k_0}nx\eq t_i^0\land\bigwedge_{i=1}^{k_1}t_i^1<nx\land\bigwedge_{i=1}^{k_2}nx<t_i^2\land\bigwedge_{i=1}^{k_3}m_i\bdivs nx+t_i^3} $$
    Let $y=nx$ and $m_0=n$, we get that this is equivalent to
    $$ \exists y\parens{\bigwedge_{i=1}^{k_0}y\eq t_i^0\land\bigwedge_{i=1}^{k_1}t_i^1<y\land\bigwedge_{i=1}^{k_2}y<t_i^2\land\bigwedge_{i=1}^{k_3}m_i\bdivs y+t_i^3\land m_0\bdivs y} $$
    By the above theorem, we can assume $k_0=0$ and so by substituting $x$ for $y$ and setting $t_0^3=0$, this is equivalent to
    $$ \exists x\parens{\bigwedge_{i=1}^{k_1}t_i^1<x\land\bigwedge_{i=1}^{k_2}x<t_i^2\land\bigwedge_{i=0}^{k_3}m_i\bdivs x+t_i^3} $$
    Let $m$ be the least common multiple of $m_0,\dots,m_{k_3}$.
    We split into cases:

    {\bf Case 1}: $k_1,k_2=0$.
    Then the formula is $\exists x\bigwedge_{i=0}^{k_3}m_i\bdivs x+t_i^3$, which is equivalent to $\bigvee_{j=1}^m\bigwedge_{i=0}^{k_3}m_i\bdivs x+t_i^3$.
    This is since if there exists an $x$ such that $m_i\bdivs x+t_i^3$ for all $0\leq i\leq k_3$, then there must exist some $1\leq j\leq m$ which satifies this: by $\vartheta_m$ there exists such a $j$
    where $m\bdivs x+(m-j)$ and so $m\bdivs x-j$ and so $m_i\bdivs x-j$ for all relevant $i$, and so $m_i\bdivs x+t_i^3-(x-j)=j+t_i^3$ as required.

    {\bf Case 2}: $k_1\neq0$.
    Let $j$ be as above, then this is equivalent to
    $$ \bigvee_{\mu=1}^{k_1}\parens{\bigwedge_{i=1}^{k_1}t_i^1\leq t_\mu^1\land\bigvee_{j=1}^m\parens{\bigwedge_{i=1}^{k_2}t^1_\mu+j<t_i^2\land\bigwedge_{i=0}^{k_3}m_i\bdivs t_\mu^1+j+t_i^3}} $$
    This implies the formula, as we can take $x=t^1_\mu+j$ for the $\mu$ which is satisfied.
    Now suppose $x$ witnesses the formula, then if $\bigwedge_{i=1}^{k_1}t_i^1\leq t_\mu^1$ holds (which it must for some $\mu$ of course), then the rest of the disjuncts hold as well.
    We can take the same $j$ as before, where $m_i\bdivs j+t_i^3$.
    To show $t^1_\mu+j<t_i^2$ it is sufficient to show $t^1_\mu+j\leq x$.
    Suppose the converse then $0<x-t_\mu^1<j$, which means $x-t_\mu^1=k$ for some $k<j$ by $\eta_j$, so $x=k+t_\mu^1$.
    So $m_i\bdivs t_\mu^1+j-x=j-k$ for all $i$, but then $m\bdivs j-k<m$.

    {\bf Case 3}: $k_1=0$ and $k_2\neq0$.
    This is similar to the previous case but using the smallest term among $t_i^2$ instead of the largest of $t_i^1$.

\eexam

\bcoro

    ${\sf ZGE}$ is model complete.
    Furthermore, ${\sf ZGE}$ and ${\sf ZG}$ are both complete and decidable.

\ecoro

Since ${\bb Z}$ is a prime model of ${\sf ZGE}$, it is complete due to it being model complete (by quantifier elimination) and having a prime model.
Since ${\sf ZGE}$ is an explicit extension of ${\sf ZG}$, it being complete implies so too is ${\sf ZG}$.
These theories are complete and axiomatizable and so they are decidable.
\qed

\bdefn

    Let $X\subseteq{\cal L}$ be a set of ${\cal L}$-formulas, then it is a {\emphcolor boolean basis}\addtoindex{boolean basis} for ${\cal L}$ if every formula $\phi\in{\cal L}$ is in $\gen X_T$, ie. every
    formula is equivalent in $T$ to a boolean combination of formulas from $X$.

\edefn

Let ${\cal M},{\cal M}'$ be ${\cal L}$-models, write ${\cal M}\equiv_X{\cal M}'$ to mean that ${\cal M}\vDash\phi\iff{\cal M}'\vDash\phi$ for all $\phi\in X$.

\bthrm[title=Basis Theorem for Formulas, name=formulabasistheorem]

    Let $T$ be an ${\cal L}$-theory and $X\subseteq{\cal L}$.
    Then suppose that ${\cal M}\vDash_X{\cal M}'\implies{\cal M}\equiv{\cal M}'$ for all ${\cal M},{\cal M}'\vDash T$.
    Then $X$ is a boolean basis for ${\cal L}$ in $T$.

\ethrm

Let $\alpha\in{\cal L}$ and define $Y_\alpha\coloneqq\set{\gamma\in\gen X}[\alpha_T\gamma]$.
Then as in the proof of the \refmath{basistheorem}, show that $Y_\alpha\vdash\alpha$ arguing analogously but considering ${\cal L}$-models instead of ${\cal L}$-structures.
\qed

\bdefn

    Call a theory $T$ {\emphcolor substructure complete}\addtoindex{substructure complete} if for all models ${\cal A}\subseteq{\cal B}$ where ${\cal B}\vDash T$, $T+D{\cal A}$ is complete.
    This is equivalent to saying that $T$ is the model completion of $T^\forall$.

\edefn

\bthrm

    For every ${\cal L}$-theory $T$, the following are equivalent:
    $$ (1)\ \hbox{$T$ allows quantifier elimination}\qquad (2)\ \hbox{$T$ is substructure complete} $$

\ethrm

$(1)\implies(2)$: let ${\cal A}$ be a substructure of a $T$-model, $\phi(\vec x)\in{\cal L}$ and $a\in A^n$ such that ${\cal A}\vDash\phi(\vec a)$.
Let ${\cal B}\vDash T,D{\cal A}$ so we can assume ${\cal A}\subseteq{\cal B}$.
Since $T$ allows quantifier elimination, we can assume that $\phi$ is quantifier-free and so ${\cal B}\vDash\phi(\vec a)$.
And since ${\cal B}\vDash T$ is arbitrary we have $D{\cal A}\vdash_T\phi(\vec a)$, so $D{\cal A}+T=\Th{\cal A}_A$, meaning it is complete.

$(2)\implies(1)$: let $X$ be the set of literals in ${\cal L}$, then saying that $T$ allows quantifier elimination is the same as saying that $X$ is a boolean basis for ${\cal L}$ in $T$.
So we will prove that for all ${\cal M},{\cal M}'\vDash T$, ${\cal M}\equiv_X{\cal M}'\implies{\cal M}\equiv{\cal M}'$.
Suppose ${\cal M}=({\cal A},w)\vDash T$ and suppose ${\cal M}\vDash\phi(\vec x)$ where $x\neq\varnothing$, let $a_i=x_i^w$.
Denote ${\cal A}^E$ the substructure of ${\cal A}$ generated by $E\coloneqq\set{a_1,\dots,a_n}$.
By $(2)$, $T+D{\cal A}^E$ is complete and consistent with $\phi(\vec a)$ since ${\cal A}_A$ satisfies $T+D{\cal A}^E+\phi(\vec a)$, this means $D{\cal A}^E\vdash_T\phi(\vec a)$ as it is complete.
Now we showed that this means $D{\cal A}^E\cap{\cal L}E\vdash_T\phi(\vec a)$ (proven in an exercise, boils down to the coincidence theorem).
By the compactness theorem there exist literals $\lambda_1(\vec a),\dots,\lambda_n(\vec a)\in D{\cal A}^E$ such that $\bigwedge_{i=1}^k\lambda_i(\vec a)\vdash_T\phi(\vec a)$.
Since $\vec a$ does not occur in $T$, this means $\bigwedge_{i=1}^k\lambda_i(\vec x)\vdash_T\phi(\vec x)$.
And since ${\cal M}\vDash\bigwedge_{i=1}^k\lambda_i(\vec x)$ (since $\vec x^{\cal M}=\vec a$ and these are all literals satisfied by ${\cal A}^E$), and ${\cal M}\equiv_X{\cal M}'$, we have that
${\cal M}'\vDash\bigwedge_{i=1}^k\lambda_i(\vec x)$ and so ${\cal M}'\vDash\phi(\vec x)$.
And so by symmetry we have ${\cal M}\vDash\phi(\vec x)\iff{\cal M}'\vDash\phi(\vec x)$, ie. ${\cal M}\equiv{\cal M}'$.
\qed

\bcoro

    An $\forall$-theory $T$ allows quantifier elimination if and only if it is model complete.

\ecoro

This is since ${\cal A}\subseteq{\cal B}\vDash T\implies{\cal A}\vDash T$, so $(2)$ in the above theorem is satisfied if and only if $T+D{\cal A}$ is complete for all ${\cal A}\vDash T$, ie. that $T$ is
model complete.
\qed

\bthrm

    ${\sf ACF}$ allows quantifier elimination.

\ethrm

We must show that ${\sf ACF}$ is substructure complete, meaning it is the model completion of ${\sf ACF}^\forall$.
But we already showed that ${\sf ACF}^\forall$ is simply the theory of integral domains and indeed ${\sf ACF}$ is its model completion.
\qed

