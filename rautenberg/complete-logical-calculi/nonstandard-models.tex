By the completeness theorem, ${\vdash}={\vDash}$ and so immediately we get

\bthrm[title=The Finiteness Theorem, name=finitenesstheorem]

    $X\vDash\alpha$ implies $X_0\vDash\alpha$ for a finite $X_0\subseteq X$.

\ethrm

Let us define the sentence ${\tt char}_p=1+\cdots+1\eq0$ ($1$ is added $p$ times) in the language of rings ${\cal L}\set{+,\cdot,0,1}$.
A {\it field of characteristic $p$} is a field $F$ such that $F\vDash{\tt char}_p$ and $F\nvDash{\tt char}_q$ for $q<p$ (it is simple to show that if a field is of characteristic $p$, then $p$ is prime).
We define the first-order theory of fields of characteristic $0$ to be the theory axiomatized by $X$: the set of axioms for the theory of fields, as well as $\neg{\tt char}_p$ for all prime $p$.

\bprop

    A sentence $\alpha$ which is valid in all fields of characteristic $0$ is also valid in all fields of sufficiently high prime characteristic $p$ ($p$ is dependent on $\alpha$).

\eprop

A sentence $\alpha$ is valid in all fields of characteristic $0$ if and only if $X\vDash\alpha$.
By the finiteness theorem this means $X_0\vDash\alpha$ for some finite $X_0\subseteq X$.
Since $X_0$ is finite, this means it can only contain finitely many sentences of the form $\neg{\tt char}_q$, thus there exists some $p$ such that for all $\neg{\tt char}_q\in X_0$, $q<p$.
Thus for all fields $F$ of characteristic $\geq p$, we have $F\vDash X_0$ and so $F\vDash\alpha$.
Meaning $\alpha$ is valid in all fields whose characteristic is at least $p$, as required.
\qed

\bdefn

    A set of strings $Z$ is {\emphcolor decidable}\addtoindex{decidable} if there is an algorithm (this will be formally defined later) which after finitely many steps provides an answer to whether a 
    string of symbols $\xi$ belongs to $Z$.
    Otherwise $Z$ is {\emphcolor undecidable}.
    For example, it is intuitively clear that it is decidable whether $\xi$ is a formula (or a sentence).
    A theory $T$ is {\emphcolor recursively axiomatizable}\addtoindex{recursively axiomatizable} (for short {\emphcolor axiomatizable}) if it possess a decidable axiom system.
    For example, $T$ is recursively axiomatizable if it is finitely axiomatizable: if it has a finite axiom system.

\edefn

So by the above proposition, the theory of fields of characteristic zero is not finitely axiomatizable.
For if $F$ were a finite set of axioms, then their conjunction $\alpha=\bigwedge F$ would be valid in fields of sufficiently large characteristic.
But then these fields would be of characteristic zero, in contradiction.

Here is another example: an abelian group ${\cal G}$ is {\it $n$-divisible} if $G\vDash\vartheta_n$ where $\vartheta_n=\forall x\exists y\,x\eq ny$ where $ny=y+\cdots+y$ ($n$ times).
And ${\cal G}$ is {\it divisible} if it is $n$-divisible for all $n\geq1$.
The theory of divisible abelian groups is denoted ${\sf DAG}$ and is axiomatized by the axioms of abelian groups and the sentences $\vartheta_n$ for $n\geq1$.

\bprop

    A sentence $\alpha\in{\cal L}\set{+,0}$ valid in all divisible abelian groups is also valid in at least one nondivisible abelian group.

\eprop

If ${\sf DAG}\vDash\alpha$ then $X_0\vDash\alpha$ for a finite subset of the axioms $X_0$.
In particular there exists some $m$ such that for every $\vartheta_n\in X_0$, $n<m$, let $p$ be a prime larger than $m$.
Now let ${\bb Z}_p$ be the cyclic group of order $p$, then ${\bb Z}_p\vDash\vartheta_n$ for $0<n<p$ (since $n$ is coprime with $p$, and thus is divisible in ${\bb Z}_p$ thus take $y=n^{-1}x$).
Thus ${\bb Z}_p\vDash X_0$, but ${\bb Z}_p$ is not divisible, as ${\bb Z}_p\nvDash\vartheta_p$ (since $py=0$).
\qed

So ${\sf DAG}$ is not finitely axiomatizable, for the same reason that the theory of fields of characteristic zero is not finitely axiomatizable.

\bthrm[title=The Compactness Theorem, name=compactnesstheorem]

    A set of formulas $X$ is satisfiable if and only if every finite subset $X_0\subseteq X$ is satisfiable.

\ethrm

If $X$ is satisfiable, then so is every $X_0\subseteq X$.
If $X$ is not satisfiable, then by the model existence theorem this is equivalent to $X$ being inconsistent so $X\vdash\bot$ and thus $X_0\vdash\bot$ meaning $X_0\vDash\bot$ for some finite subset
$X_0\subseteq X$.
\qed

\bdefn

    An ${\cal L}$-theory $T$ is {\emphcolor complete}\addtoindex{theory}[complete] if it is consistent and has no consistent proper extension in ${\cal L}^0$.

\edefn

If $T$ is complete and if $\nvdash_T\alpha,\neg\alpha$ then $T+\alpha$ and $T+\neg\alpha$ are consistent by ${\tt C}^+$ and ${\tt C}^-$ and are therefore consistent proper extensions in contradiction.
So if $T$ is complete then $\vdash_T\alpha$ or $\vdash_T\neg\alpha$ (but not both, for then $T$ would not be consistent) for every $\alpha\in{\cal L}^0$.
Conversely if this is true then $T$ is consistent as if $\vdash_T\bot$ then $\vdash_T\alpha,\neg\alpha$ contradicting the ``not both'' part.
And it is maximally consistent as then if $\alpha\notin T$, $\neg\alpha\in T$ and so $T+\alpha\vdash\alpha,\neg\alpha$ and is therefore inconsistent.
So $T$ is complete if and only if $\vdash_T\alpha$ or $\vdash_T\neg\alpha$ but not both, or equivalently $\vdash_T\alpha\iff\nvdash_T\neg\alpha$ for all sentences $\alpha$.

So for example, if ${\cal A}$ is an ${\cal L}$-structure, then since for every sentence $\alpha$, either ${\cal A}\vDash\alpha$ or ${\cal A}\nvDash\alpha$, which is equivalent to ${\cal A}\vDash\neg\alpha$
(since $\alpha$ this is true for specific models over ${\cal A}$, and since $\alpha$ is a sentence, all the models agree on the validity of $\alpha$).
Thus for every sentence $\alpha$, ${\cal A}\vDash\alpha$ or ${\cal A}\nvDash\neg\alpha$, and it obviously cannot satisfy both.
Thus $\Th{\cal A}$ is a complete theory for every structure ${\cal A}$.

Now, a frequently occurring theory is the theory $\Th{\cal N}$, which is the theory of the structure ${\cal N}=({\bb N},0,{\tt S},+,\cdot)$ where ${\tt S}$ is the
{\it successor function}\addtoindex{successor function}: ${\tt S}\colon n\varmapsto n+1$.
${\cal N}$ is the standard structure of the arithmetical language ${\cal L}_{\it ar}\coloneqq{\cal L}\set{0,{\tt S},+,\cdot}$.
We can define the relationship $\leq$ in ${\cal L}_{\it ar}$ by $x\leq y\oto\exists z\,x+z\eq y$ and the relationship $<$ by $x<y\oto x\leq y\land x\neqb y$.

Even more frequent than $\Th{\cal N}$ is one of its subtheories: {\it Peano arithmetic}\addtoindex{Peano arithmetic} denoted ${\sf PA}$.
It is axiomatized by

\medskip
\tabskip=0pt plus 1fil
{\openup1\jot\halign to\hsize{$#$\hfil\tabskip=2cm&$#$\hfil\tabskip=2cm&$#$\hfil\tabskip=0pt plus 1fil\cr
    \forall x\,{\tt S}x\neqb0 & \forall x\,x+0\eq x & \forall x\,x\cdot x\eq0\cr
    \forall x\forall y\,({\tt S}x\eq{\tt S}y\to x\eq y) & \forall x\forall y\,x+{\tt S}y\eq{\tt S}(x+y) & \forall x\forall y\,x\cdot{\tt S}y\eq x\cdot y+x\cr
    \multispan3\hfil IS: $\phi\frac 0x\land\forall x\parens{\phi\to\phi\frac{{\tt S}x}x}\to\forall x\phi$\hfil\cr
}}
\medskip

IS is the {\it induction schema} and is a {\it schema} of rules: it runs over all formulas $\phi$ in ${\cal L}_{\it ar}$.
Since $\phi$ may not be a sentence, and axioms are, by our convention IS must be generalized, so the correct axiom is
$\bigl(\phi\frac 0x\land\forall x\parens{\phi\to\phi\frac{{\tt S}x}x}\to\forall x\phi\bigr)^g$.
The purpose of IS is to prove $\forall x\phi$ {\it by induction on $x$}: first one proves $\phi$ for when $x=0$ (hence showing $\phi\frac 0x$), and then one shows that if $\phi$ then $\phi\frac{{\tt S}}x$.
Proving $\vdash_{\sf PA}\forall x\parens{\phi\to\phi\frac{{\tt S}x}x}$ is equivalent to showing $\phi\vdash_{\sf PA}\phi\frac{{\tt S}x}x$.

For example, let $\phi=x\eq0\lor\exists v\,{\tt S}v\eq x$.
We will prove $\vdash_{\sf PA}\forall x\phi$, meaning every $x\neq0$ has a predecessor.
Obviously $\phi\frac0x\mathrel{}(=0\eq0\lor\exists v\,{\tt S}v\eq x)$.
Now we must show that $\phi\vdash_{\sf PA}\phi\frac{{\tt S}x}x$.
In general we have ${\tt S}v\eq x\vdash_{\sf PA}{\tt SS}v\eq{\tt S}x$, and so by particularization we get $\exists v\,{\tt S}v\eq x\vdash_{\sf PA}\exists v\,{\tt SS}v\eq{\tt S}x$.
And since $x\eq0\vdash_{\sf PA}\exists v\,{\tt S}v\eq{\tt S}x$, we get $\phi\vdash_{\sf PA}\exists v\,{\tt S}v\eq{\tt S}x\vdash_{\sf PA}{\tt S}x\eq0\lor\exists v\,{\tt S}v\eq{\tt S}x=\phi\frac{{\tt S}x}x$ as
required.

We will now show that $\Th{\cal N}$ and ${\sf PA}$ have what are called {\it nonstandard models}\addtoindex{nonstandard}[models], models which are not isomorphic to the standard model of the theory, which
is ${\cal N}$ in this case.
Let us define $\underline n\coloneqq{\tt S}^n0\coloneqq{\tt S}\cdots{\tt S}0$ where ${\tt S}$ is composed with itself $n$ times.
So for example $\underline1={\tt S}0$, $\underline2={\tt S}1$ and in general $\underline{{\tt S}n}={\tt S}\underline n$.
Let $x\in\Var$ and we define $X\coloneqq\Th{\cal N}\cup\set{\underline n<x}[n\in{\bb N}]$.

Let $X_0\subseteq X$ be finite, then $X_0$ is satisfiable: since $X_0$ is finite, there must be some $m$ such that $X_0\subseteq\Th{\cal N}\cup\set{\underline n<x}[n<m]$.
Thus, we can simply assign $x$ the value of $m$ and then $({\cal N},m)$ (meaning the valuation function maps $x$ to $m$) is a model of $X_0$.
So by \refmath{compactnesstheorem}, $X$ has a model $({\cal N}',c)$ where the domain of the model is ${\bb N}'$ and $c\in{\bb N}'$ is the valuation of $x$.
Since $\Th{\cal N}$ is a subset of $X$ and is complete, ${\bb N}'$ satisfies precisely the same sentences as ${\cal N}$.
In particular the following sentences are valid in ${\cal N}'$: ${\tt S}\underline n\eq\underline{{\tt S}n}$, $\underline{n+m}=\underline n+\underline m$, and
$\underline{n\cdot m}=\underline n\cdot\underline m$.
Thus $n\varmapsto\underline n^{\cal N'}$ is an embedding of ${\cal N}$ into ${\cal N}'$.
We can identify the image of this embedding with ${\cal N}$, meaning it is legitimate to assume $\underline n^{\cal N'}=n$ so ${\cal N}\subseteq{\cal N}'$.

Since ${\cal N}'\vDash X$, ${\cal N}$ and ${\cal N}'$ are elementarily equivalent as $\Th{\cal N}$ is complete.
But on the other hand, $n<a$ for any $a\in{\bb N}'\setminus{\bb N}$, since in ${\cal N}$ and thus in ${\cal N}'$ we have $(\forall x\leq\underline n)\bigvee_{i\leq\underline n}x\eq\underline i$.
So if $a\leq n$ then we'd have $a\eq\underline i$ and so $a\in{\bb N}$ in contradiction.
And since ${\bb N}'\setminus{\bb N}$ is nonempty by the formulas added to $X$, ${\bb N}$ is a proper initial segment of ${\bb N}'$.
The elements of ${\bb N}'\setminus{\bb N}$ are called {\it nonstandard numbers}\addtoindex{nonstandard}[numbers] (the existence of nonstandard numbers are precisely what make ${\bb N}'$ and ${\bb N}$
non-isomorphic).
$c$ is an example of a nonstandard number, as well as $c+c$ and so on.

Since $(\forall x\neqb0)\exists!y\,{\tt S}y\eq x$ is a theorem of ${\cal N}$ it too must be valid in ${\cal N}'$ and so $c$ must have an immediate successor.
But if we were to chase the successors of $c$, we would never find one which is in ${\bb N}$ (ie. there is no natural $x$ and $n$ such that ${\tt S}^nx=c$ as then $c$ would be natural).

In no nonstandard model ${\cal N}'$ of $\Th{\cal N}$ is ${\bb N}$ definable (for the same reason as above, ${\bb N}$ can be embedded in all models of $\Th{\cal N}$, so we really mean that the image of this
embedding is not definable).
In fact it is not even {\it parameter definable}, meaning there is no $\alpha=\alpha(x,\vec y)$ and $b_1,\dots,b_n\in{\bb N}'$ such that ${\bb N}=\set{a\in{\bb N}'}[{\cal N}'\vDash\alpha{[a,\vec b]}]$.
Otherwise, we'd have ${\cal N}'\vDash\alpha\frac0x[\vec b]$ since $0\in{\bb N}$ and ${\cal N}'\vDash\forall x\parens{\alpha\to\alpha\frac{{\tt S}x}x}{}[\vec b]$ since ${\bb N}$ is closed under ${\tt S}$.
Thus by IS we have ${\cal N}'\vDash\forall x\alpha[\vec b]$, which contradicts ${\bb N}'\setminus{\bb N}$ being nonempty.

Similar to ${\cal N}$ we can find nonstandard models for the theory of real numbers, ${\cal R}=({\bb R},+,\cdot,<,\set{\boldsymbol a}[a\in{\bb R}])$ which contains a constant symbol $\boldsymbol a$ for every
number $a\in{\bb R}$.
Consider $X=\Th{\cal R}\cup\set{\boldsymbol a<x}[a\in{\bb R}]$, which is finitely satisfiable and therefore by \refmath{compactnesstheorem} $X$ has a model ${\cal R}^*$ which is a {\it nonstandard
model of analysis}.
${\cal R}^*$ models $\Th{\cal R}$ and therefore every theorem valid in ${\cal R}$ is valid in ${\cal R}^*$ and vice versa.
We could also add functions like $\exp,\ln,\sin,\cos$ to the signature of ${\cal R}$ and obtain a model ${\cal R}^*$ where the properties of these functions which can be formulated in first-order language
are preserved, like
$$ \forall x,y\,\exp(x+y)=\exp x\cdot\exp y,\qquad (\forall x>0)\exp\ln x\eq x,\qquad \forall x\,\sin^2x+\cos^2x\eq1 $$
Since continuity and differentiability can be formulated in first-order language (via $\epsilon$-$\delta$ shenanigans), these functions remain continuous and repeatedly differentiable.
Though topological properties like Bolzano-Weirestrass are not generally true, but can be replaced by infinitesimal arguments.

For a nonstandard model ${\cal R}^*$ of $\Th{\cal R}$, with ${\cal R}\subseteq{\cal R}^*$ contains infinitely large numbers (meaning numbers $c$ where $r<c$ for all $r\in{\bb R}$), but also infinitely
small numbers (meaning numbers $\epsilon$ where $0<\epsilon<r$ for all $r\in{\bb R}$, as we can take $\epsilon\coloneqq\frac1c$).
These infinitely small numbers are termed {\it infinitesimal numbers}, and utilizing their properties one can study {\it nonstandard analysis} which is a branch of analysis utilizing these nonstandard
models of analysis and their infinitesimal numbers.
This branch gives arguably more intuitive proofs for analytic results, which can more closely follow the arguments of the original analytical researchers like Leibniz.

While every bounded set in ${\cal R}$ possesses a supremum, this is not true in ${\cal R}^*$.
Rather we can only say that every bounded definable set in ${\cal R}^*$ possesses a supremum, as this is first-order formulatable,
$$ \exists x\phi\land\exists y\forall x(\phi\to x\leq y)\to\exists z\forall x\bigl((\phi\to x\leq z)\land\forall y((\phi\to x\leq y)\to z\leq y)\bigr) $$
(the set is defined by $\phi$, ie. it is the set $\set{x}[{\cal R}^*\vDash\phi]$.)

\bexerc

    Prove in ${\sf PA}$ the associativity, commutativity, and distributivity of $+$ and $\cdot$.

\eexerc

We will first derive ${\tt S}x+y\eq x+{\tt S}y$ by induction on $y$, ie. we define $\phi={\tt S}x+y\eq x+{\tt S}y$.
The base case $\phi\frac0y={\tt S}x+0\eq x+{\tt S}0$: since ${\tt S}x+0\eq{\tt S}x$ and $x+{\tt S}0\eq{\tt S}(x+0)\eq{\tt S}x$, this is true.
Now we show $\phi\vdash\phi\frac{{\tt S}y}y={\tt S}x+{\tt S}y\eq x+{\tt SS}y$.
Now, ${\tt S}x+{\tt S}y\eq{\tt S}({\tt S}x+y)\eq{\tt S}(x+{\tt S}y)\eq x+{\tt SS}y$ as required.
So we have shown $\forall y\phi$ as required.

Now, we will also show that $0+y\eq y$ by induction on $y$: let $\phi=0+y\eq y$.
Then $\phi\frac0y=0+0\eq0$ which is valid in ${\sf PA}$.
Now we must show $0+y\eq y\vdash0+{\tt S}y\eq{\tt S}y$.
Since $0+{\tt S}y\eq{\tt S}(0+y)\eq{\tt S}y$ we have finished.

Now we will show associativity, by inducting $\phi=(x+y)+z\eq x+(y+z)$ on $y$.
$\phi\frac0y=(x+0)+z\eq x+(0+z)$ and since $x+0\eq x$ and $0+z\eq z$ this is true.
For the inductive step, $(x+{\tt S}y)+z\eq {\tt S}(x+y)+z\eq {\tt S}((x+y)+z)\eq {\tt S}(x+(y+z))\eq x+{\tt S}(y+z)\eq x+(y+{\tt S}z)\eq x+({\tt S}y+z)$ as required.

For commutativity, we induct $x+y\eq y+x$ on $y$ (we could also on $x$, but why change?).
This is obviously true when substituting $0$ for $y$, as $x+0\eq x\eq x+0$.
Now $x+{\tt S}y\eq{\tt S}(x+y)\eq{\tt S}(y+x)\eq{\tt S}y+x$, as required.
Proofs for $\cdot$ and distributivity are similar.

\bexerc

    Prove the antisymmetry of $\leq$ in ${\sf PA}$.

\eexerc

So we must show that $x\leq y$ and $y\leq x$ implies $x\eq y$.
Now,
$$ x\leq y\oto\exists z\,x+z\eq y,\qquad y\leq x\oto\exists u\,y+u\eq x $$
This means that $x\leq y\land y\leq x\to\exists u,z\,y+u+z\eq y$.
Now simple induction on $x$ will yield $y+x\eq y\to x\eq0$, and so $u+z\eq0$.
And induction on $z$ will yield $u+z\eq0\to z\eq0,u\eq0$ (the base case is trivial, the inductive step holds vacuously).
And so $u+z\eq0$ and thus $y\eq x$.

\bexerc

    Prove $x<y\equiv_{\sf PA}{\tt S}x\leq y$.
    Use this to prove $\vdash_{\sf PA}x\leq y\lor y\leq x$ by induction on $x$.

\eexerc

We will first prove $x<y\vdash_{\sf PA}{\tt S}x\leq y$ 
By definition if $x<y$ then $\exists z\,x+z\eq y$ and since $x\neqb y$ we have that $z\neqb0$ and so $z$ has a predecessor, $u$.
Thus $x<y$ implies $\exists u\,x+{\tt S}u\eq y$ and since $x+{\tt S}u\eq{\tt S}x+u$ this means ${\tt S}\leq y$ as required.
Conversely, since $x<{\tt S}x$ by induction on $x$, we have that if ${\tt S}x\leq y$ by transitivity $x<y$.

Let $\phi=x\leq y\lor y\leq x$, then substituting $0$ for $x$, we get $0\leq y\lor y\leq0$ and since we know that $\forall y\,0\leq y$ (by induction on $y$), this is true.
Now for the inductive step we must show ${\tt S}x\leq y\lor y\leq{\tt S}x\equiv x<y\lor y\leq{\tt S}x$.
We know that $x\leq y\lor y\leq x$, so either $x<y$ in which case this is true, or $y\leq x$ in which case $y\leq{\tt S}x$.

\bexerc

    Let $\alpha,\beta,\gamma\in{\cal L}_{\it ar}$ and $y\notin\var\set{\alpha,\beta}$ and $z\notin\var\gamma$.
    Verify the following:
    \benum
        \item $\vdash_{\sf PA}\forall x\parens{(\forall y<x)\alpha\frac yx\to\alpha}\to\forall x\alpha$, the {\it schema of $<$-induction} (or {\it strong induction}).
        \item $\vdash_{\sf PA}\exists x\beta\to\exists x\parens{\beta\land(\forall y<x)\neg\beta\frac yx}$, the {\it minimum schema} (or {\it well-ordering principle}).
        \item $\vdash_{\sf PA}(\forall x<v)\exists y\gamma\to\exists z(\forall x<v)(\exists y<z)\gamma$, the {\it schema of collection}.
    \eenum

\eexerc

\benum
    \item Let $\phi=(\forall y<x)\alpha\frac yx$, we will prove $\forall x(\phi\to\alpha)\vdash_{\sf PA}\phi\frac0x,\,\phi\to\phi\frac{{\tt S}x}x$ which proves $\forall x\phi$ by IS.
    This means that in particular $\forall x\phi\frac{{\tt S}x}x$, and so for every $y<{\tt S}x$, $\alpha\frac yx$ and since $x<{\tt S}x$ we get $\alpha$ as required.
    Since $\phi\frac0x=(\forall y<0)\alpha\frac yx$ which holds vacuously.
    And since $\forall x(\phi\to\alpha)=\forall x\parens{(\forall y<x)\alpha\frac yx\to\alpha}$, we have that if $\phi=(\forall y<x)\alpha\frac yx$ then $\alpha$, meaning $(\forall y\leq x)\alpha\frac yx$,
    which is equivalent to $\phi\frac{{\tt S}x}x$.
    Thus $\forall x(\phi\to\alpha)\vdash_{\sf PA}\phi\to\phi\frac{{\tt S}x}x$, as required.
    \item Note that the contrapositive of this is $\forall x\parens{(\forall y<x)\neg\beta\frac yx\to\neg\beta}\to\forall x(\neg\beta)$, which follows from $(1)$.
    \item Let $\phi=(\forall x<v)\exists y\gamma\to\exists z(\forall x<v)(\exists y<z)\gamma$, we will prove $\forall v\phi$ by induction on $v$.
    $\vdash_{\sf PA}\phi\frac 0v$ holds vacuously.
    Now we must prove $\phi\vdash_{\sf PA}\phi\frac{{\tt S}v}v\equiv_{\sf PA}(\forall x\leq v)\exists y\gamma\to\exists z(\forall x\leq v)(\exists y<z)\gamma$.
    By $\phi$, if $(\forall x\leq v)\exists y\gamma$, then we have $\exists z(\forall x<v)(\exists y<z)\gamma$.
    So the only ``issue'' is when $x=v$, but in which case we know there exists a $y_x$ such that $\gamma$, so let $z'=\maxof{z,{\tt S}y_x}$ and we get that for every $x<v$ there exists a $y<z\leq z'$ such
    that $\gamma$ and for $x=v$ there exists a $y_x<{\tt S}y_x\leq z'$ such that $\gamma$.
    And so we get $\exists z(\forall x<v)(\exists y<z)\gamma$ (where the $z$ is $z'$) as required.
\eenum

