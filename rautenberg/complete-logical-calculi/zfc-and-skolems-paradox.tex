We briefly review fundamentals of set theory.
Recall that a set $M$ is {\it countable}\addtoindex{countable set} if there exists a surjection $f\colon{\bb N}\longto M$ (assuming the axiom of choice, this is equivalent to there being an injection
$M\longto{\bb N}$).
A set which is not countable is called {\it uncountable}.
Two sets $M$ and $N$ are {\it equipotent}\addtoindex{equipotency} if there exists a bijection between $M$ and $N$, we denote this as $M\sim N$.
If $M\sim{\bb N}$, $M$ is {\it countably infinite}.
Recall that ${\bb R}$ is uncountable and equipotent to $\powsetof{\bb N}$.
In general due to Cantor's theorem which states that the powerset of a set, $\powsetof M$, is of a higher cardinality than $M$ (meaning there is an injection $M\longto\powsetof M$ but no surjection).

And if $M$ and $N$ are countable, than so is $M\times N$.
And if $\set{M_i}_{i\in I}$ is a countable family of countable sets (meaning $I$ is countable and $M_i$ is countable for every $i\in I$), then their union $\bigcup_{i\in I}M_i$ is also countable.
Notice then that if $M$ is countable, the set of all finite sequences of symbols in $M$ is $\bigcup_{n\in{\bb N}}M^n$, which is countable.
Thus if ${\cal L}$ is a first-order language over a countable signature then ${\cal L}$ is countable as well.

A {\it countable theory} is a theory formalized in a countable first-order language.

\bthrm[title=L\"owenheim-Skolem Theorem, name=lowenheimskolem]

    A countable consistent theory $T$ has a countable model.

\ethrm

By \refmath{modelexistencetheorem}, $T$ has a model whose domain $A$ consists of the equivalence classes $\overline c$ for $c\in C$, where $C=\bigcup_{n\in{\bb N}}C_n$ is a new set of constants.
By definition, $C_0$ is equipotent to ${\cal L}\times\Var$ which is countable, and so ${\cal L}_1\coloneqq{\cal L}C_0$ is equipotent.
And in general, $C_n$ is equipotent to ${\cal L}_n\times\Var$ so $C_n$ and ${\cal L}_{n+1}$ are countable for every $n$, by induction.
Thus $C$ is also countable as the countable union of countable sets.
The map $c\varmapsto\overline c$ is a surjection from $C$ to $A$, and therefore $A$ is also countable, as required.
\qed

This theorem will later be generalized, but this theorem already leads to interesting consequences.
For example, there must be countable ordered fields which are nonstandard models of the first-order theory $\Th({\bb R},0,1,+,\cdot,<,\exp,\sin,\dots)$.
Even more surprising (for reasons explained shortly) is that this theorem implies the existence of countable models of set theory.

To explain why, we must first define set theory.
The most common and thus most important formalization of set theory is ${\sf ZFC}$ (short for Zermelo-Fraenkel+Choice, after the two mathematicians which first formalized it as well as the axiom of choice
${\sf AC}$.
Excluding ${\sf AC}$ from the axioms of the theory, gives rise to a proper (this is not a trivial fact) subtheory, ${\sf ZF}$.).

The language upon which ${\sf ZFC}$ is defined is ${\cal L}\set{\inb}$, it contains only the extralogical symbols $\inb$ (and of course $\eq$).
We denote this set-theoretic language by ${\cal L}_{\inb}$.
We use a boldface $\inb$ symbol in order to differentiate it from the metalogical $\in$ symbol, much like $\eq$.
Importantly, in ${\sf ZFC}$ all objects are sets, there are no other types of objects like in other formulations of set theory.

We define the relation $\subseteqb$ by $x\subseteqb y\oto\forall z(z\inb x\to z\inb y)$.

We begin to list the axioms of ${\sf ZFC}$ (recall that these axioms must be generalized):

\medskip
\tabskip=0pt plus 1fil
{\openup1\jot\halign to\hsize{${\sf#}:$\hfil\tabskip=.5cm&$#$\hfil\tabskip=.5cm&(#)\hfil\tabskip=0pt plus 1fil\cr
    AE & \forall z(x\inb x\oto z\inb y)\to x\eq y & axiom of extensionality\cr
    AS & \exists y\forall z(z\inb y\oto\phi\land z\inb x) & axiom of separation\cr
}}
\medskip

${\sf AS}$ is an axiom schema: $\phi$ runs over all ${\cal L}_{\inb}$-formulas with $y\notin\free\phi$.
Now, let $\phi=\phi(x,z,\vec a)$, then we can derive $\forall x\exists!y\forall z(z\inb y\oto\phi\land z\inb x)$, meaning that $y\eq\set{z\inb x}[\phi]\oto\forall z(z\inb y\oto\phi\land z\inb x)$ is a
legitimate definition of the {\it set term} $\set{z\inb x}[\phi]$ (the set of all $z\in x$ such that $\phi(x,z,\vec a)$ holds).
The set term $\set{z\inb x}[\phi]$ is simply a suggestive way of writing a function term $f_{\vec a}x$ (which depends on the ``parameter'' term $\vec a$).
To derive the required formula for legitimacy, let $y,y'\notin\free\phi$ then it is obvious to see
$$ (x\inb y\oto\phi\land z\inb x)\land(z\inb y'\oto\phi\land z\inb x)\to(z\inb y\oto z\inb y') $$
This implies that $\forall z(x\inb y\oto\phi\land z\inb x)\land\forall z(z\inb y'\oto\phi\land z\inb x)\to y\eq y'$, and so we have proven the claim of legitimacy.

We can explicitly define the empty set by $y\eq\varnothing\oto\forall z\,z\notinb y$.
But we must show that this is legitimate.
According to ${\sf AS}$, $\exists y\forall z(z\inb y\oto z\notinb x\land z\inb x)$.
But since $z\inb y\oto z\inb x\land z\inb x\equiv z\notinb y$, so this is equivalent to $\exists y\forall z(z\notinb y)$ (proving the existence part of legitimacy; we must still prove uniqueness).
Now by ${\sf AE}$, $\forall z\,z\notinb y\land\forall z\,z\notinb y'\to y\eq y'$, and since we just showed $\exists y\forall z(z\notinb y)$ gives the required $\exists!y\forall z(z\notinb y)$.

We also have
$$ {\sf AU}:\quad \forall x\exists y\forall z(z\inb y\oto(\exists u\inb x)z\inb u)\quad (\hbox{axiom of union}) $$
$y$ here is the {\it union} of $x$, $z\inb y$ if and only if there exists a set $u\inb x$ such that $z\inb u$.
By ${\sf AE}$, the $\exists y$ may be replaced with $\exists!y$.
So we can define an operator (we use the word operator instead of function as the word function has a special meaning in ${\sf ZFC}$), the union operator $x\varmapsto\bigcup x$.
In view of ${\sf AS}$, the axiom of union could be weakened to $\forall x\exists y\forall z((\exists u\inb x)z\inb u\to z\inb y)$ as then we can use ${\sf AS}$ to get the union from this set $y$.
Similarly the following axiom could be weakened:
$$ {\sf AP}:\quad \forall x\exists y\forall z(z\inb y\oto z\subseteqb x)\quad (\hbox{power set axiom}) $$
And again, $\exists y$ can be replaced with $\exists!y$ due to ${\sf AE}$.
We define the {\it powerset} of a set $x$ to be this unique $y$, which we denote ${\cal P}x$, which contains all the subsets of $x$.
Now, since $x\subseteq\varnothing\oto\forall y\inb x\,y\inb\varnothing$, and $y\inb\varnothing\equiv_{\sf ZFC}\bot$, we have that $x\subseteq\varnothing\oto\bot$, so
$\forall x(x\in{\cal P}\varnothing\oto x\eq\varnothing)$.
Similarly $\forall x(x\in{\cal PP}\varnothing\oto x\eq\varnothing\lor x\eq{\cal P}\varnothing)$.
Since ${\cal P}\varnothing\neqb\varnothing$ as it is not empty, ${\cal PP}\varnothing$ has precisely two elements.

The following axiom was added by Fraenkel to the axioms formulated by Zermelo:
$$ {\sf AR}:\quad \forall x\exists!y\phi\to\forall u\exists v\forall y(y\inb v\oto(\exists x\in u)\phi)\quad (\hbox{axiom of replacement}) $$
Here $\phi=\phi(x,y,\vec a)$ is viewed as a function which maps $x$ to $y$, where $y$ is the unique value satisfying $\phi$ (hence the $\forall x\exists!y\phi$).
If $\phi$ does represent such a function, ${\sf AR}$ says that for every $u$ we can define the image of $u$ under this function.
More suggestively, we can define an operator $x\varmapsto Fx$ where $y\eq Fx\oto\phi$ (since $F$ actually is dependent on $\vec a$, we should instead write $F_{\vec a}$ for $F$).
Then ${\sf AR}$ states that for every $u$ we can define its image under $u$ which is denoted $\set{Fx}[x\in u]$.

For example, let us define $\phi=\phi(x,y,a,b)\coloneqq x\eq\varnothing\land y\eq a\lor x\neqb\varnothing\land y\eq b$.
$\forall x\exists!y\phi$ is obvious, and so this defines an operator $F=F_{a,b}$ which satisfies $F\varnothing\eq a$ and $Fx\eq b$ for all $x\neqb\varnothing$ (and such an $x$ exists, for example
${\cal P}\varnothing$).
We then define
$$ \set{a,b} \coloneqq \set{F_{a,b}x}[x\in{\cal PP}\varnothing] $$
(since ${\cal PP}\varnothing$ contains ${\cal P}\varnothing$ and $\varnothing$, two distinct elements.)
This is called the {\it pair set} of $a,b$.
Then we can define the union of two sets (which would not necessarily exist, as we have only defined the union of a set) to be $a\cupb b\coloneqq\bigcup\set{a,b}$.
The intersection of two sets already exists by ${\sf AS}$: $a\capb b\coloneqq\set{z\in a}[z\in b]$.
But we can also define the intersection of a set as $\bigcap x\coloneqq\set{z\in\bigcup x}[(\forall y\in x)z\in y]$.
Notice that $\bigcap\set{a,b}=\set{z\in a\cup b}[z\in a\land a\in b]$ which by ${\sf AE}$ is equal to $a\capb b$.
We further define $\set a\coloneqq\set{a,a}$ and inductively $\set{a_1,\dots,a_{n+1}}\coloneqq\set{a_1,\dots,a_n}\cup\set{a_{n+1}}$.
So for example, we can prove ${\cal P}\varnothing=\set{\varnothing},\,{\cal PP}\varnothing=\set{\varnothing,\set\varnothing}$.
We define the {\it ordered pair} of $a,b$ to be $(a,b)\coloneqq\set{\set a,\set{a,b}}$.
This definition is useful as it has the basic property $(a,b)\eq(c,d)\oto a\eq c\land b\eq d$.

So far our theory has many features, but it lacks an important one: there does not necessarily exist an infinite set!
Let us define the successor operator, defined by ${\tt S}\colon x\varmapsto x\cup\set x$.
So for example ${\tt S}\varnothing=\set\varnothing,\,{\tt SS}\varnothing=\set{\varnothing,\set\varnothing},\dots$.
Notice that ${\tt S}x$ contains precisely one more element than $x$ (assuming that $x\notin x$, which requires another axiom which will be stated next).
So there is a natural correspondence between this successor operator and the one from ${\sf PA}$ (in fact, in some definitions of ${\bb N}$, these two operators are one and the same).
Now we state the following axiom which asserts the existence of an infinite set
$$ {\sf AI}:\quad \exists u(\varnothing\in u\land\forall(x\in u){\tt S}x\in u)\quad (\hbox{axiom of infinity}) $$
This set $u$ is a so-called {\it inductive set}: it is closed under successors and contains $\varnothing$.

We also have
$$ {\sf AF}:\quad (\forall x\neqb\varnothing)(\exists y\inb x)x\capb y\eq\varnothing\quad (\hbox{axiom of foundation}) $$
This means that every nonempty set $x$ has a $\inb$-minimal element: an element $y$ such that no other $y'\inb x$ is in $y$.
The axiom of foundation implies that there cannot be a circular chain of inclusion, ie. $x_0\inb\cdots\inb x_n\inb x_0$, as then $y=\set{x_0,\dots,x_n}$ would have no $\inb$-minimal element.

These previously mentioned axioms axiomatize the theory ${\sf ZF}$.
If we adjoin to ${\sf ZF}$ the {\it axiom of choice}, we get the theory ${\sf ZFC}$.
$$ {\sf AC}:\quad \forall u\bigl(\varnothing\notinb u\land(\forall x\inb u)(\forall y\inb u)(x\neqb y\to x\capb y\eq\varnothing)\to\exists z(\forall x\inb u)\exists!y(y\inb x\capb z)\bigr)\quad
(\hbox{axiom of choice}) $$
The axiom of choice states that if $u$ is a set of nonempty disjoint sets, then there exists a {\it choice set} $z$ which contains precisely one element from each $x\inb u$.
Equivalent to ${\sf AC}$ (among many other important equivalences) is that $\prod_{i\in I}A_i\eq\varnothing$ for any index set $I$.

${\sf ZFC}$ is often viewed as the purest of all first-order theories, as it is sophisticated and thought to be able to formalize (almost) all proofs in mathematics (the {\it almost} here is technical:
${\sf ZFC}$ is not complete, and so some mathematicians may study results independent of it.
One such example is the {\it continuum hypothesis}).
But whether or not this is true is not of too much significance, as most mathematics does not care whether or not it can be formalized in ${\sf ZFC}$.

If ${\sf ZFC}$ is consistent (which no one really doubts, although we cannot prove), then by the \refmath{lowenheimskolem}, it must contain a countable model ${\cal V}=(V,\inb^{\cal V})$.
But certainly the existence of an uncountable set is provable within ${\sf ZFC}$: for every inductive set $u$, ${\cal P}u$ is uncountable.
But at the same time, $({\cal P}u)^{\cal V}\subseteq V$ and so ${\cal P}u$ must be countable.
The fallacy here is that our notion of {\it countable} ``outside'' of ${\cal V}$ is different from the notion ``within'' ${\cal V}$.
This ``paradox'' is known as {\it Skolem's paradox}.

The explanation of Skolem's paradox is that the countable model ${\cal V}$ contains fewer sets and functions than expected, and in particular there do not exist enough to satisfy a bijection between
$u^{\cal V}$ and $({\cal P}u)^{\cal V}$ although ``outside'' of ${\cal V}$ such a bijection exists.

Another ``paradox'' is that while $V$ is by definition a set, but $\vdash_{\sf ZFC}\neg\exists u\forall z\,z\inb u$, ie. there does not exist a universal set.
This is derived as follows: $\exists u\forall z\,z\inb u$ implies, using ${\sf AE}$ and ${\sf AS}$, the existence of a Russelian set $v=\set{x\inb u}[x\notinb x]$.
Thus $\exists u\forall z\,z\inb u\vdash_{\sf ZFC}\exists u\forall x(x\inb u\oto x\notinb x)$.
Though we have previously shown that the right-hand formula is logically invalid, and thus equivalent to $\bot$.
So $\exists u\forall z\,z\inb u\vdash_{\sf ZFC}\bot$ and so $\vdash_{\sf ZFC}\neg\exists u\forall z\,z\inb u$.
So similar to how there is no absolute concept of countability, there is no absolute concept of a set: a set ``outside'' a model of ${\sf ZFC}$ is not necessarily a set within ${\sf ZFC}$.

\bexerc

    Let $T$ be an elementary theory with arbitrarily large finite models.
    Prove that $T$ also has an infinite model.

\eexerc

Since $T$ has arbitrarily large finite models, $\exists_n$ does not contradict $T$ for every $n$ (since $T+\exists_n$ is not unsatisfiable).
Thus let $X=\set{\exists_n}[n\in{\bb N}]$, then $T+X$ is satisfiable: let $X_0\subseteq T\cup X$ be finite then $X_0\subseteq T\cup\set{\exists_n}[n\leq m]$ for some $m$.
Then $X_0$ is satisfiable as $T+\exists_m$ is satisfiable, and therefore so is $T\cup\set{\exists_n}[n\leq m]$ (as $\exists_m\to\exists_n$ is a tautology) and thus so is $X_0$.
So every finite subset of $T\cup X$ is satisfiable, and so by \refmath{compactnesstheorem}, so is $T\cup X$.

Now, a model of $T+X$ must satisfy $\exists_n$ for every $n$, and so it must have at least $n$ elements for every $n$.
So it cannot be finite, and thus must be infinite.
Furthermore, by definition a model of $T+X$ models $T$ as well.

\bexerc

    Suppose ${\cal A}=(A,<)$ is an infinite well-ordered set.
    Show that there is a not well-ordered set elementarily equivalent to ${\cal A}$.
    Thus being well-ordered is not a first-order property (at least in the language ${\cal L}\set<$).

\eexerc

Let us define $X=\Th{\cal A}\cup\set{\v_{n+1}<\v_n}[n\in{\bb N}]$.
$X$ is finitely satisfiable (as finite subsets of $X$ can be modeled over ${\cal A}$), but every model of $X$ has an infinitely descending chain (and such a chain therefore has no minimum element) and
therefore models of $X$ are not well-ordered.

\bexerc

    Prove that a consistent theory $T$ is equal to the intersection of all its complete extensions.
    Meaning $T=\bigcap\set{T'\supseteq T}[T'\hbox{ complete}]$.

\eexerc

Firstly, $T$ does have a complete extension as there exists a ${\cal A}\vDash T$ and $\Th{\cal A}$ is a complete extension of $T$.
Obviously then $T\subseteq\bigcap\set{T'\supseteq T}[T'\hbox{ complete}]$.
Now suppose $\phi\in T'$ for every complete extension of $T$, then if $\phi\notin T$ then $\neg\phi$ doesn't contradict $T$ since $T,\neg\phi\nvdash\bot$ (in general if $X\nvdash\phi$ then by ${\tt C}^+$,
$X,\neg\phi\nvdash\bot$).
But then $T,\neg\phi$ is consistent and therefore has a complete extension, which contradicts $\phi$ being in every complete extension of $T$.

\bexerc

    Derive ${\sf AS}$ from ${\sf AR}$.

\eexerc

Let $\phi=\phi(x,z,\vec a)$ be a formula and let $y,b\notin\free\phi$.
$$ \psi = \Bigl(\phi\tfrac bz\land\bigl((y\eq z\land\phi)\lor(y\eq b\land\neg\phi)\bigr)\Bigr)\lor\Bigl(\neg\phi\tfrac bx\land y\eq\varnothing\Bigr) $$
Then $\forall z\exists!y\psi$, as either $\phi$ or $\neg\phi$ is true.
This operator defined by $\psi$ maps $z$ to $z$ if $\phi$ and $z$ to $b$ otherwise.
If $\set{z\in x}[\phi]$ is nonempty then there exists a $b$ such that $\phi\tfrac bz$ and for this $b$, the operator defined by $\psi$ when parameterized with $b$ and $x$ gives this set.
If $\phi$ is invalid, then $\set{z\in x}[\phi]$ is the empty set which exists by definition.

