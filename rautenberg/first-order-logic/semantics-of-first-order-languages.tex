Similar to how in propositional logic we defined models to give meaning to propositional formulas, we do the same for first-order logic.

\bdefn

    Suppose ${\cal L}$ is a first-order language, then an {\emphcolor ${\cal L}$-model}\addtoindex{model} (or an {\emphcolor ${\cal L}$-interpretation}) is a pair ${\cal M}=({\cal A},w)$ where ${\cal A}$ is
    an ${\cal L}$-structure and $w$ is a {\emphcolor valuation function}\addtoindex{valuation}, $w\colon\Var\longto A,\,x\varmapsto x^w$.
    We denote $f^{\cal A}$, $r^{\cal A}$, $c^{\cal A}$, and $x^w$ also by $f^{\cal M}$, $r^{\cal M}$, $c^{\cal M}$, and $x^{\cal M}$ respectively.

\edefn

We can extend valuations to ${\cal T}$ in an obvious manner:
$$ c^{\cal M} = c\hbox{ for constant symbols $c$},\qquad (ft_1\cdots t_n)^{\cal M} = f^{\cal M}t_1^{\cal M}\cdots t_n^{\cal M} $$
In place of $t^{\cal M}$ we may write $t^{{\cal A},w}$ or simply $t^w$ if the structure is understood.
But we will usually stick with $t^{\cal M}$.

Notice that the valuation of a term $t$ depends only on the valuation of the variables and extralogical symbols occurring in $t$:

\bprop[name=termcoincidenceprop]

    Suppose $t$ is an ${\cal L}$-term, and ${\cal M}$ and ${\cal M}'$ are two ${\cal L}$-models.
    Let $V$ be a set of variables where $\var t\subseteq V$.
    Now suppose that ${\cal M}$ and ${\cal M}'$ agree on their valuations of $V$ and extralogical symbols in $t$: for every $x\in V$, $x^{\cal M}=x^{{\cal M}'}$ and for every extralogical symbol ${\sf s}$
    occuring in $t$, ${\sf s}^{\cal M}={\sf s}^{{\cal M}'}$.
    Then $t^{\cal M}=t^{{\cal M}'}$.

\eprop

This is done by term induction.
If $t$ is a prime term, then $t=c$ for some constant or $t=x$ for some variable.
In either case the proposition is satisfied by its assumption (that ${\cal M}$ and ${\cal M}'$ agree on variables and extralogical symbols occurring in $t$).
Now suppose $t=ft_1\cdots t_n$ then by the assumption of the proposition, $f^{\cal M}=f^{{\cal M}'}$ and by the induction hypothesis $t_i^{\cal M}=t_i^{{\cal M}'}$.
Thus
$$ t^{\cal M} = f^{\cal M}t_1^{\cal M}\cdots t_n^{\cal M} = f^{{\cal M}'}t_1^{{\cal M}'}\cdots t_n^{{\cal M}'} = t^{{\cal M}'} $$
as required.
\qed

\bdefn

    We now define the {\emphcolor satisfiability relation}\addtoindex{satisfiability relation} for first-order models.
    Let ${\cal M}=({\cal A},w)$ be a model.
    For every $a\in{\cal A}$ and $x\in\Var$ let us define the model ${\cal M}^a_x=({\cal A},w')$ where $y^{w'}=y^w$ for variables $y$ distinct from $x$, and $x^{w'}=a$.
    Meaning
    $$ y^{{\cal M}^a_x} = \cases{a & $y=x$\cr y^w & else} $$
    So now we define the satisfiability relation $\vDash$ recursively as follows:

    \medskip
    \tabskip=0pt plus 1fil
    \halign to\hsize{$#$\hfil\tabskip=0pt&${}\iff#$\hfil\tabskip=1cm&$#$\hfil\tabskip=0pt&${}\iff#$\hfil\tabskip=0pt plus 1fil\cr
        {\cal M}\vDash s\eq t & s^{\cal M}=t^{\cal M}, & {\cal M}\vDash r\vec t & r^{\cal M}\vec t^{\cal M},\cr
        {\cal M}\vDash (\alpha\land\beta) & {\cal M}\vDash\alpha\hbox{ and }{\cal M}\vDash\beta, & {\cal M}\vDash\neg\alpha & {\cal M}\nvDash\alpha,\cr
        \multispan4\hfil${\cal M}\vDash \forall x\alpha\iff{\cal M}^a_x\vDash\alpha \hbox{ for all $a\in{\cal A}$}$\hfil\cr
    }
    \medskip

    If ${\cal M}\vDash\phi$, then ${\cal M}$ is said to model $\phi$.
    And if $X\subseteq{\cal L}$ is a set of formulas, we write ${\cal M}\vDash X$ if for all $\phi\in X$, ${\cal M}\vDash\phi$, and we similarly say ${\cal M}$ models  $X$.

\edefn

We can generalize ${\cal M}^a_x$ to ${\cal M}^{\vec a}_{\vec x}$ where the underlying structure remains the same and
$$ y^{{\cal M}^{\vec a}_{\vec x}} = \cases{a_i & $y=x_i$\cr y^{\cal M} & else} $$
Notice that ${\cal M}^{\vec a}_{\vec x} = ({\cal M}^{a_1}_{x_1})^{a_2}_{x_2}\dots$.
It follows immediately that if we use $\forall\vec x$ as an abbreviation for $\forall x_1\forall x_2\cdots\forall x_n$, then we get
$$ {\cal M}\vDash\forall\vec x\alpha \iff {\cal M}^{\vec a}_{\vec x}\vDash\alpha\hbox{ for all $\vec a\in{\cal A}^n$} $$

It is easily verifiable that

\medskip
\tabskip=0pt plus 1fil
\halign to\hsize{$#$\hfil\tabskip=0pt&${}\iff#$\hfil\tabskip=1cm&$#$\hfil\tabskip=0pt&${}\iff#$\hfil\tabskip=0pt plus 1fil\cr
    {\cal M}\vDash(\alpha\lor\beta) & {\cal M}\vDash\alpha\hbox{ or }{\cal M}\vDash\beta & {\cal M}\vDash(\alpha\to\beta) & \hbox{if }{\cal M}\vDash\alpha\hbox{ then }{\cal M}\vDash\beta\cr
    {\cal M}\vDash(\alpha\oto\beta) & {\cal M}\vDash\alpha\hbox{ if and only if }{\cal M}\vDash\beta\cr
}
\medskip

And also ${\cal M}\vDash\exists x\alpha=\neg\forall x\neg\alpha$ if and only if ${\cal M}\nvDash\forall x\neg\alpha$ so there exists an $a\in{\cal A}$ such that ${\cal M}^a_x\nvDash\neg\alpha$,
meaning there exists an $a\in{\cal A}$ such that ${\cal M}^a_x\vDash\alpha$.
This chain of reasoning is readily seen to be reversible.
So we have shown
$$ {\cal M}\vDash\exists x\alpha \iff \hbox{there exists an $a\in{\cal A}$ such that }{\cal M}^a_x\vDash\alpha $$

\bdefn

    A formula or set of formulas is said to be {\emphcolor satisfiable} if it has a model.
    $\phi\in{\cal L}$ is called a {\emphcolor tautology}\addtoindex{tautology} (or {\emphcolor generally/logically valid}), denoted $\vDash\phi$, if ${\cal M}\vDash\phi$ for every model ${\cal M}$.
    Two formulas $\alpha$ and $\beta$ are said to be {\emphcolor logically equivalent}\addtoindex{equivalence}[logical], denoted $\alpha\equiv\beta$, if for every model ${\cal M}$,
    $$ {\cal M}\vDash\alpha \iff {\cal M}\vDash\beta $$

    Now, say ${\cal A}$ is an ${\cal L}$-structure, then we write ${\cal A}\vDash\phi$ for a formula $\phi$ if $({\cal A},w)\vDash\phi$ for all valuations $w\colon\Var\longto A$.
    Similarly one writes ${\cal A}\vDash X$ for a set of formulas $X$ if ${\cal A}\vDash\phi$ for all $\phi\in X$.

\edefn

\bdefn

    Finally we define the {\emphcolor consequence relation}\addtoindex{consequence relation} for first-order logic.
    Suppose $X$ is a set of formulas and $\phi$ is a formula, then we write $X\vDash\phi$ if every model of $X$ models $\phi$.
    Meaning ${\cal M}\vDash X\implies{\cal M}\vDash\phi$.

\edefn

Again, $\vDash$ is used to denote both the satisfaction and consequence relations.
The meaning of the notation is to be understood from context.
Moreso, $\vDash$ is also used for the satisfaction relation of structures.
And again we write $\phi_1,\dots,\phi_n\vDash\phi$ in place of $\set{\phi_1,\dots,\phi_n}\vDash\phi$ and all the usual shorthands.

Notice that while by definition if ${\cal M}$ is a model then ${\cal M}\vDash\phi$ or ${\cal M}\vDash\neg\phi$ for all formulas $\phi$.
But if ${\cal A}$ is a structure, then it is possible for ${\cal A}$ to satisfy neither $\phi$ nor $\neg\phi$ (but if it does satisfy one, it cannot satisfy the other obviously).
Take for example the formula $x\eq y$, then if ${\cal A}$ is a structure with at least two elements, suppose $a\neq b\in{\cal A}$, then we can define a valuation which satisfies $x=y$ and one which does not.
And so ${\cal A}$ satisfies neither $x\eq y$ nor $\neg x\eq y=x\neqb y$.

Now suppose $\phi$ is a formula and let $x_1,\dots,x_n$ be an enumeration of $\free\phi$ (according to some accepted total order of $\Var$, for example by index), then we define the
{\it generalized of $\phi$} or its {\it universal closure}\addtoindex{universal closure} to be the sentence
$$ \phi^g \coloneqq \forall x_1\cdots\forall x_n\phi $$
From the definitions provided above, it is immediate that if ${\cal A}$ is a structure then
$$ {\cal A}\vDash\phi \iff {\cal A}\vDash\phi^g $$
And in general ${\cal A}\vDash X\iff{\cal A}\vDash X^g\coloneqq\set{\phi^g}[\phi\in X]$.

\bthrm[title=The Coincidence Theorem, name=coincidencetheorem]

    Let $\phi$ be a formula, and $V$ be a set of variables such that $\free\phi\subseteq V$.
    Let ${\cal M}$ and ${\cal M}'$ be two models over the same domain $A$ such that $x^{\cal M}=x^{{\cal M}'}$ for all variables $x\in V$, and ${\sf s}^{\cal M}={\sf s}^{\cal M'}$ for all extralogical
    symbols ${\sf s}$ occurring in $\phi$.
    Then ${\cal M}\vDash\phi$ if and only if ${\cal M}'\vDash\phi$.

\ethrm

We prove this by induction on $\phi$.
If $\phi$ is a prime formula of the form $rt_1\cdots t_n$, by the assumptions of the theorem and \refmath[proposition]{termcoincidenceprop}, $t_i^{\cal M}=t_i^{\cal M'}$ for all $1\leq i\leq n$, and
$r^{\cal M}=r^{\cal M'}$, so $r^{\cal M}\vec t^{\cal M}\iff r^{\cal M'}\vec t^{\cal M'}$ as required.
This proof holds for equations as well.
Now by the inductive hypothesis we get
$$ {\cal M}\vDash\alpha\land\beta \iff {\cal M}\vDash\alpha\hbox{ and }{\cal M}\vDash\beta \iff {\cal M}'\vDash\alpha\hbox{ and }{\cal M}'\vDash\beta \iff {\cal M}'\vDash\alpha\land\beta $$
Similar for formulas of the form $\neg\alpha$.

Now, let $a\in{\cal A}$ and suppose ${\cal M}^a_x\vDash\phi$.
Then let $V'=V\cup\set x$ then $\free\phi\subseteq V'$ (since $\free\phi\subseteq\free\forall x\phi\cup\set x\subseteq V\cup\set x$) and ${\cal M}^a_x$ and ${\cal M'}^a_x$ coincide for all $y\in V'$
(though it is possible that $x^{\cal M}\neq x^{\cal M'}$).
Thus by our inductive hypothesis ${\cal M}^a_x\vDash\phi$ if and only if ${\cal M'}^a_x\vDash\phi$.
Thus
$$ {\cal M}\vDash\forall x\phi \iff {\cal M}^a_x\vDash\phi\hbox{ for all $a\in{\cal A}$} \iff {\cal M'}^a_x\vDash\phi\hbox{ for all $a\in{\cal A}$} \iff {\cal M}'\vDash\forall x\phi $$
as required.
\qed

Let $\sigma\subseteq\sigma'$ be two signatures, and ${\cal L}\subseteq{\cal L'}$ be their respective first-order languages.
Now, if ${\cal M}=({\cal A},w)$ is an ${\cal L}$-model, it can be arbitrarily extended to an ${\cal L}'$-model ${\cal M}'=({\cal A}',w)$, where ${\cal A}'$ is the $\sigma'$-expansion of ${\cal A}$, by
arbitrarily setting ${\sf s}^{\cal M'}$ for ${\sf s}\in\sigma'\setminus\sigma$.
Now, let us set $V=\Var$ and by the coincidence theorem we get that for every $\phi\in{\cal L}$ since ${\cal M}$ and ${\cal M}'$ agree on the extralogical symbols (as ${\cal A}'$ is an expansion of
${\cal A}$) and variables n $V$ (since the valuation remains the same), we get that
$$ {\cal M}\vDash\phi \iff {\cal M}'\vDash\phi $$

If we denote the consequence relation of ${\cal L}$ by $\vDash_{\cal L}$, then it follows that if ${\cal L}\subseteq{\cal L}'$, $\vDash_{\cal L'}$ is a {\it conservative} extension of $\vDash_{\cal L}$:
for every $\phi\in{\cal L}$ and $X\subseteq{\cal L}$, $X\vDash_{\cal L'}\phi$ if and only if $X\vDash_{\cal L}\phi$.
Indeed: if ${\cal M}'$ is an ${\cal L'}$-model then let ${\cal M}$ be the ${\cal L}$-reduct of ${\cal M}'$ and so ${\cal M}\vDash_{\cal L} X$ if and only if ${\cal M}'\vDash_{\cal L'}X$, and same for $\phi$.

So the satisfiability of $\phi$ depends only on the symbols occurring in $\phi$, we need not the subscripts in $\vDash$.

Another consequence of the coincidence theorem is the {\it omission of superfluous quantifiers}:
$$ \forall x\phi \equiv \phi \equiv \exists x\phi \hbox{ if $x\notin\free\phi$} $$
To see this, let ${\cal M}$ be a model and $a\in{\cal A}$ be arbitrary.
Then let $V=\free\phi$ and ${\cal M}'={\cal M}^a_x$, and by the coincidence theorem since $y^{\cal M}=y^{\cal M'}$ for all $y\in V$ (since $x\notin\free\phi$) we have that ${\cal M}\vDash\phi$ if and
only if ${\cal M}^a_x\vDash\phi$.
So ${\cal M}\vDash\forall x\phi$ if and only if ${\cal M}^a_x\vDash\phi$ for all $a\in{\cal A}$, which is if and only if ${\cal M}\vDash\phi$, which is if and only if ${\cal M}^a_x\vDash\phi$ for some
$a\in{\cal A}$, which is by definition ${\cal M}\vDash\exists x\phi$.

This fact should be intuitive, for example $\forall x\exists x(x>0)$ is the same as $\exists x(x>0)$ and $\exists x\exists x(x>0)$ since the outermost quantifier is superfluous.

Another thing which is simple to show is that if ${\cal A}$ is a substructure of ${\cal B}$, and ${\cal M}=({\cal A},w)$ and ${\cal M}'=({\cal B},w)$ are models with the same valuation
$w\colon\Var\longto A$, then for every term $t$, $t^{\cal M}=t^{\cal M'}$.
For prime terms this is obvious (for terms of the form $x$, this is since the same valuation is used, and for constant terms this is by definition of substructures).
Now inductively, if $t_i^{\cal M}=t_i^{\cal M'}$ then since $t_i^{\cal M}\in{\cal A}$, and $f^{\cal M'}\vec a=f^{\cal M}\vec a$ for values $\vec a\in{\cal A}^n$, we have that
$$ (f\vec t)^{\cal M'} = f^{\cal M'}\vec t^{\cal M'} = f^{\cal M}\vec t^{\cal M} = (f\vec t)^{\cal M} $$

Now by the coincidence theorem, we know that the satisfaction of a formula $\phi$ is dependent only on the valuations of variables occurring in $\phi$.
So let $\phi=\phi(\vec x)$ (meaning $\free\phi\subseteq\set{x_1,\dots,x_n}$), then saying
$$ ({\cal A},w)\vDash\phi\hbox{ for a valuation $w$ where $x_1^w=a_1,\dots,x_n^w=a_n$} $$
can be written more suggestively by
$$ ({\cal A},\vec a)\vDash\phi,\hbox{ or } {\cal A}\vDash\phi[a_1,\dots,a_n],\hbox{ or }{\cal A}\vDash\phi[\vec a] $$
So there is no need to mention the global valuation $w$.

\bthrm[title=The Substructure Theorem, name=substructuretheorem]

    For $\cal L$-structures ${\cal A}$ and ${\cal B}$ such that $A\subseteq B$ (where $A$ represents the domain of ${\cal A}$, and so on), the following conditions are equivalent:
    \benum
        \item ${\cal A}$ is a substructure of ${\cal B}$: ${\cal A}\subseteq{\cal B}$,
        \item ${\cal A}\vDash\phi[\vec a]$ if and only if ${\cal B}\vDash\phi[\vec a]$ for all quantifier-free $\phi=\phi(\vec x)$ and $\vec a\in A^n$,
        \item ${\cal A}\vDash\phi[\vec a]$ if and only if ${\cal B}\vDash\phi[\vec a]$ for all prime $\phi=\phi(\vec x)$ and $\vec a\in A^n$.
    \eenum

\ethrm

First we prove $(1)\implies(2)$: so we must show that for models ${\cal M}=({\cal A},w)$ and ${\cal M}'=({\cal B},w)$, ${\cal M}\vDash\phi$ if and only if ${\cal M}'\vDash\phi$.
For prime formulas this is obvious as we showed that $\vec t^{\cal M}=\vec t^{\cal M'}$ for all vectors of terms $\vec t$.
The inductive step for $\land$ and $\neg$ are simple.
$(2)\implies(3)$ is trivial as prime formulas are quantifier-free.
Now to show that $(3)\implies(1)$ we must show that $f^{\cal A}\vec a=f^{\cal B}\vec a$ and $r^{\cal A}\vec a\iff r^{\cal B}\vec a$ for $\vec a\in A^n$.
By $(3)$, we know that
$$ r^{\cal A}\vec a \iff {\cal A}\vDash r\vec x[\vec a] \iff {\cal B}\vDash r\vec x[\vec a] \iff r^{\cal B}\vec a $$
($\vec x=(x_1,\dots,x_n)$ is any vector of $n$ distinct variables.)
Similarly
$$ f^{\cal A}\vec a = b \iff {\cal A}\vDash f\vec x\eq y[\vec a,b] \iff {\cal B}\vDash f\vec x\eq y[\vec a,b] \iff f^{\cal B}\vec a=b $$
and so $f^{\cal A}\vec a=f^{\cal B}\vec a$, meaning ${\cal A}\subseteq{\cal B}$ as required (constants are viewed as $0$-ary functions; the proof is the same).
\qed

\bdefn

    A {\emphcolor universal formula}\addtoindex{formulas}[universal] (or a {\emphcolor $\forall$-formula}) is a formula $\alpha$ of the form $\forall\vec x\beta$ where $\beta$ is quantifier-free.
    If $\alpha$ is a sentence, it is also called a {\emphcolor universal sentence} or {\emphcolor $\forall$-sentence}.
    Similarly, an {\emphcolor existential formula}\addtoindex{formulas}[existential] (or a {\emphcolor $\exists$-formula}) is a formula $\alpha$ of the form $\exists\vec x\beta$ where $\beta$ is
    quantifier-free.
    And again if $\alpha$ is a sentence, it is also called an {\emphcolor existential sentence} or {\emphcolor $\exists$-sentence}.

\edefn

For example, we can define the following existential sentences:
$$ \exists_1 \coloneqq \exists\v_0\v_0\eq\v_0,\qquad \exists_n\coloneqq \exists\v_0\cdots\exists\v_{n-1}\bigwedge_{0\leq i<j<n}\v_i\neqb\v_j $$
So $\exists_n$ states ``there exists at least $n$ (distinct) elements''.
This means that $\neg\exists_{n+1}$ means ``there exists at most $n$ elements'', and so we can define $\exists_{=n}\coloneqq\exists_n\land\neg\exists_{n+1}$ which means ``there exists exactly $n$
elements''.

Since structures are non-empty by definition, $\exists_1$ is a tautology, and so we define $\top\coloneqq\exists_1$, and $\bot\coloneqq\exists_0\coloneqq\neg\top$.
From the substructure theorem we get the following corollary:

\bcoro

    Let ${\cal A}\subseteq{\cal B}$, then every $\forall$-sentence which is valid in ${\cal B}$ is also valid in ${\cal A}$.
    Conversely, every $\exists$-sentence which is valid in ${\cal A}$ is also valid in ${\cal B}$.

\ecoro

Suppose ${\cal B}\vDash\forall\vec x\beta$ where $\beta$ is quantifier-free.
Then let $\vec a\in A^n$, and so ${\cal B}\vDash\beta[\vec a]$, and by the substructure theorem since $\beta$ is quantifier-free, we get ${\cal A}\vDash\beta[\vec a]$.
Since $\vec a$ is arbitrary, this means ${\cal A}\vDash\forall\vec x\beta$ as required.
And similarly, if ${\cal A}\vDash\exists\vec x\beta$ then there exists a $\vec a\in A^n$ such that ${\cal A}\vDash\beta[\vec a]$ and so again by the substructure theorem we get ${\cal B}\vDash\beta[\vec a]$,
so ${\cal B}\vDash\exists\vec x\beta$.
\qed

\bthrm[title=The Invariance Theorem, name=invariancetheorem]

    Suppose ${\cal A}$ and ${\cal B}$ are isomorphic $\cal L$-structures with an isomorphism $\iota\colon{\cal A}\longto{\cal B}$.
    Then for every formula $\phi=\phi(\vec x)$, and all $\vec a\in A^n$:
    $$ {\cal A}\vDash\phi[\vec a] \iff {\cal B}\vDash\phi[\iota\vec a] $$
    where again, $\iota\vec a=(\iota a_1,\dots,\iota a_n)$.

\ethrm

It is sufficient to show that for models ${\cal M}=({\cal A},w)$ and ${\cal M'}=({\cal B},w')$ where $w'=\iota\circ w$, ${\cal M}\vDash\phi\iff{\cal M}'\vDash\phi$.
We will first prove by term induction that for every term $t$, $\iota(t^{\cal M})=t^{\cal M'}$.
For prime terms, this is obvious:
$$ x^{\cal M'} = x^{w'} = \iota x^w = \iota(x^{\cal M}),\qquad c^{\cal M'} = c^{\cal B} = \iota(c^{\cal A}) = \iota(c^{\cal M}) $$
For compound terms, suppose $\iota(t_i^{\cal M})=t_i^{\cal M'}$ for $1\leq i\leq n$, then we get $\vec t^{\cal M'}=\iota\vec t^{\cal M}$ and so
$$ (f\vec t)^{\cal M'} = f^{\cal M'}\vec t^{\cal M} = f^{\cal B}\iota\vec t^{\cal M} = \iota(f^{\cal A}\vec t^{\cal M}) = \iota(f\vec t)^{\cal M} $$
the third equality is due to $\iota$ being a homomorphism.

Now we will prove the goal: ${\cal M}\vDash\phi\iff{\cal M}'\vDash\phi$, by formula induction on $\phi$.
For prime formulas $\phi$:
$$ \displaylines{
    {\cal M'}\vDash t\eq s \iff t^{\cal M'}=s^{\cal M'} \iff \iota t^{\cal M}=\iota s^{\cal M} \iff t^{\cal M} = s^{\cal M} \iff {\cal M}\vDash t\eq s\cr
    {\cal M'}\vDash r\vec t \iff r^{\cal B}\vec t^{\cal M'} \iff r^{\cal B}\iota\vec t^{\cal M} \iff r^{\cal A}\vec t^{\cal M} \iff {\cal M}\vDash r\vec t\cr
} $$
For compound formulas we use the inductive hypothesis:
$$ \displaylines{
    {\cal M'}\vDash\alpha\land\beta \iff {\cal M}'\vDash\alpha\hbox{ and }{\cal M'}\vDash\beta \iff {\cal M}\vDash\alpha\hbox{ and }{\cal M}\vDash\beta \iff {\cal M}\vDash\alpha\land\beta\cr
    {\cal M'}\vDash\neg\alpha \iff {\cal M}'\nvDash\alpha \iff {\cal M}\nvDash\alpha \iff {\cal M}\vDash\neg\alpha\cr
} $$
And for quantified formulas, ${\cal M'}\vDash\forall x\phi$ if and only if ${\cal M'}^b_x\vDash\phi$ for all $b\in{\cal B}$.
Since $\iota$ is surjective, $b=\iota a$ for some $a\in A$ and so if ${\cal M}^a_x=({\cal A},w_0)$ then ${\cal M'}^b_x=({\cal B},\iota\circ w_0)$ and so by our inductive hypothesis (as ${\cal M}$ and
${\cal M'}$ are arbitrary), this is if and only if ${\cal M}^a_x\vDash\phi$.
Since $b$ is arbitrary, and therefore so is $a$, ${\cal M}\vDash\forall x\phi$.
This chain of logic is reversible, which can also be seen since $\iota^{-1}$ is also an isomorphism.
So we have indeed shown that for all formulas, ${\cal M}\vDash\phi\iff{\cal M'}\vDash\phi$ as required.
\qed

This proof obviously covers sentences $\phi\in{\cal L}^0$.
So for example, if $G$ is a group and $\iota$ is an isomorphism to some other $\circ$-structure, then $\iota(G)$ is also a group.
This is as we can have $\phi$ run over the axioms of group theory (which are sentences) and $G\vDash\phi\implies\iota(G)\vDash\phi$.

\bdefn

    Two ${\cal L}$-structures are {\emphcolor elementarily equivalent}\addtoindex{equivalence}[elementary] if ${\cal A}\vDash\phi\iff{\cal B}\vDash\phi$ for all sentences $\phi\in{\cal L}^0$.
    This is denoted ${\cal A}\equiv{\cal B}$.

\edefn

So by the invariance theorem isomorphic structures are elementarily equivalent.

Notice that substitutions do not pay attention to {\it collision of variables}, for example if $\phi=\exists y(x\neqb y)$ then certainly $\forall x\phi$ is true when the domain has at least two elements.
But $\phi\frac yx=\exists y(y\neqb y)$ is false.
So we have shown that $\forall x\phi$ does not necessarily imply $\phi\frac tx$, as one may expect.
But the issue is that we have changed the context of the variable $x$ from a free to a bound variable, and so the meaning of $\phi\frac yx$ is different from $\phi$.

\bdefn

    A simple substitution $\phi,\frac tx$ is called {\emphcolor collision-free}\addtoindex{substitution}[collision-free] if every $y\in\var t$ distinct from $x$ does not occur bound in $\phi$
    (meaning $\bnd\phi\cap\var t\subseteq\set x$, or $y\in\var t\setminus\set x\implies y\notin\bnd\phi$).
    This means that a variable being inserted into $\phi$ does not get potentially inserted into a scope of its own quantifier.

    A global substitution $\sigma$ is collision-free if $\phi,\frac{x^\sigma}x$ is collision-free for every variable $x\in\Var$.
    So for a simultaneous substitution $\phi,\frac{\vec t}{\vec x}$ it is only necessary to verify that $\phi,\frac{t_i}{x_i}$ is collision-free.

\edefn

The reason for not requiring $x\notin\bnd\phi$ in a substitution $\frac tx$ is since $t$ is only substituted at free occurrences of $x$, so whether or not $x$ occurs bound in $\phi$ is immaterial.
(Though such a restriction wouldn't matter practically: it is bad practice to have a formula where the same variable occurs both free and bound.
In some texts, different symbols are used for free and bound variables even.)

This is a crude definition, for example if $\phi=\forall x(x=x)\land\forall y(y=y)$, then $\phi\frac yx$ is not collision-free since $y$ occurs bound in $\phi$.
But such a substitution would not alter $\phi$.
We could refine the definition of collision-free substitutions to mean that $x$ does not occur within the scope of any variable in $\var t$, other than potentially $x$ itself.
Some texts refer to this definition as {\it $t$ is free for $x$ in $\phi$}.
Such a refinement is unnecessary for our purposes though.

For a model ${\cal M}=({\cal A},w)$ and a global substitution $\sigma$, we define ${\cal M}^\sigma=({\cal A},w^\sigma)$ where $x^{w^\sigma}=(x^\sigma)^{\cal M}$ (first substitute $x$ to get a term, and then
valuate it via ${\cal M}$) for every $x\in\Var$.
In other words $x^{{\cal M}^\sigma} = x^{\sigma{\cal M}}\mathrel{}(=(x^\sigma)^{\cal M})$.
This extends to terms: $t^{{\cal M}^\sigma}=t^{\sigma{\cal M}}$.
This is obvious for prime terms (variables and constants), and for compound terms:
$$ (ft_1\cdots t_n)^{{\cal M}^\sigma} = f^{\cal M}t_1^{\cal M^\sigma}\cdots t_n^{\cal M^\sigma} = f^{\cal M}t_1^{\sigma\cal M}\cdots t_n^{\sigma\cal M} = (ft_1\cdots t_n)^{\sigma\cal M} $$
Notice that in the case $\sigma=\frac{\vec t}{\vec x}$,
$$ x^{\cal M^\sigma} = x^{\sigma\cal M} = \parens{x\frac{\vec t}{\vec x}}^{\cal M} $$
and so ${\cal M}^\sigma$ coincides with ${\cal M}^{\vec t^{\cal M}}_{\vec x}$ ($\vec t^{\cal M}$ since the terms must be interpreted, ${\cal M}^{\vec t}_{\vec x}$ makes no sense sine $\vec t$ is not a
vector in the domain).

\bthrm[title=The Substitution Theorem, name=substitutiontheorem]

    Let ${\cal M}$ be a model and $\sigma$ be a global substitution.
    If $\phi,\sigma$ are collision-free then
    $$ {\cal M}\vDash\phi^\sigma \iff {\cal M}^\sigma\vDash\phi $$
    In particular, ${\cal M}\vDash\phi\frac{\vec t}{\vec x}$ if and only if ${\cal M}^{\vec t^{\cal M}}_{\vec x}\vDash\phi$, provided $\phi,\frac{\vec t}{\vec x}$ is collision-free.

\ethrm

This will be proven by formula induction.
In the case that $\phi$ is an equation $t_1\eq t_2$,
$$ {\cal M}\vDash(t_1\eq t_2)^\sigma \iff t_1^{\sigma\cal M} = t_2^{\sigma\cal M} \iff t_1^{\cal M^\sigma} = t_2^{\cal M^\sigma} \iff {\cal M}^\sigma\vDash t_1\eq t_2 $$
Similarly, for prime formulas of the form $r\vec t$:
$$ {\cal M}\vDash (r\vec t)^\sigma \iff r^{\cal M}\vec t^{\sigma\cal M} \iff r^{\cal M^\sigma}\vec t^{\cal M^\sigma} \iff {\cal M}^\sigma\vDash r\vec t $$
The induction steps for $\land$ and $\neg$ are obvious and simple.
Now, recall that ${\cal M}\vDash\forall x\phi$ if and only if ${\cal M}\vDash\forall x\phi^\tau$ where $x^\tau=x$ and $y^\tau=y^\sigma$ for $x\neq y$,
$$ \iff {\cal M}^a_x\vDash\phi^\tau \hbox{ for every $a\in A$} \iff ({\cal M}^a_x)^\tau \vDash \phi $$
now we claim that $({\cal M}^a_x)^\tau=({\cal M}^\sigma)^a_x$.
Since $\forall x\phi,\sigma$ is collision-free, meaning $\forall x\phi,\frac{y^\sigma}y$ for every $y$ is collision-free so $x\notin\var y^\sigma$ for every $y\neq x$.
And since $y^\tau=y^\sigma$, we get $y^{({\cal M}^a_x)^\tau}=y^{\tau{\cal M}^a_x}=y^{\sigma{\cal M}^a_x}=y^{\sigma\cal M}=y^{\cal M^\sigma}=y^{({\cal M}^\sigma)^a_x}$.
And $x^{({\cal M}^a_x)^\tau} = x^{{\cal M}^a_x} = a = x^{({\cal M^\sigma})^a_x}$.
Thus
$$ \iff ({\cal M}^\sigma)^a_x\vDash\phi \hbox{ for every $a\in A$} \iff {\cal M}^\sigma\vDash\forall x\phi $$
as required.
\qed

This proof also obviously works for the refinement of the concept of collision-free substitutions (what we termed ``\dots free for $x$ in $\phi$'').

\bcoro[name=universaltosubstitution]

    For all formulas $\phi$ and $\frac{\vec t}{\vec x}$ such that $\phi,\frac{\vec t}{\vec x}$ is collision-free, the following are true:
    \benum
        \item $\forall\vec x\phi\vDash\phi\frac{\vec t}{\vec x}$,
        \item $\phi\frac{\vec t}{\vec x}\vDash\exists\vec x\phi$,
        \item $\phi\frac sx,\,s\eq t\vDash\phi\frac tx$, provided $\phi,\frac sx$ and $\phi,\frac tx$ are collision-free.
    \eenum

\ecoro

\benum
    \item Let ${\cal M}\vDash\forall\vec\phi$, so ${\cal M}^{\vec a}_{\vec x}\vDash\phi$ for all $\vec a\in A^n$.
    This means that ${\cal M}^{\vec t^{\cal M}}_{\vec x}\vDash\phi$, and so by the substitution theorem this is equivalent to ${\cal M}\vDash\phi\frac{\vec t}{\vec x}$.
    \item Since $\neg\exists\vec x\phi=\forall\vec x\neg\phi$, by (1) $\neg\exists\vec x\phi\vDash\neg\phi\frac{\vec t}{\vec x}$.
    And so $\phi\frac{\vec t}{\vec x}\vDash\exists\vec\phi$ (since $\neg X\vDash\neg Y$ if and only if $Y\vDash X$).
    \item Suppose ${\cal M}\vDash\phi\frac sx,\,s\eq t$ so $s^{\cal M}=t^{\cal M}$ and ${\cal M}^{s^{\cal M}}_x\vDash\phi$ by the substitution theorem, and so ${\cal M}^{t^{\cal M}}_x\vDash\phi$, and
    by applying the substitution theorem again we get ${\cal M}\vDash\phi\frac tx$.
    \qed
\eenum

This shows that collision-free substitutions act as substitutions should: if $\phi$ is true for all $\vec x$, then we should be able to substitute any terms $\vec t$ in place of $\vec x$ and $\phi$ should
hold.
What we need to be careful of is that this substitution doesn't mess with the syntax of $\phi$, which is why it must be collision-free.

Let us define the following quantifier, ``there exists exactly one'':
$$ \exists!x\phi \coloneqq \exists x\phi\land\forall x\forall y(\phi\land\phi\tfrac yx\to y\eq x) $$
where $y$ is any variable not in $\var\phi$.
So if ${\cal M}\vDash\forall x\forall y(\phi\land\phi\frac yx\to y\eq x)$ means that ${\cal M}^a_x\vDash\phi$ and ${\cal M}^b_x\vDash\phi\frac yx$ implies $a=b$ (since $y\notin\var\phi$).
Putting it all together, ${\cal M}\vDash\exists!x\phi$ if and only if there exists a single $a$ such that ${\cal M}^a_x\vDash\phi$.
An example of a tautology is $\exists!x x\eq t$ for any term $t$.

An equivalent definition of $\exists!$ is
$$ \exists!x\phi \coloneqq \exists x\forall y(\phi\tfrac yx\oto y\eq x) $$

\bexerc

    Suppose $X\vDash\phi$ and $x\notin\free\phi$, then show that $X\vDash\forall x\phi$.

\eexerc

Suppose ${\cal M}\vDash X$ and let $a\in A$, then notice that ${\cal M}^a_x$ and ${\cal M}$ agree on all variables in $\free\phi$ and extralogical symbols (as they are defined over the same structure).
So by \refmath{substitutiontheorem} since ${\cal M}\vDash\phi$, ${\cal M}^a_x\vDash\phi$.
Since $a$ is arbitrary, this means ${\cal M}\vDash\forall x\phi$.

\bexerc

    Show that $\forall x(\alpha\to\beta)\vDash\forall x\alpha\to\forall x\beta$.

\eexerc

Let ${\cal M}\vDash\forall x(\alpha\to\beta)$.
Then if ${\cal M}\vDash\forall x\alpha$ then let $a\in A$, so ${\cal M}^a_x\vDash\alpha,\alpha\to\beta$ so ${\cal M}^a_x\vDash\beta$ by modus ponens.
Since $a$ is arbitrary, this means ${\cal M}\vDash\forall x\beta$.
Thus ${\cal M}\vDash\forall x\alpha\to\forall x\beta$.

\bexerc

    Suppose ${\cal A}$ is an ${\cal L}$-structure and ${\cal A}'$ is obtained by adjoining a constant symbol $\boldsymbol{a}$ for some $a\in A$ (meaning $\boldsymbol{a}^{\cal A'}=a$).
    Show that ${\cal A}\vDash\alpha[a]\iff{\cal A}'\vDash\alpha\frac{\boldsymbol a}{x}$ for a formula $\alpha=\alpha(x)$.

\eexerc

We will first show that $t(x)^{{\cal A},a}=(t\frac{\boldsymbol a}x)^{\cal A'}$.
For prime terms this is obvious.
For compound terms this is simple by induction.

It is sufficient to show that for arbitrary ${\cal M}=({\cal A},w)$ and ${\cal M'}=({\cal A'},w)$ such that $x^w=a$, ${\cal M}\vDash\alpha\iff{\cal M}'\vDash\alpha\frac{\boldsymbol a}x$.
By formula induction on $\alpha$: in the case that $\alpha=t_1\eq t_2$,
$$ {\cal M'}\vDash t_1\tfrac{\boldsymbol a}x\eq t_2\tfrac{\boldsymbol a}x \iff \parens{t_1\tfrac{\boldsymbol a}x}^{\cal M'} = \parens{t_2\tfrac{\boldsymbol a}x}^{\cal M'} \iff
t_1^{\cal M}=t_2^{\cal M} \iff {\cal M}\vDash t_1\eq t_2 $$
Similar for prime formulas of the form $r\vec t$ (there is of course a similarity between equations and prime formulas of the form $r\vec t$, $\eq$ is a ``primitive'' binary relation; it represents the
identity relation).
For compound formulas using connectives $\land$ and $\neg$ the inductive step is clear.
And for formulas $\alpha=\forall y\beta$, if $y\neq x$ then this is clear.
If $y=x$ then this is clear by the substitution theorem, as ${\cal A}$ and ${\cal A'}$ act the same on $\alpha$ as it is a sentence.

\bexerc

    Prove the following:
    \benum
        \item A conjunction of formulas $\exists_i$ and their negations is logically equivalent to a formula of the form $\exists_n\land\neg\exists_m$ for suitable $n$ and $m$.
        \item A boolean combination of $\exists_i$ is equivalent to a formula of the form $\bigvee_{i\leq n}\exists_{=k_i}$ or $\exists_k\lor\bigvee_{i\leq n}\exists_{=k_i}$ where $k_0<\cdots<k_n<k$.
    \eenum

\eexerc

\benum
    \item Every such conjunction is equivalent, by commutativity, to a formula of the form
    $$ \bigwedge_{1\leq i\leq n}\exists_{k_i}\land\bigwedge_{1\leq j\leq m}\neg\exists_{\ell_j} $$
    Where $k_1<\cdots<k_n$ and $\ell_1<\cdots<\ell_m$.
    Since $\exists_i$ represents there existing at least $i$ elements, and $\neg\exists_i$ means there exists $<i$ elements, this is equivalent to $\equiv\exists_{k_n}\land\neg\exists_{\ell_1}$.
    If $n=0$ then take $k_0=1$ (since $\exists_1$ is a tautology) and if $m=0$ then take $\ell_1=0$ (since $\neg\exists_0$ is a tautology).

    \item Let us prove this by induction on the rank of the boolean combination, $r$.
    If $r=0$ then the formula is $\exists_n$, which is of the second form.
    Otherwise, there are three cases:
    \benum
        \item The formula is of the form $\neg\alpha$ where $\alpha\equiv\bigvee_{i\leq n}\exists_{=k_i}$ or $\exists_k\lor\bigvee_{i\leq n}\exists_{=k_i}$.
        In the first case,
        $$ \neg\alpha \equiv \bigwedge_{i\leq n}\neg\exists_{=k_i} \equiv \exists_{k_n+1}\lor\bigwedge_{i\leq k_n,i\neq k_j}\exists_{=i} $$
        This is as the negation of $\alpha$ refers to there not existing precisely $k_i$ elements, so this means there is either $\geq k_n+1$ elements or precisely $i$ elements for $i\leq k_n$ where
        $i$ differs from any $k_j$.
        In the second case,
        $$ \neg\alpha \equiv \neg\exists_k\land\bigwedge_{i\leq n}\neg\exists_{=k_i} $$
        By above, both parts of the conjunction is equivalent to one of the forms required by the exercise, so this case is covered by case (3).
        \item The formula is of the form $\alpha\land\beta$ where $\alpha$ and $\beta$ are boolean combinations.
        It is obvious how this formula is equivalent to a formula of one of the forms required.
        \item If the formula is of the form $\alpha\lor\beta$, then it is already (after $\alpha$ and $\beta$ are converted to the correct form) in a required form.
    \eenum
\eenum

