Similar to the expansion and reduct of structures, if ${\cal L}\subseteq{\cal L'}$, ${\cal L}$ is called the {\it reduct} of ${\cal L}'$ and ${\cal L'}$ is called the {\it expansion} of ${\cal L}$.
Since the satisfaction requirements for $\land$ and $\neg$ are the same as in propositional logic, the basic rules for $\vDash$ in propositional logic carry over to first order logic.
These are

\medskip
\tabskip=0pt plus 1fil
{\openup1\jot\halign to\hsize{($#$)\hfil\tabskip=.2cm&$#$\hfil\tabskip=2cm&($#$)\hfil\tabskip=.2cm&$#$\hfil\tabskip=0pt plus 1fil\cr
    \hbox{IS} & \gentzen{}{\alpha\vDash\alpha} & \hbox{MR} & \gentzen{X\vDash\alpha}{X'\vDash\alpha}\quad (X\subseteq X')\cr
    \land1 & \gentzen{X\vDash\alpha,\beta}{X\vDash\alpha\land\beta} & \land2 & \gentzen{X\vDash\alpha\land\beta}{X\vDash\alpha,\beta}\cr
    \neg1 & \gentzen{X\vDash\alpha,\neg\alpha}{X\vDash\beta} & \neg2 & \gentzen{X,\alpha\vDash\beta & X,\neg\alpha\vDash\beta}{X\vDash\beta}\cr
}}
\medskip

But we also have new properties:

\medskip
\tabskip=0pt plus 1fil
\mlcount=0
{\openup1\jot\halign to\hsize{\global\advance\mlcount by 1\relax\hfil$(\the\mlcount)$#\tabskip=.25cm&$\displaystyle#$\hfil\tabskip=.5cm&(#)\hfil\tabskip=0pt plus 1fil\cr
    & \gentzen{X\vDash\forall x\alpha}{X\vDash\alpha\frac tx} & $\alpha,\frac tx$ collision-free\cr
    & \gentzen{X\vDash\alpha\tfrac sx,\,s\eq t}{X\vDash\alpha\frac tx} & $\alpha,\frac sx$ and $\alpha,\frac tx$ collision-free\cr
    & \gentzen{X,\beta\vDash\alpha}{X,\forall x\beta\vDash\alpha} & anterior generalization\cr
    & \gentzen{X\vDash\alpha}{X\vDash\forall x\alpha} & $x\notin\free X$, posterior generalization\cr
    & \gentzen{X,\beta\vDash\alpha}{X,\exists x\beta\vDash\alpha} & $x\notin\free X,\free\alpha$, anterior particularization\cr
    & \gentzen{X\vDash\alpha\tfrac tx}{X\vDash\exists x\alpha} &  $\alpha,\frac tx$ collision-free, posterior particularization\cr
}}

\medskip
$(1)$ is due to $\forall x\alpha\vDash\alpha\frac tx$ and the transititvity of $\vDash$.
Similarly $(2)$ is due to $\alpha\frac sx,\,s\eq t\vDash\alpha\frac tx$.
And $(3)$ is due to $\forall x\beta\vDash\beta$.
We now prove $(4)$.
Suppose $X\vDash\alpha$, ${\cal M}\vDash X$, and $x\notin\free X$.
Then by \refmath{coincidencetheorem}, for every $a\in A$, ${\cal M}^a_x\vDash X$ and so ${\cal M}^a_x\vDash\alpha$ meaning ${\cal M}\vDash\forall x\alpha$ as required.
For $(5)$, notice that by $(4)$:
$$ X,\beta\vDash\alpha \implies X,\neg\alpha\vDash\neg\beta \implies X,\neg\alpha\vDash\forall x\neg\beta \implies X,\neg\forall x\neg\beta\vDash\alpha \implies X,\exists x\beta\vDash\alpha $$
And by \refmath[corollary]{universaltosubstitution}, $\alpha\frac tx\vDash\exists x\alpha$ proving $(6)$.

Some texts define a stricter consequence relation, which we call the {\it global consequence relation}\addtoindex{consequence relation}[global], denoted here by $\gvDash$.
For a set of formulas $X\subseteq{\cal L}$ and a formula $\phi\in{\cal L}$, we define $X\gvDash\phi$ if and only if ${\cal A}\vDash X$ implies ${\cal A}\vDash\phi$ for all {\it structures} ${\cal A}$.
(The difference here is subtle: the ``local'' consequence relation deals in models, while the global consequence relation deals in structures.)

Obviously $X\vDash\phi$ implies $X\gvDash\phi$: let ${\cal A}\vDash X$ then since for every valuation ${\cal A},w\vDash X\implies{\cal A},w\vDash\phi$ so ${\cal A}\vDash\phi$.
The converse is not true though: for example $x\eq y\gvDash\forall x\forall y\,x\eq y$ since if ${\cal A}\vDash x\eq y$ then ${\cal A}$ has a single element.
But $x\eq y\nvDash\forall x\forall y\,x\eq y$, since any model over a structure with two or more elements whose valuations of $x$ and $y$ are equal is a counterexample.
By posterior generalization, $X\vDash\phi\implies X\vDash\phi^g$ is true in general only if the free variables of $\phi$ do not occur free in $X$.
But on the other hand $X\gvDash\phi\implies X\gvDash\phi^g$ is always true (since ${\cal A}\vDash\phi\iff{\cal A}\vDash\phi^g$).

Now, since for every structure ${\cal A}$, valuation $w$, and set of formulas $X$: ${\cal A},w\vDash X^g\iff{\cal A}\vDash X^g$ by the coincidence theorem ($\free X^g=\varnothing$).
Thus
$$ X\gvDash\phi \iff X^g\vDash\phi $$
If $X\gvDash\phi$, let ${\cal A},w\vDash X^g$ then ${\cal A}\vDash X^g$ (explained above) and so ${\cal A}\vDash X$ meaning ${\cal A}\vDash\phi$ and in particular ${\cal A},w\vDash\phi$.
Conversely, if $X^g\vDash\phi$ suppose ${\cal A}\vDash X$ then ${\cal A}\vDash X^g$ and so ${\cal A}\vDash\phi$.
If $S$ is a set of sentences, then $S^g=S$ and so we get that
$$ S\gvDash\phi \iff S\vDash\phi $$
so for sets of sentences, there is no difference.
But otherwise the two consequence relations act differently, for example neither the rule of case distinction nor the deduction theorem hold.
These are the rules:
$$ \gentzen{X,\alpha\gvDash\beta & X,\neg\alpha\gvDash\beta}{X\gvDash\beta}\ (\hbox{rule of case distinction}),\qquad \gentzen{X,\alpha\gvDash\beta}{X\gvDash\alpha\to\beta}\ (\hbox{deduction theorem}) $$
For example, $x\eq y\gvDash\forall x\forall y\,x\eq y$ but it is not necessarily true that $\gvDash x\eq y\to\forall x\forall y\,x\eq y$.

\bdefn

    A {\emphcolor first-order theory}\addtoindex{theory} over a language ${\cal L}$ (also a {\emphcolor ${\cal L}$-theory}), is a set of sentences $T\subseteq{\cal L}^0$ which is
    {\emphcolor deductively closed} in ${\cal L}$, meaning $T\vDash\phi\iff\phi\in T$ for all sentences $\phi\in{\cal L}^0$.
    If $\phi\in T$ then we say that $\phi$ is a {\emphcolor theorem}\addtoindex{theorem} of $T$ (or holds or is true in $T$).
    If $T\subseteq T'$ are both ${\cal L}$-theories, then $T$ is called a {\emphcolor subtheory}\addtoindex{theory}[subtheory] of $T'$ and $T'$ is called an
    {\emphcolor extension}\addtoindex{theory}[extension] of $T$.
    An ${\cal L}$-structure ${\cal A}$ such that ${\cal A}\vDash T$ is called an {\emphcolor model} of $T$, we denote the class of all models of $T$ by $\Md T$.

\edefn

Different authors define theories slightly differently, for example some may not require the theory be deductively closed, etc.

For example, if $X\subseteq{\cal L}$ is a set of formulas we define its {\it deductive closure} to be $\set{\alpha\in{\cal L}^0}[X\vDash\alpha]$.
Since $\vDash$ is transitive, the deductive closure of a set of formulas is indeed a theory.
When we discuss the {\it theory of $S$}, where $S$ is a set of sentences, we mean the deductive closure of $S$.
A set $X\subseteq{\cal L}$ is called an {\it axiom system} for a theory $T$ if $T=\set{\alpha\in{\cal L}^0}[X^g\vDash\alpha]$.
Notice how we tacitly generalize the free variables in axioms.

Furthermore, if $T$ is a theory then $T\vDash\phi\iff{\cal A}\vDash\phi$ for all ${\cal A}\vDash T$ since $T$ is a set of sentences (in general we require that ${\cal M}\vDash\phi$ for all ${\cal M}\vDash T$
but as discussed previously, there is no distinction between global and local consequence).
And $T\vDash\phi\iff T\vDash\phi^g$.

The smallest theory in ${\cal L}$ is the theory containing only tautologies, which we denote ${\sl Taut}_{\cal L}$.
An axiom system for ${\sl Taut}_{\cal L}$ is the empty set.
The set of all sentences, ${\cal L}^0$, is the largest theory in ${\cal L}$ which is also called the {\it inconsistent theory}, and all other theories are called or {\it satisfiable}\addtoindex{satisfiable}.

Notice that if $\set{T_i}_{i\in I}$ is a family of ${\cal L}$-theories, $T=\bigcap_{i\in I}T_i$ is also an ${\cal L}$-theory: if $T\vDash\alpha$ then $T_i\vDash\alpha$ for all $i\in I$ (by (MR)) and so
$\alpha\in T$.

\bdefn

    Let $T$ be an ${\cal L}$-theory, and $\phi\in{\cal L}^0$ be a sentence.
    Then we define $T+\phi$ to be the smallest theory which extends $T$ and contains $\phi$.
    Similarly if $S\subseteq{\cal L}^0$ a set of sentences, then $T+S$ is the smallest theory which extends $T$ and contains $S$.
    Alternatively,
    $$ T + S = \bigcap\set{T'}[\hbox{$T'$ is a theory which extends $T$ and contains $S$}] $$
    If $T+\phi$ is satisfiable, then $\phi$ is said to be {\emphcolor compatible with $T$}.
    If $\neg\phi$ is compatible with $T$, then $\phi$ is said to be {\emphcolor refutable in $T$}.
    If $\phi$ is both compatible and refutable (meaning $\neg\phi$ is also compatible) with $T$, then $\phi$ is said to be {\emphcolor independent of $T$}.

\edefn

\bdefn

    If $T$ is a ${\cal L}$-theory and $\alpha,\beta\in{\cal L}$ are two formulas, we say $\alpha$ and $\beta$ are {\emphcolor equivalent modulo $T$}\addtoindex{equivalence}[in a theory], denoted
    $\alpha\equiv_T\beta$, if $\alpha\equiv_{\cal A}\beta$ for all ${\cal A}\vDash T$.
    Meaning
    $$ {\equiv_T} \coloneqq \bigcap_{{\cal A}\in\Md T}{\equiv_{\cal A}} $$
    So $\equiv_T$ is an intersection of congruences, and is therefore also a congruence.
    Obviously
    $$ \alpha\equiv_T\beta \iff T\vDash\alpha\oto\beta \iff T\vDash(\alpha\oto\beta)^g $$

    Similarly if $t$ and $s$ are ${\cal L}$-terms, we call them {\emphcolor equivalent modulo $T$}\addtoindex{equivalence}[of terms], denoted $t\approx_Ts$, if $T\vDash s\eq t$ (meaning for all
    ${\cal A}\vDash T$ and valuations $w\colon\Var\longto A$, ${\cal A},w\vDash s\eq t$).

\edefn

For example, let $T_G^=$ be the theory of groups in the signature $\set{\circ,e,{}^{-1}}$ (an equivalent formulation can be defined over the signature $\set{\circ,e}$ and gives us the equivalent theory
$T_G$), then $(x\circ y)^{-1}\approx_{T_G^=}y^{-1}\circ x^{-1}$.

Let ${\cal A}$ be an ${\cal L}$-structure, then we define its {\it theory} to be:
$$ \Th{\cal A} \coloneqq \set{\alpha\in{\cal L}^0}[{\cal A}\vDash\alpha] $$
this is indeed a theory, as if $\Th{\cal A}\vDash\alpha$ then since ${\cal A}\vDash\Th{\cal A}$ by definition, we have ${\cal A}\vDash\alpha$ so $\alpha\in\Th{\cal A}$.
And if $\boldsymbol K$ is a class of ${\cal L}$-structures, then its theory is defined to be
$$ \Th\boldsymbol K \coloneqq \bigcap_{{\cal A}\in\boldsymbol K}\Th{\cal A} $$
which is a theory as the intersection of theories.

\bexerc

    Suppose $x\notin\free X$ and $c$ is not in $X$ or $\alpha$.
    Prove
    $$ X\vDash\alpha \iff X\vDash\forall x\alpha \iff X\vDash\alpha\tfrac cx $$

\eexerc

Since $x\notin\free X$, we have $X\vDash\alpha\implies\forall x\alpha$.
And similarly $X\vDash\forall x\alpha\implies X\vDash\alpha$ since $\alpha,\frac xx$ is collision-free.
And since $\alpha,\frac cx$ is collision-free, we have $X\vDash\forall x\alpha\implies X\vDash\alpha\tfrac cx$.

Now suppose $X\vDash\alpha\tfrac cx$, then suppose ${\cal M}\vDash X$ then ${\cal M}^c_x\vDash\alpha$ by the substitution theorem.
Let $a\in A$, since $c$ is not in $X$, if we set $c^{\cal M}=a$ then ${\cal M}$ still satisfies $X$, and so ${\cal M}^a_x\vDash\alpha$, thus ${\cal M}\vDash\forall x\alpha$.
Thus $X\vDash\forall x\alpha$ as required.

\bexerc

    Let $S$ be a set of sentences, $\alpha$ and $\beta$ be formulas, $x\notin\free\beta$, and $c$ be a constant not occurring in $S$, $\alpha$, or $\beta$.
    Show that
    $$ S\vDash\alpha\tfrac cx\to\beta \iff S\vDash\exists x\alpha\to\beta $$

\eexerc

If $S\vDash\alpha\tfrac cx\to\beta$, then if ${\cal M}\vDash S$, if ${\cal M}\vDash\exists x\alpha$ then there exists an $a\in A$ such that ${\cal M}^a_x\vDash\alpha$.
Since we can set $c^{\cal M}=a$ as $c$ does not occur in $S,\alpha,\beta$ (again, the coincidence theorem), so ${\cal M}\vDash\alpha\tfrac cx$ by the substitution theorem.
Thus ${\cal M}\vDash\beta$, meaning ${\cal M}\vDash\exists x\alpha\to\beta$ so $S\vDash\exists x\alpha\to\beta$.

And if $S\vDash\exists x\alpha\to\beta$ then since $\vDash\alpha\tfrac cx\to\exists x\alpha$ we have $S\vDash\alpha\tfrac cx\to\beta$ as required.

\bexerc

    Show that for all sentences $\alpha,\beta\in{\cal L}^0$, if $T$ is an ${\cal L}$-theory then
    $$ \beta\in T+\alpha \iff \alpha\to\beta \in T $$

\eexerc

If $\beta\in T+\alpha$ then $T,\alpha\vDash\beta$ so $T\vDash\alpha\to\beta$ by the deduction theorem, thus $\alpha\to\beta\in T$ since $T$ is deductively closed.
Conversely if $\alpha\to\beta\in T$ then $T,\alpha\vDash\alpha\to\beta,\alpha$ and so $T,\alpha\vDash\beta$ thus $\beta\in T+\alpha$.

\bexerc

    Let $T\subseteq{\cal L}$ be an ${\cal L}$-theory and ${\cal L}_0\subseteq{\cal L}$ a reduction of the language ${\cal L}$.
    Show that $T_0\coloneqq T\cap{\cal L}_0$ is also a (${\cal L}_0$-)theory.

\eexerc

Suppose $T_0\vDash\alpha$ for a ${\cal L}_0$-sentence $\alpha$, then $T\vDash\alpha$ and so $\alpha\in T$ and $\alpha\in{\cal L}_0$ so $\alpha\in T_0$.
(Note that $T_0$ is not necessarily a ${\cal L}$-theory.
For example if $c$ is a constant symbol not occurring in ${\cal L}_0$ then $T_0\vDash c\eq c$ but $c\eq c$ is a sentence not in $\notin{\cal L}_0$, so $c\eq c\notin T_0$).

