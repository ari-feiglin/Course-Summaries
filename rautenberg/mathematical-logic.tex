\input pdfToolbox

\setlayout{horizontal margin=2cm, vertical margin=2cm}
\parindent=0pt

\input preamble
\input pdfmsym

\pdfmsymsetscalefactor{10}

\@Arrow@def{varLeftRightarrow}\@Larrow\@Rarrow{1}
\let\to=\varrightarrow
\let\longto=\longvarrightarrow
\let\oto=\varleftrightarrow
\def\implies{\,\longvarRightarrow\,}
\def\iff{\,\longvarLeftRightarrow\,}
\def\coloneqq{\mathrel{{\mathop:}{=}}}

\mathchardef\nvDash="3532
\mathchardef\nvdash="3530
\mathchardef\varnothing="53F

\catcode`_=11
\def\_addtoindex#1[#2]{%
    \indexize{category=#1, item=#2, value=\the\pageno, expand value, add hyperlink}%
}
\def\addtoindex#1{%
    \_ifnextchar[ {\_addtoindex{#1}}{\_addtoindex{#1}[]}%
}

\def\_alsosee#1[#2]#3#4{%
    \seealso{category=#1, item=#2, dest=#3, hyperlink=#4, index link}%
}
\def\alsosee#1{%
    \_ifnextchar[ {\_alsosee{#1}}{\_alsosee{#1}[]}%
}
\catcode`_=8

\def\createfontselector#1#2#3{%
    \edef#1{%
        \noexpand\setfont{#2}%
        \unless\ifnum#3<0 %
            \fam=#3\relax%
        \fi%
    }%
}
\createfontselector\rm{rm}0
\createfontselector\bf{bf}6
\createfontselector\it{it}8
\createfontselector\tt{tt}7

\createcounter{math counter}[subsection]

\def\curvewidth{1.5pt}
\def\curvebuffer{10pt}

\def\curremphcolor{black}
{\catcode`_=11
\gdef\emphcolor{\expanded{\noexpand\_setcolor{}{\curremphcolor}}\setfont{bf}}
}

\def\printmcount{\the\counter{section}.\the\counter{subsection}.\the\counter{math counter}}

\catcode`_=11

\def\createmathbox#1#2#3#4#5{%
    \_xp\def\csname _b#1\endcsname[##1]{
        \mapkeys{%
            title={%
                name=_title,
                default=\novalue%
            },
            name={%
                name=_name,
                default=\novalue%
            }%
        }{##1}%
        \bppbox{#3}{#4}{#5}%
        {\globalsetters\advancecounter{math counter}{1}}%
        \def\curremphcolor{#5}%
        \hbox to\hsize{%
            \emphcolor\printmcount{} #2%
            \unless\ifx\_title\novalue%
                \ (\_title)%
            \fi%
            \unless\ifx\_name\novalue%
                \_xp\anchor\_xp{\_name}%
                \_xp\xdef\csname math@\_name\endcsname{\printmcount}%
                \unless\ifx\_title\novalue%
                    \_xp\xdef\csname math@\_name\endcsname{\_title}%
                    \addthreedepth{\_title}{\_name}%
                \fi%
            \fi%
        \hfil}%
        \hrule height\z@\relax%
        \medskip%
    }%
    \_xp\edef\csname b#1\endcsname{\noexpand\_ifnextchar[ {\_xp\noexpand\csname _b#1\endcsname}{\_xp\noexpand\csname _b#1\endcsname[]}}%
    \_xp\let\csname e#1\endcsname=\eppbox%
}

\def\bproof{%
    \blinedppbox{rgb{1 1 1}}{rgb{0 0 0}}{rgb{0 0 0}}
    \hbox{\setfont{bf}Proof}
    \medskip
}
\let\eproof=\elinedppbox

\def\blankpp{%
    \blinedppbox{rgb{1 1 1}}{rgb{0 0 0}}{rgb{0 0 0}}
}
\let\eblankpp=\elinedppbox

\def\bnote{%
    \bppbox{rgb{1 1 .5}}{rgb{.5 .4 0}}{rgb{.5 .4 0}}
    \hbox{\setfont{bf}Note}
    \medskip
}
\let\enote=\eppbox

\def\__refmath#1{\gotoanchor{#1}{\csname math@#1\endcsname}}
\def\_refmath[#1]#2{\gotoanchor{#2}{#1 \csname math@#2\endcsname}}
\def\refmath{\_ifnextchar[ {\_refmath}{\__refmath}}
\catcode`_=8

\createmathbox{defn}{Definition}{rgb{1 .9 .9}}{rgb{1 .3 .3}}{rgb{.8 .1 .1}}
\createmathbox{thrm}{Theorem}{rgb{.9 .9 1}}{rgb{.3 .3 1}}{rgb{.1 .1 .8}}
\createmathbox{coro}{Corollary}{rgb{.9 .9 1}}{rgb{.3 .3 1}}{rgb{.1 .1 .8}}
\createmathbox{lemm}{Lemma}{rgb{1 .9 1}}{rgb{1 .3 1}}{rgb{.8 .1 .8}}
\createmathbox{prop}{Proposition}{rgb{1 .9 1}}{rgb{1 .3 1}}{rgb{.6 .1 .6}}
\createmathbox{prin}{Principle}{rgb{1 1 .5}}{rgb{.5 .3 0}}{rgb{.5 .3 0}}

\def\qed{%
    \ifmmode \eqno\mathchar"404%
    \else%
        \hskip1cm\penalty0\null\nobreak\hfill$\mathchar"404$%
        \par\medskip%
    \fi%
}

%\def\bdefn{%
%    \bppbox{rgb{1 .9 .9}}{rgb{1 .3 .3}}{rgb{.8 .1 .1}}
%    \advancecounter{math counter}{1}
%    \def\curremphcolor{rgb{.8 .1 0}}
%    \hbox{\emphcolor\the\counter{section}.\the\counter{subsection}.\the\counter{math counter} Definition}
%}
%
%\def\edefn{
%    \eppbox
%}

\newcount\gentzencount
\def\gentzen#1#2{%
    \,\vcenter{%
        \tabskip=.15cm\relax%
        \gentzencount=1\relax%
        \offinterlineskip%
        \halign{\strut\hfil$\displaystyle{##}$\hfil&&\global\advance\gentzencount by 1\relax\vrule\kern.15cm$\displaystyle{##}$\cr%
            #1\cr\noalign{\hrule\kern1\jot}%
            \multispan{\the\gentzencount}\hfil$#2$\hfil\cr
        }%
    }\,%
}

%%%%%%%%%%%%%%%%%%%%%%%%%%%%%%%%%%%%%%%%%%%%%%%%%%%%%%%%%%%%%%%%

\pageborder{rgb{1 1 .1}}{rgb{1 .4 0}}{5}

\color rgb{.8 .1 0}

{\def\boxshadowcolor{rgb{.8 .8 0}}
\bppbox{rgb{1 1 .5}}{rgb{1 .4 0}}{rgb{.8 .1 0}}

    \centerline{\setfontandscale{bf}{20pt}Mathematical Logic}
    \smallskip
    \centerline{\setfont{it}A summary of ``A Concise Introduction to Mathematical Logic'', W. Rautenberg}
    \centerline{\setfont{it}Ari Feiglin}

\eppbox

\bigskip

\bppbox{rgb{1 1 .5}}{rgb{1 .4 0}}{rgb{.8 .1 0}}
\section*{Contents}

\tableofcontents
\eppbox

}

\vfill\break

\color{black}

\section{Propositional Logic}

\subsection{Semantics of Propositional Logic}

\input propositional-logic/semantics

\subsection{Propositional Calculi}

\input propositional-logic/calculi

\vfill\break

\index

\bye

