\input pdfToolbox

\setlayout{horizontal margin=2cm, vertical margin=2cm}
\parindent=0pt
\parskip=3pt plus 2pt minus 2pt

\input preamble
\input pdfmsym

\pdfmsymsetscalefactor{10}

\@Arrow@def{varLeftRightarrow}\@Larrow\@Rarrow{1}
\let\to=\varrightarrow
\let\longto=\longvarrightarrow
\let\oto=\varleftrightarrow
\def\implies{\,\longvarRightarrow\,}
\def\impliedby{\,\longvarLeftarrow\,}
\def\iff{\,\longvarLeftRightarrow\,}
\def\coloneqq{\mathrel{{\mathop:}{=}}}

\mathchardef\nvDash="3532
\mathchardef\nvdash="3530
\mathchardef\nless="3504
\mathchardef\varnothing="53F
\mathchardef\varkappa="57B
\mathchardef\nless="3504
\mathchardef\ngreater="3505
\mathchardef\nmid="352D
\def\divides{{\mid}}

\protected\def\gvDash{\mathrel{\vbox{\offinterlineskip\ialign{\hfil$##$\hfil\cr\scriptstyle g\cr\noalign{\kern-1pt}\vDash\cr}}}}
\protected\def\ngvDash{\mathrel{\vbox{\offinterlineskip\ialign{\hfil$##$\hfil\cr\scriptstyle g\cr\noalign{\kern-1pt}\nvDash\cr}}}}
\protected\def\gvsim{\mathrel{\vbox{\offinterlineskip\ialign{\hfil$##$\hfil\cr\scriptstyle g\cr\noalign{\kern-1pt}\vsim\cr}}}}
\protected\def\Bvdash{\mathrel{\vbox{\offinterlineskip\ialign{\hfil$##$\hfil\cr\scriptstyle\,B\cr\noalign{\kern-2pt}\vdash\cr}}}}

\def\vsim{\mathrel{\vrule height 6.994pt width.3pt depth0pt\relax\mkern-1.25mu\raise1.6625pt\hbox{$\m@th\scriptstyle\sim$}}}

\def\boldsymbol#1{\mathchoice{\fakebold[.3]{#1}}{\fakebold[.3]{#1}}{\fakebold[.1]{#1}}{\fakebold[.1]{#1}}}

\def\Var{{\sl Var}}
\def\v{{\boldsymbol{v}}}
\def\eq{\mathrel{\boldsymbol{=}}}
\def\neqb{\not\mathrel{\boldsymbol{=}}}
\def\inb{\mathrel{\boldsymbol{\in}}}
\def\notinb{\not\mathrel{\boldsymbol{\in}}}
\def\subseteqb{\mathrel{\boldsymbol{\subseteq}}}
\def\cupb{\mathbin{\boldsymbol{\cup}}}
\def\capb{\mathbin{\boldsymbol{\cap}}}
\def\setminusb{\mathbin{\boldsymbol{\setminus}}}
\def\var{{\sl var}}
\def\bnd{{\sl bnd}}
\def\free{{\sl free}}
\def\Q{{\tt Q}}
\def\Md{\mathop{\rm Md}}
\def\Th{{\it Th}}
\def\bdivs{\boldsymbol{\divides}}
\def\bndivs{\boldsymbol{\nmid}}

\catcode`_=11
\def\_addtoindex#1[#2]{%
    \indexize{category=#1, item=#2, value=\the\pageno, expand value, add hyperlink}%
}
\def\addtoindex#1{%
    \_ifnextchar[ {\_addtoindex{#1}}{\_addtoindex{#1}[]}%
}

\def\_alsosee#1[#2]#3#4{%
    \seealso{category=#1, item=#2, dest=#3, hyperlink=#4, index link}%
}
\def\alsosee#1{%
    \_ifnextchar[ {\_alsosee{#1}}{\_alsosee{#1}[]}%
}
\catcode`_=8

\def\createfontselector#1#2#3{%
    \edef#1{%
        \noexpand\setfont{#2}%
        \unless\ifnum#3<0 %
            \fam=#3\relax%
        \fi%
    }%
}
\createfontselector\rm{rm}0
\createfontselector\bf{bf}6
\createfontselector\it{it}8
\createfontselector\tt{tt}7

\createcounter{math counter}[subsection]

\def\curvewidth{1.5pt}
\def\curvebuffer{10pt}

\def\curremphcolor{black}
{\catcode`_=11
\gdef\emphcolor{\expanded{\noexpand\_setcolor{}{\curremphcolor}}\setfont{bf}}
}

\def\printmcount{\the\counter{section}.\the\counter{subsection}.\the\counter{math counter}}

\catcode`_=11

\def\createmathbox#1#2#3#4#5{%
    \_xp\def\csname _b#1\endcsname[##1]{
        \mapkeys{%
            title={%
                name=_title,
                default=\novalue%
            },
            name={%
                name=_name,
                default=\novalue%
            }%
        }{##1}%
        \bppbox{#3}{#4}{#5}%
        {\globalsetters\advancecounter{math counter}{1}}%
        \def\curremphcolor{#5}%
        \hbox to\hsize{%
            \emphcolor\printmcount{} #2%
            \unless\ifx\_title\novalue%
                \ (\_title)%
            \fi%
            \unless\ifx\_name\novalue%
                \_xp\anchor\_xp{\_name}%
                \_xp\xdef\csname math@\_name\endcsname{\printmcount}%
                \unless\ifx\_title\novalue%
                    \_xp\xdef\csname math@\_name\endcsname{\_title}%
                    \addthreedepth{\_title}{\_name}%
                \fi%
            \fi%
        \hfil}%
        \hrule height\z@\relax%
        \medskip%
    }%
    \_xp\edef\csname b#1\endcsname{\noexpand\_ifnextchar[ {\_xp\noexpand\csname _b#1\endcsname}{\_xp\noexpand\csname _b#1\endcsname[]}}%
    \_xp\let\csname e#1\endcsname=\eppbox%
}

\def\bproof{%
    \blinedppbox{rgb{1 1 1}}{rgb{0 0 0}}{rgb{0 0 0}}
    \hbox to \hsize{\setfont{bf}Proof\hfil}
    \medskip
}
\let\eproof=\elinedppbox

\def\blankpp{%
    \blinedppbox{rgb{1 1 1}}{rgb{0 0 0}}{rgb{0 0 0}}
}
\let\eblankpp=\elinedppbox

\def\bnote{%
    \bppbox{rgb{1 1 .5}}{rgb{.5 .4 0}}{rgb{.5 .4 0}}
    \hbox to\hsize{\setfont{bf}Note\hfil}
    \medskip
}
\let\enote=\eppbox

\def\bexerc{%
    \bppbox{rgb{.9 1 .9}}{rgb{.3 .8 .3}}{rgb{.1 .65 .1}}
    \hbox to \hsize{\setfont{bf}Exercise\hfil}
    \def\curremphcolor{rgb{.1 .65 .1}}%
    \hrule height\z@%
    \medskip
}
\let\eexerc=\eppbox

\def\__refmath#1{\gotoanchor{#1}{\csname math@#1\endcsname}}
\def\_refmath[#1]#2{\gotoanchor{#2}{#1 \csname math@#2\endcsname}}
\def\refmath{\_ifnextchar[ {\_refmath}{\__refmath}}
\catcode`_=8

\createmathbox{defn}{Definition}{rgb{1 .9 .9}}{rgb{1 .3 .3}}{rgb{.8 .1 .1}}
\createmathbox{thrm}{Theorem}{rgb{.9 .9 1}}{rgb{.3 .3 1}}{rgb{.1 .1 .8}}
\createmathbox{coro}{Corollary}{rgb{.9 .9 1}}{rgb{.3 .3 1}}{rgb{.1 .1 .8}}
\createmathbox{lemm}{Lemma}{rgb{1 .9 1}}{rgb{1 .3 1}}{rgb{.8 .1 .8}}
\createmathbox{prop}{Proposition}{rgb{1 .9 1}}{rgb{1 .3 1}}{rgb{.6 .1 .6}}
\createmathbox{prin}{Principle}{rgb{1 1 .5}}{rgb{.5 .3 0}}{rgb{.5 .3 0}}
\createmathbox{exam}{Example}{rgb{.9 1 .9}}{rgb{.3 .8 .3}}{rgb{.1 .65 .1}}

\def\qed{%
    \ifmmode \eqno\mathchar"404%
    \else%
        \hskip1cm\penalty0\null\nobreak\hfill$\mathchar"404$%
        \par\medskip%
    \fi%
}

%\def\bdefn{%
%    \bppbox{rgb{1 .9 .9}}{rgb{1 .3 .3}}{rgb{.8 .1 .1}}
%    \advancecounter{math counter}{1}
%    \def\curremphcolor{rgb{.8 .1 0}}
%    \hbox{\emphcolor\the\counter{section}.\the\counter{subsection}.\the\counter{math counter} Definition}
%}
%
%\def\edefn{
%    \eppbox
%}

\newcount\mlcount
\def\gentzen#1#2{%
    \,\vcenter{%
        \tabskip=.15cm\relax%
        \mlcount=1\relax%
        \offinterlineskip%
        \halign{\strut\hfil${##}$\hfil&&\global\advance\mlcount by 1\relax\vrule\kern.15cm${##}$\cr%
            #1\cr\noalign{\kern.1\jot\hrule\kern1\jot}%
            \multispan{\the\mlcount}\hfil$#2$\hfil\cr
        }%
    }\,%
}

\footline={}

%%%%%%%%%%%%%%%%%%%%%%%%%%%%%%%%%%%%%%%%%%%%%%%%%%%%%%%%%%%%%%%%

\headline={\pageborder{rgb{1 1 .1}}{rgb{1 .4 0}}{5}}

\color rgb{.8 .1 0}

{\def\boxshadowcolor{rgb{.8 .8 0}}
\bppbox{rgb{1 1 .5}}{rgb{1 .4 0}}{rgb{.8 .1 0}}

    \centerline{\setfontandscale{bf}{20pt}Mathematical Logic}
    \smallskip
    \centerline{\setfont{it}A summary of ``A Concise Introduction to Mathematical Logic'', W. Rautenberg}
    \centerline{\setfont{it}Ari Feiglin}

\eppbox

\bigskip

\bppbox{rgb{1 1 .5}}{rgb{1 .4 0}}{rgb{.8 .1 0}}
\section*{Contents}

\tableofcontents
\eppbox

}

\vfill\break

\color{black}

%\pageborder{rgb{1 1 .1}}{rgb{1 .4 0}}{5}
%\null
%\vfill\break

\pageno=1
\newif\ifpageodd
\pageoddtrue
\headline={%
    \hbox to \hsize{\color{black}%
        \ifpageodd\hfil{\it\currsubsection\quad\bf\folio}\global\pageoddfalse%
        \else{\bf\folio\quad\it\currsubsection}\hfil\global\pageoddtrue\fi%
    }%
}

\section{Propositional Logic}

\subsection{Semantics of Propositional Logic}

\input propositional-logic/semantics

\subsection{Gentzen Calculi}

\input propositional-logic/gentzen-calculi

\subsection{Hilbert Calculi}

\input propositional-logic/hilbert-calculi

\vfill\break

\section{First Order Logic}

\subsection{Mathematical Structures}

\input first-order-logic/mathematical-structures

\subsection{Syntax of First-Order Languages}

\input first-order-logic/syntax-of-first-order-languages

\subsection{Semantics of First-Order Languages}

\input first-order-logic/semantics-of-first-order-languages

\subsection{General Validity and Logical Equivalence}

\input first-order-logic/general-validity-and-logical-equivalence

\subsection{Logical Consequence and Theories}

\input first-order-logic/logical-consequence-and-theories

\subsection{Explicit Definitions---Language Expansions}

\input first-order-logic/explicit-definitions-language-expansions

\vfill\break

\section{Complete Logical Calculi}

\subsection{A Calculus of Natural Deduction}

\input complete-logical-calculi/a-calculus-of-natural-deduction

\subsection{The Completeness Proof}

\input complete-logical-calculi/the-completeness-proof

\subsection{First Applications: Nonstandard Models}

\input complete-logical-calculi/nonstandard-models

\subsection{{\sf ZFC} and Skolem's Paradox}

\input complete-logical-calculi/zfc-and-skolems-paradox

\subsection{Enumerability and Decidability}

\input complete-logical-calculi/enumerability-and-decidability

\subsection{Complete Hilbert Calculi}

\input complete-logical-calculi/complete-hilbert-calculi

\subsection{First-Order Fragments}

\input complete-logical-calculi/first-order-fragments

\vfill\break

\section{Foundations of Logic Programming}

\subsection{Term Models and Herbrand's Theorem}

\input foundations-of-logic-programming/term-models-and-herbrands-theorem

\subsection{Horn Formulas}

\input foundations-of-logic-programming/horn-formulas

%\subsection{Propositional Resolution}
%
%\input foundations-of-logic-programming/propositional-resolution

\vfill\break

\section{Elements of Model Theory}

\subsection{Elementary Extensions}

\input elements-of-model-theory/elementary-extensions

\subsection{Complete and $\kappa$-Categorical Theories}

\input elements-of-model-theory/complete-and-categorical-theories

\subsection{Ehrenfeucht-Fra\"\i ss\'e Games}

\input elements-of-model-theory/ehrenfeucht-fraisse-games

\subsection{Embedding and Characterization Theorems}

\input elements-of-model-theory/embedding-and-characterization-theorems

\subsection{Model Completeness}

\input elements-of-model-theory/model-completeness.tex

\subsection{Quantifier Elimination}

\input elements-of-model-theory/quantifier-elimination

\subsection{Reduced Products and Ultraproducts}

\input elements-of-model-theory/reduced-products-and-ultraproducts

\vfill\break

\section{Incompleteness and Undecidability}

\subsection{Recursive and Primitive Recursive Functions}

\input incompleteness-and-undecidability/recursive-and-primitive-recursive-functions

\vfill\break

\parskip=\z@
\def\currsubsection{Index}

\index

\bye

