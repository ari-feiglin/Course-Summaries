We define a language $\lambda$-OCaml similar to untyped $\lambda$-calculus as follows:

\bdefn

    Terms in $\lambda$-OCaml are defined recursively as follows:
    \benum
        \item all variables are terms,
        \item if $\tt x$ is a variable and $\tt t$ a term, then \lcalc{fun x\to t} is a term,
        \item if $\tt t_1$ is a term, then \lcalc{t\scr1 t\scr2} is a term.
    \eenum

\edefn

This is obviously equivalent to untyped $\lambda$-calculus where instead of \lcalc{\lambda x.t} we write \lcalc{fun x\to t}.
We also define types:

\bdefn

    Suppose we have an infinite set of type variables, then a type is defined recursively as follows:
    \benum
        \item all type variables are types,
        \item if $\T$ and $\S$ are types, so is \lcalc{\T\to\S}.
    \eenum

\edefn

Similar to typed $\lambda$-calculus we define the {\it type relation} $\Gamma\vdash\lcalc{t:\T}$ where {\tt t} is a term, $\T$ is a type, and $\Gamma$ is a variable type set of which contains elements of the
form \lcalc{x:\S} for variables {\tt x} and types \S, such that every variable is given a single type.
It is a Gentzen calculus defined using the rules:
$$ \gentzen{\lcalc{x:\T}\in\Gamma}{\Gamma\vdash\lcalc{x:\T}}\eqno{(\hbox{\tensc O-Var})} $$
$$ \gentzen{\Gamma\vdash\lcalc{t\scr{12}:\T\scr1\to\T\scr2}&\Gamma\vdash\lcalc{t\scr1:\T\scr1}}{\Gamma\vdash\lcalc{t\scr{12}t\scr1:\T\scr2}} \eqno{(\hbox{\tensc O-App})} $$
$$ \gentzen{\Gamma\vdash\lcalc{x:\T}&\Gamma\vdash\lcalc{t:\S}}{\Gamma\vdash\lcalc{(fun x\to t):\T\to\S}} \eqno{(\hbox{\tensc O-Abs})} $$
Notice that this is similar to simply typed $\lambda$-calculus except for {\tensc O-Abs}, where instead of viewing what type has {\tt t} has under the assumption that {\tt x} has type \T, we give them both
a type under the plain assumptions in $\Gamma$.

\bdefn

    The problem of {\emphcolor type inference} is the problem of finding mapping between terms and types.
    Its input is a term $t$, and its output is a variable type set $\Gamma$ and a map $m$ between subterms of $t$ (including $t$) such that $\Gamma\vdash t'\colon m(t')$ for all subterms $t'$.

\edefn

We will solve this problem in three steps: $(1)$ creating a system of equations between types, $(2)$ solving the system, and $(3)$ converting the solution to the appropriate $\Gamma$ and $m$.

\bdefn

    A term $t$ is called {\emphcolor normalized} if for every two subterms \lcalc{t\scr1 = fun x\to t\scr{11}} and \lcalc{t\scr2 = fun y\to t\scr{22}}, {\tt x} and {\tt y} are distinct variables.

\edefn

By $\alpha$-equivalence, every term has an equivalent normalized term.

\bdefn

    Let $t$ be a term, let us define the set of equations $A_t$ as follows: for every subterm $t'$ correspond a unique type variable $\alpha$, then
    \benum
        \item if $\alpha$ and $\beta$ correspond to different occurrences of the same subterm, then $\alpha\eq\beta\in A_t$,
        \item suppose $t_1t_2$ is a subterm such that $\alpha$ is the variable of $t_1$, $\beta$ of $t_2$, and $\gamma$ of $t_1t_2$, then $\alpha\eq\beta\to\gamma\in A_t$,
        \item for every subterm \lcalc{fun x\to t'}, if $\alpha$ is the variable of $x$, $\beta$ of $t'$, and $\gamma$ of \lcalc{fun x\to t'}, then $\gamma\eq\alpha\to\beta\in A_t$.
    \eenum

\edefn

For example, let {\tt t} be \lcalc{(fun x\to x)y}, then let us map the subterms to type variables as follows:
$$ y\mapsto\alpha_y,\quad x\mapsto\alpha_x^1,\quad x\mapsto\alpha_x^2,\quad \lcalc{fun x\to x}\mapsto\alpha_f,\quad t\mapsto\alpha_t $$
Then
$$ A_t = \set{\alpha_x^1\eq\alpha_x^2,\;\alpha_f\eq\alpha^1_x\to\alpha^2_x,\;\alpha_f\eq\alpha_y\to\alpha_t} $$

Now that we have finished step $(1)$, we skip step $(2)$ and progress to step $(3)$.
We will return to step $(2)$ later.

\bdefn

    A {\emphcolor substitution} is a function $\sigma$ which maps between type variables such that $\sigma(\T_1\to\T_2)=\sigma(\T_1)\to\sigma(\T_2)$.
    $\sigma$ {\emphcolor preserves} an equality $\T_1\eq\T_2$ if $\sigma(\T_1)=\sigma(\T_2)$, and it preserves a set of equations if it preserves every equality in the set.

\edefn

So in the example above, one such substitution which preserves $A_t$ is $\sigma(\alpha_x^1)=\sigma(\alpha_x^2)=\sigma(\alpha_y)=\sigma(\alpha_t)=\alpha$, and $\sigma(\alpha_f)=\alpha\to\alpha$.o

\bdefn

    Let $t$ be a term and $\beta$ a function which maps subterms $t'$ to their type variables.
    Suppose $A_t$ is the resulting set of equations, and $\sigma$ a substitution which preserves it.
    Then we define
    $$ \Gamma^\beta_\sigma = \set{x\colon\sigma(\beta(x))}[x\in\var t] $$

\edefn

So using the above example where $\beta$ is the map
$$ \beta\colon\qquad y\mapsto\alpha_y,\quad x\mapsto\alpha_x^1,\quad x\mapsto\alpha_x^2,\quad \lcalc{fun x\to x}\mapsto\alpha_f,\quad t\mapsto\alpha_t $$
Then using the above substitution $\sigma$, we have that
$$ \Gamma^\beta_\sigma = \set{x\colon\alpha,\ y\colon\alpha} $$

\bthrm

    Let $t$ be a term, $\beta$ a correspondence between subterms and type variables, and $\sigma$ a substitution which preserves $A_t$.
    Then for every subterm $t'$ of $t$,
    $$ \Gamma^\beta_\sigma\vdash t'\colon\sigma(\beta(t')) $$
    Thus if we define $m\coloneqq\sigma\circ\beta$ and $\Gamma\coloneqq\Gamma^\beta_\sigma$, we have a solution to the problem of type inference for $t$.
    And if there is no $\sigma$ which preserves $A_t$ then there is no solution to the problem of type inference for $t$.

\ethrm

\Proof by induction on $t'$.
\benum
    \item If $t'$ is a variable then $t'\colon\sigma(\beta(t'))$ is in $\Gamma^\beta_\sigma$ and thus this follows from {\tensc O-Var}.
    \item If $t'$ is of the form $\fun x\to t''$, then let $\alpha_1,\alpha_2,\alpha_3$ be the types of $x,t'',t'$ respectively.
    Then $\alpha_3\eq\alpha_1\to\alpha_2$ is an equation in $A_t$ so $\sigma(\alpha_3)=\alpha(\alpha_1)\to\sigma(\alpha_2)$.
    In other words, $\sigma(\beta(t'))=\sigma(\beta(x))\to\sigma(\beta(t''))$.
    Now, by induction we have that
    $$ \Gamma^\beta_\sigma\vdash x\colon\sigma(\beta(x)),\quad \Gamma^\beta_\sigma\vdash t''\colon\sigma(\beta(t'')) $$
    So applying {\tensc O-Abs} yields
    $$ \Gamma^\beta_\sigma\vdash t'\colon\sigma(\beta(x))\to\sigma(\beta(t'')) = \sigma(\beta(t')) $$
    as required.
    \item if $t'=t_1t_2$ then this follows similarly to the above case.
\eenum

If $m$ solves the problem of type inference, then define $\sigma=m\circ\beta^{-1}$ and we claim that this is a substitution which preserves $A_t$.
We split into cases by the type of equations in $A_t$:
\benum
    \item Equations arising from different occurrences of the same subterm and so $m$ will map this term to the same type, independent of the occurrence.
    \item Equations arising from $t'=\fun x\to t''$, then this follows from {\tensc O-Abs}.
    \item Equations arising from $t_1t_2$ follows from {\tensc O-App}.
    \qed
\eenum

So to solve step $(2)$ all we must do is find a suitable substitution.
This is called the problem of {\it unification}: given a set of equations of terms (since $\beta$ is a correspondence, we can view the equation of type variables as equations of terms) finding a
substitution which preserves it.

