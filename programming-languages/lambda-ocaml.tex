We define a language $\lambda$-OCaml similar to untyped $\lambda$-calculus as follows:

\bdefn

    Terms in $\lambda$-OCaml are defined recursively as follows:
    \benum
        \item all variables are terms,
        \item if $\tt x$ is a variable and $\tt t$ a term, then \lcalc{fun x\to t} is a term,
        \item if $\tt t_1$ is a term, then \lcalc{t\scr1 t\scr2} is a term.
    \eenum

\edefn

This is obviously equivalent to untyped $\lambda$-calculus where instead of \lcalc{\lambda x.t} we write \lcalc{fun x\to t}.
We also define types:

\bdefn

    Suppose we have an infinite set of type variables, then a type is defined recursively as follows:
    \benum
        \item all type variables are types,
        \item if $\T$ and $\S$ are types, so is \lcalc{\T\to\S}.
    \eenum

\edefn

Similar to typed $\lambda$-calculus we define the {\it type relation} $\Gamma\vdash\lcalc{t:\T}$ where {\tt t} is a term, $\T$ is a type, and $\Gamma$ is a variable type set of which contains elements of the
form \lcalc{x:\S} for variables {\tt x} and types \S, such that every variable is given a single type.
It is a Gentzen calculus defined using the rules:
$$ \gentzen{\lcalc{x:\T}\in\Gamma}{\Gamma\vdash\lcalc{x:\T}}\eqno{(\hbox{\tensc O-Var})} $$
$$ \gentzen{\Gamma\vdash\lcalc{t\scr{12}:\T\scr1\to\T\scr2}&\Gamma\vdash\lcalc{t\scr1:\T\scr1}}{\Gamma\vdash\lcalc{t\scr{12}t\scr1:\T\scr2}} \eqno{(\hbox{\tensc O-App})} $$
$$ \gentzen{\Gamma\vdash\lcalc{x:\T}&\Gamma\vdash\lcalc{t:\S}}{\Gamma\vdash\lcalc{(fun x\to t):\T\to\S}} \eqno{(\hbox{\tensc O-Abs})} $$
Notice that this is similar to simply typed $\lambda$-calculus except for {\tensc O-Abs}, where instead of viewing what type has {\tt t} has under the assumption that {\tt x} has type \T, we give them both
a type under the plain assumptions in $\Gamma$.

\bdefn

    The problem of {\emphcolor type inference} is the problem of finding mapping between terms and types.
    Its input is a term $t$, and its output is a variable type set $\Gamma$ and a map $m$ between subterms of $t$ (including $t$) such that $\Gamma\vdash t'\colon m(t')$ for all subterms $t'$.

\edefn

We will solve this problem in three steps: $(1)$ creating a system of equations between types, $(2)$ solving the system, and $(3)$ converting the solution to the appropriate $\Gamma$ and $m$.

\bdefn

    A term $t$ is called {\emphcolor normalized} if for every two subterms \lcalc{t\scr1 = fun x\to t\scr{11}} and \lcalc{t\scr2 = fun y\to t\scr{22}}, {\tt x} and {\tt y} are distinct variables.

\edefn

By $\alpha$-equivalence, every term has an equivalent normalized term.

\bdefn

    Let $t$ be a term, let us define the set of equations $A_t$ as follows: for every subterm $t'$ correspond a unique type variable $\alpha$, then
    \benum
        \item if $\alpha$ and $\beta$ correspond to different occurrences of the same subterm, then $\alpha\eq\beta\in A_t$,
        \item suppose $t_1t_2$ is a subterm such that $\alpha$ is the variable of $t_1$, $\beta$ of $t_2$, and $\gamma$ of $t_1t_2$, then $\alpha\eq\beta\to\gamma\in A_t$,
        \item for every subterm \lcalc{fun x\to t'}, if $\alpha$ is the variable of $x$, $\beta$ of $t'$, and $\gamma$ of \lcalc{fun x\to t'}, then $\gamma\eq\alpha\to\beta\in A_t$.
    \eenum

\edefn

For example, let {\tt t} be \lcalc{(fun x\to x)y}, then let us map the subterms to type variables as follows:
$$ y\mapsto\alpha_y,\quad x\mapsto\alpha_x^1,\quad x\mapsto\alpha_x^2,\quad \lcalc{fun x\to x}\mapsto\alpha_f,\quad t\mapsto\alpha_t $$
Then
$$ A_t = \set{\alpha_x^1\eq\alpha_x^2,\;\alpha_f\eq\alpha^1_x\to\alpha^2_x,\;\alpha_f\eq\alpha_y\to\alpha_t} $$

Now that we have finished step $(1)$, we skip step $(2)$ and progress to step $(3)$.
We will return to step $(2)$ later.

\bdefn

    A {\emphcolor substitution} is a function $\sigma$ which maps between type terms such that $\sigma(\T_1\to\T_2)=\sigma(\T_1)\to\sigma(\T_2)$.
    $\sigma$ {\emphcolor preserves} an equality $\T_1\eq\T_2$ if $\sigma(\T_1)=\sigma(\T_2)$, and it preserves a set of equations if it preserves every equality in the set.

\edefn

So in the example above, one such substitution which preserves $A_t$ is $\sigma(\alpha_x^1)=\sigma(\alpha_x^2)=\sigma(\alpha_y)=\sigma(\alpha_t)=\alpha$, and $\sigma(\alpha_f)=\alpha\to\alpha$.o

\bdefn

    Let $t$ be a term and $\beta$ a function which maps subterms $t'$ to their type variables.
    Suppose $A_t$ is the resulting set of equations, and $\sigma$ a substitution which preserves it.
    Then we define
    $$ \Gamma^\beta_\sigma = \set{x\colon\sigma(\beta(x))}[x\in\var t] $$

\edefn

So using the above example where $\beta$ is the map
$$ \beta\colon\qquad y\mapsto\alpha_y,\quad x\mapsto\alpha_x^1,\quad x\mapsto\alpha_x^2,\quad \lcalc{fun x\to x}\mapsto\alpha_f,\quad t\mapsto\alpha_t $$
Then using the above substitution $\sigma$, we have that
$$ \Gamma^\beta_\sigma = \set{x\colon\alpha,\ y\colon\alpha} $$

\bthrm

    Let $t$ be a term, $\beta$ a correspondence between subterms and type variables, and $\sigma$ a substitution which preserves $A_t$.
    Then for every subterm $t'$ of $t$,
    $$ \Gamma^\beta_\sigma\vdash t'\colon\sigma(\beta(t')) $$
    Thus if we define $m\coloneqq\sigma\circ\beta$ and $\Gamma\coloneqq\Gamma^\beta_\sigma$, we have a solution to the problem of type inference for $t$.
    And if there is no $\sigma$ which preserves $A_t$ then there is no solution to the problem of type inference for $t$.

\ethrm

\Proof by induction on $t'$.
\benum
    \item If $t'$ is a variable then $t'\colon\sigma(\beta(t'))$ is in $\Gamma^\beta_\sigma$ and thus this follows from {\tensc O-Var}.
    \item If $t'$ is of the form $\fun x\to t''$, then let $\alpha_1,\alpha_2,\alpha_3$ be the types of $x,t'',t'$ respectively.
    Then $\alpha_3\eq\alpha_1\to\alpha_2$ is an equation in $A_t$ so $\sigma(\alpha_3)=\alpha(\alpha_1)\to\sigma(\alpha_2)$.
    In other words, $\sigma(\beta(t'))=\sigma(\beta(x))\to\sigma(\beta(t''))$.
    Now, by induction we have that
    $$ \Gamma^\beta_\sigma\vdash x\colon\sigma(\beta(x)),\quad \Gamma^\beta_\sigma\vdash t''\colon\sigma(\beta(t'')) $$
    So applying {\tensc O-Abs} yields
    $$ \Gamma^\beta_\sigma\vdash t'\colon\sigma(\beta(x))\to\sigma(\beta(t'')) = \sigma(\beta(t')) $$
    as required.
    \item if $t'=t_1t_2$ then this follows similarly to the above case.
\eenum

If $m$ solves the problem of type inference, then define $\sigma=m\circ\beta^{-1}$ and we claim that this is a substitution which preserves $A_t$.
We split into cases by the type of equations in $A_t$:
\benum
    \item Equations arising from different occurrences of the same subterm and so $m$ will map this term to the same type, independent of the occurrence.
    \item Equations arising from $t'=\fun x\to t''$, then this follows from {\tensc O-Abs}.
    \item Equations arising from $t_1t_2$ follows from {\tensc O-App}.
    \qed
\eenum

So to solve step $(2)$ all we must do is find a suitable substitution.
This is called the problem of {\it unification}: given a set of equations of type variables, we must find a substitution which preserves it.

The unification algorithm, due to Hindley-Milner, functions as follows (its code written in OCaml):

\beginhi \color{white} \globcounttrue
type id = string

type term =
    | Var of id
    | Term of id * (term list)

type substitution = (id * term) list
\endhi

Let us take a quick second to understand the types here.
Firstly {\tt id} is simply an alias for {\tt string}.
{\tt term}s (denoted $\tau$) are type terms, whose formal definition is:
$$ \tau \ccoloneqq \alpha\ |\ C\tau\dots\tau $$
$\alpha$ is the set of type variables, and $C$ is a set of {\it type functions}.
So for example $C$ may contain the $0$-ary {\tt int} and {\tt string} which represent types of integers and strings respectively.
Another example is {\tt Map} which is a $2$-ary type function: \lcalc{Map string int} is a hashmap from {\tt string}s to {\tt int}s.
Yet another example is {\tt Set}: \lcalc{Set int} is a set of {\tt int}s.
In this $C$, we can chain type functions, so for example \lcalc{Map (Set string) int} is a type term.

Hindley-Milner imposes only the restriction that $C$ must include the type function $\to$ which represents a function.
Instead of $\to \tau \tau'$ though, we write $\tau\to\tau'$ (use infix instead of polish notation).

A {\tt substitution} is simply a map from {\tt Var}s to terms.

\beginhi \color{white} \globcounttrue
let rec occurs (x : id) (t : term) : bool =
    match t with
    | Var y -> x = y
    | Term(_,s) -> List.exists (occurs x) s
\endhi

{\tt occurs} takes a variable $x$ and a term $t$ and checks if $x$ occurs in $t$. 
It does so as follows: if $t$ is a variable $y$, then return if $x=y$.
Otherwise $t=Ct_1\dots t_n$, so return if $x$ occurs in any $t_i$.

\beginhi \color{white} \globcounttrue
let rec subst (s : term) (x : id) (t : term) : term =
    match t with
    | Var y -> if x = y then s else t
    | Term(f,u) -> Term(f, List.map (subst s x) u)
\endhi

\lcalc{subst s x t} corresponds to the substitution \lcalc{t[x\mapsto s]}.
So recursively, if $t$ is the variable $y$, if $x=y$ then $t[x\mapsto s]=s$ otherwise $t$ remains the same.
And
$$ (Ct_1\dots t_n)[x\mapsto s] = C(t_1[x\mapsto s])\dots(t_n[x\mapsto s]) $$

\beginhi \color{white} \globcounttrue
let apply (s : substitution) (t : term) : term =
    List.fold_right (fun (x,u) -> subst u x) s t
\endhi

Here, if $s=[(x_1,t_1),\dots,(x_n,t_n)]$ we want to apply $s$ to a term $t$.
Then what we want in the end is to get $t[x_n\mapsto t_n]\cdots[x_1\mapsto t_1]$ which is what this does.

Now we get to the meat of the algorithm: the code which actually unifies the a list of equations.
First we have the function {\tt unify\_one} which unifies a single equation $s\eq t$.
Then using this we define {\tt unify}.

\beginhi \color{white} \globcounttrue
let rec unify_one (s : term) (t : term) : substitution =
    match (s,t) with
    | (Var x, Var y) -> if x = y then [] else [(x,t)]
    | (Term(f,sc), Term(g,tc)) ->
        if f = g && List.length sc = List.length tc
        then unify (List.combine sc tc)
        else failwith "not unifiable: head symbol conflict"
    | (Var x, Term(_,_) as t) | (Term(_,_) as t, Var x) ->
        if occurs x t
        then failwith "not unifiable: circularity"
        else: [(x,t)]

and unify (e : (term * term) list) : substitution =
    match e with
    | [] -> []
    | (x,y) :: t ->
        let s2 = unify t in
        let s1 = unify_one (apply s2 x) (apply s2 y) in
        s1 @ s2
\endhi

So the algorithm is as follows: given a list of equivalences {\tt e=(x,y)::t} first unify the equivalences in {\tt t} recursively to get a substitution {\tt s2}.
Apply this substitution to {\tt x} and {\tt y} and unify them.

To unify a single equivalence $s\eq t$, we split into cases:
\benum
    \item if both $s=x$ and $t=y$ are variables, then if they are the same variable no unification is needed.
    Otherwise we simply have $x$ be substituted with $t$.
    \item If $s=Cs_1\dots s_n$ and $t=C't_1\dots t_m$, we must have that $C=C'$ and $n=m$ (we cannot unify \lcalc{Set $\alpha$} with \lcalc{Map int $\beta$} for example).
    If such is the case, we unify the list $[(s_1,t_1),\dots,(s_n,t_n)]$ recursively, since we now have the equivalences $s_i\eq t_i$.
    \item If $s=x$ is a variable and $t=Ct_1\dots t_n$ is a compound term (or vice versa), then we cannot have that $x$ occurs in $t$ (we cannot unify \lcalc{Set $\alpha$} with $\alpha$ for example).
    If such is the case (that $x$ does not occur in $t$), then we simply have that $x$ is substituted with $t$.
\eenum

Notice that the same equivalence can have multiple unifiers.
For example the equivalence between $f\ x\ (g\ y)$ and $f\ (g\ z)\ w$ has the unifiers: we can take $S=[x\mapsto g\ z,\, w\mapsto g\ y]$.
Applying this to both yields $f\ (g\ z)\ (g\ y)$.
But we can also have the unifier $T=[x\mapsto g(f\ a\ b),\, y\mapsto f\ b\ a,\, z\mapsto f\ a\ b,\, w\mapsto g(f\ b\ a)]$.
Both terms then substitute out to $f\ (g\ (f\ a\ b))\ (g\ (f\ b\ a))$.

\bdefn

    The {\emphcolor most general unifier} (mgu) for a set of equations $A_t$ is a unifier $\sigma$ such that for every other unifier $\sigma'$ there exists a substitution $\sigma''$ such that
    $\sigma'=\sigma\sigma''$.
    Meaning that all other unifiers can be obtained from $\sigma$ using further substitutions.

\edefn

In the above example, $S$ is an mgu and $T=S[z\mapsto f\ a\ b,\,y\mapsto f\ b\ a]$.
Hindlery-Milner's algorithm returns an mgu for the set of equivalences.

