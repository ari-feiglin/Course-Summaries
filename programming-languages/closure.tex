\bdefn

    An {\emphcolor environment} is a list $(x_1:v_1)\cc\cdots\cc(x_n:v_n)$ where $x_i$ is a variable and $v_i$ is a value (in whatever language, so a number, boolean, function, etc).
    Given an environment $E$, we define $E(x)$ to be the value given to $x$ by $E$ (ie. $E$ is of the form $\cdots\cc(x:E(x))\cc\cdots$).

\edefn

We define a ternary relation between environments, expressions, and values:
$$ E\vdash e\To v $$
which is to be read as ``the expression $e$ has value $v$ in environment $E$''.
We define this using the following Gentzen-style rules:

\centerline{\vbox{\everycr={\noalign{\kern3pt}}
\halign{\hfil$#$\tabskip=2cm&$#$\hfil\tabskip=\z@\crcr
    \gentzen{}{E\vdash v\To v}\ (\hbox{\tensc Val}) &
    \gentzen{E(x) = v}{E\vdash x\To v}\ (\hbox{\tensc Var})\cr
    \multispan2\hfil$\gentzen{E\vdash e_1\To v_1&\cdots&E\vdash e_n\To v_n}{E\vdash(e_1,\dots,e_n)\To(v_1,\dots,v_n)}\ (\hbox{\tensc N-Tuple})$\hfil
    \cr
    \gentzen{\quad E\vdash e_1\To{\sf true} & E\vdash e_2\To v\quad}{E\vdash{\tt if}\ e_1\ {\tt then}\ e_2\ {\tt else}\ e_3\To v}\ (\hbox{\tensc Cond1})&
    \gentzen{\quad E\vdash e_1\To{\sf false} & E\vdash e_3\To v\quad}{E\vdash{\tt if}\ e_1\ {\tt then}\ e_2\ {\tt else}\ e_3\To v}\ (\hbox{\tensc Cond2})
    \cr
}}}

These should all be pretty self-explanatory rules.

Now, for a function we must somehow give it a value which captures all the values in the environment, thus we create new values of the form $\gen{E,f}$ where $E$ is an environment and $f$ is a function.
This is called the {\it closure} of $f$ in $E$.
We continue developing rules for $\vdash$:

$$ \displaylines{
    \gentzen{}{E\vdash(\fun x\to e)\To\gen{E,(\fun x\to e)}}\ (\hbox{\tensc Fun1})\cr\noalign{\kern3pt}
    \gentzen{}{E\vdash(\fun(x_1,\dots,x_n)\to e)\To\gen{E,(\fun(x_1,\dots,x_n)\to e)}}\ (\hbox{\tensc Fun2})\cr\noalign{\kern3pt}
} $$
So we give to a function $f$ the value of its closure $\gen{E,f}$ in the environment $E$.
$$ \displaylines{
  \gentzen{E\vdash e_1\To\gen{E',(\fun x\to e)} & E\vdash e_2\To v' & (x:v')\cc E'\vdash e\To v}{E\vdash(e_1\ e_2)\To v}\ (\hbox{\tensc App1})\cr\noalign{\kern3pt}
  \gentzen{E\vdash e_1\To\gen{E',(\fun(x_1,\dots,x_n)\to e)} & E\vdash e_2\To (v_1,\dots,v_n) & (x:v_1)\cc\cdots\cc(x:v_n)\cc E'\vdash e\To v}{E\vdash(e_1\ e_2)\To v}
  \ (\hbox{\tensc App1})\cr\noalign{\kern3pt}
} $$

Now recall that the syntax for setting a variable is $\Let x=e_1\In{e_2}$
So the rule for {\tt let} is that we just add $(x:e_1)$ to the environment:
$$ \gentzen{E\vdash e_1\To v_1 & (x:v_1)\cc E\vdash e_2\To v}{E\vdash\Let{x=e_1}\In{e_2}\To v}\ (\hbox{\tensc Let}) $$
We also have {\tt let rec} whose syntax is $\Letrec f\ x=e_1\In{e_2}$.
Now the idea is that ${\tt let\ rec}f\ x=e$ will be given the closure $\gen{E,(\fun x\to e)}$ where $E$ contains an infinite pair $f:\gen{E,(\fun x\to e)}$.
Such an object can be represented in memory as an object with a pointer which points to itself.
So given an environment $E$, a function name $f$, and an expression $(\fun x\to e)$ (where importantly $e$ may contain occurrences of $f$), we assumed we can construct an environment $E'$ such that
$E'=(f:\gen{E',(\fun x\to e)})\cc E$, and then the rule is:
$$ \gentzen{E'=(f:\gen{E',(\fun x\to e)})\cc E \vdash e_2\To v}{E\vdash\Letrec f\ x=e_1\In e_2\To v}\ (\hbox{\setfont{sc}Letrec}) $$

Before giving an example, let us define some boolean and arithmetic rules:

\centerline{\vbox{\everycr={\noalign{\kern3pt}}
\halign{\hfil$#$\tabskip=2cm&$#$\hfil\tabskip=\z@\crcr
    \gentzen{E\vdash e_1\To v_1 & E\vdash e_2\To v_2 & v_1\leq v_2}{E\vdash e_1\leq e_2\To{\tt true}}\ (\hbox{\setfont{sc}Leq1}) &
    \gentzen{E\vdash e_1\To v_1 & E\vdash e_2\To v_2 & v_1>v_2}{E\vdash e_1\leq e_2\To{\tt false}}\ (\hbox{\setfont{sc}Leq2})\cr
    \gentzen{E\vdash e_1\To v_1 & E\vdash e_2\To v_2 & v_1=v_2}{E\vdash e_1=e_2\To{\tt true}}\ (\hbox{\setfont{sc}Eq1})&
    \gentzen{E\vdash e_1\To v_1 & E\vdash e_2\To v_2 & v_1\neq v_2}{E\vdash e_1=e_2\To{\tt false}}\ (\hbox{\setfont{sc}Eq2})\cr
    \gentzen{E\vdash e_1\To v_1 & E\vdash e_2\To v_2}{E\vdash e_1+e_2\To v_1+v_2}\ (\hbox{\setfont{sc}Add})&
    \gentzen{E\vdash e_1\To v_1 & E\vdash e_2\To v_2}{E\vdash e_1-e_2\To v_1-v_2}\ (\hbox{\setfont{sc}Sub})\cr
    \gentzen{E\vdash e_1\To v_1 & E\vdash e_2\To v_2}{E\vdash e_1\cdot e_2\To v_1\cdot v_2}\ (\hbox{\setfont{sc}Mul})&
    \gentzen{E\vdash e_1\To v_1 & E\vdash e_2\To v_2}{E\vdash e_1/e_2\To v_1/v_2}\ (\hbox{\setfont{sc}Div})\cr
}}}

So for example, suppose our initial environment is the empty list, and we'd like to compute the value of
$$ \Letrec f\ x = (\If x=1\Then 1\Else x\cdot f(x-1))\In{(f\ 2)} $$
Which should give $2!=2$.
Let us write ${\tt fact}$ in place of $\fun x\to(\If x=0\Then 1\Else x\cdot f(x-1))$, so
\vfill\break

{
\setlayout{page width=60cm, page height=20cm, horizontal margin=2cm, vertical margin=2cm}
\headline={}
\vbox to\vsize{\vss\centerline{\syntaxtree{
    {[]\vdash\Letrec f\ x = (\If x=1\Then 1\Else x\cdot f(x-1))\In{(f\ 2)}\To2}[\setfont{sc}Letrec]{
        {E=(f:\gen{E,{\tt fact}})\vdash(f\ 2)\To2}[\setfont{sc}App1]{
            {E\vdash f\To\gen{E,{\tt fact}}}[\setfont{sc}Var]\noc
            {E\vdash2\To2}[\setfont{sc}Val]\noc
            {E'=(x:2)\cc E\vdash(\If x=0\Then 1\Else x\cdot f(x-1))}[\setfont{sc}Cond2]{
                {E'\vdash x=1\To{\tt false}}[\setfont{sc}Eq2]{
                    {E'\vdash x\To2}[\setfont{sc}Var]\noc
                    {E'\vdash1\To1}[\setfont{sc}Val]\noc
                }
                {E'\vdash x\cdot f(x-1)\To2}[\setfont{sc}Mul]{
                    {E'\vdash x\To2}[\setfont{sc}Var]\noc
                    {E'\vdash f(x-1)\To1}[\setfont{sc}App1]{
                        {E'\vdash f\To{\tt fact}}[\setfont{sc}Var]\noc
                        {E'\vdash x-1\To1}[\setfont{sc}Sub]{
                            {E'\vdash x\To2}[\setfont{sc}Var]\noc
                            {E'\vdash 1\To1}[\setfont{sc}Val]\noc
                        }
                        {E''=(x:1)\cc E'\vdash(\If x=1\Then 1\Else x\cdot f(x-1))\To1}[\setfont{sc}Cond1]{
                            {E''\vdash x=1\To{\tt true}}[\setfont{sc}Eq]{
                                {E''\vdash x\To1}[\setfont{sc}Var]\noc
                                {E''\vdash 1\To1}[\setfont{sc}Val]\noc
                            }
                            {E''\vdash1\To1}[\setfont{sc}Val]\noc
                        }
                    }
                }
            }
        }
    }
}}\vss}
\vfill\break
}

