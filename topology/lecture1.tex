\documentclass[10pt]{article}

\usepackage{amsmath, amssymb, mathtools}
\usepackage[margin=1.5cm]{geometry}

\input pdfmsym
\input prettyprint
\input preamble

\pdfmsymsetscalefactor{10}
\initpps

\def\pmat#1{\begin{pmatrix} #1 \end{pmatrix}}

\def\bS{\mathbb{S}}
\def\mU{\mathcal{U}}
\def\mV{\mathcal{V}}
\def\mO{\mathcal{O}}

\newfunc{metric}{\rho}({})
\newfunc{metricA}{\sigma}({})

\let\divides=\mid
\let\to=\varrightarrow
\def\implies{\,\longvarRightarrow\,}

\font\bigbf = cmbx12 scaled 2000

\pdf@drawing@macro{check}{0 1 0 RG 1.2 w 1 j 1 J 0 5 m 3 0 l 8 10 l S}{9.2pt}{12pt}{1pt}{1pt}
\pdf@drawing@macro{ecks}{1 0 0 RG 0 0 m 1 j 1 J 10 10 l 0 10 m 10 0 l S}{11.2pt}{12pt}{1pt}{1pt}

\begin{document}

\c@section=1

\barcolorbox{255, 220, 255}{130, 0, 130}{200, 80, 200}{
    \leftskip=0pt plus 1fill \rightskip=\leftskip
    {\bigbf Topology}

    \medskip
    \textit{Lecture 1, Sunday March 19, 2022}

    \textit{Ari Feiglin}
}

\bigskip

\subsection{Metric Spaces}

We begin by studying \emph{metric spaces} (again).
These were also covered in Infinitesimal Calculus $3$, whose summaries can be found on my github as well.

\begin{defn*}

    A \ppemph{metric space} is a pair $(M, \metric)$ where $\metric$ is a function:
    \[ \metric\colon M\times M\longvarrightarrow\bR_{\geq0} \]
    which satisfies the following three properties:
    \benum
        \item For every $x\in M$, $\metricof{x,x}=0$.
        \item For every $x,y\in M$, $\metricof{x,y}=\metricof{y,x}$ (symmetry).
        \item For every $x,y,z\in M$, $\metricof{x,y}\leq\metricof{x,z}+\metricof{z,y}$ (the triangle inequality).
    \eenum
    $\metric$ is called a \ppemph{metric function} on $M$.
    As an abuse of notation, if the metric is understood, we commonly refer to $M$ as the metric space.

\end{defn*}

\begin{exam*}

    If $M$ is a set, we define the \ppemph{discrete metric} on $M$ as:
    \[ \metricof{x,y} = \begin{cases} 1 & x\neq y \\ 0 & x=y \end{cases} \]
    This obviously satisfies the first and second axioms of a metric, and the triangle inequality is simple to show.

\end{exam*}

\begin{exam*}

    If $M$ is a normed linear space, then we can define
    \[ \metricof{x,y} = \norm{x-y} \]
    Showing this is indeed a metric is trivial, the first two conditions are obvious and the third follows from the norm's own triangle inequality.

    In the case that $M=\bR^n$ and the norm is the euclidean norm, this is called the \ppemph{euclidean metric}.

    In the case that $M=\bR^n$ and the norm is the maximum norm, ie $\norm{\bigl(x_1,\dots,x_n\bigr)}=\max_{1\leq i\leq n}\abs{x_i}$, the induced metric is called the \ppemph{maximum metric}.
    We will show (for the fun of it), that the maximum norm is indeed a norm.
    It is obvious that the maximum norm is non-negative, and is equal to $0$ only if the maximum $\abs{x_i}$ is $0$, and since it's the maximum this means every $\abs{x_i}=0$ so $x_i=0$ for every $i$,
    and therefore the vector is the zero vector.
    Scalar multiplication multiplies each component by that scalar, and so multiplies the maximum norm by the absolute value of the scalar.
    To show the triangle inequality:
    \[ \norm{x+y}_{\max} = \max_{1\leq i\leq n}\abs{x_i+y_i} \leq \max_{1\leq i\leq n}\bigl(\abs{x_i} + \abs{y_i}\bigr) \leq \max_{1\leq i\leq n}\abs{x_i} + \max_{1\leq i\leq n}\abs{y_i} =
    \norm x_{\max} + \norm y_{\max} \]
    as required.

\end{exam*}

\begin{defn*}

    If $(M,\metric)$ is a metric space, and $A\subseteq M$ then we \ppemph{inherited metric} of $A$ to be the restriction of $\metric$ to $A\times A$.
    This makes $\Bigl(A,\metric\bigl|_{A\times A}\Bigr)$ a metric space.
    As another abuse of notation we commonly write this as just $(A,\metric)$.

\end{defn*}

\begin{defn*}

    A sequence $\set{x_n}_{n=1}^\infty\subseteq M$ in a metric space $(M,\metric)$ \ppemph{converges} to a point $x\in M$ if
    \[ \lim_{n\to\infty}\metricof{x_n,x} = 0 \]
    or in other words, for every $\epsilon>0$ there is an $N\in\bN$ such that for every $n\geq N$, $\metricof{x_n,x}<\epsilon$.
    This is denoted as one of the following:
    \[ \lim_{n\to\infty}x_n = x,\quad \lim x_n = x,\quad x_n\xvarrightarrow{}[n\to\infty] x,\quad x_n\longvarrightarrow x \]

\end{defn*}

\begin{prop*}

    A sequence $\set{x_n}_{n=1}^\infty$ in a metric space $(M,\metric)$ converges to at most one point in $M$.

\end{prop*}

\begin{proof}

    Suppose it converges to both $x$ and $y$ in $M$.
    Then by the triangle inequality, for every $n$:
    \[ \metricof{x,y} \leq \metricof{x,x_n} + \metricof{y,x_n} \]
    and by taking the limit of the right-hand side (these are real numbers after all), we get that:
    \[ 0\leq\metricof{x,y}\leq0 \implies \metricof{x,y}=0 \implies x=y \]
    So the sequence converges to only one value.

    \hfill$\blacksquare$

\end{proof}

\begin{defn*}

    A \ppemph{Cauchy Sequence} in a metric space $(M,\metric)$ is a sequence $\set{x_n}_{n=1}^\infty\in M$ such that for every $\epsilon>0$ there is an $N$ such that for every $n,m\geq N$:
    \[ \metricof{x_n, x_m} < \epsilon \]

\end{defn*}

\begin{prop*}

    Every convergent sequence is Cauchy.

\end{prop*}

\begin{proof}

    Suppose $x_n\longvarrightarrow x$ in $M$.
    Then for every $\epsilon>0$ there is an $N$ such that for every $n\geq N$, $\metricof{x,x_n}<\frac\epsilon2$.
    So for every $n,m\geq N$:
    \[ \metricof{x_n, x_m} \leq \metricof{x_n,x} + \metricof{x_m,x} < \epsilon \]
    so the sequence is Cauchy, as required.

    \hfill$\blacksquare$

\end{proof}

\begin{exam*}

    Not every Cauchy sequence is convergent.
    For example take the sequence $x_n=\frac1n$ in $(0,1]\subset\bR$.
    Since $x_n$ is convergent in $\bR$, it is Cauchy in $\bR$, and since $(0,1]$ is a submetric space of $\bR$'s, $x_n$ is Cauchy in $(0,1]$ as well.
    But we know it converges to $0$ in $\bR$, and so if it were to converge in $(0,1]$ it would have to converge to $0$ by uniqueness of the limit.
    But since $0\notin(0,1]$, the sequence is Cauchy but does not converge.

\end{exam*}

\begin{defn*}

    A metric space where every Cauchy sequence is convergent is called \ppemph{complete}.

\end{defn*}

\begin{defn*}

    Suppose $M$ is a metric space, $x\in M$, and $r>0$ is a positive real number.
    Then the \ppemph{open ball} of radius $r$ centered around $x$ is defined to be:
    \[ B_r(x) = \set{a\in M}[\metricof{x,a}<r] \]
    an alternative notation for this is $B(x,r)$.

\end{defn*}

\begin{defn*}

    Suppose $M$ is a metric space.
    A set $\mU\subseteq M$ is called an \ppemph{open set} ($\mU$ is open) if for every $x\in\mU$ there exists an $r>0$ such that $B_r(x)\subseteq\mU$.

\end{defn*}

\begin{prop*}

    An open ball is indeed open.

\end{prop*}

\begin{proof}

    Let our open ball be $B_r(x)$.
    Let $a\in B_r(x)$ then define $\epsilon=r-\metricof{x,a}$.
    Since $a\in B_r(x)$, $\metricof{x,a}<r$ and so $\epsilon>0$, so we now can claim $B_\epsilon(a)\subseteq B_r(x)$.
    Suppose $b\in B_\epsilon(a)$ then
    \[ \metricof{b,x}\leq\metricof{b,a}+\metricof{a,x}<\epsilon+\metricof{x,a}=r-\metricof{x,a}+\metricof{x,a} = r \]
    So $b\in B_r(x)$ as required.

    So for every $a\in B_r(x)$, there is a $B_\epsilon(a)\subseteq B_r(x)$, so $B_r(x)$ is open as required.

    \hfill$\blacksquare$

\end{proof}

Furthermore, notice that $\varnothing$ is also open (vaccuously) and so is $M$, the metric space itself (trivially).
We generalize this in the following theorem:

\begin{thrm*}

    If $(M,\metric)$ is a metric space then:
    \benum
        \item $\varnothing$ and $M$ are both open.
        \item If $\Lambda$ is an arbitrary indexing set (an arbitrary set, it may be infinite of any cardinality) and $\set{\mU_\lambda}_{\lambda\in\Lambda}$ is a set of open sets, then
        $\bigcup_{\lambda\in\Lambda}\mU_\lambda$ is also open.
        \item If $\set{\mU_k}_{k=1}^n$ is a finite set of open sets, then so is $\bigcap_{k=1}^n\mU_k$
    \eenum

\end{thrm*}

\begin{proof}

    \benum
        \item As stated above, $\varnothing$ being open is vaccuously true since it has no points.
        And since for every $x\in M$, $B_r(x)\subseteq M$ for every $r>0$, $M$ is also open.
        \item Let $x\in\bigcup_{\lambda\in\Lambda}\mU_\lambda$, then there exists a $\lambda\in\Lambda$ such that $x\in\mU_\lambda$.
        Since $\mU_\lambda$ is open, there exists an $r>0$ such that $B_r(x)\subseteq\mU\subseteq\bigcup_{\lambda\in\Lambda}\mU_\lambda$, and since this is true for every $x$ in the union, the union is
        open as required.
        \item Let $x\in\bigcap_{k=1}^n\mU_k$, then for every $1\leq k\leq n$ there exists a $r_k>0$ such that $B_{r_k}(x)\subseteq\mU_k$.
        Let $r=\min_{1\leq k\leq n}r_k$, then since $r_k>0$ and the minimum is taken over a finite set, $r>0$.
        Furthermore, for every $k$, $r<r_k$, so $B_r(x)\subseteq B_{r_k}(x)\subseteq\mU_k$.
        And since this is true for every $\mU_k$ in the intersection, $B_r(x)\subseteq\bigcap_{k=1}^n\mU_k$.
        And so the intersection is open as well.
    \eenum

    \hfill$\blacksquare$

\end{proof}

Notice that the set of open sets defined by a metric (this is actually an important mathematical object, which we will focus on later on in the course) is not unique to the metric.
If we take a metric $\metric$ and then multiply it by some positive to get $\alpha\metric$, this is still a metric (trivial), and the open sets defined by $\metric$ are the same as those defined by
$\alpha\metric$ since a set is an open ball if and only if it is an open ball in the other (ie. the set of open balls defined by each metric are equal).

Another example of two metrics which admit the same topologies are the maximum and euclidean metrics in $\bR^n$.
The intuition for this (in fact, a sketch of the proof) is that the open balls relative to the maximum metric are in fact open hypercubes of width $2r$ centered around the point.
So we can embed a ball of radius $r$ (an actual ball, ie a ball relative to the euclidean metric) within this hypercube, and therefore if a set is open relative to the maximum metric, it is open relative
to the euclidean metric as well.
And balls relative to the euclidean metric are in fact hyperballs, and we can embed hypercubes of radius $\frac r{\sqrt2}$ within these hyperballs.
So by similar logic as before, an open set relative to the euclidean metric is open relative to the maximum metric, as required.
All it comes down to is proving that
\[ B(x,r,\metric_{\rm euc}) \subseteq B(x,r,\metric_{\rm max})\qquad\text{and}\qquad B\parens{x,\frac r{\sqrt 2},\metric_{\rm max}}\subseteq B(x,r,\metric_{\rm euc}) \]

We define some more metric spaces:
\[ \bS^n = \set{x\in\bR^{n+1}}[\metricof{x,0}=1] \]
relative to the euclidean metric of $\bR^{n+1}$.

Notice that the convergence of a sequence is dependent only on the set of open sets defined by a metric, and so investigating not the metric itself but the open sets it defines is a worthwhile pursuit.

\begin{defn*}

    Suppose $(M,\metric)$ is a metric space, a \ppemph{neighborhood} of $x\in M$ is an open set $\mU\subseteq M$ such that $x\in\mU$.

\end{defn*}

\begin{prop*}

    A sequence $\set{x_n}_{n=1}^\infty$ in a metric space $M$ converges to $x\in M$ if and only if for every neighborhood $\mU$ of $x$ there exists an $N$ such that for every $n\geq N$, $x_n\in\mU$.

\end{prop*}

\begin{proof}

    Suppose $x_n$ converges to $x$ and let $\mU$ be a neighborhood of $x$.
    Then by the definition of an open set, there exists an $\epsilon>0$ such that $B_\epsilon(x)\subseteq\mU$.
    And since by the definition of convergence there exists an $N$ such that for every $n\geq N$, $x_n\in B_\epsilon(x)\subseteq\mU$ as required.

    The converse is similarly trivial.
    Let $\epsilon>0$ then since $B_\epsilon(x)$ is a neighborhood of $x$, there exists an $N$ such that for every $n\geq N$, $x_n\in B_\epsilon(x)$, or in other words for every $n\geq N$,
    $\metricof{x_n,x}<\epsilon$ so $x_n$ converges to $x$.

    \hfill$\blacksquare$

\end{proof}

We can give an even weaker equivalence:

\begin{prop*}

    $x_n$ converges to $x$ in a metric space $M$ if and only if for every neighborhood of $x$ $\mU\subseteq M$ there are only a finite number of $x_n\notin\mU$.

\end{prop*}

\begin{proof}

    If $x_n$ converges to $x$ then for there exists an $N$ such that if $n\geq N$, $x_n\in\mU$.
    Then if $x_n\notin\mU$, $n<N$ so there are only a finite number of $x_n$ which aren't in $\mU$, as required.

    And if there are a finite number of elements not in $\mU$, let $N$ be the maximum index not in $\mU$ plus one (which exists since the set is finite), then for every $n\geq N$, $x_n\in\mU$, so $x_n$
    converges to $x$ as required.

    \hfill$\blacksquare$

\end{proof}

Thus convergence is not dependent on the metric (only the open sets defined by the metric) or the ordering of the naturals.
This gives us the following result:

\begin{prop*}

    Suppose $x_n$ converges to $x$ in a metric space $M$, and let $\sigma\in S_\bN$ ($\sigma$ is a bijection over the naturals), then $y_n=x_{\sigma(n)}$ also converges to $x$.

\end{prop*}

The proof of this is trivial given the previous proposition.

\begin{defn*}

    If $(M,\metric)$ and $(N,\metricA)$ are two metric spaces, a function $f\colon M\longvarrightarrow N$ is \ppemph{continuous at $x$} if for every $\epsilon>0$ there exists a $\delta>0$ such that if
    $\metricof{x,y}<\delta$ then $\metricAof{f(x),f(y)}<\epsilon$.

\end{defn*}

Equivalently, for every $\epsilon>0$ there exists a $\delta>0$ such that:
\[ f\bigl(B_\delta(x)\bigr) \subseteq B_\epsilon\bigl(f(x)\bigr) \]
Note that this is still dependent on the choice of metric; it is not immediately clear whether a different choice of metric with the same open sets admits the same continuous functions.

\begin{prop*}

    Suppose $M$ and $N$ are metric spaces, then $f\colon M\longvarrightarrow N$ is continuous at $x$ if and only if for every neighborhood of $f(x)$ $\mU\subseteq N$ there exists a neighborhood
    $\mV\subseteq M$ of $x$ such that $f(\mV)\subseteq\mU$.

\end{prop*}

\begin{proof}

    Suppose $f$ is continuous at $x$, then let $\mU$ be a neighborhood of $f(x)$, then there exist an open ball $B_\epsilon^{\metricA}\bigl(f(x)\bigr)\subseteq\mU$ since $\mU$ is open.
    Then by the definition of continuity, there exists a $\delta>0$ such that $f\bigl(B_\delta^{\metric}(x)\bigr)\subseteq B_\epsilon^{\metricA}\bigl(f(x)\bigr)\subseteq\mU$, so $\mV=B_\delta^{\metric}(x)$
    is one such neighborhood of $x$ which satisfies the property.

    The proof of the converse is similar.
    Suppose $\epsilon>0$ then there exists a neighborhood $\mV$ of $x$ such that $f(\mV)\subseteq B_\epsilon\bigl(f(x)\bigr)$.
    Since $\mV$ is open there exists a $\delta>0$ such that $B_\delta(x)\subseteq\mV$ and so
    \[ f\bigl(B_\delta^{\metric}(x)\bigr)\subseteq f(\mV)\subseteq B_\epsilon^{\metricA}\bigl(f(x)\bigr) \]
    so $f$ is continuous at $x$ as required.

    \hfill$\blacksquare$

\end{proof}

\begin{prop*}

    Suppose $M$ and $N$ are metric spaces, then $f\colon M\longvarrightarrow N$ is continuous at $x\in M$ if and only if for every sequence $\set{x_n}_{n=1}^\infty$ in $M$, if $x_n\xvarrightarrow{\,M\,}x$
    then $f(x_n)\xvarrightarrow{\,N\,}f(x)$.

\end{prop*}

\begin{proof}

    Suppose $f$ is continuous at $x$ and $x_n$ converges to $x$.
    Then let $\mU$ be a neighborhood of $f(x)$ in $N$.
    Since $f$ is continuous at $x$ there exists a neighborhood $\mV$ of $x$ such that $f(\mV)\subseteq\mU$, and since $x_n$ converges to $x$ there are only a finite numberof $x_n$s not in $\mV$ and so there 
    are only a finite number of $f(x_n)\notin f(\mV)$, and so there are only a finite number of $f(x_n)\notin\mU$.
    So for every neighborhood of $x$, there are only a finite number of $f(x_n)$ which aren't in the neighborhood, so $f(x_n)\xvarrightarrow{\,N\,}f(x)$ as required.

    Now suppose the converse, suppose for the sake of a contradiction that $f$ is not continuous at $x$.
    So there exists an $\epsilon>0$ such that for every $\delta>0$, there exists a $y\in B_\delta(x)$ such that $f(y)\notin B_\epsilon\bigl(f(x)\bigr)$.
    Specifically for every $n\in\bN$, there is an $x_n\in B_{\frac1n}(x)$ such that $f(x_n)\notin B_\epsilon\bigl(f(x)\bigr)$, ie $\metricAof{f(x_n),f(x)}\geq\epsilon$.
    So the sequence $x_n$ converges to $x$ since $\metricof{x_n,x}<\frac1n$ which converges to $0$, but $f(x_n)$ does not converges to $f(x)$, which contradicts the assumption of the proposition.

    \hfill$\blacksquare$

\end{proof}

Note that while one direction of the proof is independent of the metric (continuity implies sequence convergence), the other is not.
And so when we eventually get rid of metric spaces, the above proposition will not be true in general.

\end{document}

