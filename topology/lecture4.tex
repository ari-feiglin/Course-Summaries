\documentclass[10pt]{article}

\usepackage{amsmath, amssymb, mathtools}
\usepackage[margin=1.5cm]{geometry}

\input pdfmsym
\input prettyprint
\input preamble

\pdfmsymsetscalefactor{10}
\initpps

\def\qed{\hskip1cm\penalty-100\hbox{}\hfill$\blacksquare$}
\def\pmat#1{\begin{pmatrix} #1 \end{pmatrix}}

\def\bS{\mathbb{S}}
\def\mU{\mathcal{U}}
\def\mV{\mathcal{V}}
\def\mO{\mathcal{O}}
\def\mF{\mathcal{F}}

\def\to{\varrightarrow}
\def\longto{\longvarrightarrow}

\newfunc{metric}{\rho}({})
\newfunc{metricA}{\sigma}({})
\newfunc{powset}{{\cal P}}(|)

\let\divides=\mid
\let\to=\varrightarrow
\def\implies{\,\longvarRightarrow\,}

\def\mathresize#1#2{\mathpalette\@mathresize{{#1}{#2}}}
\def\@mathresize#1#2{\@@mathresize#1#2}
\def\@@mathresize#1#2#3{\hbox to #2{\hss$#1#3$\hss}}

\font\bigbf = cmbx12 scaled 2000

\pdf@drawing@macro{check}{0 1 0 RG 1.2 w 1 j 1 J 0 5 m 3 0 l 8 10 l S}{9.2pt}{12pt}{1pt}{1pt}
\pdf@drawing@macro{ecks}{1 0 0 RG 0 0 m 1 j 1 J 10 10 l 0 10 m 10 0 l S}{11.2pt}{12pt}{1pt}{1pt}

\begin{document}

\c@section=3

\barcolorbox{255, 220, 255}{130, 0, 130}{200, 80, 200}{
    \leftskip=0pt plus 1fill \rightskip=\leftskip
    {\bigbf Topology}

    \medskip
    \textit{Lecture \thesection, Sunday April 16, 2022}

    \textit{Ari Feiglin}
}

\bigskip

\subsection{Homeomorphisms and closed sets}

\begin{defn*}

    Let $X$ be a topological space, then $\mF\subseteq X$ is called \ppemph{closed} if its complement is open.

\end{defn*}

It is immediate from the definition of open sets that
\benum
    \item $\varnothing$ and $X$ are closed.
    \item If $\set{\mF_\lambda}_{\lambda\in\Lambda}$ is an arbitrary set of closed sets then $\bigcap_{\lambda\in\Lambda}\mF_\lambda$ is also closed.
    \item If $\set{\mF_k}_{k=1}^n$ is a finite set of closed sets, then $\bigcup_{k=1}^n \mF_k$ is also closed.
\eenum

\begin{prop*}

    A function between topological spaces $f\colon X\longto Y$ is continuous if for every $\mF\subseteq Y$ closed, $f^{-1}(\mF)\subseteq X$ is also closed.

\end{prop*}

This is trivial: if $f$ is continuous, let $F\subseteq Y$ be closed then $F^c$ is open and so $f^{-1}(\mF^c)=f^{-1}(\mF)^c\subseteq X$ is open and so $f^{-1}(\mF)$ is closed.
And for the converse, let $\mU\subseteq Y$ be open then $\mU^c$ is closed so $f^{-1}(\mU^c)=f^{-1}(\mU)^c$ is closed and therefore $f^{-1}(\mU)$ is open.

\begin{defn*}

    Let $X$ and $Y$ be topological spaces, a mapping $f\colon X\longto Y$ is an \ppemph{open mapping} if for every $\mU\subseteq X$ open, $f(\mU)$ is open.
    And $f$ is a \ppemph{closed mapping} if for every $\mF\subseteq X$ closed, $f(\mF)$ is closed.

\end{defn*}

\begin{prop*}

    Let $X$ be a topological space, and $A\subseteq X$ then $\mF\subseteq A$ is closed relative to $A$ (closed in $\tau_A$) if and only if there is a $Q\subseteq X$ closed such that $\mF=Q\cap A$.

\end{prop*}

This is again trivial: if $\mF$ is closed relative to $A$ then $A\setminus\mF$ is open in $\tau_A$ so $A\setminus\mF = A\cap\mU$ for some $\mU\subseteq X$ open.
Then $\mF=A\cap\mU^c$ so $Q=\mU^c$ satisfies this.
And if $\mF=Q\cap A$ for $Q$ closed in $X$ then $A\setminus\mF=A\setminus Q=A\cap Q^c$ and $Q^c$ is open so $A\setminus\mF$ is open relative to $A$ so $\mF$ is closed relative to $A$.

\begin{exam*}

    If $X$ is a toplogical space and $A\subseteq X$, then $\tau_A$ is not necessarily a subset of $\tau_X$.
    That is, there can be a set open relative to $A$ which is not open relative to $X$.

    For example, take $X=\bR^2$ and $A=(0,1)\times\set0$.
    Then $A$ is open relative to itself but not relative to $X$ (in general if $A\subseteq X$ is not open then obviously $\tau_A\not\subseteq\tau_X$ since $A\in\tau_A$ is not in $\tau_X$).

    Similarly we can gave $S\subseteq A$ which is closed in $A$ but not in $X$.
    For example take a non-closed set $A\subseteq X$ then $A$ is closed in $A$ but not in $X$.

\end{exam*}

So we see that if $A\notin\tau_X$, then $\tau_A\not\subseteq\tau_X$.
But what if $A$ is open in $X$?

\begin{prop*}

    If $A$ is open in $X$ then and $\mU$ is open in $A$ then $\mU$ is open in $X$.
    That is, if $A\in\tau_X$ then $\tau_A\subseteq\tau_X$.

    Similarly, if $A$ is closed in $X$ and $\mF$ is closed in $A$ then $\mF$ is closed in $X$.

\end{prop*}

This is yet again trivial: if $\mU=A\cap\mU'$ is open in $\tau_A$, since the finite intersection of open sets is open, $A\cap\mU'$ is open in $X$, ie. $A\cap\mU'\in\tau_X$.
And if $A$ is closed then if $\mF=A\cap\mF'$ is closed in $A$, then $\mF$ is closed in $X$ as the intersection of closed sets.

This condition is actually equivalent, since as we said earlier, if $A$ is not open in $X$ then $\tau_A\not\subseteq\tau_X$.
And for closed sets if $A$ is not closed, then $A$ is closed in $A$ but not in $X$.
And this is an equality if and only if $A=X$ of course, since otherwise $X\notin\tau_A$.

\begin{thrm*}

    If $X$ and $Y$ are toplogical spaces and $f\colon X\longto Y$ is a mapping then the following are equivalent:
    \benum
        \item $f$ is bijective, continuous, and $f^{-1}$ is continuous.
        \item $f$ is bijective, continuous, and an open mapping.
        \item $f$ is bijective, continuous, and a closed mapping.
        \item $f$ is bijective, and for every $S\subseteq X$, $S$ is open in $X$ if and only if $f(S)$ is open in $Y$.
        \item $f$ is bijective, and for every $S\subseteq Y$, $S$ is open in $Y$ if and only if $f^{-1}(S)$ is open in $X$.
    \eenum

\end{thrm*}

\begin{proof}

    We show the first implication: suppose $\mU\subseteq X$ is open then $f(\mU)=\bigl(f^{-1}\bigr)^{-1}(\mU)$ and since $\mU$ is open and $f^{-1}$ is continuous, $f(\mU)$ is open.
    Now for the second, suppose $f$ is an open mapping, and let $\mF\subseteq X$ is closed then $f(\mF)=f\parens{{\mF^c}^c}=f(\mF^c)^c$ is closed since $f(\mF^c)$ is open (the last equivalence is due to the
    bijectivity of $f$).
    This also shows that the third implies the second.
    To show the third implication, we can assume $f$ is an open mapping then this is a direct consequence of the continuity and open-mapping nature of $f$.
    The fourth property obviously implies the fifth.
    To show the fifth implies the first, notice that this obviously implies continuity of $f$ and $f^{-1}$.
    \qed

\end{proof}

\begin{defn*}

    A mapping which satisfies any of the above equivalent conditions is called a \ppemph{homeomorphism}.
    If there exists a homeomorphism between two toplogical spaces $X$ and $Y$ then $X$ and $Y$ are considered \ppemph{homeomorphic}, which is denoted $X\cong Y$.

\end{defn*}

It is obvious that if $f$ is a homeomorphism then so is $f^{-1}$.
And the composition of homeomorphisms is also a homeomorphism.
Thus the homeomorphic relation is symmetric and transitive, it is also obviously reflexive since it is trivial to see that ${\rm id}_X$ is a homeomorphism on $X$.
So the homeomorphic relation is essentially an equivalence relation (barring the issue of constructing a set of all the toplogical spaces).

\begin{exam*}

    We can show that $[a,b]\cong[c,d]$ by defining $f(x)=\alpha x+\beta$ where $f(a)=c$ and $f(b)=d$ (this has a solution), and it can be shown that this is continuous and so is its inverse.
    We use the same function to show $(a,b)\cong(c,d)$.
    Similarly $(-\infty,a)\cong(-\infty,b)\cong(b,\infty)$, the (correct) bounds can also be closed.

    And we can use $\tan$ as a homeomorphism between $\parens{-\frac\pi2,\frac\pi2}$ and $\bR$ and thus $\bR$ is homeomorphic with every open interval.
    We can also restrict $\tan^{-1}$ from $[0,\infty)$ to $\left[0,\frac\pi2\right)$ to show that $[a,\infty)\cong[a,b)$.
    And we can take $-x$ to show $(0,1]\cong[-1,0)$ so all half-open intervals are homeomorphic, independent of which side is open.

    So we can classify the intervals in homeomorphism classes: closed intervals, open intervals (including open intervals including infinite bounds), and half open intervals (including infinite bound
    intervals with closed finite bounds like $[a,\infty)$).

    And we know this is the strictest possible classification, as injective continuous functions between intervals must be monotonic.
    So $(a,b)$ cannot be homeomorphic to $(a,b]$ as nothing can map to $b$, and neither can be homeomorphic to $[a,b]$ since nothing can map to $a$.

\end{exam*}

\begin{defn*}

    Let $X$ be a toplogical space, and $A\subseteq X$ then the \ppemph{closure} of $A$ is defined as
    \[ \overline A = \bigcap_{\substack{\mF\text{ is closed}\\ A\subseteq\mF}}\mF \]
    And the \ppemph{interior} of $A$ is defined as
    \[ \mathring A = \bigcup_{\substack{\mU\text{ is open}\\ \mU\subseteq A}}\mU \]

\end{defn*}

The closure of a set has the following properties:
\benum
    \item $\overline A$ is closed
    \item $A\subseteq\overline A$
    \item For any set $\mF$ which satisfies the above two properties, $\overline A\subseteq\mF$ (as $\mF$ is in the intersection then).
    In other words, the closure of a set is the minimal closed set which contains the set (the closure is minimal).
    \item If two sets $B$ and $C$ satisfy the above three properties, then $B=C=\overline A$ as $B\subseteq C$ and $C\subseteq B$ by the third property (and also for $\overline A$) (the closure is unique).
\eenum

The interior of a set has the following properties:
\benum
    \item $\mathring A$ is open
    \item $\mathring A\subseteq A$
    \item For any set $\mU$ which satisfies the above two properties, $\mU\subseteq\mathring A$ (the interior is maximal).
    \item If two sets $B$ and $C$ satisfy the above three properties, then $B=C=\mathring A$ (the interior is unique).
\eenum

And finally it is easy to see
\[ \bigl(\mathring A\bigr)^c = \overline{A^c},\qquad \bigl(\overline A\bigr)^c = \mathring{\bigl(A^c\bigr)} \]

\begin{prop*}

    Let $X$ be a toplogical space and $A\subseteq X$ then $p\in X$ is in $\overline A$ if and only if for every neighborhood $\mU$ of $p$, $A\cap\mU\neq\varnothing$.

\end{prop*}

\begin{proof}

    If $p\in\overline A$ then let $p\in\mU$ be a neighborhood of $p$, then $p\in\mU\cap\overline A$ so the intersection is non-empty.
    If $p\in\overline A$ then suppose there is a neighborhood $\mU$ of $p$ which is disjoint from $A$.
    Thus $A\subseteq\mU^c$ and $\mU^c$ is closed so $\overline A\subseteq\mU^c$, and since $p\in\overline A$ then $p\in\mU^c$ so $p\notin\mU$ in contradiction.
    \qed

\end{proof}

\begin{defn*}

    If $X$ is a toplogical space and $A\subseteq X$.
    We say that $A$ is \ppemph{dense} in $X$ if $\overline A=X$.

\end{defn*}

Thus $A$ is dense if and only if for every $x\in X$ and every neighborhood $\mU$ of $x$, $\mU\cap A\neq\varnothing$.
Thus $A$ is dense if and only if it intersects with every non-empty open set (as every non-empty open set is a neighborhood of its elements).

\begin{prop*}

    Suppose $A\subseteq B\subseteq X$ where $X$ is a topological space, then $\overline A^B=\overline A^X\cap B$ where $\overline A^X$ is the closure of $A$ with respect to $X$.

\end{prop*}

\begin{proof}

    We know that
    \[ \overline A^X\cap B = \bigcap_{\mathresize{1cm}{A\subseteq\mF\text{ closed}}}\mF\cap B = \bigcap_{\mathresize{1cm}{A\subseteq\mF\text{ closed in $B$}}}\mF = \overline A^B \]
    since the sets which are closed relative to $B$ are exactly $\mF\cap B$ where $\mF$ is closed in $X$.
    \qed

\end{proof}

Thus $A$ is dense in $B$ if and only if $\overline A^X\supseteq B$.

\begin{thrm*}

    Let $X$ be a toplogical space and $\set{\mU_\lambda}_{\lambda\in\Lambda}$ be an open cover of $X$.
    Let $Y$ be another toplogical space and $f\colon X\longto Y$.
    Then if for every $\lambda\in\Lambda$, $f\bigr|_{\mU_\lambda}\colon\mU_\lambda\longto Y$ is continuous, then $f$ is continuous.

\end{thrm*}

\begin{proof}

    Let $f_\lambda=f\bigr|_{\mU_\lambda}$.
    Notice that for any $S\subseteq Y$:
    \[ x\in f^{-1}(S) \iff x\in\bigcup_{\lambda\in\Lambda} f_\lambda^{-1}(S) \]
    since if $x\in f^{-1}(S)$ then $x\in\mU_\lambda$ for some $\mU_\lambda$ and since $f(x)\in S$, $f_\lambda(x)\in S$ so $x\in f^{-1}_\lambda(S)$.
    And if $x$ is in the union then $x\in f_\lambda^{-1}(S)$ so $f_\lambda(x)\in S$ and so $f(x)\in S$.
    Thus for any $S\subseteq Y$:
    \[ f^{-1}(S) = \bigcup_{\lambda\in\Lambda} f_\lambda^{-1}(S) \]

    And so if $S=\mU$ is open, then $f_\lambda^{-1}(\mU)$ is open in $\mU_\lambda$ and since $\mU_\lambda$ is open, $f_\lambda^{-1}(\mU)$ is open in $X$.
    So $f^{-1}(\mU)$ is the union of open sets ($f_\lambda^{-1}(\mU)$) and is therefore itself open.
    Thus $f$ is continuous.
    \qed

\end{proof}

\begin{thrm*}

    Let $X$ be a toplogical space and $\set{\mF_k}_{k=1}^n$ be a finite closed cover of $X$.
    Let $Y$ be another topological space and $f\colon X\longto Y$.
    Then if for every $1\leq k\leq n$ $f\bigr|_{\mF_k}=f_k\colon\mF_k\longto Y$ is continuous, then $f$ is continuous.

\end{thrm*}

\begin{proof}

    By above we have
    \[ f^{-1}(\mF) = \bigcup_{k=1}^n f_k^{-1}(\mF) \]
    for every closed set $\mF\subseteq Y$.
    Since $f_k^{-1}(\mF)$ is closed in $\mF_k$ since $f_k$ is continuous, and since $\mF_k$ is closed in $X$, $f_k^{-1}(\mF)$ is closed in $X$ for every $k$.
    Thus $f^{-1}(\mF)$ is a finite union of closed sets and is therefore also closed.
    So $f$ is continuous.
    \qed

\end{proof}

\end{document}

