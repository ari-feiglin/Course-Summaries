\documentclass[10pt]{article}

\usepackage{amsmath, amssymb, mathtools}
\usepackage[margin=1.5cm]{geometry}

\input pdfmsym
\input prettyprint
\input preamble

\@Arrow@def{varLeftRightarrow}\@Larrow\@Rarrow{1}

\pdfmsymsetscalefactor{10}
\initpps

\def\pmat#1{\begin{pmatrix} #1 \end{pmatrix}}

\def\bS{\mathbb{S}}
\def\mU{\mathcal{U}}
\def\mV{\mathcal{V}}
\def\mO{\mathcal{O}}

\newfunc{metric}{\rho}({})
\newfunc{metricA}{\sigma}({})
\newfunc{pmetric}{d_p}({})
\newfunc{diam}{{\rm diam}}(\vert)

\let\divides=\mid
\let\to=\varrightarrow
\def\implies{\,\longvarRightarrow\,}
\def\iff{\,\longvarLeftRightarrow\,}

\font\bigbf = cmbx12 scaled 2000

\begin{document}

\c@section=1

\barcolorbox{255, 220, 255}{130, 0, 130}{200, 80, 200}{
    \leftskip=0pt plus 1fill \rightskip=\leftskip
    {\bigbf Topology Recitation}

    \medskip
    \textit{Recitation 1, Sunday March 19, 2022}

    \textit{Ari Feiglin}
}

\bigskip

\begin{defn*}

    If $M$ is a metric space, then the \ppemph{diameter} of a set $A\subseteq M$ is:
    \[ \diamof{A} = \sup\set{\metricof{x,y}}[x,y\in A] \]
    and $A$ is \ppemph{bounded} if $\diamof A<\infty$.

\end{defn*}

\begin{prop*}

    The union of two bounded sets is itself bounded.

\end{prop*}

\begin{proof}

    Suppose $A$ and $B$ are bounded, let $a\in A$ and $b\in B$, then $M=\diamof A+\diamof B+\metricof{a,b}$.
    We claim $\diamof{A\cup B}\leq M$.
    Let $x,y\in A\cup B$.
    If both $x$ and $y$ are in the same set then $\metricof{x,y}\leq\diamof A$ or $\diamof B$ depending on which set they're in, and necessarily $\metricof{x,y}\leq\diamof A+\diamof B\leq M$.
    Otherwise suppose $x\in A$ and $y\in B$ then $\metricof{x,y}\leq\metricof{x,a}+\metricof{a,b}+\metricof{b,y}\leq\diamof A+\metricof{a,b}+\diamof B=M$ as required.

    \hfill$\blacksquare$

\end{proof}

\begin{defn*}

    An \ppemph{ultrametric space} is a set $M$ equipped with an \ppemph{ultrametric}, a function $\metric\colon M\times M\longvarrightarrow\bR_{\geq0}$ where
    \benum
        \item $\metricof{x,y}=0$ if and only if $x=y$
        \item $\metricof{x,y}=\metricof{y,x}$
        \item $\metricof{x,y}\leq\maxof{\metricof{x,z}, \metricof{z,y}}$
    \eenum

    An ultrametric is trivially also a metric.

\end{defn*}

Let $X$ be the set of infinite sequences above $\set{0,\dots,n-1}$, ie $X=\set{0,\dots,n-1}^\bN$.
If $w\in X$ is a sequence we understand that $w_i$ is the $i$th index of $w$.
Then for $w,v\in X$ we define:
\[ k(w,v) = \begin{cases} \infty & w=v \\ \min\set{i\in\bN}[w_i\neq v_i] \end{cases} \]
and we define:
\[ \metricof{w,v} = \begin{cases} 0 & u=v \\ \frac1{p^{k(w,v)}} & u\neq v\end{cases} \]
for $1<p$.
We claim that $\metric$ is an ultrametric.
It is obvious that it is both nonnegative and symmetric, as well as that it is $0$ only when $w=v$.
So we must prove the triangle inequality.
Suppose $w,u,v\in X$ then if any one of them are equal, this is trivial.
Otherwise we must show that
\[ \metricof{w,v} \leq \maxof{\metricof{w,u}, \metricof{u,v}} \iff \frac1{p^{k(w,v)}} \leq \max\set{\frac1{p^{k(w,u)}}, \frac1{p^{k(u,v)}}} \]
This is equivalent to
\[ p^{-k(w,v)} \leq \maxof{p^{-k(w,u)}, p^{-k(u,v)}} \]
and since $p>1$ this is only if
\[ -k(w,v) \leq \maxof{-k(w,u), -k(u,v)} \iff k(w,v) \geq \minof{k(w,u), k(u,v)} \]
Suppose for the sake of a contradiction that this is false, ie $k(w,v)<\minof{k(w,u), k(u,v)}$.
Let $i=k(w,v)$ and so $w_i\neq v_i$, and we furthermore know that $i<k(w,u)$, so $u_i=w_i$ and $i<k(u,v)$ so $u_i=v_i$ so $w_i=v_i$ in contradiction to $w_i\neq v_i$.
So $\metric$ is in fact a metric.

\begin{defn*}

    Let $p$ be prime, then we define the metric $\pmetric$ over the integers $\bZ$ by:
    \[ \pmetricof{x,y} = \begin{cases} 0 & x=y \\ \frac1{p^{k(x,y)}} & x\neq y\end{cases} \]
    where
    \[ k(x,y) = \max\set{n\in\bN_{\geq0}}[p^n\divides(x-y)] \]
    This is called the \ppemph{$p$-adic} metric.

\end{defn*}

We claim that $\pmetric$ is an ultrametric:
\[ \pmetric{x,y} \leq \maxof{\pmetric{x,z}, \pmetric{z,y}} \iff k(x,y) \geq \minof{k(x,z), k(z,y)} \]
Suppose $i=\minof{k(x,z), k(z,y)}$ then $p^i\divides (x-z), (z-y)$ and so $p^i\divides\bigl((x-z)+(z-y)\bigr)\implies p^i\divides(x-y)\implies i\leq k(x,y)$ as required.

There is a connection between the two metric spaces we defined above.

\begin{defn*}

    A \ppemph{iosmetry} between two metric spaces $(M,\metric)$ and $(N,\metricA)$ is a function $f\colon M\longvarrightarrow N$ such that for every $x,y\in M$:
    \[ \metricof{x,y} = \metricAof{f(x), f(y)} \]

\end{defn*}

Every isometry is an injection: if $f(x)=f(y)$ then $\metricof{x,y}=\metricAof{f(x),f(y)}=0$ so $x=y$.

Recall that every $z\in Z$ has a unique representation in base $p$, that is there is a unique sequence $\set{a_i}_{i=0}^k\in\set{0,\dots,p-1}$ such that:
\[ z = \pm\sum_{i=0}^k z_ip^i \]
So if we let $C_p$ be the set of all sequences above $\set{0,\dots,p-1}$ with the metric defined above, we define an isometry:
\[ \phi\colon\bZ\longvarrightarrow C_p \]
By
\[ \phi(z) = \begin{cases} \set{z_0, z_1,\dots,a_k,0,\dots} & z\geq0 \\ \set{z_0,\dots,z_k,(p-1),\dots} & z<0 \end{cases} \]
We claim that $\phi$ is indeed an isometry.
So we must prove that:
\[ \pmetricof{z,w} = p^{-k(\set{z_i}, \set{w_i})} = \metricof{\set{z_i}, \set{w_i}} \]
where $k$ is the function defined on $C_p$.
Let $t=k(\set{z_i}, \set{w_i})$, so $t$ is the first index where $z_i\neq w_i$, that is the first index where $z_i-w_i\neq0$.
So we must show that $p^t\divides(z-w)$ and $p^{t+1}$ does not.
We know that $p^t$ divides $z-w$ since $z-w=(z_i-w_i)p^t+\dots$ so $p^t$ divides $z-w$, and since $z_i-w_i\neq0$ $p^{t+1}$ does not.
So we have shown that $k(z,w)$ in $\bZ$ is equal to $k(\set{z_i},\set{w_i})=k(f(z),f(w))$ in $C_p$, and so $\pmetricof{z,w}=\metricof{f(z),f(w)}$ as required.

\begin{prop*}

    If $M$ is an ultrametric space, then if $y\in B_r(x)$ then $B_r(y)=B_r(x)$, that is every point in a ball is its center.

\end{prop*}

\begin{proof}

    We will show that $B_r(y)\subseteq B_r(x)$.
    Suppose $z\in B_r(y)$, so $\metricof{z,x}\leq\maxof{\metricof{z,y}, \metricof{x,y}}<\maxof{r,r}=r$.
    So $z\in B_r(x)$.
    Then by symmetry (since $x\in B_r(y)$), $B_r(x)\subseteq B_r(y)$.

    \hfill$\blacksquare$

\end{proof}

Notice that $p^n$ converges to $0$ in the $p$-adics: $\pmetric{p^n, 0}=p^{-n}$ which converges to $0$ (in $\bR$), so $p^n$ converges to $0$ in the $p$-adics.

\end{document}
