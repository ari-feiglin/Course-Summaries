\documentclass[10pt]{article}

\usepackage{amsmath, amssymb, mathtools}
\usepackage[margin=1.5cm]{geometry}

\input pdfmsym
\input prettyprint
\input preamble

\pdfmsymsetscalefactor{10}
\initpps

\def\qed{\hskip1cm\penalty-100\hbox{}\hfill$\blacksquare$}
\def\pmat#1{\begin{pmatrix} #1 \end{pmatrix}}

\def\bS{\mathbb{S}}
\def\mU{\mathcal{U}}
\def\mV{\mathcal{V}}
\def\mO{\mathcal{O}}
\def\mF{\mathcal{F}}

\def\to{\varrightarrow}
\def\longto{\longvarrightarrow}

\newfunc{metric}{\rho}({})
\newfunc{metricA}{\sigma}({})
\newfunc{powset}{{\cal P}}(|)

\let\divides=\mid
\let\to=\varrightarrow
\def\implies{\,\longvarRightarrow\,}

\def\mathresize#1#2{\mathpalette\@mathresize{{#1}{#2}}}
\def\@mathresize#1#2{\@@mathresize#1#2}
\def\@@mathresize#1#2#3{\hbox to #2{\hss$#1#3$\hss}}

\font\bigbf = cmbx12 scaled 2000

\begin{document}

\c@section=5

\barcolorbox{255, 220, 255}{130, 0, 130}{200, 80, 200}{
    \leftskip=0pt plus 1fill \rightskip=\leftskip
    {\bigbf Topology}

    \medskip
    \textit{Lecture \thesection, Sunday April 30, 2022}

    \textit{Ari Feiglin}
}

\bigskip

\begin{defn*}

    A \ppemph{disconnected} topological space $(X,\tau)$ is a topological space where there exists two non-empty open sets $\mU$ and $\mO$ such that $X=\mU\dcup\mO$.
    If there does not exist two such open sets, then $X$ is \ppemph{connected}.

\end{defn*}

Equivalently, $X$ is \ppemph{connected} if when $X=\mU\dcup\mO$ for open sets $\mU$ and $\mO$, $\mU$ or $\mO$ must be empty.

Notice that $X$ is disconnected if and only if there exist non-trivial (not $X$ or $\varnothing$) clopen (closed and open) sets.
This is because if $X$ is disconnected then $\mU^c=\mO$ so $\mU$ (and $\mO$) is closed, and therefore clopen.
And if $\mU$ is non-trivial and clopen then $X=\mU\dcup\mU^c$ is a non-trivial disjoint union of $X$ into open sets, and therefore $X$ is disconnected.
So $X$ is connected if and only if the only clopen sets are $X$ and $\varnothing$.

\begin{thrm*}

    Suppose $X$ and $Y$ are toplogical spaces and $X$ is connected.
    If $f\colon X\longto Y$ is continuous and onto $Y$, then $Y$ is also connected.

\end{thrm*}

\begin{proof}

    Suppose $Y=\mU\dcup\mO$ for open sets.
    Then $f^{-1}(\mU)$ and $f^{-1}(\mO)$ are open sets since $f$ is continuous.
    And so $X=f^{-1}(Y)=f^{-1}(\mU)\dcup f^{-1}(\mO)$ and so either $f^{-1}(\mU)$ or $f^{-1}(\mO)$ is empty, suppose $f^{-1}(\mU)$ is.
    Then since $f$ is surjective, $\mU=\varnothing$, so $Y$ is connected as required.
    \qed

\end{proof}

Notice then that if $f\colon X\longto Y$ is not necessarily onto and $X$ is connected, then $f(X)$ is connected.
This is because the restriction $f\colon X\longto f(X)$ is continuous and surjective, the rest follows from the previous theorem.

\begin{prop*}

    A topological space $X$ is connected if and only if there is a continuous surjective mapping from $X$ to the discrete topology on two points.

\end{prop*}

\begin{proof}

    If $X$ is connected then there cannot exist such a mapping since the discrete toplogy is disconnected (if the discrete topology is not on a singleton).
    If $X$ is disconnected then $X=\mU\dcup\mO$ non-trivial.
    Let $Y=\set{a,b}$ is a topological space (with any topology, specifically here the discrete topology), then we define
    \[ f(x) = \begin{cases} a & x\in\mU \\ b x\in\mO \end{cases} \]
    Since $\mU$ and $\mO$ are disjoint, this is well-defined, and since they cover $X$ this is surjective.
    Since $\set{\mU, \mO}$ is an open cover of $X$ and the restriction of $f$ to these sets is constant (and therefore continuous), $f$ is itself continuous as required.
    \qed

\end{proof}

\begin{thrm*}

    $\bR$ is connected.

\end{thrm*}

\begin{proof}

    Suppose $\bR$ is not connected.
    Then there exists a continuous surjection $f\colon\bR\longto\set{0,1}$.
    But this contradicts the mean value theorem (since $f$ must take on values in $(0,1)$).
    \qed

\end{proof}

Since we showed the mean value theorem for intervals as well, the above proof holds for intervals.
Thus real intervals are connected as well (any interval).
Singletons are also connected in general as they cannot form a surjection to a set of two elements.
It just so happens that these are the only connected subspaces of $\bR$.
We will show this after proving the following lemma:

\begin{lemm*}

    Let $A\subseteq\bR$ be non-empty then the following are equivalent:
    \benum
        \item $A$ is some interval.
        \item If $a<c$ are in $A$ and $a<b<c$ then $b\in A$.
        \item $(\inf A,\sup A)\subseteq A$.
    \eenum

\end{lemm*}

This is also true for the empty set, but due to pedantic definitions of the supremum and infimum of empty sets.

\begin{proof}

    The first condition obviously implies the second.
    Now suppose $\inf A<b<\sup A$, then there exists $\inf A\leq a<b$ where $a\in A$ and $b<c\leq\sup A$ where $c\in A$.
    Thus $a<b<c$ and $a,c\in A$ so $b\in A$.
    So $(\inf A,\sup A)\subseteq A$ as required.

    Now showing that the third condition implies the first is simple.
    Then $A$ either contains some of the end points $\inf A$ and $\sup A$ or it doesn't, but in any case this is an interval.
    \qed

\end{proof}

\begin{prop*}

    Every connected subspace of $\bR$ is an interval or a singleton, and every interval or singleton of $\bR$ is connected.

\end{prop*}

\begin{proof}

    Suppose $A\subseteq\bR$ is not a singleton or an interval.
    Then there exists $a,c\in A$ such that there exists a $b\in\bR$ where $a<b<c$ and $b\notin A$.
    Then let $\mU=(-\infty,b)\cap A$ and $\mV=(b,\infty)\cap A$ which are open in $A$ and cover $A$.
    But neither of them are empty since $a\in\mU$ and $c\in\mV$ so $A$ is disconnected.

    The other direction was proven above.
    \qed

\end{proof}

\begin{prop*}

    \benum
        \item If $X$ is a topological space such that $X=\mU\dcup\mV$ for disjoint open sets $\mU$ and $\mV$, then if $A\subseteq X$, $A\subseteq\mU$ or $A\subseteq\mV$.
        \item If $A\subseteq X$ and $A$ is dense in $X$ and connected.
        Then $X$ is also connected.
        \item If $A$ and $B$ are non-disjoint connected sets such that $X=A\cup B$ then $X$ is connected as well.
        \item $X$ is connected if and only if for every $a,b\in X$ there is a $A\subseteq X$ connected such that $a,b\in A$.
    \eenum

\end{prop*}

\begin{proof}

    \benum
        \item As $A=(A\cap\mU)\dcup(A\cap\mV)$, and both of these sets are open in $A$, $A=A\cap\mU$ or $A=A\cap\mV$ so $A\subseteq\mU$ or $A\subseteq\mV$ as required.
        \item Suppose $\overline A=X=\mU\dcup\mV$ is a disjoint open cover.
        $A$ is connected so $A\subseteq\mU$ or $A\subseteq\mV$, suppose the former.
        Then $\overline A\subseteq\overline\mU=\mU$ since $\mU$ is closed, so its closure is itself.
        So $X=\mU$ and therefore $X$ is connected.
        \item Suppose $X=\mU\dcup\mV$, suppose $A\subseteq\mU$.
        Then $B\subseteq\mU$ as otherwise $B\subseteq\mV$ and then $A\cap B=\varnothing$ which is a contradiction.
        So $X=A\cup B\subseteq\mU$, so $X$ is connected as required.
        \item If $X$ is connected then take $A=X$.
        If $X$ is disconnected, then $X=\mU\dcup\mV$ non-empty disjoint open cover.
        Take $a\in\mU$ and $b\in\mV$ then there cannot be a connected set $A$ such that $a,b\in A$ as we showed that $A\subseteq\mU$ or $A\subseteq\mV$.
        \qed
    \eenum

\end{proof}

We define an equivalence relation on a topological space $X$ as follows: $a\sim b$ if and only if there exists a connected subspace $E\subseteq X$ where $a,b\in E$.
This is an equivalence relation since:
\blist
    \item It is reflexive: $a\in\set a$ and singletons are connected so $a\sim a$.
    \item It is obviously symmetric.
    \item If $a\sim b$ and $b\sim c$ then $a,b\in E_1$ and $b,c\in E_2$ where $E_i$ are connected.
    Since their intersection is non trivial, $E_1\cup E_2$ is connected and $a,c\in E_1\cup E_2$ and so $a\sim c$.
\elist

\newpage
Equivalence classes under this relation are called \emph{connected components} of $X$.
We will denote the equivalence classes as $\set{C_\alpha}_{\alpha\in I}$.
These are disjoint and their union is $X$ as they form a partition of $X$/

\begin{prop*}

    If $A$ is connected then
    \benum
        \item $A\subseteq C$ for some equivalence class $C$.
        \item If $A\cap C\neq\varnothing$ then $A\subseteq C$.
        \item If $C\subseteq A$ then $A=C$.
        \item The connected components are connected.
    \eenum

\end{prop*}

\begin{proof}

    \benum
        \item Let $x\in A$ and $x\in C$.
        Then for every $y\in A$, by definition $x\sim y$ so $y\in C$, so $A\subseteq C$.
        \item Since $A\subseteq C$ for some $C$, and the $C$s are disjoint, this must mean $A\subseteq C$.
        \item $A\cap C=A\neq\varnothing$ so $A\subseteq C$ and so $A=C$.
        \item Let $a,b\in C$ then there exists $E\subseteq X$ connected such that $a,b\in E$ by the definition of $\sim$.
        So $E\cap C\neq\varnothing$ so $E\subseteq C$, and therefore $C$ is connected.
        \qed
    \eenum

\end{proof}

So the connected components are connected, but we also showed that they are maximal.
So this equivalence relation partitions $X$ into maximal connected sets.

\begin{prop*}

    Connected components are closed.

\end{prop*}

\begin{proof}

    $C$ is dense in $\overline C$ (this is true in general), and since $C$ is connected so is $\overline C$.
    But since $C$ is maximal, $\overline C=C$, so $C$ is closed.
    \qed

\end{proof}

Thus if there is a finite number of connected components, since the complement of a connected component is the union of all other connected components (as they form a partition), since this union is a finite
union of closed sets, $C^c$ is closed and so $C$ is open.
That is: if there is a finite number of connected components, the connected components are clopen.

\end{document}

