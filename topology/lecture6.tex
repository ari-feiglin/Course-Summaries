\documentclass[10pt]{article}

\usepackage{amsmath, amssymb, mathtools}
\usepackage[margin=1.5cm]{geometry}

\input pdfmsym
\input prettyprint
\input preamble

\pdfmsymsetscalefactor{10}
\initpps

\def\qed{\hskip1cm\penalty-100\hbox{}\hfill$\blacksquare$}
\def\pmat#1{\begin{pmatrix} #1 \end{pmatrix}}

\def\bS{\mathbb{S}}
\def\mU{\mathcal{U}}
\def\mV{\mathcal{V}}
\def\mO{\mathcal{O}}
\def\mF{\mathcal{F}}

\def\to{\varrightarrow}
\def\longto{\longvarrightarrow}

\newfunc{metric}{\rho}({})
\newfunc{metricA}{\sigma}({})
\newfunc{powset}{{\cal P}}(|)

\let\divides=\mid
\let\to=\varrightarrow
\def\implies{\,\longvarRightarrow\,}

\def\mathresize#1#2{\mathpalette\@mathresize{{#1}{#2}}}
\def\@mathresize#1#2{\@@mathresize#1#2}
\def\@@mathresize#1#2#3{\hbox to #2{\hss$#1#3$\hss}}

\font\bigbf = cmbx12 scaled 2000

\begin{document}

\c@section=6

\barcolorbox{255, 220, 255}{130, 0, 130}{200, 80, 200}{
    \leftskip=0pt plus 1fill \rightskip=\leftskip
    {\bigbf Topology}

    \medskip
    \textit{Lecture \thesection, Sunday April 7, 2022}

    \textit{Ari Feiglin}
}

\bigskip

\begin{defn*}

    If $X$ is a topological space, a \ppemph{curve} in $X$ is a continuous function $\gamma\colon[0,1]\longto X$.
    It is called a curve from $\gamma(0)$ to $\gamma(1)$.

\end{defn*}

If $\gamma$ is a curve we define its reverse as $\overline\gamma(t)=\gamma(1-t)$ which is a curve (continuous) as the composition of continuous functions.
Note that $\overline\gamma(0)=\gamma(1)$ and $\overline\gamma(1)=\gamma(0)$.

If $\gamma_1$ and $\gamma_2$ are two curves such that $\gamma_1(1)=\gamma_2(0)$ then we define their \emph{concatenation} as
\[ (\gamma_1*\gamma_2)(t) = \begin{cases} \gamma_1(2t) & 0\leq t\leq\frac12 \\ \gamma_2(2t-1) & \frac12\leq t\leq1 \end{cases} \]
This is continuous since its restrictions under the finite closed cover $\bracks{0,\frac12}\cup\bracks{\frac12,1}$ are continuous as the composition of continuous functions.

\begin{defn*}

    A topological space $X$ is \ppemph{pathwise connected} if for every $a,b\in X$ there exists a curve from $a$ to $b$ in $X$.

\end{defn*}

\begin{prop*}

    If $X$ is pathwise connected, $X$ is connected.

\end{prop*}

\begin{proof}

    Suppose for the sake of a contradiction that $X=\mU\dcup\mV$ which are non-empty closed sets.
    Then there exists $a\in\mU$ and $b\in\mV$, and so there exists a curve $\gamma$ from $a$ to $b$.
    But since $\mU$ is closed in $X$, $\gamma^{-1}(\mU)$ is closed in $[0,1]$ and it is open, so it is also open in $[0,1]$.
    The same is true for $\mV$, so $\gamma^{-1}(\mV)$ is also clopen in $[0,1]$.
    But $[0,1]$ is connected so the only clopen sets are $\varnothing$ and $[0,1]$.
    Since $a\in\mU$, $0\in\gamma^{-1}(\mU)$ and $1\in\gamma^{-1}(\mV)$ so they cannot be empty and so are both equal to $[0,1]$.
    But then they cannot be disjoint, as then their preimages would have to be disjoint, which is a contradiction.
    \qed

\end{proof}

Alternatively, for every $a,b\in X$ there exists a path $\gamma$ between them and so $a,b\in\gamma([0,1])=A$.
Since $\gamma$ is continuous and $[0,1]$ is connected, $A$ is connected.
So every two points in $X$ are contained in a connected set, and so $X$ is connected.

\begin{prop*}

    If $X$ and $Y$ are topological spaces where $X$ is pathwise connected, if there exists a surjective continuous function $f\colon X\longto Y$, then $Y$ is also pathwise connected.

\end{prop*}

\begin{proof}

    If $a,b\in Y$ then there exists $x,y\in X$ such that $f(x)=a$ and $f(y)=b$ since $f$ is surjective.
    There exists a path $\gamma$ from $x$ to $y$, so $f\circ\gamma$ is continuous and $f\circ\gamma(0)=f(x)=a$ and $f\circ\gamma(1)=f(y)=b$ so $f\circ\gamma$ is a curve from $a$ to $b$ as required.
    \qed

\end{proof}

Thus if $f\colon X\longto Y$ is a continuous function from a pathwise connected space $X$, then $f(X)$ is pathwise connected.
This is because $f\colon X\longto f(X)$ is surjective and continuous.

We define an equivalence relation on $X$ where $a\sim b$ if and only if there exists a curve from $a$ to $b$.
This is reflexive (constant curve), symmetric (reverse curve), and transitive (concatenation of curves), so it is indeed an equivalence relation.
This defines a partition of $X$, $\slfrac X\sim$.
The equivalence classes under this relation are called \emph{pathwise connected components}.

Note that if $a\sim b$ then $\gamma([0,1])$ is connected and contains $a$ and $b$, so if $a$ and $b$ are in a pathwise connected relation, they are in a connected relation (the relation from last lecture).

\begin{prop*}

    \benum
        \item If $A\subseteq X$ is pathwise connected, it is contained within one of pathwise connected components.
        \item Pathwise connected components are pathwise connected.
    \eenum

\end{prop*}

\begin{proof}

    \benum
        \item For every $a,b\in A$ then there exists a path $\gamma\colon[0,1]\longto A$ from $a$ to $b$.
        But expanding the codomain of $\gamma$ to $X$ (composing with the inclusion function) gives a curve in $X$ from $a$ to $b$, so $a\sim b$.
        So for every $b\in A$, $a\sim b$, so $A\subseteq\slfrac a\sim$ as required.
        \item Let $C$ be a pathwise connected component.
        Let $a,b\in C$ then $a\sim b$ so there exists a curve $\gamma\colon[0,1]\longto X$ from $a$ to $b$.
        Note that $[0,1]$ is pathwise connected (this is trivial), so $\gamma([0,1])$ is pathwise connected and so $\gamma([0,1])\subseteq C'$ for some connected component $C'$ by above.
        Since $a,b\in\gamma([0,1])$, $C'=C$ so restricting the codomain to get $\gamma\colon[0,1]\longto C$ gives a continuous curve from $a$ to $b$ in $C$.
        So for every $a,b\in C$ there exists a curve in $C$ from $a$ to $b$, so $C$ is pathwise connected.
        \qed
    \eenum

\end{proof}

Thus the partition of $X$ into pathwise connected components partitions $X$ into maximal pathwise connected components.
Thus $X$ is pathwise connected if and only if there exists a single pathwise connected component.

Since every pathwise connected component is pathwise connected and therefore connected, every pathwise connected component is contained within a connected component.

\begin{exam*}

    We define $C\subseteq\bR^2$ as:
    \[ C = [0,1]\times\set 0\cup\bigcup_{n\in\bN}\set{\tfrac1n}\times[0,1] \]
    $C$ is a line on the $x$ axis from $0$ to $1$ with lines sticking up at every $x=\frac1n$ from $y=0$ to $y=1$.
    Let $S$ be th point $(0,1)$, then define
    \[ X = C\cup\set S \]

    We now claim that $X$ is connected, but not pathwise.

    $C$ is path connected, since for every $p,q\in C$ we can take the composition of lines from $p=(p_x,p_y)$ to $(p_x,0)$ to $(q_x,0)$ to $q=(q_x,q_y)$.

    $C$ is also dense in $X$, since for every $x\in X$ if $x\in C$ then $x\in\overline C$.
    Otherwise $x=S$ and for every $B_r(S)$, there is a $\frac1n<r$ and so $\parens{0,\frac1n}$ is in $C$ and in $B_r(S)$ as required.
    So for every open neighborhood of $S$, it has non-trivial intersection with $C$ and therefore $S\in\overline C$.
    Therefore $\overline C=X$ as required.

    Since $C$ is dense in $X$ and is connected, then $X$ is connected (proven last lecture).

    But $X$ is not path connected.
    We show this by defining:
    \[ \mU_1 = \set{(x,y)\in X}[y>\frac13],\qquad \mU_2=\set{(x,y)\in X}[y<\frac23] \]
    this is an open cover of $X$.
    Let $\gamma\colon[0,1]\longto X$ be a curve such that $\gamma(0)=S$, we claim that $\gamma$ is constant.
    Since $\gamma^{-1}(\mU_1)\cup\gamma^{-1}(\mU_2)$ is an open cover of $[0,1]$
    Since $[0,1]$ is compact, this open cover has a Lebesgue number $\epsilon>0$.
    Let $m$ be a number such that $\frac1m<2\epsilon$, then we define
    \[ I_k = \bracks{\frac{k-1}m, \frac km} \]
    for $1\leq k\leq m$.
    This is a finite closed cover of $[0,1]$.
    And since every $I_k$ is contained within a ball of radius $\epsilon$, every $I_k$ is contained within one of $\gamma^{-1}(\mU_i)$.

    Since $0\in I_1$, and $S\notin\mU_2$, we have that $0\notin\gamma^{-1}(\mU_2)$ so $I_1\subseteq\gamma^{-1}(\mU_1)$.

    Now for every $k$ we define
    \[ w_k = \set{(x,y)\in\mU_1}[x<\frac1{k+\frac12}],\qquad w'_k=\set{(x,y)\in\mU_1}[x>\frac1{k+\frac12}] \]
    Since any point with $x$ value $\frac1{k+\frac12}$ cannot be in $\mU_1$ (it can only have $y$ value $0$ if it is in $X$), $w_k$ and $w'_k$ form an open cover of $\mU_1$.
    Since $S\in w_k$, and we can restrict $\gamma\colon I_1\longto\mU_1$.
    $\gamma(I_1)$ is connected, and so $\gamma(I_1)\subseteq w_k$ since $S\in\gamma(I_1)\cap w_k$, and $\gamma(I_1)\subseteq w_k\dcup w'_k$ and is connected.
    So then
    \[ \gamma(I_1) \subseteq \bigcap_{k=1}^\infty w_k = \set S \]
    Thus $\gamma\bigl|_{I_1}$ is constant.

    Since the right endpoint of $I_1$ is the left endpoint of $I_2$, this proof continues the same on $I_2$ and so on until $I_k$.
    So $\gamma$ is constant on all $I_k$ to $S$.
    So if $x\in X$ is not equal to $S$ then $S$ cannot be connected to $x$ by a continuous curve, so $X$ is not pathwise connected.

    The path connected components here are $C$ and $\set S$.

\end{exam*}

\begin{defn*}

    Suppose $V$ is a real linear space and $X\subseteq V$.
    $X$ is called \ppemph{convex} if for every $p,q\in V$ and every $0\leq t\leq1$, $p+t(q-p)\in X$.

\end{defn*}

\begin{prop*}

    If $X\subseteq\bR^n$ is convex, it is path connected.

\end{prop*}

\begin{proof}

    Let $p,q\in X$ then $\gamma\colon[0,1]\longto\bR^n$ by $\gamma(t)=p+t(q-p)$ is a continuous function from $p$ to $q$.
    Since scalar multiplication is continuous, this is indeed a curve from $p$ to $q$.
    And since $X$ is convex, $\gamma(t)\in X$ so we can restrict the codomain of $\gamma$ to $X$ and get a continuous curve $\gamma\colon[0,1]\longto X$ as required.
    \qed

\end{proof}

\end{document}

