\input pdfToolbox

\setlayout{horizontal margin=1.5cm, vertical margin=1.5cm}
\parindent=0cm
\parskip=3pt plus 2pt minus 2pt

\input pdfmsym
\input preamble

\pdfmsymsetscalefactor{10}

\footline={}

%%%%%%%%%%%%%%%%%%%%%%%%%%%%%%%%%%%%%%%%%%%%%%%%%%%%%%%%%%%%%%%%

\pageborder{rgb{1 .5 .5}}{rgb{1 0 0}}{5}

\color rgb{1 0 0}

{\def\boxshadowcolor{rgb{.8 0 0}}
\bppbox{rgb{1 .7 .7}}{rgb{1 0 0}}{rgb{.8 0 0}}

    \centerline{\setfontandscale{bf}{20pt}Introduction to Stochastic Processes}
    \smallskip
    \centerline{\setfont{it}Dr. Naomi Feldheim, \setfont{tt}naomi.feldheim@biu.ac.il}
    \centerline{\setfont{it}Summary by Ari Feiglin}

\eppbox

\bigskip

\bppbox{rgb{1 .7 .7}}{rgb{1 0 0}}{rgb{.8 0 0}}
\section*{Contents}

\tableofcontents
\eppbox

}

\vfill\break

\color{black}

\pageborder{rgb{1 .5 .5}}{rgb{1 0 0}}{5}
\null
\vfill\break

\pageno=1
\newif\ifpageodd
\pageoddtrue
\headline={%
    \hbox to \hsize{\color{black}%
        \ifpageodd\hfil{\it\currsubsection\quad\bf\folio}\global\pageoddfalse%
        \else{\bf\folio\quad\it\currsubsection}\hfil\global\pageoddtrue\fi%
    }%
}

\section{Introduction}

This course will focus on tools which can be used to study random processes.
A random process is a sequence of random variables which represent measurements of the process.
Examples of random processes are random walks (these are commonly described as the path a drunk man would take while trying to get home), card shuffles (which can be viewed as choosing a card and placing
it randomly in the deck), and branching (for example the population of bunnies in a specific area: the random variable being the number of bunnies in each generation).

\subsection{Markov Chains}

\bdefn

    A {\emphcolor discrete-time Markov process}\addtoindex{markov process}[discrete-time] is a sequence of random variables $\set{X_n}_{n\geq0}$.
    This sequence is called a {\emphcolor Markov chain}\addtoindex{markov chain} on a set of states $S$ if:
    \benum
        \item For every $n$, $X_n\in S$ almost surely (meaning $\probof{X_n\in S}=1$),
        \item For every $n\geq0$ and for every $s_0,\dots,s_{n+1}\in S$,
        $$\probof{X_{n+1}=s_{n+1}}[X_0=s_0,\dots,X_n=s_n]=\probof{X_{n+1}=s_{n+1}}[X_n=s_n]$$
        ie. the probability of the next measurement being some arbitrary value is dependent only on the previous measurement.
        This is only necessary if $\probof{X_0=s_0,\dots,X_n=s_n}>0$.
    \eenum

\edefn

In this course $S$ will always be countable.
We can also write the second condition using distributive equivalence:
$$ X_{n+1}\vert X_0,\dots,X_n \deq X_{n+1}\vert X_n $$

Notice how the Markov property can be strengthened in various ways, for example if $n>m$ then

\medskip
{\tabskip=0pt plus 1fil
\openup1\jot\halign to\hsize{\kern.5cm$\displaystyle#$\hfil\tabskip=0pt plus 1fil\cr
    \omit$\displaystyle\probof{X_n=s_n}[X_{n-1}=s_{n-1},\dots,X_m=s_m]$\hfil\cr
    = \sum_{s_m,\dots,s_0}\probof{X_n=s_n}[X_{n-1}=s_{n-1},\dots,X_0=s_0]\cdot\probof{X_{m-1}=s_{m-1},\dots,X_0=s_0}[X_{n-1}=s_{n-1},\dots,X_m=s_m]\cr
    = \probof{X_n=s_n}[X_{n-1}=s_{n-1}]\cdot\sum\probof{X_{m-1}=s_{m-1},\dots,X_0=s_0}[X_{n-1}=s_{n-1},\dots,X_m=s_m]\cr
    = \probof{X_n=s_n}[X_{n-1}=s_{n-1}]\cr
}}

\medskip
\noindent We also get
$$ \probof{X_n=s_n}[X_m=s_m,\dots,X_0=s_0] = \probof{X_n=s_n}[X_m=s_m] $$
We denote $m$ by $n$ and $n$ by $m+k$, then we prove this by induction on $k$:
{\tabskip=0pt plus 1fil
\openup1\jot\halign to\hsize{\kern.5cm$\displaystyle#$\hfil\tabskip=0pt plus 1fil\cr
    \omit$\displaystyle\probof{X_{n+k}=s_{n+k}}[X_n=s_n,\dots,X_0=s_0]$\hfil\cr
    = \sum\probof{X_{n+k}=s_{n+k}}[X_{n+1}=s_{n+1},\dots,X_0=s_0]\cdot\probof{X_{n+1}=s_{n+1}}[X_n=s_n,\dots,X_0=s_0]\cr
    = \sum\probof{X_{n+k}=s_{n+k}}[X_{n+1}=s_{n+1},X_n=s_n]\cdot\probof{X_{n+1}=s_{n+1}}[X_n=s_n]\cr
    = \probof{X_{n+k}=s_{n+k}}[X_n=s_n]\cr
}}
\medskip
\noindent as required.

\bdefn

    For a Markov chain $\set{X_n}_{n\geq0}$ on a finite set of states $S$, we define the {\emphcolor adjacency matrix} at the $n$th measurement by
    $$ P^{(n)}_{ij} = \probof{X_n=j}[X_{n-1}=i] $$
    for $i,j\in S$.
    This is also sometimes written as $P_n(i\to j)$ (the probability measuring $i$ on the $n-1$th measurement gives $j$ on the next).
    If $P^{(n)}$ is the same for all $n$, then we say that the chain is {\emphcolor homogeneus in time}, and we generally write $P$ in place of $P^{(n)}$.

\edefn

For example, suppose a frog is hopping between $N$ leaves.
The frog can hopping from every leaf to every other leaf, and it always chooses a leaf in an independent and uniform manner.
This defines a Markov chain where the states are the leaves, and $X_n$ is the leaf the frog is on after $n$ hops.
This Markov chain is even homogeneus since the frog makes its choices in a manner which does not take the current number of hops into account.
The adjacency matrix is defined by
$$ P_{ij} = \cases{\frac1{N-1} & $i\neq j$\cr 0 & $i=j$} $$
This is the simple random process on the complete graph of $N$ vertices, $K_N$.

Suppose $N=4$, and supppose that at the beginning the frog is on either the first or second leaf with equal probability.
What is the probability that after one hop the frog is on the fourth leaf?
The following notation will be used: $X\sim(a_0,\dots,a_n)$ will be used to mean $\probof{X=s_i}=a_i$, where $s_i$ is some understood ordering of the set of states $S$.
Then
$$ \probof{X_1=j}[X_0\sim\parens{\frac12,\frac12,0,0}] = \probof{X_1=j}[X_0=1]\cdot\frac12 + \probof{X_1=j}[X_0=2]\cdot\frac12 $$
as the rest of the terms are zero.
For $j=4$ we get that this is equal to $\frac13$.
Notice that we can generalize this and get
$$ \probof{X_{n+1}=j}[X_n\sim\vec v] = \sum_{i\in S}\probof{X_{n+1}=j}[X_n=i]\cdot\probof{X_n=i} = \sum_{i\in S}P^{(n+1)}_{ij}\vec v_i = (\vec v\cdot P^{(n+1)})_j $$
So we have proven the following:

\bprop

    If $X_n\sim\vec v$ then $X_{n+1}\vert X_n\sim\vec v\cdot P^{(n+1)}$, and so $X_n\vert X_0\sim\vec v\cdot P^{(n)}\cdots P^{(1)}$.
    In particular if the Markov chain is homogeneus, $X_n\vert X_0\sim\vec v\cdot P^n$.

\eprop

This simplifies dealing with Markov chains, especially homogeneus ones.

\bexam

    Suppose $\set{Y_n}_{n=1}^\infty$ is a sequence of random variables which have the distribution $Y_n\sim\Berof{\frac1n}$ (recall that $X\sim\Berof p$ means that $X$ is $1$ with probability $p$ and zero
    otherwise).
    And we define $X_n=\chi\set{(\exists m\leq n)\,Y_m=1}$, the indicator of the set of all values such that there is an index before $n$ where $Y_m=1$ ($\chi_S$ is the {\it indicator function} of the set
    $S$, defined by $\chi_S(x)=1$ for $x\in S$ and zero otherwise).
    We will prove $X_n$ is a Markov chain.
    Notice that
    $$ X_n = \chi\set{(\exists m\leq n)\,Y_m=1} = \chi\set{(\exists m\leq n-1)\,Y_m=1}\lor\chi\set{Y_n=1} = X_{n-1}\lor\chi\set{Y_n=1} $$
    $\lor$ is bitwise or, or equivalently the maximum.
    And therefore we get that $X_n=\bigvee_{i=1}^n\chi\set{Y_i=1}$.
    This means that if $X_{n-1}=1$ then $X_n=1$, and if $X_{n-1}=0$ then $X_n=1$ if and only if $Y_n=1$.
    And so $X_n$'s value depends only on $X_{n-1}$'s and not any previous $X_i$.
    So $\set{X_n}_{n=1}^\infty$ is indeed a Markov chain.

    Notice that
    $$ \displaylines{
        \probof{X_n=0}[X_{n-1}=0] = \probof{Y_n=0} = \frac{n-1}n,\quad \probof{X_n=1}[X_{n-1}=0] = \probof{Y_n=1} = \frac1n,\quad\cr
        \probof{X_n=0}[X_{n-1}=1] = 0,\quad \probof{X_n=1}[X_{n-1}=1] = 1
    } $$
    And so we get that
    $$ P^{(n)} = \pmat{\frac{n-1}n & \frac1n \cr 0 & 1} $$

\eexam

\bye

