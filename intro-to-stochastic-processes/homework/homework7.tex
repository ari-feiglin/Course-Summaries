\input pdfToolbox

\setlayout{horizontal margin=1.5cm, vertical margin=1.5cm}
\parindent=0cm
\parskip=3pt plus 2pt minus 2pt

\input pdfmsym
\input preamble

\pdfmsymsetscalefactor{10}

\footline={}

\def\printmcount{\the\counter{section}.\the\counter{math counter}}
\setcounter{section}{7}

%%%%%%%%%%%%%%%%%%%%%%%%%%%%%%%%%%%%%%%%%%%%%%%%%%%%%%%%%%%%%%%%

{\bppbox{rgb{1 .7 .7}}{rgb{1 0 0}}{rgb{.8 0 0}}

    \centerline{\setfontandscale{bf}{20pt}Introduction to Stochastic Processes}
    \smallskip
    \centerline{\setfont{it}Assignment \the\counter{section}}
    \centerline{\setfont{it}Ari Feiglin}

\eppbox}

\bigskip

\bexerc

    Suppose $(X_1,X_2,X_3)\sim{\cal N}(0,\Sigma)$ where
    $$ \Sigma = \pmatrix{14 & -4 & -3\cr -4 & 2 & 1\cr -3 & 1 & 1} $$
    \benum
        \item Find a triangle matrix $U$ such that $U^\top U=\Sigma$.
        \item Compute $\expecof{X_1^2X_2X_3}$.
    \eenum

\eexerc

\benum
    \item We will look for a lower triangle matrix, so we are trying to solve
    $$ \pmatrix{14 & -4 & -3\cr -4 & 2 & 1\cr -3 & 1 & 1} = \pmatrix{a_1 & a_2 & a_3\cr & a_4 & a_5\cr & & a_6}\pmatrix{a_1\cr a_2 & a_4\cr a_3 & a_5 & a_6} =
    \pmatrix{a_1^2+a_2^2+a_3^2 & a_2a_4 + a_3a_5 & a_3a_6\cr a_2a_4+a_3a_5 & a_4^2+a_5^2 & a_5a_6\cr a_3a_6 & a_5a_6 & a_6^2} $$
    So we can assume that $a_6=1$ and then $a_5=1$ and $a_3=-3$.
    We can then further assume $a_4=-1$ and then $a_2=-1$ and $a_1=2$.
    So we get
    $$ U = \pmatrix{2\cr -1 & 1\cr -3 & 1 & 1} $$
    \item Since $\Sigma=AA^\top$ where $A$ is a transition matrix, $U^\top$ is a transition matrix for $(X_1,X_2,X_3)$ so there exist $(Z_1,Z_2,Z_3)\sim{\cal N}(0,I)$ such that
    $$ \pmatrix{X_1\cr X_2\cr X_3} = \pmatrix{2 & -1 & -3\cr & 1 & 1\cr & & 1}\pmatrix{Z_1\cr Z_2\cr Z_3} = \pmatrix{2Z_1 - Z_2 - 3Z_3\cr Z_2+Z_3\cr Z_3} $$
    So we get that
    \multlines{
        X_1^2X_2X_3 = (2Z_1-Z_2-3Z_3)^2(Z_2+Z_3)Z_3 = 4Z_1^2Z_2Z_3 - 4Z_1Z_2^2Z_3 - 12Z_1Z_2Z_3^2 + Z_2^3Z_3 - 6Z_2^2Z_3^2 + 9Z_3^2Z_2\cr
        &+ 4Z_1^2Z_3^2 - 4Z_1Z_2Z_3^2 - 12Z_1Z_3^3 + Z_2^2Z_3^2 - 6Z_2Z_3^3 + 9Z_3^4
    }
    Now since $Z_1,Z_2,Z_3$ are independent, the expected value of their products is the product of their expected values.
    And since their expected values are all zero, we focus only on the terms which have powers of $Z_i$ greater than $1$.
    So
    $$ \expecof{X_1^2X_2X_3} = 4\expecof{Z_1^2}\expecof{Z_3^2} + \expecof{Z_2^2}\expecof{Z_3^2} + 9\expecof{Z_3^4} $$
    For $Z\sim{\cal N}(0,1)$, $\expecof{Z^2}=1$ as this is its variance, and $\expecof{Z^4}=3$ (this can be computed using its moment generating function, $e^{-x^2/2}$).
    So
    $$ \expecof{X_1^2X_2X_3} = 32 $$
\eenum

\bexerc

    Suppose $B(t)$ is Brownian motion starting from $0$ and $t_1<t_2<t_3$, then compute
    \benum
        \item $\expecof{B(t_1)B(t_2)}$,
        \item $\expecof{B(t_1)B(t_2)B(t_3)}$.
    \eenum

\eexerc

\benum
    \item We know that $B(t_2)-B(t_1)\sim{\cal N}(0,t_2-t_1)$ and $B(t_i)\sim{\cal N}(0,t_i)$ so
    $$ \expecof{B(t_1)B(t_2)} = \expecof{-\frac{(B(t_2)-B(t_1))^2 - B(t_2)^2 - B(t_1)^2}2} = -\frac{(t_2-t_1) - t_2 - t_1} = t_1 $$
    \item We will prove something more general: if $X=(X_1,\dots,X_n)$ is a Gaussian vector and $n$ is odd, then $\expecof{X_1\cdots X_n}=0$.
    This is since the covariance matrix of $X$ is the same as that of $-X$, since $\Covof{-X_i,-X_j}=\Covof{X_i,X_j}$ by bilinearity.
    And so the joint probability of $X$ and $-X$ are the same, which means that $\expecof{X_1\cdots X_n}=\expecof{(-X_1)\cdots(-X_n)}=-\expecof{X_1\cdots X_n}$, so $\expecof{X_1\cdots X_n}=0$ as required.
    And since $\bigl(B(t_1),B(t_2),B(t_3)\bigr)$ is a Gaussian vector of odd length, we get the desired result.
\eenum

\bexerc

    Suppose $B(t)$ is Brownian motion, then show that $\limsup\frac{\abs{B(t)}}{\sqrt t}\geq1$ almost surely.

\eexerc

Notice that
\multlines{
    \probof{\abs{B(n)}\geq\sqrt n\hbox{ i.o.}} = \probof{(\forall n)(\exists m\geq n)\, \abs{B(m)}\geq\sqrt m} = \lim_{n\to\infty}\probof{(\exists m\geq n)\,\abs{B(m)}\geq\sqrt m}\cr
    &\geq \lim_{n\to\infty}\probof{\abs{B(n)}\geq\sqrt n} = \probof{\abs{{\cal N}(0,n)}\geq\sqrt n} > 0
}
Now
$$ X(n) = \frac{\abs{B(n)}}{\sqrt n}=\frac{\abs{\sum_{k=1}^n\bigl(B(k)-B(k-1)\bigr)}}{\sqrt n} $$
and $\set{B(k)-B(k-1)}_{k=1}^\infty$ are by definition independent, and since $\limsup X(n)$ is invariant under finite permutations of $\set{B(k)-B(k-1)}$, by Hewitt-Savage the event $\limsup X(n)\geq1$
has trivial probability.
But if $\abs{B(n)}\geq\sqrt n\hbox{ i.o.}$ then certainly $\limsup X(n)\geq1$ since we can form a sequence of all the indexes $m_n$ where $\abs{B(m_n)}\geq\sqrt{m_n}$ and in this case we would have
$\lim X(m_n)\geq1$.
And we showed that $\abs{B(n)}\geq\sqrt n\hbox{ i.o.}$ has nonzero probability, and so $\limsup X(n)\geq1$ also has nonzero probability, meaning almost surely $\limsup X(n)\geq1$, as required.

\bye

