%%%%%%%%%%%%%%%%%%%%%%%%%%%%%%%%
%    General TeX Utilities     %
%  (Much credit to Plante <3)  %
%                              %
%     Slurp - April, 2022      %
%%%%%%%%%%%%%%%%%%%%%%%%%%%%%%%%

\catcode`@=11

\def\gobble#1{}
\def\nul{\noexpand\nul}
\long\def\afterfi#1\fi{\fi#1}

% ISSUE: non-expandable
\long\def\@map@#1#2{%
    \ifx\nul#2%
        \let\next=\gobble%
    \else%
        #1{#2}%
        \let\next=\@map@%
    \fi%
    \next{#1}%
}

\long\def\@map@#1#2{%
	\ifx\nul#2%
	\else%
		\afterfi#1{#2}%
		\@map@{#1}%
	\fi%
}

% Suppose tokens is a list of tokens: t1 t2 t3 ... tN
% \map{\cs}{tokens} -> \cs{t1}\cs{t2}...\cs{\tN}
\long\def\map#1#2{\@map@{#1}#2\nul}

\long\def\@twomap@#1#2#3{%
	\ifx\nul#3%
	\else%
		\afterfi #1{#2}{#3}%
		\@twomap@{#1}%
	\fi%
}

% \twomap{\cs}{t1t2...t(2n-1)t(2n)} -> \cs{t1}{t2}...\cs{t(2n-1)}\cs{t(2n)}
\long\def\twomap#1#2{\@twomap@{#1}#2\nul\nul}

%% Various flavors of repeat %%

% \repeatit{num}{\cs} -> \cs\cs...\cs (num times)
\def\repeatit#1#2{%
    \relax\ifnum #1=0%
        \expandafter\gobble
    \else%
        \afterfi{#2%
        \expandafter\repeatit\expandafter{\the\numexpr #1-1\relax}{#2}}%
    \fi%
}

\def\@prepeat#1#2#3{%
    \relax\ifnum #1>#2%
    \else%
        \afterfi #3{#1}%
        \expandafter\@prepeat\expandafter{\the\numexpr #1+1\relax}{#2}{#3}%
    \fi%
}

% \prepeat{num}{\cs} -> \cs{1}\cs{2}...\cs{num}
\def\prepeat#1#2{\@prepeat{1}{#1}{#2}}

% \repeated{num}{\cs}{x} -> \cs{\cs{...\cs{x}}...} num groups
\def\repeated#1#2#3{%
    \relax\ifnum #1=0%
        #3%
    \else%
        #2{\repeated{\the\numexpr #1-1}{#2}{#3}}%
    \fi%
}

\newcount\@len
\def\@len@#1{%
    \ifx\nul#1%
        \let\next=\relax%
    \else%
        \ifx#1\ %
        \else%
            \advance\@len by 1%
            \let\next=\@len@%
        \fi%
    \fi%
    \next%
}

\def\len@norm#1{\@len=0 \@len@ #1\nul}
\def\len@star#1{\len@norm{#1} \the\@len}

% \len{str} -> puts the length of <str> into the register \@len
% \len*{str} -> calls \len{str} then \the\@len
\def\len{\@ifstar \len@star \len@norm}

% \smean{cs} -> \meaning of cs and friends
\def\@smean#1{\ifcsname #1\endcsname \hangindent=2cm \hangafter=1 \noindent$\underline{\hbox{\tt\detokenize\expanded{{#1}}}}$: \expandafter\meaning\csname #1\endcsname \par\fi}
\def\smean#1{\bgroup\tt\frenchspacing
    \@smean{#1} \@smean{#1 } \@smean{@#1} \@smean{@@#1} \@smean{@@@#1} \@smean{@#1@}
\egroup%
}

% \literal<parameter text>{<replacement text>} -> defines and calls a function literal by the given parameter and replacement texts
\def\literal{\afterassignment\@function@liter@l \def\@function@liter@l}

% \fontuse <font name> -> switches font to <font name>
\def\fontuse{\afterassignment\@temp@font \font\@temp@font=}

\def\gobbletwo#1#2{}
\def\gobblethree#1#2#3{}

\def\@mapparamto#1#2{%
	\expandafter\ifx\csname #1\endcsname\empty%
		\expandafter\def\csname #1\endcsname{#2}%
	\else%
		\errmessage{mapparam->@mapparamto: macro not defined as empty}%
	\fi%
}

\def\@mapparam@test@next{%
	\ifx\@next\nul%
		\expandafter\gobbletwo%
	\else%
		\ifx,\@next%
			\expandafter\gobblethree%
		\fi%
	\fi%
	\@mapparam
}

\def\@@mapparam{\futurelet\@next\@mapparam@test@next}

\def\@mapparam#1=#2,{\@mapparamto{#1}{#2}\@@mapparam}

\def\initialize@params#1{%
	\ifx\nul#1%
		\expandafter\gobble%
	\else%
		\let#1=\empty%
	\fi%
	\initialize@params%
}

% \mapparams{m1=p1,m2=p2,...,mN=pN}{\m1\m2...\mN} -> \def\m1{p1}...\def\mN{pN}
% All macros given in #2 are defined to be \empty before being mapped to pN.
% So if you do something like \mapparams{macroA=paramA}{\macroA\macroB},
% \macroA will be defined to be {paramA} and \macroB will be \empty
\def\mapparams#1#2{%
	\initialize@params #2\nul%
	\@mapparam #1,\nul%
}

\unless\ifcsname @ifnextchar\endcsname
	\def\@nextchar@cmp{\if\reg@MA\@nextchar \let\cmp@next=\reg@MB \else \let\cmp@next=\reg@MC \fi \cmp@next}
	\def\@ifnextchar#1#2#3{\def\reg@MA{#1}\def\reg@MB{#2}\def\reg@MC{#3} \futurelet\@nextchar\@nextchar@cmp}
	\def\@ifstar#1#2{\@ifnextchar*{\expandafter#1\gobble}{#2}}
\fi

\def\newline{\hfill\penalty-10000}

\def\hexdigit#1{\ifnum #1<10 #1\else\expandafter\char\the\numexpr #1-10 + `A\relax\fi}

\def\seq#1#2{%
	\ifnum #1<#2%
		{#1}%
		\expandafter\seq\expandafter{\the\numexpr #1+1\expandafter}%
	\fi%
	{#2}%
}

\def\emap#1#2{\expanded{\unexpanded{\map{#1}}{#2}}}

\font\smalltt=cmtt8

\def\@ftable@row#1#2{%
	&\expandafter\char\the\numexpr #1 * 16 + #2\relax%
}

\def\@fonttable@row#1{%
	{\smalltt"\hexdigit{#1}x}\expanded{\emap{\@ftable@row{#1}}{\seq0{15}}}\cr\noalign{\hrule}%
}

\def\@ftable@header#1{%
	&\hss{\smalltt"x\hexdigit{#1}}\hss%
}

\def\ftable@strut{\vbox{\hrule width 0pt height 13.5pt depth 5pt}}

\def\@fonttable{%
	\toks0\expandafter{\fontname\font}
	\openup\jot\offinterlineskip%
	\halign{\ftable@strut\vrule\hskip.5em\hss##\hss\hskip.5em\vrule&&\hskip.5em\hss##\hss\hskip.5em\vrule\cr
		\multispan{17}\hss{\fontuse cmcsc10 at 14pt \the\toks0\ Fonttable}\vrule width 0pt depth 5pt\hss\cr\noalign{\hrule}%
		\emap\@ftable@header{\seq0{15}}\cr\noalign{\hrule}%
		\emap\@fonttable@row{\seq0{7}}%
	}%
}

\def\@@fonttable#1{\fontuse #1 \@fonttable}

% \fonttable{<font>} -> font table of <font>
\def\fonttable{\@ifnextchar\bgroup \@@fonttable\@fonttable}


\newread\temp@in \newwrite\temp@out
% \sctoks{<tok list>} -> retokenizes <tok list> into an auxillary macro \temp@output
\def\sctoks#1{%
	\immediate\openout\temp@out=temp%
	\immediate\write\temp@out{#1}%
	\immediate\closeout\temp@out\openin\temp@in=temp%
	\read\temp@in to\temp@output%
	\closein\temp@in%
}

\newtoks\c@toks
\def\@@grabline#1{\expandafter\global\expandafter\c@toks\expanded{{\the\c@toks#1}}}

\long\def\@grabline#1#2{%
	\ifx\par#2%
		\egroup\sctoks{\the\c@toks}\expandafter#1\expandafter{\temp@output}%
	\else%
		\afterfi\@@grabline{#2}\@grabline{#1}%
	\fi%
}

% \grabline\cs <line of text> -> \cs{<line of text>}
\long\def\grabline#1{\bgroup\obeylines\catcode`\ =11\@grabline{#1}}

\def\@endcases{\noexpand\@endcases}
\def\default{\noexpand\default}
\let\@default=\default
\def\@@default{\noexpand \@@default}

% \strcmp{str1}{str2}{true}{false} -> if str1==str2 then true else false
\def\strcmp#1#2#3#4{%
	\edef\s@A{#1}\edef\s@B{#2}%
	\ifx\s@A\s@B #3\else #4\fi%
}

\def\@@strcasecmp#1#2#3{%
	\ifx\@default#2%
		#3%
	\else%
		\strcmp{#1}{#2}{#3 \let\default=\@@default}{}%
		\afterfi\ifstrcases{#1}%
	\fi%
}

% \ifstrcases{str}{cases} -> matches str to cases
% cases will be in the form of a list {<case 1>}...{<case n>} where each case is of the form {{match}{code}}
% where if str==match, code is executed.
% if match is \default, then code is executed by default.
\def\ifstrcases#1#2{%
	\ifx#2\@endcases%
		\let\default=\@default%
	\else%
		\afterfi\@@strcasecmp{#1}#2%
	\fi%
}

\def\bracedmacro@out#1#2{\def#1##1##{##1\bgroup #2 \let\next=}}
\def\bracedmacro@in#1#2{\def#1##1##{\bgroup #2##1 \let\next=}}

% \bracedmacro(*){\name}{\cs}
% \ifstar: \name<toks>{...} -> {\cs<toks>...}
% \else:   \name<toks>{...} -> <toks>{\cs...}
\def\bracedmacro{\@ifstar \bracedmacro@in\bracedmacro@out}

\bracedmacro\slanted\sl
\bracedmacro\italics\it
\bracedmacro\bold\bf
\bracedmacro*\rehsize{\hsize=}
\bracedmacro*\usefont\fontuse


\def\funnyfonts{}

\twomap{\literal#1#2{\expandafter\edef\expandafter\funnyfonts\expandafter{\funnyfonts\par#1}\expandafter\def\csname #1\endcsname{#2}}}
	{{Calligraname}{callig}{Scriptname}{wednsl}{Acornname}{Acorn}{Starburstname}{Starburst}}

%%%%%%%%%%%%%%%%%%%%%%%%%%%%%%%%
%           Demobox            %
%                              %
%     Slurp - April, 2022      %
%  (Much credit to Plante <3)  %
%%%%%%%%%%%%%%%%%%%%%%%%%%%%%%%%

\newbox\b@xit
\def\demo@boxit#1#2{%
    \setbox\b@xit=\hbox{#1}%
    \vbox{%
        \hrule width \dimexpr \wd\b@xit + #2 * 2\relax height #2%
        \hbox{%
            \vrule height \ht\b@xit width #2%
            \rlap{\vrule width \wd\b@xit height #2}%
            \copy\b@xit%
            \vrule height \ht\b@xit width #2%
        }%
        \hrule width \dimexpr \wd\b@xit + #2 * 2\relax height #2%
    }%
}

\def\hband#1#2#3{%
    \hbox{%
        \vrule width #3 height #2%
        \vrule width #1 height \dimexpr #3 + #2 / 2\relax  depth \dimexpr -#2 / 2\relax%
        \vrule width #3 height #2%
    }%
}

\def\vband#1#2#3{%
    \vbox{%
        \hrule width #1 height #3%
        \hbox to #1{\hss\vrule height #2 width #3\hss}%
        \hrule width #1 height #3%
    }%
}

\font\fivett cmtt8

\def\demobox@formatdimens#1#2{%
    \bgroup\fivett\offinterlineskip%
    \halign{\hfil##\quad\hfil&\hfil##\cr
         &#1\cr
        +&#2\cr%
    }%
    \egroup%
}

\newdimen\dem@b@xbandwidth \newdimen\dem@b@xbandheight
\dem@b@xbandwidth=5pt      \dem@b@xbandheight=0.1pt
\newbox\dem@b@x
\newbox\demobox@dimens

\def\demobox#1{%
    \setbox\dem@b@x=\hbox{#1}%
    \setbox\demobox@dimens=\vbox{\demobox@formatdimens{\the\ht\dem@b@x}{\the\dp\dem@b@x}}%
    \vbox{%
        \baselineskip=5pt%
        \hbox to \dimexpr \dem@b@xbandwidth + 4pt + \wd\demobox@dimens + \wd\dem@b@x + \dem@b@xbandheight * 2\relax{%
            \hfil\hbox to \wd\dem@b@x{\fivett\hss\the\wd\dem@b@x\hss}%
        }%        
        \hbox to \dimexpr \dem@b@xbandwidth + 4pt + \wd\demobox@dimens + \wd\dem@b@x + \dem@b@xbandheight * 2\relax{%
            \hfil\hband{\wd\dem@b@x}{\dem@b@xbandwidth}{\dem@b@xbandheight}%
        }%
        \hbox{%
            \raise\dimexpr(\ht\dem@b@x + \dp\dem@b@x - \ht\demobox@dimens)/2\relax\copy\demobox@dimens%
            \kern2pt%
            \vband{\dem@b@xbandwidth}{\dimexpr \ht\dem@b@x + \dp\dem@b@x\relax}{\dem@b@xbandheight}%
            \kern2pt%
            \demo@boxit{#1}{\dem@b@xbandheight}%
        }%
    }%
}

%%%%%%%%%%%%%%%%%%%%%%%%%%%%%%%%%%%%
%         Gradient Macros          %
%  (Macros for gradient-ing text)  %
%                                  %
%       Slurp - April, 2022        %
%%%%%%%%%%%%%%%%%%%%%%%%%%%%%%%%%%%%

\newcount\red@diff \newcount\green@diff \newcount\blue@diff \newcount\@red \newcount\@green \newcount\@blue
\def\grad@diffs#1#2#3{%
    \len{#3}%
    \grabparams{#1}{\@red@\@green@\@blue@}%
    \grabparams{#2}{\@@red\@@green\@@blue}%
    \@red=\@red@ \@green=\@green@ \@blue=\@blue@%
    \red@diff=\numexpr (\@@red-\@red@)/\the\@len\relax%
    \green@diff=\numexpr (\@@green-\@green@)/\the\@len\relax%
    \blue@diff=\numexpr (\@@blue-\@blue@)/\the\@len\relax%
}

\def\@grad@temp#1{%
    \ifx#1\ %
        \ %
    \else%
        \textcolor[RGB]{\the\@red, \the\@green, \the\@blue}{#1}%
        \advance\@red by \red@diff%
        \advance\@green by \green@diff%
        \advance\@blue by \blue@diff%
    \fi%
}

% \grad{color1}{color2}{text} -> writes <text> in a gradient from <color1> to <color2>
\def\grad#1#2#3{\grad@diffs{#1}{#2}{#3}\map{\@grad@temp}{#3}}

%%%%%%%%%%%%%%%%%%%%%%%
%    Pixel Drawing    %
%                     %
%  Slurp - May, 2022  %
%%%%%%%%%%%%%%%%%%%%%%%

\newdimen\math@dimen@result
\newdimen\math@dimen@input

{
	\catcode`\p=12 \catcode`\t=12%
	\gdef\@smd@#1pt{#1}%
}

\def\@strip@math@dimen{\expandafter\@smd@\the\math@dimen@result}

\newdimen\gpsize \gpsize=1pt

\newdimen\boxdrawheight
\newdimen\boxdrawwidth
\newbox\drawbox

\def\setdrawbox{\wd\drawbox=\boxdrawwidth \ht\drawbox=\boxdrawheight}

\def\begindraw{\boxdrawheight=0pt \boxdrawwidth=0pt \setbox\drawbox=\vbox\bgroup\endlinechar=-1 \nullfont}
\def\enddraw{\egroup\setdrawbox\hskip 1pt\box\drawbox\hskip 1pt}

\let\beginboxdraw=\begindraw
\def\endboxdraw{\egroup\setdrawbox\hskip 1pt\fbox{\box\drawbox}\hskip 1pt}

\def\updatedrawbox#1#2{%
	\ifdim #1 > \boxdrawwidth %
		\global\boxdrawwidth = #1%
	\fi%
	\ifdim #2 > \boxdrawheight %
		\global\boxdrawheight = #2%
	\fi%
}

\def\point(#1,#2);{%
	\updatedrawbox{\dimexpr \gpsize * #1 + \gpsize\relax}{\dimexpr \gpsize * #2 + \gpsize\relax}%
	\noindent\leavevmode\hbox to 0pt{%
		\vbox to 0pt{%
			\vskip \dimexpr \gpsize * #2\relax%
			\noindent\leavevmode\hskip \dimexpr \gpsize * #1\relax%
			\vrule width \gpsize height \gpsize depth 0pt%
			\vss%
			}%
		\hss%
		}%
	\ignorespaces%
}

\def\rect(#1,#2) to (#3,#4);{%
	\ifnum #1 < #3 %
		\edef\@dleft{#1}%
		\edef\@dright{#3}%
	\else%
		\edef\@dleft{#3}%
		\edef\@dright{#1}%
	\fi%
%
	\ifnum #2 < #4 %
		\edef\@dtop{#2}%
		\edef\@dbot{#4}%
	\else%
		\edef\@dtop{#4}%
		\edef\@dbot{#2}%
	\fi%
%
	\updatedrawbox{\dimexpr \gpsize * (\@dright + 1)\relax}%
				  {\dimexpr \gpsize * (\@dbot + 1)\relax}%
	\noindent\leavevmode\hbox to 0pt{%
		\vbox to 0pt{%
			\vskip \dimexpr \gpsize * \@dtop\relax%
			\noindent\leavevmode\hskip \dimexpr \gpsize * \@dleft\relax%
			\vrule width \dimexpr \gpsize * (\@dright - \@dleft + 1)\relax height \dimexpr \gpsize * (\@dbot - \@dtop + 1)\relax depth 0pt%
			\vss%
			}%
		\hss%
		}%
	\ignorespaces%
}

\newcount\draw@x
\newcount\draw@y
\def\dline(#1,#2) to (#3,#4);{%
	\ifnum #1 = #3 %
		\draw@y=#2%
		\loop%
			\point(#1,\the\draw@y);%
			\advance\draw@y by 1%
		\ifnum\draw@y < #4\repeat%
	\else%
		\draw@x=#1%
		\loop%
			\draw@y=\numexpr((#2 - #4) * (\the\draw@x - #1)) / (#1 - #3) + #2\relax%
			\immediate\write16{(\the\draw@x,\the\draw@y)}%
			\point(\the\draw@x,\the\draw@y);
			\advance\draw@x by 1%
		\ifnum\draw@x < #3\repeat%
	\fi%
}

\newcount\draw@regA
\newcount\draw@regB
\def\circle(#1,#2) rad (#3);{%
	\draw@x=\numexpr #1-#3\relax%
	\draw@regA=#2%
	\draw@regB=#2%
	\loop%
		\draw@y=\numexpr #3 * #3 - (\draw@x - #1) * (\draw@x - #1)\relax%
		\expandafter\sqrt@into\expandafter{\the\draw@y}%
		\setbox0=\hbox{\global\draw@y=\numexpr\@strip@math@dimen\relax}%
		\draw@y=\numexpr\draw@y+#2\relax%
		\rect(\the\draw@x,\the\draw@y) to (\the\draw@x,\the\draw@regA);%
		\draw@regA=\draw@y%
		\draw@y=\numexpr-\draw@y + #2 * 2\relax%
		\rect(\the\draw@x,\the\draw@y) to (\the\draw@x,\the\draw@regB);%
		\draw@regB=\draw@y%
		\advance\draw@x by 1%
	\ifnum\draw@x < \numexpr #1 + #3\relax\repeat%
}

\def\ucircle(#1,#2) rad (#3);{%
	\draw@x=\numexpr #1-#3\relax%
	\draw@regA=#2%
	\loop%
		\draw@y=\numexpr #3 * #3 - (\draw@x - #1) * (\draw@x - #1)\relax%
		\expandafter\sqrt@into\expandafter{\the\draw@y}%
		\setbox0=\hbox{\global\draw@y=\numexpr\@strip@math@dimen\relax}%
		\draw@y=\numexpr-\draw@y + #2\relax%
		\rect(\the\draw@x,\the\draw@y) to (\the\draw@x,\the\draw@regA);%
		\draw@regA=\draw@y%
		\advance\draw@x by 1%
	\ifnum\draw@x < \numexpr #1 + #3\relax\repeat%
}

\def\bcircle(#1,#2) rad (#3);{%
	\draw@x=\numexpr #1-#3\relax%
	\draw@regA=#2%
	\loop%
		\draw@y=\numexpr #3 * #3 - (\draw@x - #1) * (\draw@x - #1)\relax%
		\expandafter\sqrt@into\expandafter{\the\draw@y}%
		\setbox0=\hbox{\global\draw@y=\numexpr\@strip@math@dimen\relax}%
		\draw@y=\numexpr\draw@y+#2\relax%
		\rect(\the\draw@x,\the\draw@y) to (\the\draw@x,\the\draw@regA);%
		\draw@regA=\draw@y;
		\advance\draw@x by 1%
	\ifnum\draw@x < \numexpr #1 + #3\relax\repeat%
}

\def\graph@reg#1(#2:#3);{%
	\edef\d@left{#2} \edef\d@right{#3}%
	\let\g@f=#1%
	\draw@x=\d@left%
	\g@f\d@left\d@right%
	\draw@regA=\draw@y%
	\loop%
		\g@f\d@left\d@right%
		\rect(\the\draw@x,\the\draw@y) to (\the\draw@x,\the\draw@regA);
		\draw@regA=\draw@y%
		\advance\draw@x by 1%
	\ifnum\draw@x<\d@right\repeat%
}

\let\graph=\graph@reg

\def\fbox#1{\vbox{\hrule \hbox{\vrule #1\vrule} \hrule}}
\def\@fbox{\fbox{\box0}}
\def\boxit{\setbox0=\hbox\bgroup\aftergroup\@fbox \let\next=}

\def\normparab#1#2{%
	\draw@y=\numexpr (\draw@x - (#1 + #2) / 2) * (\draw@x - (#1 + #2) / 2)\relax%
}

\def\bpp{%
	\beginboxdraw

	\circle(50,50) rad (50);
	\circle(150,50) rad (50);
	\rect(50,100) to (50,300);
	\rect(150,100) to (150,300);
	\bcircle(100,300) rad (50);

	\endboxdraw%
}

%%%%%%%%%%%%%%%%%%%%%%%%
%       Math-cros      %
%      Math macros     %
%                      %
%  Slurp - June, 2022  %
%%%%%%%%%%%%%%%%%%%%%%%%

\let\mbb=\mathbb
\let\mfk=\mathfrak

\map{\literal#1{\expandafter\def\csname b#1\endcsname{\mbb{#1}}}}{NZQRCPF}

\def\newmathtype#1#2#3#4{%
    \expandafter\def\csname @#1\endcsname##1[##2]{\begingroup\left#2 ##1 \;\middle#3\; ##2 \right#4\endgroup}%
    \expandafter\def\csname @@#1\endcsname##1{\begingroup\left#2 ##1\right#4\endgroup}%
    \expandafter\def\csname #1\endcsname##1{\@ifnextchar[ {\csname @#1\endcsname{##1}}{\csname @@#1\endcsname{##1}}}%
}

\newmathtype{set}\{\vert\}
\newmathtype{condevent}.\vert.
\newmathtype{excondevent}(\vert)
\newmathtype{parens}({})
\newmathtype{bracks}[{}]
\newmathtype{gen}\langle|\rangle
\newmathtype{abs}|{}|
\newmathtype{floor}\lfloor{}\rfloor
\newmathtype{ceil}\lceil{}\rceil
\newmathtype{norm}\|{}\|

\def\@newfunc@#1#2#3#4#5{%
    \expandafter\def\csname #1of@master\endcsname[##1]##2[##3]{\begingroup\csname #1\endcsname_{##1}\left#3##2\;\middle#4\;##3\right#5\endgroup}%
    \expandafter\def\csname #1of@\endcsname[##1]##2{\@ifnextchar[ {\csname #1of@master\endcsname[##1]{##2}}{\begingroup\csname #1\endcsname_{##1}\left#3##2\right#5\endgroup}}%
    \expandafter\def\csname #1of@@\endcsname##1{\@ifnextchar[ {\csname #1of@master\endcsname[]{##1}}{\begingroup\csname #1\endcsname\left#3##1\right#5\endgroup}}%
    \expandafter\def\csname #1of\endcsname{\@ifnextchar[{\csname #1of@\endcsname}{\csname #1of@@\endcsname}}%
}

\def\@newfunc@star@#1#2{\expandafter\def\csname #1\endcsname{\mathop{#2\kern\z@}}}

\def\newfunc@star#1#2#3#4#5{%
    \@newfunc@star@{#1}{#2}%
    \@newfunc@{#1}{#2}{#3}{#4}{#5}%
}

\def\@newfunc@reg@#1#2{\expandafter\def\csname #1\endcsname{\mathop{#2\kern\z@}\nolimits}}

\def\newfunc@reg#1#2#3#4#5{%
    \@newfunc@reg@{#1}{#2}%
    \@newfunc@{#1}{#2}{#3}{#4}{#5}%
}

\def\newfunc{\@ifstar \newfunc@star \newfunc@reg}

\newfunc*{max}{\rm max}\{\vert\}
\newfunc*{min}{\rm min}\{\vert\}
\newfunc{interior}{\rm int}({})
\newfunc{exterior}{\rm ext}({})
\newfunc{boundary}\partial({})
\newfunc{sin}{{\rm sin}}({})
\newfunc{cos}{{\rm cos}}({})
\newfunc{tan}{{\rm tan}}({})
\newfunc{powset}{{\cal P}}(|)
\newfunc{inf}{{\rm inf}}\{|\}
\newfunc{sup}{{\rm sup}}\{|\}
\newfunc{Mor}{{\rm Mor}}({})
\newfunc{ob}{{\rm ob}}({})
\newfunc{Im}{{\rm Im}}({})
\newfunc{lspan}{{\rm span}}\{|\}
\newfunc{Gal}{{\rm Gal}}({})
\newfunc{Aut}{{\rm Aut}}({})
% Newfuncshere

\def\swapchars#1#2{%
	\expanded{%
		\mathchardef#1=\number#2%
		\mathchardef#2=\number#1%
	}%
}

\swapchars\epsilon\varepsilon
\swapchars\phi\varphi
\def\smiley{\hbox to 10pt{\pdfliteral{q 4.8611 0 0 4.8611 0 0 cm
.03 w
.9 .9 .3 rg
2 1 m
2 1.55228 1.55228 2 1 2 c
0.44772 2 0 1.55228 0 1 c
0 0.44772 0.44772 0 1 0 c
1.55228 0 2 0.44772 2 1 c b
1 J
1.5 0.65 m
1.3 0.35 0.7 0.35 0.5 0.65 c S
.33 w
.65 1.1 m .65 1.3 l S
1.35 1.1 m 1.35 1.3 l S
1 1 1 RG
.3 w
.65 1.1 m .65 1.3 l S
1.35 1.1 m 1.35 1.3 l S
.3 .7 1 RG
.2 w
.65 1.115 m .65 1.25 l S
1.35 1.115 m 1.35 1.25 l S Q}\hss}}
\def\crown{\hbox to 10pt{\pdfliteral{q 4.8611 0 0 4.8611 0 0 cm
.1 w
0 0 0 rg 0 0 0 RG
1 j
.2 1.7 m 1.8 1.7 l 1.8 2.2 l 1.4 2 l 1 2.2 l .6 2 l .2 2.2 l b
.05 w
.7 .65 .1 rg .7 .65 .1 RG
.2 1.7 m 1.8 1.7 l 1.8 2.2 l 1.4 2 l 1 2.2 l .6 2 l .2 2.2 l b Q}\hss}}

\def\smileycrown{\rlap{\smiley}\crown}

\def\@crownit#1[#2]{{\setbox0=\hbox{#1}\leavevmode\rlap{\copy0}\lower#2\hbox to\wd0{\hss\crown\hss}}}
\def\crownit#1{\@ifnextchar[ {\@crownit{#1}}{\@crownit{#1}[0pt]}}

\def\frac#1#2{{{#1}\over{#2}}}
\def\tfrac#1#2{{\textstyle\frac{#1}{#2}}}
\def\slfrac#1#2{\left.{}^{#1}\mkern-2mu\middle/\mkern-3mu{}_{#2}\right.}
\let\ds=\displaystyle

\def\@blist[#1]{%
    \bgroup\par%
    \def\item{%
        \par\egroup\bgroup\medskip\setbox0=\hbox{#1\quad}%
        \advance\leftskip by \wd0\leavevmode\kern-\wd0\box0%
    }%
    \bgroup%
}
\def\blist{\@ifnextchar[ \@blist {\@blist[$\bullet$]}}
\def\elist{\par\egroup\egroup\medskip}

\newcount\enumcount
\def\enumstyle#1{$\m@th\bf(#1)$}
\def\enumindent{.25cm}
\def\benum{\bgroup\par%
    \enumcount=0 %
    \def\item{%
        \par\egroup\advance\enumcount by 1\bgroup\medskip%
        \setbox0=\hbox{\enumstyle{\the\enumcount}\quad}%
        \advance\leftskip by \dimexpr\wd0+\enumindent\relax\noindent\kern-\wd0\box0%
    }%
    \bgroup%
}
\def\eenum{\par\egroup\egroup\par\medskip}

\let\o@left=\left \let\o@right=\right
\def\left{\mathopen{}\mathclose\bgroup\o@left}
\def\right{\aftergroup\egroup\o@right}

\def\@multlines#1\cr{\omit\span\omit$\displaystyle#1$\hfil\cr}

\def\multlines#1{%
    \par\medskip%
    {\m@th\tabskip=\dimexpr\leftskip+.5cm\relax
    \openup1\jot\halign to\hsize{\hfil##\tabskip=0pt plus 1fil&\hfil$\displaystyle##$\tabskip=.5cm&##\hfil\tabskip=.5cm\cr
        \@multlines#1\crcr
    }}
    \medskip%
}

\def\lmultlines#1{%
    \par\medskip
    {\m@th\tabskip=\dimexpr\leftskip+.5cm\relax
    \openup1\jot\halign to\hsize{\hfil##\tabskip=0pt plus 1fil&$\displaystyle##$\hfil\tabskip=.5cm&##\hfil\tabskip=.5cm\cr
        \@multlines#1\crcr
    }}
    \medskip
}

\def\centermath#1{%
    \vcenter{\halign{\hfil##\hfil\crcr #1\crcr}}%
}

\def\cnot@w{3pt} \def\cnot@h{2.5pt}
\def\centernot#1{{%
    \setbox0=\hbox{$\m@th#1$}%
    \vcenter{\pdf@literal{q
        \@pdfmsym@trans/
        1 j 1 J .3 w
        \@nopt{.5\dimexpr\wd0 - \cnot@w\relax} -\@nopt{\cnot@h} m
        \@nopt{.5\dimexpr\wd0 + \cnot@w\relax} \@nopt{\cnot@h} l
        S
    Q}}%
    {#1}%
}}

\def\notto{\mathrel{\centernot\to}}
\def\binom#1#2{{{#1}\choose{#2}}}

\def\@stackmath#1#2{%
    \vcenter{\offinterlineskip\openup1\jot\ialign{\hfil$\m@th#1##$\hfil\crcr#2\crcr}}%
}

\def\stackmath#1{%
    \mathpalette\@stackmath{#1}%
}

\input pdfmsym

\pdfmsymsetscalefactor{10}

\@Arrow@def{varLeftRightarrow}\@Larrow\@Rarrow{1}
\let\to=\varrightarrow
\let\longto=\longvarrightarrow
\let\oto=\varleftrightarrow
\let\longembeds=\longvaruphookrightarrow
\let\embeds=\varuphookrightarrow
\def\implies{\,\longvarRightarrow\,}
\def\impliedby{\,\longvarLeftarrow\,}
\def\iff{\,\longvarLeftRightarrow\,}
\def\coloneqq{\mathrel{{\mathop:}{=}}}

\mathchardef\nvDash="3532
\mathchardef\nvdash="3530
\mathchardef\nless="3504
\mathchardef\varnothing="53F
\mathchardef\varkappa="57B
\mathchardef\nless="3504
\mathchardef\ngreater="3505
\mathchardef\nmid="352D
\def\divides{{\mid}}

\protected\def\gvDash{\mathrel{\vbox{\offinterlineskip\ialign{\hfil$##$\hfil\cr\scriptstyle g\cr\noalign{\kern-1pt}\vDash\cr}}}}
\protected\def\ngvDash{\mathrel{\vbox{\offinterlineskip\ialign{\hfil$##$\hfil\cr\scriptstyle g\cr\noalign{\kern-1pt}\nvDash\cr}}}}
\protected\def\gvsim{\mathrel{\vbox{\offinterlineskip\ialign{\hfil$##$\hfil\cr\scriptstyle g\cr\noalign{\kern-1pt}\vsim\cr}}}}
\protected\def\Bvdash{\mathrel{\vbox{\offinterlineskip\ialign{\hfil$##$\hfil\cr\scriptstyle\,B\cr\noalign{\kern-2pt}\vdash\cr}}}}

\def\vsim{\mathrel{\vrule height 6.994pt width.3pt depth0pt\relax\mkern-1.25mu\raise1.6625pt\hbox{$\m@th\scriptstyle\sim$}}}

\def\boldsymbol#1{\mathchoice{\fakebold[.3]{#1}}{\fakebold[.3]{#1}}{\fakebold[.1]{#1}}{\fakebold[.1]{#1}}}

\def\Var{{\sl Var}}
\def\v{{\boldsymbol{v}}}
\def\eq{\mathrel{\boldsymbol{=}}}
\def\neqb{\not\mathrel{\boldsymbol{=}}}
\def\inb{\mathrel{\boldsymbol{\in}}}
\def\notinb{\not\mathrel{\boldsymbol{\in}}}
\def\subseteqb{\mathrel{\boldsymbol{\subseteq}}}
\def\cupb{\mathbin{\boldsymbol{\cup}}}
\def\capb{\mathbin{\boldsymbol{\cap}}}
\def\setminusb{\mathbin{\boldsymbol{\setminus}}}
\def\var{{\sl var}}
\def\bnd{{\sl bnd}}
\def\free{{\sl free}}
\def\Q{{\tt Q}}
\def\Md{\mathop{\rm Md}}
\def\Th{{\it Th}}
\def\bdivs{\boldsymbol{\divides}}
\def\bndivs{\boldsymbol{\nmid}}

\catcode`_=11
\def\_addtoindex#1[#2]{%
    \indexize{category=#1, item=#2, value=\the\pageno, expand value, add hyperlink}%
}
\def\addtoindex#1{%
    \_ifnextchar[ {\_addtoindex{#1}}{\_addtoindex{#1}[]}%
}

\def\_alsosee#1[#2]#3#4{%
    \seealso{category=#1, item=#2, dest=#3, hyperlink=#4, index link}%
}
\def\alsosee#1{%
    \_ifnextchar[ {\_alsosee{#1}}{\_alsosee{#1}[]}%
}
\catcode`_=8

\def\createfontselector#1#2#3{%
    \edef#1{%
        \noexpand\setfont{#2}%
        \unless\ifnum#3<0 %
            \fam=#3\relax%
        \fi%
    }%
}
\createfontselector\rm{rm}0
\createfontselector\bf{bf}6
\createfontselector\it{it}8
\createfontselector\tt{tt}7

\createcounter{math counter}[subsection]

\def\curvewidth{1.5pt}
\def\curvebuffer{10pt}

\def\curremphcolor{black}
{\catcode`_=11
\gdef\emphcolor{\expanded{\noexpand\_setcolor{}{\curremphcolor}}\setfont{bf}}
}

\def\printmcount{\the\counter{section}.\the\counter{subsection}.\the\counter{math counter}}

\catcode`_=11

\def\createmathbox#1#2#3#4#5{%
    \_xp\def\csname _b#1\endcsname[##1]{
        \mapkeys{%
            title={%
                name=_title,
                default=\novalue%
            },
            name={%
                name=_name,
                default=\novalue%
            }%
        }{##1}%
        \bppbox{#3}{#4}{#5}%
        {\globalsetters\advancecounter{math counter}{1}}%
        \def\curremphcolor{#5}%
        \hbox to\hsize{%
            \emphcolor\printmcount{} #2%
            \unless\ifx\_title\novalue%
                \ (\_title)%
            \fi%
            \unless\ifx\_name\novalue%
                \_xp\anchor\_xp{\_name}%
                \_xp\xdef\csname math@\_name\endcsname{\printmcount}%
                \unless\ifx\_title\novalue%
                    \_xp\xdef\csname math@\_name\endcsname{\_title}%
                    \addthreedepth{\_title}{\_name}%
                \fi%
            \fi%
        \hfil}%
        \hrule height\z@\relax%
        \medskip%
    }%
    \_xp\edef\csname b#1\endcsname{\noexpand\_ifnextchar[ {\_xp\noexpand\csname _b#1\endcsname}{\_xp\noexpand\csname _b#1\endcsname[]}}%
    \_xp\let\csname e#1\endcsname=\eppbox%
}

\def\bproof{%
    \blinedppbox{rgb{1 1 1}}{rgb{0 0 0}}{rgb{0 0 0}}
    \hbox to \hsize{\setfont{bf}Proof\hfil}
    \medskip
}
\let\eproof=\elinedppbox

\def\blankpp{%
    \blinedppbox{rgb{1 1 1}}{rgb{0 0 0}}{rgb{0 0 0}}
}
\let\eblankpp=\elinedppbox

\def\bnote{%
    \bppbox{rgb{1 1 .5}}{rgb{.5 .4 0}}{rgb{.5 .4 0}}
    \hbox to\hsize{\setfont{bf}Note\hfil}
    \medskip
}
\let\enote=\eppbox

\def\__refmath#1{\gotoanchor{#1}{\csname math@#1\endcsname}}
\def\_refmath[#1]#2{\gotoanchor{#2}{#1 \csname math@#2\endcsname}}
\def\refmath{\_ifnextchar[ {\_refmath}{\__refmath}}
\catcode`_=8

\createmathbox{defn}{Definition}{rgb{1 .9 .9}}{rgb{1 .3 .3}}{rgb{.8 .1 .1}}
\createmathbox{thrm}{Theorem}{rgb{.9 .9 1}}{rgb{.3 .3 1}}{rgb{.1 .1 .8}}
\createmathbox{coro}{Corollary}{rgb{.9 .9 1}}{rgb{.3 .3 1}}{rgb{.1 .1 .8}}
\createmathbox{lemm}{Lemma}{rgb{1 .9 1}}{rgb{1 .3 1}}{rgb{.8 .1 .8}}
\createmathbox{prop}{Proposition}{rgb{1 .9 1}}{rgb{1 .3 1}}{rgb{.6 .1 .6}}
\createmathbox{prin}{Principle}{rgb{1 1 .5}}{rgb{.5 .3 0}}{rgb{.5 .3 0}}
\createmathbox{exam}{Example}{rgb{.9 1 .9}}{rgb{.3 .8 .3}}{rgb{.1 .65 .1}}
\createmathbox{exerc}{Exercise}{rgb{.9 1 .9}}{rgb{.3 .8 .3}}{rgb{.1 .65 .1}}

\def\qed{%
    \ifmmode \eqno\mathchar"404%
    \else%
        \hskip.75cm\penalty0\null\nobreak\hfill$\mathchar"404$%
        \par\medskip%
    \fi%
}

\createcounter{eqcount}[subsection]
\def\eqnum{\advancecounter{eqcount}{1}\leqno{\kern\enumindent\relax\hbox{\enumstyle{\the\counter{eqcount}}}}}
\def\Proof{{\bf Proof:} }

\def\zerobox#1{\vbox to\z@{\vss\hbox to\z@{#1\hss}}}
\def\rotatebox#1{{%
    \setbox0=\hbox{#1}%
    \pdfliteral{q 0 -1 1 0 0 0 cm}\zerobox{#1}\pdfliteral{Q}%
}}

\def\mathtowd#1#2{\hbox to#1{\hss$\m@th#2$\hss}}

