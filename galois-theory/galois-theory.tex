\input pdfToolbox

\setlayout{horizontal margin=2cm, vertical margin=2cm}
\parindent=0pt
\parskip=3pt plus 2pt minus 2pt

\input preamble

\footline={}

%%%%%%%%%%%%%%%%%%%%%%%%%%%%%%%%%%%%%%%%%%%%%%%%%%%%%%%%%%%%%%%%

\headline={\pageborder{rgb{1 .5 1}}{rgb{.6 .2 .2}}{5}}

\color rgb{.8 .1 .1}

{\def\boxshadowcolor{rgb{.6 .4 .6}}
\bppbox{rgb{1 .6 1}}{rgb{.6 .1 .1}}{rgb{.4 .1 .1}}

    \centerline{\setfontandscale{bf}{20pt}Fields and Galois Theory}
    \smallskip
    \centerline{\setfont{it}Lectures by Uzi Vishne}
    \centerline{\setfont{it}Summary by Ari Feiglin \setfont{rm}({\tt ari.feiglin@gmail.com})}

\eppbox

\bigskip

\bppbox{rgb{1 .6 1}}{rgb{.6 .1 .1}}{rgb{.4 .1 .1}}
    \section*{Contents}
    
    \tableofcontents
\eppbox

}

\vfill\break

\color{black}

\pageno=1
\newif\ifpageodd
\pageoddtrue
\headline={%
    \hbox to \hsize{\color{black}%
        \ifpageodd\hfil{\it\currsubsection\quad\bf\folio}\global\pageoddfalse%
        \else{\bf\folio\quad\it\currsubsection}\hfil\global\pageoddtrue\fi%
    }%
}

\section{Field Extensions}

Suppose $F\subseteq K$ are fields, then $K$ is certainly also an $F$-vector space and therefore has a dimension and we denote it $[K:F]\coloneqq\dim_FK$.

\bthrm

    Suppose $F\subseteq K$ and $V$ is a $K$-vector space, then $V$ is also a vector space over $F$ as well, and $\dim_FV=[K:F]\dim_KV$.

\ethrm

\Proof Let $B_1\subseteq V$ be a basis for $V$ over $K$ and $B_2\subseteq K$ be a basis for $K$ over $F$, then define $B=\set{\alpha v}[\alpha\in B_2,v\in B_1]$.
This is a basis for $V$ in $F$, it is linearly independent since if $\alpha_1v_1,\dots,\alpha_nv_n\in B$ and $\beta_1,\dots,\beta_n\in F$ then $\sum_{i=1}^n\beta_i\alpha_iv_i=0$ implies $\beta_i\alpha_i=0$
for all $i$ since $B_1$ is a basis, and this means that $\beta_i$ or $\alpha_i$ is zero, but $\alpha_iv_i\in B$ so $\beta_i=0$ as required.
$B$ spans $V$ since for $v\in B$ there exist $v_1,\dots,v_n\in B_1$ and $\alpha_1,\dots,\alpha_n\in K$ such that $v=\sum_{i=1}^n\alpha_iv_i$ and $\alpha_i$ can be written as the linear combination of
elements in $B_2$ by elements of $F$ which gives a linear combination of elements in $B$ of $F$.
So $B$ is indeed a basis for $V$ over $F$.
Finally $B\cong B_2\times B_1$ since $(\alpha,v)\mapsto\alpha v$ is a bijection: it is obviously surjective and $\alpha_1v_1=\alpha_2v_2$ implies $\alpha_1=\alpha_2,v_1=v_2$ since $v_1,v_2$ are independent.
Thus we have
$$ \dim_FV = \abs B = \abs{B_2\times B_1} = [K:F]\dim_K V $$
\qed

In particular if $F\subseteq K\subseteq E$ are fields then $[E:F]=[E:K]\cdot[K:F]$.

The following are methods of constructing fields:
\benum
    \item If $R$ is a commutative ring and $M\triangleleft R$ is a maximal ideal then $\slfrac RM$ is a field.
        Specifically if $R=F[x]$ and $p$ is an irreducible polynomial, $\gen p$ is maximal and $\slfrac{F[x]}{\gen p}$ is a field.
    \item If $F$ is a field, then the set of rational functions is also a field:
        $$ F \subseteq F(x) \coloneqq \set{\frac{f(x)}{g(x)}}[f,g\in F[x],\,g(x)\neq0] $$
        In general if $R$ is an integral domain then its field of fractions/quotients $q(R)\coloneqq\set{\frac ab}[a,b\in R,\,b\neq0]$ is a field.
        And $F(x)$ is the quotient field of $F[x]$.
    \item If $F_0\subseteq F_1\subseteq F_2\subseteq\cdots$ is a chain of fields then so is $\bigcup F_n$ (the theory of fields is inductive, this holds for arbitrary chains, not just inductive ones).
        So for example $F(\lambda_1,\lambda_2,\dots)$ is a field since we can define $F_n=F(\lambda_1,\dots,\lambda_n)$ (the quotient field of $F[\lambda_1,\dots,\lambda_n]$) and the union of this chain
        is $F(\lambda_1,\lambda_2,\dots)$.
\eenum

Let $F$ be a field and $F\subseteq K$ a ring with $a\in K$, we define a homomorphism $F[\lambda]\xvarrightarrow{\,\psi_a\,}K$ defined by $\alpha\mapsto\alpha$ for $\alpha\in F$
and $\lambda\mapsto a$, meaning
$$ \psi_a\parens{\sum\alpha_i\lambda^i} = \sum\alpha_i a^i \qquad (\psi_a(f)=f(a)) $$
In particular $\psi_a$ is a linear transformation from $F$ to $K$, and is called the {\it evaluation homomorphism} at $a$.
The kernel of the homomorphism is
$$ \ker\psi_a = \set{f\in F[\lambda]}[f(a)=0] \triangleleft F[\lambda] $$

\bdefn

    $a\in K$ is {\emphcolor algebraic} if $\ker\psi_a\neq0$ and {\emphcolor transcendental} if the kernel is trivial.

\edefn

If $a$ is transcendental then $\ker\psi_a$ and so $\Im\psi_a=\set{f(a)}[{f\in F[\lambda]}]=F[a]\cong F[\lambda]$.
In fact we get
$$ \vcenter{\ialign{&${}\m@th#{}$\cr
    F &\subseteq& F[a] &\subseteq& F(a) &\subseteq& K\cr 
      &         & \cong &        & \cong\cr
      &         & F[x]  &        & F(x)\cr
    }}
$$

Now if $a$ is algebraic, since $F[x]$ is a euclidean domain and therefore a PID, the kernel has a generator $\ker\psi_a=\gen h=h\cdot F[\lambda]$.
So $h(a)=0$ and $f(a)=0\implies h\divides f$, and $h$ is called the {\it minimal polynomial} of $a$.
And so
$$ \slfrac{F[\lambda]}{\gen h} = \slfrac{F[\lambda]}{\ker\psi_a} \cong \Im\psi_a = \set{f(a)}[{f\in F[\lambda]}] = F[a] = \lspanof{1,a,\dots,a^{n-1}} \subseteq K $$
where $n=\deg h$, since $f(x)=q(x)h(x)+r(x)$ where $\deg r<\deg h=n$ and so $f(a)=r(a)$.
$\set{1,\dots,a^{n-1}}$ is a basis due to $h$ being minimal, a zeroing linear combination would give a zeroing polynomial of $a$ of degree less than $h$.
This means that the dimension of $F[a]$ as an $F$-vector space is $n$, ie. $\bigl[F[a]:F\bigr]=n$.

Since $K$ is an integral domain and therefore so too is $F[a]$ and this means that $\gen h$ is a prime ideal (since $\slfrac RI$ is an integral domain if and only if $I$ is prime), this means
that $h$ is a prime (irreducible) polynomial.
And since $F[a]$ is a PID, prime and maximal ideals are one and the same, so $\gen h$ is maximal and therefore $\slfrac{F[\lambda]}{\gen h}\cong F[a]$ is a field.
Let us summarize this:

\bprop

    Let $F\subseteq K$ where $K$ is an integral domain and $a\in K$ is algebraic in $F$, let $h_a$ be its minimal polynomial.
    Then $(1)$ $h_a$ is irreducible, $(2)$ $F[a]$ is a field, $(3)$ $\bigl[F[a]:F\bigr]=\deg h_a$.

\eprop

So for example let $a\in K\setminus F$ be algebraic then $F\subseteq F[a]\subseteq K$ and suppose $[K:F]=p$ is prime.
Then $p=[K:F]=[K:F[a]]\cdot[F[a]:F]$, and since $a\in F[a]\setminus F$ this means $[F[a]:F]>1$ so $[F[a]:F]=p$ and $[K:F[a]]=1$ since $p$ is prime so $F[a]=K$.

\bcoro

    Suppose $F$ is a field and $F\subseteq K$ is an integral domain with finite dimension.
    Then every element of $K$ is algebraic and $K$ is a field.

\ecoro

\Proof Let $a\in K$ then $[K:F]=[K:F[a]]\cdot[F[a]:F]$ so $[F[a]:F]$ is finite.
If $a$ were transcendental then $F[a]\cong F[x]$ and $F[x]$ has infinite dimension over $F$.
$K$ is a field since every $a\in K$ must have a multiplicative inverse, since $F[a]$ is a field.
\qed

Notice that $[F[a,b]:F[a]]\leq [F[b]:F]$ since if $h_b$ is $b$'s minimal polynomial in $F$ then it is also a zeroing polynomial in $F[a]$.
This means that
$$ [F[a,b]:F] = [F[a,b]:F[a]]\cdot[F[a]:F] \leq [F[b]:F]\cdot[F[a]:F] $$

\bcoro

    Let $F$ be a field and $K$ a field extension, define
    $$ {\rm Alg}_F(K) \coloneqq \set{a\in K}[\hbox{$a$ is algebraic over $F$}] . $$
    This is a field.
    Furthermore $F\subseteq{\rm Alg}_F(K)$ is an algebraic extension (all elements of ${\rm Alg}_F(K)$ are algebraic in $F$), and ${\rm Alg}_F(K)\subseteq K$ is a purely transcendental extension (all
    elements in $K\setminus{\rm Alg}_F(K)$ are transcendental in ${\rm Alg}_F(K)$).

\ecoro

\Proof Notice that $F[a\cdot b],F[a+b]\subseteq F[a,b]$ and so $[F[a,b]:F]\leq[F[b]:F]\cdot[F[a]:F]<\infty$, so ${\rm Alg}_F(K)$ is closed under addition and multiplication (and obviously additive inverses).
For $a$ algebraic, $F[a]$ is a field so $a^{-1}\in F[a]$ and so $F[a^{-1}]\subseteq F[a]$ and therefore $[F[a^{-1}]:F]<\infty$ so $a^{-1}$ is algebraic as well (and so by symmetry $F[a]=F[a^{-1}]$).
So ${\rm Alg}_F(K)$ is indeed a field.

To show that ${\rm Alg}_F(K)\subseteq K$ is a pure transcendental extension, notice that if $F_1\subseteq F_2\subseteq F_3$ where $F_1\subseteq F_2$ is algebraic, if $a\in F_3$ is algebraic in $F_2$
it is also algebraic in $F_1$.
Indeed if $f\in F_2[x]$ such that $f(a)=0$, let its coefficients be $b_i$ then $a$ is algebraic in $F_1[b_0,\dots,b_n]$ and so
$$ [F_1[b_0,\dots,b_n,a]:F_1[b_0,\dots,b_n]] = [F_1[b_0,\dots,b_n,a]:F_1[b_0,\dots,b_n]]\cdot[F_1[b_0,\dots,b_n]:F_1] $$
and this is finite since $b_0,\dots,b_n$ are algebraic in $F_1$ as they are in $F_2$, so both terms are finite.
So if $K$ had any algebraic numbers not in ${\rm Alg}_F(K)$, they would be algebraic in $F$ and thus in ${\rm Alg}_F(K)$ in contradiction.
\qed

\bprop

    Let $F$ be a field and $f\in F[\lambda]$ be irreducible, then there exists a field extension $F\subseteq K$ such that $f$ has a root in $K$, and $[K:F]=\deg f$.

\eprop

\Proof since $f$ is irreducible, $\gen f$ is prime and $F[\lambda]$ is a PID so it is maximal.
So $K\coloneqq\slfrac{F[\lambda]}{\gen f}$ is a field, and its dimension is $\deg f$, since it can be generated by $\set{1,x,\dots,x^{\deg f-1}}$.
Now recall that by the second isomorphism theorem, $\slfrac F{F\cap\gen f}\cong\slfrac{F+\gen f}{\gen f}\subseteq\slfrac{F[\lambda]}{\gen f}=K$.
But since elements of $\gen f$ are multiples of $f$, which is disjoint from $F$, so $F\cap\gen f=(0)$ so $\slfrac F{F\cap\gen f}\cong F$, and so $F$ can be embedded into $K$ and is thus for all intents and
purposes, a subfield of $K$.
Now define $\alpha\coloneqq\lambda+\gen f$, and suppose $f(\lambda)=\sum_{i=0}^n a_i\lambda^i$ where $a_i\in F$ (viewing $f$ as a polynomial over $K$, $a_i$ is actually $a_i+\gen f$).
Then
$$ f(\alpha) = \sum_{i=0}^na_i(\lambda + \gen f)^i = \sum_{i=0}^na_i(\lambda^i+\gen f) = \sum_{i=0}^na_i\lambda^i + \gen f = f + \gen f = \gen f = 0_K $$
so $\alpha$ is indeed a root of $f(\lambda)$, as required.
\qed

\bcoro

    Let $F$ be a field and $f\in F[\lambda]$ any polynomial.
    Then there exists a field extension $F\subseteq K$ such that $f$ has a root in $K$ and $[K:F]\leq\deg f$.

\ecoro

\Proof find $f$'s irreducible factorization $f=f_1\cdots f_t$, then extend $F$ to a field $K$ such that $f_1$ has a root in $K$, and by above $[K:F]=\deg f_1\leq\deg f$.
\qed

\bdefn

    Let $F$ be a field, and $f$ a polynomial over $F$.
    A field $F\subseteq K$ {\emphcolor splits $f$} if there exist $\alpha_1,\dots,\alpha_n\in K$ such that $f(\lambda)=(\lambda-\alpha_1)\cdots(\lambda-\alpha_n)$.

\edefn

\bthrm

    Every polynomial $f$ over a field $F$ has a field $K$ which splits it, such that $[K:F]\leq(\deg f)!$.

\ethrm

\Proof by induction on $n=\deg f$.
For $n=1$ then $f$ already has a root, and so take $F=K$ and $[K:F]=1=(\deg f)!$.
Now suppose $\deg f=n+1$, then by above there exists a field extension $F\subseteq K_0$ such that there exists an $\alpha_1\in K_0$ such that $f(\alpha_1)=0$ and $[K_0:F]\leq\deg f=n+1$.
And so $(\lambda-\alpha_1)\divides f(\lambda)$, so $f(\lambda)=(\lambda-\alpha_1)g(\lambda)$.
Then $\deg g=n$, and $g$ is a polynomial over $K_0$, so there exists a field extension $F\subseteq K_0\subseteq K$ such that $g(\lambda)=(\lambda-\alpha_2)\cdots(\lambda-\alpha_{n+1})$ for $\alpha_i\in K$
and $[K:K_0]\leq n!$.
Then $f(\lambda)=(\lambda-\alpha_1)\cdots(\lambda-\alpha_{n+1})$ for $\alpha_i\in K$ and $[K:F]=[K:K_0][K_0:F]\leq (n+1)n!=(n+1)!$.
\qed

Notice the following
\benum
    \item the split of a polynomial over any field into its roots is unique,
    \item the number of roots is $\leq\deg f$.
\eenum

Recall that a field $F$ is {\it algebraically closed} if it splits every polynomial in $F[\lambda]$.

\bdefn

    Let $F$ be a field, then $F\subseteq\overline F$ is an {\emphcolor algebraic closure} of $F$ if $\overline F$ is algebraically closed.

\edefn

\bnote

    Every field has a unique (up to isomorphism) algebraic closure.

\enote

So let $f(\lambda)\in F[\lambda]$, then $f(\lambda)\in\overline F[\lambda]$ and so $f=(\lambda-\alpha_1)\cdots(\lambda-\alpha_n)$ for $\alpha_i\in\overline F$.
Then take $F\subseteq K=F[\alpha_1,\dots,\alpha_n]\subseteq\overline F$, it can be shown that $[K:F]\leq(\deg f)!$.

Now suppose $F\subseteq K$ are fields, and $E$ is a field which $F$ is embeddable into, suppose $\phi\colon F\longembeds E$ is an embedding.
An embedding $\phi'\colon K\longembeds E$ is an {\it extension} of $\phi$ if $\phi'\bigr|_F=\phi$.
Denote
$$ \eta_{F\subseteq K}^E \coloneqq \#\set{\hbox{$\phi'$ is an extension of $\phi$}} $$
where $\phi$ is held constant and understood.
Then

\bprop

    Suppose $K=F[\alpha]$, then $\eta_{F\subseteq K}^E$ is equal to the number of roots the minimal polynomial of $\alpha$ in $F$ has in $E$.

\eprop

\Proof since $\alpha$ generates $K$ over $F$, every extension of $\phi$ is defined by its image on $\alpha$.
Let $h$ be the minimal polynomial of $\alpha$ over $F$.
Denote $\hat b\coloneqq\phi(b)$ for all $b\in F$, and this definition extends to polynomials, $\varwidehat{\sum_{i=0}^n b_ix^i}=\sum_{i=0}\hat b_ix^i$.
Then if $\phi'$ is an extension of $\phi$,
$$ \hat h(\phi'(\alpha)) = \phi'(h(\alpha)) = \phi'(0) = 0 $$
this is since if $h(\lambda)=\sum_{i=0}^n a_i\lambda^i$, then $\hat h(\lambda)=\sum_{i=0}^n\hat a_i\lambda^i$, so
$$ \hat h(\phi'(\alpha)) = \sum_{i=0}^n\hat a_i\phi'(\alpha)^i = \sum_{i=0}^n\phi(a_i)\phi'(\alpha)^i = \sum_{i=0}^n\phi'(a_i)\phi'(\alpha)^i = \phi'\parens{\sum_{i=0}^n a_i\alpha^i} = \phi'(h(\alpha)) $$
so $\phi'(\alpha)$ must be one of $\hat h$'s roots, precisely as stated.
\qed

\bdefn

    A polynomial $f$ which splits over $E$ is called {\emphcolor separable} in $E$ if its linear factors are distinct (ie. all of its roots in $E$ are distinct).

\edefn

\bthrm

    Let $F\subseteq K$ be a finite extension (meaning $[K:F]<\infty$), and let $\phi\colon F\longembeds E$ be a given embedding.
    Then
    \benum
        \item $\eta_{F\subseteq K}^E\leq[K:F]$,
        \item if $K$ is generated by the roots of $f$, assuming that $E$ splits $f$, then $1\leq\eta_{F\subseteq K}^E$,
        \item if $f$ is separable over $E$, then $\eta_{F\subseteq K}^E=[K:F]$.
    \eenum

\ethrm

\Proof suppose $K=F[\alpha_1,\dots,\alpha_n]$ (the generators of $K$ can be taken to be the basis of $K$ as an $F$-vector space).
We prove this by induction on $n$, for $n=1$ this is given by the previous proposition, since $\eta_{F\subseteq K}^E$ is the number of roots $h$ has in $E$, and $[K:F]=\deg h$ which is at least this.
Define $F_1\coloneqq F[\alpha_1]$, then
$$ \eqalign{
    \eta_{F\subseteq K}^E &= \#\set{\hbox{$\phi''\colon K\longto E$ is an extension of $\phi$}} \cr
    &= \#\bigcup\set{{\hbox{$\phi''\colon F_1\longto E$ is an extension of $\phi'$}}}[\hbox{$\phi'\colon F_1\longto E$ is an extension of $\phi$}]\cr
    &= \sum_{\phi'}\eta_{F_1\subseteq K}^E = \eta_{F\subseteq F_1}^E\cdot\eta_{F_1\subseteq K}^E \subseteq [F_1:F]\cdot[K:F_1] = [K:F]
} $$

For $(2)$, by the assumption there is an extension of $F\longembeds E$ to $F_1\longembeds E$, and continue inductively.
For $(3)$, since $f$ is separable, makes the bound an equality.
\qed

\bdefn

    Let $f$ be a polynomial over $F$, a field $F\subseteq K$ is a {\emphcolor splitting field} if it is the smallest field in which the polynomial splits.

\edefn

Notice that if $K$ is a splitting field, it is of the form $K=F[\alpha_1,\dots,\alpha_n]$ where $\alpha_i$ are roots of the polynomial, so they are algebraic.
This means that $[K:F]\leq\prod_i[F:\alpha_i]<\infty$.

Furthermore, if $K$ is a splitting field of $f$, then it is generated by the roots of $f$: $K=F[\alpha_1,\dots,\alpha_n]$, then if $E$ is any field which splits $f$, we have $\eta^E_{F\subseteq K}\geq1$,
meaning there exists an embedding $K\embeds E$ which extends the embedding $F\embeds E$.
And in particular if $K,K'$ are two splitting fields of $f$, there exists two embeddings $K\embeds K'$ and $K'\embeds K$, which means $[K:F]=[K':F]$ and so $K$ and $K'$ are isomorphic as $F$-vector spaces.
And so $K\cong K'$ as fields.

Recall that there exists a unique ring homomorphism $f\colon{\bb Z}\longto F$, and $\slfrac{{\bb Z}}{\ker f}\cong\Im f\subseteq F$.
Since $\Im f$ is a subring of $F$, it is an integral domain and so $\ker f$ is a prime ideal.
Thus $\ker f=p{\bb Z}$ for $p$ prime or $0$, and this $p$ is called {\it $F$'s characteristic}.
In other words $F$ has characteristic $p$ if and only if $1+\cdots+1=0$ ($p$ times) since then $p\in\ker f$ and so $(p)\subseteq\ker f$, but ${\bb Z}$ is a PID and so $(p)$ is maximal.
And $F$ has characteristic $0$ if $1+\cdots+1$ is never zero.

If $F$ has characteristic $0$, then $f$ is an embedding into $F$, so ${\bb Z}\subseteq F$ and since it is a field ${\bb Q}\subseteq F$, up to embedding.
And for characteristic $p$, $\slfrac{{\bb Z}}{p{\bb Z}}={\bb F}_p\subseteq F$.

Notice that in characteristic $p$, $\binom pk=\frac{p!}{k!(p-k)!}$ is zero for $k\neq0,p$.
$$ (a+b)^p = \sum_{k=0}^p\binom pka^kb^{p-k} = a^p + b^p $$
And so $e(x)=x^p$ is a field homomorphism $F\longto F^p=\set{x^p}[x\in F]$, and it has a trivial kernel, and so $F\cong\slfrac F{\ker f}\cong F^p$.

\bdefn

    We define the {\emphcolor derivative} over a field $F$ to be the function $F[\lambda]\longto F[\lambda]$ defined by
    $$ \parens{\sum_{i=0}^n\alpha_i\lambda^i}' = \sum_{i=1}^n\alpha_i\cdot i\lambda^{i-1} $$

\edefn

It is trivial to show that $(f+g)'=f'+g'$ and $(fg)'=fg+f'g$, meaning that $(f^2g)'=f^2g'+2ff'g$.
This means that if $f^2\divides h$ then $f\divides h'$.
In particular if $f$ is not separable, then there exists some $(\lambda-\alpha)^2$ which divides $f$ over a field which splits it, then $\lambda-\alpha$ divides $f'$, meaning $f'(\alpha)=0$.
But this means that $f'=0$, so $\alpha_ii=0$ for all $i$, and so if $p$ doesn't divide $i$ this means $i\neq0$ so $\alpha_i=0$.
Thus
$$ f(\lambda) = \sum_{p\divides i}\alpha_i\lambda^i = \sum_j\alpha_{pj}(\lambda^p)^j $$
So we get that

\bprop

    Let $f$ be irreducible over a field of characteristic $p>0$, then $f$ is not separable if and only if $f'=0$ if and only if $f(\lambda)=g(\lambda^p)$ for some polynomial $g$.

\eprop

\bexam

    Let $\lambda^p-a$ be a polynomial over $F$ of characteristic $p$, and $\alpha$ a root in a field which splits it.
    Then
    $$ \lambda^p-a = \lambda^p-\alpha^p = (\lambda-\alpha)^p $$
    so $\lambda^p-a$ is not separable (which we can see since it is $g(\lambda^p)$ for $g(\lambda)=\lambda-a$).

\eexam

\bdefn

    Let $K/F$ be a field extension (meaning $F\subseteq K$), then an automorphism of $K$ over $F$ is an automorphism $\sigma\colon K\longto K$ which holds $F$ constant: $\sigma(a)=a$ for all
    $a\in F$.

\edefn

Notice that all field homomorphisms are either injective or trivial, since the kernel is an ideal and fields only have trivial ideals, so if $\sigma$ is a field homomorphism there is no need
to check injectivity.
And $\sigma(ax)=\sigma(a)\sigma(x)=a\sigma(x)$ for $a\in F$ and $x\in K$ so $\sigma$ is an $F$-linear transformation, so if $[K:F]$ is finite $\sigma$ must be surjective.
Thus in the case that $K/F$ is a finite field extension, all monomorphisms of $K$ over $F$ are automorphisms.

\bdefn

    Let $K/F$ be a field extension, then define its {\emphcolor Galois group} to be
    $$ \Galof{K/F} \coloneqq \set{\sigma}[\hbox{$\sigma$ is an automorphism of $K$ over $F$}] $$
    and this is indeed a group relative to composition.

\edefn

Notice that if $K/F$ is a field extension and $\alpha\in K$ algebraic.
Let $h$ be its minimal polynomial and $\sigma\in\Galof{K/F}$, then
$$ h(\sigma(\alpha)) = \sigma(h(\alpha)) = \sigma(0) = 0 $$
This is since $\sigma\parens{\sum_ia_i\alpha^i}=\sum_i\sigma(a_i)\sigma(\alpha)^i=\sum_ia_i\sigma(\alpha)^i=h(\sigma(\alpha))$.
So permutations in Galois groups map roots to roots of polynomials.

So for example, let $G=\Galof{{\bb Q}[\sqrt3]/{\bb Q}}$ and $\lambda^2-3=(\lambda-\sqrt3)(\lambda+\sqrt3)$ and so $\sigma$ must map $\sqrt3$ to $\pm\sqrt3$.
And since all automorphisms of ${\bb Q}[\sqrt3]$ over ${\bb Q}$ are defined by $\sqrt3$'s image,
$$ G = \set{1,\sqrt3\xvarmapsto{\sigma}-\sqrt3} \cong {\bb Z}_2 $$
And similarly let $G=\Galof{{\bb Q}[\sqrt3,\sqrt2]/{\bb Q}}$, $\sqrt3$ must be mapped to $\pm\sqrt3$ (due to $\lambda^2-3$) and $\sqrt2$ must be mapped to $\pm\sqrt3$, so
$$ G = \set{1,\,\stackmath{\sqrt2\mapsto\sqrt2\cr\sqrt3\mapsto-\sqrt3},\,\stackmath{\sqrt2\mapsto-\sqrt2\cr\sqrt3\mapsto\sqrt3},\,\stackmath{\sqrt2\mapsto-\sqrt2\cr\sqrt3\mapsto-\sqrt3}} \cong
{\bb Z}_2\times{\bb Z}_2 $$

Notice that if $K$ has characteristic $p$, every automorphism must keep elements of ${\bb F}_p$ constant (since $\sigma(1)=1$).
And if $K$ has characteristic $0$, every automorphism must keep elements of ${\bb Q}$ constant (since $\sigma(a/b)=\sigma(a)/\sigma(b)=a/b$).
So let $F_0$ be the characteristic field of $K$ (either ${\bb F}_p$ or ${\bb Q}$), so
$$ \Aut(k) = \Galof{K/F_0} $$

\bdefn

    Let $K$ be a field, then for every subfield $G\leq\Autof K$, define the {\emphcolor fixed-point field},
    $$ K^G \coloneqq \set{a\in K}[\forall\sigma\in G\colon \sigma(a)=a] $$
    This is indeed a field.

\edefn

Notice that if $F\subseteq K$ is a subfield, then $\Galof{K/F}$ is a subgroup of $\Aut(K)$.
And if $G\leq\Aut(K)$ is a subgroup, then $K^G$ is a subfield of $K$.
So we have the following correspondences:

\penalty-100\medskip
{\tabskip=0pt plus1fil
\offinterlineskip\halign to\hsize{$#$\hfil\tabskip=.25cm&\hfil$#$\hfil\tabskip=.25cm&$#$\hfil\tabskip=0pt plus1fil\cr
& \xvarmapsto{\mathtowd{3cm}{\Galof{K,\bullet}}}\cr
\set{\hbox{Subgroups of $\Autof K$}} & &\set{\hbox{Subfields of $K$}}\cr
& \xvarmapsfrom{\mathtowd{3cm}{K^{\textstyle\bullet}}}\cr
}}
\medskip

And if $F\subseteq K$ is a subfield, and $F\subseteq L\subseteq K$ is a field between them, $\Galof{K,L}$ is a subgroup of $\Galof{K/F}$ (since $\sigma\in\Galof{K/L}$ keeps elements of $L$, and thus $F$
constant).
And if $G\leq\Galof{K/F}$, then $K^G$ is a field between $F$ and $K$.
So we have

\medskip
{\tabskip=0pt plus1fil
\offinterlineskip\halign to\hsize{$#$\hfil\tabskip=.25cm&\hfil$#$\hfil\tabskip=.25cm&$#$\hfil\tabskip=0pt plus1fil\cr
& \xvarmapsto{\mathtowd{3cm}{\Galof{K,\bullet}}}\cr
\set{\hbox{Subgroups of $\Galof{K/F}$}} & &\set{\hbox{Fields between $F$ and $K$}}\cr
& \xvarmapsfrom{\mathtowd{3cm}{K^{\textstyle\bullet}}}\cr
}}
\medskip

Some properties:
\benum
    \item If $L_2\subseteq L_1$ then $\Galof{K/L_2}\supseteq\Galof{K/L_1}$ since an automorphism which keeps elements of $L_1$ constant keeps elements of $L_1$ constant.
    \item If $H_2\subseteq H_1$ then $K^{H_2}\supseteq K^{H_1}$ since if $a$ is held constant by every $\sigma\in H_1$, it is held constant by every $\sigma\in H_2$.
    \item For every $L$, $L\subseteq K^{\Galof{K/L}}$ since $K^{\Galof{K/L}}$ are elements held constant by every automorphism in $\Galof{K/L}$, which includes all elements of $L$ by definition.
    \item For every $H$, $H\subseteq\Galof{K/K^H}$ since for $\sigma\in H$ every element of $K^H$ is held constant.
\eenum

\bdefn

    Let $X,Y$ be posets (partially ordered sets), then a pair of functions $\alpha\colon X\longto Y$ and $\beta\colon Y\longto X$ is an {\emphcolor Galois correspondence} if
    \benum
        \item $\alpha$ and $\beta$ reverse order, meaning if $x_1\leq x_2$ then $\alpha(x_2)\leq\alpha(x_1)$ and similar for $\beta$,
        \item for every $x\in X$ and $y\in Y$, $x\leq\beta(\alpha(x))$ and $y\leq\alpha(\beta(y))$.
    \eenum
\edefn

For example, let $X$ and $Y$ both be the lattice of subgroups of a group $G$, $\alpha=\beta\colon H\mapsto C_G(H)$.
But our important example is $\alpha\colon F\mapsto\Galof{K/F}$ and $\beta\colon H\mapsto K^H$.

\blemm

    $\alpha,\beta$ form a Galois correspondence if and only if for all $x\in X$ and $y\in Y$ $y\leq\alpha(x)\iff x\leq\beta(y)$.

\elemm

\Proof suppose $\alpha,\beta$ form a Galois correspondence.
If $x\leq\beta(y)$ then $y\leq\alpha(\beta(y))\leq\alpha(x)$ and similar for $\beta$, so we get the desired result.
Now suppose $y\leq\alpha(x)\iff x\leq\beta(y)$.
Since $\beta(y)\leq\beta(y)$, we get $y\leq\alpha(\beta(y))$, similar for $\beta(\alpha(x))$.
And if $x\leq x'$ then $x\leq x'\leq\beta(\alpha(x'))=\beta(y)$ which is equivalent to $\alpha(x')=y\leq\alpha(x)$.
Similar for $\beta$.
\qed

\bprop

    Let $\alpha,\beta$ be a Galois correspondence.
    Then
    \benum
        \item $\alpha\circ\beta\circ\alpha=\alpha$ and $\beta\circ\alpha\circ\beta=\beta$.
        \item $\beta(\alpha(x))=x$ if and only if $x\in\beta(Y)$ and $\alpha(\beta(y))=y$ if and only if $y\in\alpha(X)$.
        \item $\alpha$ and $\beta$ are inverses as functions between $\beta(Y)$ and $\alpha(X)$.
    \eenum

\eprop

\Proof 
\benum
    \item Since $x\leq\beta(\alpha(x))$, we get $\alpha\beta\alpha(x)\leq\alpha x$.
    On the other hand let $y=\alpha(x)$ then $y\leq\alpha\beta y=\alpha\beta\alpha(x)$, so we have equality.
    \item This is direct from $(1)$, since if $\alpha\beta(y)=y$ then trivially $y\in\alpha(X)$, and if $y\in\alpha(X)$ then $y=\alpha(x)$ so $\alpha\beta(y)=\alpha\beta\alpha(x)=\alpha(x)=y$.
    \item This is direct from $(2)$.
    \qed
\eenum

In particular $\alpha(X)$ is isomorphic to the reverse order of $\beta(X)$, $\alpha(X)\cong\beta(X)^{\rm op}$.

\bdefn

    A field extension $K/F$ is a {\emphcolor separable extension} if the minimal polynomial of every $a\in K$ over $F$ is separable (meaning $f$ splits into distinct linear factors over its splitting field).
    And it is a {\emphcolor normal extension} if the minimal polynomial of every $a\in K$ over $F$ splits over $K$.
    Equivalently for every irreducible polynomial $f$ over $F$, if $f$ has a root in $K$ then $f$ splits in $K$.
    If it is both a normal and separable extension then it is called a {\emphcolor Galois extension}.

\edefn

\bthrm

    Let $K/F$ be a finite field extension, then the following are equivalent:
    \benum
        \item $K/F$ is a Galois extension,
        \item $K$ is the splitting field of a separable polynomial over $F$,
        \item $F=K^G$ for some $G\leq\Autof K$,
        \item $F=K^{\Galof{K/F}}$,
        \item $\abs{\Galof{K/F}}=[K:F]$.
    \eenum

\ethrm

\Proof $(1)\implies(2)$: suppose $K=F[a_1,\dots,a_n]$, then since $K/F$ is separable the minimal polynomial $f_i$ of every $a_i$ is separable (meaning its linear factors are distinct in its splitting field).
By normality, since $f_i$ has a root in $K$ it splits, and the factors must be distinct.
Define $f=\prod f_i$, which is separable and splits over $K$.
$K$ must be the splitting field of $f$ since $f$ splits into distinct linear terms over $K$ and $K$ is generated from its roots.

$(2)\implies(5)$: we showed that if $K$ is generated by the roots of $f$ which has a splitting field $E$, then $\eta_{F\subseteq K}^E=[K:F]$.
Take $E=K$ so $\eta_{F\subseteq K}^K=[K:F]$.
Extensions of $F\embeds K$ to $K\embeds K$ are simply automorphisms which hold $F$ constant (since field homomorphisms are either injective or trivial, and if it holds $F$ constant it cannot be trivial).
Thus $\eta_{F\subseteq K}^K=\abs{\Galof{K/F}}$, so we have $\abs{\Galof{K/F}}=[K:F]$.

$(2)\implies(4)$: let $F'=K^{\Galof{K/F}}$ then by the Galois correspondence, $F\subseteq F'$.
We now that $(2)\implies(5)$ so $\abs{\Galof{K/F'}}=[K:F']$ and $\abs{\Galof{K/F}}=[K:F]$
Since $\Galof{K/F'}=\alpha\beta\alpha(F)$ we know that $\Galof{K/F'}=\Galof{K/F}$ so $[K:F]=[K:F']$ and $F\subseteq F'$ so $F=F'$.

$(4)\implies(3)$ is trivial.

$(3)\implies(1)$: let $a\in K$ and let $g$ be its minimal polynomial in $F$.
Let $a_1,\dots,a_k$ be its roots in $K$, then let $h=\prod(\lambda-a_i)\in K[\lambda]$, so $h\divides g$ in $K[\lambda]$.
Now let $\sigma\in G$, this will permute a root of $g$ to another root of $g$ which is in $K$, we have $h\in K^G[\lambda]=F[\lambda]$ (since $a_i$ is mapped to $a_j$), we then get that $h\divides h$ in $F$.
But $g$ is the minimal polynomial so $h=g$, meaning the minimal polynomial of every $a\in K$ splits into distinct linear terms.

$(5)\implies(4)$: let $G=\Galof{K/F}$ and $F'=K^G$, then it satisfies the condition for $(3)$, which implies $(1)$ which implies $(5)$, so we get $\abs{\Galof{K/F'}}=[K:F']$.
Again since $\Galof{K/F'}=\alpha\beta\alpha(F)=\Galof{K/F}$, we get $[K:F]=[K:F']$ and $F\subseteq F'$ so $F=F'$.
\qed

Notice that if $F\subseteq L\subseteq K$ are fields such that $K/F$ is a Galois extension, then $K/L$ is also a Galois extension, since if the minimal polynomial of $a\in K$ over $F$ splits, then since the
minimal polynomial of $a$ in $L$ divides it, it must also split.
Thus
$$ K^{\Galof{K/L}} = L $$
So if we look at our previous diagram

\penalty-100\medskip
{\tabskip=0pt plus1fil
\offinterlineskip\halign to\hsize{$#$\hfil\tabskip=.25cm&\hfil$#$\hfil\tabskip=.25cm&$#$\hfil\tabskip=0pt plus1fil\cr
& \xvarmapsto{\mathtowd{3cm}{\alpha=\Galof{K,\bullet}}}\cr
\set{\hbox{Subgroups of $\Galof{K/F}$}} & &\set{\hbox{Fields between $F\subseteq K$}}\cr
& \xvarmapsfrom{\mathtowd{3cm}{\beta=K^{\textstyle\bullet}}}\cr
}}
\medskip

We have that $\beta\alpha=1$, meaning we know that for every $F\subseteq L\subseteq K$ there exists a subgroup of $\Galof{K/F}$ such that $K^G=L$.
But for which subgroups $H\leq G$ is there a field $F\subseteq L\subseteq K$ such that $\Galof{K/L}=G$?

\blemm[title=Artin's Lemma, name=artinlemma]

    Let $H\leq\Autof K$ be a finite subgroup, then $[K:K^H]\leq\abs H$

\elemm

\Proof let $H=\set{\sigma_1=1,\sigma_2,\dots,\sigma_n}$, $n<m$, and $x_1,\dots,x_m\in K$.
We need to show that $x_1,\dots,x_m$ are linearly dependent in $K$ over $K^H$.
So we'd like to find $a_1,\dots,a_m\in K^H$ such that $\sum_ia_ix_i=0$.
Applying $\sigma_i\in H$, since $a_j\in K^H$ we have that by definition $\sigma_ia_j=a_j$ so
$$ \sigma_i\parens{\sum_ja_jx_j} = \sum_ja_j\sigma_i(x_j) = 0 $$
Let $X$ be an $n\times m$ matrix defined by $X=(\sigma_i(x_j))_{ij}$ and $\vec a=(a_1,\dots,a_m)^\top$.
Then we need to solve for $\vec a$ in
$$ X\vec a = 0 $$
But $X$ is an $M_{n\times m}(K)$ matrix, so it must have a nontrivial nullspace, meaning there exists a solution $\vec a\neq0$ in $K$.
Recall that our goal is to find such a $\vec a$ in $K^H$.

Let us choose one non-trivial solution $\vec a$ such that its number of zeroes is minimal (meaning $\#\set{1\leq i\leq m}[a_i=0]$ is minimal).
We can reorder the solutions to assume that $a_1\neq0$, and since $a_1^{-1}\vec a$ is also a solution, we can assume $a_1=1$.
Now we claim that $a_i\in K^H$ for all $1\leq i\leq m$, and so once we prove this we have finished.
Suppose that $a_2\notin K^H$, then there exists a $\sigma_k\in H$ such that $\sigma_k(a_2)\neq a_2$.
Then we know that $\sum_ja_j\sigma_i(x_j)=0$ for all $i$, so compose this with $\sigma_k$ to get
$$ \sum_j\sigma_k(a_j)\sigma_k(\sigma_i(x_j)) = 0 $$
since $\sigma_k\sigma_i\in H$, this is just a permutation of the indexing of $i$, so we still have
$$ \sum_j\sigma_k(a_j)\sigma_k(x_j) = 0 $$
meaning $(1,\sigma_k(a_2),\dots,\sigma_k(a_m))$ is another solution to the system of equations, and since the set of solutions form a vector space, this means that
$(0,a_2-\sigma_k(a_2),\dots,a_m-\sigma_k(a_m))$ is a solution.
This is a non-trivial solution, but it has fewer zeroes than $\vec a$ since now the first coefficient is zero (and all zero coefficients in $\vec a$ remain zero here), in contradiction.
\qed

Since $K^H$ is Galois by the above theorem, we have $[K:K^H]=\abs{\Galof{K/K^H}}$.
Now, $H\subseteq\Galof{K/K^H}$ by the definition of a Galois correspondence, and so we have $\abs H\leq\abs{\Galof{K/K^H}}=[K:K^H]\leq\abs H$.
Thus $H=\Galof{K/K^H}$.

And so we have shown that $H\mapsto K^J$ and $L\mapsto\Galof{K/L}$ are inverse functions:

\bthrm[title=The Fundamental Theorem of Galois Theory, name=ftgalois]

    Let $K/F$ be a finite-dimensional field extension.
    Then the maps $H\mapsto K^J$ and $L\mapsto\Galof{K/L}$ are inverse functions which invert order between fields in between $F$ and $K$ and subgroups of $\Galof{K/F}$.

\ethrm

\bcoro

    For every finite-dimensional Galois field extension, there is a finite number of in-between fields.

\ecoro

\Proof since the number of in-between fields is equal to the number of subgroups of $\Galof{K/F}$, we need simply to show that $\Galof{K/F}$ is finite.
This is since $\abs{\Galof{K/F}}=[K:F]$ since $K/F$ is Galois, and by assumption this is finite.
\qed

\bcoro

    Let $G=\Galof{K/F}$, then $H$ is normal in $G$ if and only if $\sigma(K^H)=K^H$ for every $\sigma\in G$.

\ecoro

\Proof if $H$ is normal in $G$, then let us consider the map $\sigma\mapsto\sigma\bigl|_{K^H}$ from $\Galof{K/F}$ to $\Galof{K^H/F}$.
The kernel of this is the set of all permutations which hold $K^H$ constant, $\Galof{K/K^H}$ which is just $H$.
So $H$ is therefore normal.
Now notice that
\multlines{
    K^{\sigma H\sigma^{-1}} = \set{x\in K}[\forall h\in H\colon \sigma h\sigma^{-1}(x)=x] = \set{\sigma(y)}[\forall h\in H\colon \sigma h(y)=\sigma(y)]\cr
    &= \sigma\set{y\in K}[\forall h\in H\colon h(y)=y] = \sigma(K^H)
}
so if $\sigma(K^H)=K^H$ then $K^{\sigma H\sigma^{-1}}=K^H$, meaning $H=\sigma H\sigma^{-1}$ (the $H\mapsto K^H$ map is injective; it has an inverse).
\qed

Notice that we defined a homomorphism $\Galof{K/F}\longto\Galof{K^H/F}$ whose kernel is $\Galof{K/K^H}$, thus
$$ \Galof{K^H/F} \cong \slfrac{\Galof{K/F}}{\Galof{K/K^H}} $$

\bprop

    If $K/F$ is Galois and $G=\Galof{K/F}$, then $H\leq G$ is normal if and only if $K^H/F$ is a normal extension.

\eprop

\Proof let $L=K^H$, $a\in L$, and $h$ be its minimum polynomial over $F$.
Thus it splits in $K$, meaning $h(\lambda)=\prod(\lambda-a_i)$ for $a_i\in K$.
We showed previously that $\sigma\in G$ permutes the roots of $h$.
If $H$ is normal, then $\sigma(K^H)=K^H$ meaning $\sigma(a)\in L$.
So if there is an $a_i$ not in $L$, then we can define a permutation $a\mapsto a_i$, but then $\sigma(a)\notin L$ in contradiction.
So all $a_i$ are in $L$, meaning $h$ splits in $L$, so $L/F$ is normal.

And if $L/F$ is normal, then $a_i\in L$ so $\sigma(a)=a_i$ for some $i$, since again $G$ permutes the roots, and so $\sigma(a)\in L$.
Thus $\sigma(L)=L$, meaning $H$ is normal.
\qed

\bye

