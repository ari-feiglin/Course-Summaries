\input pdfToolbox

\setlayout{horizontal margin=2cm, vertical margin=2cm}
\parindent=0pt
\parskip=3pt plus 2pt minus 2pt

\input preamble

\footline={}

%%%%%%%%%%%%%%%%%%%%%%%%%%%%%%%%%%%%%%%%%%%%%%%%%%%%%%%%%%%%%%%%

\headline={\pageborder{rgb{1 .5 1}}{rgb{.6 .2 .2}}{5}}

\color rgb{.8 .1 .1}

{\def\boxshadowcolor{rgb{.6 .4 .6}}
\bppbox{rgb{1 .6 1}}{rgb{.6 .1 .1}}{rgb{.4 .1 .1}}

    \centerline{\setfontandscale{bf}{20pt}Fields and Galois Theory}
    \smallskip
    \centerline{\setfont{it}Lectures by Uzi Vishne}
    \centerline{\setfont{it}Summary by Ari Feiglin \setfont{rm}({\tt ari.feiglin@gmail.com})}

\eppbox

\bigskip

\bppbox{rgb{1 .6 1}}{rgb{.6 .1 .1}}{rgb{.4 .1 .1}}
    \section*{Contents}
    
    \tableofcontents
\eppbox

}

\vfill\break

\color{black}

\pageno=1
\newif\ifpageodd
\pageoddtrue
\headline={%
    \hbox to \hsize{\color{black}%
        \ifpageodd\hfil{\it\currsubsection\quad\bf\folio}\global\pageoddfalse%
        \else{\bf\folio\quad\it\currsubsection}\hfil\global\pageoddtrue\fi%
    }%
}

\section{Field Extensions}

Suppose $F\subseteq K$ are fields, then $K$ is certainly also an $F$-vector space and therefore has a dimension and we denote it $[K:F]\coloneqq\dim_FK$.

\bthrm

    Suppose $F\subseteq K$ and $V$ is a $K$-vector space, then $V$ is also a vector space over $F$ as well, and $\dim_FV=[K:F]\dim_KV$.

\ethrm

\Proof Let $B_1\subseteq V$ be a basis for $V$ over $K$ and $B_2\subseteq K$ be a basis for $K$ over $F$, then define $B=\set{\alpha v}[\alpha\in B_2,v\in B_1]$.
This is a basis for $V$ in $F$, it is linearly independent since if $\alpha_1v_1,\dots,\alpha_nv_n\in B$ and $\beta_1,\dots,\beta_n\in F$ then $\sum_{i=1}^n\beta_i\alpha_iv_i=0$ implies $\beta_i\alpha_i=0$
for all $i$ since $B_1$ is a basis, and this means that $\beta_i$ or $\alpha_i$ is zero, but $\alpha_iv_i\in B$ so $\beta_i=0$ as required.
$B$ spans $V$ since for $v\in B$ there exist $v_1,\dots,v_n\in B_1$ and $\alpha_1,\dots,\alpha_n\in K$ such that $v=\sum_{i=1}^n\alpha_iv_i$ and $\alpha_i$ can be written as the linear combination of
elements in $B_2$ by elements of $F$ which gives a linear combination of elements in $B$ of $F$.
So $B$ is indeed a basis for $V$ over $F$.
Finally $B\cong B_2\times B_1$ since $(\alpha,v)\mapsto\alpha v$ is a bijection: it is obviously surjective and $\alpha_1v_1=\alpha_2v_2$ implies $\alpha_1=\alpha_2,v_1=v_2$ since $v_1,v_2$ are independent.
Thus we have
$$ \dim_FV = \abs B = \abs{B_2\times B_1} = [K:F]\dim_K V $$
\qed

In particular if $F\subseteq K\subseteq E$ are fields then $[E:F]=[E:K]\cdot[K:F]$.

The following are methods of constructing fields:
\benum
    \item If $R$ is a commutative ring and $M\triangleleft R$ is a maximal ideal then $\slfrac RM$ is a field.
        Specifically if $R=F[x]$ and $p$ is an irreducible polynomial, $\gen p$ is maximal and $\slfrac{F[x]}{\gen p}$ is a field.
    \item If $F$ is a field, then the set of rational functions is also a field:
        $$ F \subseteq F(x) \coloneqq \set{\frac{f(x)}{g(x)}}[f,g\in F[x],\,g(x)\neq0] $$
        In general if $R$ is an integral domain then its field of fractions/quotients $q(R)\coloneqq\set{\frac ab}[a,b\in R,\,b\neq0]$ is a field.
        And $F(x)$ is the quotient field of $F[x]$.
    \item If $F_0\subseteq F_1\subseteq F_2\subseteq\cdots$ is a chain of fields then so is $\bigcup F_n$ (the theory of fields is inductive, this holds for arbitrary chains, not just inductive ones).
        So for example $F(\lambda_1,\lambda_2,\dots)$ is a field since we can define $F_n=F(\lambda_1,\dots,\lambda_n)$ (the quotient field of $F[\lambda_1,\dots,\lambda_n]$) and the union of this chain
        is $F(\lambda_1,\lambda_2,\dots)$.
\eenum

Let $F$ be a field and $F\subseteq K$ a ring with $a\in K$, we define a homomorphism $F[\lambda]\xvarrightarrow{\,\psi_a\,}K$ defined by $\alpha\mapsto\alpha$ for $\alpha\in F$
and $\lambda\mapsto a$, meaning
$$ \psi_a\parens{\sum\alpha_i\lambda^i} = \sum\alpha_i a^i \qquad (\psi_a(f)=f(a)) $$
In particular $\psi_a$ is a linear transformation from $F$ to $K$, and is called the {\it evaluation homomorphism} at $a$.
The kernel of the homomorphism is
$$ \ker\psi_a = \set{f\in F[\lambda]}[f(a)=0] \triangleleft F[\lambda] $$

\bdefn

    $a\in K$ is {\emphcolor algebraic} if $\ker\psi_a\neq0$ and {\emphcolor transcendental} if the kernel is trivial.

\edefn

If $a$ is transcendental then $\ker\psi_a$ and so $\Im\psi_a=\set{f(a)}[{f\in F[\lambda]}]=F[a]\cong F[\lambda]$.
In fact we get
$$ \vcenter{\ialign{&${}\m@th#{}$\cr
    F &\subseteq& F[a] &\subseteq& F(a) &\subseteq& K\cr 
      &         & \cong &        & \cong\cr
      &         & F[x]  &        & F(x)\cr
    }}
$$

Now if $a$ is algebraic, since $F[x]$ is a euclidean domain and therefore a PID, the kernel has a generator $\ker\psi_a=\gen h=h\cdot F[\lambda]$.
So $h(a)=0$ and $f(a)=0\implies h\divides f$, and $h$ is called the {\it minimal polynomial} of $a$.
And so
$$ \slfrac{F[\lambda]}{\gen h} = \slfrac{F[\lambda]}{\ker\psi_a} \cong \Im\psi_a = \set{f(a)}[{f\in F[\lambda]}] = F[a] = \lspanof{1,a,\dots,a^{n-1}} \subseteq K $$
where $n=\deg h$, since $f(x)=q(x)h(x)+r(x)$ where $\deg r<\deg h=n$ and so $f(a)=r(a)$.
$\set{1,\dots,a^{n-1}}$ is a basis due to $h$ being minimal, a zeroing linear combination would give a zeroing polynomial of $a$ of degree less than $h$.
This means that the dimension of $F[a]$ as an $F$-vector space is $n$, ie. $\bigl[F[a]:F\bigr]=n$.

Since $K$ is an integral domain and therefore so too is $F[a]$ and this means that $\gen h$ is a prime ideal (since $\slfrac RI$ is an integral domain if and only if $I$ is prime), this means
that $h$ is a prime (irreducible) polynomial.
And since $F[a]$ is a PID, prime and maximal ideals are one and the same, so $\gen h$ is maximal and therefore $\slfrac{F[\lambda]}{\gen h}\cong F[a]$ is a field.
Let us summarize this:

\bprop

    Let $F\subseteq K$ where $K$ is an integral domain and $a\in K$ is algebraic in $F$, let $h_a$ be its minimal polynomial.
    Then $(1)$ $h_a$ is irreducible, $(2)$ $F[a]$ is a field, $(3)$ $\bigl[F[a]:F\bigr]=\deg h_a$.

\eprop

So for example let $a\in K\setminus F$ be algebraic then $F\subseteq F[a]\subseteq K$ and suppose $[K:F]=p$ is prime.
Then $p=[K:F]=[K:F[a]]\cdot[F[a]:F]$, and since $a\in F[a]\setminus F$ this means $[F[a]:F]>1$ so $[F[a]:F]=p$ and $[K:F[a]]=1$ since $p$ is prime so $F[a]=K$.

\bcoro

    Suppose $F$ is a field and $F\subseteq K$ is an integral domain with finite dimension.
    Then every element of $K$ is algebraic and $K$ is a field.

\ecoro

\Proof Let $a\in K$ then $[K:F]=[K:F[a]]\cdot[F[a]:F]$ so $[F[a]:F]$ is finite.
If $a$ were transcendental then $F[a]\cong F[x]$ and $F[x]$ has infinite dimension over $F$.
$K$ is a field since every $a\in K$ must have a multiplicative inverse, since $F[a]$ is a field.
\qed

Notice that $[F[a,b]:F[a]]\leq [F[b]:F]$ since if $h_b$ is $b$'s minimal polynomial in $F$ then it is also a zeroing polynomial in $F[a]$.
This means that
$$ [F[a,b]:F] = [F[a,b]:F[a]]\cdot[F[a]:F] \leq [F[b]:F]\cdot[F[a]:F] $$

\bcoro

    Let $F$ be a field and $K$ a field extension, define
    $$ {\rm Alg}_F(K) \coloneqq \set{a\in K}[\hbox{$a$ is algebraic over $F$}] . $$
    This is a field.
    Furthermore $F\subseteq{\rm Alg}_F(K)$ is an algebraic extension (all elements of ${\rm Alg}_F(K)$ are algebraic in $F$), and ${\rm Alg}_F(K)\subseteq K$ is a purely transcendental extension (all
    elements in $K\setminus{\rm Alg}_F(K)$ are transcendental in ${\rm Alg}_F(K)$).

\ecoro

\Proof Notice that $F[a\cdot b],F[a+b]\subseteq F[a,b]$ and so $[F[a,b]:F]\leq[F[b]:F]\cdot[F[a]:F]<\infty$, so ${\rm Alg}_F(K)$ is closed under addition and multiplication (and obviously additive inverses).
For $a$ algebraic, $F[a]$ is a field so $a^{-1}\in F[a]$ and so $F[a^{-1}]\subseteq F[a]$ and therefore $[F[a^{-1}]:F]<\infty$ so $a^{-1}$ is algebraic as well (and so by symmetry $F[a]=F[a^{-1}]$).
So ${\rm Alg}_F(K)$ is indeed a field.

To show that ${\rm Alg}_F(K)\subseteq K$ is a pure transcendental extension, notice that if $F_1\subseteq F_2\subseteq F_3$ where $F_1\subseteq F_2$ is algebraic, if $a\in F_3$ is algebraic in $F_2$
it is also algebraic in $F_1$.
Indeed if $f\in F_2[x]$ such that $f(a)=0$, let its coefficients be $b_i$ then $a$ is algebraic in $F_1[b_0,\dots,b_n]$ and so
$$ [F_1[b_0,\dots,b_n,a]:F_1[b_0,\dots,b_n]] = [F_1[b_0,\dots,b_n,a]:F_1[b_0,\dots,b_n]]\cdot[F_1[b_0,\dots,b_n]:F_1] $$
and this is finite since $b_0,\dots,b_n$ are algebraic in $F_1$ as they are in $F_2$, so both terms are finite.
So if $K$ had any algebraic numbers not in ${\rm Alg}_F(K)$, they would be algebraic in $F$ and thus in ${\rm Alg}_F(K)$ in contradiction.
\qed

\bprop

    Let $F$ be a field and $f\in F[\lambda]$ be irreducible, then there exists a field extension $F\subseteq K$ such that $f$ has a root in $K$, and $[K:F]=\deg f$.

\eprop

\Proof since $f$ is irreducible, $\gen f$ is prime and $F[\lambda]$ is a PID so it is maximal.
So $K\coloneqq\slfrac{F[\lambda]}{\gen f}$ is a field, and its dimension is $\deg f$, since it can be generated by $\set{1,x,\dots,x^{\deg f-1}}$.
Now recall that by the second isomorphism theorem, $\slfrac F{F\cap\gen f}\cong\slfrac{F+\gen f}{\gen f}\subseteq\slfrac{F[\lambda]}{\gen f}=K$.
But since elements of $\gen f$ are multiples of $f$, which is disjoint from $F$, so $F\cap\gen f=(0)$ so $\slfrac F{F\cap\gen f}\cong F$, and so $F$ can be embedded into $K$ and is thus for all intents and
purposes, a subfield of $K$.
Now define $\alpha\coloneqq\lambda+\gen f$, and suppose $f(\lambda)=\sum_{i=0}^n a_i\lambda^i$ where $a_i\in F$ (viewing $f$ as a polynomial over $K$, $a_i$ is actually $a_i+\gen f$).
Then
$$ f(\alpha) = \sum_{i=0}^na_i(\lambda + \gen f)^i = \sum_{i=0}^na_i(\lambda^i+\gen f) = \sum_{i=0}^na_i\lambda^i + \gen f = f + \gen f = \gen f = 0_K $$
so $\alpha$ is indeed a root of $f(\lambda)$, as required.
\qed

\bcoro

    Let $F$ be a field and $f\in F[\lambda]$ any polynomial.
    Then there exists a field extension $F\subseteq K$ such that $f$ has a root in $K$ and $[K:F]\leq\deg f$.

\ecoro

\Proof find $f$'s irreducible factorization $f=f_1\cdots f_t$, then extend $F$ to a field $K$ such that $f_1$ has a root in $K$, and by above $[K:F]=\deg f_1\leq\deg f$.
\qed

\bdefn

    Let $F$ be a field, and $f$ a polynomial over $F$.
    A field $F\subseteq K$ {\emphcolor splits $f$} if there exist $\alpha_1,\dots,\alpha_n\in K$ such that $f(\lambda)=(\lambda-\alpha_1)\cdots(\lambda-\alpha_n)$.

\edefn

\bthrm

    Every polynomial $f$ over a field $F$ has a field $K$ which splits it, such that $[K:F]\leq(\deg f)!$.

\ethrm

\Proof by induction on $n=\deg f$.
For $n=1$ then $f$ already has a root, and so take $F=K$ and $[K:F]=1=(\deg f)!$.
Now suppose $\deg f=n+1$, then by above there exists a field extension $F\subseteq K_0$ such that there exists an $\alpha_1\in K_0$ such that $f(\alpha_1)=0$ and $[K_0:F]\leq\deg f=n+1$.
And so $(\lambda-\alpha_1)\divides f(\lambda)$, so $f(\lambda)=(\lambda-\alpha_1)g(\lambda)$.
Then $\deg g=n$, and $g$ is a polynomial over $K_0$, so there exists a field extension $F\subseteq K_0\subseteq K$ such that $g(\lambda)=(\lambda-\alpha_2)\cdots(\lambda-\alpha_{n+1})$ for $\alpha_i\in K$
and $[K:K_0]\leq n!$.
Then $f(\lambda)=(\lambda-\alpha_1)\cdots(\lambda-\alpha_{n+1})$ for $\alpha_i\in K$ and $[K:F]=[K:K_0][K_0:F]\leq (n+1)n!=(n+1)!$.
\qed

Notice the following
\benum
    \item the split of a polynomial over any field into its roots is unique,
    \item the number of roots is $\leq\deg f$.
\eenum

Recall that a field $F$ is {\it algebraically closed} if it splits every polynomial in $F[\lambda]$.

\bdefn

    Let $F$ be a field, then $F\subseteq\overline F$ is an {\emphcolor algebraic closure} of $F$ if $\overline F$ is algebraically closed.

\edefn

\bnote

    Every field has a unique (up to isomorphism) algebraic closure.

\enote

So let $f(\lambda)\in F[\lambda]$, then $f(\lambda)\in\overline F[\lambda]$ and so $f=(\lambda-\alpha_1)\cdots(\lambda-\alpha_n)$ for $\alpha_i\in\overline F$.
Then take $F\subseteq K=F[\alpha_1,\dots,\alpha_n]\subseteq\overline F$, it can be shown that $[K:F]\leq(\deg f)!$.

Now suppose $F\subseteq K$ are fields, and $E$ is a field which $F$ is embeddable into, suppose $\phi\colon F\longembeds E$ is an embedding.
An embedding $\phi'\colon K\longembeds E$ is an {\it extension} of $\phi$ if $\phi'\bigr|_F=\phi$.
Denote
$$ \eta_{F\subseteq K}^E \coloneqq \#\set{\hbox{$\phi'$ is an extension of $\phi$}} $$
where $\phi$ is held constant and understood.
Then

\bprop

    Suppose $K=F[\alpha]$, then $\eta_{F\subseteq K}^E$ is equal to the number of roots the minimal polynomial of $\alpha$ in $F$ has in $E$.

\eprop

\Proof since $\alpha$ generates $K$ over $F$, every extension of $\phi$ is defined by its image on $\alpha$.
Let $h$ be the minimal polynomial of $\alpha$ over $F$.
Denote $\hat b\coloneqq\phi(b)$ for all $b\in F$, and this definition extends to polynomials, $\varwidehat{\sum_{i=0}^n b_ix^i}=\sum_{i=0}\hat b_ix^i$.
Then if $\phi'$ is an extension of $\phi$,
$$ \hat h(\phi'(\alpha)) = \phi'(h(\alpha)) = \phi'(0) = 0 $$
this is since if $h(\lambda)=\sum_{i=0}^n a_i\lambda^i$, then $\hat h(\lambda)=\sum_{i=0}^n\hat a_i\lambda^i$, so
$$ \hat h(\phi'(\alpha)) = \sum_{i=0}^n\hat a_i\phi'(\alpha)^i = \sum_{i=0}^n\phi(a_i)\phi'(\alpha)^i = \sum_{i=0}^n\phi'(a_i)\phi'(\alpha)^i = \phi'\parens{\sum_{i=0}^n a_i\alpha^i} = \phi'(h(\alpha)) $$
so $\phi'(\alpha)$ must be one of $\hat h$'s roots, precisely as stated.
\qed

\bdefn

    A polynomial $f$ which splits over $E$ is called {\emphcolor separable} in $E$ if its linear factors are distinct (ie. all of its roots in $E$ are distinct).

\edefn

\bthrm

    Let $F\subseteq K$ be a finite extension (meaning $[K:F]<\infty$), and let $\phi\colon F\longembeds E$ be a given embedding.
    Then
    \benum
        \item $\eta_{F\subseteq K}^E\leq[K:F]$,
        \item if $K$ is generated by the roots of $f$, assuming that $E$ splits $f$, then $1\leq\eta_{F\subseteq K}^E$,
        \item if $f$ is separable over $E$, then $\eta_{F\subseteq K}^E=[K:F]$.
    \eenum

\ethrm

\Proof suppose $K=F[\alpha_1,\dots,\alpha_n]$ (the generators of $K$ can be taken to be the basis of $K$ as an $F$-vector space).
We prove this by induction on $n$, for $n=1$ this is given by the previous proposition, since $\eta_{F\subseteq K}^E$ is the number of roots $h$ has in $E$, and $[K:F]=\deg h$ which is at least this.
Define $F_1\coloneqq F[\alpha_1]$, then
$$ \eqalign{
    \eta_{F\subseteq K}^E &= \#\set{\hbox{$\phi''\colon K\longto E$ is an extension of $\phi$}} \cr
    &= \#\bigcup\set{{\hbox{$\phi''\colon F_1\longto E$ is an extension of $\phi'$}}}[\hbox{$\phi'\colon F_1\longto E$ is an extension of $\phi$}]\cr
    &= \sum_{\phi'}\eta_{F_1\subseteq K}^E = \eta_{F\subseteq F_1}^E\cdot\eta_{F_1\subseteq K}^E \subseteq [F_1:F]\cdot[K:F_1] = [K:F]
} $$

For $(2)$, by the assumption there is an extension of $F\longembeds E$ to $F_1\longembeds E$, and continue inductively.
For $(3)$, since $f$ is separable, makes the bound an equality.
\qed

\bdefn

    Let $f$ be a polynomial over $F$, a field $F\subseteq K$ is a {\emphcolor splitting field} if it is the smallest field in which the polynomial splits.

\edefn

\bye

