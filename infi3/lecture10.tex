\documentclass[10pt]{article}

\usepackage{amsmath, amssymb, mathtools}
\usepackage{tikz}
\usepackage[margin=1.5cm]{geometry}

\input mathex
\input prettyprint
\input preamble

\setscalefactor{10}
\initpps

\def\pmat#1{\begin{pmatrix} #1 \end{pmatrix}}

\let\divides=\mid
\newfunc{metric}\rho({})
\newfunc{metricc}\sigma({})

\font\bigbf = cmbx12 scaled 2000

\def\openset{{\cal O}}
\let\lineseg=\overleftrightvecc
\let\ds=\displaystyle

\begin{document}

\c@section=10

\barcolorbox{220, 255, 220}{0, 130, 0}{80, 200, 80}{
    \leftskip=0pt plus 1fill \rightskip=\leftskip
    {\bigbf Infinitesimal Calculus 3}

    \medskip
    \textit{Lecture 10, Wednsday November 23, 2022}

    \textit{Ari Feiglin}
}

\bigskip

\begin{thrm*}

    If $(X,\metric)$ and $(Y,\sigma)$ are two metric spaces and $f$ is a mapping from $X$ to $Y$, then $f$ is continuous if and only if for every open set $\openset\subseteq Y$, $f^{-1}(\openset)\subseteq X$
    is also open.

\end{thrm*}

\begin{proof}

    Suppose $f$ is continuous, let $\openset\subseteq Y$ be open.
    Suppose $x\in f^{-1}(\openset)$ then $f(x)\in\openset$ and since this is open there exists an $\epsilon>0$ such that $B_\epsilon^Y\bigl(f(x)\bigr)\subseteq\openset$.
    By the $\epsilon-\delta$ criterion/definition for continuity, there exists a $\delta>0$ such that $f\bigl(B_\delta^X(x)\bigr)\subseteq B_\epsilon^Y\bigl(f(x)\bigr)\subseteq\openset$.
    And therefore $B_\delta^X(x)\subseteq f^{-1}(\openset)$, and so for every $x\in f^{-1}(\openset)$ there is a ball around $x$ contained in $f^{-1}(\openset)$, and so by definition $f^{-1}(\openset)$ is
    open.

    To prove the converse suppose that the preimage of every openset in $Y$ under $f$ is open in $X$.
    Let $x\in X$ and $\epsilon>0$, then $f^{-1}\bigl(B_\epsilon(f(x))\bigr)$ is open, and since $x\in f^{-1}\bigl(B_\epsilon(f(x))\bigr)$, there must be a $\delta>0$ such that
    $B_\delta(x)\subseteq f^{-1}\bigl(B_\epsilon(f(x))\bigr)$, and so we have that
    \[ f\bigl(B_\delta(x)\bigr) \subseteq B_\epsilon\bigl(f(x)\bigr) \]
    and since this is true for every $\epsilon>0$, $f$ is continuous at $x$ for every $x\in X$ so $f$ is continuous as required.

    \hfill$\blacksquare$

\end{proof}

This theorem gives us an equivalent definition of continuity which allows us to extend this notion of continuity to more general topological spaces.

Also notice that this is equivalent to saying that a mapping between $X$ and $Y$ is continuous if and only if for every $K\subseteq Y$ which is closed, $f^{-1}(K)$ is closed.
This is since $f^{-1}(K)$ is closed if and only if $f^{-1}(K)^c=f^{-1}(K^c)$ is open.
Since the complement of a closed set is an open set (and vice versa), this is equivalent to saying that for every $\openset\subseteq Y$ which is open, $f^{-1}(\openset)$ is open, which we showed above is
equivalent to continuity.
This proves the following corollary:

\begin{coro*}

    If $(X,\metric)$ and $(Y,\sigma)$ are two metric spaces and $f$ is a mapping from $X$ to $Y$, then $f$ is continuous if and only if for every closed set $K\subseteq Y$, $f^{-1}(K)\subseteq X$
    is also closed.

\end{coro*}

It is not true that the image of an open set under a continuous function is open, as demonstrated by the following example:

\begin{exam}

    If we take $f\colon\bR\longvarrightarrow\bR$ where $f(x)=x^2$ then we know that $f$ is continuous.
    But if we take the image of $f$ under the open set $\bR$, then we get $f(\bR)=[0,\infty)$ which is not open.

\end{exam}

\begin{exam}

    If we take the real valued function $f\colon[0,\infty)\longvarrightarrow\bR$ where $f((x)=\sqrt x$, then $f^{-1}(\bR)=[0,\infty)$.
    But $f$ is continuous and $\bR$ is open, so shouldn't $f^{-1}(\bR)$ be open as well?
    In fact $f^{-1}(\bR)=[0,\infty)$ \emph{is} open since openness is relative to the metric space we're dealing with.
    In this case the domain of $f$ is $[0,\infty)$, and $[0,\infty)$ \emph{is} open in $[0,\infty)$ ($X$ is open in $X$).

\end{exam}

\begin{defn*}

    Suppose $(X,\metric)$ is a metric space.
    Suppose $S\subseteq E\subseteq X$ then $S$ is \ppemph{open} relative to $E$ if it is open in the metric space $\bigl(E,\metric\vert_{E\times E}\bigr)$.

\end{defn*}

\begin{prop*}

    If $(X,\metric)$ is a metric space and $E\subseteq X$.
    $S\subseteq E$ is open relative to $E$ if and only if there exists an open set $\openset\subseteq X$ such that $S=E\cap\openset$.

\end{prop*}

\begin{proof}

    Suppose $S$ is open relative to $E$, then for every $x\in S$ let $r_x>0$ such that $B_{r_x}^E(x)\subseteq S$.
    Then we have that $S$ is the union of all the $B_{r_x}^E(x)$s.
    Let:
    \[ \openset = \bigcup_{x\in S} B_{r_x}^X(x) \]
    This is open as the union of open sets, and we have that:
    \[ E\cap\openset = \bigcup_{x\in S} B_{r_x}^X(x)\cap E = \bigcup_{x\in S}B_{r_X}^E(x) = S \]
    As required.

    To prove the converse suppose $\openset\subseteq X$ is open and $S=\openset\cap E$.
    Let $x\in S$ then $x\in\openset$ so there exists an $r>0$ such that $B_r(x)\subseteq\openset$.
    And therefore $B_r^E(x)=B_r(x)\cap E\subseteq\openset\cap E=S$, so for every $x\in S$ there exists an $r>0$ such that $B_r^E(x)\subseteq S$, so $S$ is open as required.

    \hfill$\blacksquare$

\end{proof}

\begin{prop*}

    If $(X,\metric)$ is a metric space and $E\subseteq X$.
    $S\subseteq E$ is open relative to $E$ if and only if there exists a closed set $K\subseteq X$ such that $S=E\cap K$.

\end{prop*}

\begin{proof}

    $S$ is open in $E$ if and only if $E\setminus S$ is open in $E$ which is equivalent to $E\setminus S=E\cap\openset$ for some open set $\openset$ and so
    $S=E\setminus\bigl(E\cap\openset)\bigr)=E\cap\openset^c$ and so $S=E\cap K$ for some closed $K$.

    \hfill$\blacksquare$

\end{proof}

\begin{prop*}

    If $E\subseteq X$ is open then $S\subseteq E$ is open in $E$ if and only if $S$ is open in $X$.

\end{prop*}

\begin{proof}

    If $S$ is open in $E$ then $S=E\cap\openset$ where $\openset$ is open in $X$ and since $E$ is as well, $S$ is open.
    If $S$ is open in $X$ then it is trivial that $S$ is open in $E$ (this is true in general).

    \hfill$\blacksquare$

\end{proof}

\begin{prop*}

    Suppose $(X,\metric)$ and $(Y,\sigma)$ are metric spaces and $E\subseteq X$.
    Then $f\colon E\longvarrightarrow Y$ is continuous in $E$ if and only if for every $\openset\subseteq Y$ open, there exists an open set ${\cal U}\subseteq X$ such that $f^{-1}(\openset)=E\cap{\cal U}$.

\end{prop*}

\begin{proof}

    This proof is trivial.
    $f$ is continuous if and only if for every $\openset\subseteq Y$ open, $f^{-1}(Y)$ is open in $E$, which is equivalent to $f^{-1}(\openset)=E\cap{\cal U}$ for some open set ${\cal U}\subseteq X$.

    \hfill$\blacksquare$

\end{proof}

\end{document}

