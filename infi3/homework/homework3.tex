\documentclass[10pt]{article}

\usepackage{amsmath, amssymb, mathtools}
\usepackage{tikz}
\usepackage[margin=1.5cm]{geometry}

\input mathex
\input prettyprint
\input ../preamble

\setscalefactor{10}
\initpps

\def\pmat#1{\begin{pmatrix} #1 \end{pmatrix}}

\def\@blist[#1]{%
    \bgroup\bgroup\par\vskip-\medskipamount%
    \gdef\item{%
        \par\egroup\bgroup\medskip\setbox0=\hbox{#1\quad}%
        \advance\leftskip by \wd0\leavevmode\kern-\wd0\box0%
    }%
}
\def\blist{\@ifnextchar[ \@blist {\@blist[$\bullet$]}}
\def\elist{\par\egroup\egroup\medskip}

\let\divides=\mid
\newfunc{metric}\rho({})
\def\openset{{\cal O}}

\font\bigbf = cmbx12 scaled 2000

\pdf@drawing@macro{check}{0 1 0 RG 1.2 w 1 j 1 J 0 5 m 3 0 l 8 10 l S}{9.2pt}{12pt}{1pt}{1pt}
\pdf@drawing@macro{ecks}{1 0 0 RG 0 0 m 1 j 1 J 10 10 l 0 10 m 10 0 l S}{11.2pt}{12pt}{1pt}{1pt}

\begin{document}

\c@section=3

\barcolorbox{220, 255, 220}{0, 130, 0}{80, 200, 80}{
    \leftskip=0pt plus 1fill \rightskip=\leftskip
    {\bigbf Infintesimal Calculus 3}

    \medskip
    \textit{Assignment 3}

    \textit{Ari Feiglin}
}

\bigskip

\begin{exercise*}

    For every $n\in\bN$ the set $A_n\subseteq\bR^d$ is compact.
    Prove or disprove the following:
    \blist
        \item $\bigcup_{n=1}^m A_n$ is compact.
        \item $\bigcap_{n=1}^m A_n$ is compact.
        \item $A_m\setminus A_n$ is compact.
        \item $\bigcup_{n\in\bN} A_n$ is compact.
        \item $\bigcap_{n\in\bN} A_n$ is compact.
    \elist

\end{exercise*}

\begin{blankpp}

    \blist
        \item This is true since if $\set{\openset_\lambda}_{\lambda\in\Lambda}$ is an open cover of the union then:
        \[ \bigcup_{n=1}^m A_n \subseteq \bigcup_{\lambda\in\Lambda}\openset_\lambda \implies A_n \subseteq \bigcup_{\lambda\in\Lambda} \openset_\lambda \]
        So $\set{\openset_\lambda}$ is an open cover of every $A_n$.
        So for every $n$ there exists a finite subcover $\set{\openset_{n_k}}_{k=1}^{k_n}$ of $A_n$.
        And therefore
        \[ \bigcup_{n=1}^m A_n \subseteq \bigcup_{n=1}^m\bigcup_{k=1}^{k_n} \openset_{n_k} \]
        And so the union of these finite subcovers of $A_n$ is itself a finite subcover of the union.
        And so the union is compact.
        \item This is true.
        An arbitrary intersection of compact sets in $\bR^d$ is compact: suppose $\set{A_\lambda}_{\lambda\in\Lambda}$ are compact.
        Then each of the sets are closed, and therefore $\bigcap_{\lambda\in\Lambda}A_\lambda$ is closed.
        And each of the sets are bounded since they are compact, and therefore the intersection is also bounded (it is a subset of the ball containing any $A_\lambda$).
        And therefore the intersection is compact.
        This also shows that $5$ is true.
        \item This is false.
        The sets $[0,2]$ and $[1,2]$ are compact since they are closed and bounded, but $[0,2]\setminus[1,2]=[0,1)$ is not closed and is therefore not compact.
        \item This is false.
        For every $n$ let $A_n=[-n,n]$, then $A_n$ is compact since it is closed and bounded.
        But the union of $A_n$ gives $\bR$ which is not compact since it is not bounded.
        \item This is true, refer to the proof of $2$.
    \elist

\end{blankpp}

\begin{exercise*}

    Suppose $(X,\norm\cdot)$ is a normed linear space.
    Show that every closed ball in $X$ is path connected.

\end{exercise*}

\let\lineseg=\overleftrightvecc
\begin{blankpp}

    Suppose $\bar B_r(x)$ is the offending ball.
    If $z,y\in\bar B_r(x)$ then the line segment between them, denoted $\lineseg{zy}$ is a path between the two points contained in $\bar B_r(x)$.
    It is contained in $\bar B_r(x)$ since for every $0\leq t\leq1$:
    \[ \norm{ty + (1-t)z - x} \leq \norm{t(y-x)} + \norm{(1-t)(z-x)} = t\norm{y-x} + (1-t)\norm{z-x} \]
    and since $z,y$ are in the ball, this is less than or equal to $tr+(1-t)r=r$.
    So every point on the line segment is in the ball.

\end{blankpp}

\newpage
\begin{exercise*}

    Suppose $A,B\subseteq\bR^n$ are connected.
    Which of the following are also connected?
    \blist
        \item $A\cap B$
        \item If $A\cap B\neq\varnothing$, $\interiorof{A\cup B}$.
        \item $A\setminus B$
    \elist

\end{exercise*}

\begin{blankpp}

    \blist
        \item This is not necessarily connected.
        
        We will first show that rectangles are connected.
        Suppose we have the rectangle $R=\prod(a_n,b_n)$, and $x,y\in R$.
        Then the line $\lineseg{xy}$ is contained in $R$, so it is path connected so $R$ is connected.
        Then if $x_k$ and $y_k$ are the $k$th components of $x$ and $y$ respectively, notice that for every relevant $k$:
        \[ a_k = ta_k + (1-t)a_k < tx_k + (1-t)y_k < tb_k + (1-t)b_k = b_k \]
        So the line segment is conntained in $R$ as required.

        So if we take the rectangle: $A=(0,3)\times(0,1)$ and the union of non-disjoint rectangles $B=\bigl((0,1)\times(0,2)\bigr)\cup\bigl((0,3)\times(1,2)\bigr)\cup\bigl((2,3)\times(0,2)\bigr)$.
        Then these are both connected and $A\cap B=\bigl((0,1)\times(0,1)\bigr)\dcup\bigl((2,3)\times(0,1)\bigr)$, which is the disjoint union of two open sets.
        So $A\cap B$ is disconnected.

        \item This is not necessarily connected.
        
        %We will first prove a simple lemma: if $A$ is connected and $x\in\overline A$ then $A\cup\set x$ is also connected.
        %Suppose $\openset_1$ and $\openset_2$ are disjoint open sets such that $A\cup\set x\subseteq\openset_1\dcup\openset_2$.
        %Then $A\subseteq\openset_1\dcup\openset_2$, and since $A$ is connected suppose $A\subseteq\openset_1$.
        %If $x\in\openset_2$, since it is open then there is a $r>0$ such that $B_r(x)\subseteq\openset_2$.
        %And since $x$ is a point of closure of $A$'s, $B_r(x)\cap A\neq\varnothing$ which implies $A\cap\openset_2\neq\varnothing$, but $A$ is disjoint from $\openset_2$ since it is a subset of $\openset_1$.
        %So this is a contradiction, and therefore $x\in\openset_1$ and therefore $A\cup\set x\subseteq\openset_1$, so $A\cup\set x$ is connected.

        So if we define $A=\bar B_1(0,0)$ and $B=\bar B_1(2,0)$ then by the previous question both of these sets are path connected and therefore connected.
        But $\interiorof{A\cup B}=B_1(0,0)\cup B_2(2,0)$ which is the disjoint union of two open sets which is therefore not connected.

        \item This is not necessarily connected.
        If $A=(-2,2)$ and $B=[-1,1]$ these two sets are connected since they are path connected, but $A\setminus B=(-2,-1)\cup(1,2)$ which is the disjoint union of two open sets and is therefore not
        connected.
    \elist

\end{blankpp}

\end{document}
