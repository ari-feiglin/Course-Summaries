\documentclass[10pt]{article}

\usepackage{amsmath, amssymb, mathtools}
\usepackage[margin=1.5cm]{geometry}

\input mathex
\input prettyprint
\input preamble

\setscalefactor{10}
\initpps

\let\divides=\mid
\newfunc{metric}\rho({})

\font\bigbf = cmbx12 scaled 2000

\def\openset{{\cal O}}
\let\lineseg=\overleftrightvecc

\begin{document}

\barcolorbox{220, 255, 220}{0, 130, 0}{80, 200, 80}{
    \leftskip=0pt plus 1fill \rightskip=\leftskip
    {\bigbf Infinitesimal Calculus 3}

    \medskip
    \textit{Lecture 5, Sunday November 6, 2022}

    \textit{Ari Feiglin}
}

\bigskip

\c@section=4
\subsection{Connected Spaces}

\begin{defn*}

    If $X$ is a metric space, $E\subseteq X$ is \ppemph{disconnected} if there exists two disjoint open sets $\openset_1$
    and $\openset_2$ such that $E\subseteq \openset_1\dcup\openset_2$ and $E\cap\openset_1, E\cap\openset_2\neq\varnothing$.
    A \ppemph{connected} space is a space which is not disconnected.

\end{defn*}

This is equivalent to saying that if $E\subseteq\openset_1\dcup\openset_2$ then $S\subseteq\openset_1$ or $S\subseteq\openset_2$.

\begin{prop*}

    A metric space is connected if and only if the only clopen sets is the entire space and the empty set.

\end{prop*}

\begin{proof}

    Suppose $X$ is disconnected, then there exists open sets $\openset_1$ and $\openset_2$ such that $X=\openset_1\dcup\openset_2$.
    Therefore the complement of $\openset_1$ is $\openset_2$ and so it is also closed.
    And since $\openset_1$ and $\openset_2$ are disjoint and have non-empty intersection with $X$, they are neither $X$ nor the
    empty set.
    Therefore if $X$ is disconnected there exists clopen sets other than $X$ and $\varnothing$.

    Suppose $\openset_1$ is a clopen set in $X$ not equal to $X$ or $\varnothing$, then $\openset_2=\openset_1^c$ is open and
    disjoint from $\openset_1$.
    Since $\openset_1\neq X,\varnothing$, $\openset_2\neq X,\varnothing$ since $X=\openset_1\dcup\openset_2$, $X$ is disconnected.

    \hfill$\blacksquare$

\end{proof}

So $\bR$ for example is connected.
And $\bR\setminus\set0$ is disconnected since $(-\infty,0)$ and $(0,\infty)$ are open, disjoint, and cover $\bR\setminus\set0$.

\begin{defn*}

    A \ppemph{line segment} between two vectors $P$ and $Q$ in a linear space $X$ is
    \[ \lineseg{PQ} = \set{P + t(Q - P)}[t\in{[0,1]}] \]

\end{defn*}

Notice that $\lineseg{PQ}=\lineseg{QP}$, and both $P$ and $Q$ are in this segment.
If the line segment in focus is understood, $x_t$ is understood to be the point $P + t(Q-P)$.

\begin{prop*}

    A line segment in a normed linear space $X$ is connected.

\end{prop*}

\begin{proof}

    Suppose $P,Q\in X$, suppose for the sake of a contradiction that $A$ and $B$ are open such that
    $\lineseg{PQ}\subseteq A\dcup B$ as well as having non-empty intersections with the segment.
    We know $x_1=Q$ and suppose $x_1\in B$, we define:
    \[ K = \set{t\in[0,1]}[x_t\in A] \qquad u =\sup K \]
    $K$ is non-empty since $A$ has non-empty intersection with $\lineseg{PQ}$ and $K$ is bound by $1$, so it has a supremum.
    We will show that $x_u\notin A$.
    If $u=1$ then $x_u=Q\in B$ which is disjoint from $A$, so $x_u\notin A$.
    If $u<1$ and $x_u\in A$, since $A$ is open there is an $\epsilon>0$ such that $B_{\epsilon}(x_u)\subseteq A$.
    Notice then that $x_{t'}=x_{t+\frac\epsilon{2\norm{Q-P}}}$ must be in $A$ since
    $\norm{x_t-x_{t'}} = \frac\epsilon2<\epsilon$.
    But $t'>t$ and $t'\in K$ which is a contradiction to $t$'s supremumness.

    And $x_u\notin B$ since if it were, since $B$ is open so there is a ball $B_\epsilon(x_u)\subseteq B$.
    And so similar to before there must be a $\delta>0$ such that for every $u-\delta<t<u+\delta$, $x_t\in B$.
    So $u-\delta$ is an upper bound to $K$, which is a contradiction.

    So $x_u\notin A\dcup B$, but it must be in $\lineseg{PQ}$ since $u\in[0,1]$ which is a contradiction.

    \hfill$\blacksquare$

\end{proof}

\begin{thrm*}

    If $\set{S_\lambda}_{\lambda\in\Lambda}$ are connected sets such that their intersection is non-empty, then
    \[ S = \bigcup_{\lambda\in\Lambda} S_\lambda \]
    is also connected.

\end{thrm*}

\begin{proof}

    Suppose $B$ and $C$ are open and disjoint such that $S\subseteq B\dcup C$.
    Let $\lambda\in\Lambda$ and so $S_\lambda B\dcup C$, so $S_\lambda\subseteq B$ or $S_\lambda\subseteq C$.
    Without loss of generality suppose it is a subset of $B$.
    Then take $\gamma\neq\lambda\in\Lambda$ then $S_\gamma\subseteq B$ or $S_\gamma\subseteq C$.
    Since there exists an $x\in S_\lambda\cap S_\gamma$ and $x\notin C$, $S_\lambda\subseteq B$.
    So for every $\lambda\in\Lambda$, $S_\lambda\subseteq B$, and therefore $S\subseteq B$, so $S$ is connected.

    \hfill$\blacksquare$

\end{proof}

\begin{defn*}

    If $X$ is a normed linear space and $P_1,\dots,P_n\in X$, the \ppemph{polygon chain} is
    \[ \lineseg{P_1P_2\cdots P_n} = \lineseg{P_1P_2}\cup\lineseg{P_2P_3}\cup\cdots\cup\lineseg{P_{n-1}P_n} \]

\end{defn*}

\begin{prop*}

    Polygonal chains are connected.

\end{prop*}

\begin{proof}

    We will prove so through induction on the number of vectors in the chain.
    For $n=2$ this is simply a line segment, which we know is connected.
    We know that $\lineseg{P_1\cdots P_{n+1}} = \lineseg{P_1\cdots P_n}\cup\lineseg{P_nP_{n+1}}$.
    By our inductive hypothesis, these are both connected and have a non empty intersection (since $P_n$ is in it), so by the above
    theorem, their union is connected as well.

    \hfill$\blacksquare$

\end{proof}

\begin{defn*}

    A \ppemph{path} between two vectors $P$ $Q$ in a normed linear space $X$ is a continuous function
    $f\colon[0,1]\longvarrightarrow X$ such that $f(0)=P$ and $f(1)=Q$.

\end{defn*}

Notice then that a line segment represents a path (let $f(t)=x_t$).
And similarly so does a polygonal chain.

\begin{defn*}

    A subset $S$ of a normed linear space $X$ is \ppemph{connected pathwise} if for every $x,y\in S$ there is a path $\gamma_{xy}$
    between $x$ and $y$ whose image is contained in $S$.

\end{defn*}

When talking about paths it is often useful to focus on their image, so we when discussing paths we may use the function in place of
its image.

\begin{prop*}

    Every path is connected.

\end{prop*}

A similar proof to showing line segments are connected can be used.

\begin{prop*}

    If a set is connected pathwise, then it is connected.

\end{prop*}

\begin{proof}

    Suppose $S$ is connected pathwise, then let $x\in S$.
    We know that
    \[ S = \bigcup_{y\in S}\gamma_{xy} \]
    Since $y\in\gamma_{xy}$ and $\gamma_{xy}\subseteq S$.
    And since paths are connected and the intersection of $\set{\gamma_{xy}}[y\in S]$ contains $x$ and is therefore non-empty, by the above
    theorem, $S$ is also connected.

    \hfill$\blacksquare$

\end{proof}

\begin{thrm*}

    An open set is connected if and only if it is connected pathwise.

\end{thrm*}

\begin{proof}

    Suppose $\openset$ is open, we know if it is connected pathwise then it is connected, so all that remains is to prove the converse.
    Take some $x\in S$ and define:
        \[ A = \set{y\in S}[\text{there exists a path between $x$ and $y$ contained entirely in $S$}] \]
    Let $B=S\setminus A$.
    We will show that $A$ and $B$ are open.
    We know that $S=A\dcup B$ and $A\cap B=\varnothing$, so if we succeed in showing that they both are open, since $S$ is connected, one must be empty.
    And since $x\in A$, it must be that $B=\varnothing$ so $S=A$ and therefore $S$ is open.

    First we will show that $A$ is open.
    If $y\in A$ then since $\gamma_{xy}\subseteq S$, $y\in S$ which is open, so there exists an $r>0$ such that $B_r(y)\subseteq S$.
    If $z\in B_r(y)$ then taking $\gamma_{xy}\cup\lineseg{yz}$ gives a path between $x$ and $y$ contained in $S$ (we define the path of the union to squish $\gamma_{xy}$ and then be $\lineseg{yz}$).
    So $z\in A$, and therefore $B_r(y)\subseteq A$, so $A$ is open.

    Let $y\in B$, since $y\in S$ then there exists an $r>0$ such that $B_r(y)\subseteq S$.
    If there is a $z\in B_r(y)\cap A$ then there is a path $\gamma_{zx}$ contained in $S$, and since $z\in B_r(y)$, $\lineseg{yz}$ is also contained in $S$, so the path $\lineseg{yz}\cup\gamma_{zx}$
    is a path between $y$ and $x$ contained in $S$, so $y\in A$ in contradiction.
    So $B_r(y)\subseteq B$, so $B$ is open.

    \hfill$\blacksquare$

\end{proof}

\end{document}

