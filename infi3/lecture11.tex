\documentclass[10pt]{article}

\usepackage{amsmath, amssymb, mathtools}
\usepackage{tikz}
\usepackage[margin=1.5cm]{geometry}

\input mathex
\input prettyprint
\input preamble

\setscalefactor{10}
\initpps

\def\pmat#1{\begin{pmatrix} #1 \end{pmatrix}}

\let\divides=\mid
\newfunc{metric}\rho({})
\newfunc{metricc}\sigma({})

\font\bigbf = cmbx12 scaled 2000

\def\openset{{\cal O}}
\let\lineseg=\overleftrightvecc
\let\ds=\displaystyle

\def\pdv#1#2{\frac{\partial #1}{\partial #2}}

\begin{document}

\c@section=11

\barcolorbox{220, 255, 220}{0, 130, 0}{80, 200, 80}{
    \leftskip=0pt plus 1fill \rightskip=\leftskip
    {\bigbf Infinitesimal Calculus 3}

    \medskip
    \textit{Lecture \thesection, Wednsday November 23, 2022}

    \textit{Ari Feiglin}
}

\bigskip

\subsection{Funky Functions}

\begin{prop*}

    If $f\colon X\longvarrightarrow Y$ is continuous and $E\subseteq X$ is compact, $f(E)$ is also compact.

\end{prop*}

\begin{proof}

    Suppose $\set{\openset_\lambda}_{\lambda\in\Lambda}$ is an open cover of $f(E)$, that is:
    \[ f(E) \subseteq \bigcup_{\lambda\in\Lambda}\openset_\lambda \implies E \subseteq \bigcup_{\lambda\in\Lambda} f^{-1}(\openset_\lambda) \]
    Since $f$ is continuous, $f^{-1}(\openset_\lambda)$ is open so $\set{f^{-1}(\openset_\lambda)}_{\lambda\in\Lambda}$ is an open cover of $E$, which is compact so there is an open subcover of $E$, let
    it be $\set{f^{-1}(\openset_k)}_{k=1}^n$, so:
    \[ E \subseteq \bigcup_{k=1}^n f^{-1}(\openset_k) \implies f(E) \subseteq \bigcup_{k=1}^n\openset_k \]
    And so $\set{\openset_k}_{k=1}^n$ is an open subcovering of $f(E)$, and therefore $f(E)$ is compact.

    \hfill$\blacksquare$

\end{proof}

\begin{prop*}

    If $E\subseteq X$ is compact and $f\colon E\longvarrightarrow Y$ is continuous, $f$ takes a minimum and maximum in $E$.

\end{prop*}

\begin{proof}

    Since $f(E)$ itself is compact, it is bounded.
    And so there exists $s=\sup f(E)$, which means that there exists $x_n\in E$ such that $f(x_n)\longvarrightarrow s$.
    And since $E$ is compact there exists a subsequence $x_{n_k}$ which converges to some value $x\in E$, and so $f\bigl(x_{n_k}\bigr)\longvarrightarrow s$, but at the same time since $f$ is continuous
    so $f\bigl(x_{n_k}\bigr)\longvarrightarrow f(x)$, and so $f(x)=s$.
    So $f$ takes a maximum in $E$.
    A similar proof can be used to show it takes a minimum.

    \hfill$\blacksquare$

\end{proof}

\begin{exam}

    The continuous preimage of a compact set is not necessarily compact.
    Take for example $f\colon\bR\longvarrightarrow\bR$ where $f(x)=c$.
    Then even though $\set c$ is compact, $f^{-1}(\set c)=\bR$ is not compact.

\end{exam}

\begin{prop*}

    If $E\subseteq X$ is compact and $f\colon E\longvarrightarrow Y$ is continuous and injective, then $f^{-1}\colon f(E)\longvarrightarrow E$ is continuous.

\end{prop*}

\begin{proof}

    Let $g=f^{-1}$ on $f(E)$.
    We will show that for every $K\subseteq E$ closed, $g^{-1}(K)$ is also closed.
    Since $E$ is compact and $K$ is closed in $E$, $K$ is compact in $X$.
    And $g^{-1}=f$, so $g^{-1}(K)=f(K)$ which is compact since $f$ is continuous, and so it is closed.
    So the preimage under $g$ of any closed set is also closed, therefore $g$ is continuous.

    \hfill$\blacksquare$

\end{proof}

\begin{defn*}

    A mapping $f$ between two metric spaces $X$ and $Y$ is \ppemph{uniformly continuous} if for every $\epsilon>0$ there is a $\delta>0$ such that for every $x,y\in X$ if $\metricof{x,y}<\delta$ then
    $\sigma\parens{f(x),f(y)}<\epsilon$.

\end{defn*}

%%% TODO: FIX

\begin{thrm*}

    If $E\subseteq X$ is compact and $f\colon E\longvarrightarrow X$ is continuous, then it is uniformly continuous.

\end{thrm*}

\begin{proof}

    Let $\epsilon>0$ then for every $x\in E$ let $\delta_x>0$ satisfy continuity for $\epsilon$, that is $f\bigl(B_{\delta_x}(x)\bigr)\subseteq B_\epsilon\bigl(f(x)\bigr)$.
    And so:
    \[ E \subseteq \bigcup_{x\in E}B_{\frac12\delta_x}(x) \]
    This is an open cover so there is an open subcover $\set{B_{\frac12\delta_k}(x_k)}_{k=1}^n$.
    We can take $\delta=\frac12\min\limits_{1\leq k\leq n}\delta_k>0$.
    Let $y,z\in E$ such that $\metricof{x,y}<\delta$, then there exists some $x_1$ and $x_2$ in the open subcover such that $y\in B_1(x_1)$ and $z\in B_2(x_2)$ so
    \[ \metricof{z,x_1}\leq\metricof{z,y}+\metricof{y,x_1}<\frac{\delta_1}2+\delta\leq\delta_1 \]
    So $z\in B_1(x_1)$ and therefore $\sigma\parens{f(z),f(x_1)}<\epsilon$ and so $\sigma\parens{f(y),f(z)}<2\epsilon$.
    And so $f$ is uniformly continuous.

    \hfill$\blacksquare$

\end{proof}

\begin{prop*}

    Suppose $f\colon X\longvarrightarrow Y$ is continuous and $E\subseteq X$ is connected, then $f(E)$ is connected in $Y$.

\end{prop*}

\begin{proof}

    Suppose $f(E)\subseteq\openset_1\dcup\openset_2$, so $E\subseteq f^{-1}(\openset_1)\dcup\ f^{-1}(\openset_2)$, which are open since $f$ is continuous.
    Since $E$ is connected, $E\subseteq f^{-1}(\openset_1)$ or $f^{-1}(\openset_2)$ and therefore $f(E)\subseteq\openset_1$ or $\openset_2$.
    So $f(E)$ is connected as required.

    \hfill$\blacksquare$

\end{proof}

Note that if $E$ is connected in $\bR$ and $a<b\in E$ than $(a,b)\subseteq E$.
Suppose $c\in(a,b)\setminus E$ then $E\subseteq\bR_{<c}\dcup\bR_{>c}$ which are both open and have non-empty intersections with $E$ (namely $a$ and $b$), in contradiction to $E$'s connectedness.
This helps prove our next corollary.

\begin{coro*}

    Suppose $X$ is a metric space and $f\colon X\longvarrightarrow\bR$ is continuous.
    Suppose $E\subseteq X$ is connected, then if $a,b\in X$ for $f(a)<f(b)$ then for every $f(a)<y<f(b)$, there is an $x\in E$ such that $f(x)=y$.

\end{coro*}

\begin{proof}

    Since $f$ is continuous and $E$ is connected, $f(E)$ is connected in $\bR$.
    Since it contains $f(a)$ and $f(b)$, it must contain everything between them (since it is a connected set in $\bR$), ie $\bigl(f(a), f(b)\bigr)\subseteq f(E)$, as required.

    \hfill$\blacksquare$

\end{proof}

Thus concludes the topological portion of our course.

\newpage
\subsection{Partial Derivatives}

\begin{defn*}

    If $f$ is a real valued function defined at $(x_0,y_0)$ then the \ppemph{partial derivative} of $f$ relative to $x$ is defined as:
    \[ \pdv fx(x_0,y_0) = \lim_{\Delta x\varrightarrow0}\frac{f(x_0+\Delta x, y_0) - f(x_0, y_0)}{\Delta x} \]
    And the partial derivative relative to $y$ is:
    \[ \pdv fy(x_0,y_0) = \lim_{\Delta y\varrightarrow0}\frac{f(x_0,y_0+\Delta y)-f(x_0,y_0)}{\Delta y} \]

    Other notations include:
    \[ \pdv fx(x_0,y_0) = f_x(x_0,y_0) = \partial_x f(x_0,y_0) \]

\end{defn*}

\begin{lemm*}

    If we define $g(x)=f(x,y_0)$ then $\partial_x f(x_0,y_0)=g'(x_0)$.

\end{lemm*}

This lemma is trivial, but helps us understand how to compute partial derivatives.
We simply keep the other variable constant and diffeentiate.

\begin{proof}

    The proof is trivial:
    \[ g'(x_0) = \lim_{h\varrightarrow0}\frac{g(x_0+h) - g(x_0)}h = \lim_{h\varrightarrow0}\frac{f(x_0+h,y_0)-f(x_0,y_0)}h = \partial_x f(x_0, y_0) \]

\end{proof}

\begin{exam}

    \[ f(x,y) = \sinof{x^2y + y^3} \]
    We will compute $\partial_x f(1,2)$:
    \[ \partial_x f(1,2) = \bigl(\sinof{2x^2+8}\bigr)'(1) = \bigl(4x\cdot\cosof{2x^2 + 8}\bigr)(1) = 4\cosof{10} \]

\end{exam}

\end{document}

