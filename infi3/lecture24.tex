\documentclass[10pt]{article}

\usepackage{amsmath, amssymb, mathtools}
\usepackage{tikz}
\usepackage[margin=1.5cm]{geometry}

\input pdfmsym
\input prettyprint
\input preamble

\pdfmsymsetscalefactor{10}
\initpps

\def\pmat#1{\begin{pmatrix} #1 \end{pmatrix}}

\let\divides=\mid
\newfunc{metric}\rho({})
\newfunc{metricc}\sigma({})
\newfunc{spa}{{\rm span}}(\vert)
\newfunc{diam}{{\rm diam}}(\vert)

\font\bigbf = cmbx12 scaled 2000
\@undervecc@def{underbar}\@linecap\@linecap

\def\openset{{\cal O}}
\let\lineseg=\overleftrightvecc
\let\ds=\displaystyle

\def\pdv#1#2{\frac{\partial #1}{\partial #2}}

\def\differ#1#2{\left.d#1\strut\right|_{#2}}

\def\@ppmathcount{\thesection.\thepp@mathcount}

\begin{document}

\c@section=24

\barcolorbox{220, 255, 220}{0, 130, 0}{80, 200, 80}{
    \leftskip=0pt plus 1fill \rightskip=\leftskip
    {\bigbf Infinitesimal Calculus 3}

    \medskip
    \textit{Lecture \thesection, Wednsday January 25, 2023}

    \textit{Ari Feiglin}
}

\bigskip

\begin{defn*}

    Alternative notation for the integral of a multivariable function $f(x_1,\dots,x_n)$ over $D\subseteq\bR^n$ is
    \[ \int\cdots\int_D f(x_1,\dots,x_n)\,dx_1\dots dx_n \]

\end{defn*}

Notice that if $D=[a,b]\times[c,d]$ and $f$ is integrable over $D$ then we can choose partions $a=x_0<\dots<x_n=b$ and $c=y_0<\dots<x_m=d$ we have that
\[ s(f,P) = \sum_{i=1}^n\sum_{j=1}^m f(a_{ij})(x_i-x_{i-1})(y_{j-1}-y_j) = \sum_{i=1}^n\parens{\sum_{j=1}^m f(a_{ij})(y_{j-1}-y_j)}(x_i-x_{i-1}) \]
we can take partitions which refine $[c,d]$ and get (let $a_{ij}=(a_i,a_j)$):
\[ \sum_{i=1}^n\parens{\int_c^d f(a_i, y)dy}(x_i-x_{i-1}) \]
which is a Riemann sum of $A(x)$ over $[a,b]$ where $A$ is the integral of $f(x,y)$ relative to $y$:
\[ A(x) = \int_c^d f(x, y)\,dy \]
and so $s(f,P) = s(A,P_{a,b})$, and so $\int_D f = \int_a^b A\,dx$, ie
\[ \int_D f = \int_a^b\parens{\int_c^d f(x,y)\,dy}\,dx \]
Using an identical proof, we can swap the order of integration.
We summarize this in the following theorem:

\begin{thrm*}

    Suppose $f$ is integrable in $D=[a,b]\times[c,d]$ and for every $x\in[a,b]$ the following is defined
    \[ I(x) = \int_c^d f(x,y)\,dy \]
    then
    \[ \iint_D f(x,y)\,dxdy = \int_a^b I(x)\,dx = \int_a^b\parens{\int_c^d f(x,y)\,dy}\,dx \]
    Similarly if
    \[ I(y) = \int_a^b f(x,y)\,dx \]
    is defined then the integral is equal to
    \[ \iint_D f(x,y)\,dxdy = \int_c^d I(x) = \int_c^d\parens{\int_a^b f(x,y)\,dx}\,dy \]

\end{thrm*}

We can generalize this to $\bR^3$, if the domain is a prism $[a,b]\times[c,d]\times[e,g]$ then
\[ \iiint_D f = \int_a^b\int_c^d\int_e^g f \]
And in general we can extend this to $n$ dimensions.

\begin{defn*}

    A domain $D\bR^2$ is a \ppemph{normal domain} relative to $x$ if there exists functions $\phi_1,\phi_2$ such that
    \[ D = \set{(x,y)\in\bR^2}[x\in[a,b],\, \phi_1(x)\leq y\leq\phi_2(x)] \]
    Similar for normal domains relative to $y$.

\end{defn*}

\begin{prop*}

    Suppose $D$ is a normal domain relative to $x$: $D=\set{(x,y)}[a\leq x\leq b,\, \phi_1(x)\leq y\leq\phi_2(x)]$ and $\phi_i$ are continuous in $[a,b]$ and so is $f$.
    Then
    \[ \iint_D f(x,y) = \int_a^b\int_{\phi_1(x)}^{\phi_2(x)}f(x,y)\,dydx \]

\end{prop*}

\begin{proof}

    Since $\phi_i$ are continuous, they have extrema: $c\leq\phi_1,\phi_2\leq d$ and so if we define a function $g$ on $R=[a,b]\times[c,d]$ by
    \[ g(x) = \begin{cases} f(x) & x\in D \\ 0 & x\notin D \end{cases} \]
    then
    \[ \int_D f = \int_R g = \int_a^b \int_c^d g(x,y)\,dxdy \]
    and since $g(x,y)=0$ if $y$ is outside $[\phi_1(x),\phi_2(x)]$ this is equal to
    \[ \int_a^b\int_{\phi_1(x)}^{\phi_2(x)}f(x,y)\,dxdy \]

    \hfill$\blacksquare$

\end{proof}

\end{document}
