\documentclass[10pt]{article}

\usepackage{amsmath, amssymb, mathtools}
\usepackage{tikz}
\usepackage[margin=1.5cm]{geometry}

\input mathex
\input prettyprint
\input preamble

\setscalefactor{10}
\initpps

\def\pmat#1{\begin{pmatrix} #1 \end{pmatrix}}

\def\@blist[#1]{%
    \bgroup\par%\vskip-\medskipamount%
    \def\item{%
        \par\egroup\bgroup\medskip\setbox0=\hbox{#1\quad}%
        \advance\leftskip by \wd0\leavevmode\kern-\wd0\box0%
    }%
    \bgroup%
}
\def\blist{\@ifnextchar[ \@blist {\@blist[$\bullet$]}}
\def\elist{\par\egroup\egroup\medskip}

\let\divides=\mid
\newfunc{metric}\rho({})
\newfunc{metricc}\sigma({})

\font\bigbf = cmbx12 scaled 2000

\def\openset{{\cal O}}
\let\lineseg=\overleftrightvecc
\let\ds=\displaystyle

\begin{document}

\c@section=8

\barcolorbox{220, 255, 220}{0, 130, 0}{80, 200, 80}{
    \leftskip=0pt plus 1fill \rightskip=\leftskip
    {\bigbf Infinitesimal Calculus 3}

    \medskip
    \textit{Lecture 8, Wednesday November 16, 2022}

    \textit{Ari Feiglin}
}

\bigskip

\subsection{Continuous Limits}

\begin{prop*}

    If $(X,\metric)$ is a metric space and $p$ is a limit point of some $E\subseteq X$, then for every mapping between $E$ and $\bR$ (or a normed linear space):
    \blist
        \item $\displaystyle\lim_{x\varrightarrow p} f(x) + g(x) = \lim_{x\varrightarrow p} f(x) + \lim_{x\varrightarrow p}g(x)$
        \item If $c\in\bR$ then $\displaystyle\lim_{x\varrightarrow p} cf(x) = c\lim_{x\varrightarrow p}f(x)$
        \item If $f$ and $g$ map to $\bR$ then $\displaystyle\lim_{x\varrightarrow p}f(x)\cdot g(x) = \lim_{x\varrightarrow p}f(x)\cdot\lim_{x\varrightarrow p}g(x)$
        \item If $f$ and $g$ map to $\bR$ and $\displaystyle\lim_{x\varrightarrow p}g(x)\neq0$ then
        $\displaystyle\lim_{x\varrightarrow p} \frac{f(x)}{g(x)} = \frac{\lim_{x\varrightarrow p}f(x)}{\lim_{x\varrightarrow p}g(x)}$
    \elist

\end{prop*}

The proof of this is trivial using that the limit $\lim_{x\varrightarrow p}f(x)$ is equal to the discrete limit $\lim f(x_n)$ if $x_n\longvarrightarrow p$.

\begin{prop*}[squeezeProp,The\ Squeeze\ Theorem]

    If $p$ is a limit point of $E$ and $f,g,h\colon E\longvarrightarrow\bR$ such that $f(x)\leq g(x)\leq h(x)$ and $\ds\lim_{x\varrightarrow p} f(x)=\lim_{x\varrightarrow p}h(x)=c$.
    Then $\ds\lim_{x\varrightarrow p}g(x)=c$ as well.

\end{prop*}

The proof of this is again trivial using discrete limits.

\begin{prop*}

    If $f$ is a mapping from a metric space $E$ to a normed linear space $X$ then $\ds\lim_{x\varrightarrow p}f(x)=0$ if and only if $\ds\lim_{x\varrightarrow p}\norm{f(x)}=0$.

\end{prop*}

Notice that the $0$ in $\ds\lim_{x\varrightarrow p}f(x)=0$ is the $0$ of $X$, and the $0$ in the norm's limit is $0\in\bR$.

\begin{proof}

    Suppose the limit is $0$, then for any $x_n\longvarrightarrow p$ we have that $f(x_n)\longvarrightarrow0$ which by definition means $\norm{f(x_n)}\longvarrightarrow0$ as required.
    If the limit of the norm is $0$ then for any $x_n\longvarrightarrow p$ we have $\norm{f(x_n)}\longvarrightarrow0$ which means $f(x_n)\longvarrightarrow0$ so the limit is $0$ as required.

    \hfill$\blacksquare$

\end{proof}

\begin{prop*}

    If $\ds\lim_{x\varrightarrow p}f(x)=0$ and $g(x)$ is bounded then $\ds\lim_{x\varrightarrow p}g(x)\cdot f(x)=0$ as well.

\end{prop*}

Notice that $\ds\lim_{x\varrightarrow p}g(x)$ doesn't need to exist.
If it did, then this proof would be immediate.

\begin{proof}

    Suppose $\abs{g(x)}<M$ so $\abs{g(x)f(x)}<M\abs{f(x)}$ which converges to $0$ and so by the above proposition so does $g(x)\cdot f(x)$.
   
    \hfill$\blacksquare$

\end{proof}

\begin{thrm*}

    If $(X,\metric_1)$, $(Y,\metric_2)$, and $(Z,\metric_3)$ are metric spaces, and $f$ is a mapping between $X$ and $Y$ and $g$ is a mapping between $Y$ and $Z$.
    If
    \[ \lim_{x\varrightarrow p}f(x) = q \in Y \qquad \lim_{x\varrightarrow q}g(x) = r \in Z \]
    then
    \[ \lim_{x\varrightarrow p} g(f(x)) = r \]

\end{thrm*}

\begin{proof}

    Suppose $x_n\longvarrightarrow p$ then $f(x_n)\longvarrightarrow \ds\lim_{x\varrightarrow p}f(x) = q$ by the definition of limits.
    Then since $f(x_n)$ is a sequence converging to $q$, $g(f(x_n))\longvarrightarrow \ds\lim_{x\varrightarrow q}g(x) = r$.
    And so for every $x_n\longvarrightarrow p$ we have that $g(f(x_n))\longvarrightarrow r$ so the limit of $g(f(x))$ is $r$ as required.

    \hfill$\blacksquare$

\end{proof}

\begin{exam}

    Suppose we'd ``like'' to compute:
    \[ \lim_{(x,y)\varrightarrow(2,1)} \bigl(x^2y^2 - 4xy\bigr)\cdot\frac{\arcsin(xy-2)}{\arctan(3xy-6)} \]
    Let $t=xy-2$ then we have that $t^2=x^2y^2-4xy+4$ so the limit is equal to:
    \[ = \lim_{t\varrightarrow0} \bigl(t^2 - 4\bigr)\cdot\frac{\arcsin(t)}{\arctan(3t)} \]
    Essentially what we have done is defined a function $f(x,y)=xy-2$ and $g(x)=\bigl(t^2 - 4\bigr)\cdot\frac{\arcsin(t)}{\arctan(3t)}$ and the original limit is simply the limit of $g(f(x))$ as $(x,y)$
    approaches $(2,1)$.
    Now since the limit of $f(x,y)$ as $(x,y)$ approaches to $(2,1)$ is $0$.
    So the original limit is equal to the limit of $g(x)$ as $x$ approaches $0$, which is what we wrote above.
    Now this is the limit of a single variable function which we know how to compute.
    The limit is the product of two limits, the polynomial and the rational trignometric function, whose limit is (after applying L'h\^opital's rule) is $-\frac43$.

\end{exam}

\begin{exam}

    It is \emph{not} always the case that
    \[ \lim_{(x,y)\varrightarrow(a,b)}f(x,y) = \lim_{x\varrightarrow a}\lim_{y\varrightarrow b}f(x,y) \]
    For example take
    \[ \lim_{(x,y)\varrightarrow(0,1)}x\cdot\sin\parens{\frac1{1-y}} \]
    Then since $\sin$ is bounded and $x$ converges to $0$, this limit is $0$.
    But the limit of this as $y$ approaches $1$ does not exist, that is:
    \[ \lim_{x\varrightarrow0}\lim_{y\varrightarrow} x\cdot\sin\parens{\frac1{1-y}} \]
    does not exist, but the original limit does.

\end{exam}

\begin{prop*}

    Suppose a function $f\colon E\longvarrightarrow\bR^n$ is defined by $f=\parens{f_1,\dots,f_n}$.
    If $p=\parens{p_1,\dots,p_n}$ is a limit point of $E$ then $\ds\lim_{x\varrightarrow p}f(x)$ exists if and only if for every relevant $k$, the limit $\ds\lim_{x\varrightarrow p_k}f(x_k)$ exists and is
    equal to $q_k$.
    Then
    \[ \lim_{x\varrightarrow p}f(x) = q = \parens{q_1,\dots,q_n} \]

\end{prop*}


This is true since convergence in $\bR^n$ is equivalent to pointwise convergence.
So $f(x_m)=\parens{f_1(x_m),\dots,f_n(x_m)}$ converges to $\parens{q_1,\dots,q_n}$ if and only if every $f_k(x_k)$ converges to $q_k$.

\newpage
\subsection{Continuous Functions}

\begin{defn*}

    If $X$ and $Y$ are metric spaces and $E\subseteq X$ is a subset then $f\colon E\longvarrightarrow Y$ is \ppemph{continuous} at $p\in E$ if for every $\epsilon>0$ there is a $\delta>0$ such that if
    $\metricof{x,p}<\delta$ then $\sigma\parens{f(x),f(p)}<\epsilon$.

    And $f$ is \ppemph{continuous} in $E$ if it is continuous for every $p\in E$.

\end{defn*}

This is equivalent to saying that for every $\epsilon>0$ there is a $\delta>0$ such that:
\[ f\bigl(B_\delta^E(p)\bigr) \subseteq B_\epsilon^Y(f(p)) \]

If $p$ is not an isolated point then this is equivalent to saying
\[ \lim_{x\varrightarrow p}f(x) = f(p) \]
if $p$ is an isolated point though, then the limit doesn't exist but $f$ \emph{is} continuous at $p$.
This is because for any $\epsilon>0$ we can take a $\delta>0$ such that $B_\delta^E(p)=\set p$ and then for every $x$ in this ball (which is only $p$), we have that
$\sigma\parens{f(x),f(p)}=\sigma\parens{f(p),f(p)}=0<\epsilon$.

So an equivalent definition would be to say that $f$ is continuous at $p$ if and only if $p$ is either an isolated point or $\ds\lim_{x\varrightarrow p}f(x)=f(p)$.

\begin{prop*}

    If $X$, $Y$, and $Z$ are metric spaces and $f\colon X\supseteq E\longvarrightarrow Y$ is continuous at $x\in E$ and $g\colon f(E)\longvarrightarrow Z$ is continuous at $f(x)\in f(E)$ then
    $g\circ f$ is continuous at $x$.

\end{prop*}

\begin{proof}

    If $x$ is an isolated point of $E$, by above $g\circ f$ is continuous at $x$.
    Otherwise take $E\ni x_n\longvarrightarrow x$, so $f(x_n)\longvarrightarrow f(x)$.
    Since $g$ is continuous at $f(x)$, we know that $g(f(x_n))\longvarrightarrow g(f(x))$ (since if $f(x_n)=f(x)$ this is true trivially, and otherwise it is true since
    $\lim_{y\varrightarrow f(x)}g(y)=f(x)$).
    And so all in all for every $x_n\longvarrightarrow x$, $g(f(x_n))\longvarrightarrow g(f(x))$, so $g\circ f$ is continuous at $x$.

    \hfill$\blacksquare$

\end{proof}

This shows that if $f$ is continuous in $E$ and $g$ is continuous in $f(E)$, $g\circ f$ is continuous in $E$.

\end{document}

