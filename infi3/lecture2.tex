\documentclass[10pt]{article}

\usepackage{amsmath, amssymb, mathtools}
\usepackage{tikz}
\usepackage[margin=1.5cm]{geometry}

\input mathex
\input prettyprint
\input preamble

\setscalefactor{10}
\initpps

\def\pmat#1{\begin{pmatrix} #1 \end{pmatrix}}

\def\@blist[#1]{%
    \bgroup\bgroup\par\vskip-\medskipamount%
    \gdef\item{%
        \par\egroup\bgroup\medskip\setbox0=\hbox{#1\quad}%
        \advance\leftskip by \wd0\leavevmode\kern-\wd0\box0%
    }%
}
\def\blist{\@ifnextchar[ \@blist {\@blist[$\bullet$]}}
\def\elist{\par\egroup\egroup\medskip}

\let\divides=\mid
\newfunc{metric}\rho({})

\font\bigbf = cmbx12 scaled 2000

\pdf@drawing@macro{check}{0 1 0 RG 1.2 w 1 j 1 J 0 5 m 3 0 l 8 10 l S}{9.2pt}{12pt}{1pt}{1pt}
\pdf@drawing@macro{ecks}{1 0 0 RG 0 0 m 1 j 1 J 10 10 l 0 10 m 10 0 l S}{11.2pt}{12pt}{1pt}{1pt}

\begin{document}

\c@section=2

\barcolorbox{220, 255, 220}{0, 130, 0}{80, 200, 80}{
    \leftskip=0pt plus 1fill \rightskip=\leftskip
    {\bigbf Infintesimal Calculus 3}

    \medskip
    \textit{Lecture 2, Wednesday October 26, 2022}

    \textit{Ari Feiglin}
}

\bigskip

\c@subsection=1

Notice that if $(X,\rho)$ is a metric space, if $Y\subseteq X$, then by restricting $\rho$ onto $Y\times Y$, we define another metric space, $(Y,\rho')$ where $\rho'$ is
the restriction of $\rho$.
This new metric space is called a \emph{metric subspace} of $X$.

\begin{defn*}

    Suppose $(X,\metric)$ is a metric space and $r>0$ is a positive real number.
    If $x\in X$, then we define $B_r(x)$ to be the \ppemph{open ball} centered at $x$ with radius $r$:
    \[ B_r(x) \coloneqq \set{y\in X}[\metricof{x,y}<r] \]
    And the \ppemph{closed ball} is defined similarly:
    \[ \bar B_r(x) \coloneqq \set{y\in X}[\metricof{x,y}\leq r] \]

\end{defn*}

These balls are called the \emph{basic neighborhoods}.

\def\openset{\mathcal{O}}

\begin{defn*}

    If $X$ is a metric space, $\openset\subseteq X$ is \ppemph{open} if for every $x\in S$, there is a $r>0$ such that $B_r(x)\subseteq\openset$.
    A set $F\subseteq X$ is \ppemph{closed} if $F^c$ is open.

\end{defn*}

\begin{exam}

    Every open ball $B_r(x)$ is indeed open.
    This is because if $y\in B_r(x)$ then if we let $s=r-\metricof{x,y}$ then $B_s(y)\subseteq B_r(x)$, since if:
    \[ \metricof{z,y} < s \implies \metricof{z,y} < r-\metricof{x,y} \implies \metricof{z,y} + \metricof{x,y} < r \]
    By the triangle inequality, this means $\metricof{x,z}<r$, so $z\in B_r(x)$, as required.

\end{exam}

\begin{exam}

    The closed ball $\bar B_r(x)$ is indeed closed.
    To prove this, we need to show that $\parens{\bar B_r(x)}^c$ is open.
    Suppose that $y\notin\bar B_r(x)$.
    That means that $\metricof{y,x}>r$, so take $\epsilon>0$ such that $r<r+\epsilon<\metricof{x,y}$.
    Then for every $z\in B_\epsilon{y}$, $\metricof{y,z}<\epsilon$, so $\metricof{z,x}\geq\metricof{y,x}-\metricof{z,y}>r+\epsilon-\epsilon=r$.
    So $z\notin\bar B_r(x)$, and therefore $B_\epsilon(y)\subseteq\parens{\bar B_r(x)}^c$ as required.

\end{exam}

\begin{exam}

    $X$ and $\varnothing$ are both open and closed.
    $\varnothing$ is open vaccuously.
    If $x\in X$, then for any $r>0$, $B_r(x)\subseteq X$ so $X$ is open.
    Since $\varnothing^c=X$, $X$ and $\varnothing$ are also closed.

\end{exam}

\begin{exam}

    If $X=\bR^+\cup\bR^-$, then $\bR^+$ is open since if $x\in\bR^+$ since $B_x(x)\in\bR^+$.
    Similarly so is $\bR^-$.
    So both $\bR^+$ and $\bR^-$ are closed and open in $X$.

\end{exam}

Such sets which are both open and closed are sometimes called \emph{clopen} sets.

\begin{exam}

    If $X=\bR$ let $S=[0,1)$.
    Then $S$ is neither open nor closed.
    $S$ is not open since no ball around $0$ is contained entirely in $S$.
    And since $S^c=(-\infty,0)\cup[1,\infty)$ so no ball around $1$ is contained entirely in $S^c$, so $S^c$ is not open, and therefore $S$ is not closed.
    So $S$ is neither closed nor open.

\end{exam}

\begin{defn*}

    Suppose $X$ is a metric space and $S\subseteq X$.
    \blist
        \item $x\in S$ is an \ppemph{interior point} of $S$ if there is an $r>0$ such that $B_r(x)\subseteq S$.
        \item $x\in X$ is an \ppemph{exterior point} of $S$ if there is an $r>0$ such that $B_r(x)\subseteq S^c$.
        \item $x\in X$ is a  \ppemph{boundary point} of $S$ if every open ball containing $x$ intersects with both $S$ and $S^c$.
        \item $x\in X$ is a  \ppemph{isolated point} of $S$ if there is an open ball containing $x$ which does not contain any other point of $S$.
            That is, there is an $r>0$ such that $B_r(x)\cap S=\set{x}$.
        \item $x\in X$ is a  \ppemph{limit point} of $S$ if every open ball containing $x$ contains another element of $S$.
            That is, for all $r>0$ $\exists x\neq s\in B_r(x)\cap S$.
    \elist

\end{defn*}

\begin{prop*}

    If $X$ is a metric space and $S\subseteq X$, then the following are equivalent:
    \blist
        \item $S$ is open.
        \item Every $x\in S$ is an interior point.
        \item $S$ does not contain any of its boundary points.
    \elist

\end{prop*}

\begin{proof}

    The equivalence of the first two points is a direct consequence of the definition of open sets and interior points.
    Now, suppose $S$ is open and $x$ is a boundary point.
    Then for every $r>0$, $B_r(x)\cap S^c\neq\varnothing$, so $B_r(x)$ is not a subset of $S$, so $x$ is not in $S$.
    Therefore if $S$ is open, it does not contain any of its boundary points.
    
    Now suppose that $S$ doesn't contain any of its boundary points.
    So if $x\in S$, there is an $r>0$ such that $B_r(x)$ doesn't intersect both $S$ and $S^c$.
    Since $x\in B_r(x)$, it must intersect $S$, so $B_r(x)$ cannot intersect $S^c$.
    Therefore $B_r(x)\subseteq S$.
    So for every $x\in S$, there is a $r>0$ such that $B_r(x)\subseteq S$, and therefore $S$ is open.

    \hfill$\blacksquare$

\end{proof}

\end{document}

