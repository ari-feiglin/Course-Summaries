\documentclass[10pt]{article}

\usepackage{amsmath, amssymb, mathtools}
\usepackage{tikz}
\usepackage[margin=1.5cm]{geometry}

\input pdfmsym
\input prettyprint
\input preamble

\pdfmsymsetscalefactor{10}
\initpps

\def\pmat#1{\begin{pmatrix} #1 \end{pmatrix}}

\let\divides=\mid
\newfunc{metric}\rho({})
\newfunc{metricc}\sigma({})
\newfunc{spa}{{\rm span}}(\vert)
\newfunc{diam}{{\rm diam}}(\vert)

\font\bigbf = cmbx12 scaled 2000
\@undervecc@def{underbar}\@linecap\@linecap

\def\openset{{\cal O}}
\let\lineseg=\overleftrightvecc
\let\ds=\displaystyle

\def\pdv#1#2{\frac{\partial #1}{\partial #2}}

\def\differ#1#2{\left.d#1\strut\right|_{#2}}

\def\@ppmathcount{\thesection.\thepp@mathcount}

\begin{document}

\c@section=23

\barcolorbox{220, 255, 220}{0, 130, 0}{80, 200, 80}{
    \leftskip=0pt plus 1fill \rightskip=\leftskip
    {\bigbf Infinitesimal Calculus 3}

    \medskip
    \textit{Lecture \thesection, Sunday January 22, 2023}

    \textit{Ari Feiglin}
}

\bigskip

By the previous theorem, $f$ is integrable if and only if for every $\epsilon>0$ there is a
partition such that $\bar s(f,P) - \underbar s(f,P) < \epsilon$, and this is simply 
\[ \sum_{j=1}^n(M_j-m_j)\abs{D_j} < \epsilon \]
where $P=\set{D_j}_{j=1}^n$.

\begin{thrm*}

    Let $D\subseteq\bR^n$ compact and content and $f\colon D\longvarrightarrow\bR$ continuous, then
    $f$ is integrable over $D$.

\end{thrm*}

\begin{proof}

    By Weirstrauss, $f$ is bounded, which is a necessary condition for integrability.
    Let $\epsilon>0$, since $D$ is compact, $f$ is uniformly continuous over it, so there exist a
    $\delta>0$ such that for every $\norm{x-y}<\delta$ in $D$,
    $\abs{f(x)-f(y)}<\frac{\epsilon}{2\abs D}$.
    Now we take a partition $P=\set{D_j}_{j=1}^k$ where $\lambda(P)<\delta$.
    So for every $x,y\in D$ in the same $D_j$, $\norm{x-y}<\delta$ so
    $\abs{f(x)-f(y)}<\frac\epsilon{2\abs D}$ and so
    $\abs{M_j-m_j}\leq\frac\epsilon{2\abs D}<\frac\epsilon{\abs D}$, so
    \[ \bar s(f,P) - \underbar s(f,P) = \sum_{j=1}^k(M_j-m_j)\abs{D_j} <
    \sum_{j=1}^k\frac\epsilon{\abs{D}}\abs{D_j} = \epsilon \]
    so $f$ is integrable as required.

    \hfill$\blacksquare$

\end{proof}

Note that we showed that if $f$ is continuous in $D$ contented compact then
\[ \lim_{\lambda(P)\varrightarrow0}\bar s(f,P)-\underbar s(f,P) = 0 \]
since it is less than every $\epsilon>0$, and since
\[ \underbar s(f,P) \leq \int_D f \leq \bar s(f,P) \implies
0\leq\bar s(f,P) - \int_D f \leq \bar s(f,P) - \underbar s(f,P) \longvarrightarrow 0 \]
and so
\[ \int_D f = \lim_{\lambda(P)\varrightarrow0}\bar s(f,P) =
\lim_{\lambda(P)\varrightarrow0}\underbar s(f,P) \]

\begin{defn*}

    Suppose $P=\set{D_j}_{j=1}^k$ is a partition of $D$ and $f\colon D\longvarrightarrow\bR$ is
    bounded.
    Then a \ppemph{Riemann sum} of $f$ over $P$ is a sum of the form
    \[ \sum_{j=1}^k f(x_j)\abs{D_j} \]
    where $x_j\in D_j$.

\end{defn*}

Notice then that by definition $\bar s(f,P)$ is the supremum of all Riemann sums over $P$ and
$\underbar s(f,P)$ is the infimum, and so if $s$ is a Riemman sum:
\[ \underbar s(f,P) \leq s\leq \bar s(f,P) \]

\begin{defn*}

    A function is \ppemph{Riemann Integrable} in $D$ if when $\lambda(P)\longvarrightarrow0$ all
    the Riemann sums over $P$ converge to the same limit $L$.
    Then we say
    \[ \int_D f = L \]

\end{defn*}

We say that for our previous definition of integrability, $f$ is \emph{Darboux Integrable}.
But we will prove that it doesn't matter, both definitions are equivalent.

\begin{thrm*}

    Let $D\in\bR^n$ compact and $f\colon D\longvarrightarrow\bR$ bounded, then
    $f$ is Riemann integrable if and only if it is Darboux integrable.

\end{thrm*}

\begin{proof}

    Since for every Riemman sum over $P$ we have
    \[ \underbar s(f,P) \leq s(P) \leq \bar s(f,P) \]
    and if $f$ is Darboux integrable then the limits on both sides are the same, say $L$.
    So the limit of $s(P)$ as $\lambda(P)\longvarrightarrow0$ is equal to $L$ by the squeeze
    theorem.

    Now suppose $f$ is Riemann integrable, then for every Riemman sum over the partition $P$, we
    have that $\lim\limits_{\lambda(P)\varrightarrow0}s(P)=L$.
    And since the upper and lower Darboux sums are the supremum and infimum of the Riemman sums,
    this means that they too must converge to $L$.

    \hfill$\blacksquare$

\end{proof}

\begin{prop*}

    Suppose $D\subseteq\bR^n$ is a compact domain and $f,g\colon D\longvarrightarrow\bR$ integrable and $c\in\bR$ constant, then
    \benum
        \item $f+cg$ is integrable over $D$ and satisfies
            \[ \int_D f+cg = \int_D f + c\int_D g \]
        \item If $f\leq g$ in $D$ then
            \[ \int_D f \leq \int_D g \]
        \item If $\set{D_i}_{i=1}^k$ is a partition of $D$ then $f$ is integrable over each $D_i$ and
            \[ \int_D = \sum_{i=1}^k\int_{D_i}f \]
        \item
            \[ \int_D 1 = \abs D \]
        \item If $m\leq f\leq M$ in $D$ then
            \[ m\abs D\leq\int_D f\leq M\abs D \]
        \item If $f$ is continuous in $D$ then there is an $x_0\in D$ such that
             \[ \int_D f = f(x_0)\abs D \]
        \item $\abs f$ is integrable and
             \[ \abs{\int_D f} \leq \int_D\abs f \]
    \eenum

\end{prop*}

\begin{proof}

    \benum
        \item For any partition $P$:
            \[ s(f+cg,P) = \sum_{i=1}^k\bigl(f(x_i)+cg(x_i)\bigr)\abs{D_i} = \sum_{i=1}^k f(x_i)\abs{D_i} + c\sum_{i=1}^k g(x_i)\abs{D_i} =
            s(f,P) + c\cdot s(g,P) \]
            and thus taking limits as $\lambda(P)$ approaches $0$ yields
            \[ \int_D f+cg = \int_D f + c\int_D g \]
        \item Since $\abs{D_i}\geq0$ we have that
            \[ s(f,P) = \sum_{i=1}^k f(x_i)\abs{D_i} \leq \sum_{i=1}^k g(x_i)\abs{D_i} = s(g,P) \]
            so taking the limit gives the inequality.
        \item Take $P_j$ to be a partition of $D_j$ then $P=\bigcup P_j$ is a partition of $D$ then it is simple to show that
            \[ \bar s(f,P) = \sum_{j=1}^k \bar s(f,P_j) \qquad \underbar s(f,P) = \sum_{j=1}^k \underbar s(f,P_j) \]
            And so
            \[ \bar s(f,P) - \underbar s(f,P) = \sum_{j=1}^k\bar s(f,P_j) - \underbar s(f,P_j) \]
            since the right side is a sum of non-negatives it follows that for every $j$
            \[ 0 \leq \bar s(f,P_j) - \underbar s(f,P_j) \leq \bar s(f,P) - \underbar s(f,P) \]
            thus taking the limit of the right side (since for every $\lambda(P)$ we can create a partition of $P_j$s as needed) gives that the
            difference between the upper and lower sums converges to $0$ and therefore $f$ is integrable over $D_j$.
            And so
            \[ \int_D = \lim_{\lambda(P)\varrightarrow0}\bar s(f,P) = \lim\sum_{j=1}^k\bar s(f,P_j) = \sum_{j=1}^k\int_{D_j} f \]
        \item This is simple: for every partition $P$:
            \[ s(1,P) = \sum_{i=1}^k \abs{D_j} = \abs D \]
            and so the integral is also $\abs D$.
        \item By above we know that $\int m = m\abs D$ and $\int M = M\abs D$ and since $m\leq f\leq M$:
            \[ m\abs D = \int_D m \leq \int_D f \leq \int_D M = M\abs D \]
        \item Since $D$ is compact, $f$ has a maximum and minimum so $m\abs D\leq\int f\leq M\abs D$, so
            \[ m\leq\frac1{\abs D}\int_D f\leq M \]
            And since $f$ is continuous, for every value between $m$ and $M$, there is an element $x_0\in D$ such that $f(x_0)$ is equal to it.
            And specifically there is an $x_0\in D$ such that
            \[ f(x_0) = \frac1{\abs D}\int_D f \]
            as required.
        \item For any partition $P$ of $D$, we know that $M_j(\abs f)-m_j(\abs f)\leq M_j(f)-m_j(f)$.
            And so
            \[ 0\leq\bar s(\abs f, P) - \underbar s(\abs f, P) \leq \bar s(f,P) - \underbar s(f,P) \]
            and since $f$ is integrable, this means that so is $\abs f$.
            And since both $f$ and $-f$ are both less than or equal to $\abs f$, their integrals, which are $\int f$ and $-\int f$ are less than
            $\int\abs f$, so
            \[ \abs{\int_D f} = \pm\int_D f \leq \int_D\abs f \]
    \eenum

    \hfill$\blacksquare$

\end{proof}

\begin{prop*}

        Suppose $D$ is a domain which can be separated into $D=\bigcup_{i=1}^k D_i$.
        If $f$ is integrable over each $D_i$ then it is integrable over $D$ and
        \[ \int_Df = \sum_{i=1}^k\int_{D_i}f \]

\end{prop*}

The proof of this is similar to the proof of the converse we showed above.
Since
\[ \bar s(f,P) - \underbar s(f,P) = \sum_{i=1}^k \bar s(f,P_j) - \underbar s(f,P_j) \]
choose $P_j$s such that the right difference is less than $\frac\epsilon k$ and then we get that $\bar s(f,P)-\underbar s(f,P)<\epsilon$ as required.
Then by above
\[ \int_Df = \sum_{i=1}^k\int_{D_i}f \]

\end{document}
