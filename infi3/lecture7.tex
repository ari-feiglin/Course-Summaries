\documentclass[10pt]{article}

\usepackage{amsmath, amssymb, mathtools}
\usepackage{tikz}
\usepackage[margin=1.5cm]{geometry}

\input mathex
\input prettyprint
\input preamble

\setscalefactor{10}
\initpps

\def\pmat#1{\begin{pmatrix} #1 \end{pmatrix}}

\def\@blist[#1]{%
    \bgroup\bgroup\par%\vskip-\medskipamount%
    \gdef\item{%
        \par\egroup\bgroup\medskip\setbox0=\hbox{#1\quad}%
        \advance\leftskip by \wd0\leavevmode\kern-\wd0\box0%
    }%
}
\def\blist{\@ifnextchar[ \@blist {\@blist[$\bullet$]}}
\def\elist{\par\egroup\egroup\medskip}

\let\divides=\mid
\newfunc{metric}\rho({})
\newfunc{metricc}\sigma({})

\font\bigbf = cmbx12 scaled 2000

\def\openset{{\cal O}}
\let\lineseg=\overleftrightvecc

\begin{document}

\c@section=7

\barcolorbox{220, 255, 220}{0, 130, 0}{80, 200, 80}{
    \leftskip=0pt plus 1fill \rightskip=\leftskip
    {\bigbf Infinitesimal Calculus 3}

    \medskip
    \textit{Lecture 7, Sunday November 13, 2022}

    \textit{Ari Feiglin}
}

\bigskip

\subsection{Complete Metric Spaces}

\begin{defn*}

    A metric space $(X,\metric)$ is \ppemph{complete} if every cauchy sequence in $X$ is convergent.

\end{defn*}

For example, $\bR$ is complete but $\bQ$ is not.

\begin{prop*}

    If $\set{x_n}_{n=1}^\infty$ is a cauchy sequence, it is bounded.
    And if $\set{x_n}_{n=1}^\infty$ has a convergent subsequence, it itself is convergent.

\end{prop*}

\begin{proof}

    Let $\epsilon>0$ then there exists a $N$ such that for every $n,m\geq N$: $\metricof{x_n,x_m}<\epsilon$.
    Let $x=X_N$, and we define
    \[ M = \maxof[1\leq n< N]{\metricof{x_n,x}, \epsilon} \]
    then for every $n\in\bN$ we have that $\metricof{x_n,x}\leq M$ since if $n<N$ then by definition $\metricof{x_n,x}\leq M$ since $M$ is the maximum distance.
    And if $n\geq N$ then $\metricof{x_n,x}<\epsilon\leq M$.
    So $\set{x_n}_{n=1}^\infty$ is bounded.

    Suppose $x_{n_k}$ is a convergent subsequence which converges to $x\in X$.
    Then let $\epsilon>0$, and so there exists an $N$ which satsifes the definition of cauchy sequences for $\epsilon$.
    Let $n\geq N$, and there must be a $k$ such that $n_k\geq N$ and $\metricof{x_{n_k},x}<\epsilon$ (since it converges to $x$).
    So for every $n\geq N$ by the definition of a cauchy sequence:
    \[ \metricof{x_n, x} \leq \metricof{x_n, x_{n_k}} + \metricof{x_{n_k}, x} < 2\epsilon \]
    And so for every $\epsilon>0$ there is an $N$ such that for every $n\geq N$: $\metricof{x_n,x} < 2\epsilon$, so $x_n$ converges to $x$ as required.

    \hfill$\blacksquare$

\end{proof}

\begin{prop*}

    $\bR^n$ is complete.

\end{prop*}

\begin{proof}

    Suppose $\set{x_n}_{n=1}^\infty$ is a cauchy sequence in $\bR^n$.
    Then it is bound, and by Weierstrauss it has a convergent subsequence.
    So by above since it is a cauchy sequence with a convergent subsequence, it itself is convergent.
    So $\bR^n$ is complete.

    \hfill$\blacksquare$

\end{proof}

\begin{prop*}

    If $(X,\metric)$ is a compact metric space, it is complete.

\end{prop*}

\begin{proof}

    Suppose $\set{x_n}_{n=1}^\infty$ is a cauchy sequence in $X$, then since $X$ is a compact space $x_n$ has a convergent subsequence.
    And since it is cauchy, $x_n$ is therefore convergent.
    So $X$ is complete.

    \hfill$\blacksquare$

\end{proof}

\begin{prop*}

    If $(X,\metric)$ is a complete metric space and $S\subseteq X$, then $S$ is closed if and only if $(S,\metric)$ is complete (we restrict $\metric$ to $S\times S$).

\end{prop*}

\begin{proof}

    Suppose $S$ is closed and $\set{x_n}_{n=1}^\infty$ is a cauchy sequence in $S$, then it is cauchy in $X$ and therefore converges to some $x\in X$.
    Therefore since $x_n\longvarrightarrow x$ and $x_n\in S$, $x\in\overline S$.
    Because $S$ is closed, $\overline S=S$ and therefore $x\in S$.
    So $x_n$ converges to a value in $S$, that is it is convergent in $S$.
    So every cauchy sequence in $S$ is convergent, and therefore $S$ is complete.

    Suppose $S$ is complete and $x\in S'$ then there exists a sequence $x_n\in S$ such that $x_n\longvarrightarrow x$.
    Since $\set{x_n}$ is convergent in $X$, it is cauchy in $S$, and therefore converges to a value in $S$.
    Therefore $x\in S$, that is $S'\subseteq S$, so $S$ is closed.

    \hfill$\blacksquare$

\end{proof}

\subsection{Continuous Mappings Between Metric Spaces}

\begin{defn*}

    If $(X,\metric)$ and $(Y,\metricc)$ are metric spaces, a \ppemph{mapping} (or a function) $f$ between them is a function:
    \[ f\colon X\longvarrightarrow Y \]
    And a \ppemph{restriction} of $f$ onto $E\subseteq X$ is a mapping $f\vert_E$ between $(E,\metric)$ and $(Y,\metricc)$ such that for every $x\in E: f\vert_E(x)=f(x)$.

\end{defn*}

\begin{defn*}

    If $f$ is a mapping between $X$ and $Y$ and $p$ is a limit point of $X$, we say
    \[ \lim_{x\varrightarrow p}f(x) = q \]
    if for every sequence $p\neq x_n\longvarrightarrow p$ in $X$, $f(x_n)\longvarrightarrow q$.

\end{defn*}

\begin{thrm*}

    Suppose $f$ is a mapping between metric space, then the following are equivalent:
    \blist
        \item $\lim_{x\varrightarrow p}f(x)=q$
        \item For every $\epsilon>0$ there is a $\delta>0$ such that for every $p\neq x\in B_\delta(p)$, $f(x)\in B_\epsilon(q)$.
        \item For every $K\subseteq X$ where $p$ is a limit point of $K$: $\lim_{x\varrightarrow p}f\vert_K(x)=q$ in $K$.
    \elist

\end{thrm*}

\begin{proof}

    Suppose $\lim f(x)=q$ and assume for the sake of a contradiction that there is an $\epsilon>0$ such that for every $\delta>0$ there is a $p\neq x\in B_\delta(p)$ such that $f(x)\notin B_\epsilon(q)$.
    Take $\delta_n=\frac1n$ and $x_n$ to be the $x_n$ which satisfies the above for $\delta_n$.
    Then $p\neq x_n\longvarrightarrow p$, but $\metricof{x_n, q}\geq\epsilon$, so $x_n$ doesn't converge to $q$ in contradiction.

    To prove the converse, suppose $p\neq x_n\longvarrightarrow p$.
    Then let $\epsilon>0$, so there exists a $\delta>0$ which satisfies the $\epsilon-\delta$ criterion, and since $x_n\longvarrightarrow p$, there exists an $N$ such that for every $n\geq N$ we have that
    $x_n\in B_\delta(p)$, so $f(x_n)\in B_\epsilon(q)$.
    So for every $\epsilon>0$ there is an $N$ such that for every $n\geq N$ we have $\metricof{f(x_n), q}<\epsilon$ so $f(x_n)$ converges to $q$.

    We will now show that $1$ is equivalent to $3$.
    If we assume $1$ then $3$ is trivial.
    Now assume $3$, suppose $p\neq x_n\varrightarrow p$, then $p$ is a limit point of $K=\set{x_n}[n\in\bN]$, and so:
    \[ \lim_{x\varrightarrow p}f\vert_K(x) = q \]
    and since $\set{x_n}$ is a sequence in $K$ which converges to $p$ in $X$ and isn't equal to $p$:
    \[ p = \lim_{x\varrightarrow p}f\vert_K(x) = \lim f\vert_K(x_n) = \lim f(x_n) \]
    as required.

    \hfill$\blacksquare$

\end{proof}

\begin{exam}

    We define the function $f\colon\bR^2\setminus\set{(0,0)}\longvarrightarrow\bR$ by:
    \[ f(x,y) = \frac{xy}{x^2 + y^2} \]
    and we'd like to compute the limit
    \[ \lim_{(x,y)\varrightarrow(0,0)}f(x,y) \]
    If we take $K_k=\set{(x,xk)}[x\in\bR]$ then $(0,0)$ is a limit point of every $K_k$.
    But the limit in $K$ is equal to:
    \[ \lim_{(x,xk)\varrightarrow(0,0)} f(x) = \lim_{x\varrightarrow0}\frac{kx^2}{x^2(1+k^2)} = \lim_{x\to0}\frac{k}{1+k^2} = \frac{k}{1+k^2} \]
    which is different for every $k$, and therefore the limit doesn't exist.

\end{exam}

\begin{exam}

    \[ \lim_{(x,y)\varrightarrow(0,0)}\frac{x^2y}{x^4+y^2} \]
    What happens on $y=kx$? We get:
    \[ \lim_{x\varrightarrow0}\frac{kx^3}{x^2(x^2+k^2)} = \lim_{x\varrightarrow0}\frac{kx}{x^2+k^2} = 0 \]
    So if the limit exists, it is $0$.
    But if we take $y=kx^2$ then the limit:
    \[ \lim_{x\varrightarrow0}f(x,kx^2) = \lim_{x\varrightarrow0}\frac{kx^4}{x^4(1+k^2)} = \frac{k}{1+k^2} \]
    which is not equal to $0$ if $k\neq0$, and therefore the limit doesn't exist.

\end{exam}

\end{document}

