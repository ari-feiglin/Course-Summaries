\documentclass[10pt]{article}

\usepackage{amsmath, amssymb, mathtools}
\usepackage{tikz}
\usepackage[margin=1.5cm]{geometry}

\input pdfmsym
\input prettyprint
\input preamble

\pdfmsymsetscalefactor{10}
\initpps

\def\pmat#1{\begin{pmatrix} #1 \end{pmatrix}}

\let\divides=\mid
\newfunc{metric}\rho({})
\newfunc{metricc}\sigma({})
\newfunc{spa}{{\rm span}}(\vert)
\newfunc{diam}{{\rm diam}}(\vert)

\font\bigbf = cmbx12 scaled 2000
\@undervecc@def{underbar}\@linecap\@linecap

\def\openset{{\cal O}}
\let\lineseg=\overleftrightvecc
\let\ds=\displaystyle

\def\pdv#1#2{\frac{\partial #1}{\partial #2}}

\def\differ#1#2{\left.d#1\strut\right|_{#2}}

\def\@ppmathcount{\thesection.\thepp@mathcount}

\begin{document}

\c@section=22

\barcolorbox{220, 255, 220}{0, 130, 0}{80, 200, 80}{
    \leftskip=0pt plus 1fill \rightskip=\leftskip
    {\bigbf Infinitesimal Calculus 3}

    \medskip
    \textit{Lecture \thesection, Wednsday January 18, 2023}

    \textit{Ari Feiglin}
}

\bigskip

\begin{thrm*}

    If $f,h_1,\dots,h_k\colon\bR^{n+k}\longvarrightarrow\bR$ functions in $C^1$.
    We define
    \[ S = \set{x\in\bR^{n+k}}[h_1(x)=\dots=h_k(x)=0] \]
    If $f\big|_S$ has a critical point $p\in S$ and $\set{\nabla h_i(p)}_{i=1,\dots,k}$ are linearly independent then there exists $\lambda_1,\dots,\lambda_k$ such that
    \[ \nabla f(p) = \sum_{i=1}^k\lambda_i\cdot\nabla h_i(p) \]

\end{thrm*}

\begin{proof}

    Since $\set{\nabla h_i}$ are independent, then the function $h=(h_1,\dots,h_k)$ has an invertible Jacobian at $p$ and therefore by the implicit function theorem, $S$ is the graph of an implicit function
    around $p$ in $C^1$.
    Let $\mathrm{TS}$ be the tangent space to $S$.
    We know that by the previous lemma $\nabla f(p)\in\mathrm{TS}$, and since $S$ is defined via $k$ variables, $\mathrm{TS}$ has a dimension of $n$ so
    $\mathrm{TS}=\spa\set{\nabla h_1(p),\dots,\nabla h_k(p)}^\top$, and so $\nabla f(p)$ is in this span.

    \hfill$\blacksquare$

\end{proof}

\subsection{Integrals}

\begin{defn*}

    A set $D\subseteq\bR^n$ is a \ppemph{open domain} if it is an open and connected set.
    And $\bar D$ is called a \ppemph{closed domain}.

\end{defn*}

A prism $T=\set{(x_1,\dots,x_n)}[a_i<x_j<b_j]=\prod_{i=1}^n (a_i,b_i)$ has volume
\[ \abs T = \prod_{i=1}^n (b_i-a_i) \]
For a set $D$ we define its \emph{internal volume} to be
\[ \abs{D}_{\rm int} = \sup\sum_{i=1}^k\abs{T_i} \]
where the supremum is taken over all sets of prisms $\set{T_i}_{i=1}^k$ where $\dcup T_i\subseteq D$.
The \emph{external volume} is defined to be
\[ \abs{D}_{\rm ext} = \inf\sum_{i=1}^k\abs{T_i} \]
where the infimum is taken over all sets of prisms where $\dcup T_i\supseteq D$.

\begin{defn*}

    A set $D$ is \ppemph{contented} if its internal and external volumes are equal.

\end{defn*}

It can be shown that $D$ is contented if and only if $\abs{\partial D}_{\rm ext}=0$.

\begin{defn*}

    A \ppemph{partition} of the domain $\bar D$ is a set $\set{\bar D_i}[1\leq i\leq k]$ of contented closed domains which are pairwise disjoint and
    \[ \bigdcup_{i=1}^k \bar D_i = \bar D \]

\end{defn*}

\begin{defn*}

    Given a partition $P=\set{\bar D_i}_{i=1}^k$ we define
    \[ M_i = \sup\set{f(x)}[x\in\bar D_i] \qquad m_i = \inf\set{f(x)}[x\in\bar D_i] \]
    The \ppemph{upper sum} of $P$ is
    \[ \bar S(f,P) = \sum_{i=1}^k M_j\cdot\abs{\bar D_j} \]
    and the \ppemph{lower sum} of $P$
    \[ \underbar S(f,P) = \sum_{i=1}^k m_j\cdot\abs{\bar D_j} \]

\end{defn*}

\begin{defn*}

    The \ppemph{width} of a partition $P$ is $\lambda(P)=\max\diam(D_j)$.

\end{defn*}


Suppose $D$ is contented, and so is every $D_j$.
Further suppose $f\colon\bar D\longvarrightarrow\bR$ is bounded, then
\[ m\abs{\bar D} = m\sum_{i=1}^k\abs{D_j} = \sum_{i=1}^k m\abs{D_j} \leq \sum_{i=1}^k m_j\abs{D_j} \leq \sum_{i=1}^k M_j\abs{D_j} \leq M\sum_{i=1}^k\abs{D_j} \leq M\abs{\bar D_j} \]
And specifically
\[ m\abs{\bar D} \leq \underbar S(f,P) \leq \bar S(f,P) \leq M\abs D \]
Thus all the lower and upper sums are bounded by constants and thus the following is well-defined:

\begin{defn*}

    The \ppemph{upper integral} and \ppemph{lower integral} are defined as followed, respectively:
    \[ \overline{\int_D}f = \inf_P\bar S(f,P) \qquad \underline{\int_D} f = \sup_P\underbar S(f,P) \]
    And $f$ is \ppemph{integrable} over $D$ if
    \[ \int_D f(x_1,\dots,x_n)dx_1\dots dx_n = S_D = \underline{\int_D}f = \overline{\int_D}f \]

\end{defn*}

\begin{defn*}

    A partition $Q$ is a \ppemph{refinement} of a partition $P=\set{P_i}_{i=1}^k$ if it is obtained from $P$ by further partitioning each $P_i$.

\end{defn*}

\begin{prop*}

    If $Q$ is a refinement of $P$ then
    \[ \underbar S(f,P) \leq \underbar S(f,Q) \leq \bar S(f,Q) \leq \bar S(f,P) \]

\end{prop*}

\begin{proof}

    We can write $Q$ as the ``union'' of partitions $Q_j$ (partitions, not domains) which partition $P_j$.
    It is then trivial to see that
    \[ \bar S(f,Q) = \sum_{j=1}^k \bar S(f,Q_j) \]
    and we know by the previous proposition that
    \[ \bar S(f,Q_j) \leq M_j\abs{P_j} \]
    where $M_j$ is taken from $P_j$, since $Q_j$ is a partition of the domain $P_j$.
    Thus
    \[ \bar S(f,Q) \leq \sum_{j=1}^k M_j\abs{P_j} = \bar S(f,P) \]
    as required.
    Similar for lower sums.

    \hfill$\blacksquare$

\end{proof}

\begin{prop*}

    If $P$ and $Q$ are \emph{any} two partitions then
    \[ \underbar S(f,P) \leq \bar S(f,Q) \]

\end{prop*}

\begin{proof}

    We define a new partition $T$ which is a refinement of both $P$ and $Q$, this can be done by taking all possible intersections of $P_i$ and $Q_j$, that is:
    \[ T = \set{T_{ij}}[T_{ij} = P_i\cap Q_j\neq\varnothing] \]
    This is a contented partition since $\partial(P_i\cap Q_j)\subseteq\partial P_i\cup\partial Q_j$ and it can be shown that $\abs{A\cup B}\leq\abs A+\abs B$, and so $\partial{P_i\cap Q_j}=0$.
    Thus
    \[ \underbar S(f,P) \leq \underbar S(f,T) \leq \bar S(f,T) \leq \bar S(f,Q) \]
    as required.

    \hfill$\blacksquare$

\end{proof}

\begin{prop*}

    \[ \underline{\int_D}f\leq\overline{\int_D}f \]

\end{prop*}

This is true because the set of lower integrals is less than the set of upper integrals.

\begin{prop*}

    $f$ is integrable in $D$ if and only if for every $\epsilon>0$ there is a partition $P$ such that
    \[ \bar S(f,P) - \underbar S(f,P) < \epsilon \]

\end{prop*}

\begin{proof}

    Suppose this is true for every $\epsilon>0$, then
    \[ \overline{\int_D}f - \underline{\int_D}f \leq \bar S(f,P) - \underbar S(f,P) < \epsilon \]
    so the upper and lower integrals are equal, and therefore $f$ is integrable.

    And if $f$ is integrable, take $\epsilon>0$ then there are partitions $P$ and $Q$ such that
    \[ 0\leq \bar s(f,P) - \overline\int f < \frac\epsilon2 \]
    and
    \[ 0\leq \underline{\int} f - \underbar s(f,Q) < \frac\epsilon 2 \]
    If we take a common refinement $K$ of $P$ and $Q$ then we get the same inequalities.
    Then adding both inequalities gives
    \[ 0 \leq \bar s(f,K) - \underbar S(f,K) < \epsilon \]
    since the upper and lower integrals are equal.

    \hfill$\blacksquare$

\end{proof}

\end{document}

