\documentclass[10pt]{article}

\usepackage{amsmath, amssymb, mathtools}
\usepackage{tikz}
\usepackage[margin=1.5cm]{geometry}

\input mathex
\input prettyprint
\input preamble

\setscalefactor{10}
\initpps

\def\pmat#1{\begin{pmatrix} #1 \end{pmatrix}}

\def\@blist[#1]{%
    \bgroup\par%\vskip-\medskipamount%
    \def\item{%
        \par\egroup\bgroup\medskip\setbox0=\hbox{#1\quad}%
        \advance\leftskip by \wd0\leavevmode\kern-\wd0\box0%
    }%
    \bgroup%
}
\def\blist{\@ifnextchar[ \@blist {\@blist[$\bullet$]}}
\def\elist{\par\egroup\egroup\medskip}

\let\divides=\mid
\newfunc{metric}\rho({})
\newfunc{metricc}\sigma({})

\font\bigbf = cmbx12 scaled 2000

\def\openset{{\cal O}}
\let\lineseg=\overleftrightvecc
\let\ds=\displaystyle

\begin{document}

\c@section=9

\barcolorbox{220, 255, 220}{0, 130, 0}{80, 200, 80}{
    \leftskip=0pt plus 1fill \rightskip=\leftskip
    {\bigbf Infinitesimal Calculus 3}

    \medskip
    \textit{Lecture 9, Sunday November 20, 2022}

    \textit{Ari Feiglin}
}

\bigskip

\subsection{Continuity Continued}

Notice that $f$ is continuous at $x$ if and only if for every $x_n\longvarrightarrow x$ (and there exists such a sequence, since we can take $x_n=x$), $f(x_n)\longvarrightarrow f(x)$.
Suppose $f$ is continuous at $x$, let $\epsilon>0$ and $\delta>0$ satisfy continuity.
Then since $\metricof{x_n,x}\longvarrightarrow0$, so for some $N$ for every $n\geq N$, $\metricof{x_n,x}<\delta$ and so $\metricof{f(x_n),f(x)}<\epsilon$.
And therefore $f(x_n)\longvarrightarrow f(x)$.
To show the converse, suppose that $f$ isn't continuois at $x$.
Then there is an $\epsilon>0$ such that every $\delta>0$ doesn't satisfy continuity.
So for $\frac1n$ there is a $x_n$ such that $\metricof{x_n,x}<\frac1n$ but $\metricof{f(x_n),f(x)}\geq\epsilon$.
So while $x_n\longvarrightarrow x$, $f(x_n)$ doesn't converge to $f(x)$, in contradiction.

\begin{prop*}

    We define the function $\chi_k\colon\bR^n\longvarrightarrow\bR$ by $f_k(x_1,\dots,x_n)=x_k$.
    Then $f_k$ is continuous.

\end{prop*}

\begin{proof}

    Suppose $x^m\longvarrightarrow x=(x_1,\dots,x_n)$.
    Then we must have that $x^m_k\longvarrightarrow x_k$ since pointwise convergence in $\bR^n$ is equivalent to convergence.
    And so $f_k(x^m)=x^m_k\longvarrightarrow x_k = f_k(x)$, and so $f_k$ is continuous.

    \hfill$\blacksquare$

\end{proof}


\begin{exam}

    The function $f(x,y,z)=z^3e^{\sin z}$ is continuous since if we define $g(z)=z^3 e^{\sin z}$, we have that $f(x,y,z)=g(f_3(x,y,z))$.
    And $g$ and $f_3$ are continuous, and the composition of continuous functions is continuous, so $f$ is continuous.

\end{exam}

\begin{prop*}

    If $f,g\colon E\longvarrightarrow\bR$ are continuous at $p\in E$ and if $c\in\bR$ then the following are also continuous at $p$:
    \begin{gather*}
        f + g \\
        f + cg \\
        f\cdot g \\
        \frac fg \rlap{\quad If $g(p)\neq0$}
    \end{gather*}

\end{prop*}

This is trivial to prove.

Also notice that $f\colon X\longvarrightarrow\bR^n$ is continuous if and only if $\chi_k\circ f$ is continuous for every $1\leq k\leq n$.
This is due to a proposition we proved in the previous lecture.

\begin{exam}

    Circles are continuous.
    We define the function:
    \[ f\colon[0,2\pi)\longvarrightarrow\bR,\qquad t\varmapsto(\cos(t), \sin(t)) \]
    which is the parametric representation of the unit circle.
    Since both of the coordinate functions, $\chi_1\circ f$ and $\chi_2\circ f$ ($\cos t$ and $\sin t$ respectively) are continuous, so is $f$.
    Notice that this parametric representation is the \emph{counter-clockwise} representation.
    The clockwise representation, $(\cos t, -\sin t)$ is also continuous.

\end{exam}

\newpage
\subsection{Surfaces}

\begin{defn*}

    If $\vec v,\vec n\in\bR^k$ if $\vec n\neq0$, the \ppemph{hyperplane} normal to $\vec n$ which contains $\vec v$ is defined as:
    \[ H_{v,n} = \set{\vec u\in\bR^k}[n\cdot(u-v)=0] = \set{\vec u\in\bR^k}[n\cdot u = n\cdot v] \]

\end{defn*}

Notice that if $u\in H_{v,n}$ then $H_{v,n}=H_{u,n}$ and for every $\alpha\neq0$, $H_{v,\alpha n}=H_{v,n}$.
Further notice that the set $\set{\vec u\in\bR^k}[n\cdot u=d]$ for any $d\in\bR$ defines a hyperplane.
And if $u\in\bR^n$ then $H_{v,n}+u=\set{u+w\in\bR^k}[n\cdot(w-v)=0] = \set{w\in\bR^k}[n\cdot(w-v-u)=0] = H_{v+u,n}$.
And if $0\in H_{v,n}$ then $H_{v,n}$ represents a subspace of $\bR^k$ whose dimension is $k-1$, and so $H_{v,n}-v=H_{0,n}$ is a $k-1$ dimension subspace of $\bR^k$.

\begin{defn*}

    If $f\colon\bR\longvarrightarrow\bR$ is a real-valued function, then its rotation around an axis, for example the $z$ axis is defined as the set:
    \[ \set{(x,y,f(\sqrt{x^2+y^2}))}[x,y\in\bR] \]

\end{defn*}

If we intersect a rotation with some plane $z=c$ then we get the set of points $(x,y,c)$ where $f(\sqrt{x^2+y^2})=c$.
This is either empty or contains circles whose radii are in $f^{-1}(c)$.

\begin{exam}

    If $f(x)=\alpha x$, then rotating it around $z$ creates the set: $\set{(x,y,\alpha\sqrt{x^2+y^2})}[x,y\in\bR]$ which is called a \ppemph{cone}.
    And for $f(x)=\alpha x^2$, rotating it gives $z=\alpha(x^2+y^2)$, this is called an elliptical parabola.

\end{exam}

\begin{exam}

    If we look at $f(x,y)=\frac{x^2}{a^2}+\frac{y^2}{b^2}$, $z=f(x,y)$.
    Cutting this at $z=c$ yields an ellipse, and thus this is not a rotation.
    Cutting this at $y=0$ gives $z=\frac{x^2}{a^2}$ which is a parabola.

\end{exam}

\end{document}

