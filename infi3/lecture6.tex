\documentclass[10pt]{article}

\usepackage{amsmath, amssymb, mathtools}
\usepackage{tikz}
\usepackage[margin=1.5cm]{geometry}

\input mathex
\input prettyprint
\input preamble

\setscalefactor{10}
\initpps

\def\pmat#1{\begin{pmatrix} #1 \end{pmatrix}}

\def\@blist[#1]{%
    \bgroup\bgroup\par\vskip-\medskipamount%
    \gdef\item{%
        \par\egroup\bgroup\medskip\setbox0=\hbox{#1\quad}%
        \advance\leftskip by \wd0\leavevmode\kern-\wd0\box0%
    }%
}
\def\blist{\@ifnextchar[ \@blist {\@blist[$\bullet$]}}
\def\elist{\par\egroup\egroup\medskip}

\let\divides=\mid
\newfunc{metric}\rho({})

\font\bigbf = cmbx12 scaled 2000

\def\openset{{\cal O}}
\let\lineseg=\overleftrightvecc

\begin{document}

\c@section=6

\barcolorbox{220, 255, 220}{0, 130, 0}{80, 200, 80}{
    \leftskip=0pt plus 1fill \rightskip=\leftskip
    {\bigbf Infinitesimal Calculus 3}

    \medskip
    \textit{Lecture 6, Wednsday November 9, 2022}

    \textit{Ari Feiglin}
}

\bigskip

\subsection{Sequences and Limits in Metric Spaces}

\begin{defn*}

    If $(X,\metric)$ is a metric space, a sequence $\set{x_n}_{n=1}^\infty$ in $X$ \ppemph{converges} to $x\in X$ if
    \[ \lim_{n\to\infty}\metricof{x_n, x} = 0 \]
    We denote this by $\lim\limits_{n\to\infty}x_n=x$ or $x_n\varrightarrow x$.

\end{defn*}

\begin{prop*}

    \blist
        \item Limits, if they exist, are unique.
        \item Constant sequences $\set{x}_{n=1}^\infty$ converge to $x$.
        \item If $x_n\varrightarrow x$ and $y_n\varrightarrow y$ then $\metricof{x_n, y_n}\varrightarrow\metricof{x,y}$
    \elist

\end{prop*}

\begin{proof}

    \blist
        \item Suppose $x_n\varrightarrow x,y$ so by the triangle inequality for every $n\in\bN$:
            \[ \metricof{x,y} \leq \metricof{x,x_n} + \metricof{x_n,y} \]
            Taking the limit of the right side gives $0$ by definition, so $\metricof{x,y}=0$ and therefore $x=y$.
        \item This is trivial since $\metricof{x,x}=0$.
        \item Since:
            \[ \metricof{x,y} \leq \metricof{x, x_n} + \metricof{x_n, y_n} + \metricof{y_n, y} \]
            The limit of the right side is $\lim\metricof{x_n,y_n}$, so $\metricof{x,y}\leq\lim\metricof{x_n,y_n}$.
            And:
            \[ \metricof{x_n,y_n} \leq \metricof{x_n,x} + \metricof{x,y} + \metricof{y,y_n} \]
            The limit of the right side is $\metricof{x,y}$ so
            \[ \lim\metricof{x_n,y_n} \leq \metricof{x,y} \leq \lim\metricof{x_n,y_n} \]
            And therefore $\metricof{x_n,y_n}\varrightarrow\metricof{x,y}$.
    \elist

    \hfill$\blacksquare$

\end{proof}

\begin{prop*}

    Suppose $X$ is a normed linear space and $x_n\varrightarrow x$ and $y_n\varrightarrow y$ and let $c\in\bR$.
    \blist
        \item $x_n+y_n\varrightarrow x+y$
        \item $cx_n\varrightarrow cx$
        \item $\norm{x_n}\varrightarrow\norm x$
        \item If $X=\bR^n$ and if $x^{(m)}=\parens{x_1^{(m)},\dots,x_n^{(m)}}$ and $x=\parens{x_1,\dots,x_n}$ then $x^{(m)}\varrightarrow x$ if and only if $x_k^{(m)}\varrightarrow x_k$ for every relevant
            $k$.
        \item If $X=\bR^n$ and $x^{(m)}\varrightarrow x$ and $y^{(m)}\varrightarrow y$ then $x^{(m)}\cdot y^{(m)}\varrightarrow x\cdot y$ (dot product).
    \elist

\end{prop*}

\begin{proof}

    \blist
        \item We know $\norm{x_n+y_n-x-y} \leq \norm{x_n-x} + \norm{y_n-y} \varrightarrow 0$ so $x_n+y_n\varrightarrow x+y$ as required.
        \item We know $\norm{cx_n - cx} = \abs c\norm{x_n-x} \varrightarrow 0$ so $cx_n\varrightarrow cx$ as required.
        \item Since $\abs{\norm{x_n}-\norm{x}}\leq\norm{x_n-x}\varrightarrow0$, it must be that $\abs{\norm{x_n}-\norm x}\varrightarrow0$ so $\norm{x_n}\varrightarrow\norm x$ as required.
        \item Notice that
                \[ \norm{x^{(m)}-x}^2 = \sum_{k=1}^n \big(x^{(m)}_k - x_k\big)^2 \varrightarrow 0 \]
            So the left converges to $0$ if and only if every $\big(x^{(m)}_k-x_k\big)^2$ converges to $0$ since squares are non-negative.
            And this in turn is equivalent to $x^{(m)}_k\varrightarrow x_k$.
            It is easy to see how this is actually true for any $p$-norm, not just for $p=2$.
        \item By above we know that $x^{(m)}_k\varrightarrow x_k$ and $y^{(m)}_k\varrightarrow y_k$, and since:
                \[ x^{(m)}\cdot y^{(m)} = \sum_{k=1}^n x^{(m)}_k\cdot y^{(m)}_k \]
            And by limit arithmetic, we know this converges to
                \[ \sum_{k=1}^n x_k\cdot y_k = x\cdot y \]
            As required.
    \elist

    \hfill$\blacksquare$

\end{proof}

\begin{defn*}

    Suppose $(X,\metric)$ is a metric space and $\set{x_n}_{n=1}^\infty$ is a sequence in it.
    If $\set{n_k}$ is a strictly increasing sequence ($n_k<n_{k+1}$), then $\set{x_{n_k}}_{k=1}^\infty$ is a \ppemph{subsequence} of $\set{x_n}_{n=1}^\infty$.
    If $x_{n_k}$ converges to $x$, then $x$ is a \ppemph{partial limit} of $\set{x_n}_{n=1}^\infty$.

\end{defn*}

\begin{prop*}

    If $x_n\varrightarrow x$ then every subsequence of $\set{x_n}$ converges to $x$.

\end{prop*}

This is trivial.

\begin{thrm*}

    If $S\subseteq X$ is compact, then every sequence $\set{x_n}_{n=1}^\infty$ in $S$ has a convergent subsequence.

\end{thrm*}

\begin{proof}

    If there is an element $x$ which is in the sequence an infinite amount of times, we can construct a subsequence of all of its instances, and this subsequence converges to $x$.
    Otherwise there are an infinite number of different elements in $\set{x_n}$.
    Let $x\in S$, then if for every $\epsilon>0$ there is an element $x\neq x_n\in B_\epsilon(x)$, we can taken a sequence $\epsilon_n\varrightarrow0$ such and the associated $x_{n_k}$s converge to $x$,
    and thus we have a convergent subsequence.
    Otherwise, there must be some $\epsilon_x$ such that there is no $x\neq x_n\in B_{\epsilon_x}(x)$.
    So we can take an open cover of $S$ by $\set{B_{\epsilon_x}(x)}_{x\in S}$, and since every one of these balls contains at most one element in $x_n$, every finite subcover contains only a finite
    number of $x_n$s, so it can't cover $S$.
    This contradicts the compactness of $S$.
    And therefore $\set{x_n}$ must have a convergent subsequence.

    \hfill$\blacksquare$

\end{proof}

\begin{thrm*}[bolzWeirTheorem,Bolzano-Weierstrauss\ Theorem]

    Suppose $\set{x_m}$ is bounded in $\bR^n$ then there exists a convergent subsequence of it.

\end{thrm*}

\begin{proof}

    Since $x_m\in B_M(0)$ for some $M>0$, so $x_m\in\bar B_M(0)$.
    And since this ball is closed and bounded (by $B_{M+1}(0)$) for example, then by Heine-Borel, it is compact.
    So $x_m$ is contained inside a compact space and therefore by the above theorem it has a convergent subsequence (moreso, its limit is in $\bar B_M(0)$).

    \hfill$\blacksquare$

\end{proof}

\begin{exam}

    Let $e_n=\set{0,\dots,1,\dots}$ be the sequence in $\ell^2$ which is $0$ except for at its $n$th position.
    Then $\set{e_n}$ is bounded since it is contained in the closed unit ball.
    But no subsequence of it is convergent: let $x\in\ell^2$ then:
    \[ \norm{x-e_{n_k}}^2 \geq (x_{n_k}-1)^2 \]
    And since $x\in\ell^2$, this converges to $1$, so $e_{n_k}$ is not convergent to $x$.

\end{exam}

\begin{defn*}

    Suppose $(X,\metric)$ is a metric space.
    Then a sequence $\set{x_n}_{n=1}^\infty$ in $X$ is a \ppemph{cauchy sequence} if for every $\epsilon>0$ there is some $N\in\bN$ such that for every $n,m\geq N$:
        \[ \metricof{x_n, x_m} < \epsilon \]

\end{defn*}

\begin{prop*}

    Every convergent sequence is also Cauchy.

\end{prop*}

\begin{proof}

    Suppose $x_n\varrightarrow x$, then let $\epsilon>0$ then there exists an $N$ such that for every $n\geq N$:
        \[ \metricof{x_n, x} < \frac\epsilon2 \]
    And so if $n,m\geq N$:
        \[ \metricof{x_n, x_m} \leq \metricof{x_n, x} + \metricof{x, x_m} < \epsilon \]
    And so $\set{x_n}$ is a cauchy sequence, as required.

    \hfill$\blacksquare$

\end{proof}

The reverse of this proposition is not true.
Take $x\in\bR\setminus\bQ$ and take a sequence $q_n$ of rationals which converge to $\bQ$.
Then $q_n$ is cauchy in $\bQ$ (since it is cauchy in $\bR$), but it is not convergent in $\bQ$.

\end{document}

