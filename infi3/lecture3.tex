\documentclass[10pt]{article}

\usepackage{amsmath, amssymb, mathtools}
\usepackage{tikz}
\usepackage[margin=1.5cm]{geometry}

\input mathex
\input prettyprint
\input preamble

\setscalefactor{10}
\initpps

\def\pmat#1{\begin{pmatrix} #1 \end{pmatrix}}

\def\@blist[#1]{%
    \bgroup\bgroup\par\vskip-\medskipamount%
    \gdef\item{%
        \par\egroup\bgroup\medskip\setbox0=\hbox{#1\quad}%
        \advance\leftskip by \wd0\leavevmode\kern-\wd0\box0%
    }%
}
\def\blist{\@ifnextchar[ \@blist {\@blist[$\bullet$]}}
\def\elist{\par\egroup\egroup\medskip}

\let\divides=\mid
\newfunc{metric}\rho({})

\font\bigbf = cmbx12 scaled 2000

\pdf@drawing@macro{check}{0 1 0 RG 1.2 w 1 j 1 J 0 5 m 3 0 l 8 10 l S}{9.2pt}{12pt}{1pt}{1pt}
\pdf@drawing@macro{ecks}{1 0 0 RG 0 0 m 1 j 1 J 10 10 l 0 10 m 10 0 l S}{11.2pt}{12pt}{1pt}{1pt}
\def\openset{{\cal O}}

\begin{document}

\c@section=3

\barcolorbox{220, 255, 220}{0, 130, 0}{80, 200, 80}{
    \leftskip=0pt plus 1fill \rightskip=\leftskip
    {\bigbf Infinitesimal Calculus 3}

    \medskip
    \textit{Lecture 3, Sunday October 30, 2022}

    \textit{Ari Feiglin}
}

\bigskip

\c@subsection=1

\begin{prop*}

    Any arbitrary union of open sets is itself open, and a finite intersection of open sets is open as well.
    Similarly, any intersection of closed sets is closed, and a finite union of closed sets is closed as well.

\end{prop*}

\begin{proof}

    This is trivial.
    Suppose $\set{\openset_i}_{i\in I}$ is a set of open sets, then if:
    \[ x\in\bigcup_{i\in I}\openset_i \]
    there must be some $i\in I$ such that $x\in\openset_i$.
    Since $\openset_i$ is open, there is a $r>0$ such that $B_r(x)\subseteq\openset_i$, and therefore $B_r(x)\subseteq\bigcup\openset_i$, so the union is open.

    Now suppose $\set{\openset_n}_{n=1}^N$ is a finite set of open sets.
    Then if
    \[ x\in\bigcap_{n=1}^N \openset_n \]
    $x$ must be in $\openset_n$ for every $n=1\dots N$.
    Then for every $n=1\dots N$, there must be a $r_n>0$ such that $B_{r_n}(x)\subseteq\openset_n$.
    If we let $r=\minof{r_1,\dots,r_N}>0$, $B_r(x)\subseteq B_{r_n}(x)\subseteq\openset_n$ for every $n$, so:
    \[ B_r(x)\subseteq\bigcap_{n=1}^N\openset_n \]
    as required.

    Now suppose $\set{F_i}_{i\in I}$ is a set of closed sets, then:
    \[ \parens{\bigcap_{i\in I} F_i}^c = \bigcup_{i\in I} F_i^c \]
    which is a union of open sets since $F_i^c$ is open, which we just proved is open.
    So the complement of the intersection is open, and therefore the intersection is closed.

    Similarly for a finite set of closed sets:
    \[ \parens{\bigcup_{n=1}^N F_n}^c = \bigcap_{n=1}^N F_n^c \]
    which is open, and therefore the union is closed.

    \hfill$\blacksquare$

\end{proof}

Notice that an arbitrary intersection of open sets, and thus an arbitrary intersection of closed sets, is not necessarily open (or closed).
Take for instance $\openset_n=\parens{-\frac1n,\frac1n}$, then:
\[ \bigcap_{n\in\bN}\openset_n = \set0 \]
And $\set0$ is not open.

\begin{defn*}

    For a subset $S$ of a metric space $X$, we denote its \ppemph{interior}, the set of all interior points of $S$ as $S^\circ$ or $\interior S$.
    Similarly, the \ppemph{exterior}, the set of all exterior points is $\exterior S$ (or ${S^c}^\circ$), and the \ppemph{boundary} of $S$ is denoted $\boundary S$.

\end{defn*}

Notice that $\interior S\subseteq S$.
Furthermore, $S\cup\boundary S=\interior S\dcup \boundary S$.
These sets are obviously disjoint by definition, and if $x\in S$, $x$ is either in the interior of $S$ or the boundary of $S$.

\begin{lemm*}

    Every limit point of $S$ is in $\interior S\dcup\boundary S = S\cup\boundary S$.

\end{lemm*}

\begin{proof}

    Suppose $x$ is a limit point of $S$ and $x\notin\interior S$.
    Since $x$ is a limit point, for every $r>0$ there is a $x\neq y\in S$ such that $y\in B_r(x)$, so $B_r(x)$ is not a subset of $S^c$.
    And since $x$ is not an interior point of $S$, for every $r>0$, $B_r(x)$ is not a subset of $S$.
    So for every $r>0$, $B_r(x)\cap S$ and $B_r(x)\cap S^c$ are not empty and therefore $x$ is a boundary point of $S$.

    \hfill$\blacksquare$

\end{proof}

\begin{thrm*}

    Suppose $S\subseteq X$ for $X$ metric space.
    Then the following are equivalent:
    \blist
        \item $S$ is closed.
        \item $S$ contains all of its boundary points.
        \item $S$ contains all of its limit points.
    \elist

\end{thrm*}

\begin{proof}

    We will prove the first relation.
    Suppose $x\in\boundary S$, then if $x\notin S$, $x\in S^c$, and since $\boundaryof{S^c}=\boundary S$, $x\in\boundaryof{S^c}\cap S^c$.
    But $S^c$ is open and thus doesn't contain its boundary points, in contradiction, so $x\in S$.
    Therefore $\boundary S\subseteq S$.

    Now suppose $\boundary S\subseteq S$.
    Then since every limit point is in $\boundary S\cup S=S$, every limit point is in $S$.

    Now suppose $S$ contains all of its limit points.
    Suppose $x\in S^c$, then $x$ is not a limit point of $S$, so there must be a $r>0$ such that $B_r(x)$ contains no points in $S$ other than $x$.
    Since $x$ is not in $S^c$, $B_r(x)\subseteq S^c$.
    So $S^c$ is open and therefore $S$ is closed.

    \hfill$\blacksquare$

\end{proof}

\begin{defn*}

    For $S\subseteq X$ a metric space, $S'$ is the set of all limit points of $S$, and is also called the \ppemph{limit set} of $S$.

\end{defn*}

\begin{lemm*}

    Let $S\subseteq X$ a metric space and $x\in S^c$.
    Then $x$ is a boundary point of $S$ if and only if $x$ is a limit point of $S$.

\end{lemm*}

This proof is trivial, if $x$ is a boundary point of $S$, for every $r>0$ there must be a $y\in B_r(x)\cap S$, and since $x\notin S$, $y\neq x$, so $x$ is a limit point.
And if $x$ is a limit point, then since $x$ is in $B_r(x)\cap S^c$, it is not open, and since there must be a $y\in B_r(x)\cap S$ since it's a limit point, $x$ is a boundary
point of $S$.

Notice then that
\[ S' \subseteq S\cup \boundary S = \interior S\dcup\boundary S \]
Since if $x\in S'$ and $x\notin S$, then we just showed $x\in\boundary S$.

\begin{defn*}

    If $S$ is a subset of a metric space $X$, then the closure of $S$, denoted $\overline S$, is the smallest possible closed set containing $S$:
    \[ \overline S = \bigcap_{\substack{F \text{ closed} \\ S\subseteq F}} F \]

\end{defn*}

\begin{prop*}

    If $S$ is a subset of a metric space $X$, then $\overline S = \interior S\dcup\boundary S = \interior S\dcup\boundaryof{S^c}$.

\end{prop*}

\begin{proof}

    Since $X = \interior S\dcup\boundary S\dcup\interiorof{S^c}$, $\interiorof{S^c}$ is open so its complement, $\interior S\dcup\boundary S$ is closed.
    This contains $S$ so $\overline S\subseteq\interior S\dcup\boundary S$.
    And if $F$ is a closed set which contains $S$, it must contain $S$'s interior (since it is a subset of $S$) and its boundary, so the sets are equal.

    \hfill$\blacksquare$.

\end{proof}

\begin{thrm*}

    If $S\subseteq X$ a metric space, the following are equal:
    \blist
        \item $\overline S$
        \item $S\cup S'$
        \item $S\cup\boundary S$
    \elist

\end{thrm*}

\begin{proof}

    The first equality was proven above.
    We know that $S'\subseteq S\cup\boundary S$, so $S\cup S'\subseteq S\cup\boundary S$.
    Suppose $x\in\boundary S$ and $x\notin S$.
    Then we know that $x$ is a limit point, so $S\cup\boundary S\subseteq S\cup S'$.
    Therefore $S\cup S'=S\cup\boundary S$.

    \hfill$\blacksquare$

\end{proof}

\begin{defn*}

    A set $S\subseteq X$ is \ppemph{compact} if for every set of open sets $\set{\openset_i}_{i\in I}$ such that
    \[ S\subseteq\bigcup_{i\in I}\openset_i \]
    there is a finite subcovering, that is a finite indexing set $I'\subseteq I$ such that:
    \[ S\subseteq\bigcup_{i\in I'}\openset_i \]

\end{defn*}

\end{document}

