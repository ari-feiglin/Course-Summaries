\documentclass[10pt]{article}

\usepackage{amsmath, amssymb, mathtools}
\usepackage{tikz}
\usepackage[margin=1.5cm]{geometry}

\input pdfmsym
\input prettyprint
\input preamble

\pdfmsymsetscalefactor{10}
\initpps

\def\pmat#1{\begin{pmatrix} #1 \end{pmatrix}}

\let\divides=\mid
\newfunc{metric}\rho({})
\newfunc{metricc}\sigma({})

\font\bigbf = cmbx12 scaled 2000

\def\openset{{\cal O}}
\let\lineseg=\overleftrightvecc
\let\ds=\displaystyle

\def\pdv#1#2{\frac{\partial #1}{\partial #2}}

\def\differ#1#2{\left.d#1\strut\right|_{#2}}

\def\@ppmathcount{\thesection.\thepp@mathcount}

\begin{document}

\c@section=18

\barcolorbox{220, 255, 220}{0, 130, 0}{80, 200, 80}{
    \leftskip=0pt plus 1fill \rightskip=\leftskip
    {\bigbf Infinitesimal Calculus 3}

    \medskip
    \textit{Lecture \thesection, Wednsday January 11, 2023}

    \textit{Ari Feiglin}
}

\bigskip

Notice that if we have $y=(y_1,\dots,y_s)$ and $x=(x_1,\dots,x_k)$ and $s$ functions $F_i$ with the system
\[ \left\{\begin{gathered}F_1(x_1,\dots,x_k,y_1,\dots,y_s) = 0 \\ \vdots \\ F_s(x_1,\dots,x_k,y_1,\dots,y_s) = 0 \end{gathered}\right. \]
We can assume $y_i=\phi_i(x_1,\dots,x_k)$ where $\phi_i\in C_1$ by the Implicit Function Theorem (we assume $J_{F_i}(x,y)\neq0$).
We can then differentiate these equations relative to $x_j$ to get
\[ \left\{\begin{gathered}\pdv{F_1}{x_j} + \sum_{i=1}^s\pdv{F_1}{y_i}\cdot\pdv{y_i}{x_j} = 0 \\ \vdots \\ \pdv{F_s}{x_j} + \sum_{i=1}^s\pdv{F_s}{y_i}\cdot\pdv{y_i}{x_j} = 0 \end{gathered}\right. \]
And therefore we get that
\[ \sum_{i=1}^s\pdv{F_t}{y_i}\pdv{y_i}{x_j} = -\pdv{F_i}{x_j} \]
This is equivalent to saying that
\[ J_F\begin{pmatrix}\pdv{y_1}{x_j}\\\vdots\\\pdv{y_s}{x_j}\end{pmatrix} = -\begin{pmatrix}\pdv{F_1}{x_j}\\\vdots\\\pdv{F_s}{x_j}\end{pmatrix} \]
Where
\[ J_F = \pdv{(F_1,\dots,F_s)}{(y_1,\dots,y_s)} \]

\end{document}
