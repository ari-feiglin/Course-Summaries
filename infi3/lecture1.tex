\documentclass[10pt]{article}

\usepackage{amsmath, amssymb, mathtools}
\usepackage{tikz}
\usepackage[margin=1.5cm]{geometry}

\input mathex
\input prettyprint
\input preamble

\setscalefactor{10}
\initpps

\def\pmat#1{\begin{pmatrix} #1 \end{pmatrix}}

\def\@blist[#1]{%
    \bgroup\bgroup\par\vskip-\medskipamount%
    \gdef\item{%
        \par\egroup\bgroup\medskip\setbox0=\hbox{#1\quad}%
        \advance\leftskip by \wd0\leavevmode\kern-\wd0\box0%
    }%
}
\def\blist{\@ifnextchar[ \@blist {\@blist[$\bullet$]}}
\def\elist{\par\egroup\egroup\medskip}

\let\divides=\mid

\font\bigbf = cmbx12 scaled 2000

\pdf@drawing@macro{check}{0 1 0 RG 1.2 w 1 j 1 J 0 5 m 3 0 l 8 10 l S}{9.2pt}{12pt}{1pt}{1pt}
\pdf@drawing@macro{ecks}{1 0 0 RG 0 0 m 1 j 1 J 10 10 l 0 10 m 10 0 l S}{11.2pt}{12pt}{1pt}{1pt}

\begin{document}

\c@section=1

\barcolorbox{220, 255, 220}{0, 130, 0}{80, 200, 80}{
    \leftskip=0pt plus 1fill \rightskip=\leftskip
    {\bigbf Infintesimal Calculus 3}

    \medskip
    \textit{Lecture 1, Sunday October 23, 2022}

    \textit{Ari Feiglin}
}

\bigskip

Just like how the focus of Calculus 1 and 2 were of single value functions:
\[ f\colon X\subseteq\bR\longvarrightarrow\bR \]
The focus of this course will be on functions:
\[ f\colon X\subseteq\bR\longvarrightarrow\bR^m \]

\subsection{A Soft Review of Linear Algebra}

\begin{defn*}

    We will recall that a \ppemph{normed vector space} $V$ is a vector space equipped with a norm function:
        \[ \norm\cdot\colon W\longvarrightarrow\bR \]
    Which satisfies the following axioms:
    \blist
        \item Positivity: $\norm v\geq0$ for every $v\in V$.
        \item Homogeneity: $\norm{\alpha v}=\abs\alpha\cdot\norm v$ for every $\alpha\in\bR$ and $v\in V$.
        \item The Triangle Inequality: $\norm{v+u}\leq\norm{v}+\norm{u}$ for every $v,u\in V$.
    \elist

\end{defn*}

Notice that by the triangle inequality:
\[ \norm{v-u}+\norm u\geq\norm v \quad \norm{v-u}+\norm v\geq\norm u \]
So we have that
\[ \norm{v-u} \geq \abs{\norm v - \norm u} \]

\begin{exam}

    Some examples of norms over $\bR^n$ are:
        \blist
            \item $\norm x_\infty=\max{\abs{x_1},\dots,\abs{x_n}}$
            \item $\norm x_p=\displaystyle\parens{\sum_{k=1}^n \abs{x_k}^p}^{\frac1p}$ for $1\leq p<\infty$.
        \elist
        These are actually special cases of the more general $\ell^p$ norm, which itself can be seen as a special case of the $L^p(\bR)$ norm.

\end{exam}

\begin{exam}

    The set:
        \[ C\parens{[a,b]}\coloneqq\set{f\colon[a,b]\longvarrightarrow\bR}[f\text{ is continuous}] \]
    is also a vector space, and we can norm it by the $\norm\cdot_{\rm max}$ norm:
        \[ \norm{f}_{\rm max}\coloneqq\max_{x\in[a,b]} f(x) \]
    We can also define the $\norm\cdot_p$ norm:
        \[ \norm f_p\coloneqq\parens{\int_a^b \abs{f(x)}^p\,dx}^{\frac1p} \]
    This too is a specific case of the $L^p[a,b]$ norm.

\end{exam}

Note that $C[a,b]$ is infinite-dimensional.
We can show this by showing that the set $\set{x^n}_{n\in\bN}\subseteq C[a,b]$ is linearly independent.
Suppose then that there is a finite sum of $x^n$s which equals $0$:
\[ \sum_{k=1}^n a_k x^{k_n} = 0 \]
Notice that the sum above is a polynomial, we will let it equal $p(x)$, so $p\equiv 0$.
But we know that a polynomial is identically $0$ if and only if its coefficients are all $0$, meaning that $a_k=0$.
So $\set{x^n}_{n\in\bN}$ is indeed linearly independent, so $C[a,b]$ is infinite dimensional.

\begin{defn*}

    Again recall that given a vector space $V$, an \ppemph{inner product} on $V$ is a function:
        \[ \iprod{\cdot,\cdot}\colon V\times V\longvarrightarrow\bR \]
    Such that:
    \blist
        \item $\iprod{v,v}\geq0$ and is $0$ if and only if $v=0$.
        \item $\iprod{\alpha v,u}=\alpha\iprod{v,u}$
        \item $\iprod{v,u}=\overline{\iprod{u,v}}$.
              Since we are working over $\bR$ we can simplify this to $\iprod{v,u}=\iprod{u,v}$.
    \elist
    If $V$ has an inner product, it is called an \ppemph{inner product space}.

\end{defn*}

An inner product also generates a norm, we can define:
\[ \norm{v} = \sqrt{\iprod{v,v}} \]

\begin{exam}

    Over $C[a,b]$ we can define:
        \[ \iprod{f,g} = \int_a^b f(x)\cdot g(x)\,dx \]
    This inner product actually generates the $\norm\cdot_2$ norm.

\end{exam}

\begin{exam}

    The $\ell^p$ space is the space of all infinite sequences $\set{a_n}_{n=1}^\infty$ such that:
        \[ \sum_{n=1}^\infty \abs{a_n}^p < \infty \]
    And we define the $\ell^p$ norm, also denoted $\norm\cdot_p$ to be the $p$th root of this.

    Over $\ell^2$ we can define an inner product:
        \[ \iprod{\set{a_n}_{n=1}^\infty, \set{b_n}_{n=1}^\infty} = \sum_{n=1}^\infty a_n\cdot b_n \]
    It can be seen that this is well-defined and it is trivial to see that this generates the $\ell^2$ norm.
\end{exam}

\begin{defn*}

    Given a normed vector space $V$, we can define the \ppemph{distance} metric to be:
        \[ d(v,u) = \norm{v-u} \]

\end{defn*}

\begin{defn*}

    A set $M$ equipped with a function:
        \[ d(\cdot, \cdot)\colon M\times M\longvarrightarrow\bR_{\geq0} \]
    Which satisfies the following:
    \blist
        \item Positivity: $d(v,u)\geq0$ and is $0$ if and only if $v=u$.
        \item Symmetry: $d(v,u)=d(u,v)$.
        \item The Triangle Inequality: $d(v,u)\leq d(v,w)+d(w,u)$.
    \elist

\end{defn*}

Notice then that a normed vector space is a metric space since:
\[ d(v,w) + d(w,u) = \norm{v-w} + \norm{w-u} \geq \norm{v-u} = d(v,u) \]
And the other two requirements are simple to prove.

\def\bS{\mathbb{S}}

\begin{exam}
    Not every metric space is a vector space.
    We can define the $\bS^n$ space as:
        \[ \bS^n\coloneqq\set{v\in\bR^{n+1}}[\norm v = 1] \subset \bR^{n+1} \]
    We can define a metric on $\bS^n$ to be the length of the smallest arc between two points.
    This is obviously not a vector space, but it \emph{is} a metric space.
\end{exam}

\end{document}

