\documentclass[10pt]{article}

\usepackage{amsmath, amssymb, mathtools}
\usepackage{tikz}
\usepackage[margin=1.5cm]{geometry}

\input mathex
\input prettyprint
\input preamble

\setscalefactor{10}
\initpps

\def\pmat#1{\begin{pmatrix} #1 \end{pmatrix}}

\def\@blist[#1]{%
    \bgroup\bgroup\par\vskip-\medskipamount%
    \gdef\item{%
        \par\egroup\bgroup\medskip\setbox0=\hbox{#1\quad}%
        \advance\leftskip by \wd0\leavevmode\kern-\wd0\box0%
    }%
}
\def\blist{\@ifnextchar[ \@blist {\@blist[$\bullet$]}}
\def\elist{\par\egroup\egroup\medskip}

\let\divides=\mid
\newfunc{metric}\rho({})

\font\bigbf = cmbx12 scaled 2000

\def\openset{{\cal O}}

\begin{document}

\c@section=4

\barcolorbox{220, 255, 220}{0, 130, 0}{80, 200, 80}{
    \leftskip=0pt plus 1fill \rightskip=\leftskip
    {\bigbf Infinitesimal Calculus 3}

    \medskip
    \textit{Lecture 4, Wednsday November 2, 2022}

    \textit{Ari Feiglin}
}

\bigskip

\c@subsection=1

\begin{prop*}

    If $S$ is compact, $S$ is closed.

\end{prop*}

\begin{proof}

    Suppose for the sake of a contradiction that $S$ is not closed.
    Then there is a limit point $x$ which is not in $S$.
    We will take a descending sequence of closed sets $F_n = \bar B_{\frac1{2^n}}(x)$, then let $\openset_n=F_n^c$ which are open.
    Notice then that the intersection of $F_n$ is $\set x$, since if $y$ is in the intersection $\metricof{x,y}\leq\frac1{2^n}$ for all $n$, so $\metricof{x,y}=0$, so $y=x$.
    And therefore the union of $\openset_n$ is $X\setminus\set{x}$, and since $x$ isn't in $S$, $\set{\openset_n}_{n=1}^\infty$ is an open cover of $S$.
    Since $S$ is compact, there is a finite subcover $\set{\openset_{m_n}}_{n=1}^N$ which covers $S$.
    Now if we assume $m_n<m_{n+1}$, then since $\openset_n$ is an increasing sequence:
    \[ S\subseteq\openset_{m_N} \]
    But then that means that $F_{m_n}$ and $S$ are disjoint, so there exists a ball around $x$ which doesn't contain points of $S$.
    So $x$ can't be a limit point, in contradiction. \lightning

    \hfill$\blacksquare$

\end{proof}

\begin{defn*}

    If $X$ is a metric space, $S\subseteq X$ is \ppemph{bounded} if there is an $x$ in $X$ and an $r>0$ such that $S\subseteq B_r(x)$.

\end{defn*}

Notice then that $S$ is bounded if and only if for \emph{every} $x\in X$ there is an $r>0$ such that $S\subseteq B_r(x)$.
If $S$ is bounded, then suppose $S\subseteq B_r(x)$, then let $y\in X$, if we let $r'=r+\metricof{x,y}$ $S\subseteq B_{r'}(y)$.

\begin{prop*}

    If $S$ is compact, then $S$ is bounded.

\end{prop*}

\begin{proof}

    Notice that for any $x\in X$, $\set{B_n(x)}_{n\in\bN}$'s union is $X$ since for any $y\in X$, there must be some $n\in\bN$ such that $\metricof{x,y}<n$.
    Since $S$ is compact, there is a subcovering $\set{B_{m_n}}_{n=1}^N$ which covers $S$.
    But then since these balls form an increasing subsequence:
    \[ S\subseteq B_{m_N}(x) \]
    and thus by definition $S$ is bounded.

    \hfill$\blacksquare$

\end{proof}

\begin{exam}

    Notice that not every bounded set is compact.
    Over a set $X$ we can define the following \ppemph{discrete metric}:
    \[ \metricof{x,y} = \begin{cases} 1 & x\neq y \\ 0 & x=y \end{cases} \]
    This is obviously a metric (and quite an interesting one as well).
    It has the interesting characteristic that every subset of $X$ is open and thus it must also be closed (since its complement is open).
    This is because for $\epsilon\leq1$, $B_\epsilon(x)=\set x$.

    So if we take $X=\bN$ then we can create a covering $\set{\set{x}}_{x\in\bN}$ (since every set is open), but for any finite subcovering, the union contains
    only a finite number of points and thus cannot cover $X$.
    So $X$ is not compact.
    But $X$ is bounded (this is true for every discrete metric space) since for every $x,y\in X$, then $\metricof{x,y}\leq1$, so $X\subseteq B_2(x)$.

\end{exam}

\begin{prop*}

    If $X$ is a metric space and $T\subseteq S\subseteq X$ where $S$ is compact and $T$ is closed, then $T$ is also compact.

\end{prop*}

\begin{proof}

    Suppose $\set{\openset_\lambda}_{\lambda\in\Lambda}$ is an open covering of $T$.
    Then we can add $T^c$ to the cover which is open since $T$ is closed, and that creates an open cover of $X$ and therefore also of $S$.
    So there exists a finite subcovering of this which covers $S$:
    \[ T\subseteq S\subseteq\bigcup_{k=1}^n\openset_k\cup T^c \]
    Then since $T$ and $T^c$ are disjoint, we must have that
    \[ T\subseteq\bigcup_{k=1}^n\openset_k \]
    so $\set{\openset_k})_{k=1}^n$ is a finite subcovering of $T$, so $T$ is compact.

    \hfill$\blacksquare$

\end{proof}

\begin{defn*}

    A \ppemph{closed rectangle} in $\bR^n$ is a set of the form:
    \[ [a_1,b_1]\times\cdots\times[a_n,b_n] \]
    Where $a_k<b_k$.
    And the $k$th \ppemph{vertex} of such a rectangle is $[a_k, b_k]$.

\end{defn*}

\newfunc{diam}{\rm diam}(\vert)

\begin{defn*}

    If $X$ is a metric space and $S\subseteq X$, the \ppemph{diameter} of $S$ is:
    \[ \diam S=\sup_{x,y\in S}\metricof{x,y} \]
    A \ppemph{contracting sequence} of sets $\set{E_n}_{n=1}^\infty$ in $X$ is a descending sequence of sets whose diameter approaches $0$, that is:
    \[ E_{n+1}\subseteq E_n \text{ and } \lim_{n\to\infty} \diamof{E_n} = 0 \]

\end{defn*}

\begin{thrm*}[cantorLemma,Cantor's\ Lemma]

    If $\set{T_k}_{k\in\bN}$ is a contracting sequence of rectangles in $\bR^n$ then there exists an $x\in\bR^n$ such that:
    \[ \bigcap_{k\in\bN} T_k = \set x \]

\end{thrm*}

\begin{proof}

    For every $m$ we can take the $k$th vertex of $T_k$: $[a_k^{(m)}, b_k^{(m)}]$.
    And since $T_k$ is decreasing:
    \[ [a_k^{(1)}, b_k^{(1)}] \supseteq \cdots \supseteq [a_k^{(m)}, b_k^{(m)}] \supseteq \cdots \]
    Then by Cantor's Lemma in $\bR$, the intersection of these intervals is non-empty, so there exists an $x_k$ in the intersection.
    Then $\parens{x_1,\dots,x_n}$ is in the intersection of $T_k$.
    This must be the only point in the intersection since if there is another $y$ in the intersection, since $x,y\in T_k$ for every $k$,
    $\diam T_k\geq\metricof{x,y}$, and then the limit of the diameters wouldn't be $0$.

    \hfill$\blacksquare$

\end{proof}

This theorem can be generalized to any complete metric space (which we will learn about later).
In fact this trait is actually equivalent to completeness.

\begin{thrm*}

    A rectangle $T=[a_1,b_1]\times\cdots\times[a_n,b_n]$ in $\bR^n$ is compact.

\end{thrm*}

\begin{proof}

    Let $\set{\openset_\lambda}_{\lambda\in\Lambda}$ be an open covering of $T$.
    Suppose for the sake of a contradiction that $T$ has no finite open subcovering.
    For every vertex, we note that $[a_k,b_k]=[a_k,c_k]\times[c_k,b_k]$ where $c_k=\frac12\big(a_k+b_k\big)$.
    Then we now have $2^n$ smaller rectangles.
    And one of these smaller rectangles must also not have a finite open subcovering (if they all did, we could take the union of these finite open subcoverings which would
    also be a finite open subcovering of $T$ which is a contradiction).
    Let $T_1$ be this subrectangle which doesn't have a finite open subcovering.
    We can find a subrectangle $T_2$ of $T_1$ which also doesn't have a finite open subcovering, and thus we can recursively define a sequence $\set{T_n}_{n\in\bN}$.
    And since the diameter of $T_{n+1}$ is half that of $T_n$, $\diam T_n\longvarrightarrow0$.
    Since $T_n$ is closed, by Cantor's lemma:
    \[ \bigcap_{n\in\bN} T_n = \set x \]
    for some $x\in\bR^n$.
    Then there must be some $\lambda\in\Lambda$ such that $x\in\openset_\lambda$, and so there must be an $r>0$ such that $x\in B_r(x)\subseteq\openset_\lambda$.
    Since $\diam T_n\longvarrightarrow0$, at some point $\diam T_n<r$, so therefore
    \[ T_n\subseteq B_r(x) \subseteq \openset_\lambda \]
    So $T_n$ \emph{does} have a finite open subcovering, in contradiction. \lightning

    \hfill$\blacksquare$

\end{proof}

\begin{thrm*}[heineBorelTheorem,Heine-Borel\ Theorem]

    $T\subseteq\bR^n$ is compact if and only if $T$ is closed and bounded.

\end{thrm*}

\begin{proof}

    We have already shown that compactness implies closedness and boundedness, so all that remains is to prove the converse.
    Since $T$ is bound, it must be contained inside of a rectangle $U$ (we can take $x\in T$, and then vertices around $x_k$ whose lengths are the diameter of $T$).
    By above, $U$ is compact.
    And since $T\subseteq U$ and $T$ is closed, $T$ is compact.

    \hfill$\blacksquare$

\end{proof}

This result does not hold in general metric spaces.

\begin{exam}

    Recall the definition of $\ell^2$:
    \[ \ell^2=\set{\set{a_n}_{n\in\bN}}[\sum_{n=1}^\infty a_n^2<\infty] \]
    Let $e_k$ be the sequence whose $k$th element is $1$ and the rest are $0$.
    Let $T=\set{e_1,\dots,e_n,\dots}$.
    $T$ is bounded since $\metricof{e_n,e_m}=\sqrt2$ and it is closed since if $x$ is a limit point of $T$, then it must be equal to some $e_k$, because the distance between
    each $e_k$ is constant.
    But $T$ is not compact since if we focus on the cover $\set{B_1(e_k)}_{k=1}^\infty$, each ball contains only one $e_k$ and thus there can't be a finite subcover since $T$
    is infinite.

\end{exam}

\end{document}

