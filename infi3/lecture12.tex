\documentclass[10pt]{article}

\usepackage{amsmath, amssymb, mathtools}
\usepackage{tikz}
\usepackage[margin=1.5cm]{geometry}

\input mathex
\input prettyprint
\input preamble

\setscalefactor{10}
\initpps

\def\pmat#1{\begin{pmatrix} #1 \end{pmatrix}}

\let\divides=\mid
\newfunc{metric}\rho({})
\newfunc{metricc}\sigma({})

\font\bigbf = cmbx12 scaled 2000

\def\openset{{\cal O}}
\let\lineseg=\overleftrightvecc
\let\ds=\displaystyle

\def\pdv#1#2{\frac{\partial #1}{\partial #2}}

\begin{document}

\c@section=12

\barcolorbox{220, 255, 220}{0, 130, 0}{80, 200, 80}{
    \leftskip=0pt plus 1fill \rightskip=\leftskip
    {\bigbf Infinitesimal Calculus 3}

    \medskip
    \textit{Lecture \thesection, Wednsday November 23, 2022}

    \textit{Ari Feiglin}
}

\bigskip

We will use $\epsilon\colon X\longvarrightarrow X$ where $X$ is some linear space as a function such that:
\[ \lim_{x\varrightarrow0}\frac{\epsilon(x)}{\norm x} = 0 \]
Then $\lim_{x\varrightarrow0}\epsilon(x)=0$.
So if $X=\bR^n$ then $\epsilon$ must satisfy:
\[ \lim_{(x_1,\dots,x_n)\varrightarrow0}\frac{\epsilon(x_1,\dots,x_n)}{\sqrt{x_1^2+\cdots+x_n^2}} = 0 \]

Notice then that $f$ is continuous if and only if $f(x)=f(x_0)+\frac{\epsilon(\Delta x)}{\Delta x}$ where $\Delta x=x-x_0$.
This is direct since the limit of $\frac{\epsilon(\Delta x)}{\Delta x}$ is $f(x)=f(x_0+\delta x)-f(x_0)$ which is $0$ if and only if $f$ is continuous at $x_0$.

\begin{prop*}

    Suppose $f\colon\bR\longvarrightarrow\bR$ defined in some neighborhood of $x_0$, then $f$ is differentiable at $x_0$ if and only if there is some $A\in\bR$ and $\epsilon(\Delta x)$ such that
    \[ f(x) = f(x_0) + A\Delta x + \epsilon(\Delta x) \]
    where $\Delta x = x-x_0$.

\end{prop*}

\begin{proof}

    If $f$ is differentiable then define $\epsilon(\Delta)=f(x)-f(x_0)-A\Delta x$:
    \[ \lim_{\Delta x\varrightarrow0}\frac{\epsilon(\Delta)}{\Delta x} = \lim_{\Delta x\varrightarrow0}\frac{f(x_0+\Delta x)-f(x_0) - A\Delta x}{\Delta x} = f'(x_0) - A \]
    so if $A=f'(x_0)$, then $\epsilon$ and $A$ satisfy the condition.

    Suppose there exists some $A$ and $\epsilon$ where this occurs, then:
    \[ \lim_{\Delta x\varrightarrow0}\frac{f(x_0+x)-f(x_0)}{\Delta x} = \lim_{\Delta x\varrightarrow0}\frac{A\Delta x+\epsilon(\Delta x)}{\Delta x} =
    A + \lim_{\Delta x\varrightarrow}\frac{\epsilon(\Delta x)}{\Delta x} = A \]
    As required.

    \hfill$\blacksquare$

\end{proof}

\begin{defn*}

    A function $f\colon\bR^2\longvarrightarrow\bR$ is differentiable at $(x_0,y_0)$ if it is defined in some neighborhood of it and there exist $A,B\in\bR$ and $\epsilon(\Delta x,\Delta y)$ such that
    \[ f(x,y) = f(x_0,y_0) + A\Delta x + B\Delta y + \epsilon(\Delta x,\Delta y) \]

\end{defn*}

\begin{prop*}

    If $f(x,y)$ is defined in some neighborhood of $(x_0,y_0)$ and differentiable at $(x_0,y_0)$ then
    \benum
        \item $f$ is continuous at $(x_0,y_0)$
        \item $\partial_x f$ and $\partial_y f$ exist at $(x_0,y_0)$ and are equal to $A$ and $B$
    \eenum

\end{prop*}

\begin{proof}

    \benum
        \item We know that:
        \[ \lim_{(x,y)\varrightarrow(x_0,y_0)}f(x,y) = f(x_0,y_0) + \lim_{(\Delta x,\Delta y)\varrightarrow0}\epsilon(\Delta x,\Delta y) = f(x_0,y_0) \]
        $\Delta x$ and $\Delta y$ approach $0$ since convergence in $\bR^n$ is pointwise.

        \item We know
        \[ \partial_x f(x_0, y_0) = \lim_{\Delta x\varrightarrow0}\frac{f(x_0+\Delta x,y_0)-f(x_0,y_0)}{\Delta x} = A + \lim_{\Delta x\varrightarrow0}\frac{B\Delta y+\epsilon(\Delta x,\Delta y)}{\Delta x} \]
        In this case $\Delta y=0$ since $y=y_0$ so this is equal to:
        \[ A + \lim_{\Delta x\varrightarrow0}\frac{\epsilon(\Delta x,0)}{\Delta x} = A \]
        And similarly for $\partial_y f$.
    \eenum

    \hfill$\blacksquare$

\end{proof}

\begin{prop*}

    Suppose $f(x,y)$ is a function which has partial derivatives in a neighborhood of $(x_0,y_0)$ and $\partial_x f$ and $\partial_y f$ are continuous at $(x_0,y_0)$.
    Then $f$ is differentiable at $(x_0,y_0)$.

\end{prop*}

\begin{proof}

    Notice that
    \[ f(x_0+\Delta x, y_0+\Delta y) - f(x_0,y_0) = \parens{f(x_0+\Delta x, y_0+\Delta y)-f(x_0,y_0+\Delta y)} + \parens{f(x_0,y_0+\Delta y) - f(x_0,y_0)} \]
    So by the mean value theorem on the ``single value representations'' of the partial derivatives, there exists $0\leq t,s\leq1$ such that
    \[ f(x_0+\Delta x,y_0+\Delta y) - f(x_0,y_0) = \partial f_x(x_0+t\Delta x,y_0+\Delta y)\Delta x + \partial f_y(x_0+\Delta x,y_0+s\Delta y)\Delta y \]
    Since $\partial_x f$ and $\partial_y f$ are continuous at $(x_0,y_0)$ this is equal to:
    \[ = \parens{\partial_x f(x_0,y_0) + \epsilon_1(t\Delta x,\Delta y)}\Delta x + \parens{\partial_y f(x_0,y_0) + \epsilon_2(\Delta x,s\Delta y)} \]
    And so if we define $A=\partial_x f(x_0,y_0)$ and $B=\partial_y f(x_0,y_0)$ and $\epsilon(\Delta x,\Delta y)=\epsilon_1\Delta x+\epsilon_2\Delta y$ we have:
    \[ f(x,y) - f(x_0,y_0) = A\Delta x + B\Delta x + \epsilon(\Delta x,\Delta y) \]
    And we know that:
    \[ \lim\frac{\epsilon(\Delta x,\Delta y)}{\sqrt{\Delta x^2+\Delta y^2}} = 0 \]
    Since $\frac{\Delta x}{\sqrt{\Delta x^2+\Delta y^2}}$ is bounded by $1$, and similiar for $\Delta y$ and the limit of this for $\epsilon_1$ and $\epsilon_2$ is $0$ (by pointwise convergence), so this
    is indeed $0$ as required.

    \hfill$\blacksquare$

\end{proof}

This is not an equivalent condition though.

\begin{exam}

    Let
    \[ f(x,y) = \begin{cases} (x^2+y^2)\sinof{\frac1{\sqrt{x^2+y^2}}} & (x,y)\neq0 \\ 0 & (x,y)=0 \end{cases} \]
    Then $f(x,0) = x^2\sinof{\frac1{x}}$ the derivative of this is $\partial_x f(x,0)=2x\sinof{\frac1x}-\cosof{\frac1x}$, which does not exist for $x=0$, but it is continuous.
    Similarly for $\partial_y f$.
    But since the limit of $f$ as $(x,y)$ approaches $0$ is $0$, $f$ is a valid $\epsilon$, so it is necessarily differentiable.

\end{exam}

\end{document}

