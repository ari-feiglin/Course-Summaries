\subsection{Linear Programs}

Suppose we have a pizzeria which sells different types of pizzas, each with an assortment of toppings.
This pizzeria has one-thousand units of dough, $500$ units of sauce, and $800$ units of cheese.
The pizzeria sells the following types of pizzas, which each require the following units of toppings and have the following prices:

$$ \vbox{\tabskip=.25cm\halign{\hfil\bf#&&\hfil#\hfil\cr
&\bf Regular Pizza&\bf Thin Pizza&\bf Extra-Cheese Pizza\cr\noalign{\kern3pt\hrule\kern3pt}
Dough&2&1&2\cr
Sauce&2&3&2\cr
Cheese&1&2&3\cr\noalign{\kern3pt\hrule\kern3pt}
&\$25&\$30&\$40\cr}} $$

So if we sell $x_1$ regular pizzas, $x_2$ thin pizzas, and $x_3$ extra-cheese pizzas, we must have the following constraints:
$$ 2x_1 + x_2 + 2x_3 \leq 1000,\qquad 2x_1 + 3x_2 + 2x_3 \leq 500,\qquad x_1 + 2x_2 + 3x_3 \leq 800 $$
And we want to maximize our total sales, i.e. maximize $25x_1+30x_2+40x_3$ where $x_1,x_2,x_3\geq0$.

This is an example of {\it linear programming} (LP).

\bdefn

    A {\emphcolor Linear Program} is the problem of maximizing some objective expression of the form
    $$ f(x_1,\dots,x_n\;;\;c_1,\dots,c_n)=\sum_{i=1}^nc_ix_i $$
    for some parameters $c_i$ under the constraints
    $$ \sum_{j=1}^n\alpha_{i,j}x_i\leq\beta_i $$
    for $1\leq i\leq m$ where $m$ is the number of constraints.

\edefn

Notice that if we want to have the constraint $\sum_{j=1}^n\alpha_{i,j}x_i=\beta_i$, this is just the same as having two constraints $\sum_{j=1}^n\alpha_{i,j}x_i\leq\beta_i$ and
$\sum_{j=1}^n\alpha_{i,j}x_i\geq\beta_i$.
And if we want the constraint $\sum_{j=1}^n\alpha_{i,j}x_i\geq\beta_i$, this is equivalent to $\sum_{j=1}^n(-\alpha_{i,j}x_i)\leq-\beta_i$.

Now say we want to {\it minimize} $f(x_1,\dots,x_n\;;\;c_1,\dots,c_n)$, this is just the same as maximizing the expression $f(x_1,\dots,x_n\;;\;-c_1,\dots,-c_n)$.

\bdefn

    An LP is in {\emphcolor standard form} if it has the constraints $x_i\geq0$ for every variable $x_i$.

\edefn

This seems to weaken the strength of LP, but in fact every general LP can be converted to an equivalent LP in standard form.
Suppose we have the LP $f(x_1,\dots,x_n\;;\;c_1,\dots,c_n)$ with constraints using $\alpha_{11},\dots,\alpha_{n,m}$ and $\beta_1,\dots,\beta_m$.
Then we can define
$$ f'(x_1^+,x_1^-,\dots,x_n^+,x_n^-) = \sum_{i=1}^n c_i(x_i^+-x_i^-) $$
and the constraints
$$ \sum_{j=1}^n\alpha_{i,j}(x_i^+-x_i^-) \leq \beta_i \qquad (1\leq i\leq m) $$
along with $x_i^+,x_i^-\geq0$.
This is equivalent, as we can then just define $x_i=x_i^+-x_i^-$, we have split $x_i$ into is positive and negative parts.

Thus given an LP in standard form, we can write it as a problem of the form
$$ \vbox{\tabskip=.25cm\openup1\jot\halign{\hfil#&\hfil#\hfil\cr
Maximize&$\vec c^\top\cdot\vec x$\cr
Such that&$A\vec x\leq\vec b$\cr
&$\vec x\geq\vec 0$\cr}} $$
where $\vec c$ is the vector of coefficients $(c_1,\dots,c_n)^\top$, $\vec x$ is the vector of variables $(x_1,\dots,x_n)^\top$, and $A$ is the matrix such that $A_{ij}=\alpha_{ij}$.
We write $\vec a\leq\vec b$ to mean $a_i\leq b_i$ for every $i$.

{\bf Example:} Suppose we have the following graph (the edges are labelled), and we'd like to find the maximum number of matches we can find.

\centerline{
\def\diagrowbuf{.5cm}
\def\diagcolbuf{1cm}
\drawdiagram{
    A&1\cr
    B&2\cr
    C&3\cr
     &4\cr
}{
    \diagarrow{from={1,1}, to={1,2}, text=1, right cap=-, y distance=.25cm}
    \diagarrow{from={1,1}, to={3,2}, text=2, right cap=-, y distance=.5cm, slide=.3}
    \diagarrow{from={1,1}, to={4,2}, text=3, right cap=-, x distance=-.1cm, slide=.1}
    \diagarrow{from={2,1}, to={1,2}, text=4, right cap=-, y distance=.15cm}
    \diagarrow{from={2,1}, to={2,2}, text=5, right cap=-, slide=.2, y distance=-.15cm}
    \diagarrow{from={3,1}, to={2,2}, text=6, right cap=-, x distance=-.5cm, y distance=-.1cm}
    \diagarrow{from={3,1}, to={3,2}, text=7, right cap=-, y distance=.15cm}
    \diagarrow{from={3,1}, to={4,2}, text=8, right cap=-, y distance=-.15cm}
}}

The LP form of this would be maximizing $\sum_{i=1}^8x_i$ (where $x_i$ denotes whether or not the $i$th edge was chosen), with the constraints ($x_i\geq0$ is implicit)
$$ \eqalign{
    x_1+x_2+x_3\leq1\cr
    x_4+x_5\leq1\cr
    x_6+x_7+x_8\leq1\cr
    x_1+x_4\leq1\cr
    x_5+x_6\leq1\cr
    x_2+x_7\leq1\cr
    x_3+x_8\leq1\cr
} $$
But we also want the constraint that $x_i$ be an integer.
This is called {\it Integer Linear Programming}.
This is an NP-hard problem, as we will see in the following example.
\qqed

{\bf Example:} Now let us look at another example: the problem of {\it Vertex Covers} (VC).
Given a graph $G=(V,E)$, we want to find the smallest vertex cover, which is a set of vertices $S$ such that every $v\in V$ has a neighbor in $S$.
Now, we can use LP as follows: associate each vertex $v\in V$ with a variable $x_v$.
Then we want to minimize $\sum_{v\in V}x_v$.
And we add the constraints that for every $(u,v)\in E$, $x_u+x_v\geq1$ (i.e. either $u$ or $v$ is in $S$).
We can solve this using integer linear programming, and since this is an NP-complete problem (whether $G$ has a VC of a particular size), so too is integer linear programming.

Using LP won't necessarily give us a valid solution to the problem since we can have that $x_v$ be a non-integer.
We only know that the answer LP gives us is at most ${\rm VC}_{\it opt}$ (the optimal solution), since the optimal solution satisfies the constraints.
In fact, if we define ${\rm Sol}=\set{v}[x_v\geq\frac12]$ then we have that
$$ {\rm LP} \leq {\rm VC}_{\it opt} \leq \abs{\rm Sol} \leq 2{\rm LP} $$
This is because ${\rm Sol}$ must be a vertex cover: if there exists a $v\in V$ with no neighbors in ${\rm Sol}$, this means that $x_v<\frac12$ and $x_u<\frac12$ for every $(v,u)\in E$.
But then we can just set $x_v=\frac12$ and this will satisfy the constraints.
And if we multiply LP by $2$, then every vector in Sol gets multiplied by $2$, and so summing them gives ${\rm Sol}$.
\qqed

\bdefn

    The {\emphcolor slack form} of an LP is one of the form:
    $$ \vbox{\tabskip=.25cm\openup1\jot\halign{\hfil#&\hfil#\hfil\cr
    Maximize&$\vec c^\top\cdot\vec x$\cr
    Such that&$A\vec x=\vec b$\cr
    &$\vec x\geq\vec 0$\cr}} $$

\edefn

Every LP in general form can be brought to one in slack form by adding variables.
For every constraint $\sum_{j=1}^n\alpha_{i,j}x_j\leq\beta_i$ we add a basic variable $s_i$ and alter the constraint to be $\sum_{j=1}^n\alpha_{i,j}x_j+s_i=\beta_i$.
Since we must also add the constraint $s_i\geq0$, this is equivalent.
The coefficients of the basic variables in the expression we optimize is of course zero.

We will also assume that in slack form, $\vec b\geq\vec 0$ as otherwise there may be no solution.

\subsection{The Simplex Algorithm}

First we begin by defining a few mathematical concepts.

\bdefn

    A {\emphcolor convex set} $S\subseteq{\bb R}^n$ is a set such that for every $x,y\in S$, $\lambda x+(1-\lambda)y\in S$ for all $\lambda\in[0,1]$.
    The {\emphcolor convex hull} of a set of points $\set{p_1,\dots,p_m}$ is the smallest convex set containing these points.
    It can be shown that the convex hull is
    $$ [p_1,\dots,p_n] = \set{\sum_{i=1}^m\lambda_ip_i}[\sum_{i=1}^m\lambda_i=1] $$
    An {\emphcolor extreme point} of a convex set is a point $x\in S$ such that there exists no $a,b\in S$ such that $x$ lies on the open segment between $a$ and $b$.
    I.e. there exist no $a,b\in S$, $\lambda\in(0,1)$ such that $x=\lambda a+(1-\lambda)b$.
    Two extreme points $x,y$ are {\emphcolor adjacent} if the line connecting them is not intersected by any other line between two points not on the line between $x$ and $y$.

\edefn

Notice that if $[p_1,\dots,p_n]$ is a convex hull, then every extreme point is one of the $p_i$s.

\bdefn

    A {\emphcolor convex polytope} is the region of some $n$-dimensional space defined by $A\vec x\leq\vec b$.
    The extreme points of a convex polytope are called {\emphcolor vertices} and there are finitely many of them (in fact a convex polytope is the convex hull of its vertices).

\edefn

Notice that the space of {\it feasible solutions} to an LP problem under the constraint $A\vec x\leq\vec b$ and $\vec x\geq0$ forms a convex polytope (it is obviously a polytope, convexity is easy to
verify).
We will use the following results for constructing an algorithm to find the maximum of some objective $\vec c^\top\vec x$.

\blemm

    Let $P$ be a convex polytope with vertices $v_1,\dots,v_n$ and $f$ be a convex function over $P$.
    Then $f$ attains its maximum at one of the vertices.

\elemm

\Proof suppose $x$ is the maximum of $f$, then since it is in $P$, it can be written as $x=\sum_{i=1}^n\lambda_iv_i$ for $\sum_{i=1}^n\lambda_i=1$ and $\lambda_i\geq0$.
So by convexity
$$ f(x) = f\parens{\sum_{i=1}^n\lambda_iv_i} \leq \sum_{i=1}^n\lambda_if(v_i) \leq \sum_{i=1}^n\lambda_if(x) = f(x) $$
Thus $f(x)=\sum_{i=1}^n\lambda_if(v_i)$, but $f(v_i)\leq f(x)$ for all $i$ so $f(v_i)=f(x)$ for all $i$ with $\lambda_i\neq0$, meaning $v_i$ must be a maximum.
\qed

So to find the maximum of the objective, it is sufficient to find its maximum over the vertices of the polyope defined by $A\vec x\leq\vec b$.
But there are many vertices, so how can we effectively find the maximum?

\blemm[name=localmax]

    Let $P$ be a convex polytope and a linear function $f(\vec x)=\vec c^\top\vec x$, then if a vertex is larger than all its adjacent vertices, it is a maximum.

\elemm

We will prove this later.
But this tells us that if we start from a vertex and traverse to adjacent vertices such that each one is larger than the next, we will eventually reach a vertex greater than all its neighbors and this must
be a maximum.

Notice that when we consider $A\vec x=\vec b$, we can assume that $A$ has full row rank, since otherwise there is no solution or $A$ has extra equations which we can remove.

\bdefn

    A {\emphcolor basis} of the polytope defined by $A\vec x\leq\vec b$ is a set of indexes $B$ such that when we take the matrix whose columns are the columns of $A$ whose index are in $B$, denoted $A_B$,
    $A_B$ is invertible.

\edefn

If we have the boundary of a convex polytope $A\vec x=\vec b$ where $A\in{\bb R}^{m\times n}$, then the vertices are the vectors $\vec x$ where $x_i=0$ for $i\notin B$ where $B$ is some basis.
Call such vectors {\it basic feasible solutions} (bfs), we will show later why these are precisely the vertices of the polytope.
For each basis $B$, such a point is unique, since they must have $A_B\vec x=\vec b$ and so $\vec x=A_B^{-1}\vec x$.

These are indeed vertices: we prove this by showing that they are extreme points.
Suppose that $\vec\alpha,\vec\beta$ are two points in the polytope such that $\lambda\vec\alpha+(1-\lambda)\vec\beta=\vec x$ for $\lambda\in(0,1)$.
Then on one hand
$$ A(\lambda\vec\alpha+(1-\lambda)\vec\beta) = \lambda A\vec\alpha + (1-\lambda)A\vec\beta \leq\lambda\vec b+(1-\lambda)\vec b=\vec b $$
but on the other hand this must be an equality since $\lambda\vec\alpha+(1-\lambda)\vec\beta=\vec x$.
So $A\vec\alpha=A\vec\beta=\vec b$.
For every $i$ for which $x_i=0$ we must have $\alpha_i=\beta_i=0$ since $\lambda\alpha_i+(1-\lambda)\beta_i=0$ and $\alpha_i,\beta_i\geq0$.
But we said that such points are unique, so $\vec\alpha=\vec\beta=\vec x$, so $\vec x$ is an extreme point.

We also make the following observation: maximizing $c^\top\vec x$ by $A\vec x=\vec b$ is the same as finding a maximal $v$ such that there exists a $\vec x$ where
$$ \pmatrix{A & 0 & b\cr -c^\top & 1 & 0}\pmatrix{\vec x\cr v\cr-1} = 0 $$
The left matrix is called the {\it tableau} of the LP.

Now suppose $B$ is a basis for $A$, then if we multiply by $A_B^{-1}\oplus1$ (which is legal since it is invertible) we get that we can solve for the tableau
$$ \pmatrix{I \mid_B A' & 0 & A_B^{-1}b\cr -c^\top & 1 & 0} $$
Where $I\mid_B A'$ means that the columns of matrix $I$ are inserted at indexes in $B$ to the matrix $A'$.
Let us assume that $I\mid_B A'=(I\ A')$, then we have a tableau of the form
$$ \pmatrix{I & A' & 0 & \overline b\cr -c_1^\top & -c_2^\top & 1 & 0} $$
Performing some row-reduction gives us
$$ \pmatrix{I & A' & 0 & \overline b\cr 0 & -\overline c_2^\top & 1 & z_B} $$
Recall that the vertex associated with the basis $B$ has zeroes when $i\notin B$, so multiplying by $(\vec x\ v\ -1)^\top$ gives us
$$ \pmatrix{I & A' & 0 & \overline b\cr 0 & -\overline c_2^\top & 1 & z_B}\pmatrix{\vec x\cr v\cr -1} = \pmatrix{\vec x_B-\overline b\cr v-z_B} $$
So we have that $\vec x_B=\overline b$ gives us the value $v=z_B$ ($\vec x_B$ is the vector $\vec x$ with indexes not in $B$ removed).
What this tells us that if we evaluate $x_i=\overline b_i$ for $i\in B$ and $x_i=0$ for $i\notin B$ we get a vertex of the polytope, and its value is $z_B$.

Using the second lemma, that a vertex is a maximum if and only if its value is greater than all its neighbors, we can continue this algorithm.
How do we determine if two vertices are adjacent?
We will answer this through the following lemmas.

\blemm

    Let $B_1,B_2$ be two bases which differ in a single index, and $x,y$ be the bfs-s corresponding to the bases.
    Then for every point $a$ in the boundary of the polytope (meaning $Aa=b$), if $a_i=0$ for every $i\notin B_1\cup B_2$, $a$ lies on the line between $x$ and $y$.

\elemm

\Proof we will show that the line passing through $a$ and $x$ contains $y$.
That is, there exists a $\lambda$ such that $\lambda a+(1-\lambda)x=y$.
Suppose that $B_1\setminus B_2=\set j$ and $B_2\setminus B_1=\set k$.
Then it is sufficient that $\lambda a_i+(1-\lambda)x_i=0$ for $i\notin B_2$, since $A(\lambda a+(1-\lambda)x)=b$ and such a point is uniquely $y$.
Therefore it is sufficient that $\lambda a_j+(1-\lambda)x_j=0$, i.e. $\lambda(a_j-x_j)=-a_j$.
This has a solution when $x_j\neq a_j$.

Thus the only issue arises when $x_j=a_j$.
Let us denote $A_i=A_{B_i}$, then
$$ Aa = b \implies A_1^{-1}Aa = x \implies (I\mid_B A_1^{-1}A_{B_1^c})a = x $$
Since $a_i=0$ for $i\notin B_1\cup B_2$, we have that
$$ a_{B_1} + a_k A_1^{-1}C_k(A) = a_{B_1} + a_k C_k(A_1^{-1}A) = x $$
Thus $x_j=a_j+a_k[A_1^{-1}A]_{jk}$, so $a_k[A_1^{-1}A]_{jk}=0$.
If $a_k=0$ then $a=x$ by uniqueness.
Otherwise, $[A_1^{-1}A]_{jk}=0$ \localcolor{red}{why can't this happen?}
\qed

Generalizing this lemma, we get the converse of what we stated earlier: a point is a vertex iff it is a bfs.
This is since we can generalize this lemma to say that if $B_1,\dots,B_k$ differ each by a point and $Aa=b$ such that $a_i=0$ for $i\notin\bigcup_j B_j$, then $a$ lies on the convex hull defined by
$x_1,\dots,x_k$ where $x_i$ is the bfs defined by by $B_i$.
So the polytope is the convex hull of all the bfs-s, and thus each extreme point must be a bfs.

\blemm

    Two bfs-s are adjacent iff their bases differ by a single point.

\elemm

\Proof suppose that $B_1,B_2$ are bases which differ by a single index.
Let $x,y$ be the corresponding bfs-s, and suppose that $\alpha,\beta$ are two points in the polytope such that the open segment between them intersects the line segment between $x$ and $y$.
Then there exists $\gamma\in(0,1),\lambda\in[0,1]$ such that
$$ \lambda x + (1-\lambda)y = \gamma\alpha + (1-\gamma)\beta $$
this means that for $i\notin B_1\cup B_2$, $\gamma\alpha_i+(1-\gamma)\beta_i=0$, since $\alpha,\beta\geq0$ this means $\alpha_i=\beta_i=0$.
But then by the above lemma, they are both on the line segment of $x$ and $y$, so $x$ and $y$ are adjacent.

\localcolor{red}{Show converse}
\qed

{\bf Proof }(of \refmath[lemm]{localmax}):

So now let us discuss the significance of these lemmas on our algorithm.
A current naive algorithm would be to iterate over all the bases $B$, and multiply the tableau by $A_B^{-1}$.
Notice that this is just equivalent to row-reducing the tableau at the columns indicated by $B$.
Recall that if we have a basis $B$, our tableau will be in the following form:
$$ \pmatrix{I\mid_BA' & 0 & \bar b\cr 0 \mid_B-\bar c^\top & 1 & z_B} $$
and the value at the vertex $x_B$ (the bfs defined by $B$) is $z_B$.
Choosing a basis $B'$ which differs from $B$ by a single index and then looking at the resulting tableau is the same as choosing a column from $A'$ and using an element from that column as a pivot.
The column we choose is what will determine the {\it entering index} (or entering variable), and the row we choose will determine which column from $B$ will leave (the {\it leaving index} or variable).

Suppose we choose the index $i$ to enter and the index $j$ to leave, i.e. the column $i$ and the row $j$.
Then to row reduce the final row, we must add $\bar c_j/A'_{ji}$, resulting in a change of $z_B$ to $z_{B'}=z_B+\frac{\bar c_j}{A'_{ji}}\bar b_j$.
So we want to find the $i,j$ which maximize $\frac{\bar c_j\cdot\bar b_j}{A'_{ji}}$.
Notice that if all of $c$ is negative, then this means that our current bfs is larger than all its neighbors and so it is maximal by an above lemma.

But we also need our new solution to be feasible, that is we cannot have that any $b_k$ turns negative.
The new value of $b_k$ is, when $i,j$ are chosen, $b_k-\frac{b_j}{A_{ji}}A_{ki}$.
So we must have that
$$ \frac{b_k}{A_{ki}} \geq \frac{b_j}{A_{ji}} $$
So after choosing $j$, we must choose $i$ to minimize $b_j/A_{ji}$.

So what we do is we choose the column $j$ such that $\bar c_j$ is maximal (the most negative value in the bottom row) (this is to find the optimal neighbor), and then choose the row $i$ such that
$b_j/A_{ji}$ is minimal (to ensure feasibility).

This is the simplex algorithm.
It is correct since at the worst case we visit every vertex, but that's still a finite number of vertices.
Why is it called the {\it simplex} algorithm?
Who knows, in my opinion a more fitting name would be ``the convex polytope algorithm'', but that's a bit of a mouthful.

\subsection{The Simplex Algorithm; In Short}

Suppose we want to solve the following LP in slack form: maximize $c^\top x$ constrained to $Ax=b$ where $A\in{\bb R}^{m\times n}$.
Then we write out the {\it initial tableau}:
$$ \pmatrix{A & 0 & b\cr -c^\top & 1 & 0} $$
If $A$ contains the $m\times m$ identity matrix, then say that the tableau is in {\it canonical form}.
Changing names of variables (i.e. column swapping), this means that the tableau is of the form
$$ \pmatrix{I & A & 0 & b\cr -d^\top & -c^\top & 1 & 0} $$
Row reducing gives us
$$ \pmatrix{I & A & 0 & b\cr 0 & -c^\top & 1 & z_B} $$
$z_B$ is a feasible solution to the problem when we set the variables in the columns corresponding to the identity matrix to the values of $b$ and all the other variables to zero.

If the tableau is not in canonical form, we can find $m$ independent columns and row reduce those to get the identity.

If $c$ has no positive values, then we finish and $z_B$ is the maximum value.
Otherwise, choose the column $i$ for which $c_i$ is maximal.
Then choose the row $j$ for which $\frac{b_j}{A_{ji}}$ is minimal, and use the index $(j,i)$ as a pivot in row reducing.
This will give (after swapping columns)
$$ \pmatrix{I & \bar A & 0 & \bar b\cr 0 & -\bar c^\top & 1 & z_{B'}} $$
where $z_{B'}>z_B$, and this is another feasible solution.
Continue iteratively choosing a pivot until $c$ has no positive values.

{\bf Example:} suppose we want to maximize
$$ 2x + 3y + 4z $$
subject to
$$ \eqalign{
    3x + 2y +  z &\leq 10\cr
    2x + 5y + 3z &\leq 15\cr
    x,y,z &\geq0
} $$
Adding slack variables, this is equivalent to
$$ \eqalign{
    3x + 2y +  z + s_1 &= 10\cr
    2x + 5y + 3z + s_2 &= 15\cr
    x,y,z,s_1,s_2 &\geq0
} $$
So our initial tableau is
$$ \pmatrix{
    3 & 2 & 1 & 1 & 0 & 0 & 10\cr
    2 & 5 & 3 & 0 & 1 & 0 & 15\cr
    -2 & -3 & -4 & 0 & 0 & 1 & 0
} $$
We now perform a pivot operation:
\benum
    \item we first choose the most negative value in the bottom row: $-3$, this will be our column.
    \item we then compute $b_j/A_{ji}$ and take the row with the minimum value.
    The values are $10/3$ and $15/2$, so we choose $2$ as the pivot.
\eenum
Row reducing will give
$$ \pmatrix{
    0 & -5.5 & -3.5 & 1 & -1.5 & 0 & -12.5\cr
    1 & 2.5 & 1.5 & 0 & 0.5 & 0 & 7.5\cr
    0 & 2 & -1 & 0 & 1 & 1 & 15
} $$
The next pivot column will be the third column, and computing the quotients gives $-12.5/-3.5=25/7$ and $7.5/1.5=5$ so we choose the second row as the pivot column.
The resulting tableau is
$$ \pmatrix{
    0 & -5.5 & -3.5 & 1 & -1.5 & 0 & -12.5\cr
    2/3 & 5/3 & 1 & 0 & 1/3 & 0 & 5\cr
    2/3 & 11/3 & 0 & 0 & 4/3 & 1 & 20
} $$
The bottom row is all positive, so we can no longer pivot.
Thus the maximum value is $20$.
\qqed

\subsection{The Simplex Algorithm; Phase I}

So far what we discussed relies on us having an initial bfs to our problem, otherwise our initial tableau is not in canonical form (and then we have to find a basis for which the bfs is non-negative).
Instead of iterating over each basis, we can instead solve an LP to find an initial bfs.

Suppose we want to maximize $c^\top x$ constrained to $Ax=b$.
Now instead let us minimize $1^\top s$ ($1$ is the vector containing all ones) constrained to $(A\mid I)\binom{x}{s}=b$ as well as $-c^\top x+w=0$ for some other variable $w$ (this is in order to keep track
of the changes to $c^\top x$ as we row reduce).
This is equivalent to $Ax+s=b$, i.e. $Ax\leq b$, and the idea is that if we can make $s=0$ then the resulting bfs will be a vertex in the polytope $Ax=b$.
So we minimize $1^\top s$ (equivalently, maximize $-1^\top s$) and if we get $s=0$ then we have an initial bfs.
Otherwise, we can't minimize $s$ to $0$ and so we have no solution to $Ax=b$, meaning our polytope is empty.

This procedure is the first step in the simplex algorithm, also called {\it Phase I}.
The second step (row reducing the tableau) is called {\it Phase II}.

{\bf Example:} suppose we want to maximize
$$ 2x + 3y + 4z $$
subject to
$$ \eqalign{
    3x + 2y +  z &=10\cr
    2x + 5y + 3z &=15\cr
    x,y,z &\geq0
} $$
So our initial tableau is
$$ \pmatrix{
    3 & 2 & 1 & 0 & 10\cr
    2 & 5 & 3 & 0 & 15\cr
    -2 & -3 & -4 & 1 & 0
} $$
which is not in canonical form.
So now according to phase I, we want to maximize $-u-v$ subject to
$$ \eqalign{
    3x + 2y +  z + u &=10\cr
    2x + 5y + 3z + v &=15\cr
    -2x - 3y - 4z + w &= 0\cr
    x,y,z,u,v &\geq0
} $$
The initial tableau for this is
$$ \pmatrix{
    3 & 2 & 1 & 0 & 1 & 0 & 0 & 10\cr
    2 & 5 & 3 & 0 & 0 & 1 & 0 & 15\cr
    -2 & -3 & -4 & 1 & 0 & 0 & 0 & 0\cr
    0 & 0 & 0 & 0 & 1 & 1 & 1 & 0\cr
} $$
Converting to its canonical form, we get
$$ \pmatrix{
    3 & 2 & 1 & 0 & 1 & 0 & 0 & 10\cr
    2 & 5 & 3 & 0 & 0 & 1 & 0 & 15\cr
    -2 & -3 & -4 & 1 & 0 & 0 & 0 & 0\cr
    -5 & -7 & -4 & 0 & 0 & 0 & 1 & -25
} $$
We could take the second column as the pivot column, but let's choose column $3$ instead.
We then perform the simplex algorithm on this to give an initial tableau.
\qqed

If some of the constraints have slack variables, we can optimize this so that we only add artificial variables to constraints without them.
This way we still start with a canonical form, and we have fewer artififical variables to deal with.

Now suppose we have the linear program of maximizing $c^\top x$ subject to $Ax=b$ where $b$ is not non-negative.
Let us assume that $A$ is in canonical form, then simply converting $R_i(A)x=b_i$ to $-R_i(A)x=-b_i$ will not work, as it will affect the canonicalness of $A$.
For every constraint where $b_i$ is negative, we substract some artificial variable $a_0$ from the constraint, so it becomes $\sum_j a_{ij}x_i - a_0 = b_i$, and we want to minimize $a_0$, i.e. maximize
$-a_0$.
We then add $a_0$ to the initial basis, we do this to make $\bar b$ nonnegative.
This is because if we pivot on the $i$th row, then this means that $b_i/A_{ij}$ (where $j$ is the row corresponding to $a_0$) is maximal, but $A_{ij}=-1$ so $-b_i$ is maximal.
This will set $b_i$ to $-b_i$ (since remember the coefficient is $-1$), and for every other $b_j$ which is negative (so $a_0$'s coefficient is $-1$), this becomes $b_j-b_i\geq0$.

This makes more sense when our constraint is $a_1x_1+\cdots+a_nx_n\leq -b$, we add a slack variable to become $a_1x_1+\cdots+a_nx_n+s=-b$.
But then setting $x_i=0$ gives us an infeasible solution $s=-b<0$.
So we add an artificial variable $a$, $a_1x_1+\cdots+a_nx_n+s-a=-b$, then our goal is to now maximize $-a$ so that it becomes zero.

{\bf Example:} we want to maximize $5x_1-x_2-x_3$ subject to
$$ \eqalign{
    3x_1 - x_2 - x_3 &\leq -1\cr
    x_1 + 2x_2 - x_3 &\leq -2\cr
    2x_1 + x_2 &\leq 2\cr
    x_1 + x_2 &= 1\cr
    x_1,x_2,x_3 &\leq 0
} $$
We need two artificial variables: one for the equality constraint, and one for the negative constraints.
We also add slack variables.
So our problem becomes maximizing $-a_1-a_2$ subject to
$$ \eqalign{
    3x_1 - x_2 - x_3 + s_1 - a_1 &=-1\cr
    x_1 + 2x_2 - x_3 + s_2 - a_1&=-2\cr
    2x_1 + x_2 + s_3 &= 2\cr
    x_1 + x_2 + a_2 &= 1\cr
    -5x_1 + x_2 + x_3 + z &= 0\cr
} $$
So the initial tableau is
$$ \pmatrix{
    x_1 & x_2 & x_3 & s_1 & s_2 & s_3 & z & a_1 & a_2 & w & b\cr\noalign{\kern3pt\hrule\kern3pt}
    3 & -1 & -1 & 1 & 0 & 0 & 0 & -1 & 0 & 0 & -1\cr
    1 & 2 & -1 & 0 & 1 & 0 & 0 & -1 & 0 & 0 & -2\cr
    2 & 1 & 0 & 0 & 0 & 1 & 0 & 0 & 0 & 0 & 2\cr
    1 & 1 & 0 & 0 & 0 & 0 & 0 & 0 & 1 & 0 & 1\cr
    -5 & 1 & 1 & 0 & 0 & 0 & 1 & 0 & 0 & 0 & 0\cr\noalign{\kern3pt\hrule\kern3pt}
    0 & 0 & 0 & 0 & 0 & 0 & 0 & 1 & 1 & 1 & 0\cr
} $$
To make this canonical, we must adjust the objective row so that $a_2$ is in the basis:
$$ \pmatrix{
    x_1 & x_2 & x_3 & s_1 & s_2 & s_3 & z & a_1 & a_2 & w & b\cr\noalign{\kern3pt\hrule\kern3pt}
    3 & -1 & -1 & 1 & 0 & 0 & 0 & -1 & 0 & 0 & -1\cr
    1 & 2 & -1 & 0 & 1 & 0 & 0 & -1 & 0 & 0 & -2\cr
    2 & 1 & 0 & 0 & 0 & 1 & 0 & 0 & 0 & 0 & 2\cr
    1 & 1 & 0 & 0 & 0 & 0 & 0 & 0 & 1 & 0 & 1\cr
    -5 & 1 & 1 & 0 & 0 & 0 & 1 & 0 & 0 & 0 & 0\cr\noalign{\kern3pt\hrule\kern3pt}
    -1 & -1 & 0 & 0 & 0 & 0 & 0 & 1 & 0 & 1 & -1\cr
} $$
Now we pivot with respect to $a_0$, which enters, and $s_2$ leaves.
We then continue normally until we get
$$ \pmatrix{
    x_1 & x_2   & x_3   & s_1   & s_2   & s_3   & z     & a_1   & a_2   & w     & b\cr\noalign{\kern3pt\hrule\kern3pt}
    1   & 0     & 0     & 1/5   & -1/5  & 0     & 0     & 3/5   & 0     & 0     & 4/5\cr
    0   & 0     & 1     & -1/5  & -4/5  & 0     & 0     & 7/5   & 1     & 0     & 16/5\cr
    0   & 0     & 0     & -1/5  & 1/5   & 1     & 0     & -8/5  & 0     & 0     & 1/5\cr
    0   & 1     & 0     & -1/5  & 1/5   & 0     & 0     & 2/5   & 0     & 0     & 1/5\cr
    0   & 0     & 0     & 7/5   & -2/5  & 0     & 1     & 6/5   & -1    & 0     & 3/5\cr\noalign{\kern3pt\hrule\kern3pt}
    0   & 0     & 0     & 0     & 0     & 0     & 0     & 1     & 1     & 1     & 0\cr
} $$
Phase I is thus successful since we have maximized to $0$.
We can then perform phase II on the canonical tableau
$$ \pmatrix{
    x_1 & x_2   & x_3   & s_1   & s_2   & s_3   & z     & b\cr\noalign{\kern3pt\hrule\kern3pt}
    1   & 0     & 0     & 1/5   & -1/5  & 0     & 0     & 4/5\cr
    0   & 0     & 1     & -1/5  & -4/5  & 0     & 0     & 16/5\cr
    0   & 0     & 0     & -1/5  & 1/5   & 1     & 0     & 1/5\cr
    0   & 1     & 0     & -1/5  & 1/5   & 0     & 0     & 1/5\cr\noalign{\kern3pt\hrule\kern3pt}
    0   & 0     & 0     & 7/5   & -2/5  & 0     & 1     & 3/5
} $$
Now $s_2$ enters and $s_3$ leaves since the most negative value in the bottom row is $-2/5$ and the largest quotient is $1$.
This gives
$$ \pmatrix{
    x_1 & x_2   & x_3   & s_1   & s_2   & s_3   & z     & b\cr\noalign{\kern3pt\hrule\kern3pt}
    1   & 0     & 0     & 0     & 0     & 1     & 0     & 1\cr
    0   & 0     & 1     & -1    & 0     & 4     & 0     & 4\cr
    0   & 0     & 0     & -1    & 1     & 5     & 0     & 1\cr
    0   & 1     & 0     & 0     & 0     & -1    & 0     & 0\cr\noalign{\kern3pt\hrule\kern3pt}
    0   & 0     & 0     & 1     & 0     & 2     & 1     & 1
} $$
So the maximum value is $1$ with $x_1=1$, $x_2=0$ and $x_3=4$.
\qqed

\subsection{Duality}

\bdefn

    Call an LP problem of the form: maximize $c^\top x$ such that $Ax\leq b,x\geq0$ {\emphcolor primal}.
    Its {\emphcolor dual} is: minimize $b^\top y$ such that $A^\top y\geq c,y\geq0$.

\edefn

\blemm[title=The Weak Duality Theorem]

    If $x$ is a feasible solution to the primal LP and $y$ to the dual LP, then $c^\top x\leq b^\top y$.

\elemm

\Proof this is just a chain of inequalities:
$$ c^\top x = x^\top c \leq x^\top (A^\top y) = (x^\top A^\top)y = (Ax)^\top y \leq b^\top y \qed $$

\bthrm[title=The Strong Duality Theorem]

    If $x$ is a maximal solution to the primal LP, and $y$ a minimal solution to the dual LP, then $c^\top x=b^\top y$.

\ethrm

\Proof let us show that if $x$ is a maximal solution to the primal problem, then there exists a solution $y$ to the dual problem such that $c^\top x=b^\top y$.
By the weak duality theorem, this is sufficient.

Suppose first that our problem is in slack form: $Ax=b$.
Let $x$ be a maximal solution, and its corresponding basis is $B$.
Suppose that $A=A_B\mid_BA_N$ and $c^\top=c_B^\top\mid_Bc_N^\top$, then our initial tableau, multiplied by $A_B^{-1}\oplus(1)$ is
$$ \pmatrix{A_B^{-1}\cr&1}\cdot\pmatrix{A & 0 & b\cr -c^\top & 1 & 0} = \pmatrix{I&\mid_B&A_B^{-1}A_N & 0 & A_B^{-1}b\cr -c_B^\top&\mid_B&-c_N^\top & 1 & 0} $$
Row reducing gives us
$$ \pmatrix{I&\mid_B&A_B^{-1}A_N & 0 & A_B^{-1}b\cr 0^\top&\mid_B&-c_N^\top+c_B^\top A_B^{-1}A_N & 1 & z_B} $$
Since $x$ is optimal, the bottom row must be positive, so
$$ c_N^\top - c_B^\top A_B^{-1}A_N \leq 0 $$
but trivially, we also know (since the following is equal to $0$)
$$ c_B^\top - c_B^\top A_B^{-1}A_B \leq 0 $$
Thus by concatenating them with their order determined by $B$,
$$ c^\top - c_B^\top A_B^{-1}A \leq 0 $$

Now, our problem was $Ax\leq b$, which is equivalent to $(A\mid I)\binom xs=b$ by introducing slack variables.
And the objective function is $(c^\top\mid0)\binom xs$, so by our above computation (by splitting according to which indexes are slack variables), we get
$$ \eqalign{
    \hbox{For non-slack variables:}\quad&c^\top - c_B^\top\bar A_B^{-1}A \leq 0\cr
    \hbox{For slack variables:}\quad&0^\top - c_B^\top\bar A_B^{-1}I \leq 0
} $$
where $\bar A=A\mid I$.
So let us define $y^\top=c_B^\top\bar A_B^{-1}$, then we have that
$$ c^\top - c_B^\top\bar A_B^{-1}A = c^\top - y^\top A \leq 0 \implies y^\top A\geq c^\top $$
and $y^\top\geq0$.
So $y$ is a solution to the dual LP problem.
Furthermore
$$ b^\top y = y^\top b = c^\top_B\bar A_B^{-1}b = c^\top_Bx_B = c^\top x $$
as required.
\qed

