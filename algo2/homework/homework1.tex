\input pdfToolbox

\setlayout{horizontal margin=2cm, vertical margin=2cm}
\parindent=0pt
\parskip=4pt plus 2pt minus 2pt

\input ../preamble

\footline={}

\setcounter{section}{1}

%%%%%%%%%%%%%%%%%%%%%%%%%%%%%%%%%%%%%%%%%%%%%%%%%%%%%%%%%%%%%%%%

\def\printmcount{\the\counter{section}.\the\counter{math counter}}

{\bppbox{rgb{.5 1 .5}}{rgb{0 .4 0}}{rgb{.1 .4 0}}

    \centerline{\setfontandscale{bf}{20pt}Advanced Algorithms}
    \smallskip
    \centerline{\setfont{it}Homework \the\counter{section}}
    \centerline{\setfont{it}Ari Feiglin}

\eppbox}

\bexerc

    Solve the following linear programs using the simplex method:
    $$
    (1)\quad\vtop{\halign{\hfil\emphcolor#\quad&\hfil${}#{}$&${}#{}$\hfil\cr
        Maximize&3x_1+4x_2\cr
        Subject to&15x_1+10x_2&\leq300\cr
        &2.5x_1+5x_2&\leq110\cr
        &x_1,x_2&\geq0\cr
        }}
    \kern3cm
    (2)\quad\vtop{\halign{\emphcolor\hfil#\quad&\hfil${}#{}$&${}#{}$\hfil\cr
        Minimize&3x_1+x_2\cr
        Subject to&x_1+x_2&\leq5\cr
        &2x_1+x_2&\geq8\cr
        &x_1,x_2&\geq0\cr
        }}
    $$

\eexerc

\benum
    \item We add slack variables and create the initial tableau:
    $$ \pmatrix{
        x_1&x_2&s_1&s_2&z&b\cr\hline
        15&10&1&0&0&300\cr
        2.5&5&0&1&0&110\cr\hline
        -3&-4&0&0&1&0
    } $$
    We then choose the second column as the pivot column, since $-4$ has the most negative value.
    Computing ratios: $300/10=30$ and $110/5=22$, we take the minimum so our pivot row is the second row.
    Pivoting:
    $$ \pmatrix{
        x_1&x_2&s_1&s_2&z&b\cr\hline
        10&0&1&-2&0&80\cr
        0.5&1&0&0.2&0&22\cr\hline
        -1&0&0&4&1&88
    } $$
    Our pivot column is now the first, and taking the minimum ratio gives the first row as the pivot row.
    Pivoting:
    $$ \pmatrix{
        x_1&x_2&s_1&s_2&z&b\cr\hline
        1&0&0.1&-0.2&0&8\cr
        0.5&1&0&0.2&0&22\cr\hline
        0&0&0.1&3.8&1&96
    } $$
    Since the bottom row is all positive, we have completed the algorithm and the maximum value is $96$.
    (Because of this we don't need to finish the computation of row reducing the second row, unless we want the values for the variables.)

    \item The second constraint is equivalent to $-2x_1-x_2\leq-8$.
    And the objective is equivalent to maximizing $-3x_1-x_2$, and then taking the inverse.
    We add slack variables and an artificial variable.
    Phase I then becomes maximizing $-a$ subject to $x_1+x_2+s_1=5$ and $-2x_1-x_2+s_2-a=-8$.
    We also want to save the original objective, so we add the constraint $z+3x_1+x_2=0$.
    Our initial tableau for phase I becomes:
    $$ \pmatrix{
        x_1&x_2&s_1&s_2&z&a&w&b\cr\hline
        1&1&1&0&0&0&0&5\cr
        -2&-1&0&1&0&-1&0&-8\cr
        3&1&0&0&1&0&0&0\cr\hline
        0&0&0&0&0&1&1&0
    } $$
    $a$ must enter the initial basis, so row reducing gives
    $$ \pmatrix{
        x_1&x_2&s_1&s_2&z&a&w&b\cr\hline
        1&1&1&0&0&0&0&5\cr
        2&1&0&-1&0&1&0&8\cr
        3&1&0&0&1&0&0&0\cr\hline
        -2&-1&0&1&0&0&1&-8
    } $$
    We begin pivoting.
    Our pivot column is the first, and pivot row the second.
    Pivoting gives
    $$ \pmatrix{
        x_1&x_2&s_1&s_2&z&a&w&b\cr\hline
        0&0.5&1&0.5&0&-0.5&0&1\cr
        1&0.5&0&-0.5&0&0.5&0&4\cr
        0&-0.5&0&1.5&1&-1.5&0&-12\cr
        0&0&0&0&0&1&1&0
    } $$
    So we have finished applying the simplex method in phase I, and we have a maximum value of $0$, as required.
    So we can now abandon the artificial variable and simplex on the resulting tableau:
    $$ \pmatrix{
        x_1&x_2&s_1&s_2&z&b\cr\hline
        0&0.5&1&0.5&0&1\cr
        1&0.5&0&-0.5&0&4\cr
        0&-0.5&0&1.5&1&-12\cr
    } $$
    The pivot column is now the second, and the pivot row is the first.
    Pivoting:
    $$ \pmatrix{
        x_1&x_2&s_1&s_2&z&b\cr\hline
        0&1&2&1&0&2\cr
        1&0&-1&-1&0&3\cr
        0&0&1&2&1&-11
    } $$
    So the maximum value of $-3x_1-x_2$ is $-11$, and so the minimum value of $2x_1+x_2$ is $11$.
\eenum

\bexerc

    Consider the following linear program: maximize $x_1+x_2$ subject to $-2x_1+x_2\leq2$ and $x_1-x_2\leq1$.
    \benum
        \item Use the simplex method to show that the problem is unbounded.
        \item Make a picture of the feasible region of the dual problem.
    \eenum

\eexerc

\benum
    \item Adding slack variables, we get an initial tableau:
    $$ \pmatrix{
        x_1&x_2&s_1&s_2&z&b\cr\hline
        -2&1&1&0&0&2\cr
        1&-1&0&1&0&1\cr\hline
        -1&-1&0&0&1&0
    } $$
    Pivoting on the first column, second row gives
    $$ \pmatrix{
        x_1&x_2&s_1&s_2&z&b\cr\hline
        0&-1&1&2&0&4\cr
        1&-1&0&1&0&1\cr
        0&-2&0&1&1&1
    } $$
    Now we want to pivot on the second column, but none of the rows have a positive coefficient.
    Thus the current vertex/bfs has no adjacent vertices other than the one we just traversed.
    So the problem is unbounded.

    \item The transpose of $A$ is
    $$ A^\top = \pmatrix{-2 & 1\cr 1 & -1} $$
    So the dual problem is minimizing $2x+y$ subject to $-2x+y\geq1$ and $x-y\geq1$.
    These two graphs show us that there is no feasible region (the colors represent each of the constraints, the feasible region is their intersection above the $x$-axis, empty).

    \centerline{\vbox to70pt{\vss\hbox to40pt{\hss\pdfliteral{q 1.25 0 0 1.25 0 0 cm
        1 J
        1 .4 .4 rg
        -30 -30 m
        -30 50 l
        20 50 l
        -20 -30 l f
        .4 .4 1 rg
        -20 -30 m
        30 -30 l
        30 10 l
        20 10 l f
        -20 -30 m
        20 50 l
        -20 -30 m
        20 10 l S
        .9 w
        -30 0 m
        30 0 l
        0 -35 m
        0 55 l S
        Q
    }\hss}}}

    \vfill\break
    By duality, since the dual is infeasible, the primal must be unbounded.
\eenum

\bexerc

    Construct a linear program with two variables and two constraints such that both the primal and dual problem have no feasible solution.

\eexerc

Take
$$ A = \pmatrix{0 & 1\cr -1 & -1},\qquad b = \pmatrix{-1\cr1},\qquad c = \pmatrix{1\cr-1} $$
The primal has the constraint $y\leq-1$ which cannot be satisfied, so it is infeasible.
The transpose of $A$ is
$$ A^\top = \pmatrix{0 & -1\cr1 & -1} $$
and so it has the constraint $-y\geq1$, equivalently $y\leq-1$, which cannot be satisfied.
So both the primal and the dual are infeasible.

\bexerc

    Construct an integer program for the 3D-Matching problem.
    3D-Matching is defined as follows: given three disjoint sets each of $q$ elements $X,Y,Z$ and a set $M\subseteq X\times Y\times Z$, determine if there exists a subset $M'\subseteq M$ with $q$ elements
    such that for every two elements in $M'$, they share no coefficient.

\eexerc

For every $(i,j,k)\in M$ let us define a variable $m_{i,j,k}$.
Our problem is then to maximize the objective function $\sum_{(i,j,k)\in M}m_{i,j,k}$ subject to $m_{i,j,k}+m_{a,b,c}\leq1$ when $i=a$ or $j=b$ or $k=c$ for every $(i,j,k),(a,b,c)\in M$.
This also means that for every $(i,j,k)\in M$, $m_{i,j,k}\in\set{0,1}$ since it is a nonnegative integer bound by $1$.

Now suppose some valuation of the variables gives a value of $p$ for the objective function.
Define $M'=\set{(i,j,k)}[m_{i,j,k}=1]$.
This then has $p$ elements, as $m_{i,j,k}\in\set{0,1}$ for every $(i,j,k)\in M$.
And for every $(i,j,k),(a,b,c)\in M'$ suppose they have a coefficient in common, wlog $i=a$.
Then $m_{i,j,k}+m_{a,b,c}\leq1$ is one of the constraints, but because they're in $M'$ both their values are $1$ so we get $1+1\leq1$ in contradiction.

So all we need to do is return if the maximum value is at least $q$ ($M'$ can only have at most $q$ elements, since every element in $X,Y,Z$ can be used in only a single element in $M'$, so it is sufficient
to check if it {\it equals} $q$).

\bye

