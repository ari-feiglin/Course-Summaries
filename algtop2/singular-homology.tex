\subsection{Chain Complexes}

We begin by defining a {\it chain complex}.
A chain complex is a sequence of Abelian groups with homomorphisms between them:

\medskip
\centerline{\drawdiagram{
    $\cdots$ & $C_{n+1}$ & $C_n$ & $C_{n-1}$ & $\cdots$ & $C_0$ & $0$\cr
}{
    \diagarrow{from={1,1}, to={1,2}, dashed}
    \diagarrow{from={1,2}, to={1,3}, text=$\partial_{n+1}$, y distance=.25cm}
    \diagarrow{from={1,3}, to={1,4}, text=$\partial_{n}$, y distance=.25cm}
    \diagarrow{from={1,4}, to={1,5}, dashed}
    \diagarrow{from={1,5}, to={1,6}, dashed}
    \diagarrow{from={1,6}, to={1,7}}
}}
\medskip

such that for every $n$, $\partial_n\circ\partial_{n+1}=0$, in other words ${\rm Im}\partial_{n+1}\subseteq\ker\partial_n$.
Define $Z_n=\ker\partial_n$, and its elements will be called {\it $n$-dimensional cycles}.
And define $B_n={\rm Im}\partial_{n+1}$, its elements will be called {\it boundaries}.
Elements of the groups $C_n$ will be called {\it $n$-dimensional chains}.

We now want to define a category of chain complexes.
To do so we must define morphisms between chain complexes.
So suppose we have two chain complexes ${\scr C}=\set{C_n,\partial_n}$ and ${\scr D}=\set{D_n,\partial'_n}$.
We define a morphism from ${\scr C}$ to ${\scr D}$ to be a sequence of homomorphisms $f_n\colon C_n\longto D_n$ which preserves the structure of the chain.
Meaning $\partial_n'\circ f_n=f_{n-1}\circ\partial_n$, in other words the following diagram commutes:

\bigskip
\centerline{\def\diagrowbuf{.5cm}\drawdiagram{
    $\cdots$ & $C_{n+1}$ & $C_n$ & $C_{n-1}$ & $\cdots$ & $C_0$ & $0$\cr
    $\cdots$ & $D_{n+1}$ & $D_n$ & $D_{n-1}$ & $\cdots$ & $D_0$ & $0$\cr
}{
    \diagarrow{from={1,1}, to={1,2}, dashed}
    \diagarrow{from={1,2}, to={1,3}, text=$\partial_{n+1}$, y distance=.25cm}
    \diagarrow{from={1,3}, to={1,4}, text=$\partial_{n}$, y distance=.25cm}
    \diagarrow{from={1,4}, to={1,5}, dashed}
    \diagarrow{from={1,5}, to={1,6}, dashed}
    \diagarrow{from={1,6}, to={1,7}}
    \diagarrow{from={2,1}, to={2,2}, dashed}
    \diagarrow{from={2,2}, to={2,3}, text=$\partial'_{n+1}$, y distance=.25cm}
    \diagarrow{from={2,3}, to={2,4}, text=$\partial'_{n}$, y distance=.25cm}
    \diagarrow{from={2,4}, to={2,5}, dashed}
    \diagarrow{from={2,5}, to={2,6}, dashed}
    \diagarrow{from={2,6}, to={2,7}}
    \diagarrow{from={1,2}, to={2,2}, text=$f_{n+1}$, x distance=-.5cm}
    \diagarrow{from={1,3}, to={2,3}, text=$f_{n}$, x distance=-.25cm}
    \diagarrow{from={1,4}, to={2,4}, text=$f_{n-1}$, x distance=-.5cm}
    \diagarrow{from={1,6}, to={2,6}, text=$f_{0}$, x distance=-.25cm}
}}
\medskip

To simplify writing, we will write $\partial\circ f=f\circ\partial$, which $f$ and which $\partial$ is being referred to will be understood from context.

The composition of two morphisms $\set{f_n}\colon{\scr C}\longto{\scr D}$ and $\set{g_n}\colon{\scr D}\longto{\scr E}$ is defined to be $\set{g_n\circ f_n}\colon{\scr C}\longto{\scr E}$.
This is indeed a morphism:
$$ \partial\circ f\circ g = f\circ\partial\circ g = f\circ g\circ\partial $$
And then this implies that the identity morphism is just ${\rm Id}_{\scr C}=\set{\rm Id_{C_n}}\colon{\scr C}\longto{\scr C}$, as if $\set{f_n}\colon{\scr C}\longto{\scr D}$ then
$$ \set{f_n}\circ{\rm Id}_{\scr C} = \set{f_n\circ {\rm Id}_{C_n}} = \set{f_n},\qquad {\rm Id}_{\scr D}\circ\set{f_n} = \set{{\rm Id}_{D_n}\circ f_n} = \set{f_n} $$
Associativity is clear, so ${\bf Comp}$, the category of chain complexes, is indeed a category.

Now recall that by definition $\partial_n\circ\partial_{n+1}=0$, meaning
$$ B_n \subseteq Z_n \subseteq C_n $$
Since these groups are all Abelian, they are normal in one another, so let us define the {\it $n$th homology group} of a chain complex $\scr C$ as
$$ H_n({\scr C})\coloneqq\slfrac{\displaystyle Z_n}{\displaystyle B_n} = \slfrac{\displaystyle\ker\partial_n}{\displaystyle{\rm Im}\partial_{n+1}} $$

\bprop

    A chain complex morphism $\set{f_n}\colon\C\longto\D$ maps cycles to cycles and boundaries to boundaries.

\eprop

\Proof let $z\in C_n$ be a cycle, i.e. $\partial z=0$, but then $f(z)$ is a cycle since $\partial f(z)=f(\partial z)=f(0)=0$.
And let $b\in C_n$ be a boundary, so there exists an $a\in C_{n+1}$ such that $b=\partial a$.
Then $f(b)=f\partial(a)=\partial f(a)=\partial b$, so $f(b)$ is a boundary as well.
\qed

This means that if $\set{f_n}\colon\C\longto\D$ is a morphism of chain complexes, $\set{f_n}\colon Z_n(\C)\longto Z_n(\D)$ is well-defined, and so we have that

\centerline{\def\diagrowbuf{.5cm}\def\diagcolbuf{.5cm}
\drawdiagram{
    $Z_n(\C)$ & $Z_n(\D)$\cr
    $H_n(\C)$ & $H_n(\D)$\cr
}{
    \diagarrow{from={1,1}, to={1,2}}
    \diagarrow{from={1,1}, to={2,1}}
    \diagarrow{from={1,2}, to={2,2}}
    \diagarrow{from={1,1}, to={2,2}, curve=1cm, color=blue}
}}

Where the blue arrow $\psi$ is just the quotient map composed with $f_n$.
This induces a group morphism
$$ H_n(\set{f_n}) = f_*\colon H_n(\C)\longto H_n(\D) $$
since we can define $f_*([z])=\psi(z)$ since if $[z]=[z']$ then $z-z'\in B_n(\C)$ and so $f(z-z')\in B_n(\D)$ and thus the quotient of $f(z-z')$ is just $0$, so $\psi(z)=\psi(z')$.
Explicitly,
$$ f_*[z] = [f_nz] $$

We now claim that $H_n$ is a functor from the category of chain complexes ${\bf Comp}$ to the category of Abelian groups ${\bf Ab}$.
Now suppose $\set{f_n}\colon\C\longto\D$ and $\set{g_n}\colon\D\longto\E$ are chain complex morphisms, then the following diagram commutes

\bigskip
\centerline{\def\diagrowbuf{.5cm}\def\diagcolbuf{.5cm}
\drawdiagram{
    $Z_n(\C)$ & $Z_n(\D)$ & $Z_n(\E)$\cr
    $H_n(\C)$ & $H_n(\D)$ & $H_n(\E)$\cr
}{
    \diagarrow{from={1,1}, to={1,2}, text=$f$, y distance=.25cm}
    \diagarrow{from={1,1}, to={2,1}}
    \diagarrow{from={1,2}, to={2,2}}
    \diagarrow{from={1,2}, to={1,3}, text=$g$, y distance=.25cm}
    \diagarrow{from={1,3}, to={2,3}}
    \diagarrow{from={2,1}, to={2,2}, text=$f_*$, y distance=-.25cm}
    \diagarrow{from={2,2}, to={2,3}, text=$g_*$, y distance=-.25cm}
}}
\bigskip

And so $(g\circ f)_*=g_*\circ f_*$, and it is easily verified that ${\rm id}_*={\rm id}$ so $H_n$ is a functor ${\bf Comp}\longto{\bf Ab}$ (the category of Abelian groups).

\subsection{Singular Complex}

We now define a functor from ${\bf Top}$ to ${\bf Comp}$.

\bdefn

    Let $B$ be a set, then define the {\emphcolor free Abelian group} over $B$ to be
    $$ \FAof B = \bigoplus_{b\in B}{\bb Z} = \set{\phi\colon B\longto{\bb Z}}[\phi(b)\neq0\hbox{ for only finitely many $b\in B$}] $$

\edefn

Note then that there is a correspondence between $B$ and $\FAof B$: $b\oto\phi_b$ where
$$ \phi_b(x) = \cases{1 & $x=b$\cr 0 & $x\neq b$} $$
so we can identify $b$ with $\phi_b$, and it is easy to see that every element of $\FAof B$ can be written as $\sum_{i=1}^kn_i\phi_{b_i}$, abusing notation $\sum_{i=1}^knb_i$ and such a representation is
unique.

Notice that if $B$ is a set, $G$ an Abelian group, and $g\colon B\longto G$ a function, then there exists a unique group homomorphism $L\colon\FAof B\longto G$ which extends $g$.
This is defined by
$$ L\colon\sum_{i=1}^kn_ib_i \longmapsto \sum_{i=1}^kn_ig(b_i) $$

\bdefn

    The {\emphcolor $n$-dimensional simplex} is defined to be
    $$ \Delta^n \coloneqq \set{(x_0,\dots,x_n)\in{\bb R}^{n+1}}[x_i\geq0,\sum_{i=0}^nx_i=1] $$

\edefn

$\Delta^n$ has $n+1$ faces, and is homeomorphic to $D^n$.

\bdefn

    Let $X$ be a topological space, then an {\emphcolor $n$-dimensional singular simplex} in $X$ is a morphism (in the category of topological spaces; a continuous map) $\Delta^n\longto X$.
    Define $S_n(x)$ to be the set of all $n$-dimensional singular simplexes in $X$, and define $C_n(X)=\FAof{S_n(x)}$.

\edefn

We now want to define a chain complex on the sequence $C_n(X)$.

Let us define a set of maps $\tau_i^n\colon\Delta^{n-1}\longto\Delta^n$ for $0\leq i\leq n$ which maps
$$ \tau_i^n\colon (x_0,\dots,x_{n-1})\mapsto(x_0,\dots,x_{i-1},0,x_i,\dots,x_{n-1}) $$
This is a well-defined continuous map, and geometrically it maps $\Delta^{n-1}$ to one of the faces of $\Delta^n$.

Let $\sigma\in S_n(x)$, then let us define
$$ \partial(\sigma) \coloneqq \sum_{i=0}^n(-1)^i\sigma\circ\tau_i^n $$
Note that the composition is well-defined since $\Delta^{n-1}\xvarrightarrow{\tau_i^n}\Delta^n\xvarrightarrow{\,\sigma\,}X$, meaning $\sigma\circ\tau_i^n$ is an $n-1$-dimensional singular simplex.
Thus $\partial$ can be extended to a map $\partial\colon C_n(X)=\FAof{S_n(X)}\longto\FAof{S_{n-1}(X)}=C_{n-1}(X)$
Notice that
$$ \partial_{n-1}(\partial_n\sigma) = \partial_{n-1}\parens{\sum_{i=0}^n(-1)^i\sigma\circ\tau_i^n} = \sum_{i=0}^n(-1)^i\partial_{n-1}(\sigma\circ\tau_i^n) =
\sum_{i=0}^n\sum_{j=0}^{n-1}(-1)^{i+j}\sigma\circ\tau_i^n\circ\tau_j^{n-1} $$
Notice that $\tau^n_i\circ\tau_j^{n-1}=\tau^n_j\circ\tau^{n-1}_{i-1}$ which can be verified from its definition, but the first has a sign of $(-1)^{i+j}$ in the sum and the second has $-(-1)^{i+j}$.
And so the sum is zero.

Thus we have defined a chain complex on $C_n(X)$, let us denote it by $\C(X)$, this is the first step in defining the functor.
Next we must define the correspondence between morphisms.

Let $f\colon X\longto Y$ be a continuous map between topological spaces.
Let us define $f_\sharp\colon\C(X)\longto\C(Y)$.
First we define it for $\sigma\in S_n(X)$ by $f_\sharp(\sigma)=f\circ\sigma$.
Since $\sigma\colon\Delta^n\longto X$ is continuous, so is $f\circ\sigma\colon\Delta^n\longto Y$ and so $f_\sharp$ is well-defined on the generators of $C_n(X)$.
This can be extended by linearity to $f_\sharp\colon C_n(X)\longto C_n(Y)$.
Notice that we ignore the subscripts and superscripts $(f_\sharp)_n^X$ for brevity and readability.

Now we must verify that this is a morphism of chain complexes, i.e. that $\partial f_\sharp = f_\sharp\partial$.
So
$$ f_\sharp\partial\sigma = f_\sharp\parens{\sum_{i=0}^n(-1)^i\sigma\circ\tau_i^n} = \sum_{i=0}^n(-1)^if_\sharp(\sigma\circ\tau_i^n) = \sum_{i=0}^n(-1)^if\circ\sigma\circ\tau_i^n
= \sum_{i=0}^n(-1)^i(f\circ\sigma)\circ\tau_i^n = \partial f_\sharp\sigma $$
and since this holds for generators, by linearity it holds for all $C_n(X)$.
Thus $f_\sharp$ is indeed a morphism of chain complexes.

Thus we have defined a functor ${\bf Top}\longto{\bf Comp}$.

\subsection{Singular Homology}

We have two functors ${\bf Top}\longto{\bf Comp}\longto{\bf Ab}$, and so composing them together gives us a functor ${\bf Top}\longto{\bf Ab}$.
For a topological space $X$, we will denote its image under this functor as $H_n(X)$, called the {\it $n$th homological group} of $X$.
And for a continuous map $f$, we denote its image as $f_*$ or $H_n(f)$.

Let us compute the homological groups of the trivial space: $X=\set{p}$.
Notice that $S_n(X)=\set{K_n}$ where $K_n$ is the constant map $\Delta^n\longto\set p$, and so $C_n(X)={\bb Z}$.
We want to now compute what the boundary operators are, so
$$ \partial K_n = \sum_{i=0}^n(-1)^iK_n\circ\tau_i^n $$
but $K_n\circ\tau_i^n$ is a morphism $\Delta^{n-1}\longto\set p$ meaning it is equal to $K_{n-1}$, thus $\partial K_n=\parens{\sum_{i=0}^n(-1)^i}K_{n-1}$.
For $n$ even this is then $K_{n-1}$ (or $1$), and $0$ for $n$ odd.
This means that either $\ker\partial=0$ or ${\rm Im}\partial={\bb Z}$, thus $H_n=0$ for $n>0$.
For $n=0$, we have that $\partial_0\colon{\bb Z}\longto0$ and so its kernel is ${\bb Z}$, but $\partial_1$ is trivial and so its image is $0$.
Thus $H_0={\bb Z}$.

So we have shown

\bprop

    Let $X=\set p$ be the trivial topological space, then its homological groups are
    $$ H_n(X) = \cases{{\bb Z} & $n=0$\cr0 & $n>0$} $$

\eprop

\bprop

    Let $X$ be path connected, then $H_0(X)\cong{\bb Z}$.

\eprop

\Proof we are concerned with the chain:

\bigskip
\centerline{\drawdiagram{
    $C_1(X)$ & $C_0(X)$ & $0$\cr
}{
    \diagarrow{from={1,1}, to={1,2}, text=$\partial_1$, y distance=.25cm}
    \diagarrow{from={1,2}, to={1,3}, text=$\partial_0$, y distance=.25cm}
}}
\bigskip

So first let us understand $C_0(X)$, this is generated by $S_0(X)$, all the maps $\Delta^0\longto X$ which are just all the points in $X$.
And $S_1(X)$ is generated by all the maps $I\cong\Delta^1\longto X$, so all the paths in $X$.
The boundary of a $1$-simplex is then
$$ \partial_1\sigma = \sigma(1) - \sigma(0) $$
and thus $B_1(X)={\rm Im}\partial_1$ is generated by elements of the form $a-b$ where there exists a path between $a$ and $b$.
Since $X$ is path-connected, this means that $B_1(X)$ is generated by $a-b$ for $a,b\in X$.
Now, the subgroup generated by this is $\set{\sum n_ip_i}[p_i\in X,\,\sum n_i=0]$.

And now $\partial_0$'s kernel is just $C_0(X)$ which is simply the free group generated by $X$.
Thus
$$ H_0(X) = \slfrac{\ds\set{\sum n_ip_i}}{\ds\set{\sum n_ip_i}[\sum n_i=0]} $$
This is isomorphic to ${\bb Z}$ since we can define $\phi\colon C_0(X)\longto{\bb Z}$ by $\sum n_ip_i\mapsto\sum n_i$ and this is a group homomorphism whose image is ${\bb Z}$ and whose kernel is all the
points $\sum n_ip_i$ where $\sum n_i=0$.
Thus by the isomorphism theorem, $H_0(X)\cong{\bb Z}$.
\qed

\bthrm

    Let $X$ be a topological space where $\set{A_\alpha}_{\alpha\in I}$ are its path connected components.
    Then for every $n$,
    $$ H_n(X) \cong \bigoplus_{\alpha\in I}H_n(A_\alpha) $$

\ethrm

\Proof notice that if $\sigma\colon\Delta^n\longto X$ is an $n$-simplex, then its image is contained within a path connected component.
This is since $\Delta^n$ is path-connected, so $\sigma\Delta^n$ must be too.
Thus for every $\gamma=\sum n_i\sigma_i\in S_n(X)$ we can write it as $\gamma=\sum\gamma_i$ for $\gamma_i\in S_n(A_i)$.
And so $C_n(X)=\bigoplus_{\alpha\in I}C_n(A_\alpha)$.

Notice that $\gamma$ is a cycle iff every $\gamma_i$ is a cycle, since $\partial\gamma=\sum\partial\gamma_i$ and this is an element of a direct sum, so it is zero iff $\partial\gamma_i=0$.
Thus $Z_n(X)=\bigoplus_{\alpha\in I}Z_n(A_\alpha)$.
And similarly we see that $B_n(X)=\bigoplus_{\alpha\in I}B_n(A_\alpha)$.
Thus $H_n(X)=\bigoplus_{\alpha\in I}H_n(A_\alpha)$.
\qed

\bcoro

    If $X$ is a topological space with $\set{A_\alpha}_{\alpha\in I}$ path connected components, $H_n(X)=\bigoplus_{\alpha\in I}{\bb Z}$.

\ecoro

\bthrm[name=abelofhomotopy]

    Let $X$ be path-connected and $a\in X$, then $H_1(X)\cong\Ab\pi_1(X,a)$.

\ethrm

For two chains, $a,b\in C_n(X)$ say that they are {\it homological} if $a-b$ is a boundary (i.e. $a-b\in B_n(X)$).
Write this as $a\approx b$.

\blemm

    Let $\sigma,\tau$ be $1$-simplexes.
    \benum
        \item if $\sigma$ is constant, then it is a boundary, i.e. $\sigma\approx0$.
        \item if $\sigma\hdi\tau$ (since they are maps from $I\cong\Delta^1\longto X$), then $\sigma\approx\tau$.
        \item if $\sigma(1)=\tau(0)$ then $\sigma*\tau\approx\sigma+\tau$.
        \item $\sigma+\bar\sigma\approx0$
    \eenum

\elemm

\Proof
\benum
    \item If $\sigma$ is constant, then it is $K^1_p$ for some $p\in X$.
        And as we have already computed
        $$ \partial K^n_p = \cases{K^{n-1}_p & $n$ even\cr 0 & $n$ odd} $$
        Thus $\partial K^2_p=K^{n-1}_p$, meaning $\sigma$ is a boundary.
    \item Let us look at the homotopy

        \centerline{\def\coordhbuf{0pt}\def\coordvbuf{-5pt}
        \drawdiagram{
            &&\cr
            &$H$&\cr
            &&\cr
        }{
            \diagarrow{from={1,1}, to={1,3}, text=$\tau$, y distance=.25cm}
            \diagarrow{from={3,1}, to={3,3}, text=$\sigma$, y distance=-.25cm}
            \diagarrow{from={1,1}, to={3,1}, right cap=-, text=$K_a$, x distance=-.25cm}
            \diagarrow{from={1,3}, to={3,3}, right cap=-, text=$K_b$, x distance=.25cm}
        }}

        Since $H$ is surjective, it induces a map on the quotient space $\slfrac{I\times I}{I\times\set1}$, the map $G$:

        \centerline{\def\coordhbuf{0pt}\def\coordvbuf{-5pt}
        \drawdiagram{
            &&\cr
            &$G$&\cr
            &&\cr
        }{
            \diagarrow{from={1,1}, to={2,3}, text=$\tau$, y distance=.25cm}
            \diagarrow{from={3,1}, to={2,3}, text=$\sigma$, y distance=-.25cm}
            \diagarrow{from={1,1}, to={3,1}, right cap=-, text=$K_a$, x distance=-.25cm}
        }}

        The quotient space can be viewed as a $2$-simplex by assigning an order to its vertices.
        Then its boundary is
        $$ \partial G = K_a - \sigma + \tau $$
        and since $\partial G$ is a boundary, we have that
        $$ K_a - \sigma + \tau \approx 0 $$
        by $(1)$ we have that $K_a\approx0$ so $\sigma-\tau\approx0$.

    \item The idea is to define a simplex of the form

        \centerline{\def\coordhbuf{0pt}\def\coordvbuf{-2.5pt}
        \drawdiagram{
            &&\cr
            &$G$&\cr
            &&\cr
        }{
            \diagarrow{from={2,3}, to={1,1}, text=$\tau$, y distance=.25cm}
            \diagarrow{from={3,1}, to={2,3}, text=$\sigma$, y distance=-.25cm}
            \diagarrow{from={1,1}, to={3,1}, text=$\sigma*\tau$, x distance=-.5cm}
        }}

        Notice that such a simplex is possible: each horizontal line in the domain can be made constant.
        And its boundary is
        $$ \partial G = \tau - \sigma*\tau + \sigma $$
        so $\sigma*\tau\approx\sigma+\tau$ since $\partial G\approx0$.

    \item This is direct from the previous three points:
        $$ \sigma+\overline\sigma \buildrel(3)\over\approx \sigma*\overline\sigma \buildrel(2)\over\approx K_b \buildrel(1)\over\approx 0 $$
\eenum

{\bf Proof} (of \refmath[theorem]{abelofhomotopy}): let us define a homomorphism
$$ F\colon\pi_1(X,a) \longto H_1(X) $$
Denote homotopy equivalence classes by $\gen\bullet$ and the equivalence classes of $H_1(X)$ by $[\bullet]$.
Then we define
$$ \gen\phi\xvarmapsto{\quad F\quad}\relax[\phi] $$
This is well-defined: if $\phi\hdi\psi$ then $\phi\approx\psi$ and so $[\phi]=[\psi]$ (since $H_n(X)$ is the partition of $Z_n(X)$ relative to $\approx$).
Notice that $\gen{\phi*\psi}\mapsto[\phi*\psi]=[\phi+\psi]=[\phi]+[\psi]$.
So this is indeed a homomorphism.
Since $H_1(X)$ is Abelian, this induces a homomorphism
$$ \overline F\colon\Ab\pi_1(X,a)\longto H_1(X) $$
Let us now define a homomorphism
$$ G\colon C_1(X)\longto\Ab\pi_1(X,a) $$
denote the equivalence classes of $\Ab\pi_1(X,a)$ by $\gen{\gen\bullet}$.
For every $x\in X$, choose a path $\gamma_x$ from $a$ to $x$, then for $\sigma\in S_1(X)$ define
$$ \hat\sigma = \gamma_{\sigma(0)}*\sigma*\overline\gamma_{\sigma(1)} \hbox{ from $a$ to $a$} $$
And define
$$ \sigma\xvarmapsto{\quad G\quad}\gen{\gen{\hat\sigma}} $$
And extend by linearity to $G\colon C_1(X)\longto\Ab\pi_1(X,a)$.
We can then restrict $G$ to $Z_1(X)$, and in order for this to induce a map on $\slfrac{Z_1(X)}{B_1(X)}$ we must have that $G\bigl|_{B_1(X)}=0$.
So let $A$ be a $2$-simplex, then we must show $G(\partial A)=0$.
We know
$$ G(\partial A) = G(A\circ\tau_0-A\circ\tau_1+A\circ\tau_2) = \ggen{\widehat{A\circ\tau_0}} - \ggen{\widehat{A\circ\tau_1}} + \ggen{\widehat{A\circ\tau_2}} $$
Now, $\ggen\sigma+\ggen\tau=\gen{\!\gen\sigma\gen\tau\!}$ and $-\ggen\sigma=\gen{\!\sigma^{-1}\!}$ by Abelianization, so this is equal to
$$ \gen{\!\gen{\widehat{A\circ\tau_0}}\gen{\widehat{A\circ\tau_1}}\gen{\widehat{A\circ\tau_2}}\!} = \ggen{\widehat{A\circ\tau_0}*\overline{\widehat{A\circ\tau_1}}*\widehat{A\circ\tau_2}} $$
As is easily verified,
$$ = \ggen{\widehat{A\circ\tau_0}*\widehat{\overline{A\circ\tau_1}}*\widehat{A\circ\tau_2}} = \ggen{\varwidehat{A\circ\tau_0*\overline{A\circ\tau_1}*A\circ\tau_2}} $$
Since $A\colon\Delta^2\longto X$ is a simplex, $A\circ\tau_0*\overline{A\circ\tau_1}*A\circ\tau_2$ is null-homotopic (the homotopy can condense the curve to a point through the image of $A$).
Therefore its hat is as well, meaning this is all equal to zero, as required.

So $G$ induces a homomorphism
$$ \overline G\colon H_1(X)\longto\Ab\pi_1(X,a) $$
Notice that
$$ \overline G\circ\overline F\ggen\phi = \overline G[\phi] = \ggen{\hat\phi} $$
We know that $\hat\phi=\gamma_a\phi\overline\gamma_a$ which is conjugate to $\phi$, so in the Abelianization they are equal.
So $\overline G\circ\overline F={\rm id}$.
Now suppose $[z]\in H_1(X)$ where $z=\sum n_i\sigma_i$ then
$$ \overline F\circ\overline G[z] = \overline F\parens{\sum n_i\ggen{\hat\sigma_i}} = \sum n_i[\hat\sigma_i] = \bracks{\sum n_i\hat\sigma_i} $$
So we need to show that if $\sum n_i\sigma_i$ is a cycle then $\sum n_i\hat\sigma_i\approx\sum n_i\sigma_i$.
Define $T\colon C_0(X)\longto C_1(X)$ by $T(p)=\gamma_p$, so
$$ \hat\sigma = \gamma_{\sigma0}*\sigma*\overline\gamma_{\sigma1} \approx \gamma_{\sigma0} + \sigma - \gamma_{\sigma1} = \sigma - T\partial\sigma $$
And so
$$ \sum n_i\hat\sigma_i \approx \sum n_i\sigma_i - T\partial\sum n_i\sigma_i = z - T\partial z $$
since $z$ is a cycle, $\partial z=0$ and so this is equal to $z$.
Thus $\hat z\approx z$ as required.

So $\overline F,\overline G$ are inverse isomorphisms, meaning $H_1(X)\cong\Ab\pi_1(X,a)$.
\qed

\bdefn

    Let $\C,\D$ be two categories and let $F,G\colon\C\longto\D$ be functors.
    Then a {\emphcolor natural transformation} $\eta$ from $F$ to $G$ is a correspondence such that
    \benum
        \item for every object $X\in\C$, $\eta$ associates a morphism $\eta_X\colon F(X)\longto G(X)$ called the {\emphcolor component} of $X$.
        \item for every $f\colon X\longto Y$ morphism, $\eta_Y\circ F(f)=G(f)\circ\eta_X$, i.e. the following diagram commutes

            \bigskip
            \centerline{\def\diagrowheight{1cm}\drawdiagram{
                $F(X)$ & $G(X)$\cr
                $F(Y)$ & $G(Y)$\cr
            }{
                \diagarrow{from={1,1}, to={1,2}, text=$\eta_X$, y distance=.25cm, color=rgb{.8 .1 .1}}
                \diagarrow{from={2,1}, to={2,2}, text=$\eta_Y$, y distance=-.25cm, color=rgb{.8 .1 .1}}
                \diagarrow{from={1,1}, to={2,1}, text=$F(f)$, x distance=-.5cm, color=rgb{.8 .1 .1}}
                \diagarrow{from={1,2}, to={2,2}, text=$G(f)$, x distance=.5cm, color=rgb{.8 .1 .1}}
            }}
    \eenum

\edefn

So for every pointed topology $(X,a)$ we defined a group homomorphism $F_{X,a}\colon\pi_1(X,a)\longto H_1(X)$.
We claim that this is a natural transformation from $\pi_1$ to $H_1$.

Suppose there is a morphism $h\colon(X,a)\longto(Y,b)$, so we need the following diagram to commute:

\bigskip
\centerline{\def\diagrowheight{1cm}\drawdiagram{
    $\pi_1(X,a)$ & $H_1(X)$\cr
    $\pi_1(Y,b)$ & $H_1(Y)$\cr
}{
    \diagarrow{from={1,1}, to={1,2}, text=$F_{X,a}$, y distance=.25cm}
    \diagarrow{from={2,1}, to={2,2}, text=$F_{Y,b}$, y distance=-.25cm}
    \diagarrow{from={1,1}, to={2,1}, text=$\pi_1(h)$, x distance=-.5cm}
    \diagarrow{from={1,2}, to={2,2}, text=$H_1(h)$, x distance=.5cm}
}}

This is indeed the case:
$$ \gen\phi \xvarrightarrow{\;F_{X,a}\;} [\phi] \xvarrightarrow{\;H_1(h)\;} [h\circ\phi],\qquad \gen\phi \xvarrightarrow{\;\pi_1(h)\;} \gen{h\circ\phi} \xvarrightarrow{\;F_{Y,b}\;} [h\circ\phi] $$

\bexam

    If we look at the identity functor (on the category of groups) and Abelianization, then $\rho_\bullet$, which is the quotient map $\bullet\longto\Ab\bullet$, is a natural transformation.
    Indeed

    \bigskip
    \centerline{\def\diagrowheight{1cm}\drawdiagram{
        $G$ & $\Ab G$\cr
        $H$ & $\Ab H$\cr
    }{
        \diagarrow{from={1,1}, to={1,2}, text=$\rho_G$, y distance=.25cm, color=rgb{.1 .65 .1}}
        \diagarrow{from={2,1}, to={2,2}, text=$\rho_H$, y distance=-.25cm, color=rgb{.1 .65 .1}}
        \diagarrow{from={1,1}, to={2,1}, text=$\phi$, x distance=-.25cm, color=rgb{.1 .65 .1}}
        \diagarrow{from={1,2}, to={2,2}, text=$\hat\phi$, x distance=.25cm, color=rgb{.1 .65 .1}}
    }}

    Where $\hat\phi[g]=[\phi(g)]$.
    This is indeed natural:
    $$ \rho_H\circ\phi(g) = [\phi(g)],\qquad \hat\phi\circ\rho_G(g)=\hat\phi[g] = [\phi(g)] $$

\eexam

\bdefn

    The {\emphcolor simplified singular chain complex} of a topological space $X$ is the chain complex

    \bigskip
    \centerline{\drawdiagram{
        $\cdots$ & $C_{n+1}(X)$ & $C_n(X)$ & $C_{n-1}(X)$ & $\cdots$ & $C_0(X)$ & ${\bb Z}$ & $0$\cr
    }{
        \diagarrow{from={1,1}, to={1,2}, dashed, color=rgb{.8 .1 .1}}
        \diagarrow{from={1,2}, to={1,3}, text=$\partial_{n+1}$, y distance=.25cm, color=rgb{.8 .1 .1}}
        \diagarrow{from={1,3}, to={1,4}, text=$\partial_{n}$, y distance=.25cm, color=rgb{.8 .1 .1}}
        \diagarrow{from={1,4}, to={1,5}, dashed, color=rgb{.8 .1 .1}}
        \diagarrow{from={1,5}, to={1,6}, dashed, color=rgb{.8 .1 .1}}
        \diagarrow{from={1,6}, to={1,7}, text=$\epsilon$, y distance=.25cm, color=rgb{.8 .1 .1}}
        \diagarrow{from={1,7}, to={1,8}, color=rgb{.8 .1 .1}}
    }}
    \medskip

    Where we define $\epsilon$ as follows:
    $$ \epsilon\sum n_ip_i = \sum n_i $$
    i.e. $\epsilon p=1$ for every $p\in X$.
    And a morphism between two simplified singular chain complexes differ only from morphisms between normal singular chain complexes in that the map from ${\bb Z}$ to ${\bb Z}$ is the identity.

    The homology induced by a simplified singular chain complex is called the {\emphcolor reduced homology} and denoted $\tilde H_n(X)$.

\edefn

Obviously for every $n\geq1$, $\tilde H_n(X)=H_n(X)$.
Recall that if $X$ is path-connected, then $B_0(X)$ is generated by $a-b$ for $a,b\in X$, so it is $\set{\sum n_ip_i}[\sum n_i=0]$.
Now $\ker\epsilon=\set{\sum n_ip_i}[\sum n_i=0]$ as well, and so we get that when $X$ is path-connected, $\tilde H_0(X)=0$.

\bdefn

    A chain of Abelian groups

    \bigskip
    \centerline{\drawdiagram{
        $\cdots$ & $A$ & $B$ & $C$ & $\cdots$\cr
    }{
        \diagarrow{from={1,1}, to={1,2}, dashed, color=rgb{.8 .1 .1}}
        \diagarrow{from={1,2}, to={1,3}, text=$f$, y distance=.25cm, color=rgb{.8 .1 .1}}
        \diagarrow{from={1,3}, to={1,4}, text=$g$, y distance=.25cm, color=rgb{.8 .1 .1}}
        \diagarrow{from={1,4}, to={1,5}, dashed, color=rgb{.8 .1 .1}}
    }}
    \bigskip

    is {\emphcolor exact} at $B$ if ${\rm Im}f=\ker g$.
    If the sequence is exact at every group, then the sequence itself is called an {\emphcolor exact sequence}.
    (Recall that chain complexes require ${\rm Im}f\subseteq\ker g$.)

\edefn

If we have an exact sequence in one of the following forms, then:

\benum
    \item $0\longto A\xvarrightarrow{\;f\;}B$, then $0=\ker f$ so $f$ is injective.
    \item $A\xvarrightarrow{\;f\;}B\longto0$, then ${\rm Im}f=B$ so $f$ is surjective.
    \item $0\longto A\xvarrightarrow{\;f\;}B\longto0$, then $f$ is an isomorphism.
\eenum

\bdefn

    A {\emphcolor short exact sequence} is an exact sequence of the form
    $$ 0\longto A\xvarrightarrow{\;f\;}B\xvarrightarrow{\;g\;}C\longto0 $$

\edefn

In a short exact sequence, by above $f$ is injective and $g$ is surjective, and furthermore ${\rm Im}f=\ker g$.
In such a case, we can view $A$ as being a subgroup of $B$ (since $f$ is an embedding) and since by the isomorphism theorem $C\cong\slfrac B{\ker g}=\slfrac B{{\rm Im}f}=\slfrac BA$, a short exact sequence
can be viewed as
$$ 0 \longto A\xvarrightarrow{\;{\it inclusion}\;}B\xvarrightarrow{\;{\it quotient}\;}\slfrac BA\longto0 $$

\localcolor{white}{This next lemma took fucking forever to \TeX{} out, so you'd better appreciate me!}

\blemm[title=The Lemma of Five]

    Suppose the chains $\set{A_i}_i,\set{B_i}_i$ are exact, and the following diagram commutes:

    \medskip
    \centerline{\def\diagrowheight{1cm}\def\diagcolwidth{1cm}\drawdiagram{
        $A_1$ & $A_2$ & $A_3$ & $A_4$ & $A_5$\cr
        $B_1$ & $B_2$ & $B_3$ & $B_4$ & $B_5$\cr
    }{
        \diagarrow{from={1,1}, to={1,2}, color=rgb{.8 .1 .8}}
        \diagarrow{from={1,2}, to={1,3}, color=rgb{.8 .1 .8}}
        \diagarrow{from={1,3}, to={1,4}, color=rgb{.8 .1 .8}}
        \diagarrow{from={1,4}, to={1,5}, color=rgb{.8 .1 .8}}
        \diagarrow{from={2,1}, to={2,2}, color=rgb{.8 .1 .8}}
        \diagarrow{from={2,2}, to={2,3}, color=rgb{.8 .1 .8}}
        \diagarrow{from={2,3}, to={2,4}, color=rgb{.8 .1 .8}}
        \diagarrow{from={2,4}, to={2,5}, color=rgb{.8 .1 .8}}
        \diagarrow{from={1,1}, to={2,1}, color=rgb{.8 .1 .8}, text=$f_1$, x distance=-.25cm}
        \diagarrow{from={1,2}, to={2,2}, color=rgb{.8 .1 .8}, text=$f_2$, x distance=-.25cm}
        \diagarrow{from={1,3}, to={2,3}, color=rgb{.8 .1 .8}, text=$f_3$, x distance=-.25cm}
        \diagarrow{from={1,4}, to={2,4}, color=rgb{.8 .1 .8}, text=$f_4$, x distance=-.25cm}
        \diagarrow{from={1,5}, to={2,5}, color=rgb{.8 .1 .8}, text=$f_5$, x distance=-.25cm}
    }}

    \kern-.5cm
    \localcolor rgb{1 .9 1}{That actually wasn't too bad :)}

    \benum
        \item If $f_2,f_4$ are injective and $f_1$ is surjective, then $f_3$ is injective.
        \item If $f_2,f_4$ are surjective and $f_5$ is injective, then $f_3$ is surjective.
    \eenum

\elemm

\Proof We write $x\xvarmapsto Ay$ to mean $x$ maps to $y$ in the exact sequence ($x\in A_i$).
\benum
    \item Suppose $f_3a$, then $a\xmaps{f_3}0\xmaps{B}0$, now suppose $a\xmaps{A}b\xmaps{f_4}c$.
        Since the diagram commutes, we must have that $c=0$, but $f_4$ is injective so $b=0$.
        This means $a\in\ker A$, so there exists some $d$ such that $d\xmaps{A}a$.
        Suppose $d\xmaps{f_2}e$, then $e\xmaps{B}0$ by commutativity, so there exists an $f$ such that $f\xmaps{B}e$, and since $f_1$ is surjective there exists a $g\xmaps{f_1}f$.
        Now suppose $g\xmaps{A}h$.
        By commutativity, since $g\xmaps{f_1}f\xmaps{B}e$ we have $f_2h=e$ and since $f_2$ is injective, $h=d$.
        So $d$ is in the image of $A$, so it is in the kernel and so $a=0$.
    \item is a little more complicated, but it's just chasing.
        \qed
\eenum

\bdefn

    Suppose $\C$ and $\D$ are two chain complexes, with two morphisms $f,g\colon\C\longto\D$.
    Then a {\emphcolor chain homotopy} from $f$ to $g$ is a sequence of maps $T_n\colon C_n\longto D_{n+1}$ such that $\partial T+T\partial=f-g$.
    If there exists a chain homotopy between $f$ and $g$, we write $f\hch g$.

\edefn

In a diagram, we have that $T$ are the red arrows.

\bigskip
\centerline{\def\diagrowbuf{1cm}\def\diagcolbuf{1cm}\drawdiagram{
    $\cdots$ & $C_{n+1}$ & $C_n$ & $C_{n-1}$ & $\cdots$\cr
    $\cdots$ & $D_{n+1}$ & $D_n$ & $D_{n-1}$ & $\cdots$\cr
}{
    \diagarrow{from={1,1}, to={1,2}, dashed}
    \diagarrow{from={1,2}, to={1,3}}
    \diagarrow{from={1,3}, to={1,4}}
    \diagarrow{from={1,4}, to={1,5}, dashed}
    \diagarrow{from={2,1}, to={2,2}, dashed}
    \diagarrow{from={2,2}, to={2,3}}
    \diagarrow{from={2,3}, to={2,4}}
    \diagarrow{from={2,4}, to={2,5}, dashed}
    \diagarrow{from={1,2}, to={2,2}, text=$f_{n+1}$, x distance=-.75cm, x off=-.25cm}
    \diagarrow{from={1,3}, to={2,3}, text=$f_{n}$, x distance=-.5cm, x off=-.25cm}
    \diagarrow{from={1,4}, to={2,4}, text=$f_{n-1}$, x distance=-.75cm, x off=-.25cm}
    \diagarrow{from={1,2}, to={2,2}, text=$g_{n+1}$, x distance=.5cm}
    \diagarrow{from={1,3}, to={2,3}, text=$g_{n}$, x distance=.25cm}
    \diagarrow{from={1,4}, to={2,4}, text=$g_{n-1}$, x distance=.5cm}
    \diagarrow{from={1,2}, to={2,1}, dashed, color=red}
    \diagarrow{from={1,3}, to={2,2}, color=red}
    \diagarrow{from={1,4}, to={2,3}, color=red}
}}
\medskip

Let $X\subseteq{\bb R}^k$ be convex.
For $a\in X$ let us define the {\it cone construction} $C_a\colon C_n(X)\longto C_{n+1}(X)$ as follows: we start with generators of $C_n(X)$, i.e. we define $C_a\sigma$ for $\sigma\colon\Delta^n\longto X$
an $n$-simplex.
Geometrically, $C_a\sigma$ will be a cone whose tip is $a$ and whose base is $\sigma$.
We define this by:
$$ C_a\sigma(t_0,\dots,t_{n+1}) = t_0b + (1-t_0)\sigma\parens{\frac{t_1}{1-t_0},\dots,\frac{t_{n+1}}{1-t_0}} $$
Let us now compute the faces of $C_a\sigma$.
For $i=0$ then
$$ (C_a\sigma)\tau_0^{n+1}(t_0,\dots,t_n) = C_a\sigma(0,t_0,\dots,t_n) = \sigma(t_0,\dots,t_n) $$
For $i>0$ then
$$ (C_a\sigma)\tau_i^{n+1}(t_0,\dots,t_n) = C_a\sigma(t_0,\dots,0,\dots,t_n) $$
if $t_0=1$ as well, then this is just
$$ C_a\sigma(1,0,\dots,0) = a $$
Otherwise,
$$ \eqalign{
    &= t_0b + (1-t_0)\sigma\parens{\frac{t_1}{1-t_0},\dots,0,\dots,\frac{t_n}{1-t_0}}\cr
    &= t_0b + (1-t_0)\sigma\tau^n_{i-1}\parens{\frac{t_1}{1-t_0},\dots,\frac{t_n}{1-t_0}}\cr
    &= C_a^{n-1}(\sigma\tau^n_{i-1})(t_0,\dots,t_n)
} $$
So we see that
$$ (C_a\sigma)\tau_0^{n+1} = \sigma,\qquad (C_a\sigma)\tau_i^{n+1} = C_a^{n-1}(\sigma\tau^n_{i-1}) $$
So
$$ \eqalign{
    \partial_{n+1} C_a^n(\sigma) = \sum_{i=0}^{n+1}(-1)^i(C_a^n\sigma)\tau_i^{n+1} &= \sigma + \sum_{i=1}^{n+1}C_a^{n-1}(\sigma\tau^n_{i-1})\cr
    &= \sigma - \sum_{i=0}^n(-1)^iC_a^{n-1}(\sigma\tau^n_i)\cr
    &= \sigma - C_a^{n-1}\parens{\sum_{i=0}^n(-1)^i\sigma\tau^n_i}\cr
    &= \sigma - C_a^{n-1}\partial_n\sigma
} $$
So we see that
$$ \partial C_a - C_a\partial = {\rm id} $$
so in other words, $C_a$ is a chain homotopy from ${\rm id}$ to $0$.

\bthrm

    Let $X$ be a convex set in ${\bb R}^k$, then for all $n>0$, $H_n(X)=0$.

\ethrm

\Proof let $\gamma\in C_n(X)$, then $\gamma=\partial C_a\gamma + C_a\partial\gamma$.
If $\gamma\in Z_n(X)$, i.e. it is a cycle, then $\partial\gamma=0$ and so $\gamma=\partial C_a\gamma$.
Thus $\gamma\in B_n(X)$, so $Z_n(X)=B_n(X)$, and then $H_n(X)=0$.
\qed

\blemm

    If $f,g\colon X\longto Y$ are two homotopic continuous maps, then $f_\sharp$ and $g_\sharp$ are chain homotopic.

\elemm

\Proof let us define $i,j\colon X\longto X\times I$ where $i(x)=(x,0)$ and $j(x)=(x,1)$.
If $H\colon X\times I\longto Y$ is a homotopy from $f$ to $g$, then $f=H\circ i$ and $g=H\circ j$.
Also $i\sim j$, so if we can show that $i_\sharp\hch j_\sharp$ then we have that
$$ f_\sharp = H_\sharp\circ i_\sharp \hch H_\sharp\circ j_\sharp = g_\sharp $$
so it is sufficient to show that $i_\sharp\hch j_\sharp$.

So we need to define a sequence of morphisms $T_n^X\colon C_n(X)\longto C_{n+1}(X\times I)$ such that $\partial T^X+T^X\partial=i^X_\sharp-j^X_\sharp$.
We will define $T^X_n$ by induction on $n$, such that $T^X$ is natural.
Natural between what two functors?
The first functor maps topological spaces $X$ to their chain complexes $\C(X)$ and maps morphisms $X\xvarrightarrow{\,f\,}Y$ to $f_\sharp\colon\C(X)\longto\C(Y)$.
The second functor maps topological spaces $X$ to the chain complex $C_{n+1}(X\times I)$ and morphisms $X\xvarrightarrow{\,f\,}Y$ to $(f\times{\rm id})_\sharp\colon C_{n+1}(X)\longto C_{n+1}(Y)$.

$T^X$ being natural means that the diagram commutes for all $f\colon X\longto Y$:

\bigskip
\centerline{\def\diagrowbuf{.5cm}\def\diagcolbuf{.5cm}
\drawdiagram{
    $C_n(X)$ & $C_{n+1}(X\times I)$\cr
    $C_n(Y)$ & $C_{n+1}(Y\times I)$\cr
}{
    \diagarrow{from={1,1}, to={1,2}, text=$T^X$, y distance=.25cm}
    \diagarrow{from={2,1}, to={2,2}, text=$T^Y$, y distance=-.25cm}
    \diagarrow{from={1,1}, to={2,1}, text=$f_\sharp$, x distance=-.25cm}
    \diagarrow{from={1,2}, to={2,2}, text=$(f\times{\rm id})_\sharp$, x distance=.75cm}
}}
\medskip

So $T_Y\circ f_\sharp=(f\times{\rm id})_\sharp\circ T_X$.

Let $I_n\colon\Delta^n\longto\Delta^n$ be the identity $n$-dimensional simplex.
If we determine $T^{\Delta^n}(I_n)$, then we have determined $T^X(\sigma)$ for all $\sigma\in C_n(X)$ for all $X$.
This is because $\sigma=\sigma\circ I_n=\sigma_\sharp(I_n)$, since we can view $\sigma$ as a continuous map $X\longto\Delta^n$ and so $\sigma_\sharp$ is defined.
Thus
$$ T^X(\sigma) = T^X\circ\sigma_\sharp(I_n) = (\sigma\times{\rm id})_\sharp\circ T^{\Delta^n}(I_n) $$
And so determining $T^{\Delta^n}(I_n)$ determines $T^X(\sigma)$.
So if we define $A=T^{\Delta^n}(I_n)$, then
$$ T^X(\sigma)=(\sigma\times{\rm id})_\sharp(A) $$
$A$ is some simplex in $C_{n+1}(\Delta^n\times I)$, and we claim that for any choice of $A$, this defines a natural transformation.

This is because
$$ T^Y\circ f_\sharp(\sigma) = T^Y(f\circ\sigma) = \bigl((f\circ\sigma)\times{\rm id}\bigr)_\sharp(A) = (f\times{\rm id})_\sharp\circ(\sigma\times{\rm id})_\sharp(A) $$
And
$$ (f\times{\rm id})_\sharp\circ T^X(\sigma) = (f\times{\rm id})_\sharp\circ(\sigma\times{\rm id})_\sharp(A) $$
so these are indeed equal, as required.

Now we claim that
$$ (\partial T^X + T^X\partial)(\sigma) = (i^X_\sharp - j^X_\sharp)(\sigma) $$
for all $X,\sigma$.
It is sufficient to show this for $X=\Delta^n$ and $\sigma=I_n$, since if
$$ (\partial T^{\Delta^n} + T^{\Delta^n}\partial)(I_n) = (i^{\Delta^n}_\sharp - j^{\Delta^n}_\sharp)(I_n) $$
if we compose it on the left with $(\sigma\times{\rm id})_\sharp$, the LHS gives
$$ (\partial(\sigma\times{\rm id})_\sharp T^{\Delta^n} + (\sigma\times{\rm id})_\sharp T^{\Delta^n}\partial)(I_n) = (\partial T^X\sigma_\sharp + T^X\partial\sigma_\sharp)(I_n)
= \partial T^X\sigma + T^X\partial\sigma $$
since $T$ is natural, $T^Y\circ f_\sharp=(f\times{\rm id})_\sharp\circ T^X$ and $\partial f_\sharp=f_\sharp\partial$.
The RHS is
$$ \bigl((\sigma\times{\rm id})_\sharp\circ i^{\Delta^n}_\sharp - (\sigma\circ{\rm id})\circ j_\sharp^{\Delta^n}\bigr)(I_n) $$
Now notice that

\centerline{
\drawdiagram{
    $\Delta^n$ & $\Delta^n\times I$ & $X\times I$\cr
    $s$ & $(s,0)$ & $(\sigma(s),0)$\cr
}{
    \diagarrow{from={1,1}, to={1,2}, text=$i^{\Delta^n}$, y distance=.25cm}
    \diagarrow{from={1,2}, to={1,3}, text=$\sigma\times{\rm id}$, y distance=.25cm}
    \diagarrow{from={2,1}, to={2,2}, left cap=|-}
    \diagarrow{from={2,2}, to={2,3}, left cap=|-}
}}

So $(\sigma\times{\rm id})\circ i^{\Delta^n}=i^X\circ\sigma$, and similar for $j$.
So the RHS is just
$$ i^X_\sharp\circ\sigma_\sharp(I_n) - j^X_\sharp\circ\sigma_\sharp(I_n) = i^X_\sharp(\sigma) - j^X_\sharp(\sigma) $$
So we get
$$ \partial T^X(\sigma) + T^X\partial\sigma = i^X_\sharp(\sigma) - j^X_\sharp(\sigma) $$
as required.

So we must show that
$$ \partial TI_n + T\partial I_n = i_\sharp I_n - j_\sharp I_n $$
in order to get this for every $\sigma\in C_n(\Delta^n)$.
So we must show $\partial TI_n=-T\partial I_n+i_\sharp-j_\sharp I_n$, since $\partial TI_n\in C_n(\Delta^n\times I)$, and $\Delta^n\times I$ is a convex set in ${\bb R}^{n+2}$.
In a convex set so a simplex is a boundary if and only if it is a cycle.
We want $-T\partial I_n+i_\sharp I_n-j_\sharp I_n$ to be a boundary, and so it is sufficient to check that it is a cycle:
$$ -\partial T\partial I_n + \partial i_\sharp I_n - \partial j_\sharp I_n $$
Since $\partial I_n\in C_{n-1}(\Delta^n)$, we have that
$$ \partial T\partial I_n + T\partial\partial I_n = i_\sharp\partial I_n - j_\sharp\partial I_n $$
and thus we must have that the following is zero:
$$ T\partial\partial I_n - i_\sharp\partial I_n + j_\sharp\partial I_n + \partial i_\sharp I_n - \partial j_\sharp I_n $$
Since $\partial\partial=0$, and $i_\sharp,j_\sharp$ are chain homomorphisms, this is indeed zero.
So $-T\partial I_n+i_\sharp I_n-j_\sharp I_n$ is a cycle and thus a boundary since the universe is convex.
So let us take $A$ to be a chain such that $\partial A$ is this element.
\qed

So notice now that if $f\sim g$, then $f_\sharp\sim g_\sharp$ are chain homotopic, and so $f_*=g_*$.

\bcoro

    If $f\colon X\longto Y$ is a homotopy equivalence, then $f_*\colon H_n(X)\longto H_n(Y)$ is an isomorphism.

\ecoro

\Proof there exists a $g\colon Y\longto X$ such that $fg\sim{\rm id}_Y$ and $gf\sim{\rm id}_X$.
Thus
$$ g_*\circ f_* = (g\circ f)_* = ({\rm id}_X)_* = {\rm id}_{H_n(X)} $$
and similarly $f_*\circ g_*={\rm id}_{H_n(Y)}$, so $f_*$ is an isomorphism.
\qed

\subsection{Mayer-Vietoris}

\bdefn

    Let $p_1,\dots,p_n$ be vectors in a vector space, then their {\emphcolor affine hull} is
    $$ \AHof{p_1,\dots,p_n} = \set{\sum_{i=1}^n\alpha_ip_i}[\sum_{i=1}^n\alpha_i=1] $$
    Elements of the affine hull are called {\emphcolor affine combinations}.
    We similarly define the {\emphcolor convex hull}:
    $$ \CHof{p_1,\dots,p_n} = \set{\sum_{i=1}^n\alpha_ip_i}[\sum_{i=1}^n\alpha_i=1,\,\alpha_i\geq0] $$
    And its elements are called {\emphcolor convex combinations}.

\edefn

\bdefn

    $p_1,\dots,p_n$ are {\emphcolor affine independent} if $\sum_{i=1}^n\alpha_iv_i=0$ and $\sum_{i=1}^n\alpha_i=0$ implies every $\alpha_i$ is $0$.

\edefn

\bdefn

    $A\subseteq{\bb R}^k$ is an {\emphcolor $n$-simplex} if it is the convex hull of a set of $n+1$ affine independent set of vectors.

\edefn

\bdefn

    Let $\Sigma=\CHof{p_0,\dots,p_n}$ be an $n$-simplex, then its $i$th {\emphcolor face} is $\CHof{p_0,\dots,p_{i-1},p_i,\dots,p_n}$.
    And its {\emphcolor barycenter} is
    $$ b = \frac1{n+1}\sum_{i=0}^np_i $$
    We define the {\emphcolor barycentric subdivision} of $\Sigma$, denoted $\Sd\Sigma$, to be a set of $n$-simplices which we define inductively on $n$ as follows:
    \benum
        \item For a $0$-simplex, $\Sd\Sigma=\Sigma$.
        \item If $\Sigma$ is an $n$-simplex, then let $\phi_0,\dots,\phi_n$ be its faces (which are $n-1$-simplices) and $b$ its barycenter.
            Then define $\Sd\Sigma$ to be the $n$-simplices spanned by $b$ and the simplices in the barycentric subdivisions of $\phi_i$.
            I.e.
            $$ \Sd\Sigma = \set{\CHof{b,\Sigma^{n-1}}}[\Sigma^{n-1}\in\Sd\phi_i,0\leq i\leq n] $$
    \eenum

\edefn

Inductively, $\Sigma=\bigcup\Sd\Sigma$ and $\#\Sd\Sigma=(n+1)!$.

\bthrm

    For every $n$, there exists a constant $c<1$ such that for every $n$-simplex $\Sigma$ then for every $\Sigma'\in\Sd\Sigma$:
    $$ \diamof{\Sigma'} \leq c\diamof\Sigma $$

\ethrm

\bdefn

    We define $\Sd_n\colon C_n(\Delta^n)\longto C_n(\Delta^n)$ by induction on $n$.
    Let $\sigma\colon\Delta^n\longto\Delta^n$ be a generator, then
    \benum
        \item $\Sd_0(\sigma)=\sigma$
        \item $\Sd_n(\sigma)=C_{\sigma(b)}(\Sd_{n-1}(\partial\sigma))$ where $b$ is the barycenter of $\Delta^n$.
    \eenum
    Let $X$ be a topological space, then let $\Sd_n\colon C_n(X)\longto C_n(X)$ be defined on generators $\sigma\colon\Delta^n\longto X$ by $\Sd\sigma=\sigma_\sharp\Sd_n{\rm id}_{\Delta^n}$.

\edefn

\bthrm

    $\Sd$ is a chain map ($\Sd=\set{\Sd_n}_{n=0}^\infty$) and is natural (between the chain functor ${\bf Top}\to{\bf Comp}$ and itself).

\ethrm

$\Sd$ being natural means the following diagram commutes

\bigskip
\centerline{\def\diagrowheight{1cm}\drawdiagram{
    $C_n(X)$ & $C_n(X)$\cr
    $C_n(Y)$ & $C_n(Y)$\cr
}{
    \diagarrow{from={1,1}, to={1,2}, text=$\Sd_n$, y distance=.25cm}
    \diagarrow{from={2,1}, to={2,2}, text=$\Sd_n$, y distance=-.25cm}
    \diagarrow{from={1,1}, to={2,1}, text=$f_\sharp$, x distance=-.25cm}
    \diagarrow{from={1,2}, to={2,2}, text=$f_\sharp$, x distance=.25cm}
}}

\bthrm

    $\Sd$ is chain homotopic to ${\rm id}_{\C(X)}$.

\ethrm

\bdefn

    Let $X$ be a topological space, and $\U=\set{\U_\alpha}_{\alpha\in I}$ a collection of subsets of $X$ such that $\bigcup\Dot\U_\alpha=X$ (where $\Dot\U$ is the interior of $\U$).
    Such a collection will be called a {\emphcolor good cover} of $X$.

    We will say that $\sigma\colon\Delta^n\longto X$ {\emphcolor preserves} the cover if there exists an $\alpha\in I$ such that $\sigma(\Delta^n)\subseteq\U_\alpha$.
    And we will say that $\sum_in_i\sigma_i\in C_n(X)$ {\emphcolor preserves} the cover if each $\sigma_i$ preserves the cover.

    Let us define
    $$ C_n^\U(X) = \set{\sigma\in C_n(X)}[\sigma\hbox{ preserves $\U$}] $$

\edefn

$C_n^\U(X)$ is a subgroup of $C_n(X)$, as can be easily verified.
Notice that if $\sigma(\Delta^n)\subseteq\U_\alpha$ then $\sigma\tau_i(\Delta^{n-1})=\sigma(\tau_i\Delta^{n-1})\subseteq\U_\alpha$ so that $\sigma\tau_i\in C_{n-1}^\U(X)$.
Thus $\partial\sigma\in C_{n-1}^\U(X)$, so we can define a subcomplex of $\C(X)$, $\C^\U(X)$ whose coefficients are $C_n^\U(X)$.
So we can define $H_n^\U(X)$ to be the $n$th homology group of $\C^\U(X)$.

The inclusion map $\iota\colon C_n^\U(X)\longto C_n(X)$ is a chain morphism, so this induces a $\iota_*\colon H_n^\U(X)\longto H_n(X)$.

\bthrm

    This $\iota_*$ is an isomorphism.

\ethrm

This is not a trivial proof, and it relies on the following observations.
But from here on, I will only be putting in the simpler/enlightening proofs so that I can finish this summary.
Notice that
$$ \Sd_n\colon C_n^\U(X) \longto C_n^\U(X) $$
is defined, since if $\sigma\in C_n^\U(X)$ then that means for some $\alpha\in I$ $\sigma(\Delta^n)\subseteq\U_\alpha$, and $\Sd_n\sigma=\sigma_\sharp\Sd_n{\rm id}_n$.
Thus the image of $\Sd_n\sigma$ is contained in the image of $\sigma$, which in turn is contained in $\U_\alpha$.
Now, the chain homotopy between $\Sd$ and ${\rm id}_{\C(X)}$ can also be restricted to $\C^\U(X)\longto\C^\U(X)$.
Thus $\Sd$ is chain homotopic to ${\rm id}_{\C^\U(X)}$.

\bdefn

    A {\emphcolor short exact sequence} of chain complexes is a chain of chain morphisms $\C\xto{f}\D\xto{g}E$ such that for every $n$, $0\to C_n\xto{f_n}D_n\xto{g_n}E_n\to0$ is a short exact sequence.

\edefn

\blemm

    A short exact sequence of chain complexes $\C\xto{f}\D\xto{g}\E$ induces a long exact sequence on the homology groups:

    \centerline{\drawdiagram{
        && $H_{n+1}\E$\cr
        $H_n\C$ & $H_n\D$ & $H_n\E$\cr
        $H_{n-1}\C$ &&\cr
    }{
        \diagarrow{from={1,1}, to={1,3}, dashed, color=rgb{.8 .1 .8}}
        \diagarrow{from={1,3}, to={2,1}, color=rgb{.8 .1 .8}}
        \diagarrow{from={2,1}, to={2,2}, color=rgb{.8 .1 .8}}
        \diagarrow{from={2,2}, to={2,3}, color=rgb{.8 .1 .8}}
        \diagarrow{from={2,3}, to={3,1}, color=rgb{.8 .1 .8}}
        \diagarrow{from={3,1}, to={3,3}, dashed, color=rgb{.8 .1 .8}}
    }}

\elemm

\Proof a diagram chase.\qed

\bdefn

    If $\C,\D$ are chain complexes then their {\emphcolor direct sum} is the chain complex $\C\oplus\D$ whose terms are $C_n\oplus D_n$ and whose boundary operator is $\partial_\C\oplus\partial_\D$ (i.e.
    $(a,b)\mapsto(\partial a,\partial b)$.

\edefn

\blemm

    If $X$ is a topological space, $\U,V\subseteq X$ such that $\Dot\U\cup\Dot\V=X$, then there exists a short exact sequence of chain complexes
    $$ 0 \longto \C(\U\cap\V) \longto \C(\U) \oplus \C(\V) \longto \C^{\U,\V}(X) \longto 0 $$
    where $\C^{\U,\V}(X)$ is the chain complex modulo the cover $\set{\U,\V}$.

\elemm

\Proof we have the inclusions, which commute:

\centerline{\drawdiagram{
    & $\U$\cr
    $\U\cap\V$ & & $X$\cr
    & $\V$\cr
}{
    \diagarrow{from={2,1}, to={1,2}, text=$i$, y distance=.25cm}
    \diagarrow{from={2,1}, to={3,2}, text=$j$, y distance=-.25cm}
    \diagarrow{from={1,2}, to={2,3}, text=$k$, y distance=.25cm}
    \diagarrow{from={3,2}, to={2,3}, text=$\ell$, y distance=-.25cm}
}}

And from them we build:

\centerline{\drawdiagram{
    $0$ & $C_n(\U\cap\V)$ & $C_n(\U)\oplus C_n(\V)$ & $C_n^{\U,\V}(X)$ & $0$\cr
    & $a$ & $(i_\sharp a,-j_\sharp a)$\cr
    && $(a,b)$ & $k_\sharp a+\ell_\sharp b$\cr
}{
    \diagarrow{from={1,1}, to={1,2}}
    \diagarrow{from={1,2}, to={1,3}}
    \diagarrow{from={1,3}, to={1,4}}
    \diagarrow{from={1,4}, to={1,5}}
    \diagarrow{from={2,2}, to={2,3}, left cap=|-}
    \diagarrow{from={3,3}, to={3,4}, left cap=|-}
}}

This is exact because composing the two maps gives $k_\sharp i_\sharp a-\ell_\sharp j_\sharp a=(ki)_\sharp a-(\ell j)_\sharp a$, and since $ki=\ell j$, this is zero.
So the image of the first is contained within the kernel of the second.
And if $k_\sharp a=-\ell_\sharp b$, then $a,b$ must be chains in $\U\cap\V$ (since $k$ maps chains of $\U$ to $X$, and $\ell$ maps chains of $\V$), so they must be in the image of the first map.
It can be verified that these are chain morphisms.
\qed

Notice that the homology group of $C_n(X)\oplus C_n(Y)$ is just $H_n(X)\oplus H_n(Y)$ since the image of $\partial\oplus\partial$ is just $\Im\partial\oplus\Im\partial$, and similar for kernel.
From the previous two lemmas, the following is immediate (recall that $H^\U_n(X)\cong H_n(X)$):

\bthrm[title=Mayer-Vietoris]

    If $\U,\V\subseteq X$ such that $\Dot\U\cup\Dot\V=X$, then there is an exact sequence

    \centerline{\drawdiagram{
        && $H_{n+1}(X)$\cr
        $H_n(\U\cap\V)$ & $H_n(\U)\oplus H_n(\V)$ & $H_n(X)$\cr
        $H_{n-1}(\U\cap\V)$ &&\cr
    }{
        \diagarrow{from={1,1}, to={1,3}, dashed, color=rgb{.1 .1 .8}}
        \diagarrow{from={1,3}, to={2,1}, color=rgb{.1 .1 .8}}
        \diagarrow{from={2,1}, to={2,2}, color=rgb{.1 .1 .8}}
        \diagarrow{from={2,2}, to={2,3}, color=rgb{.1 .1 .8}}
        \diagarrow{from={2,3}, to={3,1}, color=rgb{.1 .1 .8}}
        \diagarrow{from={3,1}, to={3,3}, dashed, color=rgb{.1 .1 .8}}
    }}

\ethrm

Notice that at $n=0$ for the reduced homology if $\U\cap\V\neq\varnothing$, then we get the same exact sequence but with the reduced homology.

\bthrm

    $$ \tilde H_i(S^n) = \cases{{\bb Z} & $i=n$\cr 0 & else} $$

\ethrm

\Proof by induction on $n$.
For $n=0$, we have that $S^0$ is just the space of two points, so $H_0(S^0)={\bb Z}\oplus{\bb Z}$ and for $i>0$ it is zero since the homology of the one-point space is zero.
The reduced homology removes a factor of ${\bb Z}$ and so $\tilde H_0(S^0)={\bb Z}$ and for $n>0$ $\tilde H_i(S^0)=0$.
Now inductively, we can choose contractible $\U,\V$ such that $\U\cap\V$ are homotopic to $S^{n-1}$ (by choosing hemispheres which overlap), and so $H_i(\U\cap\V)\cong H_i(S^{n-1})$.
We have an exact sequence by Mayer-Vietoris:
$$ \tilde H_i(\U)\oplus\tilde H_i(\V) \longto \tilde H_i(S^n) \longto \tilde H_{i-1}(S^{n-1}) \longto \tilde H_{i-1}(\U) \oplus \tilde H_{i-1}(\V) $$
since $\U,\V$ are contractible, $\tilde H_i(\U)=\tilde H_i(\V)=0$ for all $i$ and so we get the exact sequence
$$ 0 \longto \tilde H_i(S^n) \longto \tilde H_{i-1}(S^{n-1}) \longto 0 $$
which means that $\tilde H_i(S^n)\cong\tilde H_{i-1}(S^{n-1})$, and so inductively we have our result.
\qed

Since their homology groups differ, we immediately get

\bthrm

    If $n\neq m$ then $S^n$ is not homotopic to $S^m$.

\ethrm

\bcoro

    If $n\neq m$ then ${\bb R}^n$ is not homeomorphic to ${\bb R}^m$.

\ecoro

\Proof suppose $f\colon{\bb R}^n\longto{\bb R}^m$ is a homeomorphism, then it is a homeomorphism $f\colon{\bb R}^n\setminus\set0\longto{\bb R}^m\setminus\set{f(0)}$.
So we have
$$ S^n \simeq {\bb R}^n\setminus\set0 \cong {\bb R}^m\setminus\set{f(0)} \simeq S^m $$
in contradiction.
\qed

\bthrm

    $\partial D^n$ is not a retract of $D^n$.

\ethrm

\Proof suppose $r\colon D^n\longto\partial D^n$ is a retraction, then $r\iota={\rm id}_{\partial D^n}$ where $\iota$ is the inclusion $\partial D^n\longto D$.
Thus $r_*\iota_*={\rm id}_{H_i(\partial D^n)}$.
This implies that $\iota_*$ is injective, in particular for $i=n-1$ and so $i_*\colon\tilde H_{n-1}(\partial D^n)\longto\tilde H_{n-1}(D^n)$.
Since $\partial D^n\cong S^{n-1}$ and $D^n$ is contractible, we have an injective map ${\bb Z}\longto0$ in contradiction.
\qed

\blemm

    Let us define $R\colon S^n\longto S^n$ by $R(x_1,\dots,x_n)=(-x_1,x_2,\dots,x_n)$.
    Then $R_*\colon\tilde H_n(S^n)\longto\tilde H_n(S^n)$ satisfies $R_*=-{\rm id}_{\tilde H_n(S^n)}$.

\elemm

\Proof by induction on $n$.
For $n=0$, $R(1)=-1$ and $R(-1)=1$, and $\tilde H_0(S^0)={\bb Z}$.
Now, $\epsilon$ must map the generator of the reduced homology to zero, so the generator must be $kp_1-kp_2$, and composing $R_*$ on this gives $kp_2-kp_1$ which is the inverse of the generator, so
$R_*$ is indeed minus the identity.

Now for $n>0$, let us split the sphere $S^n$ into two hemispheres $\U$ and $\V$ whose intersection is homotopic to $S^{n-1}$.
By Mayer-Vietoris, we have

\centerline{\drawdiagram{
    $0$ & $H_n(S^n)$ & $H_{n-1}(S^{n-1})$ & $0$\cr
    $0$ & $H_n(S^n)$ & $H_{n-1}(S^{n-1})$ & $0$\cr
}{
    \diagarrow{from={1,1}, to={1,2}}
    \diagarrow{from={1,2}, to={1,3}}
    \diagarrow{from={1,3}, to={1,4}}
    \diagarrow{from={2,1}, to={2,2}}
    \diagarrow{from={2,2}, to={2,3}}
    \diagarrow{from={2,3}, to={2,4}}
    \diagarrow{from={1,2}, to={2,2}, text=$R_*$, x distance=-.25cm}
    \diagarrow{from={1,3}, to={2,3}, text={$R_*=-{\rm id}$}, x distance=1cm}
}}

This diagram commutes by naturality, so $R_*=-{\rm id}$ for $S^n$.
\qed

By symmetry, we can define $R_i\colon(x_1,\dots,x_i,\dots,x_{n+1})\mapsto(x_1,\dots,-x_i,\dots,x_{n+1})$ and we have that $R_{i,*}=-{\rm id}$.
Let us define
$$ A\colon S^n\longto S^n,\qquad x\mapsto -x $$
the {\it antipodal map}.
Since $A=R_1\circ\cdots\circ R_{n+1}$, we have that $A_*=(-{\rm id})^{n+1}=(-1)^{n+1}{\rm id}$.

\bcoro

    If $n$ is even, then the antipodal map is not homotopic to the identity.

\ecoro

Note that for $n=2k-1$, we can view $S^n$ as the unit sphere in ${\bb C}^k$ and take the homotopy $H(z,t)=e^{\pi it}z$ which is a homotopy from ${\rm id}$ to $A$.

\blemm

    Let $n\geq0$, and let $f,g\colon S^n\longto S^n$ such that for all $x\in S^n$, $f(x)\neq-g(x)$.
    Then $f\sim g$.

\elemm

\Proof we define the homotopy
$$ H(x,t) = \frac{(1-t)f(x)+tg(x)}{\norm{(1-t)f(x)+tg(x)}} $$
this cannot be zero, since the line $(1-t)f(x)+tg(x)$ connects $f(x)$ and $g(x)$, and it can only be zero when $f(x)$ and $g(x)$ are antipodal points on the sphere.
\qed

\bthrm

    Let $n$ be even and $f\colon S^n\longto S^n$, then there exists an $x\in S^n$ such that either $f(x)=x$ or $f(x)=-x$.

\ethrm

\Proof suppose not.
Then for all $x$, $f(x)\neq x$, so $f(x)\neq -A(x)$ so $f\sim A$.
And for all $x$, $f(x)\neq -x$, i.e. $f(x)\neq -{\rm id}(x)$ so $f\sim{\rm id}$.
Thus $A\sim{\rm id}$, which contradicts $n$ being even.
\qed

\bdefn

    A {\emphcolor vector field} of $S^n$ is a continuous map $f\colon S^n\longto{\bb R}^{n+1}$ such that for all $x\in S^n$, $\iprod{f(x),x}=0$.

\edefn

\bthrm[title=Hairy Ball Theorem]

    Let $n$ be even.
    Then for every vector field on $S^n$, there is an $x\in S^n$ such that $f(x)=0$.

\ethrm

\Proof suppose not, then we can define a continuous map $x\mapsto\frac{f(x)}{\norm{f(x)}}$ which is a map $S^n\longto S^n$.
These points are still tangent to $x$, in particular they cannot be $x$ or antipodal to $x$, in contradiction to $n$ being even.
\qed

Note that in general if $\U,\V$ is a good cover of $X$ and $\U\cap\V$ is contractible, then by Mayer-Vietoris we have
$$ 0 = \tilde H_i(\U\cap\V) \longto \tilde H_i(\U)\oplus\tilde H_i(\V) \longto \tilde H_i(X) \longto \tilde H_{i-1}(\U\cap\V) = 0 $$
so $\tilde H_i(X)\cong\tilde H_i(\U)\oplus\tilde H_i(\V)$.
In particular let us look at $S^n\vee S^m$, we can take $\U$ to be $S^n$ with a bit of $S^m$ and $\V$ to be $S^m$ with a bit of $S^n$, then $\U\cap\V$ is contractible and $\U$ is homotopic to $S^n$ and $\V$
to $S^m$ so
$$ \tilde H_i(S^n\vee S^m) \cong \tilde H_i(S^n) \oplus \tilde H_i(S^m) $$
and similarly by induction
$$ \tilde H_i\parens{\bigvee_{j=1}^k S^{n_j}} \cong \bigoplus_{j=1}^k\tilde H_i(S^{n_j}) $$

Now let us look at $X=nT$.
Let us take $\U$ to be a disk in $X$, and $\V$ to be the rest of $X$ with a bit of $\U$.
Then $\U$ is homotopic to a point, $\U\cap\V\simeq S^1$ and we showed last semester that $\V\simeq\bigvee_{2n}S^1$.
Mayer-Vietoris gives us
$$ \tilde H_i(\U\cap\V) \longto \tilde H_i(\U)\oplus\tilde H_i(\V) \longto \tilde H_i(X) \longto \tilde H_{i-1}(\U\cap\V) $$
when $i\geq2$ $\tilde H_i(\U\cap\V)=\tilde H_i(\U)\oplus\tilde H_i(\V)=0$, but we require $i\geq3$ for $\tilde H_{i-1}(\U\cap\V)=0$.
So when $i\geq3$, $\tilde H_i(nT)=0$.
So let us look at $i=2$:
$$ \displaylines{
    H_2(\U\cap\V) \longto H_2\U\oplus H_2\V \longto H_2X \longto H_1(\U\cap\V) \longto H_1\U\oplus H_1\V \longto H_1X \longto\cr
    \longto \tilde H_0(\U\cap\V) \longto \tilde H_0\U\oplus\tilde H_0\V \longto\tilde H_0X \longto 0
} $$
We get from this
$$ 0 \longto 0 \longto H_2X \longto {\bb Z} \longto {\bb Z}^{2n} \longto H_1X \longto 0 \longto 0 \longto \tilde H_0X \longto 0 $$
So we get that $\tilde H_0X=0$.
Let us focus on the map ${\bb Z}\longto{\bb Z}^{2n}$ here, that is we need to understance $H_1(\U\cap\V)\longto H_1(\U)\oplus H_1(\V)=H_1(\V)$.
Visually, this can be shown to just be zero (using abelianization of $\pi_1$).
We can then just insert $0$ into the sequence where the zero morphism was:
$$ 0 \longto 0 \longto H_2X \longto {\bb Z} \longto 0 \longto {\bb Z}^{2n} \longto H_1X \longto 0 \longto 0 \longto \tilde H_0X \longto 0 $$
So we get that $H_2X\cong{\bb Z}$ and $H_1X\cong{\bb Z}^{2n}$.

\bthrm

    If $f\colon D^k\longto S^n$ is injective, then $\tilde H_i(S^n-f(D^k))=0$ for all $i$.

\ethrm

\Proof by induction on $k$.
For $k=0$, $S^n-\set\cdot\cong{\bb R}^n$ which is contractible and thus has a homotopy group of zero.
We will be working with the $k$-dimensional cube $I^k\cong D^k$.
So $f\colon I^k\times I\longto S^n$ is injective.
Define $A_1=I^k\times[0,1/2]$ and $B_1=I^k\times[1/2,1]$, and let $\U=S^n-f(A_1)=f(A_1)^c$ and $\V=S^n-f(B_1)=f(B_1)^c$.
So $\U\cup\V= f(A_1\cap B_1)^c=f(I^k\times\set{1/2})^c$.
So inductively, $\tilde H_i(\U\cup\V)=0$ for all $i$.
And $\U\cap\V=f(A_1\cup B_1)^c=f(I_{k+1})^c$ which is the space we want to compute the homology groups of.
By Mayer-Vietoris:
$$ 0 = \tilde H_{i+1}(\U\cup\V) \longto \tilde H_i(\U\cap\V) \longto \tilde H_i(\U)\oplus\tilde H_i(\V) \longto \tilde H_i(\U\cup\V) = 0 $$
So $\tilde H_i(\U\cap\V)\cong\tilde H_i\U\oplus\tilde H_i\V$.
So suppose that $\tilde H_i(\U\cap\V)\neq0$, then take $[z]\neq0$ in $\tilde H_i(\U\cap\V)$.
Taking the inclusion maps $i,j$ we have that one of $i_*[z]$ and $-j_*[z]$ is nonzero.
Continue.
\qed

\bthrm

    Let $f\colon S^k\longto S^n$ be injective, then
    $$ \tilde H_i(S^n-f(S^k)) = \cases{{\bb Z} & $i=n-k-1$\cr0 & else} $$

\ethrm

\Proof similarly by induction on $k$.
\qed

\bthrm[title=Jordan's Theorem]

    Let $f\colon S^{n-1}\longto S^n$ injective then $S^n-f(S^{n-1})$ has two path-connected components and they are open.

\ethrm

\bthrm[title=Invariance of Domain]

    Let $\U\subseteq{\bb R}^n$ be open, and $f\colon\U\longto{\bb R}^n$ injective.
    Then $f(\U)$ is open.

\ethrm

\Proof let $x\in\U$ then let us look at the restriction of $f$ to a closed ball around $x$, and show that the image of its interior is open.
If we choose this closed ball to lie in $\U$, looking at the union of these balls, we see that $\U$'s image is open.
So we must show that if $f\colon D^n\longto{\bb R}^n$ is injective, then $f(\Dot D^n)$ is open.

Now, ${\bb R}^n-f(\partial D^n)=f(\Dot D^n)\sqcup({\bb R}^n-f(D^n))$.
This is the union of two disjoint path-connected spaces (the left is the continuous map of a path-connectedt space, and the right is because $\tilde H_i({\bb R}^n-f(D^n))=0$ as an exercise).
By Jordan's theorem, there are two path-connected components and they are open.
So these are the two open path-connected components, in particular $f(\Dot D^n)$ is open.
\qed

Note then that if $\U\subseteq{\bb R}^n$ is open, then it is not homeomorphic to any subspace $A\subseteq{\bb R}^n$ which is not open.
This is because the homeomorphism $\U\longto A$ would mean by the invariance of domain that $A$ is open.
In particular, an open set in ${\bb R}^n$ is not homeomorphic to any open set in ${\bb R}^m$ for $n\neq m$.
This is because ${\bb R}^m\subset{\bb R}^n$ assuming $m<n$, and so $\U\longto A$ means that $A$ is open, but the last coordinates of $A$ are all zero and so it cannot be open.

\bdefn

    An $n$-dimensional {\emphcolor manifold} is a Hausdorff topological space $M$ with a countable basis such that for every $x\in M$ there exists a neighborhood homeomorphic to an open ball in ${\bb R}^n$.
    An $n$-dimensional {\emphcolor manifold with boundary} is a Hausdorff topological space $M$ with a countable basis such that every $x\in M$ has a neighborhood homeomorphic either to an open ball or
    to the half-open ball (which is defined to be $\set{(x_1,\dots,x_n)}[\norm{(x_1,\dots,x_n)}<1,x_1\geq0]$).
    A {\emphcolor closed manifold} is a compact manifold (without a boundary).

\edefn

\subsection{Excision}

Let $A\subseteq X$ be a subspace, then $\C(A)\subseteq\C(X)$ is a subcomplex, so we can define the {\it quotient complex} $\C(X,A)=\C(X)/\C(A)$.
Explicitly, $C_n(X,A)=C_n(X)/C_n(A)$.
The boundary operator $\partial$ maps from $C_n(A)$ to $C_{n-1}(A)$, so we can simply take $\partial[z]=[\partial z]$ in the quotient complex.
Thus we can define the {\it relative homology} groups of $X$ with respect to $A$ to be
$$ H_n(X,A) \coloneqq H_n(\C(X,A)) $$
Now, suppose $f\colon(X,A)\longto(Y,B)$ is a map, then we have $f_\sharp\colon C_n(X)\longto C_n(Y)$.
Do we have that this induces a map $f_\sharp\colon C_n(X,A)\longto C_n(Y,B)$?
In order for this to occur we must have $f_\sharp(C_n(A))\subseteq C_n(B)$, which is indeed the case (since $f\colon A\longto B$).
Thus we have a function $f_\sharp\colon C_n(X,A)\longto C_n(Y,B)$ and we can see that this is a chain morphism.
So we have defined a functor ${\bf Top}^2\longto{\bf Comp}$.
And in particular we can compose this with our functor ${\bf Comp}\longto{\bf Ab}$ to get ${\bf Top}^2\longto{\bf Ab}$.

Now, we have a short exact sequence of chain complexes
$$ 0 \longto C_n(A) \longto C_n(X) \longto C_n(X,A) \longto 0 $$
since this is precisely an inclusion-quotient chain, and the boundary operators are defined in such a way so that the diagram commutes.
Thus we have an exact sequence of homology groups:

\centerline{\drawdiagram{
    && $H_{n+1}(X,A)$\cr
    $H_n(A)$ & $H_n(X)$ & $H_n(X,A)$\cr
    $H_{n-1}(A)$ &&\cr
}{
    \diagarrow{from={1,1}, to={1,3}, dashed}
    \diagarrow{from={1,3}, to={2,1}}
    \diagarrow{from={2,1}, to={2,2}}
    \diagarrow{from={2,2}, to={2,3}}
    \diagarrow{from={2,3}, to={3,1}}
    \diagarrow{from={3,1}, to={3,3}, dashed}
}}

And this short exact sequence of chain complexes is natural, so this exact sequence is natural as well.

\bthrm

    $$ H_i(D^n,\partial D^n) = \cases{{\bb Z} & $i=n$\cr 0 & else} $$

\ethrm

\Proof by the exact sequence of homology groups we have
$$ 0 = \tilde H_i(D^n) \longto H_i(D^n,\partial D^n) \longto \tilde H_{i-1}(\partial D^n) \longto \tilde H_{i-1}(D^n) = 0 $$
so $H_i(D^n,\partial D^n)\cong\tilde H_{i-1}(\partial D^n)=\tilde H_{i-1}(S^{n-1})$ which is exactly what we want.
\qed

We can generalize this:

\blemm

    Let $A\subseteq X$ then
    \benum
        \item if $A$ is contractible, then $\tilde H_i(X)\cong H_i(X,A)$;
        \item if $X$ is contractible, then $\tilde H_{i-1}(A)\cong H_i(X,A)$.
    \eenum

\elemm

\Proof again we use the exact sequence:
$$ 0 = \tilde H_i(A) \longto \tilde H_i(X) \longto H_i(X,A) \longto \tilde H_{i-1}(A) = 0 $$
so $H_i(X,A)\cong\tilde H_i(X)$.
Similar for the second case.
\qed

\bprop

    Note if we have $f\colon(X,A)\longto(Y,B)$ then we have
    $$ f_*\colon H_n(X)\longto H_n(Y),\quad f_*\colon H_n(A)\longto H_n(Y),\quad f_*\colon H_n(X,A)\longto H_n(Y,B) $$
    If any two of these are isomorphisms, so is the third.

\eprop

\Proof immediate from the naturality of the exact sequence of homology groups, and the lemma of five.
\qed

In particular we have that

\bcoro

    If $f\colon X\longto Y$ and $f\colon A\longto B$ are both homotopic equivalences, then $f_*\colon H_n(X,A)\longto H_n(Y,B)$ is an isomorphism.

\ecoro

In particular, the inclusion map $(D^n,\partial D^n)\subseteq(D^n,D^n-\set0)$ is a homotopic equivalence, and so $H_i(D^n,\partial D^n)\cong H_i(D^n,D^n-\set0)$.
Thus
$$ H_i(D^n,D^n-\set0) = \cases{{\bb Z} & $i=n$\cr 0 & else} $$
And we can look at our good friend $R\colon(x_1,\dots,x_n)\mapsto(-x_1,x_2,\dots,x_n)$ which can be viewed as $R\colon(D^n,\partial D^n)\longto(D^n,\partial D^n)$ and we have the commutative diagram

\centerline{\drawdiagram{
    $H_n(D^n,\partial D^n)$ & $\tilde H_{n-1}(\partial D^n)$\cr
    $H_n(D^n,\partial D^n)$ & $\tilde H_{n-1}(\partial D^n)$\cr
}{
    \diagarrow{from={1,1}, to={1,2}, text=$\cong$, y distance=.25cm}
    \diagarrow{from={2,1}, to={2,2}, text=$\cong$, y distance=-.25cm}
    \diagarrow{from={1,1}, to={2,1}, text=$R_*$, x distance=-.25cm}
    \diagarrow{from={1,2}, to={2,2}, text={$R_*=-{\rm id}$}, x distance=1cm}
}}

And so $R_*=-{\rm id}$ for the map over $H_n(D^n,\partial D^n)$.

\bdefn

    Let $\U=\set{\U_\alpha}_{\alpha\in I}$ be a good covering of $X$, and let $A\subseteq X$, then we define
    $$ C_n^\U(X,A) = \slfrac{C_n^\U(X)}{C_n^\U(X)\cap C_n(A)} $$
    This is indeed a chain complex, since if a chain preserves $\U$ and is contained in $A$, then its boundary preserves $\U$ and is contained in $A$.

\edefn

Similar to before, the inclusion map $\iota\colon C_n^\U(X,A)\longto C_n(X,A)$ induces an isomorphism $\iota_*\colon H_n^\U(X,A)\longto H_n(X,A)$.

\bthrm[title=Excision]

    Let $K\subseteq A\subseteq X$ such that $\overline K\subseteq\Dot A$, then the inclusion $(X-K,A-K)\longto(X,A)$ induces an isomorphism of all homology groups $H_n(X-K,A-K)\longto H_n(X,A)$.

\ethrm

\Proof note that $\overline K\subseteq\Dot A$ is equivalent to $\U=\set{A,K^c}$ being a good cover.
So $C_n(A),C_n(X-K)\subseteq C_n^\U(X)$.
So we can compose the inclusion map with the quotient map to get $C_n(X-K)\longto C_n^\U(X)/C_n(A)\cap C_n^\U(X)=C_n^\U(X)/C_n(A)$.
We claim that this is a surjective map, as chains in $C_n^\U(X)/C_n(A)$ are classes of chains which respect $\set{X-K,A}$, but the simplexes which respect $A$ are identified with zero, so we are left with
formal sums of simplexes which respect $X-K$.
The kernel is just $C_n(X-K)\cap C_n(A)=C_n((X-K)\cap A)=C_n(A-K)$.
Thus by the first isomorphism theorem
$$ C_n(X-K,A-K) = \slfrac{C_n(X-K)}{C_n(A-K)} \cong \slfrac{C_n^\U(X)}{C_n(A)} = C_n^\U(X,A) $$
Thus we get that
$$ H_n(X-K,A-K) \cong H_n^\U(X,A) \cong H_n(X,A) \qed $$

\bthrm

    Let $M$ be an $n$-dimensional manifold with or without a boundary, and $p\in M$ be a point in its interior.
    Then
    $$ H_i(M,M-\set p) = \cases{{\bb Z} & $i=n$\cr 0 & else} $$

\ethrm

\Proof let $j\colon D^n\longto M$ be an embedding of $D^n$ into $M$ which maps $0$ to $p$ and which maps $\Dot D^n$ to a neighborhood of $p$.
Let us identify $D^n$ with its image in $M$ and $p$ with $0$.
Then $(D^n,D^n-\set0)\subseteq(M,M-\set0)$ is an excision: take $A=M-\set0$ and $K=M-D^n$.
Thus $H_i(M,M-\set0)\cong H_i(D^n,D^n-\set0)$ which is precisely what we want.
\qed

\bcoro

    The dimension of a manifold $M$ is a topological property of $M$ (i.e. it is unique).

\ecoro

\Proof this is since it is determined by its homology groups.
\qed

\bthrm

    Let $M$ be a manifold with a boundary and $p$ a point on its boundary.
    Then $H_i(M,M-\set p)=0$ for all $i$.

\ethrm

\Proof take $j\colon C\longto M$ an embedding of the half-open ball into $M$.
Then as before $(C,C-\set0)\subseteq(M,M-\set0)$ is an excision and both $C$ and $C-\set0$ are contractible.
We have the exact sequence
$$ 0 = \tilde H_i(C) \longto H_i(C,C-\set0) \longto \tilde H_i(C-\set0) = 0 $$
so $H_i(M,M-\set p)\cong H_i(C,C-\set0)=0$.
\qed

\bcoro

    The boundary of a manifold is a topological property of $M$ (a point cannot be both in its boundary and interior).

\ecoro

Note that if $p\in M$ is a boundary point, then it has a neighborhood homeomorphic to $\set{\vec x\in B_1^n(0)}[x_n\geq0]$.
Thus it has a neighborhood homeomorphic to $B_1^{n-1}(0)$ (taking the last coordinate equal to $0$), and all the points in this neighborhood must also be boundary points.
Thus the boundary of an $n$-dimensional manifold is an $n-1$-dimensional manifold.

\bthrm

    $[{\rm id}_n]\in H_n(\Delta^n,\partial\Delta^n)$ generates the homological group.

\ethrm

\bthrm

    Let $A\subseteq X$ be closed and suppose that there exists an open $\U$ such that $A\subseteq\U\subseteq X$ and $A$ is a deformation retract of $\U$.
    Then
    $$ H_n(X,A) = \tilde H_n(X/A) $$

\ethrm

\Proof since $A$ is a deformation retract, the inclusion $(X,A)\longto(X,\U)$ induces an isomorphism $H_n(X,A)\cong H_n(X,\U)$.
Furthermore by excision, $H_n(X-A,\U-A)\cong H_n(X,\U)$.
Let $a$ be the point which represents $A$ in $X/A$.
Then $H_n(X/A,\set a)\cong H_n(X/A,\U/A)$ similar to above.
And by excision $H_n(X/A-\set a,\U/A-\set a)\cong H_n(X/A,\U/A)$.
Note though that $(X-A,\U-A)$ is homeomomorphic to $(X/A-\set a,\U/A-\set a)$ (both just remove $A$).
Thus
$$ H_n(X,A) \cong H_n(X,\U) \cong H_n(X-A,\U-A) \cong H_n(X/A-\set a,\U/A-\set a) \cong H_n(X/A,\U/A) \cong H_n(X/A,\set a) $$
But since $\set a$ is contractible, by our exact sequence we see that this isomorphic to $H_n(X/A)$.

\bdefn

    An {\emphcolor orientation} on an $n$-dimensional manifold $M$ is a choice of a generator of $a_p\in H_n(M,M-\set p)$ such that for every $p\in M$ there is a euclidean neighborhood $\U$ and a choice
    of generator $a\in H_n(M,M-\U)$ such that for every $q\in\U$ with the inclusion $i_q\colon(M,M-\U)\longto(M,M-\set q)$ we have $i_{q,*}(a)=a_q$.

    If we can choose an orientation of a manifold, call it {\emphcolor orientable}.

\edefn

Note that since $H_n(M,M-\set p)\cong{\bb Z}$, there are two choices of orientation for each $p\in M$.

Further note that if we have a path on a manifold, $\gamma\colon I\longto M$, we can choose an orientation for $p=\gamma(0)$.
Then by covering the path with open balls, we can ensure that this orientation is consistent in each open ball.
This will give us an orientation for $q=\gamma(1)$.
This is independent on the choice of covering of the path.
If $M$ is orientable then the orientation of $q$ is also independent on the choice of the path (and is dependent only on $p$'s orientation).
Notice then that a closed loop in $M$ must start and end with the same orientation (i.e. it is {\it orientation-preserving}), in fact this is equivalent to $M$ being orientable.

\bthrm

    $M$ is orientable if and only if every closed loop in $M$ is orientation-preserving.

\ethrm

So for example, since a loop on the center of the M\"obius strip is not orientation-preserving, the M\"obius strip is not orientable.

Furthermore, if we have two paths $\gamma,\delta$ which are homotopic relative to their endpoints, then they have the same orientation (i.e. if $p=\gamma(0)=\delta(0)$ is given an orientation, both paths
give the same orientation to $q=\gamma(1)=\delta(1)$).
Further note that if $\gamma$ preserves orientation and $\delta$ flips orientation then $\gamma*\delta$ flips orientation, and so on for all combinations.
So we can assign to orientation-preserving loops the value $0$, and to orientation-flipping loops the value $1$.
For example if $\gamma,\delta$ are both orientation-flipping, the value of $\gamma*\delta$ is $0$.
By these two facts, we can define a homomorphism
$$ \phi\colon\pi_1(M,b) \longto {\bb Z}/2{\bb Z} $$
which assigns to each closed loop on $b$ $0$ if it preserves orientation and $1$ if it flips orientation.
$M$ is orientable if and only if this is the trivial homomorphism for all $b$ (all closed loops preserve orientation).

The issue is to check if $M$ is orientable we must check this homomorphism for every path-connected component of $M$.
But since ${\bb Z}/2{\bb Z}$ is Abelian and $H_1(M)={\rm Ab}\pi_1(M,b)$, there is an induced homomorphism $H_1(M)\longto{\bb Z}/2{\bb Z}$.
And $M$ is orientable if and only if this homomorphism is trivial.
And this homomorphism is trivial if it is trivial on the generators of $H_1(M)$.

So $M$ is orientable if and only if the generators of $H_1(M)$ preserve orientation.

For example take $M=nT$.
All of the generators of $H_1(M)$ (which are the center circles of the torii) preserve orientatiom, so $M$ is orientable.

Note that if $M$ is not orientable, there exists a closed loop on $M$ which flips orientation.
This loop can be blown up (since the orientation is taken in a neighborhood) to a quotient of $D^{n-1}\times I$ where $D^{n-1}\times\set0$ and $D^{n-1}\times\set1$ are identified but with the orientation
swapped.
Such a space is called a {\it full Klein bottle}.
So $M$ is not orientable if and only if a full Klein bottle can be embedded into it.

\subsection{Homology of CW Complexes}

Let a CW complex $K$ be constructed out of skeletons $K^0\subseteq K^1\subseteq\cdots\subseteq K^m=K$.
We would like to compute $H_i(K^n,K^{n-1})$.
We claim that there exists an open $\U$ such that $K^{n-1}\subseteq\U\subseteq K^n$ and $K^{n-1}$ is a deformation retract of $\U$.
This $\U$ can be taken to include part of the cells added to $K^{n-1}$ (in particular, something like $D^n-\set0$).
Thus we have that $H_i(K^n,K^{n-1})=\tilde H_i(K^n/K^{n-1})$.

Recall that $K^n$ is obtained by adding disks to $K^{n-1}$.
So if we contract $K^{n-1}$ to a point, we have essentially just added these disks to a point.
And we know that contracting the disk $D^n$ at its boundary to a point is just $S^n$, so we have that $K^{n-1}/K^n=\bigvee_{f_n}S^n$, and thus
$$ H_i(K^n,K^{n-1}) = H_i\parens{\bigvee_{f_n}S^n} = \cases{{\bb Z}^{f_n} & $i=n$\cr 0 & else} $$

We have an exact sequence
$$ H_{i+1}(K^n,K^{n-1}) \longto H_i(K^{n-1}) \longto H_i(K^n) \longto H_i(K^n,K^{n-1}) $$
If $n\neq i,i+1$ then $H_i(K^n)=H_i(K^{n-1})$.
In particular for $n<i$ we have $H_i(K^n)=0$ (since $H_i(K^n)=H_i(K^0)$ and the homology group of a set of points is $0$).
We have a sequence of homomorphisms (not necessarily exact, it is induced by the inclusion maps):
$$ 0 = H_i(K^{i-1}) \longto H_i(K^i) \longto H_i(K^{i+1}) $$
So let $A=H_i(K^i)$ and $B=H_i(K^{i+1})$, then we know that for $n>i+1$ we have $H_i(K^n)=H_i(K^{i+1})=B$.
In particular $H_i(K)=H_i(K^{i+1})$.
Thus we get the following

\bthrm

    Let $K^0\subseteq\cdots\subseteq K^m=K$ be a CW complex.
    Then
    \benum
        \item $$ H_i(K^n,K^{n-1}) = \cases{{\bb Z}^{f_n} & $i=n$\cr 0 & else} . $$
        \item $H_i(K^n)=0$ for $i<n$.
        \item $H_i(K^n)=H_i(K)$ for $n>i$.
    \eenum

\ethrm

Let us define $E_n=H_n(K^n,K^{n-1})={\bb Z}^{f_n}$.
Now, recall that we have two exact sequences:
$$ E_n = H_n(K^n,K^{n-1}) \longto H_{n-1}(K^{n-1}) $$
and
$$ E_{n-1} = H_{n-1}(K^{n-1}) \longto H_{n-1}(K^{n-1},K^{n-2}) $$
composing them gives a sequence (not necessarily exact):
$$ E_n \xvarrightarrow{\ \Delta\ } H_{n-1}(K^{n-1}) \xvarrightarrow{\ i_*\ } E_{n-1} $$
If we now look at the composition of these maps, we get a sequence
$$ \cdots \longto E_n \longto E_{n-1} \longto \cdots \longto E_0 $$
This is a chain complex, as if we look at
$$ E_n \xto{\ \Delta\ } H_{n-1}(K^{n-1}) \xto{\ i_*\ } E_{n-1} \xto{\ \Delta\ } H_{n-2}(K^{n-2}) \xto{\ i_*\ } E_{n-2} $$
the middle part of teh chain $H_{n-1}(K^{n-1}) \xto{\ i_*\ } E_{n-1} \xto{\ \Delta\ } H_{n-2}(K^{n-2})$ is part of $(K^{n-1},K^{n-2})$'s sequence, and so the composition is zero.
It turns out

\bthrm

    $H_i(K)$ is equal to the $i$th homology group of the chain complex of $E_n$.

\ethrm

Let us look at the $K^n$ skeleton of a CW complex, it is of the form $K^{n-1}\coprod_{i=1,\phi_i}^{f_n}D^n$, meaning
$$ K^n = \slfrac{K^{n-1}\coprod_{i=1}^{f_n}D^n_i}{x\sim\phi_i(x)\hbox{ for $x\in\partial D^n_i$}} $$
where $\phi_i\colon\partial D^n_i\longto K^{n-1}$ are the attaching maps.
We can look at the sequence
$$ D^n\amalg\cdots\amalg D^n \xvarrightarrow{\ i\ } K^{n-1}\amalg D^n\amalg\cdots\amalg D^n \xvarrightarrow{\ \rho\ } K^n $$
where $i$ is the inclusion map and $\rho$ is the quotient map.
Note that $\rho\circ i$ restricted to $\partial D^n_i$ s the attaching map, i.e. $\phi_i$.
So we can look at
$$ g = \rho\circ i\colon (D^n\amalg\cdots\amalg D^n,\partial D^n\amalg\cdots\amalg\partial D^n) \longto (K^n,K^{n-1}) $$
and on the boundaries, this is just the attaching map.
Note that $H_n\parens{\coprod D^n,\coprod\partial D^n}=\bigoplus H_n(D^n,\partial D^n)={\bb Z}^{f_n}$.
So $H_n\parens{\coprod D^n,\coprod\partial D^n}=H_n(K^n,K^{n-1})$ and indeed $g_*$ is actually an isomorphism.

