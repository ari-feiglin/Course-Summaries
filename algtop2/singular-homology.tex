\subsection{Chain Complexes}

We begin by defining a {\it chain complex}.
A chain complex is a sequence of Abelian groups with homomorphisms between them:

\medskip
\centerline{\drawdiagram{
    $\cdots$ & $C_{n+1}$ & $C_n$ & $C_{n-1}$ & $\cdots$ & $C_0$ & $0$\cr
}{
    \diagarrow{from={1,1}, to={1,2}, dashed}
    \diagarrow{from={1,2}, to={1,3}, text=$\partial_{n+1}$, y distance=.25cm}
    \diagarrow{from={1,3}, to={1,4}, text=$\partial_{n}$, y distance=.25cm}
    \diagarrow{from={1,4}, to={1,5}, dashed}
    \diagarrow{from={1,5}, to={1,6}, dashed}
    \diagarrow{from={1,6}, to={1,7}}
}}
\medskip

such that for every $n$, $\partial_n\circ\partial_{n+1}=0$, in other words ${\rm Im}\partial_{n+1}\subseteq\ker\partial_n$.
Define $Z_n=\ker\partial_n$, and its elements will be called {\it $n$-dimensional cycles}.
And define $B_n={\rm Im}\partial_{n+1}$, its elements will be called {\it boundaries}.
Elements of the groups $C_n$ will be called {\it $n$-dimensional chains}.

We now want to define a category of chain complexes.
To do so we must define morphisms between chain complexes.
So suppose we have two chain complexes ${\scr C}=\set{C_n,\partial_n}$ and ${\scr D}=\set{D_n,\partial'_n}$.
We define a morphism from ${\scr C}$ to ${\scr D}$ to be a sequence of homomorphisms $f_n\colon C_n\longto D_n$ which preserves the structure of the chain.
Meaning $\partial_n'\circ f_n=f_{n-1}\circ\partial_n$, in other words the following diagram commutes:

\bigskip
\centerline{\def\diagrowbuf{.5cm}\drawdiagram{
    $\cdots$ & $C_{n+1}$ & $C_n$ & $C_{n-1}$ & $\cdots$ & $C_0$ & $0$\cr
    $\cdots$ & $D_{n+1}$ & $D_n$ & $D_{n-1}$ & $\cdots$ & $D_0$ & $0$\cr
}{
    \diagarrow{from={1,1}, to={1,2}, dashed}
    \diagarrow{from={1,2}, to={1,3}, text=$\partial_{n+1}$, y distance=.25cm}
    \diagarrow{from={1,3}, to={1,4}, text=$\partial_{n}$, y distance=.25cm}
    \diagarrow{from={1,4}, to={1,5}, dashed}
    \diagarrow{from={1,5}, to={1,6}, dashed}
    \diagarrow{from={1,6}, to={1,7}}
    \diagarrow{from={2,1}, to={2,2}, dashed}
    \diagarrow{from={2,2}, to={2,3}, text=$\partial'_{n+1}$, y distance=.25cm}
    \diagarrow{from={2,3}, to={2,4}, text=$\partial'_{n}$, y distance=.25cm}
    \diagarrow{from={2,4}, to={2,5}, dashed}
    \diagarrow{from={2,5}, to={2,6}, dashed}
    \diagarrow{from={2,6}, to={2,7}}
    \diagarrow{from={1,2}, to={2,2}, text=$f_{n+1}$, x distance=-.5cm}
    \diagarrow{from={1,3}, to={2,3}, text=$f_{n}$, x distance=-.25cm}
    \diagarrow{from={1,4}, to={2,4}, text=$f_{n-1}$, x distance=-.5cm}
    \diagarrow{from={1,6}, to={2,6}, text=$f_{0}$, x distance=-.25cm}
}}
\medskip

To simplify writing, we will write $\partial\circ f=f\circ\partial$, which $f$ and which $\partial$ is being referred to will be understood from context.

The composition of two morphisms $\set{f_n}\colon{\scr C}\longto{\scr D}$ and $\set{g_n}\colon{\scr D}\longto{\scr E}$ is defined to be $\set{g_n\circ f_n}\colon{\scr C}\longto{\scr E}$.
This is indeed a morphism:
$$ \partial\circ f\circ g = f\circ\partial\circ g = f\circ g\circ\partial $$
And then this implies that the identity morphism is just ${\rm Id}_{\scr C}=\set{\rm Id_{C_n}}\colon{\scr C}\longto{\scr C}$, as if $\set{f_n}\colon{\scr C}\longto{\scr D}$ then
$$ \set{f_n}\circ{\rm Id}_{\scr C} = \set{f_n\circ {\rm Id}_{C_n}} = \set{f_n},\qquad {\rm Id}_{\scr D}\circ\set{f_n} = \set{{\rm Id}_{D_n}\circ f_n} = \set{f_n} $$
Associativity is clear, so ${\bf Comp}$, the category of chain complexes, is indeed a category.

Now recall that by definition $\partial_n\circ\partial_{n+1}=0$, meaning
$$ B_n \subseteq Z_n \subseteq C_n $$
Since these groups are all Abelian, they are normal in one another, so let us define the {\it $n$th homology group} of a chain complex $\scr C$ as
$$ H_n({\scr C})\coloneqq\slfrac{\displaystyle Z_n}{\displaystyle B_n} = \slfrac{\displaystyle\ker\partial_n}{\displaystyle{\rm Im}\partial_{n+1}} $$

\bprop

    A chain complex morphism $\set{f_n}\colon\C\longto\D$ maps cycles to cycles and boundaries to boundaries.

\eprop

\Proof let $z\in C_n$ be a cycle, i.e. $\partial z=0$, but then $f(z)$ is a cycle since $\partial f(z)=f(\partial z)=f(0)=0$.
And let $b\in C_n$ be a boundary, so there exists an $a\in C_{n+1}$ such that $b=\partial a$.
Then $f(b)=f\partial(a)=\partial f(a)=\partial b$, so $f(b)$ is a boundary as well.
\qed

This means that if $\set{f_n}\colon\C\longto\D$ is a morphism of chain complexes, $\set{f_n}\colon Z_n(\C)\longto Z_n(\D)$ is well-defined, and so we have that

\centerline{\def\diagrowbuf{.5cm}\def\diagcolbuf{.5cm}
\drawdiagram{
    $Z_n(\C)$ & $Z_n(\D)$\cr
    $H_n(\C)$ & $H_n(\D)$\cr
}{
    \diagarrow{from={1,1}, to={1,2}}
    \diagarrow{from={1,1}, to={2,1}}
    \diagarrow{from={1,2}, to={2,2}}
    \diagarrow{from={1,1}, to={2,2}, curve=1cm, color=blue}
}}

Where the blue arrow $\psi$ is just the quotient map composed with $f_n$.
This induces a group morphism
$$ H_n(\set{f_n}) = f_*\colon H_n(\C)\longto H_n(\D) $$
since we can define $f_*([z])=\psi(z)$ since if $[z]=[z']$ then $z-z'\in B_n(\C)$ and so $f(z-z')\in B_n(\D)$ and thus the quotient of $f(z-z')$ is just $0$, so $\psi(z)=\psi(z')$.

We now claim that $H_n$ is a functor from the category of chain complexes ${\bf Comp}$ to the category of Abelian groups ${\bf Ab}$.
Now suppose $\set{f_n}\colon\C\longto\D$ and $\set{g_n}\colon\D\longto\E$ are chain complex morphisms, then the following diagram commutes

\bigskip
\centerline{\def\diagrowbuf{.5cm}\def\diagcolbuf{.5cm}
\drawdiagram{
    $Z_n(\C)$ & $Z_n(\D)$ & $Z_n(\E)$\cr
    $H_n(\C)$ & $H_n(\D)$ & $H_n(\E)$\cr
}{
    \diagarrow{from={1,1}, to={1,2}, text=$f$, y distance=.25cm}
    \diagarrow{from={1,1}, to={2,1}}
    \diagarrow{from={1,2}, to={2,2}}
    \diagarrow{from={1,2}, to={1,3}, text=$g$, y distance=.25cm}
    \diagarrow{from={1,3}, to={2,3}}
    \diagarrow{from={2,1}, to={2,2}, text=$f_*$, y distance=-.25cm}
    \diagarrow{from={2,2}, to={2,3}, text=$g_*$, y distance=-.25cm}
}}
\bigskip

And so $(g\circ f)_*=g_*\circ f_*$, and it is easily verified that ${\rm id}_*={\rm id}$ so $H_n$ is a functor ${\bf Comp}\longto{\bf Ab}$ (the category of Abelian groups).

\subsection{Singular Complex}

We now define a functor from ${\bf Top}$ to ${\bf Comp}$.

\bdefn

    Let $B$ be a set, then define the {\emphcolor free Abelian group} over $B$ to be
    $$ \FAof B = \bigoplus_{b\in B}{\bb Z} = \set{\phi\colon B\longto{\bb Z}}[\phi(b)\neq0\hbox{ for only finitely many $b\in B$}] $$

\edefn

Note then that there is a correspondence between $B$ and $\FAof B$: $b\oto\phi_b$ where
$$ \phi_b(x) = \cases{1 & $x=b$\cr 0 & $x\neq b$} $$
so we can identify $b$ with $\phi_b$, and it is easy to see that every element of $\FAof B$ can be written as $\sum_{i=1}^kn_i\phi_{b_i}$, abusing notation $\sum_{i=1}^knb_i$ and such a representation is
unique.

Notice that if $B$ is a set, $G$ an Abelian group, and $g\colon B\longto G$ a function, then there exists a unique group homomorphism $L\colon\FAof B\longto G$ which extends $g$.
This is defined by
$$ L\colon\sum_{i=1}^kn_ib_i \longmapsto \sum_{i=1}^kn_ig(b_i) $$

\bdefn

    The {\emphcolor $n$-dimensional simplex} is defined to be
    $$ \Delta^n \coloneqq \set{(x_0,\dots,x_n)\in{\bb R}^{n+1}}[x_i\geq0,\sum_{i=0}^nx_i=1] $$

\edefn

$\Delta^n$ has $n+1$ faces, and is homeomorphic to $D^n$.

\bdefn

    Let $X$ be a topological space, then an {\emphcolor $n$-dimensional singular simplex} in $X$ is a morphism (in the category of topological spaces; a continuous map) $\Delta^n\longto X$.
    Define $S_n(x)$ to be the set of all $n$-dimensional singular simplexes in $X$, and define $C_n(X)=\FAof{S_n(x)}$.

\edefn

We now want to define a chain complex on the sequence $C_n(X)$.

Let us define a set of maps $\tau_i^n\colon\Delta^{n-1}\longto\Delta^n$ for $0\leq i\leq n$ which maps
$$ \tau_i^n\colon (x_0,\dots,x_{n-1})\mapsto(x_1,\dots,x_{i-1},0,x_{i+1},\dots,x_{n-1}) $$
This is a well-defined continuous map, and geometrically it maps $\Delta^{n-1}$ to one of the faces of $\Delta^n$.

Let $\sigma\in S_n(x)$, then let us define
$$ \partial(\sigma) \coloneqq \sum_{i=0}^n(-1)^i\sigma\circ\tau_i^n $$
Note that the composition is well-defined since $\Delta^{n-1}\xvarrightarrow{\tau_i^n}\Delta^n\xvarrightarrow{\,\sigma\,}X$, meaning $\sigma\circ\tau_i^n$ is an $n-1$-dimensional singular simplex.
Thus $\partial$ can be extended to a map $\partial\colon C_n(X)=\FAof{S_n(X)}\longto\FAof{S_{n-1}(X)}=C_{n-1}(X)$
Notice that
$$ \partial_{n-1}(\partial_n\sigma) = \partial_{n-1}\parens{\sum_{i=0}^n(-1)^i\sigma\circ\tau_i^n} = \sum_{i=0}^n(-1)^i\partial_{n-1}(\sigma\circ\tau_i^n) =
\sum_{i=0}^n\sum_{j=0}^{n-1}(-1)^{i+j}\sigma\circ\tau_i^n\circ\tau_j^{n-1} $$
Notice that $\tau^n_i\circ\tau_j^{n-1}=\tau^n_j\circ\tau^{n-1}_{i-1}$ which can be verified from its definition, but the first has a sign of $(-1)^{i+j}$ in the sum and the second has $-(-1)^{i+j}$.
And so the sum is zero.

Thus we have defined a chain complex on $C_n(X)$, let us denote it by $\C(X)$, this is the first step in defining the functor.
Next we must define the correspondence between morphisms.

Let $f\colon X\longto Y$ be a continuous map between topological spaces.
Let us define $f_\sharp\colon\C(X)\longto\C(Y)$.
First we define it for $\sigma\in S_n(X)$ by $f_\sharp(\sigma)=f\circ\sigma$.
Since $\sigma\colon\Delta^n\longto X$ is continuous, so is $f\circ\sigma\colon\Delta^n\longto Y$ and so $f_\sharp$ is well-defined on the generators of $C_n(X)$.
This can be extended by linearity to $f_\sharp\colon C_n(X)\longto C_n(Y)$.
Notice that we ignore the subscripts and superscripts $(f_\sharp)_n^X$ for brevity and readability.

Now we must verify that this is a morphism of chain complexes, i.e. that $\partial f_\sharp = f_\sharp\partial$.
So
$$ f_\sharp\partial\sigma = f_\sharp\parens{\sum_{i=0}^n(-1)^i\sigma\circ\tau_i^n} = \sum_{i=0}^n(-1)^if_\sharp(\sigma\circ\tau_i^n) = \sum_{i=0}^n(-1)^if\circ\sigma\circ\tau_i^n
= \sum_{i=0}^n(-1)^i(f\circ\sigma)\circ\tau_i^n = \partial f_\sharp\sigma $$
and since this holds for generators, by linearity it holds for all $C_n(X)$.
Thus $f_\sharp$ is indeed a morphism of chain complexes.

Thus we have defined a functor ${\bf Top}\longto{\bf Comp}$.

\subsection{Singular Homology}

We have two functors ${\bf Top}\longto{\bf Comp}\longto{\bf Ab}$, and so composing them together gives us a functor ${\bf Top}\longto{\bf Ab}$.
For a topological space $X$, we will denote its image under this functor as $H_n(X)$, called the {\it $n$th homological group} of $X$.
And for a continuous map $f$, we denote its image as $f_*$ or $H_n(f)$.

Let us compute the homological groups of the trivial space: $X=\set{p}$.
Notice that $S_n(X)=\set{K_n}$ where $K_n$ is the constant map $\Delta^n\longto\set p$, and so $C_n(X)={\bb Z}$.
We want to now compute what the boundary operators are, so
$$ \partial K_n = \sum_{i=0}^n(-1)^iK_n\circ\tau_i^n $$
but $K_n\circ\tau_i^n$ is a morphism $\Delta^{n-1}\longto\set p$ meaning it is equal to $K_{n-1}$, thus $\partial K_n=\parens{\sum_{i=0}^n(-1)^i}K_{n-1}$.
For $n$ even this is then $K_{n-1}$ (or $1$), and $0$ for $n$ odd.
This means that either $\ker\partial=0$ or ${\rm Im}\partial={\bb Z}$, thus $H_n=0$ for $n>0$.
For $n=0$, we have that $\partial_0\colon{\bb Z}\longto0$ and so its kernel is ${\bb Z}$, but $\partial_1$ is trivial and so its image is $0$.
Thus $H_0={\bb Z}$.

So we have shown

\bprop

    Let $X=\set p$ be the trivial topological space, then its homological groups are
    $$ H_n(X) = \cases{{\bb Z} & $n=0$\cr0 & $n>0$} $$

\eprop

\bprop

    Let $X$ be path connected, then $H_0(X)\cong{\bb Z}$.

\eprop

\Proof we are concerned with the chain:

\bigskip
\centerline{\drawdiagram{
    $C_1(X)$ & $C_0(X)$ & $0$\cr
}{
    \diagarrow{from={1,1}, to={1,2}, text=$\partial_1$, y distance=.25cm}
    \diagarrow{from={1,2}, to={1,3}, text=$\partial_0$, y distance=.25cm}
}}
\bigskip

So first let us understand $C_0(X)$, this is generated by $S_0(X)$, all the maps $\Delta^0\longto X$ which are just all the points in $X$.
And $S_1(X)$ is generated by all the maps $I\cong\Delta^1\longto X$, so all the paths in $X$.
The boundary of a $1$-simplex is then
$$ \partial_1\sigma = \sigma(1) - \sigma(0) $$
and thus $B_1(X)={\rm Im}\partial_1$ is generated by elements of the form $a-b$ where there exists a path between $a$ and $b$.
Since $X$ is path-connected, this means that $B_1(X)$ is generated by $a-b$ for $a,b\in X$.
Now, the subgroup generated by this is $\set{\sum n_ip_i}[p_i\in X,\,\sum n_i=0]$.

And now $\partial_0$'s kernel is just $C_0(X)$ which is simply the free group generated by $X$.
Thus
$$ H_0(X) = \slfrac{\ds\set{\sum n_ip_i}}{\ds\set{\sum n_ip_i}[\sum n_i=0]} $$
This is isomorphic to ${\bb Z}$ since we can define $\phi\colon C_0(X)\longto{\bb Z}$ by $\sum n_ip_i\mapsto\sum n_i$ and this is a group homomorphism whose image is ${\bb Z}$ and whose kernel is all the
points $\sum n_ip_i$ where $\sum n_i=0$.
Thus by the isomorphism theorem, $H_0(X)\cong{\bb Z}$.
\qed

\bthrm

    Let $X$ be a topological space where $\set{A_\alpha}_{\alpha\in I}$ are its path connected components.
    Then for every $n$,
    $$ H_n(X) \cong \bigoplus_{\alpha\in I}H_n(A_\alpha) $$

\ethrm

\Proof notice that if $\sigma\colon\Delta^n\longto X$ is an $n$-simplex, then its image is contained within a path connected component.
This is since $\Delta^n$ is path-connected, so $\sigma\Delta^n$ must be too.
Thus for every $\gamma=\sum n_i\sigma_i\in S_n(X)$ we can write it as $\gamma=\sum\gamma_i$ for $\gamma_i\in S_n(A_i)$.
And so $C_n(X)=\bigoplus_{\alpha\in I}C_n(A_\alpha)$.

Notice that $\gamma$ is a cycle iff every $\gamma_i$ is a cycle, since $\partial\gamma=\sum\partial\gamma_i$ and this is an element of a direct sum, so it is zero iff $\partial\gamma_i=0$.
Thus $Z_n(X)=\bigoplus_{\alpha\in I}Z_n(A_\alpha)$.
And similarly we see that $B_n(X)=\bigoplus_{\alpha\in I}B_n(A_\alpha)$.
Thus $H_n(X)=\bigoplus_{\alpha\in I}H_n(A_\alpha)$.
\qed

\bcoro

    If $X$ is a topological space with $\set{A_\alpha}_{\alpha\in I}$ path connected components, $H_n(X)=\bigoplus_{\alpha\in I}{\bb Z}$.

\ecoro

