\subsection{Chain Complexes}

We begin by defining a {\it chain complex}.
A chain complex is a sequence of Abelian groups with homomorphisms between them:

\medskip
\centerline{\drawdiagram{
    $\cdots$ & $C_{n+1}$ & $C_n$ & $C_{n-1}$ & $\cdots$ & $C_0$ & $0$\cr
}{
    \diagarrow{from={1,1}, to={1,2}, dashed}
    \diagarrow{from={1,2}, to={1,3}, text=$\partial_{n+1}$, y distance=.25cm}
    \diagarrow{from={1,3}, to={1,4}, text=$\partial_{n}$, y distance=.25cm}
    \diagarrow{from={1,4}, to={1,5}, dashed}
    \diagarrow{from={1,5}, to={1,6}, dashed}
    \diagarrow{from={1,6}, to={1,7}}
}}
\medskip

such that for every $n$, $\partial_n\circ\partial_{n+1}=0$, in other words ${\rm Im}\partial_{n+1}\subseteq\ker\partial_n$.
Define $Z_n=\ker\partial_n$, and its elements will be called {\it $n$-dimensional cycles}.
And define $B_n={\rm Im}\partial_{n+1}$, its elements will be called {\it boundaries}.
Elements of the groups $C_n$ will be called {\it $n$-dimensional chains}.

We now want to define a category of chain complexes.
To do so we must define morphisms between chain complexes.
So suppose we have two chain complexes ${\scr C}=\set{C_n,\partial_n}$ and ${\scr D}=\set{D_n,\partial'_n}$.
We define a morphism from ${\scr C}$ to ${\scr D}$ to be a sequence of homomorphisms $f_n\colon C_n\longto D_n$ which preserves the structure of the chain.
Meaning $\partial_n'\circ f_n=f_{n-1}\circ\partial_n$, in other words the following diagram commutes:

\bigskip
\centerline{\def\diagrowbuf{.5cm}\drawdiagram{
    $\cdots$ & $C_{n+1}$ & $C_n$ & $C_{n-1}$ & $\cdots$ & $C_0$ & $0$\cr
    $\cdots$ & $D_{n+1}$ & $D_n$ & $D_{n-1}$ & $\cdots$ & $D_0$ & $0$\cr
}{
    \diagarrow{from={1,1}, to={1,2}, dashed}
    \diagarrow{from={1,2}, to={1,3}, text=$\partial_{n+1}$, y distance=.25cm}
    \diagarrow{from={1,3}, to={1,4}, text=$\partial_{n}$, y distance=.25cm}
    \diagarrow{from={1,4}, to={1,5}, dashed}
    \diagarrow{from={1,5}, to={1,6}, dashed}
    \diagarrow{from={1,6}, to={1,7}}
    \diagarrow{from={2,1}, to={2,2}, dashed}
    \diagarrow{from={2,2}, to={2,3}, text=$\partial'_{n+1}$, y distance=.25cm}
    \diagarrow{from={2,3}, to={2,4}, text=$\partial'_{n}$, y distance=.25cm}
    \diagarrow{from={2,4}, to={2,5}, dashed}
    \diagarrow{from={2,5}, to={2,6}, dashed}
    \diagarrow{from={2,6}, to={2,7}}
    \diagarrow{from={1,2}, to={2,2}, text=$f_{n+1}$, x distance=-.5cm}
    \diagarrow{from={1,3}, to={2,3}, text=$f_{n}$, x distance=-.25cm}
    \diagarrow{from={1,4}, to={2,4}, text=$f_{n-1}$, x distance=-.5cm}
    \diagarrow{from={1,6}, to={2,6}, text=$f_{0}$, x distance=-.25cm}
}}
\medskip

To simplify writing, we will write $\partial\circ f=f\circ\partial$, which $f$ and which $\partial$ is being referred to will be understood from context.

The composition of two morphisms $\set{f_n}\colon{\scr C}\longto{\scr D}$ and $\set{g_n}\colon{\scr D}\longto{\scr E}$ is defined to be $\set{g_n\circ f_n}\colon{\scr C}\longto{\scr E}$.
This is indeed a morphism:
$$ \partial\circ f\circ g = f\circ\partial\circ g = f\circ g\circ\partial $$
And then this implies that the identity morphism is just ${\rm Id}_{\scr C}=\set{\rm Id_{C_n}}\colon{\scr C}\longto{\scr C}$, as if $\set{f_n}\colon{\scr C}\longto{\scr D}$ then
$$ \set{f_n}\circ{\rm Id}_{\scr C} = \set{f_n\circ {\rm Id}_{C_n}} = \set{f_n},\qquad {\rm Id}_{\scr D}\circ\set{f_n} = \set{{\rm Id}_{D_n}\circ f_n} = \set{f_n} $$
Associativity is clear, so ${\bf Comp}$, the category of chain complexes, is indeed a category.

Now recall that by definition $\partial_n\circ\partial_{n+1}=0$, meaning
$$ B_n \subseteq Z_n \subseteq C_n $$
Since these groups are all Abelian, they are normal in one another, so let us define the {\it $n$th homology group} of a chain complex $\scr C$ as
$$ H_n({\scr C})\coloneqq\slfrac{\displaystyle Z_n}{\displaystyle B_n} = \slfrac{\displaystyle\ker\partial_n}{\displaystyle{\rm Im}\partial_{n+1}} $$

\bprop

    A chain complex morphism $\set{f_n}\colon\C\longto\D$ maps cycles to cycles and boundaries to boundaries.

\eprop

\Proof let $z\in C_n$ be a cycle, i.e. $\partial z=0$, but then $f(z)$ is a cycle since $\partial f(z)=f(\partial z)=f(0)=0$.
And let $b\in C_n$ be a boundary, so there exists an $a\in C_{n+1}$ such that $b=\partial a$.
Then $f(b)=f\partial(a)=\partial f(a)=\partial b$, so $f(b)$ is a boundary as well.
\qed

This means that if $\set{f_n}\colon\C\longto\D$ is a morphism of chain complexes, $\set{f_n}\colon Z_n(\C)\longto Z_n(\D)$ is well-defined, and so we have that

\centerline{\def\diagrowbuf{.5cm}\def\diagcolbuf{.5cm}
\drawdiagram{
    $Z_n(\C)$ & $Z_n(\D)$\cr
    $H_n(\C)$ & $H_n(\D)$\cr
}{
    \diagarrow{from={1,1}, to={1,2}}
    \diagarrow{from={1,1}, to={2,1}}
    \diagarrow{from={1,2}, to={2,2}}
    \diagarrow{from={1,1}, to={2,2}, curve=1cm, color=blue}
}}

Where the blue arrow $\psi$ is just the quotient map composed with $f_n$.
This induces a group morphism
$$ H_n(\set{f_n}) = f_*\colon H_n(\C)\longto H_n(\D) $$
since we can define $f_*([z])=\psi(z)$ since if $[z]=[z']$ then $z-z'\in B_n(\C)$ and so $f(z-z')\in B_n(\D)$ and thus the quotient of $f(z-z')$ is just $0$, so $\psi(z)=\psi(z')$.

We now claim that $H_n$ is a functor from the category of chain complexes ${\bf Comp}$ to the category of Abelian groups ${\bf Ab}$.
Now suppose $\set{f_n}\colon\C\longto\D$ and $\set{g_n}\colon\D\longto\E$ are chain complex morphisms, then the following diagram commutes

\bigskip
\centerline{\def\diagrowbuf{.5cm}\def\diagcolbuf{.5cm}
\drawdiagram{
    $Z_n(\C)$ & $Z_n(\D)$ & $Z_n(\E)$\cr
    $H_n(\C)$ & $H_n(\D)$ & $H_n(\E)$\cr
}{
    \diagarrow{from={1,1}, to={1,2}, text=$f$, y distance=.25cm}
    \diagarrow{from={1,1}, to={2,1}}
    \diagarrow{from={1,2}, to={2,2}}
    \diagarrow{from={1,2}, to={1,3}, text=$g$, y distance=.25cm}
    \diagarrow{from={1,3}, to={2,3}}
    \diagarrow{from={2,1}, to={2,2}, text=$f_*$, y distance=-.25cm}
    \diagarrow{from={2,2}, to={2,3}, text=$g_*$, y distance=-.25cm}
}}
\bigskip

And so $(g\circ f)_*=g_*\circ f_*$, and it is easily verified that ${\rm id}_*={\rm id}$ so $H_n$ is a functor ${\bf Comp}\longto{\bf Ab}$ (the category of Abelian groups).

\subsection{Singular Complex}

We now define a functor from ${\bf Top}$ to ${\bf Comp}$.

\bdefn

    Let $B$ be a set, then define the {\emphcolor free Abelian group} over $B$ to be
    $$ \FAof B = \bigoplus_{b\in B}{\bb Z} = \set{\phi\colon B\longto{\bb Z}}[\phi(b)\neq0\hbox{ for only finitely many $b\in B$}] $$

\edefn

Note then that there is a correspondence between $B$ and $\FAof B$: $b\oto\phi_b$ where
$$ \phi_b(x) = \cases{1 & $x=b$\cr 0 & $x\neq b$} $$
so we can identify $b$ with $\phi_b$, and it is easy to see that every element of $\FAof B$ can be written as $\sum_{i=1}^kn_i\phi_{b_i}$, abusing notation $\sum_{i=1}^knb_i$ and such a representation is
unique.

Notice that if $B$ is a set, $G$ an Abelian group, and $g\colon B\longto G$ a function, then there exists a unique group homomorphism $L\colon\FAof B\longto G$ which extends $g$.
This is defined by
$$ L\colon\sum_{i=1}^kn_ib_i \longmapsto \sum_{i=1}^kn_ig(b_i) $$

\bdefn

    The {\emphcolor $n$-dimensional simplex} is defined to be
    $$ \Delta^n \coloneqq \set{(x_0,\dots,x_n)\in{\bb R}^{n+1}}[x_i\geq0,\sum_{i=0}^nx_i=1] $$

\edefn

$\Delta^n$ has $n+1$ faces, and is homeomorphic to $D^n$.

\bdefn

    Let $X$ be a topological space, then an {\emphcolor $n$-dimensional singular simplex} in $X$ is a morphism (in the category of topological spaces; a continuous map) $\Delta^n\longto X$.
    Define $S_n(x)$ to be the set of all $n$-dimensional singular simplexes in $X$, and define $C_n(X)=\FAof{S_n(x)}$.

\edefn

We now want to define a chain complex on the sequence $C_n(X)$.

Let us define a set of maps $\tau_i^n\colon\Delta^{n-1}\longto\Delta^n$ for $0\leq i\leq n$ which maps
$$ \tau_i^n\colon (x_0,\dots,x_{n-1})\mapsto(x_0,\dots,x_{i-1},0,x_i,\dots,x_{n-1}) $$
This is a well-defined continuous map, and geometrically it maps $\Delta^{n-1}$ to one of the faces of $\Delta^n$.

Let $\sigma\in S_n(x)$, then let us define
$$ \partial(\sigma) \coloneqq \sum_{i=0}^n(-1)^i\sigma\circ\tau_i^n $$
Note that the composition is well-defined since $\Delta^{n-1}\xvarrightarrow{\tau_i^n}\Delta^n\xvarrightarrow{\,\sigma\,}X$, meaning $\sigma\circ\tau_i^n$ is an $n-1$-dimensional singular simplex.
Thus $\partial$ can be extended to a map $\partial\colon C_n(X)=\FAof{S_n(X)}\longto\FAof{S_{n-1}(X)}=C_{n-1}(X)$
Notice that
$$ \partial_{n-1}(\partial_n\sigma) = \partial_{n-1}\parens{\sum_{i=0}^n(-1)^i\sigma\circ\tau_i^n} = \sum_{i=0}^n(-1)^i\partial_{n-1}(\sigma\circ\tau_i^n) =
\sum_{i=0}^n\sum_{j=0}^{n-1}(-1)^{i+j}\sigma\circ\tau_i^n\circ\tau_j^{n-1} $$
Notice that $\tau^n_i\circ\tau_j^{n-1}=\tau^n_j\circ\tau^{n-1}_{i-1}$ which can be verified from its definition, but the first has a sign of $(-1)^{i+j}$ in the sum and the second has $-(-1)^{i+j}$.
And so the sum is zero.

Thus we have defined a chain complex on $C_n(X)$, let us denote it by $\C(X)$, this is the first step in defining the functor.
Next we must define the correspondence between morphisms.

Let $f\colon X\longto Y$ be a continuous map between topological spaces.
Let us define $f_\sharp\colon\C(X)\longto\C(Y)$.
First we define it for $\sigma\in S_n(X)$ by $f_\sharp(\sigma)=f\circ\sigma$.
Since $\sigma\colon\Delta^n\longto X$ is continuous, so is $f\circ\sigma\colon\Delta^n\longto Y$ and so $f_\sharp$ is well-defined on the generators of $C_n(X)$.
This can be extended by linearity to $f_\sharp\colon C_n(X)\longto C_n(Y)$.
Notice that we ignore the subscripts and superscripts $(f_\sharp)_n^X$ for brevity and readability.

Now we must verify that this is a morphism of chain complexes, i.e. that $\partial f_\sharp = f_\sharp\partial$.
So
$$ f_\sharp\partial\sigma = f_\sharp\parens{\sum_{i=0}^n(-1)^i\sigma\circ\tau_i^n} = \sum_{i=0}^n(-1)^if_\sharp(\sigma\circ\tau_i^n) = \sum_{i=0}^n(-1)^if\circ\sigma\circ\tau_i^n
= \sum_{i=0}^n(-1)^i(f\circ\sigma)\circ\tau_i^n = \partial f_\sharp\sigma $$
and since this holds for generators, by linearity it holds for all $C_n(X)$.
Thus $f_\sharp$ is indeed a morphism of chain complexes.

Thus we have defined a functor ${\bf Top}\longto{\bf Comp}$.

\subsection{Singular Homology}

We have two functors ${\bf Top}\longto{\bf Comp}\longto{\bf Ab}$, and so composing them together gives us a functor ${\bf Top}\longto{\bf Ab}$.
For a topological space $X$, we will denote its image under this functor as $H_n(X)$, called the {\it $n$th homological group} of $X$.
And for a continuous map $f$, we denote its image as $f_*$ or $H_n(f)$.

Let us compute the homological groups of the trivial space: $X=\set{p}$.
Notice that $S_n(X)=\set{K_n}$ where $K_n$ is the constant map $\Delta^n\longto\set p$, and so $C_n(X)={\bb Z}$.
We want to now compute what the boundary operators are, so
$$ \partial K_n = \sum_{i=0}^n(-1)^iK_n\circ\tau_i^n $$
but $K_n\circ\tau_i^n$ is a morphism $\Delta^{n-1}\longto\set p$ meaning it is equal to $K_{n-1}$, thus $\partial K_n=\parens{\sum_{i=0}^n(-1)^i}K_{n-1}$.
For $n$ even this is then $K_{n-1}$ (or $1$), and $0$ for $n$ odd.
This means that either $\ker\partial=0$ or ${\rm Im}\partial={\bb Z}$, thus $H_n=0$ for $n>0$.
For $n=0$, we have that $\partial_0\colon{\bb Z}\longto0$ and so its kernel is ${\bb Z}$, but $\partial_1$ is trivial and so its image is $0$.
Thus $H_0={\bb Z}$.

So we have shown

\bprop

    Let $X=\set p$ be the trivial topological space, then its homological groups are
    $$ H_n(X) = \cases{{\bb Z} & $n=0$\cr0 & $n>0$} $$

\eprop

\bprop

    Let $X$ be path connected, then $H_0(X)\cong{\bb Z}$.

\eprop

\Proof we are concerned with the chain:

\bigskip
\centerline{\drawdiagram{
    $C_1(X)$ & $C_0(X)$ & $0$\cr
}{
    \diagarrow{from={1,1}, to={1,2}, text=$\partial_1$, y distance=.25cm}
    \diagarrow{from={1,2}, to={1,3}, text=$\partial_0$, y distance=.25cm}
}}
\bigskip

So first let us understand $C_0(X)$, this is generated by $S_0(X)$, all the maps $\Delta^0\longto X$ which are just all the points in $X$.
And $S_1(X)$ is generated by all the maps $I\cong\Delta^1\longto X$, so all the paths in $X$.
The boundary of a $1$-simplex is then
$$ \partial_1\sigma = \sigma(1) - \sigma(0) $$
and thus $B_1(X)={\rm Im}\partial_1$ is generated by elements of the form $a-b$ where there exists a path between $a$ and $b$.
Since $X$ is path-connected, this means that $B_1(X)$ is generated by $a-b$ for $a,b\in X$.
Now, the subgroup generated by this is $\set{\sum n_ip_i}[p_i\in X,\,\sum n_i=0]$.

And now $\partial_0$'s kernel is just $C_0(X)$ which is simply the free group generated by $X$.
Thus
$$ H_0(X) = \slfrac{\ds\set{\sum n_ip_i}}{\ds\set{\sum n_ip_i}[\sum n_i=0]} $$
This is isomorphic to ${\bb Z}$ since we can define $\phi\colon C_0(X)\longto{\bb Z}$ by $\sum n_ip_i\mapsto\sum n_i$ and this is a group homomorphism whose image is ${\bb Z}$ and whose kernel is all the
points $\sum n_ip_i$ where $\sum n_i=0$.
Thus by the isomorphism theorem, $H_0(X)\cong{\bb Z}$.
\qed

\bthrm

    Let $X$ be a topological space where $\set{A_\alpha}_{\alpha\in I}$ are its path connected components.
    Then for every $n$,
    $$ H_n(X) \cong \bigoplus_{\alpha\in I}H_n(A_\alpha) $$

\ethrm

\Proof notice that if $\sigma\colon\Delta^n\longto X$ is an $n$-simplex, then its image is contained within a path connected component.
This is since $\Delta^n$ is path-connected, so $\sigma\Delta^n$ must be too.
Thus for every $\gamma=\sum n_i\sigma_i\in S_n(X)$ we can write it as $\gamma=\sum\gamma_i$ for $\gamma_i\in S_n(A_i)$.
And so $C_n(X)=\bigoplus_{\alpha\in I}C_n(A_\alpha)$.

Notice that $\gamma$ is a cycle iff every $\gamma_i$ is a cycle, since $\partial\gamma=\sum\partial\gamma_i$ and this is an element of a direct sum, so it is zero iff $\partial\gamma_i=0$.
Thus $Z_n(X)=\bigoplus_{\alpha\in I}Z_n(A_\alpha)$.
And similarly we see that $B_n(X)=\bigoplus_{\alpha\in I}B_n(A_\alpha)$.
Thus $H_n(X)=\bigoplus_{\alpha\in I}H_n(A_\alpha)$.
\qed

\bcoro

    If $X$ is a topological space with $\set{A_\alpha}_{\alpha\in I}$ path connected components, $H_n(X)=\bigoplus_{\alpha\in I}{\bb Z}$.

\ecoro

\bthrm[name=abelofhomotopy]

    Let $X$ be path-connected and $a\in X$, then $H_1(X)\cong\Ab\pi_1(X,a)$.

\ethrm

For two chains, $a,b\in C_n(X)$ say that they are {\it homological} if $a-b$ is a boundary (i.e. $a-b\in B_n(X)$).
Write this as $a\approx b$.

\blemm

    Let $\sigma,\tau$ be $1$-simplexes.
    \benum
        \item if $\sigma$ is constant, then it is a boundary, i.e. $\sigma\approx0$.
        \item if $\sigma\hdi\tau$ (since they are maps from $I\cong\Delta^1\longto X$), then $\sigma\approx\tau$.
        \item if $\sigma(1)=\tau(0)$ then $\sigma*\tau\approx\sigma+\tau$.
        \item $\sigma+\bar\sigma\approx0$
    \eenum

\elemm

\Proof
\benum
    \item If $\sigma$ is constant, then it is $K^1_p$ for some $p\in X$.
        And as we have already computed
        $$ \partial K^n_p = \cases{K^{n-1}_p & $n$ even\cr 0 & $n$ odd} $$
        Thus $\partial K^2_p=K^{n-1}_p$, meaning $\sigma$ is a boundary.
    \item Let us look at the homotopy

        \centerline{\def\coordhbuf{0pt}\def\coordvbuf{-5pt}
        \drawdiagram{
            &&\cr
            &$H$&\cr
            &&\cr
        }{
            \diagarrow{from={1,1}, to={1,3}, text=$\tau$, y distance=.25cm}
            \diagarrow{from={3,1}, to={3,3}, text=$\sigma$, y distance=-.25cm}
            \diagarrow{from={1,1}, to={3,1}, right cap=-, text=$K_a$, x distance=-.25cm}
            \diagarrow{from={1,3}, to={3,3}, right cap=-, text=$K_b$, x distance=.25cm}
        }}

        Since $H$ is surjective, it induces a map on the quotient space $\slfrac{I\times I}{I\times\set1}$, the map $G$:

        \centerline{\def\coordhbuf{0pt}\def\coordvbuf{-5pt}
        \drawdiagram{
            &&\cr
            &$G$&\cr
            &&\cr
        }{
            \diagarrow{from={1,1}, to={2,3}, text=$\tau$, y distance=.25cm}
            \diagarrow{from={3,1}, to={2,3}, text=$\sigma$, y distance=-.25cm}
            \diagarrow{from={1,1}, to={3,1}, right cap=-, text=$K_a$, x distance=-.25cm}
        }}

        The quotient space can be viewed as a $2$-simplex by assigning an order to its vertices.
        Then its boundary is
        $$ \partial G = K_a - \sigma + \tau $$
        and since $\partial G$ is a boundary, we have that
        $$ K_a - \sigma + \tau \approx 0 $$
        by $(1)$ we have that $K_a\approx0$ so $\sigma-\tau\approx0$.

    \item The idea is to define a simplex of the form

        \centerline{\def\coordhbuf{0pt}\def\coordvbuf{-2.5pt}
        \drawdiagram{
            &&\cr
            &$G$&\cr
            &&\cr
        }{
            \diagarrow{from={2,3}, to={1,1}, text=$\tau$, y distance=.25cm}
            \diagarrow{from={3,1}, to={2,3}, text=$\sigma$, y distance=-.25cm}
            \diagarrow{from={1,1}, to={3,1}, text=$\sigma*\tau$, x distance=-.5cm}
        }}

        Notice that such a simplex is possible: each horizontal line in the domain can be made constant.
        And its boundary is
        $$ \partial G = \tau - \sigma*\tau + \sigma $$
        so $\sigma*\tau\approx\sigma+\tau$ since $\partial G\approx0$.

    \item This is direct from the previous three points:
        $$ \sigma+\overline\sigma \buildrel(3)\over\approx \sigma*\overline\sigma \buildrel(2)\over\approx K_b \buildrel(1)\over\approx 0 $$
\eenum

{\bf Proof} (of \refmath[theorem]{abelofhomotopy}): let us define a homomorphism
$$ F\colon\pi_1(X,a) \longto H_1(X) $$
Denote homotopy equivalence classes by $\gen\bullet$ and the equivalence classes of $H_1(X)$ by $[\bullet]$.
Then we define
$$ \gen\phi\xvarmapsto{\quad F\quad}\relax[\phi] $$
This is well-defined: if $\phi\hdi\psi$ then $\phi\approx\psi$ and so $[\phi]=[\psi]$ (since $H_n(X)$ is the partition of $Z_n(X)$ relative to $\approx$).
Notice that $\gen{\phi*\psi}\mapsto[\phi*\psi]=[\phi+\psi]=[\phi]+[\psi]$.
So this is indeed a homomorphism.
Since $H_1(X)$ is Abelian, this induces a homomorphism
$$ \overline F\colon\Ab\pi_1(X,a)\longto H_1(X) $$
Let us now define a homomorphism
$$ G\colon C_1(X)\longto\Ab\pi_1(X,a) $$
denote the equivalence classes of $\Ab\pi_1(X,a)$ by $\gen{\gen\bullet}$.
For every $x\in X$, choose a path $\gamma_x$ from $a$ to $x$, then for $\sigma\in S_1(X)$ define
$$ \hat\sigma = \gamma_{\sigma(0)}*\sigma*\overline\gamma_{\sigma(1)} \hbox{ from $a$ to $a$} $$
And define
$$ \sigma\xvarmapsto{\quad G\quad}\gen{\gen{\hat\sigma}} $$
And extend by linearity to $G\colon C_1(X)\longto\Ab\pi_1(X,a)$.
We can then restrict $G$ to $Z_1(X)$, and in order for this to induce a map on $\slfrac{Z_1(X)}{B_1(X)}$ we must have that $G\bigl|_{B_1(X)}=0$.
So let $A$ be a $2$-simplex, then we must show $G(\partial A)=0$.
We know
$$ G(\partial A) = G(A\circ\tau_0-A\circ\tau_1+A\circ\tau_2) = \ggen{\widehat{A\circ\tau_0}} - \ggen{\widehat{A\circ\tau_1}} + \ggen{\widehat{A\circ\tau_2}} $$
Now, $\ggen\sigma+\ggen\tau=\gen{\!\gen\sigma\gen\tau\!}$ and $-\ggen\sigma=\gen{\!\sigma^{-1}\!}$ by Abelianization, so this is equal to
$$ \gen{\!\gen{\widehat{A\circ\tau_0}}\gen{\widehat{A\circ\tau_1}}\gen{\widehat{A\circ\tau_2}}\!} = \ggen{\widehat{A\circ\tau_0}*\overline{\widehat{A\circ\tau_1}}*\widehat{A\circ\tau_2}} $$
As is easily verified,
$$ = \ggen{\widehat{A\circ\tau_0}*\widehat{\overline{A\circ\tau_1}}*\widehat{A\circ\tau_2}} = \ggen{\varwidehat{A\circ\tau_0*\overline{A\circ\tau_1}*A\circ\tau_2}} $$
Since $A\colon\Delta^2\longto X$ is a simplex, $A\circ\tau_0*\overline{A\circ\tau_1}*A\circ\tau_2$ is null-homotopic (the homotopy can condense the curve to a point through the image of $A$).
Therefore its hat is as well, meaning this is all equal to zero, as required.

So $G$ induces a homomorphism
$$ \overline G\colon H_1(X)\longto\Ab\pi_1(X,a) $$
Notice that
$$ \overline G\circ\overline F\ggen\phi = \overline G[\phi] = \ggen{\hat\phi} $$
We know that $\hat\phi=\gamma_a\phi\overline\gamma_a$ which is conjugate to $\phi$, so in the Abelianization they are equal.
So $\overline G\circ\overline F={\rm id}$.
Now suppose $[z]\in H_1(X)$ where $z=\sum n_i\sigma_i$ then
$$ \overline F\circ\overline G[z] = \overline F\parens{\sum n_i\ggen{\hat\sigma_i}} = \sum n_i[\hat\sigma_i] = \bracks{\sum n_i\hat\sigma_i} $$
So we need to show that if $\sum n_i\sigma_i$ is a cycle then $\sum n_i\hat\sigma_i\approx\sum n_i\sigma_i$.
Define $T\colon C_0(X)\longto C_1(X)$ by $T(p)=\gamma_p$, so
$$ \hat\sigma = \gamma_{\sigma0}*\sigma*\overline\gamma_{\sigma1} \approx \gamma_{\sigma0} + \sigma - \gamma_{\sigma1} = \sigma - T\partial\sigma $$
And so
$$ \sum n_i\hat\sigma_i \approx \sum n_i\sigma_i - T\partial\sum n_i\sigma_i = z - T\partial z $$
since $z$ is a cycle, $\partial z=0$ and so this is equal to $z$.
Thus $\hat z\approx z$ as required.

So $\overline F,\overline G$ are inverse isomorphisms, meaning $H_1(X)\cong\Ab\pi_1(X,a)$.
\qed

\bdefn

    Let $\C,\D$ be two categories and let $F,G\colon\C\longto\D$ be functors.
    Then a {\emphcolor natural transformation} $\eta$ from $F$ to $G$ is a correspondence such that
    \benum
        \item for every object $X\in\C$, $\eta$ associates a morphism $\eta_X\colon F(X)\longto G(X)$ called the {\emphcolor component} of $X$.
        \item for every $f\colon X\longto Y$ morphism, $\eta_Y\circ F(f)=G(f)\circ\eta_X$, i.e. the following diagram commutes

            \bigskip
            \centerline{\def\diagrowheight{1cm}\drawdiagram{
                $F(X)$ & $G(X)$\cr
                $F(Y)$ & $G(Y)$\cr
            }{
                \diagarrow{from={1,1}, to={1,2}, text=$\eta_X$, y distance=.25cm, color=rgb{.8 .1 .1}}
                \diagarrow{from={2,1}, to={2,2}, text=$\eta_Y$, y distance=-.25cm, color=rgb{.8 .1 .1}}
                \diagarrow{from={1,1}, to={2,1}, text=$F(f)$, x distance=-.5cm, color=rgb{.8 .1 .1}}
                \diagarrow{from={1,2}, to={2,2}, text=$G(f)$, x distance=.5cm, color=rgb{.8 .1 .1}}
            }}
    \eenum

\edefn

So for every pointed topology $(X,a)$ we defined a group homomorphism $F_{X,a}\colon\pi_1(X,a)\longto H_1(X)$.
We claim that this is a natural transformation from $\pi_1$ to $H_1$.

Suppose there is a morphism $h\colon(X,a)\longto(Y,b)$, so we need the following diagram to commute:

\bigskip
\centerline{\def\diagrowheight{1cm}\drawdiagram{
    $\pi_1(X,a)$ & $H_1(X)$\cr
    $\pi_1(Y,b)$ & $H_1(Y)$\cr
}{
    \diagarrow{from={1,1}, to={1,2}, text=$F_{X,a}$, y distance=.25cm}
    \diagarrow{from={2,1}, to={2,2}, text=$F_{Y,b}$, y distance=-.25cm}
    \diagarrow{from={1,1}, to={2,1}, text=$\pi_1(h)$, x distance=-.5cm}
    \diagarrow{from={1,2}, to={2,2}, text=$H_1(h)$, x distance=.5cm}
}}

This is indeed the case:
$$ \gen\phi \xvarrightarrow{\;F_{X,a}\;} [\phi] \xvarrightarrow{\;H_1(h)\;} [h\circ\phi],\qquad \gen\phi \xvarrightarrow{\;\pi_1(h)\;} \gen{h\circ\phi} \xvarrightarrow{\;F_{Y,b}\;} [h\circ\phi] $$

\bexam

    If we look at the identity functor (on the category of groups) and Abelianization, then $\rho_\bullet$, which is the quotient map $\bullet\longto\Ab\bullet$, is a natural transformation.
    Indeed

    \bigskip
    \centerline{\def\diagrowheight{1cm}\drawdiagram{
        $G$ & $\Ab G$\cr
        $H$ & $\Ab H$\cr
    }{
        \diagarrow{from={1,1}, to={1,2}, text=$\rho_G$, y distance=.25cm, color=rgb{.1 .65 .1}}
        \diagarrow{from={2,1}, to={2,2}, text=$\rho_H$, y distance=-.25cm, color=rgb{.1 .65 .1}}
        \diagarrow{from={1,1}, to={2,1}, text=$\phi$, x distance=-.25cm, color=rgb{.1 .65 .1}}
        \diagarrow{from={1,2}, to={2,2}, text=$\hat\phi$, x distance=.25cm, color=rgb{.1 .65 .1}}
    }}

    Where $\hat\phi[g]=[\phi(g)]$.
    This is indeed natural:
    $$ \rho_H\circ\phi(g) = [\phi(g)],\qquad \hat\phi\circ\rho_G(g)=\hat\phi[g] = [\phi(g)] $$

\eexam

\bdefn

    The {\emphcolor simplified singular chain complex} of a topological space $X$ is the chain complex

    \bigskip
    \centerline{\drawdiagram{
        $\cdots$ & $C_{n+1}(X)$ & $C_n(X)$ & $C_{n-1}(X)$ & $\cdots$ & $C_0(X)$ & ${\bb Z}$ & $0$\cr
    }{
        \diagarrow{from={1,1}, to={1,2}, dashed, color=rgb{.8 .1 .1}}
        \diagarrow{from={1,2}, to={1,3}, text=$\partial_{n+1}$, y distance=.25cm, color=rgb{.8 .1 .1}}
        \diagarrow{from={1,3}, to={1,4}, text=$\partial_{n}$, y distance=.25cm, color=rgb{.8 .1 .1}}
        \diagarrow{from={1,4}, to={1,5}, dashed, color=rgb{.8 .1 .1}}
        \diagarrow{from={1,5}, to={1,6}, dashed, color=rgb{.8 .1 .1}}
        \diagarrow{from={1,6}, to={1,7}, text=$\epsilon$, y distance=.25cm, color=rgb{.8 .1 .1}}
        \diagarrow{from={1,7}, to={1,8}, color=rgb{.8 .1 .1}}
    }}
    \medskip

    Where we define $\epsilon$ as follows:
    $$ \epsilon\sum n_ip_i = \sum n_i $$
    i.e. $\epsilon p=1$ for every $p\in X$.
    And a morphism between two simplified singular chain complexes differ only from morphisms between normal singular chain complexes in that the map from ${\bb Z}$ to ${\bb Z}$ is the identity.

    The homology induced by a simplified singular chain complex is called the {\emphcolor simplified homology} and denoted $\tilde H_n(X)$.

\edefn

Obviously for every $n\geq1$, $\tilde H_n(X)=H_n(X)$.

\bdefn

    A chain of Abelian groups

    \bigskip
    \centerline{\drawdiagram{
        $\cdots$ & $A$ & $B$ & $C$ & $\cdots$\cr
    }{
        \diagarrow{from={1,1}, to={1,2}, dashed, color=rgb{.8 .1 .1}}
        \diagarrow{from={1,2}, to={1,3}, text=$f$, y distance=.25cm, color=rgb{.8 .1 .1}}
        \diagarrow{from={1,3}, to={1,4}, text=$g$, y distance=.25cm, color=rgb{.8 .1 .1}}
        \diagarrow{from={1,4}, to={1,5}, dashed, color=rgb{.8 .1 .1}}
    }}
    \bigskip

    is {\emphcolor exact} at $B$ if ${\rm Im}f=\ker g$.
    If the sequence is exact at every group, then the sequence itself is called an {\emphcolor exact sequence}.
    (Recall that chain complexes require ${\rm Im}f\subseteq\ker g$.)

\edefn

If we have an exact sequence in one of the following forms, then:

\benum
    \item $0\longto A\xvarrightarrow{\;f\;}B$, then $0=\ker f$ so $f$ is injective.
    \item $A\xvarrightarrow{\;f\;}B\longto0$, then ${\rm Im}f=B$ so $f$ is surjective.
    \item $0\longto A\xvarrightarrow{\;f\;}B\longto0$, then $f$ is an isomorphism.
\eenum

\bdefn

    A {\emphcolor short exact sequence} is an exact sequence of the form
    $$ 0\longto A\xvarrightarrow{\;f\;}B\xvarrightarrow{\;g\;}C\longto0 $$

\edefn

In a short exact sequence, by above $f$ is injective and $g$ is surjective, and furthermore ${\rm Im}f=\ker g$.
In such a case, we can view $A$ as being a subgroup of $B$ (since $f$ is an embedding) and since by the isomorphism theorem $C\cong\slfrac B{\ker g}=\slfrac B{{\rm Im}f}=\slfrac BA$, a short exact sequence
can be viewed as
$$ 0 \longto A\xvarrightarrow{\;{\it inclusion}\;}B\xvarrightarrow{\;{\it quotient}\;}\slfrac BA\longto0 $$

\localcolor{white}{This next lemma took fucking forever to \TeX{} out, so you'd better appreciate me!}

\blemm[title=The Lemma of Five]

    Suppose the chains $\set{A_i}_i,\set{B_i}_i$ are exact, and the following diagram commutes:

    \medskip
    \centerline{\def\diagrowheight{1cm}\def\diagcolwidth{1cm}\drawdiagram{
        $A_1$ & $A_2$ & $A_3$ & $A_4$ & $A_5$\cr
        $B_1$ & $B_2$ & $B_3$ & $B_4$ & $B_5$\cr
    }{
        \diagarrow{from={1,1}, to={1,2}, color=rgb{.8 .1 .8}}
        \diagarrow{from={1,2}, to={1,3}, color=rgb{.8 .1 .8}}
        \diagarrow{from={1,3}, to={1,4}, color=rgb{.8 .1 .8}}
        \diagarrow{from={1,4}, to={1,5}, color=rgb{.8 .1 .8}}
        \diagarrow{from={2,1}, to={2,2}, color=rgb{.8 .1 .8}}
        \diagarrow{from={2,2}, to={2,3}, color=rgb{.8 .1 .8}}
        \diagarrow{from={2,3}, to={2,4}, color=rgb{.8 .1 .8}}
        \diagarrow{from={2,4}, to={2,5}, color=rgb{.8 .1 .8}}
        \diagarrow{from={1,1}, to={2,1}, color=rgb{.8 .1 .8}, text=$f_1$, x distance=-.25cm}
        \diagarrow{from={1,2}, to={2,2}, color=rgb{.8 .1 .8}, text=$f_2$, x distance=-.25cm}
        \diagarrow{from={1,3}, to={2,3}, color=rgb{.8 .1 .8}, text=$f_3$, x distance=-.25cm}
        \diagarrow{from={1,4}, to={2,4}, color=rgb{.8 .1 .8}, text=$f_4$, x distance=-.25cm}
        \diagarrow{from={1,5}, to={2,5}, color=rgb{.8 .1 .8}, text=$f_5$, x distance=-.25cm}
    }}

    \kern-.5cm
    \localcolor rgb{1 .9 1}{That actually wasn't too bad :)}

    \benum
        \item If $f_2,f_4$ are injective and $f_1$ is surjective, then $f_3$ is injective.
        \item If $f_2,f_4$ are surjective and $f_5$ is injective, then $f_3$ is surjective.
    \eenum

\elemm

\Proof We write $x\xvarmapsto Ay$ to mean $x$ maps to $y$ in the exact sequence ($x\in A_i$).
\benum
    \item Suppose $f_3a$, then $a\xmaps{f_3}0\xmaps{B}0$, now suppose $a\xmaps{A}b\xmaps{f_4}c$.
        Since the diagram commutes, we must have that $c=0$, but $f_4$ is injective so $b=0$.
        This means $a\in\ker A$, so there exists some $d$ such that $d\xmaps{A}a$.
        Suppose $d\xmaps{f_2}e$, then $e\xmaps{B}0$ by commutativity, so there exists an $f$ such that $f\xmaps{B}e$, and since $f_1$ is surjective there exists a $g\xmaps{f_1}f$.
        Now suppose $g\xmaps{A}h$.
        By commutativity, since $g\xmaps{f_1}f\xmaps{B}e$ we have $f_2h=e$ and since $f_2$ is injective, $h=d$.
        So $d$ is in the image of $A$, so it is in the kernel and so $a=0$.
    \item is a little more complicated, but it's just chasing.
        \qed
\eenum

\bdefn

    Suppose $\C$ and $\D$ are two chain complexes, with two morphisms $f,g\colon\C\longto\D$.
    Then a {\emphcolor chain homotopy} from $f$ to $g$ is a sequence of maps $T_n\colon C_n\longto D_{n+1}$ such that $\partial T+T\partial=f-g$.
    If there exists a chain homotopy between $f$ and $g$, we write $f\hch g$.

\edefn

In a diagram, we have that $T$ are the red arrows.

\bigskip
\centerline{\def\diagrowbuf{1cm}\def\diagcolbuf{1cm}\drawdiagram{
    $\cdots$ & $C_{n+1}$ & $C_n$ & $C_{n-1}$ & $\cdots$\cr
    $\cdots$ & $D_{n+1}$ & $D_n$ & $D_{n-1}$ & $\cdots$\cr
}{
    \diagarrow{from={1,1}, to={1,2}, dashed}
    \diagarrow{from={1,2}, to={1,3}}
    \diagarrow{from={1,3}, to={1,4}}
    \diagarrow{from={1,4}, to={1,5}, dashed}
    \diagarrow{from={2,1}, to={2,2}, dashed}
    \diagarrow{from={2,2}, to={2,3}}
    \diagarrow{from={2,3}, to={2,4}}
    \diagarrow{from={2,4}, to={2,5}, dashed}
    \diagarrow{from={1,2}, to={2,2}, text=$f_{n+1}$, x distance=-.75cm, x off=-.25cm}
    \diagarrow{from={1,3}, to={2,3}, text=$f_{n}$, x distance=-.5cm, x off=-.25cm}
    \diagarrow{from={1,4}, to={2,4}, text=$f_{n-1}$, x distance=-.75cm, x off=-.25cm}
    \diagarrow{from={1,2}, to={2,2}, text=$g_{n+1}$, x distance=.5cm}
    \diagarrow{from={1,3}, to={2,3}, text=$g_{n}$, x distance=.25cm}
    \diagarrow{from={1,4}, to={2,4}, text=$g_{n-1}$, x distance=.5cm}
    \diagarrow{from={1,2}, to={2,1}, dashed, color=red}
    \diagarrow{from={1,3}, to={2,2}, color=red}
    \diagarrow{from={1,4}, to={2,3}, color=red}
}}
\medskip

Let $X\subseteq{\bb R}^k$ be convex.
For $a\in X$ let us define the {\it cone construction} $C_a\colon C_n(X)\longto C_{n+1}(X)$ as follows: we start with generators of $C_n(X)$, i.e. we define $C_a\sigma$ for $\sigma\colon\Delta^n\longto X$
an $n$-simplex.
Geometrically, $C_a\sigma$ will be a cone whose tip is $a$ and whose base is $\sigma$.
We define this by:
$$ C_a\sigma(t_0,\dots,t_{n+1}) = t_0b + (1-t_0)\sigma\parens{\frac{t_1}{1-t_0},\dots,\frac{t_{n+1}}{1-t_0}} $$
Let us now compute the faces of $C_a\sigma$.
For $i=0$ then
$$ (C_a\sigma)\tau_0^{n+1}(t_0,\dots,t_n) = C_a\sigma(0,t_0,\dots,t_n) = \sigma(t_0,\dots,t_n) $$
For $i>0$ then
$$ (C_a\sigma)\tau_i^{n+1}(t_0,\dots,t_n) = C_a\sigma(t_0,\dots,0,\dots,t_n) $$
if $t_0=1$ as well, then this is just
$$ C_a\sigma(1,0,\dots,0) = a $$
Otherwise,
$$ \eqalign{
    &= t_0b + (1-t_0)\sigma\parens{\frac{t_1}{1-t_0},\dots,0,\dots,\frac{t_n}{1-t_0}}\cr
    &= t_0b + (1-t_0)\sigma\tau^n_{i-1}\parens{\frac{t_1}{1-t_0},\dots,\frac{t_n}{1-t_0}}\cr
    &= C_a^{n-1}(\sigma\tau^n_{i-1})(t_0,\dots,t_n)
} $$
So we see that
$$ (C_a\sigma)\tau_0^{n+1} = \sigma,\qquad (C_a\sigma)\tau_i^{n+1} = C_a^{n-1}(\sigma\tau^n_{i-1}) $$
So
$$ \eqalign{
    \partial_{n+1} C_a^n(\sigma) = \sum_{i=0}^{n+1}(-1)^i(C_a^n\sigma)\tau_i^{n+1} &= \sigma + \sum_{i=1}^{n+1}C_a^{n-1}(\sigma\tau^n_{i-1})\cr
    &= \sigma - \sum_{i=0}^n(-1)^iC_a^{n-1}(\sigma\tau^n_i)\cr
    &= \sigma - C_a^{n-1}\parens{\sum_{i=0}^n(-1)^i\sigma\tau^n_i}\cr
    &= \sigma - C_a^{n-1}\partial_n\sigma
} $$
So we see that
$$ \partial C_a - C_a\partial = {\rm id} $$
so in other words, $C_a$ is a chain homotopy from ${\rm id}$ to $0$.

\bthrm

    Let $X$ be a convex set in ${\bb R}^k$, then for all $n>0$, $H_n(X)=0$.

\ethrm

\Proof let $\gamma\in C_n(X)$, then $\gamma=\partial C_a\gamma + C_a\partial\gamma$.
If $\gamma\in Z_n(X)$, i.e. it is a cycle, then $\partial\gamma=0$ and so $\gamma=\partial C_a\gamma$.
Thus $\gamma\in B_n(X)$, so $Z_n(X)=B_n(X)$, and then $H_n(X)=0$.
\qed

\blemm

    If $f,g\colon X\longto Y$ are two homotopic continuous maps, then $f_\sharp$ and $g_\sharp$ are chain homotopic.

\elemm

\Proof let us define $i,j\colon X\longto X\times I$ where $i(x)=(x,0)$ and $j(x)=(x,1)$.
If $H\colon X\times I\longto Y$ is a homotopy from $f$ to $g$, then $f=H\circ i$ and $g=H\circ j$.
Also $i\sim j$, so if we can show that $i_\sharp\hch j_\sharp$ then we have that
$$ f_\sharp = H_\sharp\circ i_\sharp \hch H_\sharp\circ j_\sharp = g_\sharp $$
so it is sufficient to show that $i_\sharp\hch j_\sharp$.

So we need to define a sequence of morphisms $T_n^X\colon C_n(X)\longto C_{n+1}(X\times I)$ such that $\partial T^X+T^X\partial=i^X_\sharp-j^X_\sharp$.
We will define $T^X_n$ by induction on $n$, such that $T^X$ is natural.
Natural between what two functors?
The first functor maps topological spaces $X$ to their chain complexes $\C(X)$ and maps morphisms $X\xvarrightarrow{\,f\,}Y$ to $f_\sharp\colon\C(X)\longto\C(Y)$.
The second functor maps topological spaces $X$ to the chain complex $C_{n+1}(X\times I)$ and morphisms $X\xvarrightarrow{\,f\,}Y$ to $(f\times{\rm id})_\sharp\colon C_{n+1}(X)\longto C_{n+1}(Y)$.

$T^X$ being natural means that the diagram commutes for all $f\colon X\longto Y$:

\bigskip
\centerline{\def\diagrowbuf{.5cm}\def\diagcolbuf{.5cm}
\drawdiagram{
    $C_n(X)$ & $C_{n+1}(X\times I)$\cr
    $C_n(Y)$ & $C_{n+1}(Y\times I)$\cr
}{
    \diagarrow{from={1,1}, to={1,2}, text=$T^X$, y distance=.25cm}
    \diagarrow{from={2,1}, to={2,2}, text=$T^Y$, y distance=-.25cm}
    \diagarrow{from={1,1}, to={2,1}, text=$f_\sharp$, x distance=-.25cm}
    \diagarrow{from={1,2}, to={2,2}, text=$(f\times{\rm id})_\sharp$, x distance=.75cm}
}}
\medskip

So $T_Y\circ f_\sharp=(f\times{\rm id})_\sharp\circ T_X$.

Let $I_n\colon\Delta^n\longto\Delta^n$ be the identity $n$-dimensional simplex.
If we determine $T^{\Delta^n}(I_n)$, then we have determined $T^X(\sigma)$ for all $\sigma\in C_n(X)$ for all $X$.
This is because $\sigma=\sigma\circ I_n=\sigma_\sharp(I_n)$, since we can view $\sigma$ as a continuous map $X\longto\Delta^n$ and so $\sigma_\sharp$ is defined.
Thus
$$ T^X(\sigma) = T^X\circ\sigma_\sharp(I_n) = (\sigma\times{\rm id})_\sharp\circ T^{\Delta^n}(I_n) $$
And so determining $T^{\Delta^n}(I_n)$ determines $T^X(\sigma)$.
So if we define $A=T^{\Delta^n}(I_n)$, then
$$ T^X(\sigma)=(\sigma\times{\rm id})_\sharp(A) $$
$A$ is some simplex in $C_{n+1}(\Delta^n\times I)$, and we claim that for any choice of $A$, this defines a natural transformation.

This is because
$$ T^Y\circ f_\sharp(\sigma) = T^Y(f\circ\sigma) = \bigl((f\circ\sigma)\times{\rm id}\bigr)_\sharp(A) = (f\times{\rm id})_\sharp\circ(\sigma\times{\rm id})_\sharp(A) $$
And
$$ (f\times{\rm id})_\sharp\circ T^X(\sigma) = (f\times{\rm id})_\sharp\circ(\sigma\times{\rm id})_\sharp(A) $$
so these are indeed equal, as required.

Now we claim that
$$ (\partial T^X + T^X\partial)(\sigma) = (i^X_\sharp - j^X_\sharp)(\sigma) $$
for all $X,\sigma$.
It is sufficient to show this for $X=\Delta^n$ and $\sigma=I_n$, since if
$$ (\partial T^{\Delta^n} + T^{\Delta^n}\partial)(I_n) = (i^{\Delta^n}_\sharp - j^{\Delta^n}_\sharp)(I_n) $$
if we compose it on the left with $(\sigma\times{\rm id})_\sharp$, the LHS gives
$$ (\partial(\sigma\times{\rm id})_\sharp T^{\Delta^n} + (\sigma\times{\rm id})_\sharp T^{\Delta^n}\partial)(I_n) = (\partial T^X\sigma_\sharp + T^X\partial\sigma_\sharp)(I_n)
= \partial T^X\sigma + T^X\partial\sigma $$
since $T$ is natural, $T^Y\circ f_\sharp=(f\times{\rm id})_\sharp\circ T^X$ and $\partial f_\sharp=f_\sharp\partial$.
The RHS is
$$ \bigl((\sigma\times{\rm id})_\sharp\circ i^{\Delta^n}_\sharp - (\sigma\circ{\rm id})\circ j_\sharp^{\Delta^n}\bigr)(I_n) $$
Now notice that

\centerline{
\drawdiagram{
    $\Delta^n$ & $\Delta^n\times I$ & $X\times I$\cr
    $s$ & $(s,0)$ & $(\sigma(s),0)$\cr
}{
    \diagarrow{from={1,1}, to={1,2}, text=$i^{\Delta^n}$, y distance=.25cm}
    \diagarrow{from={1,2}, to={1,3}, text=$\sigma\times{\rm id}$, y distance=.25cm}
    \diagarrow{from={2,1}, to={2,2}, left cap=|-}
    \diagarrow{from={2,2}, to={2,3}, left cap=|-}
}}

So $(\sigma\times{\rm id})\circ i^{\Delta^n}=i^X\circ\sigma$, and similar for $j$.
So the RHS is just
$$ i^X_\sharp\circ\sigma_\sharp(I_n) - j^X_\sharp\circ\sigma_\sharp(I_n) = i^X_\sharp(\sigma) - j^X_\sharp(\sigma) $$
So we get
$$ \partial T^X(\sigma) + T^X\partial\sigma = i^X_\sharp(\sigma) - j^X_\sharp(\sigma) $$
as required.
