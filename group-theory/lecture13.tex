\documentclass[10pt]{article}

\usepackage{amsmath, amssymb, mathtools}
\usepackage[margin=1.5cm]{geometry}

\usepackage{tikz-cd}

\catcode`\@=11
\input pdfmsym

\input prettyprint
\input preamble

\pdfmsymsetscalefactor{10}
\initpps

\def\slfrac#1#2{\left.{}^{#1}\mkern-4mu\middle/\mkern-3mu{}_{#2}\right.}

\def\pmat#1{\begin{pmatrix} #1 \end{pmatrix}}

\let\divides=\mid

\font\bigbf = cmbx12 scaled 2000

\newfunc{eulerg}{{\rm Euler}}({})
\newfunc{order}o({})
\newfunc{ker}{{\rm Ker}}({})
\newfunc{image}{{\rm Im}}({})
\newfunc{lattice}{{\cal L}}(\vert)
\newfunc{powset}{{\cal P}}(\vert)
\newfunc{units}{{\cal U}}(\vert)
\newfunc{sign}{{\rm sgn}}({})
\newfunc{aut}{{\rm Aut}}({})
\newfunc{speclinear}{{\rm SL}}({})
\newfunc{genlinear}{{\rm GL}}({})
\newfunc{projlinear}{{\rm PGL}}({})
\newfunc{specprojlinear}{{\rm PSL}}({})
\newfunc{core}{{\rm Core}}(\vert)
\newfunc{inner}{{\rm Inn}}(\vert)
\newfunc{outer}{{\rm Out}}(\vert)
\newfunc{conj}{{\rm conj}}({})
\newfunc{fp}{{\rm FP}}({})

\def\maxdivs{\mathrel{\|}}

\let\normalsub=\trianglelefteq
\def\normal{\mathrel\triangleleft}
\let\mmorph=\longvarhookrightarrow

\begin{document}

\c@section=13

\barcolorbox{220, 220, 255}{0, 0, 130}{80, 80, 200}{
    \leftskip=0pt plus 1fill \rightskip=\leftskip
    {\bigbf Group Theory}

    \medskip
    \textit{Lecture \thesection, Friday January 20 2023}

    \textit{Ari Feiglin}
}

\bigskip

\subsection{Subgroups}

\begin{defn*}

    A group is \ppemph{solvable} if it has a subnormal series with abelian factors.

\end{defn*}

Note that this means that an abelian group is necessarily solvable since the series $\set e\normal A$ is a subnormal series whose factor is isomorphic to $A$.
We can further prove inductively that a finite abelian group has cyclic factors, we start by taking a non-trivial element $a\in A$ and constructing $\set e\normal\cycle a\normal A$.
Then the factor $\slfrac A{\cycle a}$ is abelian and so inductively has a subnormal series with cyclic factors:
\[ \set{\cycle a} = \slfrac{A_n}{\cycle a} \normal \cdots \normal \slfrac{A_0}{\cycle a} = \frac{A}{\cycle a} \]
note that these factors are
\[ \slfrac{\slfrac{A_i}{\cycle a}}{\slfrac{A_{i+1}}{\cycle a}} \cong \slfrac{A_i}{A_{i+1}} \]
thus the quotient of 
\[ \set e \normal \cycle a = A_n \normal \cdots \normal A_0 = A \]
are cyclic, and this is a subnormal series of cyclic factors, we can further refine this series to a series of cyclic groups of prime order (since these are the only simple cyclic groups).
Thus, if a finite group is solvable, it has a subnormal series of abelian factors, which in turn have subnormal series with factors of cyclic groups of prime order.
This condition is obviously sufficient, and so we summarize this result in the following proposition:

\begin{prop*}

    A finite group is solvable if and only if it has a subnormal series with factors which are cyclic groups of prime order.

\end{prop*}

\begin{lemm*}

    Suppose $A,B,C\leq G$ where $C$ is normal and $A\subseteq B$, then
    \[ (B\cap C)A = B\cap(CA) \]

\end{lemm*}

Note that this multiplication is a group because $C$ is normal.

\begin{proof}

    We already know $(B\cap C)A\subseteq B\cap(CA)$ since $B\cap C\subseteq B, CA$ and $A\subseteq B,CA$.
    Now suppose $b\in B\cap(CA)$, then $b=ca$ for some $c\in C$ and $a\in A$.
    Then $c=ba^{-1}\in B$, so $c\in B\cap C$ so $ca\in(B\cap C)A$ as required.

    \hfill$\blacksquare$

\end{proof}

\begin{prop*}

    Suppose $N$ is a normal subgroup of $G$, then $G$ is solvable if and only if both $N$ and $\slfrac GN$ are solvable.

\end{prop*}

\begin{proof}

    First, suppose $G$ is solvable, so there is a subnormal series
    \[ \set e = G_t \normal G_{t-1} \normal \cdots \normal G_1 \normal G_0 = G \]
    where $\slfrac{G_i}{G_{i+1}}$ are abelian.
    Then we can take the two series: $G_{i+1}\cap N\normal G_i\cap N$, and $\slfrac{G_{i+1}\cdot N}N\normal\slfrac{G_i\cdot N}N$.
    Note that by the lemma above:
    \[ (G_i\cap N)G_{i+1} = G_i\cap(N\cdot G_{i+1}) \]
    And since $G_{i+1}\cap N=G_{i+1}\cap(G_i\cap N)$ (since $G_{i+1}\subseteq G_i$) so:
    \[ \slfrac{G_i\cap N}{G_{i+1}\cap N} = \slfrac{G_i\cap N}{G_{i+1}\cap(G_i\cap N)} \cong \slfrac{(G_i\cap N)G_{i+1}}{G_{i+1}} = \slfrac{G_i\cap(N\cdot G_{i+1}}{G_{i+1}} \leq \slfrac{G_i}{G_{i+1}} \]
    and so the factors of $G_i\cap N$ are abelian, as the subgroups of abelian groups.
    This defines a subnormal series with abelian factors:
    \[ \set e = G_t\cap N \normal G_{t-1}\cap N \normal \cdots \normal G_1\cap N \normal G_0\cap N = G\cap N = N \]
    so $N$ is solvable, as required.

    And:
    \[ \slfrac{\slfrac{G_iN}{N}}{\slfrac{G_{i+1}N}{N}} \cong \slfrac{G_iN}{G{i+1}N} \cong \slfrac{G_i}{G_i\cap(G_{i+1}N)} = \slfrac{G_i}{(G_i\cap N)G_{i+1}} \]
    and this can be embedded in $\slfrac{G_i}{G_{i+1}}$, so it is also abelian.
    Thus the subnormal series:
    \[ \set{G} = \slfrac{G_tN}{N} \normal \slfrac{G_{t-1}N}N \normal \cdots \normal \slfrac{G_1N}N \normal \slfrac{G_0N}N = \slfrac{G}N \]
    has abelian factors and so $\slfrac GN$ is solvable.

    To show the converse, suppose that there are subnormal series
    \begin{gather*}
        \set e = N_s \normal \cdots \normal N_1 \normal N_0 = N \\
        \set N = B_r \normal \cdots \normal B_1 \normal B_0 = \slfrac GN
    \end{gather*}
    with abelian factors.
    Since $B_i$ is a subgroup of $\slfrac GN$ there is some subgroup $N\leq H_i\leq G$ such that $B_i=\slfrac{H_i}N$.
    And so
    \[ \set N = \slfrac{H_r}N \normal \cdots \normal \slfrac{H_1}N \normal \slfrac{H_0}N = \slfrac GN \]
    This means that
    \[ N = H_r \normal \cdots \normal H_1 \normal H_0 = G \]
    where the factors are abelian.
    Appending this to $N$'s series gives
    \[ \set e = N_s \normal \cdots \normal N_1 \normal N_0 = N = H_r \normal \cdots \normal H_1 \normal H_0 = G \]
    which has abelian factors (since it is concatenation of two series with abelian factors) and so $G$ is solvable, as required.

    \hfill$\blacksquare$

\end{proof}

We could use this as an alternative definition for the solvability of a finite group.

\begin{exam}

    We can construct the following subnormal series:
    \[ 1 \overset{\bZ_3}\normal A_3 \overset{\bZ_2}\normal S_3 \]
    so $S_3$ is solvable, and
    \[ 1 \overset{\bZ_2\times\bZ_2}\normal K_4 \overset{\bZ_3}\normal A_4\overset{\bZ_2}\normal S_4 \]
    so is $S_4$.
    But for $n\geq5$ then since $A_n$ is simple and the only normal subgroup of $S_n$'s then the only non-trivial subnormal series is
    \[ 1\overset{A_n}\normal A_n\overset{\bZ_2}\normal S_n \]
    which doesn't have abelian factors, so $S_n$ is not solvable for $n\geq5$.

\end{exam}

\subsection{Commutator Subgroups}

\begin{defn*}

    For $a,b\in G$, their \ppemph{commutator} is
    \[ [a,b] = aba^{-1}b^{-1} \]
    We define the commutator of two subgroups $A$ and $B$ to be
    \[ [A,B] = \cycle{[a,b]}[a\in A, b\in B] \]
    The group generated by all commutators of $G$ is defined to be
    \[ G' = [G, G] = \cycle{[a,b]}[a,b\in G] \]
    this is the \ppemph{commutator subgroup} of $G$.

\end{defn*}

Note that $[a,b]$ is trivial if and only if $a$ and $b$ commute.
Further note that
\[ [a,b]\cdot[b,a] = aba^{-1}b^{-1}bab^{-1}a^{-1} = e \]
so $[a,b]^{-1}=[b,a]$, so commutators are closed under inversions, so $G'$ is simply the group of all finite products of commutators.
We will quickly prove some basic characteristics of the commutator subgroup

\benum
    \item $G'$ is normal: suppose $c\in G$ then
    \[ c[a,b]c^{-1} = caba^{-1}b^{-1}c^{-1}=(cac^{-1})(cbc^{-1})(cac^{-1})^{-1}(cbc^{-1})^{-1}=\bigl[cac^{-1},cbc^{-1}\bigr] \]
    and since elements of $G'$ are products of commutators, and this holds for products of commutators, $G'$ is normal.

    \item $G'=\set e$ if and only if $G$ is abelian, this is trivial.

    \item If $K\normalsub G$ then
    \[ \parens{\slfrac GK}' = \cycle{[aK,bK]}[a,b\in G] = \cycle{[a,b]K}[a,b\in G] = \slfrac{G'K}K \]

    \item And so
    \[ \parens{\slfrac G{G'}}' = \slfrac{G'G'}{G'} = \slfrac{G'}{G'} = 1 \]
    So $\slfrac G{G'}$ is abelian.

    \item And $\slfrac GK$ is abelian if and only if $\slfrac{G'K}{K}=1$ which is if and only if $G'K=K$ which occurs if and only if $G'\subseteq K$.
    That is, $\slfrac GK$ is abelian if and only if $G'\subseteq K$.
    So $G'$ is the minimal normal subgroup whose quotient is abelian, and so $\slfrac{G}{G'}$ is the maximal abelian quotient, this is called the \emph{abelization}.

    \item If $A,B\normalsub G$ then $A\ni a(ba^{-1}b^{-1})=(aba^{-1})b^{-1}\in B$, so $[A,B]\leq A\cap B$.
    Furthermore, $c[a,b]c^{-1}=\bigl[cac^{-1},cbc^{-1}\bigr]\in[A,B]$, so $[A,B]$ is normal.
\eenum

\begin{exam}

    Notice that $S_n'\leq A_n$ since the sign of a commutator must be even since it is multiplicative.
    And since $\bigl[\pmat{i & j}, \pmat{i & k}\bigr] = \pmat{i & j & k}$, so all the three-cycles are commutators.
    $A_n$ is generated by the three-cycles, and so every element of $A_n$ is a product of commutators and therefore in $S_n'$.

    So $S'_n=A_n$.
    And so $\slfrac{S_n}{A_n}\cong\bZ_2$ is the maximal abelian quotient.

\end{exam}

\begin{defn*}

    The \ppemph{derived series} of a group $G$ is
    \[ \cdots \normal G''' \normal G'' \normal G' \normal G \]
    we further recursively define $G^{(n+1)}=\bigl(G^{(n)}\bigr)'$, where $G^{(0)}=G$.

\end{defn*}

\begin{thrm*}

    A group $G$ is solvable if and only if $G^{(n)}=1$ for some $n$.

\end{thrm*}

\begin{proof}

    If $G^{(n)}=1$ then the derived series
    \[ 1 = G^{(n)} \normal G^{(n-1)} \normal \cdots \normal G^{(1)} \normal G^{(0)} = G \]
    is a subnormal series whose quotients are abelian (recall that $\slfrac{G}{G'}$ is abelian), so $G$ is solvable.

    To show the converse, suppose $G$ is solvable, so it has a subnormal series
    \[ 1 = G_t \normal \cdots \normal G_1 \normal G_0 = G \]
    where $\slfrac{G_i}{G_{i+1}}$ is abelian, so $G_i'\subseteq G_{i+1}$, thus $G'\subseteq G_1$, and so $G''\subseteq G_1'\subseteq G_2$, and inductively $G^{(i)}\subseteq G_i$.
    So $G^{(t)}\subseteq G_t=1$, as required.

    \hfill$\blacksquare$

\end{proof}

Notice that we showed in this proof that the derived series is the quickest subnormal series with abelian factors.

\begin{coro*}

    If $G$ is solvable, so are all of its subgroups.

\end{coro*}

This is because if $H\leq G$, $H^{(n)}\subseteq G^{(n)}=1$, so $H$ is solvable by the theorem above.

Note that $(A\times B)'=A'\times B'$ since the group operation is done elementwise.
And more general:
\[ \parens{\prod_{\lambda\in\Lambda}A_\lambda} = \prod_{\lambda\in\Lambda}A_\lambda' \]
Thus $\prod A_\lambda$ is solvable if and only each $A_\lambda$ is solvable in $n$th class for a constant $n$ (meaning $A_\lambda^{(n)}=1$).

\subsection{Nilpotent Groups}

\begin{defn*}

    A \ppemph{central series} is a normal series (every element is normal in $G$) where for every $i$:
    \[ \slfrac{G_i}{G_{i+1}} \leq Z\parens{\slfrac{G}{G_{i+1}}} \]

\end{defn*}

In order to give an example of such a series, we define the \emph{lower central series} where $G_{(1)}=G$ and recursively $G_{(n)}=[G,G_{(n-1)}]\subseteq G_{(n-1)}$.
This is inductively a normal series (since $[A,B]$ is normal when $A$ and $B$ are).
Now notice that $\slfrac BA\in Z\parens{\slfrac GA}$ if and only if for every $g\in G$ and $b\in B$, $bA$ and $gA$ commute, that is $[bA,gA]=[b,g]A=1$.
So $\slfrac BA\in Z\parens{\slfrac GA}$ if and only if $1=\bracks{\slfrac BA,\slfrac GA}=\slfrac{[B,G]A}A$ which is if and only if $[B,G]\subseteq A$ (the order of $B$ and $G$ doesn't matter here).
So a normal series is central if and only if for every $i$:
\[ \bigl[G, G_i\bigr] \subseteq G_{i+1} \]
Thus by definition the lower central series is in fact a central series.
We define this inductively: $\zeta_0=1$ and $\zeta_{i+1}=\set{x\in G}[\forall y\in G: [x,y]\in\zeta_i]$, notice here that inductively $\zeta_i$ are all normal, and so if $x\in\zeta_i$ then for any $y\in G$,
$[x,y]\in\zeta_i$ so $\zeta_i\leq\zeta_{i+1}$.
So here we have an ascending series
\[ 1 = \zeta_0 \normal \zeta_1 \normal \zeta_2 \normal \cdots \]
This is central since
\[ \bracks{G, \zeta_{i+1}} \subseteq \zeta_i \]
by definition (we reverse $i$ and $i+1$ since the series is descending).
But notice that if $x\zeta_i\in Z\parens{\slfrac G{\zeta_i}}$ then for every $y\zeta_i\in\slfrac G{\zeta_i}$ we have that
\[ \zeta_i = \bracks{x\zeta_i, y\zeta_i} = \bracks{x,y}\zeta_i \]
so $[x,y]\in\zeta_i$ and then by definition $x\in\zeta_{i+1}$, so $x\zeta_i\in\slfrac{\zeta_{i+1}}{\zeta_i}$, and therefore we have that
\[ \slfrac{\zeta_{i+1}}{\zeta_i} = Z\parens{\slfrac G{\zeta_i}} \]

\begin{prop*}

    If $\cdots\normal H_2\normal H_1=G$ is a central series then $G_{(n)}\subseteq H_n$.
    And if $1=K_0\normal K_1\normal K_2\normal\cdots$ then $K_n\subseteq\zeta_n$.

\end{prop*}

This means that the lower central series is the quickest descending central series and the upper central series is the quickest ascending central series.

\begin{proof}

    We show this inductively (the base being $H_1=G_{(1)}=G$):
    \[ G_{(n+1)} = [G, G_{(n)}] \subseteq [G,H_n] \subseteq H_{n+1} \]
    where the last equality is because $H_n$ is a central series.

    Again we show this inductively: let $x\in K_{n+1}$ and $y\in G$ then $[x,g]\in K_n\subseteq\zeta_n$, so $\bracks{x\zeta_n,g\zeta_n}=1$ so
    $x\zeta_n\in Z\parens{\slfrac G{\zeta_n}}=\slfrac{\zeta_{n+1}}{\zeta_n}$, so $x\in\zeta_{n+1}$.

    \hfill$\blacksquare$

\end{proof}

\begin{defn*}

    A group is \ppemph{nilpotent} if it has a central sequence from $G$ to $1$.

\end{defn*}

So $G$ is nilpotent if there is a series
\[ 1 = H_n \normal H_{n-1} \normal \cdots \normal H_2 \normal H_1 = G \]
and we know that $G_{(i)}\subseteq H_i\subseteq\zeta_{n-i}(G)$ (we need to reindex $\zeta$ since it is defined as an increasing series).

\begin{prop*}

    $G$ is nilpotent if and only if there exists an $n$ and $m$ such that $G_{(n)}=1$ and $\zeta_m(G)=G$ (even one is sufficient).

\end{prop*}

\begin{proof}

    Suppose $G$ is nilpotent then $G_{(n)}\subseteq H_n=1$ and $G=H_1\subseteq\zeta_{n-1}$
    For the converse notice that the series (either $G_{(n)}$ or $\zeta_{m-n}$) defines a central series from $1$ to $G$, and thus $G$ is nilpotent by definition.

    \hfill$\blacksquare$

\end{proof}

Notice that every abelian group is nilpotent since $A_{(2)}=1$ and $\zeta_1(A)=A$.
Furthermore, since the factors of a central group are subgroups of the center of another group, they are necessarily abelian, and therefore every nilpotent group is solvable.

It isn't hard to see that $\zeta_i(G\times H)=\zeta_i(G)\times\zeta_i(H)$ and so the finite direct product of nilpotent groups is also nilpotent.
Note that by the definition of $\zeta_i$, $G$ is nilpotent if and only if $G=1$ or $Z(G)\neq1$ and $\slfrac G{Z(G)}$ is nilpotent.

\begin{lemm*}

    For every proper subgroup of a nilpotent group $H<G$, $H$ is a proper subgroup of its normalizer.

\end{lemm*}

\begin{proof}

    Take the maximal $n$ such that $\zeta_n(G)\subseteq H$ (meaning $\zeta_{n+1}\not\subseteq H$).
    We will show that $\zeta_{n+1}(G)\subseteq N_G(H)$.
    Let $x\in\zeta_{n+1}(G)$ and $h\in H$ then $xhx^{-1}=[x,h]h\in\bigl[\zeta_{n+1},G\bigr]\cdot H\subseteq\zeta_n\cdot H=\zeta_n$, where the final inequality comes from the necessary and sufficient
    condition for central series, and the final equality since $\zeta_n\subseteq H$.
    So $x\in N_G(H)$, and since $\zeta_{n+1}$ is not a subset of $H$, $H$ must be a proper subset of $N_G(H)$.

\end{proof}

\begin{thrm*}

    Let $G$ be a group, then the following are equivalent:
    \benum
        \item $G$ is nilpotent
        \item For every proper subgroup of $H$ of $G$, $H$ is a proper subgroup of its normalizer
        \item Every maximal subgroup is normal
        \item Every $p$-Sylow subgroup is normal
        \item $G$ is the direct product of $p$-groups
    \eenum

\end{thrm*}

\let\implies=\longvarRightarrow

\begin{proof}

    \blist
        \item $(1\implies2)$ We just proved this.
        \item $(2\implies3)$ Since $H\subset N_G(H)$, and $H$ is maximal, this means that $N_G(H)=G$, so $H$ is normal.
        \item $(3\implies4)$ If $P$ is a $p$-Sylow subgroup, suppose $P$ is not normal.
        Then $P\subseteq N_G(P)<G$, then let $N_G(P)\leq M<G$ be maximal.
        Then we know $N_G(M)=M$ so $M$ isn't normal but this is a contradiction.
        \item $(4\implies5)$ We proved this last lecture.
        \item $(5\implies1)$ We first show that a $p$-group is nilpotent.
        Since the center of a $p$-group is non-trivial, and $\slfrac{P}{Z(P)}$ is itself a $p$-group, this is proven inductively.
        So $G$ is the product of $p$-groups, specifically $G$ is the product of nilpotent groups, and is therefore itself nilpotent.
    \elist

    \hfill$\blacksquare$

\end{proof}

\end{document}

