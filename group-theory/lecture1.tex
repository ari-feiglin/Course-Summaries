\documentclass[10pt]{article}

\usepackage{amsmath, amssymb, mathtools}
\usepackage[margin=1.5cm]{geometry}

\input mathex
\input prettyprint
\input preamble

\setscalefactor{10}
\initpps

\def\pmat#1{\begin{pmatrix} #1 \end{pmatrix}}

\def\@blist[#1]{%
    \bgroup\bgroup\par\vskip-\medskipamount%
    \gdef\item{%
        \par\egroup\bgroup\medskip\setbox0=\hbox{#1\quad}%
        \advance\leftskip by \wd0\leavevmode\kern-\wd0\box0%
    }%
}
\def\blist{\@ifnextchar[ \@blist {\@blist[$\bullet$]}}
\def\elist{\par\egroup\egroup\medskip}

\let\divides=\mid

\font\bigbf = cmbx12 scaled 2000

\pdf@drawing@macro{check}{0 1 0 RG 1.2 w 1 j 1 J 0 5 m 3 0 l 8 10 l S}{9.6pt}{12pt}{1pt}{1pt}
\pdf@drawing@macro{ecks}{1 0 0 RG 0 0 m 1 j 1 J 10 10 l 0 10 m 10 0 l S}{11.6pt}{12pt}{1pt}{1pt}
\begin{document}

\c@section=1

\barcolorbox{220, 220, 255}{0, 0, 130}{80, 80, 200}{
    \leftskip=0pt plus 1fill \rightskip=\leftskip
    {\bigbf Group Theory}

    \medskip
    \textit{Lecture 1, Sunday October 23, 2022}

    \textit{Ari Feiglin}
}

\bigskip

\subsection{Introduction to Algebraic Structures}

\begin{defn*}

    A \ppemph{Magma} is a set $S$ with a \emph{binary operation}:
        \[ \circ\colon S\times S\longvarrightarrow S \]

\end{defn*}

Notice then that a Magma has no requirements on its binary operation, thus for a magma of size $n$ there are $n^{n^2}$ possible magmas.
Despite this, some of these Magmas are equivalent in some sense, as a renaming of the elements of one magma produces the other.
For example of a magma: $M=\set{0, 1}$:

\[ \begin{array}{c|cc} \circ & 0 & 1 \\\hline 0 & 1 & 1 \\ 1 & 1 & 1 \end{array} \cong
   \begin{array}{c|cc} \circ & 0 & 1 \\\hline 0 & 0 & 0 \\ 1 & 0 & 0 \end{array} \]

We say a magma is \emph{associative} if for every $x,y,z\in S$ it follows that:
\[ x\circ(y\circ z) = (x\circ y)\circ z \]

\begin{exam}

    Given a set $X$, the set $X^{X}$ is an associative magma under function composition.

\end{exam}

\begin{defn*}

    A \ppemph{Semigroup} is an associative magma.

\end{defn*}

So the set $X^X$ is a semigroup.
The set $M_n(\bF)$, the set of matrices of size $n\times n$, is also a semigroup under both addition and multiplication.

\begin{defn*}

    Let $(S,\circ)$ be a semigroup.
    An element $e\in S$ is a \ppemph{left-identity} if for every $a\in S$: $e\circ a=a$.
    Similarly, $e$ is a \ppemph{right-identity} if for every $a\in S$: $a\circ e=a$.
    $e$ is an \ppemph{identity} if it is both a left and right identity.

\end{defn*}

\begin{prop*}

    If $S$ is a semigroup with $e$ a left-identity and $e'$ a right-identity.
    Then $e=e'$.

\end{prop*}

\begin{proof}
    This is trivial.
    By definition of the left-identity: $e\circ e'=e'$, but by the definition of the right-identity, $e\circ e'=e$.
    So $e=e'$.

    \hfill$\blacksquare$
\end{proof}

\begin{exam}

    Let:
        \[ S\coloneqq\set{\begin{pmatrix} a & b \\ 0 & 0 \end{pmatrix}}[a,b\in\bR] \]
    equipped with the operation of multiplication.
    Note that if we alter the definition of $S$ slightly:
        \[ S^*\coloneqq\set{\pmat{a & b\\c & d}}[a,b,c\in\bR] \]
    this is not a magma, as multiplication is not well-defined over this set:
        \[ \pmat{1 & 1 \\ 1 & 0}\cdot\pmat{1 & 1 \\ 1 & 0} = \pmat{* & * \\ * & 1} \notin S^* \]
    But $S$ is a semigroup.
    Notice that:
        \[ \pmat{a & b \\ 0 & 0}\cdot\pmat{a' & b' \\ 0 & 0} = \pmat{aa' & ab' \\ 0 & 0} \]
    So in order for a matrix to be a left-identity, we must have that $aa'=a'$, so $a=1$.
    But there are no requirements for $b$ and so any matrix of the form
        \[ \pmat{1 & b \\ 0 & 0} \]
    is a left-identity.
    But for a matrix to be a right-identity, we must have that $aa'=a$ and $ab'=b$.
    But for any $a$ and $b$ we can find a $b'$ where this doesn't hold.
    So $S$ has an infinite number of left identities, but no right identities.

\end{exam}

Notice that if a semigroup has multiple left (or right) identities, it cannot have any right (or left) identities.
Suppose $e\neq e'$ are left identities and $t$ is a right identity.
Then by our proposition above $e=t$ and $e'=t$, so $e=e'$, which is a contradiction. \lightning

\medskip
\begin{center}
\begin{tabular}{c|ccc} \lower1ex\hbox{\footnotesize Left}\kern-1ex\raise1ex\hbox{\footnotesize Right} &  None & One & Many \\
        \hline
        None & \check & \check & \check \\
        One & \check & \check & \ecks \\
        Many & \check & \ecks & \ecks
\end{tabular}
\end{center}
\medskip

\begin{defn*}

    A semigroup $S$ is a \ppemph{Monoid} if it has an identity element.

\end{defn*}

\begin{prop*}

    If $S$ is a monoid, its identity element is unique.

\end{prop*}

\begin{proof}

    This too is trivial, as if $e$ and $e'$ are both identities, $e\circ e'=e=e'$.

    \hfill$\blacksquare$

\end{proof}

\begin{exam}

    $(\bN,\cdot)$ is a monoid whose identity is $1$.

    $(\bZ,-)$ is not even a semigroup as subtraction is not associative.

\end{exam}

\begin{defn*}

    If $M$ is a monoid with identity element $1$.
    If $a,b\in M$ such that $a\circ b=1$, then $a$ is \ppemph{right-invertible} and $b$ is \ppemph{left-invertible}.
    And we call $a$ $b$'s \ppemph{left inverse} and $b$ is $a$'s \ppemph{right inverse}.

    An element $a\in M$ is \ppemph{invertible} if there is a $b\in M$ such that $a\circ b=b\circ a=1$.
    $b$ is called $a$'s inverse and is denoted $a^{-1}$.

\end{defn*}

Note that the inverse of $a^{-1}$ is $a$, this comes directly from the definition.

\begin{prop*}

    An element $a$ of a monoid is invertible if and only if it is right and left invertible.

\end{prop*}

\begin{proof}

    By definition if $a$ is right and left invertible.
    For the converse suppose $a\circ b=c\circ a=1$.
    Then $c\circ(a\circ b)=c$, but $c\circ(a\circ b)=(c\circ a)\circ b=b$, so $c=b$ and therefore:
        \[ a\circ b = b\circ a = 1 \]
    So $a$ is invertible.

    \hfill$\blacksquare$

\end{proof}

\begin{defn*}[group,Group]

    If every element in a monoid $M$ is invertible, then $G$ is called a \ppemph{Group}.
    This means that:

        %\begin{itemize}
        \blist
            \item $G$'s binary operation $\circ$ is associative.
            \item $G$ has an identity element $e$ such that for every $a\in G$: $a\circ e=e\circ a=a$.
            \item For every $a\in G$, there exists an $a^{-1}\in G$ such that $a\circ a^{-1}=a^{-1}\circ a=e$.
        \elist

\end{defn*}

\begin{defn*}

    A monoid is \ppemph{left reducible} if for every $a,x,y\in M$, if $a\circ x=a\circ y$ then $x=y$.

\end{defn*}

Note then that a group is left (and right) reducible.

\begin{prop*}

    If $M$ is a monoid where every element is left-invertible (or every element is right-invertible) then $M$ is a group.

\end{prop*}

\begin{proof}

    Let $a\in M$, then there is a $b\in M$ such that $b\circ a=1$.
    But $b$ itself is left invertible so there is an element $c$ such that $c\circ b=1$.
    So:
        \[ c = c\circ b\circ a = a \]
    So $ab=ba=1$, and thus $a$ is invertible.

    \hfill$\blacksquare$

\end{proof}

\begin{thrm*}

    A finite monoid $M$ which is left reducible is a group.

\end{thrm*}

\begin{proof}

    We will show that every element is right invertible.
    We will define a function for every $a\in M$:
        \[ \ell_a\colon M\longvarrightarrow M,\qquad x\varmapsto ax \]
    Then $\ell_a$ is injective since $M$ is left reducible, if $\ell_a(x)=\ell_a(y)$ then $ax=ay$ so $x=y$.
    Because $M$ is finite, $\ell_a$ is surjective, and thus there must be an element $x$ such that $\ell_a(x)=1$, so $ax=1$.
    
    So for every $a\in M$ there is an element $x\in M$ such that $ax=1$, so every $a$ is right invertible.
    Therefore $M$ is a group.

    \hfill$\blacksquare$

\end{proof}

\subsection{$\bZ$ and an Introduction to Number Theory}

We will now focus a bit on the integers, which are the focal point of a field of math called \emph{number theory}.

\blist
    \item $(\bZ,+)$ is a group as it has an identity ($1$) and inverses ($-a$).
    \item $(\bZ,\cdot)$ is a monoid but not a group since $0$ doesn't have an inverse.
\elist

\begin{defn*}

    $\alpha$ is the \ppemph{greatest common divisor} of $a$ and $b$ if it is the maximum number which divides them both.
    We denote this as $\gcd(a,b)$.
    This maximum exists for every $a$ and $b$ unless $a$ and $b$ are both $0$, since zero is divisible by every number.

    $a,b\in\bZ$ are \ppemph{coprime} if their greatest common divisor is $1$.

\end{defn*}

\begin{defn*}

    $\pm1,0\neq p\in\bZ$ is \ppemph{prime} if for every $a,b\in\bZ$, if $p\divides ab$ then $p\divides a$ or $p\divides b$.
    And a number $\pm1,0\neq a\in\bZ$ is \ppemph{non-compound} if $a=bc$ implies $b=\pm 1$ or $c=\pm 1$.

\end{defn*}

Every prime is non-compound since if $p=ab$ then $p\divides ab$ so $p\divides a$ or $p\divides b$.
Suppose that $p\divides a$, and since $a\divides p$ (since $p=ab$) then $p\divides a\divides p$ so $a=\pm p$.
Therefore $b=\pm1$.

\begin{thrm*}

    For every $a,b\in\bZ$ then there exists $\alpha,\beta=\bZ$ such that:
        \[ \alpha a + \beta b = \gcd(a,b) \]

\end{thrm*}

\begin{prop*}

    Every non-compound number is prime.

\end{prop*}

\begin{proof}

    Let $p$ be non-compound.
    Suppose $p\divides ab$, if $p\divides a$ then we have finished.
    Otherwise $\gcd(p,a)=1$ since the greatest common divisor must divide $p$ so it must be $1$ or $p$, and since $p$ doesn't divide $a$ it must be $1$.
    So there must be $\alpha,\beta$ such that $\alpha p + \beta a = 1$, so $b=\alpha pb + \beta ab$.
    We know $p$ divides $\alpha pb$ and $\beta ab$, so $p\divides b$.

    \hfill$\blacksquare$

\end{proof}

\end{document}

