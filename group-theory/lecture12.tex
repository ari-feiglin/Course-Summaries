\documentclass[10pt]{article}

\usepackage{amsmath, amssymb, mathtools}
\usepackage[margin=1.5cm]{geometry}

\usepackage{tikz-cd}

\catcode`\@=11
\input pdfmsym

\input prettyprint
\input preamble

\pdfmsymsetscalefactor{10}
\initpps

\def\slfrac#1#2{\left.{}^{#1}\mkern-4mu\middle/\mkern-3mu{}_{#2}\right.}

\def\pmat#1{\begin{pmatrix} #1 \end{pmatrix}}

\let\divides=\mid

\font\bigbf = cmbx12 scaled 2000

\newfunc{eulerg}{{\rm Euler}}({})
\newfunc{order}o({})
\newfunc{ker}{{\rm Ker}}({})
\newfunc{image}{{\rm Im}}({})
\newfunc{lattice}{{\cal L}}(\vert)
\newfunc{powset}{{\cal P}}(\vert)
\newfunc{units}{{\cal U}}(\vert)
\newfunc{sign}{{\rm sgn}}({})
\newfunc{aut}{{\rm Aut}}({})
\newfunc{speclinear}{{\rm SL}}({})
\newfunc{genlinear}{{\rm GL}}({})
\newfunc{projlinear}{{\rm PGL}}({})
\newfunc{specprojlinear}{{\rm PSL}}({})
\newfunc{core}{{\rm Core}}(\vert)
\newfunc{inner}{{\rm Inn}}(\vert)
\newfunc{outer}{{\rm Out}}(\vert)
\newfunc{conj}{{\rm conj}}({})
\newfunc{fp}{{\rm FP}}({})

\def\maxdivs{\mathrel{\|}}

\let\normalsub=\trianglelefteq
\let\normal=\triangleleft
\let\mmorph=\longvarhookrightarrow

\begin{document}

\c@section=12

\barcolorbox{220, 220, 255}{0, 0, 130}{80, 80, 200}{
    \leftskip=0pt plus 1fill \rightskip=\leftskip
    {\bigbf Group Theory}

    \medskip
    \textit{Lecture \thesection, Sunday January 15, 2023}

    \textit{Ari Feiglin}
}

\bigskip

\subsection{Applications of Sylow's Theorems}

Remember that all the $p$-Sylow subgroups of a group $G$ is equivalent to $1$ modulo $p$.
That is if we define $n_p$ to be the number of $p$-Sylow subgroups of $G$ then
\[ n_p\equiv1\pmod p \]
Recall that if $P$ is a $p$-Sylow subgroup then the number of conjugates of $P$ is $[G:N_G(P)]$ by the Orbit-Stabilizer theorem (since $N_G(P)$ is the stabilizer of $P$ under conjugation).
And so $n_p=[G:N_G(P)]$.
Suppose $\abs G=p^t\cdot m$ where $p^t\maxdivs\abs G$ (so necessarily $m$ is coprime with $p$) so $n_p\cdot\abs{N_G(P)}=p^t\cdot m$, since we know that
\[ P\triangleleft N_G(P)\leq G \]
So $p^t$ divides $\abs{N_G(P)}$ (it is a supergroup of $P$), and so $n_p\divides m$.

\begin{defn*}

    $G$ is an \ppemph{inner direct product} of two subgroups $A,B\leq G$ if
    \benum
        \item $A,B\normalsub G$
        \item $AB=G$
        \item $A\cap B$ is trivial
    \eenum

\end{defn*}

\begin{prop*}

    If $G$ is an inner direct product of $A$ and $B$ then $G\cong A\times B$.

\end{prop*}

\begin{proof}

    If $A,B\normalsub G$ and $A\cap B$ is trivial then $ab=ba$ for every $a\in A$ and $b\in B$ since $aba^{-1}b^{-1}=a(bab^{-1})^{-1}\in A$ since $A$ is normal, and similarly it is equal to
    $(aba^{-1})b^{-1}\in B$ so $aba^{-1}b^{-1}=e$ since the intersection is trivial, and so $ab=ba$.
    So we define the isomorphism
    \[ \phi\colon A\times B\longvarrightarrow G,\qquad (a,b)\varmapsto ab \]
    this is trivially a homomorphism and obviously surjective by definition of inner direct products.
    It is injective since $ab=a'b'$ if and only if $a'a^{-1}=bb'^{-1}$ (since elements of $A$ and $B$ commute) and so $a'a^{-1}=bb'^{-1}=e$ since $A\cap B$ is trivial so $a=a'$ and $b=b'$.
    So $\phi$ is an isomorphism as required.

    \hfill$\blacksquare$

\end{proof}

We can generalize the definition of inner direct products of $A_1,\dots,A_n$ where
\benum
    \item $A_1,\dots,A_n\normalsub G$
    \item $A_1\cdots A_n=G$
    \item For every $i=1,\dots,n$, $A_i\cap\bigl(A_1\cdots A_{i-1}\cdot A_{i+1}\cdots A_n\bigr)$ is trivial
\eenum

Then we can show that if $G$ is the inner direct product of $A_1,\dots,A_n$ then $G\cong A_1\times\cdots\times A_n$ similar to how we did above.

\begin{prop*}

    If $P_i$ is the $p_i$-Sylow normal subgroup then
    \[ \abs{P_1\cdots P_k}=\abs{P_1}\cdots\abs{P_k} \]

\end{prop*}

Note that $P_i$ is normal if and only if $N_G(P_i)=G$, that is only if $n_{p_i}=1$ so $P_i$ is the only $p_i$-Sylow group.
That is $P_i$ is normal if and only if it is the only $p_i$-Sylow group.

\begin{proof}

    We do this inductively.
    We know that $P_i\cap\bigl(P_1\cdots P_{i-1}\cdot P_{i+1}\cdots P_k\bigr)$ is trivial since the order of $P_i$ is coprime with this product (since it is inductively equal to a product of powers of $p_j$)
    and so the product is an inner direct product, and therefore isomorphic to the direct product $P_1\times\cdots\times P_n$, which has order $\abs{P_1}\cdots\abs{P_n}$ as required.

    \hfill$\blacksquare$

\end{proof}

\begin{prop*}

    If every $p$-Sylow subgroup is normal then $G$ is the inner direct product of them.

\end{prop*}


The proof is simple: suppose $P_i$ are the $p$-Sylow groups of $G$ then the order of $\abs{P_1\cdots P_k}=p_1^{t_1}\cdots p_k^{t_k}=\abs G$, so $P_1\cdots P_k=G$ and this is an inner direct product.

\begin{exam}

    We can show that every group of order $5\cdot13\cdot19$.
    Suppose $P_5$ is a $5$-Sylow group, it must be unique since $n_5\equiv 1\pmod5$ and $n_5\divides 13\cdot19$.
    Since $13,19,13\cdot19\not\equiv1$ we must have that $n_5=1$ and similarly $n_{13}=n_{19}=1$, so $P_5$, $P_{13}$ and $P_{19}$ are all normal, so
    \[ G = P_5\cdot P_{13}\cdot P_{19} \cong P_5\times P_{13}\times P_{19} \]
    this is isomorphic to $\bZ_5\times\bZ_{13}\times\bZ_{19}$ since groups of prime order are cyclic, and this is isomorphic to $\bZ_{5\cdot13\cdot19}$.

\end{exam}

\begin{prop*}

    Suppose $P$ is a $p$-Sylow subgroup of $G$ then
    \[ N_G\bigl(N_G(P)\bigr) = N_G(P) \]

\end{prop*}

\begin{proof}

    Let $x\in N_G\bigl(N_G(P)\bigr)$ then
    \[ xPx^{-1}\subseteq xN_G(P)x^{-1} = P \]
    since $x$ normalizes $N_G(P)$.
    So $xPx^{-1}$ is a $p$-Sylow subgroup of $N_G(P)$, but since $P$ is normal in $N_G(P)$, it is unique in it, so $xPx^{-1}=P$ so $x\in N_G(P)$.

    \hfill$\blacksquare$

\end{proof}

\begin{prop*}

    If $P$ is a $p$-Sylow subgroup and $N_G(P)\subseteq H$ then $N_G(H)=H$.

\end{prop*}

\begin{proof}

    The proof here is similar to the proof above.
    Let $x\in N_G(H)$ then
    \[ xPx^{-1}\subseteq xN_G(P)x^{-1}\subseteq xHx^{-1} = H \]
    since $x$ normalizes $H$.
    And so $xPx^{-1}\subseteq H$ and so $xPx^{-1}$ is a $p$-Sylow subgroup of $H$, and since all $p$-Sylow groups are conjugates there must be an $h\in H$ such that $xPx^{-1}=hPh^{-1}$.
    And so $(h^{-1}x)P(h^{-1}x)^{-1}=P$ and therefore $h^{-1}x\in N_G(P)$, so $x\in h\cdot N_G(P)\subseteq H$.
    So $H\subseteq N_G(H)$ and therefore are equal.

    \hfill$\blacksquare$

\end{proof}

\begin{exam*}

    Groups of order $72$ are not simple.
    We know that $72=2^3\cdot3^2$ so
    \begin{align*}
    n_2 &= 1, 3, 9 \\
    n_3 &= 1, 4
    \end{align*}
    as these are the numbers coprime with their respective primes and equivalent to $1$ modulo the prime.
    Suppose that $72$ is simple, then the $p$-Sylow groups cannot be normal, so $n_2,n_3\neq1$ and so $n_3=4$.

    Recall that by the refinement of Cayley's Theorem, if $H\leq G$ and $m=[G:H]$ then their is a homomorphism $G\varrightarrow S_m$.
    So by this refinement on $N_G(P_3)$, there is a homomorphism $G\varrightarrow S_4$, since $72\nmid4!=24$, this cannot be an injective homomorphism, so it must have a kernel.
    Since kernels are normal subgroups, $G$ cannot be simple in contradiction.

\end{exam*}

\begin{exam*}

    We can do something similar for groups of order $90=2\cdot3^2\cdot5$ since $n_3=1,10$ and $n_5=1,6$.
    If the group is simple then $n_3=10$ and $n_5=6$, notice then that $N_G(P_3)=P_3$.
    By Cayley's refinement there is a homomorphism $G\varrightarrow S_6$, and if we assume $G$ is simple this must be a monomorphism.
    We now use $G$ to mean its image in $S_6$, notice that
    \[ \slfrac{G}{G\cap A_6} \cong \slfrac{G\cdot A_6}{A_6} \]
    and since $G$ either has odd permutations or doesn't, so this is either trivial or $\bZ_2$.
    If $G\not\subseteq A_6$ then this is $\bZ_2$ so $G\cap A_6\triangleleft G$ as it has index $2$, but $G$ is simple.
    So $G\subseteq A_6$, and since $[A_6:G]=4$ this creates a homomorphism $A_6\varrightarrow S_4$ and it must be a monomorphism since $A_6$ is simple.
    But this is cannot be since $\frac{6!}2=360$ and $4!=24$.

\end{exam*}

In general if $G$ is simple and has a subgroup of index $m$ then $G\longvarhookrightarrow A_m$.

\begin{exam*}

    We will show that the only simple group of order $60$ is $A_5$.

    Suppose $G$ is a simple group of order $60$, then since $60=2^2\cdot3\cdot5$ so
    \begin{align*}
    n_2 &= 3, 5, 15 \\
    n_3 &= 4, 10 \\
    n_5 &= 6
    \end{align*}
    $n_2$ cannot be $3$ since if it were, there'd be a monomorphism $G\longvarhookrightarrow A_3$ which cannot be.
    And similarly $n_3\neq4$, so $n_3=10$.
    In any case, $G$ has $6$ $5$-Sylow groups which then must all be isomorphic to $\bZ_5$, and their intersections must be trivial (since it divides $5$).
    Each of these subgroups has $4$ elements of order $5$, so there are at least $6\cdot4=24$ elements of order $5$.
    Similarly there must be $10\cdot2=20$ elemenets of order $3$.

    Now suppose $n_2=5$ then there is a monomorphism (in fact it is an isomorphism) $G\longvarhookrightarrow A_5$ as required.
    So suppose $n_2=15$ then there are $15$ elements of order $2$.
    If all the $2$-Sylow subgroups have trivial intersections, then since all these subgroups are of order $4$, there must be $15\cdot(4-1)=45$ elements of orders powers of $2$.
    But notice that
    \[ 60 = 1 + 20 + 24 + 15 \]
    the $1$ is the identity, there are $20$ elements of order $3$ and $24$ of power $5$, so there can only be $15$ elements of order powers of $2$.
    So there must be two $2$-Sylow groups with non-trivial intersections, let them be $P$ and $P'$.
    Let $Q=P\cap P'$ which has order $2$, and let $Q=N_G(P\cap P')$.
    Since $P,P'$ are abelian we must have that $P\subset Q$, so the order of $Q$ must be a multiple of $4$, namely $12,20,60$.
    It cannot be $60$ since then $Q$ would be normal.
    If it is $12$ then its index is $3$, which is a contradiction (as $N_G(P)\supseteq Q$ and so $n_2=15$ is smaller than its index), and similarly it cannot be $20$, in contradiction.

\end{exam*}

\newpage
\subsection{Subnormal Sequences}

\begin{defn*}

    A \ppemph{sub-normal series} is a sequence of subgroups $G_i\leq G$ such that
    \[ \set e = G_k\triangleleft \dots \triangleleft G_1 \triangleleft G_0 = G \]
    if $G_i\triangleleft G$ then the sequence is considered to be a \ppemph{normal series}.

\end{defn*}

\begin{defn*}

    A subnormal series is a \ppemph{composition series} if the quotients $\slfrac{G_i}{G_{i+1}}$ are simple.

\end{defn*}

This is equivalent to saying the sequence cannot be extended, ie we cannot add another subgroup into the sequence.
Suppose we could add a subgroup, $G_{i+1}\triangleleft G'\triangleleft G_i$ then $\slfrac{G'}{G_{i+1}}\triangleleft\slfrac{G_i}{G_{i+1}}$, in contradiction.

In a finite group we can extend every subnormal series to a composition series.

The quotients in a composition series are called the \emph{composition factors}.

\begin{thrm*}[jordanHolder,Jordan-Holder\ Theorem]

    Every two composition seriess of the same group have the same composition factors, up to order.
    Specifically, they have the same lengths.

\end{thrm*}

\begin{proof}

    Suppose we have two composition seriess:
    \begin{gather*}
        \set e\normal A_n\normal\dots\normal A_1\normal A_0 = G \\
        \set e\normal B_m\normal\dots\normal B_1\normal B_0 = G
    \end{gather*}
    If $A_1=B_1$ then the rest of the sequences (without $A_0$ and $B_0$) are composition seriess of $A_1$ and $B_1$ and so are equal inductively.

    Otherwise by the isomorphism theorems
    \[ \slfrac{A_1}{A_1\cap B_1}\cong\slfrac{A_1B_1}{A_1} \]
    and since we can't add another group between $G=A_0$ and $A_1$, we must have that $A_1B_1=G$ (since $A_1\normal A_1B_1$).

    If we create a new composition series from $A_1\cap B_1$ we get
    \usetikzlibrary{decorations.pathmorphing}
    \[\begin{tikzcd}
    	& G \\
    	{B_1} && {A_1} \\
    	& {A_1\cap B_1} \\
    	\\
    	\\
    	{B_0} && {A_0} \\
    	& {\set e}
    	\arrow["\beta"', no head, from=1-2, to=2-1]
    	\arrow["\alpha", no head, from=1-2, to=2-3]
    	\arrow["\alpha"', no head, from=3-2, to=2-1]
    	\arrow["\beta"', no head, from=2-3, to=3-2]
    	\arrow["b"', squiggly, no head, from=2-1, to=6-1]
    	\arrow["a", squiggly, no head, from=2-3, to=6-3]
    	\arrow[no head, from=6-1, to=7-2]
    	\arrow[no head, from=6-3, to=7-2]
    	\arrow["c"{description}, squiggly, no head, from=3-2, to=7-2]
    \end{tikzcd}\]
    Notice that the composition factor from $G$ to $A_1$ is $\alpha$ which is also the composition factor from $A_1$ to $A_1\cap B_1$ because
    \[ \alpha = \slfrac{G}{A_1} \cong \slfrac{A_1}{A_1\cap B_1} \]
    as explained above, similar for $\beta$.

    Inductively
    \[ a\sim\beta + c \qquad b\sim\alpha + c \]
    where $+$ denotes adding the composition factor to the composition chain, and so
    \[ \alpha + a \sim \alpha + (\beta + c) \qquad \beta + b\sim\beta + (\alpha + c) \]
    and so the composition factors in $\alpha+a$ and $\beta+b$ are the same (since they are the composition factors in $c$, and $\alpha$ and $\beta$), as required.

    \hfill$\blacksquare$

\end{proof}

There is a small loose end here where we assumed that there exists a composition series on $A_1\cap B_1$.
This is true if $A_1\cap B_1$ is finite.

\end{document}

