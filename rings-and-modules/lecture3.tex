\documentclass[10pt]{article}

\usepackage{amsmath, amssymb, mathtools}
\usepackage[margin=1.5cm]{geometry}

\usepackage{tikz-cd}

\catcode`\@=11
\input pdfmsym

\input prettyprint
\input preamble

\pdfmsymsetscalefactor{10}
\initpps

\def\implies{\,\mathrel{\longvarRightarrow}\,}

\def\slfrac#1#2{\left.{}^{#1}\mkern-4mu\middle/\mkern-3mu{}_{#2}\right.}

\def\pmat#1{\begin{pmatrix} #1 \end{pmatrix}}

\let\divides=\mid

\font\bigbf = cmbx12 scaled 2000

\newfunc{eulerg}{{\rm Euler}}({})
\newfunc{order}o({})
\newfunc{ker}{{\rm Ker}}({})
\newfunc{image}{{\rm Im}}({})
\newfunc{lattice}{{\cal L}}(\vert)
\newfunc{powset}{{\cal P}}(\vert)
\newfunc{center}{{\rm Z}}({})
\newfunc{units}{{\cal U}}(\vert)
\newfunc{sign}{{\rm sgn}}({})
\newfunc{aut}{{\rm Aut}}({})
\newfunc{speclinear}{{\rm SL}}({})
\newfunc{genlinear}{{\rm GL}}({})
\newfunc{projlinear}{{\rm PGL}}({})
\newfunc{specprojlinear}{{\rm PSL}}({})
\newfunc{core}{{\rm Core}}(\vert)
\newfunc{inner}{{\rm Inn}}(\vert)
\newfunc{outer}{{\rm Out}}(\vert)
\newfunc{conj}{{\rm conj}}({})
\newfunc{fp}{{\rm FP}}({})

\def\maxdivs{\mathrel{\|}}

\let\ideal=\trianglelefteq
\def\normal{\mathrel\triangleleft}
\let\mmorph=\longvarhookrightarrow

\begin{document}

\c@section=2

\barcolorbox{220, 220, 255}{0, 0, 130}{80, 80, 200}{
    \leftskip=0pt plus 1fill \rightskip=\leftskip
    {\bigbf Introduction to Rings and Modules}

    \medskip
    \textit{Lecture \thesection, Monday March 20 2023}

    \textit{Ari Feiglin}
}

\bigskip

\subsection{Ideals}

\begin{defn*}

    Suppose $R$ is a ring, then a subset $\varnothing\neq I\subseteq R$ is a \ppemph{left ideal} if:
    \benum
        \item $(I,+)$ is a subgroup of $(R,+)$.
        \item For every $a\in I$ and every $r\in R$, $ra\in I$.
    \eenum

    If $I$ is closed on the right by multiplication of $R$ then it is a \ppemph{right ideal}.
    If $I$ is both a left and right ideal, it is a \ppemph{bidirectional ideal} (or simply an ideal) and is denoted $I\ideal R$.

\end{defn*}

Notice that:
\benum
    \item If $R$ is commutative, left ideals, right ideals, and bidirectional ideals are all the same.
    \item Every left or right ideal is a subrng of $R$ (since $(I,\cdot)$ is necessarily a semigroup since multiplication is associative and it is closed), but it may not contain an identity.
    \item If an ideal is also a subring, that is $1_R\in I$, then for every $r\in R$ $r\cdot1_R=1_R\cdot r=r\in I$, so $I=R$.
\eenum

A trivial example of an ideal is the trivial ring $\set0$ since $r\cdot0=0\cdot r=0$.
And another trivial example of an ideal is just $R$ itself.

\begin{exam*}

    If $R$ is a ring and $a\in R$ then $Ra=\set{ra}[r\in R]$ is the smallest left ideal containing $a$.
    \benum
        \item $(Ra,+)$ is a group since $0_R\in Ra$ and if $r_1a,r_2a\in Ra$ then $r_1a+r_2a=(r_1+r_2)a\in Ra$ and $r_1a+(-r_1)a=(r_1-r_1)a=0_R$.
        \item If $r\in R$ then for any $r'a\in R$, $rr'a\in Ra$ so $Ra$ is a left ideal.
        \item If $I$ is a left ideal containing $a$ then for every $r\in R$ we have that $ra\in I$ by the definition of a left ideal, and so $Ra\subseteq I$.
    \eenum
    We also called $Ra$ the \ppemph{left ideal generated by $a$}.
    We define $aR$ in a similar fashion, it is also the smallest right ideal containing $a$ and is the \ppemph{right ideal generated by $a$}.

\end{exam*}

\begin{exam*}

    If $R$ is a ring and $a\in R$ then any bidirectional ideal must contain elements of the form $ras$ for $r,s\in R$.
    But it must also be a group under addition and so it must be closed under addition so it must contain all the elements of the form $\sum_{i=1}^n r_ias_i$ for $r_i,s_i\in R$.
    This is the smallest group generated by $\set{ras}[r,s\in R]$.
    This is a group under addition since $-\sum r_ias_i=\sum (-r_i)as_i$ and it is by definition closed under addition and contains $0_R$.
    And it is closed under both left and right multiplication so it is a bidirectional ideal.

    This is obviously the smallest bidirectional ideal containing $a$ since any bidirectional ideal containing $a$ must contain all elements of the form $ras$ and their sums.

\end{exam*}

\begin{exam*}

    \newfunc{diag}{{\rm diag}}({})
    Take $R=M_n(\bR)$ and $a=\diagof{0, 1, 1, \dots, 1}$.
    Then
    \[ Ra = \set{M\pmat{0 \\ & 1 \\ && \ddots \\ &&& 1}}[M\in M_n(\bR)] \]
    If we let $m_{ij}$ be the elements of $M$ then
    \[ M\cdot a = \pmat{0 & m_{21} & \cdots & m_{n1} \\ \vdots & & \ddots & \vdots \\ 0 & m_{2n} & \cdots & m_{nn}} \]
    So $Ra$ is the set of all matrices whose first column is $0$.
    Similarly $aR$ is the set of all matrices whose first row is $0$.

\end{exam*}

\begin{exam*}

    If $R$ is a group and $J\ideal R$ is a bidirectional ideal then $M_n(J)\ideal M_n(R)$.
    The ideality of $M_n(J)$ follows simply from the ideality of $J$.
    And as it turns out, the converse is true as well: for every bidirectional ideal $I\ideal M_n(R)$ there exists $J\ideal R$ such that $M_n(J)=I$.
    Let $J$ be the set of all $a\in R$ such that $a$ is an element of a matrix in $I$.
    Then $I\subseteq M_n(J)$.
    If $x\in J$ then there exists an $M\in I$ where $x=m_{ij}$.
    Take $A$ to be the matrix of $0$s except for $a_{jj}=1$ and $B$ the same but $b_{ii}=1$.
    Then
    \[ [BMA]_{ts} = R_t(BM)\cdot C_s(A) = R_t(B)\cdot M\cdot C_s(A) \]
    And so if $t\neq i$ or $s\neq j$ then $R_t(B)=0$ or $C_s(A)=0$ and this is $0$.
    If $t,s=i,j$ then this is equal to $m_{ij}=x$ and since $I$ is an ideal $BMA\in I$ and $BMA$ is the matrix of $0$s except for at index $i,j$ which is $x$.
    And so for any 

\end{exam*}

Notice that if $\bF$ is a field, and $\set0\neq I\ideal\bF$ is an ideal then $a\in I$ and so $a^{-1}\in\bF$ so $aa^{-1}=1\in I$ and therefore $I=\bF$.
So fields only have trivial ideals.

An important insight to have with rings is that prime decomposition need not be unique ($p\neq0,1$ is prime in $R$ if $p\divides ab$ means $p\divides a$ or $p\divides b$).
For example the ring $\set{a+b\sqrt{-5}}[a,b\in\bZ]$ has two decompositions for $6$, $2\cdot3$ and $(1+\sqrt{-5})(1-\sqrt{-5})$.
We will define later what a prime ideal is, and prime ideals do have a notion of unique decompositions.

\begin{defn*}

    If $R$ is a ring and $a,b\in R$ then:
    \[ Ra + Rb = \set{r_1a+r_2b}[r_1,r_2\in R] \]
    This is still an ideal since the product of subgroups (sum of abelian subgroups) is itself a subgroup, and this is obviously closed under left multiplication by $R$.
    And if $S\subseteq R$ is a subset, then
    \[ RS = \set{\sum_{i=1}^n r_is_i}[n\in\bN, r_i\in R, s_i\in S] \]
    So if $S=\set a$ then $RS=Ra$ and if $S=\set{a,b}$ then $RS=Ra+Rb$.
    And we similarly define this for right ideals.

\end{defn*}

\begin{defn*}

    If $R$ is a commutative ring and $I\ideal R$ an ideal, then we define
    \[ \sqrt I = \set{a\in R}[\exists n\in\bN\colon a^n\in I] \]
    This is called the \ppemph{radical ideal} of $I$.

\end{defn*}

Notice that $I\subseteq\sqrt I$, and that this is indeed an ideal:
\benum
    \item If $a\in\sqrt I$ and $r\in R$ then since $a^n\in I$ we have that $(ra)^n=r^na^n$ since $R$ is commutative and $r^n\in R$ and $a\in I$ so $(ra)^n\in I$so $ra\in\sqrt I$.
    \item If $a\in\sqrt I$ then $-a\in\sqrt I$ since $-a=(-1)a$ and $\sqrt I$ is closed under left multiplication by above.
    \item Suppose $a,b\in\sqrt I$ where $a^n,b^m\in I$.
    Then by the binomial theorem (since it is simple why this holds in commutative rings):
    \[ (a+b)^{n+m} = \sum_{k=0}^n\binom{n+m}ka^k\cdot b^{n+m-k} \]
    where for $n\in\bN$, $na=a+\cdots+a$ ($n$ summands).
    If $k\geq n$ then $a^k=a^na^{k-n}$ where $a^n\in I$ and $a^{k-n}\in R$ so $a^k\in I$ and so $a^kb^{n+m-k}\in I$ (and so is $\binom{n+m}k a^kb^{n+m-k}$ since $I$ is a group under addition).
    And if $k<n$ then $m<n+m-k$ and so as before $b^{n+m-k}\in I$ and so $\binom{n+m}k a^kb^{n+m-k}\in I$ and so our sum is in $I$.
    So $\sqrt I$ is closed under addition and therefore is an ideal.
\eenum

\begin{defn*}

    $a\in R$ is \ppemph{nilpotent} if there exists an $n\geq1$ such that $a^n=0_R$.

\end{defn*}

So if $R$ is commutative then $\set{a\in R}[a\text{ is nilpotent}]=\set{a\in R}[\exists n\in\bN\colon a^n=0_R]=\sqrt{\set{0_R}}$.
And since $\set{0_R}$ is an ideal, so is its radical and therefore the set of nilpotent elements in $R$ is also an ideal.

Suppose $R$ is a ring and $\varnothing\neq J\subseteq R$ is a subset such that $(J,+)\leq(R,+)$.
Since $(R,+)$ is abelian, $J$ is normal, and so $\slfrac RJ$ is a quotient \emph{group}, where $(a+J)+(b+J)=(a+b)+J$.
Now we'd like to define
\[ (a+J)\cdot(b+J) = (ab)+J \]
When is this well-defined?
Suppose $a_1+J=a_2+J$ and $b_1+J=b_2+J$ then $a_2=a_1+j$ for $j\in J$ and $b_2=b_1+j'$ for $j'\in J$.
We want to satisfy
\[ (a_1b_1)+J = (a_2b_2)+J \iff a_2b_2-a_1b_1\in J \]
And since
\[ a_2b_2 - a_1b_1 = (a_1+j)(b_1+j') - a_1b_1 = a_1b_1 + a_1j' + jb_1 + jj' \]
In order for this to be in $J$ we must have that $a_1j'+jb_1\in J$ for every $a_1,b_1\in R$ and $j\in J$.
And this is equivalent to having $aj\in J$ and $ja\in J$ for every $a,j\in J$, which is equivalent to $J$ being a bidirectional ideal.
So in order for $\slfrac RJ$ to have a ring structure (or at least for multiplication to be well-defined) it is necessary and sufficient for $J$ to be a bidirectional ideal.

This leads us to the following definition of the quotient ring:
\begin{defn*}

    Suppose $R$ is a ring and $I\ideal R$ is a bidirectional ideal, then we define the \ppemph{quotient ring} $\parens{\slfrac RI, +, \cdot}$ where $\slfrac RI$ is the set of cosets of $I$ under addition,
    addition is defined as normal, and multiplication of cosets as defined above.

\end{defn*}

We showed that this is indeed well-defined, and is a ring since $-(aI)=(-a)I$ and $I$ is the identity.

\begin{prop*}

    Suppose $f\colon R\longvarrightarrow S$ is a ring homomorphism, then we define
    \[ \kerof f = \set{a\in R}[f(a)=0_S]=f^{-1}(0),\qquad \imageof f = \set{s\in S}[\exists r\in R\colon f(r)=s] = f(R) \]
    and $\kerof f$ is a bidirectional ideal of $R$ and $\imageof f$ is a subring of $S$.

\end{prop*}

\begin{proof}

    We know that since $f$ is also a group homomorphism from $(R,+)$ to $(S,+)$, $\kerof f$ is a subgroup of $(R,+)$ by group theory.
    Suppose $r\in R$ and $a\in\kerof f$ then $f(ra)=f(r)f(a)=f(r)0_S=0_S$ so $ra\in\kerof f$ and $f(ar)=f(a)f(r)=0_Sf(r)=0_S$ so $ar\in\kerof f$ and so $\kerof f$ is a bidirectional ideal of $R$.

    And $\imageof f$ is a subgroup of $(R,+)$.
    Since $f(1_R)=1_S$, this means that $1_S\in\imageof f$ and since $f(a)f(b)=f(ab)$, $\imageof f$ is closed under multiplication so it is a subring of $S$.

    \hfill$\blacksquare$

\end{proof}

\begin{thrm*}[firstIsoThrm,The\ First\ Isomorphism\ Theorem]

    Suppose $f\colon R\longvarrightarrow S$ is a ring homomorphism, then there is a natural isomorphism
    \[ \slfrac R{\kerof f}\cong\imageof f \]
    (in other words, $\slfrac R{\kerof f}$ and $\imageof f$ are ring-isomorphic.)

\end{thrm*}

\end{document}

