\documentclass[10pt]{article}
\usepackage{amsmath, amssymb, mathtools}
\usepackage[margin=1.5cm]{geometry}
\usepackage{mathrsfs}

\catcode`\@=11
\input pdfmsym

\input prettyprint
\input preamble

\pdfmsymsetscalefactor{10}
\initpps

\def\implies{\,\mathrel{\longvarRightarrow}\,}

\def\pmat#1{\begin{pmatrix} #1 \end{pmatrix}}

\def\divides{{\mid}}

\font\bigbf = cmbx12 scaled 2000

\newfunc{eulerg}{{\rm Euler}}({})
\newfunc{order}o({})
\newfunc{ker}{{\rm Ker}}({})
\newfunc{image}{{\rm Im}}({})
\newfunc{lattice}{{\cal L}}(\vert)
\newfunc{powset}{{\cal P}}(\vert)
\newfunc{center}{{\rm Z}}({})
\newfunc{units}{{\cal U}}(\vert)
\newfunc{sign}{{\rm sgn}}({})
\newfunc{aut}{{\rm Aut}}({})
\newfunc{speclinear}{{\rm SL}}({})
\newfunc{genlinear}{{\rm GL}}({})
\newfunc{projlinear}{{\rm PGL}}({})
\newfunc{specprojlinear}{{\rm PSL}}({})
\newfunc{core}{{\rm Core}}(\vert)
\newfunc{inner}{{\rm Inn}}(\vert)
\newfunc{outer}{{\rm Out}}(\vert)
\newfunc{conj}{{\rm conj}}({})
\newfunc{ev}{{\rm ev}}({})
\newfunc{Frac}{{\rm Frac}}({})
\newfunc{Tor}{{\rm Tor}}(\vert)
\newfunc{Ann}{{\rm Ann}}(\vert)
\newfunc{deg}{{\rm deg}}({})

\def\maxdivs{\mathrel{\|}}

\let\ideal=\trianglelefteq
\let\pideal=\triangleleft
\def\normal{\mathrel\triangleleft}
\let\mmorph=\longvarhookrightarrow
\let\to=\varrightarrow
\let\longto=\longvarrightarrow

\def\qed{%
    \ifmmode%
        \eqno\blacksquare%
    \else%
        \hskip1cm\allowbreak\hbox{}\nobreak\hfill$\blacksquare$%
    \fi%
}

\def\mN{\mathcal N}
\def\mO{\mathcal O}
\def\mG{\mathcal G}
\def\sS{\mathscr S}

\begin{document}

\c@section=20

\barcolorbox{220, 220, 255}{0, 0, 130}{80, 80, 200}{
    \leftskip=0pt plus 1fill \rightskip=\leftskip
    {\bigbf Introduction to Rings and Modules}

    \medskip
    \textit{Lecture \thesection, Thursday June 22 2023}

    \textit{Ari Feiglin}
}

\bigskip

Recall the product of ideals is given by
\[ IJ = \set{\sum_{i=1}^n a_ib_i}[a_i\in I,\,b_i\in J] \]
and thus inductively
\[ I_1\cdots I_k = \set{\sum_{i=1}^n a_i^1\cdots a_i^k}[a_i^j\in I_j] \]

And suppose $M,N$ are sub-$R$-modules (of some larger $R$-module), then so is
\[ MN = \set{\sum_{i=1}^k m_in_i}[m_i\in M,\,n_i\in N] \]

\begin{lemm*}

    Suppose $R$ is left/right noetherian and $\sS$ is a set of left/right ideals.
    Then $R$ contains a maximal ideal.

\end{lemm*}

\begin{proof}

    Since every chain in $\sS$ has a maximal element by $R$ being noetherian, Zorn's lemma tells us $\sS$ has a maximal element.
    \qed

\end{proof}

\begin{lemm*}

    Suppose $R$ is a noetherian ring, and $0\neq I\ideal R$.
    Then $I$ contains a product of non-zero prime ideals.

\end{lemm*}

\begin{proof}

    Let
    \[ \sS = \set{0\neq I\ideal R}[I\text{ does not contain a product of non-zero prime ideals}] \]
    we must prove $\sS=\varnothing$.

    Suppose not, then by the previous lemma $\sS$ has a maximal element, $J$.
    $J$ cannot be prime as then it would contain a product of primes (itself).
    Since $J$ is not prime there exist ideals $I_1$ and $I_2$ which are not subsets of $J$, but $I_1I_2\subseteq J$.
    Thus $J$ is a proper subset of $I_1+J$ and $I_2+J$, and so by $J$'s maximality $I_1+J,I_2+J\notin\sS$.
    Thus
    \[ P_1\cdots P_n\subseteq I_1+J,\quad Q_1\cdots Q_m\subseteq I_2+J \]
    where $P_i$ and $Q_i$ are prime.
    But $(I_1+J)(I_2+J)=I_1I_2+I_1J+JI_2+J^2$, and since $I_1I_2\subseteq J$ by our definition of $I_1$ and $I_2$, and $I_1J,JI_2,j^2\subseteq J$ since $J$ is an ideal, we have that
    $(I_1+J)(I_2+J)\subseteq J$ and so
    \[ P_1\cdots P_nQ_1\cdots Q_m\subseteq J \]
    meaning $J\notin\sS$ in contradiction.
    \qed

\end{proof}

\begin{defn*}

    Suppose $R$ is an integral domain, and $(0)\neq P\pideal R$ is a prime ideal.
    We define
    \[ P^{-1} = \set{z\in\Fracof R}[zP\subseteq R] \]

\end{defn*}

$P^{-1}$ is closed under $R$-scalar multiplication and is thus an $R$-submodule of $\Fracof R$.
Since $P$ is an ideal, $R\subseteq P^{-1}$.

\begin{lemm*}

    Suppose $R$ is a noetherian integral domain, and $\dim R=1$.
    Suppose $(0)\neq P\pideal R$ a prime ideal, then $R\subset P^{-1}$.

\end{lemm*}

\begin{proof}

    Let $0\neq y\in P$ and $(0)\neq I=(y)=Ry$.
    By the previous lemma there exist $P_i$ prime such that
    \[ P_1\cdots P_n \subseteq I\subseteq P \]
    we assume $n$ is minimal.
    Since $P$ is prime, there exists an $i$ such that $P_i\subseteq P$, without loss of generality $i=n$.
    But $P_n$ and $P$ are non-zero ideals and since $\dim R=1$ this means they are maximal so $P_n=P$.
    By the minimality of $n$,
    \[ P_1\cdots P_{n-1}\nsubseteq I \]
    and so there exists an $x\in P_1\cdots P_{n-1}$ where $x\notin I$.
    Let $z=\frac xy\in\Fracof R$.
    Note that $z\notin R$ since if so $x=zy\in(y)=I$ in contradiction.

    We will show $z\in P^{-1}$, meaning $zP\subseteq R$.
    Let $a\in P$, so
    \[ za = \frac{xa}y \]
    since $x\in P_1\cdots P_{n-1}$ and $a\in P=P_n$, we have $xa\in P_1\cdots P_n\subseteq I=Ry$.
    So there exists an $r\in R$ such that $xa=ry$ and so $\frac{xa}y=r\in R$ meaning $za\in P$ as required.
    \qed

\end{proof}

\begin{lemm*}

    Suppose $R$ is a Dedekind domain and $(0)\neq I\ideal R$.
    Suppose $I\subseteq P$ for $P$ prime ideal.
    Then $I\subset IP^{-1}$.

\end{lemm*}

\begin{proof}

    Since $1\in R\subset P^{-1}$, for every $x\in I$, $x=x\cdot1\in IP^{-1}$ as required.
    Now suppose $I=IP^{-1}$.
    Let $z\in P^{-1}$ such that $z\notin R$, and since $I$ is closed under multiplication by $z$, it is closed under multiplication by any exponent of $z$.
    Since $I$ is closed under multiplication by $R$, it has the structure of an $R[z]$ module.

    Since $R$ is noetherian, $I$ is finitely generated as an ideal, meaning an $R$-module.
    We will show that $I$ is a faithful $R[z]$-module.
    Suppose
    \[ b\in\Annof[{R[z]}]I \]
    then $ba=0$ for every $i\in I$.
    But since $R[z],I\subseteq\Fracof R$ which is a field and thus has no zero divisors and $I\neq(0)$ so $b=0$, meaning
    \[ \Annof[{R[z]}]I = (0) \]

    Thus $I$ is a faithful finitely generated $R[z]$-module, and thus $z$ is integral over $R$.
    But $z\in\Fracof R$, and $R$ is a Dedekind domain so it is integrally closed, and so $z\in R$.
    Thus $\Fracof R=R$, and so $z\in R$.
    But $z\notin R$ in contradiction.
    \qed

\end{proof}

\begin{lemm*}

    Suppose $R$ is a Dedekind domain and $P$ a non-trivial prime ideal.
    Then $PP^{-1}=R$.

\end{lemm*}

\begin{proof}

    By definition $PP^{-1}\subseteq R$.
    Since $P$ and $P^{-1}$ are sub-$R$-modules of $\Fracof R$, we have $PP^{-1}$ is an $R$ module, and thus an ideal.
    But we know $P\subset PP^{-1}$, but since $P$ is non-trivial and $\dim R=1$ so $P$ is maximal, thus $PP^{-1}=R$.

\end{proof}

\begin{thrm*}

    Suppose $R$ is a Dedekind domain and $(0)\neq I\ideal R$ is a non-zero ideal.
    Then $I$ factorizes uniquely as a product of non-trivial prime ideals:
    \[ I = P_1\cdot P_2\cdots P_n \]

\end{thrm*}

\begin{proof}

    Let us first prove the existence of such a factorization.
    Let us define
    \[ \sS = \set{(0)\neq I\ideal R}[I\text{ does not factorize as a product of non-trivial prime ideals}] \]
    suppose that $\sS\neq\varnothing$.
    Thus since $R$ is noetherian, $\sS$ has a maximal ideal $I$, and $I$ is contained in a maximal (and thus prime) ideal $I\subseteq P$.
    Since $I\subset IP^{-1}$, but $IP^{-1}\subseteq PP^{-1}=R$, thus $IP^{-1}$ is an ideal of $R$.
    Since $I\subset IP^{-1}$ and $I$ is maximal, $IP^{-1}\notin\sS$ so $IP^{-1}$ factorizes, suppose
    \[ IP^{-1} = P_1\cdots P_n \]
    but then
    \[ P_1\cdots P_nP = IP^{-1}P = IR = I \]
    meaning $I\notin\sS$ in contradiction.

    Thus $\sS=\varnothing$ and so every non-zero ideal factorizes as a product of non-trivial prime ideals.

    Now suppose that
    \[ I = P_1\cdots P_n = Q_1\cdots Q_m \]
    we will assume without loss of generality that $n\leq m$.
    We will prove on induction of $n$ that this factorization is unique.
    If $n=0$ then suppose $m>0$, so $I=R=Q_1\cdots Q_m$ but $I\subseteq Q_1\subset R$ which is a contradiction.

    Now suppose that the factorization is unique for $n$, then suppose
    \[ P_1\cdots P_n\cdot P_{n+1} = Q_1\cdots Q_m \]
    Thus $Q_1\cdots Q_m\subseteq P_{n+1}$, but since $P_{n+1}$ is prime there exists an $i$ such that $Q_i\subseteq P_{n+1}$.
    But $\dim R=1$ and $Q_i$ and $P_{n+1}$ are non-trivial so $Q_i=P_{n+1}$.
    Thus
    \[ P_1\cdots P_{n+1} = Q_1\cdots Q_m = (Q_1\cdots Q_{i-1}\cdot Q_i\cdots Q_m)P_{n+1} \]
    multiplying both sides by $P^{-1}_{n+1}$ we get
    \[ P_1\cdots P_n = Q_1\cdots Q_{i-1}\cdot Q_i\cdots Q_m \]
    and thus by our inductive assumption, $m=n+1$ and the above products have the same coefficients.
    \qed

\end{proof}

\begin{prop*}

    Suppose $R$ is a Dedekind domain, then $R$ is a UFD if and only if $R$ is a PID.

\end{prop*}

\begin{proof}

    If $R$ is a PID, then $R$ is a UFD.
    For the converse, suppose $R$ is a Unique factorization Dedekind domain.
    Let $P$ be a prime ideal of $R$, if $P=(0)$ then $P$ is of course principal.
    Otherwise suppose $0\neq y\in P$ then since $R$ is a UFD, there exist $p_i$ irreducible such that
    \[ y = p_1\cdots p_n \]
    but $P$ is prime, and so there exists a $p_i\in P$.
    Since in a UFD every irreducible element is prime, $p_i$ is prime and so $(p_i)\subseteq P$ is prime.
    But since $0\neq p_i$, $(0)\neq(p_i)\subseteq P$ and $\dim R=1$ so $P=(p_i)$.

    Thus every prime ideal is principal.
    If $I$ is any ideal, then since it is the product of prime ideals, which are of the form $(p_i)$, we have
    \[ I = (p_1)\cdots(p_n) = (p_1\cdots p_n) \]
    and so $R$ is a PID.
    \qed

\end{proof}

\end{document}

