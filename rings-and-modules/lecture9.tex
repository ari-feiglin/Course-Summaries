\documentclass[10pt]{article}

\usepackage{amsmath, amssymb, mathtools}
\usepackage[margin=1.5cm]{geometry}

\catcode`\@=11
\input pdfmsym

\input prettyprint
\input preamble

\pdfmsymsetscalefactor{10}
\initpps

\def\implies{\,\mathrel{\longvarRightarrow}\,}

\def\pmat#1{\begin{pmatrix} #1 \end{pmatrix}}

\def\divides{{\mid}}

\font\bigbf = cmbx12 scaled 2000

\newfunc{eulerg}{{\rm Euler}}({})
\newfunc{order}o({})
\newfunc{ker}{{\rm Ker}}({})
\newfunc{image}{{\rm Im}}({})
\newfunc{lattice}{{\cal L}}(\vert)
\newfunc{powset}{{\cal P}}(\vert)
\newfunc{center}{{\rm Z}}({})
\newfunc{units}{{\cal U}}(\vert)
\newfunc{sign}{{\rm sgn}}({})
\newfunc{aut}{{\rm Aut}}({})
\newfunc{speclinear}{{\rm SL}}({})
\newfunc{genlinear}{{\rm GL}}({})
\newfunc{projlinear}{{\rm PGL}}({})
\newfunc{specprojlinear}{{\rm PSL}}({})
\newfunc{core}{{\rm Core}}(\vert)
\newfunc{inner}{{\rm Inn}}(\vert)
\newfunc{outer}{{\rm Out}}(\vert)
\newfunc{conj}{{\rm conj}}({})
\newfunc{ev}{{\rm ev}}({})

\def\maxdivs{\mathrel{\|}}

\let\ideal=\trianglelefteq
\let\pideal=\triangleleft
\def\normal{\mathrel\triangleleft}
\let\mmorph=\longvarhookrightarrow
\let\to=\varrightarrow
\let\longto=\longvarrightarrow

\def\qed{\hskip1cm\hbox{}\hfill$\blacksquare$}

\def\mN{\mathcal N}
\def\mG{\mathcal G}

\begin{document}

\c@section=9

\barcolorbox{220, 220, 255}{0, 0, 130}{80, 80, 200}{
    \leftskip=0pt plus 1fill \rightskip=\leftskip
    {\bigbf Introduction to Rings and Modules}

    \medskip
    \textit{Lecture \thesection, Monday May 15 2023}

    \textit{Ari Feiglin}
}

\bigskip

\begin{exam*}

    We define $\bZ[\sqrt{-5}]=\set{a+b\sqrt{-5}}[a,b\in\bZ]\subseteq\bC$.
    $\bZ[\sqrt{-5}]$ is an integral domain, since $\bC$ is a field so it has no zero divisors.
    We claim that $\bZ[\sqrt{-5}]$ is not a UFD.

    For $z\in\bZ[\sqrt{-5}]$, we define the norm by $N(z)=\abs z^2=z\cdot\overline z$.
    In this case we have $N(a+b\sqrt{-5})=a^2+5b^2$, and it is multiplicative.
    This is a norm since $a^2+5b^2\in\bN$.
    Notice that if $u$ is invertible then there exists a $v$ such that $uv=1$ and so $N(u)N(v)=1$, and since these are natural numbers, this means $N(u)=N(v)=1$.
    And $N(a+b\sqrt{-5})=a^2+5b^2=1$ if and only if $a^2=1$, since if $b\neq0$, $5b^2\geq5$.
    So the only invertible elements in $\bZ[\sqrt{-5}]$ are $\pm1$.
    Thus two elements are friends if and only if they are equal up to sign.

    Notice that
    \[ 6 = 2\cdot3 = (1+\sqrt{-5})(1-\sqrt{-5}) \]
    so if we can show that these are all irreducible, we have finished since they are not friends.

    Suppose $xy=2$ is a decomposition into irreducible elements, then $N(x)N(y)=N(2)=4$, and $N(x),N(y)\neq1$ so $N(x)=N(y)=2$, but if $x=a+b\sqrt{-5}$ then $N(x)=a^2+5b^2=2$.
    But this has no solution over the integers.
    Similar for $xy=3$, $N(x)N(y)=N(3)=9$ and so $N(x)=3$, but there are no integers which satisfy $a^2+5b^2=3$ and therefore $2$ and $3$ are irreducible.

    Similarly $N(1\pm\sqrt{-5})=6$ so if $N(x)N(y)=6$ then $N(x)=2$ or $N(x)=3$, which has no solutions, so $1\pm\sqrt{-5}$ is irreducible as well.

\end{exam*}

\begin{defn*}

    Let $R$ be a commutative ring and $a,b\in R$, then $d\in R$ is called a \ppemph{greatest common divisor} (not necessarily unique) if:
    \benum
        \item $a,b\in(d)$ (meaning $d\divides a,b$)
        \item If $I\subseteq R$ is a prime ideal such that $a,b\in I$ then $(d)\subseteq I$
    \eenum

\end{defn*}

Notice that if $R$ is a ring and $d$ and $d'$ are greatest common divisors of $a$ and $b$, then $(d)=(d')$.
This is because $a,b\in(d')$ so $(d)\subseteq(d')$ since it is a greatest common divisor, and similarly $(d')\subseteq(d)$.
So if $d$ is a greatest common divisor of $a$ and $b$, then $d'$ is a greatest common divisor if and only if $(d)=(d')$.
Thus if $d$ and $d'$ are friends, since $(d)=(d')$, $d'$ is also a greatest common divisor.
In an integral domain, this is an equivalence: $d'$ is a greatest common divisor if and only if $d$ and $d'$ are friends.

Notice that if $R$ is a PID, then $(a,b)=(d)$ and $d$ is a greatest common divisor of $a$ and $b$ (since if $a,b\in I$ then $(a,b)\subseteq I$).
This condition is both necessary and sufficient.

Since $\bZ$ is a PID, and $a\bZ+b\bZ=\gcd(a,b)\bZ$ (and this is the only non-negative number where this is true), so this satisfies our definition of greatest common divisiors in integers.
(In $\bZ$, the greatest common divisors of $a$ and $b$ are $\pm\gcd(a,b)$.)

\begin{exam*}

    Let $R=\bZ[\sqrt{-5}]$ and $\alpha=6$ and $\beta=2(1+\sqrt{-5})$.
    Then $\alpha$ and $\beta$ have no gcd.

    This is because $N(\alpha)=36$ and $N(\beta)=24$.
    If $\alpha,\beta\in(d)$ then $\alpha=d\gamma$ and $\beta=d\delta$ so $36=N(d)N(\gamma)$ and $24=N(d)N(\delta)$.
    So $N(d)$ divides $36$ and $24$, and therefore divides their gcd, $12$.
    But there are no elements with norm $12$, since if $b=1$ then $a^2=7$ and if $b=0$ then $a^2=12$..

    Since $1+\sqrt{-5}$ divides both $\alpha$ and $\beta$ (by our previous example), and so $I=(1+\sqrt{-5})$ is a prime ideal containing $\alpha$ and $\beta$, so $(d)\subseteq I$.
    Meaning $(1+\sqrt{-5})\divides d$, so $N(1+\sqrt{-5})=6\divides N(d)$, and since $N(d)$ divides $12$ and there is no element of norm $12$, $N(d)=6$.

    But $d=(1+\sqrt{-5})c$ so $N(c)=1$, and so $c$ is invertible, and therefore $1+\sqrt{-5}$ is a gcd of $\alpha$ and $\beta$.
    But $2\divides\alpha$, $2\divides\beta$ and so $\alpha,\beta\in(2)$ and so $(d)\subseteq(2)$, so $2\divides d$.
    Therefore $d=2x$ and so $6=N(d)=N(2)N(x)=4N(x)$ which is a contradiction.

    So they have no gcd.

\end{exam*}

\begin{defn*}

    An integral domain $R$ is called a \ppemph{gcd domain} if every two non-zero elements have a gcd.

\end{defn*}

\begin{prop*}

    Every UFD is a gcd domain.

\end{prop*}

\begin{proof}

    Suppose $\alpha,\beta\in R$ non-zero and non-invertible, then there exist irreducible $p_i$ such that
    \[ \alpha = p_1^{e_1}\cdots p_n^{e_n},\qquad \beta = p_1^{f_1}\cdots p_n^{f_n}u \]
    where $e_i,f_i\geq0$, $u$ is invertible, and $p_i$ are all different and not friends (hence the $u$).
    We claim that
    \[ d = p_1^{g_1}\cdots p_n^{g_n} \]
    where $g_i=\minof{e_i,f_i}$, is a gcd of $\alpha$ and $\beta$.

    This is true since $d\divides\alpha$ and $d\divides\beta$ so $\alpha,\beta\in(d)$.
    Now suppose $\alpha,\beta\in(t)$ then $\alpha=xt$ and $\beta=xs$.
    Since factorization in a UFD is by definition unique, the factorizations of $x$, $t$, and $s$ must contain only $p_i$s and a unit.
    Suppose
    \[ x = p_1^{x_1}\cdots p_n^{x_n}\cdot v \]
    Since $xt=\alpha$, and the factorization of $t$ contains only $p_i$s, we must have that $x_i\leq e_i$ and similarly $x_i\leq f_i$, so $x_i\leq g_i$.
    Therefore $x\divides d$ and therefore $d\in(x)$ so $(d)\subseteq(d)$ as required.
    \qed

\end{proof}

\end{document}

