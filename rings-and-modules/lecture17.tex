\documentclass[10pt]{article}
\usepackage{amsmath, amssymb, mathtools}
\usepackage[margin=1.5cm]{geometry}
\usepackage{mathrsfs}

\catcode`\@=11
\input pdfmsym

\input prettyprint
\input preamble

\pdfmsymsetscalefactor{10}
\initpps

\def\implies{\,\mathrel{\longvarRightarrow}\,}

\def\pmat#1{\begin{pmatrix} #1 \end{pmatrix}}

\def\divides{{\mid}}

\font\bigbf = cmbx12 scaled 2000

\newfunc{eulerg}{{\rm Euler}}({})
\newfunc{order}o({})
\newfunc{ker}{{\rm Ker}}({})
\newfunc{image}{{\rm Im}}({})
\newfunc{lattice}{{\cal L}}(\vert)
\newfunc{powset}{{\cal P}}(\vert)
\newfunc{center}{{\rm Z}}({})
\newfunc{units}{{\cal U}}(\vert)
\newfunc{sign}{{\rm sgn}}({})
\newfunc{aut}{{\rm Aut}}({})
\newfunc{speclinear}{{\rm SL}}({})
\newfunc{genlinear}{{\rm GL}}({})
\newfunc{projlinear}{{\rm PGL}}({})
\newfunc{specprojlinear}{{\rm PSL}}({})
\newfunc{core}{{\rm Core}}(\vert)
\newfunc{inner}{{\rm Inn}}(\vert)
\newfunc{outer}{{\rm Out}}(\vert)
\newfunc{conj}{{\rm conj}}({})
\newfunc{ev}{{\rm ev}}({})
\newfunc{Frac}{{\rm Frac}}({})
\newfunc{Tor}{{\rm Tor}}(\vert)
\newfunc{Ann}{{\rm Ann}}(\vert)
\newfunc{deg}{{\rm deg}}({})
\newfunc{adj}{{\rm adj}}({})

\def\maxdivs{\mathrel{\|}}

\let\ideal=\trianglelefteq
\let\pideal=\triangleleft
\def\normal{\mathrel\triangleleft}
\let\mmorph=\longvarhookrightarrow
\let\to=\varrightarrow
\let\longto=\longvarrightarrow

\def\qed{%
    \ifmmode%
        \eqno\blacksquare%
    \else%
        \hskip1cm\allowbreak\hbox{}\nobreak\hfill$\blacksquare$%
    \fi%
}

\def\mN{\mathcal N}
\def\mO{\mathcal O}
\def\mG{\mathcal G}
\def\sS{\mathscr S}

\begin{document}

\c@section=17

\barcolorbox{220, 220, 255}{0, 0, 130}{80, 80, 200}{
    \leftskip=0pt plus 1fill \rightskip=\leftskip
    {\bigbf Introduction to Rings and Modules}

    \medskip
    \textit{Lecture \thesection, Friday June 16 2023}

    \textit{Ari Feiglin}
}

\bigskip

\begin{defn*}

    Suppose $R\subseteq S$ are rings and $s\in S$, then $R[s]$ is the smallest subring of $S$ which contains both $R$ and $s$:
    \[ R[s] = \set{a_ns^n+a_{n-1}s^{n-1}+\cdots+a_1s+a_0}[n\in\bN_0,\,a_i\in R] \]

\end{defn*}

Note if $s^k\in R$ then
\[ R[s] = \set{a_ks^k+\cdots+a_0}[a_i\in R] \]

\begin{defn*}

    An $R$-module $M$ is \ppemph{faithful} if
    \[ \Annof[R]M = (0) \]
    meaning for every $r\in R$, there exists an $m\in M$ where $rm\neq0$.

\end{defn*}

\begin{prop*}

    Suppose $R\subseteq S$ are commutative.
    Let $s\in S$, then the following are equivalent
    \benum
        \item $s$ is integral over $R$.
        \item $R[s]$ is a finitely-generated $R$-module.
        \item There exists a ring $T$ such that $R[s]\subseteq T\subseteq S$ and $T$ is finitely-generated as an $R$-module.
        \item There exists a faithful $R[s]$-module $M$ which is finitely-generated as an $R$-module.
    \eenum

\end{prop*}

\begin{proof}

   Firstly, $2\varRightarrow3$ is trivial, take $T=R[s]$.
   And for $3\varRightarrow4$ let $M=T$, which is finitely-generated as an $R$-module.
   Suppose $\alpha\in\Annof[{R[s]}]T$ so $\alpha t=0$ for every $t\in T$.
   Since $T$ is a ring, take $1\in T$, so we get $\alpha=0$.
   Thus $T$ is a faithful $R[s]$-module.

   Now for $1\varRightarrow2$, we know that
   \[ s^n + b_{n-1}s^{n-1} + \cdots + b_0 = 0 \]
   for $b_i\in R$ since $s$ is integral over $R$.
   Then we claim
   \[ R[s] = \gen{1,s,\dots,s^{n-1}} \]
   It is sufficient to show that every $s^m$ is generated by these elements, since every element in $R[s]$ is a linear combination of $s^m$s.
   For $m<n$ this is trivial, and note that
   \[ s^n = -b_{n-1}s^{n-1} - \cdots - b_0 \]
   so it is true for $m=n$.
   We will show this is true for exponents $s^{n+m}$, we proved the base case for $m=0$ already.
   Now we know by our inductive hypotehsis
   \[ s^m = a_{n-1}s^{n-1} + \cdots + a_0 \]
   and so
   \[ s^{m+1} = a_{n-1}s^n + a_{n-2}s^{n-1} + \cdots + a_0s \]
   and since $s^n$ is a linear combination of $1,s,\dots,s^n$ so is $s^{m+1}$ as required.

   Finally for $4\varRightarrow1$, suppose $M$ is a faithful $R[s]$-module, which is also finitely generated as an $R$-module.
   Suppose $M=\gen{m_1,\dots,m_n}$, now for $m\in M$ we have
   \[ m = a_1m_1 + \cdots + a_nm_n \]
   for some $a_i\in R$, and this is true for $sm_i$ in particular
   \[ sm_i = a_{i,1}m_1 + \cdots + a_{i,n}m_n \]
   and thus we have that
   \[ \begin{pmatrix} sm_1\\\vdots\\sm_n \end{pmatrix} = \begin{pmatrix} a_{11} &\cdots&a_{1n}\\\vdots&\ddots&\vdots\\a_{n1}&\cdots&a_{nn} \end{pmatrix}\cdot\begin{pmatrix} m_1\\\vdots\\m_n\end{pmatrix} \]
   Thus
   \[ (sI_n-A)\begin{pmatrix} m_1\\\vdots\\m_n \end{pmatrix} = 0 \]
   notice that $sI_n-A$ is a matrix with coefficients in $R[s]$.
   The adjugate of matrices are defined still in commutative rings as there is no need for division, so we can multiply both sides by $\adj(sI_n-A)$ and get
   \[ \det(sI_n-A)\begin{pmatrix} m_1\\\vdots\\m_n \end{pmatrix} = 0 \]
   Thus $\det(sI_n-A)m_i=0$ for every $i$ and thus $\det(sI_n-A)m=0$ for every $m\in M$, so $\det(sI_n-A)\in\Annof[{R[s]}]M$.
   Thus we have $\det(sI_n-A)=0$, but we also know $\det(sI_n-A)$ is the characteristic polynomial of $A$ over $R$ evaluated at $s$ (which is in $R[s]$), meaning $p_A(s)=0$.
   Since $p_A$ is a monic polynomial over $R$, we have that $s$ is then integral over $R$.
   \qed

\end{proof}

\begin{prop*}

    Suppose $R\subseteq S$ are commutative rings, then
    \[ T=\set{s\in S}[s\text{ is integral over }R] \]
    is a subring of $S$.

\end{prop*}

\begin{proof}

    Since $R$ is integral over itself, $R\subseteq T$ and in particular $1_S=1_R\in R\subseteq T$.
    Now suppose $t_1,t_2\in T$ then let us look at
    \[ R[t_1,t_2] = \set{\sum a_{m,n}t_1^mt_2^n}[a_{m,n}\in R] \]
    which is $R[t_1][t_2]$ or the smallest ring containing $R$, $t_1$, and $t_2$.
    But since
    \[ t_1^n + a_{n-1}t^{n-1} + \cdots + a_0 = 0,\qquad t_2^m +  b_{m-1}t^{m-1} + \cdots + b_0 = 0 \]
    for $a_i,b_i\in R$.
    Thus every exponent of $t_1$ is a linear combination over $R$ of $1,t_1,\dots,t_1^{n-1}$ and similarly exponents of $t_2$ are linear combinations over $R$ of $1,t_2,\dots,t_2^{m-1}$.
    Thus the products of exponents $t_1$ and $t_2$ are linear combinations of coefficients of the form $t_1^it_2^j$ for $0\leq i\leq n-1$ and $0\leq j\leq m-1$.
    Thus
    \[ R[t_1,t_2] = \gen{t_1^it_2^j}[0\leq i\leq n-1,\,0\leq j\leq m-1] \]
    So $R[t_1,t_2]$ is a finitely-generated $R$-module.

    So we must show that $t_1+t_2\in T$ and $t_1t_2\in T$.
    Since we know
    \[ R[t_1+t_2], R[t_1t_2]\subseteq R[t_1,t_2] \subseteq S \]
    Thus by the above proposition, $t_1+t_2$ and $t_1t_2$ are integral over $R$, ie. $t_1+t_2,\,t_1t_2\in T$ as required.
    Thus $T$ is a ring, and it contains $1_S$ so it is a subring of $S$'s.
    \qed

\end{proof}

\begin{defn*}

    Suppose $R\subseteq S$ are commutative rings, then
    \[ \set{s\in S}[s\text{ is integral over }R] \]
    is called the \ppemph{integral closure of $R$ over $S$}.

\end{defn*}

\begin{lemm*}

    If $R\subseteq S\subseteq T$ such that $S$ is a finitely-generated $R$-module and $T$ a finitely generated $S$-module, then $T$ is a finitely generated $R$-module.

\end{lemm*}

\begin{proof}

    Suppose $S=\gen{a_1,\dots,a_n}$ and $T=\gen{b_1,\dots,b_m}$ then every $t\in T$ is of the form
    \[ t = \sum_{i=1}^m s_ib_i = \sum_{i=1}^m\sum_{j=1}^n r_{i,j}a_jb_i \]
    and thus $T=\gen{a_jb_i}[1\leq j\leq n,\,1\leq i\leq m]$ as required.
    \qed

\end{proof}

\begin{prop*}

    If $R\subseteq S\subseteq T$ are rings where $S$ is integral over $R$, and $T$ is integral over $S$, then $T$ is integral over $R$.

\end{prop*}

\begin{proof}

    Suppose $t\in T$, then $t$ is integral over $S$ so
    \[ t^n + s_{n-1}t^{n-1} + \cdots + s_0 = 0 \]
    Thus $t$ is integral over $R[s_0,\dots,s_{n-1}]$ and so $R[s_0,\dots,s_{n-1},t]$ is a finitely generated $R[s_0,\dots,s_{n-1}]$-module.
    But $R[s_0,\dots,s_{n-1}]$ is finitely-generated over $R$ since $s_i$ are integral over $R$.
    Thus $R\subseteq R[s_0,\dots,s_{n-1}]\subseteq R[s_0,\dots,s_{n-1},t]$ where each successive rings form an integral extension, thus by the lemma above, $R[s_0,\dots,s_{n-1},t]$ is a finitely-generated
    module over $R$.
    Thus we have $R\subseteq R[t]\subseteq R[s_0,\dots,s_{n-1},t]\subseteq T$ where $R[s_0,\dots,s_{n-1},t]$ is finitely-generated so $t$ is integral over $R$.
    \qed

\end{proof}

\begin{prop*}

    Integral closures are integrally closed.

\end{prop*}

\begin{proof}

    Suppose $R\subseteq S$ and $T$ is the integral closure of $R$ over $S$.
    Let $T'$ be $T$'s integral closure over $S$, then $R\subseteq T\subseteq T'$ where $T$ is integral over $R$ and $T'$ over $T$, thus $T'$ is integral over $R$ so $T'\subseteq T$, meaning $T'=T$.
    So every element in $S$ which is integral over $T$ is in $T$, meaning $T$ is integrally closed.
    \qed

\end{proof}

\end{document}

