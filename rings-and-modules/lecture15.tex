\documentclass[10pt]{article}

\usepackage{amsmath, amssymb, mathtools}
\usepackage[margin=1.5cm]{geometry}
\usepackage{mathrsfs}

\catcode`\@=11
\input pdfmsym

\input prettyprint
\input preamble

\pdfmsymsetscalefactor{10}
\initpps

\def\implies{\,\mathrel{\longvarRightarrow}\,}

\def\pmat#1{\begin{pmatrix} #1 \end{pmatrix}}

\def\divides{{\mid}}

\font\bigbf = cmbx12 scaled 2000

\newfunc{eulerg}{{\rm Euler}}({})
\newfunc{order}o({})
\newfunc{ker}{{\rm Ker}}({})
\newfunc{image}{{\rm Im}}({})
\newfunc{lattice}{{\cal L}}(\vert)
\newfunc{powset}{{\cal P}}(\vert)
\newfunc{center}{{\rm Z}}({})
\newfunc{units}{{\cal U}}(\vert)
\newfunc{sign}{{\rm sgn}}({})
\newfunc{aut}{{\rm Aut}}({})
\newfunc{speclinear}{{\rm SL}}({})
\newfunc{genlinear}{{\rm GL}}({})
\newfunc{projlinear}{{\rm PGL}}({})
\newfunc{specprojlinear}{{\rm PSL}}({})
\newfunc{core}{{\rm Core}}(\vert)
\newfunc{inner}{{\rm Inn}}(\vert)
\newfunc{outer}{{\rm Out}}(\vert)
\newfunc{conj}{{\rm conj}}({})
\newfunc{ev}{{\rm ev}}({})
\newfunc{Frac}{{\rm Frac}}({})
\newfunc{Tor}{{\rm Tor}}(\vert)
\newfunc{Ann}{{\rm Ann}}(\vert)

\def\maxdivs{\mathrel{\|}}

\let\ideal=\trianglelefteq
\let\pideal=\triangleleft
\def\normal{\mathrel\triangleleft}
\let\mmorph=\longvarhookrightarrow
\let\to=\varrightarrow
\let\longto=\longvarrightarrow

\def\qed{\hskip1cm\hbox{}\hfill$\blacksquare$}

\def\mN{\mathcal N}
\def\mG{\mathcal G}
\def\sS{\mathscr S}

\begin{document}

\c@section=15

\barcolorbox{220, 220, 255}{0, 0, 130}{80, 80, 200}{
    \leftskip=0pt plus 1fill \rightskip=\leftskip
    {\bigbf Introduction to Rings and Modules}

    \medskip
    \textit{Lecture \thesection, Monday June 12 2023}

    \textit{Ari Feiglin}
}

\bigskip

\begin{defn*}

    Let $R$ be a ring, an $R$-module $M$ is \ppemph{cyclic} if it is generated by a single element.
    Ie. $M$ is cyclic if and only if there exists an $m\in M$ such that
    \[ M = \gen m = \set{rm}[r\in R] \]

\end{defn*}

Note that submodules of $R$ (when viewed as a module over itself), are cyclic if and only if they are principal ideals.

\begin{prop*}

    Let $M$ be an $R$-module, then $M$ is cyclic if and only if there exists a left ideal $I\leq R$ such that $M\cong\slfrac RI$ as modules.

\end{prop*}

Note that $\slfrac RI$ is not necessarily a ring, but since $I\subseteq R$ is a submodule of $R$, $\slfrac RI$ is a module over $R$.

\begin{proof}

    If $M\cong\slfrac RI$ then it is sufficient to show that $\slfrac RI$ is cyclic.
    This is trivial as everything in the quotient can be generated by $1+I$.

    If $M$ is cyclic, let us define a module homomorphism $f\colon R\longto M$ by $f(r)=rm$ where $m$ generates $M$.
    This is a module homomorphism as $f(r_1+r_2)=(r_1+r_2)m=r_1m+r_2m=f(r_1)+f(r_2)$ and $f(r_1r_2)=r_1r_2m=r_1f(r_2)$ as required.
    This is surjective and so
    \[ \slfrac R{\ker f}\cong M \]
    and $\ker f$ is a submodule of $R$ and thus a left ideal.

\end{proof}

Suppose $R$ is a commutative ring, what is a module over the ring $R[x]$?
Let $M$ be a module over $R[x]$, thus $M$ is an abelian group, and it has left multiplication defined by polynomials in $R[x]$.
Since constants are polynomials in $R[x]$, left multiplication on $M$ by $R$ is defined.
So every module over $R[x]$ is a module over $R$ (a linear space).
Thus far, this is true for rings in general.

Note since if $R$ is commutative, the mapping
\[ \phi\colon M\longto M,\quad m\varmapsto xm \]
is a module homomorphism for $x\in R$, since $\phi(m_1+m_2)=x(m_1+m_2)=xm_1+xm_2=\phi(m_1)+\phi(m_2)$, and $\phi(rm)=xrm=rxm=r\phi(m)$.
Let us call this the ``scalar multiplication mapping of $x$''.

Thus if we are given a module $M$ over $R[x]$ as a module over $M$ and the scalar multiplication mapping of $x$ $\phi\colon M\longto M$, this is enough to define the operations of $M$ over $R[x]$.
In other words, a module over $R[x]$ is a module over $R$ and the definition of $x\cdot m$ for $m\in M$.

This is because for $m\in M$ and $f=a_nx^n+\cdots+a_0\in R[x]$ we have
\[ fm = a_nx^nm + \cdots + a_0m \]
and since we have the definition of $xm$ for every $m\in M$, we have the definition of $x^nm$ inductively as $x^nm=x^{n-1}(xm)$.
Or in other words $x^nm=\phi^n(m)$, where $\phi$ is the scalar multiplication mapping of $x$.

And for the converse, if $M$ is a module over a commutative ring $R$, and $\phi\colon M\longto M$ is a module homomorphism, if we define $x\cdot m=\phi(m)$ then this defines a module over $R[x]$:

Thus we get the following

\begin{prop*}

    If $R$ is a commutative ring, then $R[x]$ modules are equivalent to $R$-modules with a module homomorphism over themselves.

\end{prop*}

\begin{exam*}

    Let $R$ be a commutative ring, we will investigate the $R$-module 
    \[ M = \slfrac{R[x]}{((x-\lambda)^n))} \]
    for $n\in\bN$.
    Since this is a quotient of the ring $R[x]$, $M$ is cyclic.
    By polynomial division, for any $f\in R[x]$ we have that
    \[ f = q\cdot(x-\lambda)^n+r \]
    for $\deg r<n$.
    Thus as we know, $\slfrac{R[x]}{((x-\lambda)^n)}$ is the set of classes whose representatives are polynomials of degree $<n$: $a_{n-1}x^{n-1}+\cdots+a_0$.
    Such a set induces a natural $R$-module, where we simply scale each polynomial (this is independent of the representative of the class, since if $f-g\in((x-\lambda)^n$ then so is $rf-rg$).

    We take the basis $B=\set{(x-\lambda)^k + I}[0\leq k<n]$ (where $I=((x-\lambda)^n)$).
    This is a generating set as we can induct on the degree of $f\in I$.
    If $f$ is constant, then it is a multiple of $(x-\lambda)^0$.
    Otherwise we can take the degree of $f$, $m$, and we have $f=q(x-\lambda)^m+r$ where $\deg r<m$ and thus induct on $r$ (since $q$ must be constant as otherwise the degree of $q(x-\lambda)^m$ would be
    more than $m$).
    This set is a linearly independent since any sum $\sum a_k(x-\lambda)^k=0$, every $a_k=0$ as otherwise let $k$ be the maximum $k$ where $a_k\neq0$, and then the sum has a degree of $k$ in contradiction.

    And so if we let $\phi$ be the scalar multiplication mapping of $x$
    \[ \phi((x-\lambda)^{n-1} + I) = x((x-\lambda)^{n-1} + I) = x(x-\lambda)^{n-1} + I = (x-\lambda)^n + \lambda(x-\lambda)^{n-1} + I = \lambda(x-\lambda)^{n-1} + I \]
    and in general
    \[ \phi((x-\lambda)^k + I) = \bigl((x-\lambda)^{k+1} + I\bigr) + \lambda\bigl((x-\lambda)^k + I\bigr) \]
    So the matrix representing this homomorphism (relative to the basis $B$) is the Jordan block of size $n$ (we count $(x-\lambda)^{n-1}+I$ as the first element in the basis, and go in reverse).

\end{exam*}

\begin{exam*}

    Let us now look at the cyclic module
    \[ M = \slfrac{R[x]}{(p)} \]
    for some $p\in F[x]$.
    We assume that $p$ has a leading coefficient which is $1$, and thus elements of $M$ can be represented as polynomials of degree less than that of $p$'s.
    We similarly denote $I=(p)$.

    But now let us focus on the basis $B=(1+I,x+I,\dots,x^{n-1}+I)$ where $n=\deg p$.
    Then
    \[ \phi(x^k + I) = x^{k+1} + I \]
    If
    \[ p = x^n + a_{n-1}x^{n-1} + \cdots + a_0 \]
    then we have that
    \[ x^n + I = x^n - p + I = -a_{n-1}x^{n-1} - \cdots - a_0 + I \]
    and so the matrix of $\phi$ is of the form
    \[ \begin{pmatrix} 
            0 & 0 & 0 & \cdots & 0 & -a_0 \\
            1 & 0 & 0 & \cdots & 0 & -a_1 \\
            0 & 1 & 0 & \cdots & 0 & -a_2 \\
            0 & 0 & 1 & \cdots & 0 & -a_3 \\
            \vdots & \vdots & \vdots & \ddots & \vdots & \vdots \\
            0 & 0 & 0 & \cdots & 1 & -a_{n-1}
    \end{pmatrix} \]

    This is called the \ppemph{companion matrix} of $p$ and is denote $C_p$.

\end{exam*}

\begin{defn*}

    If $M$ and $N$ are $R$-modules, then we can define the $R$-module $M\times N$ by defining $r\cdot(m,n)=(rm,rn)$.

\end{defn*}

\begin{thrm*}[classificationTheorem,Classification\ Theorem]

    Let $R$ be a PID, and $M$ be a finitely generated $R$-module.
    Then
    \benum
        \item There exist $d_1,\dots,d_n\in R$ which are unique up to friends where
        \[ M \cong \slfrac R{(d_1)}\times\cdots\times\slfrac R{(d_n)} \]
        and $d_i\divides d_{i+1}$ for every $1\leq i<n$.
        These are called \ppemph{invariant elements}.

        \item There exist $p_1,\dots,p_t$ irreducible elements, unique up to friends and $r,n_1,\dots,n_t\geq0$ such that
        \[ M\cong R^r\times\slfrac R{(p_1^{n_1})}\times\cdots\times\slfrac R{(p_t^{n_t})} \]
    \eenum

\end{thrm*}

Note that if $R=\bZ$, then we get the classification theorem for abelian groups.

\end{document}

