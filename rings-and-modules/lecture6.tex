\documentclass[10pt]{article}

\usepackage{amsmath, amssymb, mathtools}
\usepackage[margin=1.5cm]{geometry}

\catcode`\@=11
\input pdfmsym

\input prettyprint
\input preamble

\pdfmsymsetscalefactor{10}
\initpps

\def\implies{\,\mathrel{\longvarRightarrow}\,}

\def\pmat#1{\begin{pmatrix} #1 \end{pmatrix}}

\let\divides=\mid

\font\bigbf = cmbx12 scaled 2000

\newfunc{eulerg}{{\rm Euler}}({})
\newfunc{order}o({})
\newfunc{ker}{{\rm Ker}}({})
\newfunc{image}{{\rm Im}}({})
\newfunc{lattice}{{\cal L}}(\vert)
\newfunc{powset}{{\cal P}}(\vert)
\newfunc{center}{{\rm Z}}({})
\newfunc{units}{{\cal U}}(\vert)
\newfunc{sign}{{\rm sgn}}({})
\newfunc{aut}{{\rm Aut}}({})
\newfunc{speclinear}{{\rm SL}}({})
\newfunc{genlinear}{{\rm GL}}({})
\newfunc{projlinear}{{\rm PGL}}({})
\newfunc{specprojlinear}{{\rm PSL}}({})
\newfunc{core}{{\rm Core}}(\vert)
\newfunc{inner}{{\rm Inn}}(\vert)
\newfunc{outer}{{\rm Out}}(\vert)
\newfunc{conj}{{\rm conj}}({})
\newfunc{ev}{{\rm ev}}({})

\def\maxdivs{\mathrel{\|}}

\let\ideal=\trianglelefteq
\let\pideal=\triangleleft
\def\normal{\mathrel\triangleleft}
\let\mmorph=\longvarhookrightarrow
\let\to=\varrightarrow
\let\longto=\longvarrightarrow

\def\qed{\hskip1cm\hbox{}\hfill$\blacksquare$}

\def\mN{\mathcal N}
\def\mG{\mathcal G}

\begin{document}

\c@section=6

\barcolorbox{220, 220, 255}{0, 0, 130}{80, 80, 200}{
    \leftskip=0pt plus 1fill \rightskip=\leftskip
    {\bigbf Introduction to Rings and Modules}

    \medskip
    \textit{Lecture \thesection, Monday May 1 2023}

    \textit{Ari Feiglin}
}

\bigskip

\subsection{Prime Ideals}

\begin{defn*}

    Let $R$ be a commutative ring, a proper ideal $I\pideal R$ is \ppemph{prime} if for every $a,b\in R$ if $ab\in I$ then $a\in I$ or $b\in I$.

\end{defn*}

\begin{exam*}

    \benum
        \item Notice then that if $p$ is prime, then $p\bZ$ is a prime ideal.
        This is because if $nm\in p\bZ$ then $p$ divides $nm$ and therefore divides $n$ or $m$ and so one is in $p\bZ$.

        And if $n$ is not prime then suppose $p$ is a prime which divides $n$, then $p\cdot\frac np\in n\bZ$ but neither $p$ nor $\frac np$ are in $n\bZ$, so $n\bZ$ is not prime.
        So the only prime ideals of $\bZ$ are $p\bZ$.

        \item And $\set 0$ is a prime ideal if and only if $R$ is an integral domain.
        This is because $ab=0$ if and only if $ab\in I$.
    \eenum

\end{exam*}

\begin{prop*}

    Let $R$ be a commutative ring, and $I\ideal R$.
    The following are equivalent:
    \benum
        \item $I$ is a prime ideal.
        \item For any $J,J'\ideal R$, if $JJ'\subseteq I$ then $J\subseteq I$ or $J'\subseteq I$.
        \item $\slfrac RI$ is an integral domain.
    \eenum

\end{prop*}

\begin{proof}

    We show the first equivalence.
    Suppose that there is an $a\in J$ which isn't in $I$ and a $b\in J'$ which isn't in $I$.
    But $ab\in JJ'\subseteq I$ and since $I$ is prime, $a\in I$ or $b\in I$.

    Now we show the second equivalence.
    Suppose that
    \[ (r + I)(r' + I) = I \implies rr' + I = I \]
    then $rr'\in I$ so one must be in $I$, so one of $r+I$ or $r'+I$ is $0$, so $\slfrac RI$ is an integral domain.

    We show that the third implies the first.
    Suppose $ab\in I$ then $(a+I)(b+I)=ab+I=0_{\slfrac RI}$.
    But since $\slfrac RI$ is an integral domain, $a+I$ or $b+I$ must be $0$ so either $a\in I$ or $b\in I$ as required.
    \qed

\end{proof}

\begin{exam*}

    Let $R=\bZ[x]$ and $I=(2)=\set{2f(x)}[f(x)\in{\bZ[x]}]$.
    Then we can form a ring homomorphism $\phi\colon\bZ[x]\longto\bZ_2[x]$ by
    \[ \phi\parens{\sum_{k=0}^n a_kx^k}=\sum_{k=0}^n [a_k]x^k \]
    This is obviously a group homomorphism, and it preserves multiplication as
    \[ \phi\parens{\parens{\sum_{k=0}^n a_kx^k}\cdot\parens{\sum_{k=0}^n b_kx^k}} = \phi\parens{\sum_{k=0}^{2n}x^k\sum_{i=0}^k a_ib_{k-i}} = \sum_{k=0}^{2n}x^k\sum_{i=0}^k[a_ib_{k-i}] \]
    which is equal to the product of the images of the polynomials under $\phi$.

    The kernel of this homomorphism is the set of polynomials with even coefficients, $\set{p\in\bZ[x]}[p_k\in2\bZ]=(2)$.
    So by the first isomorphism theorem
    \[ \slfrac{\bZ[x]}{(2)}\cong\bZ_2[x] \]
    Since $\bZ_2$ is an integral domain, so is $\bZ_2[x]$ and therefore $\slfrac{\bZ[x]}{(2)}$ is an integral domain so $(2)$ is prime.

\end{exam*}

\begin{exam*}

    Since we showed that $\slfrac{\bZ[x]}{(x)}\cong\bZ$, $(x)$ is also a prime ideal.

    And $(2,x)$ (the ideal generated by $2$ and $x$, which is $(2)+(x)$) then we can map elements of $f\in\bZ[x]$ to $[f(0)]\in\bZ_2$.
    Then the kernel of this is are polynomials with even free coefficients, which is $(2)+(x)$.
    Thus $\slfrac{\bZ[x]}{(2,x)}\cong\bZ_2$ by the first isomorphism theorem, so $(2,x)$ is prime.

\end{exam*}

Notice then that we have the following proper chain of prime ideals:
\[ (0) \subset (x) \subset (2,x) \]

\begin{defn*}

    Let $R$ be a ring, and $I$ a proper (left/right/bidirectional) ideal.
    $I$ is \ppemph{maximal} if it is not contained in any other proper (left/right/bidirectional) ideal.

\end{defn*}

We recall Zorn's Lemma:

\begin{lemm*}

    Let $(P,\leq)$ be a non-empty partial-ordered set.
    If every chain in $P$ ($p_1\leq p_2\leq p_3\leq\dots$) has an upper bound (an $M\in P$ such that for every $n$, $p_n\leq M$), then $S$ has a maximal element (an element $s\in P$ such that for every
    $t\neq s$, $s\not\leq t$).

\end{lemm*}

(This is equivalent to the axiom of choice).

\begin{prop*}

    Let $R$ be a ring and $I$ be a proper (left/right/bidirectional) ideal, then there exists a maximal (left/right/bidirectional) ideal $M$ such that $I\subseteq M$.

\end{prop*}

\begin{proof}

    We will prove this for left ideals.
    Let
    \[ P = \set{J\pideal R}[I\subseteq J] \]
    be a partially ordered set under the partial order of inclusion ($\subseteq$).
    Let $J_1\subseteq J_2\subseteq\cdots$ be a chain in $P$, then let
    \[ M = \bigcup_{n=1}^\infty J_n \]
    $M$ is a group (the union of an ascending chain of subgroups is a group) since if $a,b\in M$ then there exists a $J_n$ such that $a,b\in J_n$ (since we can take the maximum between the indexes of the
    ideals where we find $a$ and $b$), so $a+b\in J_n$ and so $a+b\in M$ and so $M$ is closed under addition.
    And if $a\in M$, then $a\in J_n$ for some $n$ and so $-a\in J_n\subseteq M$.

    And $M$ is closed under left multiplication by $R$ since if $a\in M$, there exists some $n$ where $a\in J_n$ so $Ra\subseteq J_n\subseteq M$.
    Now we must also show that $M$ is proper, that is $M\neq R$.
    If $M=R$ then $1\in M$ so there would exist a $J_n$ with $1\in J_n$ which means that $J_n=R$ and so $J_n$ is not proper, which is a contradiction (since $P$ is a set of proper ideals).
    And since $I\subseteq M$, $M\in P$ and $M$ is clearly an upper bound for the chain.

    So every chain in $P$ has an upper bound, and so $P$ has a maximal element $M$.
    This maximal element is clearly a maximal ideal containing $I$ as for any proper ideal $J$, if $M\subseteq J$ then $J\in P$ so $M=J$ since $M$ is maximal in $P$.
    \qed

\end{proof}

\begin{exam*}

    This claim is not true for rngs.
    Let $(G,+)$ be an abelian group and define $g\cdot h=0$, then $(G,+,\cdot)$ is an rng.
    Since any subgroup of $G$ is an ideal (since it contains $0$, so it is closed under multiplication by $G$), and any ideal of $G$ is necessarily a subgroup.
    So we can look for an abelian group $G$ which has no maximal (proper) subgroups.

    We can choose $(\bQ, +)$ as our group.
    Suppose $H<\bQ$ is a maximal proper subgroup.
    Then there exists an $x\notin\bQ$ and $0\neq y\in\bQ$, then let $\frac yx=\frac ab$ for integers $a,b$.
    Then $a\neq0$ and $\frac xa\notin H+\gen{x}$ since if it were then $x=ah+anx=ah+bny\in H$ which contradicts that $x\notin H$.
    So $H\subset H+\gen x\subset\bQ$, so $H$ is not maximal.

\end{exam*}

\begin{thrm*}[coroTheorem,The\ Correspondence\ Theorem]

    Suppose $I$ is some ideal of $R$, let $\mG=\set{J\ideal R}[I\subseteq J]$ and $\mN=\set{\slfrac JI\ideal\slfrac RI}$, then there is an inclusion-preserving bijection between $\mG$ and $\mN$.

\end{thrm*}

\begin{proof}

    We will focus on left ideals.

    We define the mapping
    \[ \phi\colon\mG\longto\mN,\qquad \phi(J) = \slfrac JI \]
    This is well defined since $I$ is an ideal of $J$'s, and if $j+I\in\slfrac JI$ and $r+I\in\slfrac RI$:
    \[ (r+I)(j+I) = rj + I = j' + I \in \slfrac JI \]
    Since $J$ is an ideal of $R$'s.

    Note that since an ideal of $\slfrac RI$ is a subgroup of $\slfrac RI$, it must have the form $\slfrac JI$ for some $I\leq J\leq R$ ($\leq$ meaning subgroup here) by the correspondence theorem for
    groups.
    We now claim that $J$ is an ideal, let $r\in R$ and $j\in J$, since
    \[ (r+I)(j+I) \in \slfrac JI \implies rj \in J \]
    So $J$ is indeed an ideal.
    So we can explicitly find the inverse of $\phi$:
    \[ \phi^{-1}\parens{\slfrac JI} = J \]
    This is well-defined as explained above and obviously the inverse of $\phi$.

    So $\phi$ is a bijection, and it is obviously inclusion-preserving.
    \qed

\end{proof}

Notice that we showed that $J$ is an ideal of $R$ containing $I$ if and only if $\slfrac JI$ is an ideal of $\slfrac RI$.

\begin{prop*}

    Let $R$ be a commutative ring and $I\pideal R$ a proper ideal.
    Then $I$ is maximal if and only if $\slfrac RI$ is a field.

\end{prop*}

\begin{proof}

    Notice that $I$ is maximal if and only if the set of ideals containing $I$ is $\mG=\set{I, R}$ and by \ppref{coroTheorem} this is if and only if the ideals of $\slfrac RI$ are
    $\mN=\set{0, \slfrac RI}$.

    And so we will show that $F$ is a field if and only if it has trivial ideals.
    If $F$ is a field, then if $\set0\neq I$ is an ideal of $F$, then there is a non-zero $x\in I$, since $x^{-1}\in F$ this means $xx^{-1}=1\in I$ so $I=F$.
    And if $F$ is not a field then there exists an $x\in F$ without a multiplicative inverse.
    Then if $1\in(x)$ this means that there exists a $y\in F$ with $yx=1$, and since $R$ is commutative this means $yx=xy=1$ so $y$ is $x$'s multiplicative inverse in contradiction.
    So $1\notin(x)$ so $\set0\neq(x)\neq F$, so if $F$ is not a field there exist non-trivial ideals.

    Thus since $I$ is maximal if and only if $\slfrac RI$ only has trivial ideals, $I$ is maximal if and only if $\slfrac RI$ is a field.
    \qed

\end{proof}

We showed in our proof that a commutative ring is a field if and only if it has trivial ideals, which is important as well.

\begin{coro*}

    If $R$ is a commutative ring then every maximal ideal is prime.

\end{coro*}

\begin{proof}

    We know that $I$ is a maximal ideal if and only if $\slfrac RI$ is a field, which means $\slfrac RI$ is an integral domain, which means that $I$ is a prime ideal.
    \qed

\end{proof}

\begin{defn*}

    We call ideals generated by a single element \ppemph{principal ideals}.
    If $R$ is a ring in which every (left/right/bidirectional) ideal is principal is a \ppemph{principal (left/right/bidirectional) ideal ring} 
    If a principle ideal ring is also an integral domain, it is called a \ppemph{principle ideal domain} (PID).

\end{defn*}

Note that the trivial ideals are principal:
\[ \set 0 = (0),\quad R = (1) \]

\begin{exam*}

    \benum
        \item Since the ideals of a field are trivial, all fields are principal ideal domains.
        \item $\bZ$ is also a principal ideal domain since all of its ideals are of the form $n\bZ=(n)$.
    \eenum

\end{exam*}

\begin{prop*}

    Let $R$ be a principal ideal domain, and let $P$ be a non-zero prime ideal.
    Then $P$ is maximal.

\end{prop*}

\begin{proof}

    We know that there exists a maximal ideal $M$ such that $P\subseteq M$.
    Since $R$ is a principal ideal domain, $P=(p)$ and $M=(m)$, and so $p=rm\in P$.
    Thus $r\in P$ or $m\in P$.
    If $r\in P$ then $r=tp$, and since $p=rm=tpm=ptm$ ($R$ is an integral domain), this means $p(1-tm)=0$.
    And since $R$ is an integral domain, $1-tm=0$ (since $P\neq0$ so $p\neq0$).
    So $tm=1$ and so $1\in(m)=M$ which means $M=R$ which is a contradiction since $M$ is a proper ideal.

    Therefore $m\in P$ and so $M=(m)\subseteq P$ which means $P=M$ so $P$ is maximal.
    \qed

\end{proof}

Notice that during this proof we showed the following:

\begin{prop*}

    Let $R$ be an integral domain and $P\ideal R$ a non-zero prime ideal.
    Then if $M\pideal R$ is a proper principal ideal such that $P\subseteq M$, then $M=P$.

    That is, proper principal ideals do not (properly) contain any non-zero prime ideals.

\end{prop*}

Since we showed $(x)\subset(2,x)\subset\bZ[x]$ in $\bZ[x]$, and $(x)$ and $(2,x)$ are non-zero prime ideals, $\bZ[x]$ is not a principal ideal domain since $(x)$ is not maximal.

And furthermore $(2,x)$ is not a principal ideal since $(x)\subset(2,x)$, so it contains a non-zero prime ideal $(x)$.

\begin{defn*}

    Let $R$ be a commutative ring.
    The \ppemph{dimension} of $R$ is the largest number $d$ such that there exists a proper chain of prime ideals 
    \[ P_0 \subset P_1 \subset \cdots \subset P_{d-1} \subset P_d \]
    (The dimension of $R$ is one less than the length of the chain.)

    If there doesn't exist a largest $d$, then $R$ has infinite dimension.

\end{defn*}

Thus $R$ is a field if and only if $\dim R=0$.
This is because $R$ is a field if and only if it has trivial ideals, and since if $R$ has a non-trivial ideal it has a prime ideal (since maximal ideals are prime), $R$ is a field if and only if its only
prime ideal is $\set 0$, and so the only chain in a field is a chain of length $1$.

\begin{prop*}

    If $R$ is a prime ideal domain, $\dim R\leq1$.

\end{prop*}

\begin{proof}

    Since if $P\ideal R$ is a non-zero prime ideal, $P$ is maximal, the only prime ideal chains we can form are of the form
    \[ \set 0 \subseteq P \]
    Which has length $2$ or $1$ depending on whether $P$ is zero or not.
    So $\dim R$ is $1$ or $0$.
    \qed

\end{proof}

\end{document}


