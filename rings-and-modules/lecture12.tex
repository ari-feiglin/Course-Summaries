\documentclass[10pt]{article}

\usepackage{amsmath, amssymb, mathtools}
\usepackage[margin=1.5cm]{geometry}
\usepackage{mathrsfs}

\catcode`\@=11
\input pdfmsym

\input prettyprint
\input preamble

\pdfmsymsetscalefactor{10}
\initpps

\def\implies{\,\mathrel{\longvarRightarrow}\,}

\def\pmat#1{\begin{pmatrix} #1 \end{pmatrix}}

\def\divides{{\mid}}

\font\bigbf = cmbx12 scaled 2000

\newfunc{eulerg}{{\rm Euler}}({})
\newfunc{order}o({})
\newfunc{ker}{{\rm Ker}}({})
\newfunc{image}{{\rm Im}}({})
\newfunc{lattice}{{\cal L}}(\vert)
\newfunc{powset}{{\cal P}}(\vert)
\newfunc{center}{{\rm Z}}({})
\newfunc{units}{{\cal U}}(\vert)
\newfunc{sign}{{\rm sgn}}({})
\newfunc{aut}{{\rm Aut}}({})
\newfunc{speclinear}{{\rm SL}}({})
\newfunc{genlinear}{{\rm GL}}({})
\newfunc{projlinear}{{\rm PGL}}({})
\newfunc{specprojlinear}{{\rm PSL}}({})
\newfunc{core}{{\rm Core}}(\vert)
\newfunc{inner}{{\rm Inn}}(\vert)
\newfunc{outer}{{\rm Out}}(\vert)
\newfunc{conj}{{\rm conj}}({})
\newfunc{ev}{{\rm ev}}({})
\newfunc{Frac}{{\rm Frac}}({})
\newfunc{Tor}{{\rm Tor}}(\vert)

\def\maxdivs{\mathrel{\|}}

\let\ideal=\trianglelefteq
\let\pideal=\triangleleft
\def\normal{\mathrel\triangleleft}
\let\mmorph=\longvarhookrightarrow
\let\to=\varrightarrow
\let\longto=\longvarrightarrow

\def\qed{\hskip1cm\hbox{}\hfill$\blacksquare$}

\def\mN{\mathcal N}
\def\mG{\mathcal G}

\begin{document}

\c@section=12

\barcolorbox{220, 220, 255}{0, 0, 130}{80, 80, 200}{
    \leftskip=0pt plus 1fill \rightskip=\leftskip
    {\bigbf Introduction to Rings and Modules}

    \medskip
    \textit{Lecture \thesection, Monday June 5 2023}

    \textit{Ari Feiglin}
}

\bigskip

\begin{defn*}

    Let $R$ be a ring, a \ppemph{left $R$-module} is an abelian group $(M,+)$ equipped with scalar multiplication
    \[ \cdot\colon R\times M\longto M \]
    such that the following hold
    \benum
        \item $(r+s)m=rm+sm$ for every $r,s\in R$ and $m\in M$.
        \item $r(m+n)=rm+rn$ for $r\in R$ and $m,n\in M$.
        \item $s(rm)=(sr)m$ for $r,s\in R$ and $m\in M$.
        \item $1_Rm=m$ for $m\in M$.
    \eenum

    A \ppemph{right $R$-module} is an abelian group $(M,+)$ equipped with a right multiplication function $M\times R\to M$ which satisfies the above properties, where the multiplication's order is swapped.

\end{defn*}

Note that if $R$ is commutative then if $M$ is a left module, we can induce on $M$ a right module structure by defining
\[ m\cdot r = r\cdot m \]
This satisfies the first and second properties trivially, and
\[ (mr)s = s(rm) = (sr)m = m(sr) = m(rs) \]
where the final equality is due to $R$ being commutative.
Thus if $R$ is commutative, we can think of left and right modules being equivalent and just saying $R$-modules.

\begin{note}

    If $R$ is a field, a left $R$-module is a vector space above $R$.
    Thus vector spaces are modules (the reverse is not true).

\end{note}

\begin{exam*}

    If $R$ is a ring, let $M=\set{0_M}$ be the trivial group.
    We define $r\cdot0_M=0_M$, and this defines a eft $R$-module, the so-called \ppemph{trivial $R$-module}.

\end{exam*}

\begin{prop*}

    $0_R\cdot m=0_M$ and $r\cdot0_M=0_M$.

\end{prop*}

\begin{proof}

    Note that $0_R\cdot m=(0_R+0_R)m=0_R\cdot m+0_R\cdot m$, since $M$ is a group we can subtract $0_R\cdot m$ from both sides and get $0_R\cdot m=0_M$ as required.
    And $r\cdot0_M=r\cdot(0_M+0_M)=r\cdot0_M+r\cdot0_M$ and subtracting $r\cdot0_M$ we get $r\cdot0_M=0_R$.
    \qed

\end{proof}

\begin{prop*}

    $(-1_R)m=-m$

\end{prop*}

\begin{proof}

    Notice that $(-1_R)m+m=(-1_R+1_R)m$ by distributivity, which equals $0_Rm=0_M$ so $(-1_R)m=-m$ as required.
    \qed

\end{proof}

\begin{exam*}

    \benum
        \item If $R$ is a ring, we define the module $M=(R,+)$ with multiplication $r\cdot m=rm\in R$.
        Thus $R$ is an $R$-module above itself.

        \item If $S$ is a ring and $M$ a module over $S$, and $f\colon R\longto S$ a ring homomorphism.
            We can induce on $R$-module structure on $M$ by
            \[ r\cdot m = f(r)m \]
            This satisfies the axioms since
            \[ (r_1+r_2)m = f(r_1+r_2)m = (f(r_1)+f(r_2))m = f(r_1)m+f(r_2)m = r_1m+r_2m \]
            the second axiom:
            \[ r(m+n) = f(r)(m+n) = f(r)m + f(r)n = rm+rn \]
            the third axiom:
            \[ (r_1r_2)m = f(r_1r_2)m = (f(r_1)f(r_2))m = f(r_1)(f(r_2)m) = f(r_1)(r_2m) = r_1(r_2m) \]
            the fourth axiom:
            \[ 1_Rm = f(1_R)m = 1_Sm = m \]

        \item Let $L$ be a left module over $S$ and $R=M_n(S)$, the ring of matrices of size $n\times n$ over $S$.
            Let $M=L^n$, which is a left $R$-module defined by $[s\ell]_i = \sum_{k=1}^n s_{ik}\ell_k$, where $s\in R$, $\ell\in M$ (meaning $s_{ik}\in S$ and $\ell_k\in L$, so this multiplication is
            well-defined).
    \eenum

\end{exam*}

\begin{defn*}

    If $R$ is a ring and $M$ a $R$-module, then $\varnothing\neq N\subseteq M$ is a \ppemph{submodule} of $M$ if $N$ is closed under addition, and scalar multiplication by $R$.
    Meaning that if $n_1,n_2\in N$ then $n_1+n_2\in N$ and if $r\in R$ and $n\in N$ then $rn\in N$.

\end{defn*}

Notice then that if $N$ is a submodule of $M$, then $N$ is a subgroup of $M$.
This is since $0_M=0_R\cdot n$ for $n\in N$ so $0_M\in N$.
And if $n\in N$ then $-n=(-1_R)n\in N$, so $N$ is closed under inverses.

\begin{prop*}

    The submodules of a ring $R$, when viewed as a module over itself, are exactly its left ideals.

\end{prop*}

\begin{proof}

    If $I\subseteq R$ is a left-ideal of $R$ then it is by definition closed under addition and left multiplication by $R$, so it is a submodule.
    And if $N\subseteq R$ then it is by definition closed under addition and left scalar multiplication, so is by definition a left ideal of $R$.
    \qed

\end{proof}

\begin{prop*}

    Let $M$ be an $R$-module, and $m_1,\dots,m_n\in M$.
    Then the smallest submodule containing these elements is
    \[ N = \set{r_1m_1+\cdots+r_nm_n}[r_i\in R] \]

\end{prop*}

\begin{proof}

    This set is a submodule since if $r_1m_1+\cdots+r_nm_n,s_1m_1+\cdots+s_nm_n\in N$ then
    \[ r_1m_1+\cdots+r_nm_n + s_1m_1+\cdots+s_nm_n = (r_1+s_1)m_1+\cdots+(r_ns_n)m_n \in N \]
    so $N$ is closed under addition, and if $r\in R$ then
    \[ r(r_1m_1+\cdots+r_nm_n) = (rr_1)m_1+\cdots+(rr_n)m_n \in N \]
    so $N$ is also closed under left scalar multiplication, meaning $N$ is a submodule.

    If $N'$ is another submodule containing $m_1,\dots,m_n$ then for any $r_1,\dots,r_n\in R$, it must contain $r_im_i$ for every $i$ since it is closed under scalar multiplication, and since it is also
    closed under addition it must contain $r_1m_1+\cdots+r_nm_n$, meaning $N\subseteq N'$.
    \qed

\end{proof}

\begin{defn*}

    If $M$ is an $R$-module, and $m_1,\dots,m_n\in M$ we define the \ppemph{submodule generated by $m_1,\dots,m_n$} to be
    \[ \gen{m_1,\dots,m_n} = \set{r_1m_1+\cdots+r_nm_n}[r_i\in R] \]
    the smallest submodule containing $m_1,\dots,m_n$.

    And in general if $\mathscr{S}\subseteq M$, we define the \ppemph{submodule generated by $\mathscr S$} to be
    \[ \gen{\mathscr S} = \set{r_1s_1+\cdots+r_ks_k}[k\in\bN, r_i\in R, s_i\in\mathscr S] \]
    This is the smallest submodule containing $\mathscr S$.

\end{defn*}

\begin{defn*}

    Let $R$ be an integral domain and $M$ an $R$-module.
    We define its \ppemph{torsion submodule} by
    \[ \Torof M = \set{m\in M}[\exists0_R\neq r\in R\colon rm=0_M] \]

\end{defn*}

This is indeed a submodule, since if $m_1,m_2\in\Torof M$ then there exists $r_1$ and $r_2$ such that $r_1m_1=r_2m_2=0_M$.
Since $R$ is an integral domain, $r_1r_2\neq0_R$ and
\[ (r_1r_2)(m_1+m_2) = r_1r_2m_1 + r_1r_2m_2 = r_2(r_1m_1) + r_1(r_2m_2) = 0_M \]
so $m_1+m_2\in\Torof M$, and if $m\in\Torof M$ where $rm=0_M$, and $s\in R$ then
\[ r(sm) = s(rm) = 0_M \]
so $sm\in\Torof M$ as well.

\begin{defn*}

    Let $M$ be an $R$-module.
    $B\subseteq M$ is called a \ppemph{basis} of $M$ if every element of $M$ can be written as a unique linear combination of elements in $B$.
    Meaning that for every $0_M\neq m\in M$, there exist distinct $b_i\in B$ and $r_i\in R$ such that
    \[ m = r_1b_1 + \cdots + r_nb_n \]
    and if
    \[ m = r'_1b'_1 + \cdots + r'_mb'_m \]
    then $n=m$ and there exists a permutation $\sigma\in S_n$ such that $b_{\sigma(i)}=b'_i$ and $r_{\sigma(i)}=r'_i$.

    If $M$ has a basis, it is called \ppemph{free}.

\end{defn*}

From linear algebra, we know that

\begin{thrm*}

    Let $R$ be a field, then every $R$-module is free.

\end{thrm*}

\begin{exam*}

    If $M$ is an abelian group, there is a unique way to define $M$ as a $\bZ$-module.
    This is because for $n\geq0$
    \[ n\cdot m = (1+\cdots+1)m = m+\cdots+m \]
    and
    \[ (-n)\cdot m = (-m)+\cdots+(-m) \]

    This does in fact define a $\bZ$-module.
    Thus abelian groups and $\bZ$-modules are equivalent.

\end{exam*}

\begin{exam*}

    Let $M=\slfrac\bZ{6\bZ}$, this is a $\bZ$-module.
    Now suppose $B\subseteq M$ is a basis, then let $m\in M$ so
    \[ m = r_1b_1 + \cdots + r_nb_n \]
    but we know $(r+6)b = rb+6b$ and $6b=0$ so $(r+6)b=rb$ and so
    \[ m = (r_1+6)b_1 + \cdots + r_nb_n \]
    is another linear combination equal to $m$, so these are not unique and therefore $B$ is not a basis.

    So $M$ is not free.
    This is true in general for $M=\slfrac\bZ{n\bZ}$.
    And in even more generality, this works for finite (non-trivial) abelian groups $M$, since $\abs M\cdot m=0_M$.

\end{exam*}

\end{document}

