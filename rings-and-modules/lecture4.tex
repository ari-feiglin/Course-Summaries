\documentclass[10pt]{article}

\usepackage{amsmath, amssymb, mathtools}
\usepackage[margin=1.5cm]{geometry}

\usepackage{tikz-cd}

\catcode`\@=11
\input pdfmsym

\input prettyprint
\input preamble

\pdfmsymsetscalefactor{10}
\initpps

\def\implies{\,\mathrel{\longvarRightarrow}\,}

\def\pmat#1{\begin{pmatrix} #1 \end{pmatrix}}

\let\divides=\mid

\font\bigbf = cmbx12 scaled 2000

\newfunc{eulerg}{{\rm Euler}}({})
\newfunc{order}o({})
\newfunc{ker}{{\rm Ker}}({})
\newfunc{image}{{\rm Im}}({})
\newfunc{lattice}{{\cal L}}(\vert)
\newfunc{powset}{{\cal P}}(\vert)
\newfunc{center}{{\rm Z}}({})
\newfunc{units}{{\cal U}}(\vert)
\newfunc{sign}{{\rm sgn}}({})
\newfunc{aut}{{\rm Aut}}({})
\newfunc{speclinear}{{\rm SL}}({})
\newfunc{genlinear}{{\rm GL}}({})
\newfunc{projlinear}{{\rm PGL}}({})
\newfunc{specprojlinear}{{\rm PSL}}({})
\newfunc{core}{{\rm Core}}(\vert)
\newfunc{inner}{{\rm Inn}}(\vert)
\newfunc{outer}{{\rm Out}}(\vert)
\newfunc{conj}{{\rm conj}}({})
\newfunc{ev}{{\rm ev}}({})

\def\maxdivs{\mathrel{\|}}

\let\ideal=\trianglelefteq
\def\normal{\mathrel\triangleleft}
\let\mmorph=\longvarhookrightarrow
\let\to=\varrightarrow
\let\longto=\longvarrightarrow

\def\qed{\hskip1cm\hbox{}\hfill$\blacksquare$}

\begin{document}

\c@section=4

\barcolorbox{220, 220, 255}{0, 0, 130}{80, 80, 200}{
    \leftskip=0pt plus 1fill \rightskip=\leftskip
    {\bigbf Introduction to Rings and Modules}

    \medskip
    \textit{Lecture \thesection, Wednesday April 19 2023}

    \textit{Ari Feiglin}
}

\bigskip

\subsection{The First Isomorphism Theorem}

\begin{thrm*}[firstIsoThrm,The\ First\ Isomorphism\ Theorem]

    Suppose $f\colon R\longvarrightarrow S$ is a ring homomorphism, then there is a natural isomorphism
    \[ \slfrac R{\kerof f}\cong\imageof f \]
    (in other words, $\slfrac R{\kerof f}$ and $\imageof f$ are ring-isomorphic.)

\end{thrm*}

\begin{proof}

    Let us take the group isomorphism between from $\imageof f$ to $\slfrac R{\kerof f}$ (since they are also groups under their respective addition operations) $\phi(b)=f^{-1}\set b$.
    Recall that if $f(a)=b$ then $\phi(b)=a+\kerof f$.
    Now all we must show is that $\phi$ respects multiplication:
    \benum
        \item $\phi(1_S)=f^{-1}\set{1_S}=\set{a\in R}[f(a)=1_S]=1_R+\kerof f$ which is the identity of the quotient group $\slfrac R{\kerof f}$.
        \item Now suppose $f(\alpha)=a$ and $f(\beta)=b$ then we must show that $\phi(ab)=\phi(a)\phi(b)$, now since $\phi(a)\phi(b)=(\alpha+\kerof f)(\beta+\kerof f)=\alpha\beta+\kerof f$.
        So we must show that this is equal to $\phi(ab)$, which is equivalent to $f(\alpha\beta)=ab$, which is true since $f$ is a ring homomorphism.
    \eenum
    \qed

\end{proof}

\begin{exam*}

    Let us define $f\colon\bZ\longto\bZ_n$ by $f(m)=[m]$, which we know is well-defined from group theory.
    $f$ is obviously surjective and $\kerof f=n\bZ$ so
    \[ Z_n=f(Z)\cong\slfrac\bZ{\kerof f}=\slfrac\bZ{n\bZ} \]

    So this classic equivalence is true for rings as well.

\end{exam*}

\begin{defn*}

    Recall that $Ra$ is the smallest left ideal containing $a$ and $aR$ is the smallest right ideal containing $a$.
    When $R$ is commutative, $aR=Ra$ is the smallest (bidirectional) ideal containing $a$ and is denoted $\parens a$.
    This is called the \ppemph{ideal generated by $a$}.

\end{defn*}

\begin{prop*}

    If $R$ is a ring and $f(x),g(x)\in R[x]$ such that the leading coefficient in $f(x)$ is $1$ then there exists unique $q(x),r(x)\in R[x]$ such that $g(x)=q(x)\cdot f(x)+r(x)$ and the degree
    of $r(x)$ is less than that of $f(x)$.

\end{prop*}

\begin{proof}

    If the degree of $g$'s is less than that of $f$'s then we can take $r(x)=g(x)$ and $q(x)=0$.
    Otherwise suppose
    \[ f(x) = \sum_{k=0}^n a_kx^k,\quad g(x) = \sum_{k=0}^m b_kx^k \]
    where $a_n=1$ and $n\leq m$.
    Then we have that $h(x) = g(x) - b_mx^{m-n}f(x)$ has degree $\leq m$ so proceeding inductively on $m$ (the base case of $m=0$ is trivial, as $g(x)=b$ and $f(x)=1$ so $q(x)=b$ and $r(x)=0$ satisfy)
    we have that $h(x)=q'(x)f(x)+r(x)$ where $r(x)$ has degree less than $n$.
    So
    \[ g(x) = \bigl(b_mx^{m-n}+q'(x)\bigr)f(x) + r(x) \]
    as required.

    If $q(x)f(x)+r(x) = q'(x)f(x)+r'(x)$ then $\bigl(q(x)-q'(x)\bigr)f(x) + \bigl(r(x)-r'(x)\bigr) = 0$ and since the degree of $r-r'$ is less than that of $f$'s we must have that $q-q'=0$ and so $r=r'$ as
    well so the decomposition is unique.

    \qed

\end{proof}

\begin{exam*}

    We claim that for a commutative ring $R$ and any $a\in R$ we have
    \[ \slfrac{R[x]}{(x-a)}\cong R \]

    If $a=0$ then by definition $(x)=\set{x\cdot f(x)}[{f(x)\in R[x]}]$ which is the set of all polynomials without a free coefficient.
    Notice then that $f(x)-g(x)\in (x)$ if and only if they have the same free coefficient, so the quotient group intuitively should be the set of free coefficients, $R$.

    To do this in general, we can look at the ring homomorphism $\ev_a\colon R[x]\longto R$ which is a homomorphism since $R$ is commutative and whose kernel is
    \[ \kerof{\ev_a} = \set{f\in R[x]}[f(a)=0] \]
    we claim that $\kerof{\ev_a}=(x-a)$.
    Suppose $f(x)\in(x-a)$ then $f(x)=(x-a)g(x)$ for $g(x)\in R[x]$ then $f(a)=0$ so $f\in\kerof{\ev_a}$.
    And if $f\in\kerof{\ev_a}$ then we can divide $f$ by $x-a$ due to our proposition above to get
    \[ f(x) = q(x)(x-a) + r \]
    So $0=r$ by plugging in $x=a$ ($r$ is a scalar since the degree of $f$ is $1$) so we have that $f(x)=q(x)(x-a)$ and so $f(x)\in(x-a)$ as required.

    So we have that
    \[ \slfrac{R[x]}{(x-a)} = \slfrac{R[x]}{\kerof{\ev_a}} \cong \evof[a]{R[x]} = R \]
    as required.

\end{exam*}

\begin{prop*}

    \[ \slfrac{\bR[x]}{(x^2+1)} \cong \bC \]

\end{prop*}

\begin{proof}

    We define $\phi\colon\bR[x]\longto\bC$ where we evaluate the input polynomial at $i$, in other words $\phi=\ev_i\circ\iota$ where $\iota\colon\bR\longto\bC$ is the inclusion homomorphism $\iota(x)=x$.
    $\iota$ is trivially a homomorphism and the composition of homomorphisms is a homomorphism.
    $\phi$ is surjective since $\phi(a+bx)=a+bi$.
    Now we must prove $\kerof\phi = (x^2+1)$.
    Suppose $f(x)\in(x^1+1)$ so $f(x)=g(x)(x^1+1)$ so $\phi(f)=g(i)\cdot0=0$ so $(x^1+1)\subseteq\kerof\phi$.
    If $f(x)\in\kerof\phi$ then by above $f(x)$ can be written as
    \[ f(x) = q(x)(x^2+1) + ax + b \]
    Since $f(i)=0$ we must have that $ai+b=0$ which means $a=b=0$ since they are real so we have that $f(x)=q(x)(x^1+1)$ so $f(x)\in(x^2+1)$.
    So we have that
    \[ \slfrac{\bR[x]}{(x^2+1)} = \slfrac{\bR[x]}{\kerof\phi} \cong \phi(\bR[x]) = \bC \eqno\blacksquare \]

\end{proof}

\begin{defn*}

    Similar to before, if $R$ is commutative and $I$ and $J$ are ideals we define
    \[ IJ = \set{\sum_{n=1}^N i_nj_n}[i_n\in I, j_n\in J] \]

\end{defn*}

This is obviously an ideal.
We could generalize this and require $I$ be a left ideal and $J$ be a right ideal.
Similarly $I+J=\set{i+j}[i\in I, j\in J]$ is a (left or right) ideal if $I$ and $J$ are (left or right; but the same direction) ideals.

\begin{defn*}

    Let $R$ be a ring and $I$ and $J$ be (left or right) ideals.
    If $I+J=R$ then $I$ and $J$ are called \ppemph{comaximal ideals}.

\end{defn*}

\begin{thrm*}[chineseRemainderTheorem,The\ Chinese\ Remainder\ Theorem]

    Let $R$ be a commutative ring and $I,J\ideal R$ be comaximal ideals.
    Then
    \[ \slfrac R{IJ}\cong\slfrac RI\times\slfrac RJ \]

\end{thrm*}

\begin{proof}

    We will focus on the homomorphism:
    \[ f\colon R\longto\slfrac RI\times\slfrac RJ,\quad f(a) = (a+I, a+J) \]
    in order to use the first isomorphism theorem, we must show that $f$ is surjective and its kernel is $IJ$.

    Since $R=I+J$ (they are comaximal), $1_R\in I+J$ so $1_R=i+j$ so
    \[ f(j) = (j+I, j+J) = (1_R + I, J) = \bigl(1_{\slfrac RI}, 0_{\slfrac RJ}\bigr) \]
    and similarly
    \[ f(i) = \bigl(0_{\slfrac RI}, 1_{\slfrac RJ}\bigr) \]
    so let $(a+I, b+J)\in\slfrac RI\times\slfrac RJ$ then
    \[ f(aj + bi) f(a)f(j) + f(b)f(i) = (a+I, a+J)(1, 0) + (b+I, b+J)(0, 1) = (a+I, b+J) \]
    so $f$ is indeed surjective.

    Now $a\in\kerof f$ if and only if
    \[ (a+I, a+J) = (I, J) \]
    which is only if $a\in I\cap J$, so $\kerof f=I\cap J$.
    Now since for $i_n\in I$ and $j_n\in J$, $i_nj_n\in I\cap J$ since $I$ and $J$ are ideals so $IJ\subseteq I\cap J$.
    And if $a\in I\cap J$ then $a=1_R\cdot a=(i+j)a=ia+ja$ since $a\in J$, $ia\in IJ$ and $a\in I$ so $ja\in IJ$ so $a=ia+ja\in IJ$.
    Thus $IJ=I\cap J$.o
    So $\kerof f=IJ$ as required.

    Thus
    \[ \slfrac R{IJ} = \slfrac R{\kerof f} \cong f(R) = \slfrac RI\times\slfrac RJ \eqno\blacksquare \]

\end{proof}

\end{document}

