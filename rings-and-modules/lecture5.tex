\documentclass[10pt]{article}

\usepackage{amsmath, amssymb, mathtools}
\usepackage[margin=1.5cm]{geometry}

\catcode`\@=11
\input pdfmsym

\input prettyprint
\input preamble

\pdfmsymsetscalefactor{10}
\initpps

\def\implies{\,\mathrel{\longvarRightarrow}\,}

\def\pmat#1{\begin{pmatrix} #1 \end{pmatrix}}

\let\divides=\mid

\font\bigbf = cmbx12 scaled 2000

\newfunc{eulerg}{{\rm Euler}}({})
\newfunc{order}o({})
\newfunc{ker}{{\rm Ker}}({})
\newfunc{image}{{\rm Im}}({})
\newfunc{lattice}{{\cal L}}(\vert)
\newfunc{powset}{{\cal P}}(\vert)
\newfunc{center}{{\rm Z}}({})
\newfunc{units}{{\cal U}}(\vert)
\newfunc{sign}{{\rm sgn}}({})
\newfunc{aut}{{\rm Aut}}({})
\newfunc{speclinear}{{\rm SL}}({})
\newfunc{genlinear}{{\rm GL}}({})
\newfunc{projlinear}{{\rm PGL}}({})
\newfunc{specprojlinear}{{\rm PSL}}({})
\newfunc{core}{{\rm Core}}(\vert)
\newfunc{inner}{{\rm Inn}}(\vert)
\newfunc{outer}{{\rm Out}}(\vert)
\newfunc{conj}{{\rm conj}}({})
\newfunc{ev}{{\rm ev}}({})

\def\maxdivs{\mathrel{\|}}

\let\ideal=\trianglelefteq
\def\normal{\mathrel\triangleleft}
\let\mmorph=\longvarhookrightarrow
\let\to=\varrightarrow
\let\longto=\longvarrightarrow

\def\qed{\hskip1cm\hbox{}\hfill$\blacksquare$}

\begin{document}

\c@section=5

\barcolorbox{220, 220, 255}{0, 0, 130}{80, 80, 200}{
    \leftskip=0pt plus 1fill \rightskip=\leftskip
    {\bigbf Introduction to Rings and Modules}

    \medskip
    \textit{Lecture \thesection, Monday April 24 2023}

    \textit{Ari Feiglin}
}

\bigskip

\begin{exam*}

    Suppose that $R$ is commutative and $I=(a)$ and $J=(b)$ are two ideals, then $IJ=(ab)$.
    This is because $a\in I$ and $b\in J$ so $ab\in IJ$ and thus $(ab)\subseteq IJ$ (since $(ab)=abR$).
    And every element in $IJ$ is of the form $\sum_{k=1}^n i_kj_k$ where $i_k\in I$ and $j_k\in J$ so $i_k=ar_n$ and $j_k=bs_n$.
    Thus the element is
    \[ \sum_{k=1}^n ar_nbs_n = ab\sum_{k=1}^n r_ns_n \in abR = (ab) \]
    so $IJ\subseteq(ab)$ and thus $IJ=(ab)$ as required.

\end{exam*}

Notice that if $R=\bR[x]$ and $I=(x-1)$ and $J=(x+1)$ then by above $IJ=(x^2-1)$ so by the chinese remainder theorem:
\[ \slfrac{\bR[x]}{(x^2-1)} \cong \slfrac{\bR[x]}{(x-1)} \times \slfrac{\bR[x]}{(x+1)} \cong \bR\times\bR \]
where the last equality is true since we showed that $\slfrac{\bR[x]}{(x-a)}\cong\bR$.
So notice that while $\slfrac{\bR[x]}{(x^2+1)}\cong\bC$ is a field, $\slfrac{\bR[x]}{(x^2-1)}\cong\bR^2$ is not a field since there are zero divisors: $(1,0)\cdot(0,1)=0$.

The chinese remainder theorem has its limits, specifically when the ideals are not comaximal.
For example $\slfrac{\bR[x]}{(x^2)}$ since $(x^2)=(x)(x)$ and $(x)$ is not comaximal with itself (in general groups $G$ satisfy $G\cdot G=G$, and this holds with ideals $I+I=I$).
But we can take $y=x+(x^2)\in\slfrac{\bR[x]}{(x^2)}$ and so $y^2=x^2+(x^2)=(x^2)=0_R$ ($y$ is nilpotent).
Notice that $\slfrac{\bR[x]}{(x^2-1)}$ has no such non-trivial element since $(x,y)^2=0\implies(x,y)=0$, and therefore these two rings cannot be isomorphic.

So these three rings are all non-isomorphic.

\begin{prop*}

    Suppose $f\colon R\longto S$ is a ring homomorphism, then
    \benum
        \item If $R'\subseteq R$ is a subring or subrng of $R$ then $f(R')$ is a subring or subrng of $S$.
        \item If $J\subseteq S$ is a (left, right, or bidirectional) ideal of $S$ then $f^{-1}(J)$ is a (left, right, or bidirectional) ideal of $R$.
        \item If $I\subseteq R$ is a (left, right, or bidirectional) ideal of $R$ then $f(I)$ is a (left, right, or bidirectional) ideal of $f(R)$.
    \eenum

\end{prop*}

\begin{proof}

    Notice that in all cases, $f(X)$ or $f^{-1}(X)$ are abelian groups as the image or preimage of an abelian group.
    So for these proofs all we need to show is the multiplicative nature of whatever we're investigating.
    \benum
        \item Suppose $f(x),f(y)\in f(R')$ then $f(x)f(y)=f(xy)\in R'$ as required.
        And if $f$ is a ring homomorphism and $R'$ is a subring then $1_R\in R'$ so $f(1_R)=1_S\in f(R')$ as required.
        \item We will assume that $J$ is a right ideal, the proofs for the other cases are identical.
        Suppose $x\in f^{-1}(J)$ then we must show for any $r\in R$, $xr\in f^{-1}(J)$.
        This is if and only if $f(xr)=f(x)f(r)$, since $f(x)\in J$ and $f(r)\in S$, $f(x)f(r)\in J$ so $f(xr)\in J$ and so $xr\in f^{-1}(J)$ as required.
        \item Again we suppose $I$ is a right ideal.
        Suppose $f(x)\in f(I)$ and $f(r)\in f(R)$ then $f(x)f(r)=f(xr)$, and since $xr\in I$, $f(x)f(r)\in f(I)$ as required.
        \qed
    \eenum

\end{proof}

Notice that since $\set0$ is an ideal of $S$, we get that $\ker f=f^{-1}\set0\ideal R$ from the above proposition.

\begin{note}

    If $\bF$ is a field and $I\ideal\bF$ is an ideal, if $I\neq(0)$ then there exists an $a\in I$ such that $a\neq0$ so $a^{-1}\cdot a=1\in I$ and so $I=\bF$.
    Thus the ideals of fields are all trivial.

\end{note}

\begin{thrm*}

    Suppose $R$ is a ring and $I\ideal R$ is a bidirectional ideal, then there is a pairing between right/left/bidirectional ideals of $\slfrac RI$ and right/left/bidirectional ideals of $R$ which contain
    $I$.

\end{thrm*}

\end{document}

