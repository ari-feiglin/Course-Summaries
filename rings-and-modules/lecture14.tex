\documentclass[10pt]{article}

\usepackage{amsmath, amssymb, mathtools}
\usepackage[margin=1.5cm]{geometry}
\usepackage{mathrsfs}

\catcode`\@=11
\input pdfmsym

\input prettyprint
\input preamble

\pdfmsymsetscalefactor{10}
\initpps

\def\implies{\,\mathrel{\longvarRightarrow}\,}

\def\pmat#1{\begin{pmatrix} #1 \end{pmatrix}}

\def\divides{{\mid}}

\font\bigbf = cmbx12 scaled 2000

\newfunc{eulerg}{{\rm Euler}}({})
\newfunc{order}o({})
\newfunc{ker}{{\rm Ker}}({})
\newfunc{image}{{\rm Im}}({})
\newfunc{lattice}{{\cal L}}(\vert)
\newfunc{powset}{{\cal P}}(\vert)
\newfunc{center}{{\rm Z}}({})
\newfunc{units}{{\cal U}}(\vert)
\newfunc{sign}{{\rm sgn}}({})
\newfunc{aut}{{\rm Aut}}({})
\newfunc{speclinear}{{\rm SL}}({})
\newfunc{genlinear}{{\rm GL}}({})
\newfunc{projlinear}{{\rm PGL}}({})
\newfunc{specprojlinear}{{\rm PSL}}({})
\newfunc{core}{{\rm Core}}(\vert)
\newfunc{inner}{{\rm Inn}}(\vert)
\newfunc{outer}{{\rm Out}}(\vert)
\newfunc{conj}{{\rm conj}}({})
\newfunc{ev}{{\rm ev}}({})
\newfunc{Frac}{{\rm Frac}}({})
\newfunc{Tor}{{\rm Tor}}(\vert)
\newfunc{Ann}{{\rm Ann}}(\vert)

\def\maxdivs{\mathrel{\|}}

\let\ideal=\trianglelefteq
\let\pideal=\triangleleft
\def\normal{\mathrel\triangleleft}
\let\mmorph=\longvarhookrightarrow
\let\to=\varrightarrow
\let\longto=\longvarrightarrow

\def\qed{\hskip1cm\hbox{}\hfill$\blacksquare$}

\def\mN{\mathcal N}
\def\mG{\mathcal G}
\def\sS{\mathscr S}

\begin{document}

\c@section=14

\barcolorbox{220, 220, 255}{0, 0, 130}{80, 80, 200}{
    \leftskip=0pt plus 1fill \rightskip=\leftskip
    {\bigbf Introduction to Rings and Modules}

    \medskip
    \textit{Lecture \thesection, Friday June 9 2023}

    \textit{Ari Feiglin}
}

\bigskip

\begin{thrm*}[hilbertsBasisTheorem,Hilbert's\ Basis\ Theorem]

    If $R$ is a left (right) noetherian ring, then $R[x]$ is a left (right) noetherian ring as well.

\end{thrm*}

It will be shown in recitation that if $R$ is a UFD then so is $R[x]$.

\begin{proof}

    Recall that $R$ is left noetherian if and only if every left ideal is finitely generated.
    Let $I$ be a left ideal of $R[x]$.
    For every $n\geq0$ let us define
    \[ I_n = \set{a\in R}[\exists b_0,\dots,b_{n-1}\in R\colon ax^n+b_{n-1}x^{n-1}+\cdots+b_0\in I] \]
    or in other words, $I_n$ is the set of all leading coefficients on polynomials of degree $\leq n$ in $I$ (if $a\neq0$ the degree is $n$).
    We claim that $I_n$ is a left ideal of $R$.
    If $a\in I_n$ and $a'\in I_n$ then suppose $ax^n+b_{n-1}x^{n-1}+\cdots+b_0,a'x^n+b'_{n-1}x^{n-1}+\cdots+b'_0\in I$ and so their difference is in $I$, and since the leading coefficient of their difference
    is $a-a'$, we have that $a-a'\in I_n$.
    So $I_n$ is a group under addition.
    If $a\in I_n$ and $b\in R$ then there exists $ax^n+b_{n-1}x^{n-1}+\cdots+b_0\in I$ and thus $bax^n+bb_{n-1}x^{n-1}+\cdots+b_0\in I$ since $b\in R[x]$ and $I$ is a left ideal, thus $ab\in I_n$.
    So $I_n$ is closed under left multiplication of $R$.
    Thus $I_n$ is an ideal of $R$.

    Notice that if $a\in I_n$ then we can multiply the polynomial in $I$ whose leading coefficient is $a$ by $x$ to get a polynomial of degree $n+1$ in $I$, whose leading coefficient is also $a$.
    Thus $a\in I_{n+1}$ and in particular $I_n$ is an ascending chain of ideals of $R$.
    Thus since $R$ is noetherian, at some point $I_n=I_{n+1}$ for every $n\geq N$.
    Furthermore since $R$ is noetherian, $I_n$ is finitely generated for every $n$.
    For every $n\leq N$ suppose $I_n$ is generated by $a_{n,1},\dots,a_{n,t_n}$, and for every $a_{n,k}$ we will choose the polynomial
    \[ f_{n,k} = a_{n,k}x^n + b_{n,k,n-1}x^{n-1} + \cdots + b_{n,k,0} \in I \]
    We know claim that $\set{f_{n,k}}[n\leq N,\, k\leq t_n]$ generates $I$.
    Since this set is finite, if we prove this then we have shown that $I$ is finitely generated and therefore $R$ is noetherian.

    Let $g\in I$ then
    \[ g = c_mx^m + \cdots + c_0 \]
    Suppose $g$ has degree $0$, meaning $g$ is constant: $g=c_0$, thus $c_0\in I_0$.
    We claim that $g$ is a linear combination of $f_{0,1},\dots,f_{0,t_0}$.
    We know that since $f_{n,k}\in I_n$ is of degree $\leq n$, thus $f_{0,i}$ has degree $0$ and is therefore constant, in other words $f_{0,i}=a_{0,i}$.
    Since $c_0\in I_0$ and $a_{0,i}$ generate $I_0$, $c_0$ is a linear combination of $a_{0,i}$s, and since $g=c_0$ and $f_{0,i}=a_{0,i}$, $g$ is a linear combination of $f_{0,i}$s as required.

    We make one final subclaim:
    If $g\in I$ then $g$ is a linear combination of $f_{n,k}$ (with coefficients in $R[x]$).
    We do this inductively on $m=\deg g$.
    For $m=0$, this is simply what we proved above.
    If $1\leq m\leq N$ then $c_m\in I_m$ and so
    \[ c_m = r_{m,1}a_{m,1} + \cdots + r_{m,t_m}a_{m,t_m} \]
    for $r_{m,i}\in R$.
    Since $f_{m,i}$ are polynomials in $I\subseteq R[x]$ with leading coefficients of $a_{m,i}$, the polynomial
    \[ g' = r_{m,1}f_{m,1} + \cdots + r_{m,t_m}f_{m,t_m} \]
    has a leading coefficient of $c_m$, a degree of $m$, and is a linear combination of elements of $I$ and thus $g'\in I$.
    So defining
    \[ h = g - g' \]
    gives a polynomial of degree strictly less than $m$ and is in $I$.
    Thus by induction, $h$ is a linear combination of $f_{k,i}$s and since $g'$ is a linear combination of $f_{m,i}$s, we have $g=h+g'$ is a linear combination of $f_{k,i}$s as required.

    If $m>N$, then $c_m\in I_m=J_N$.
    So we have that
    \[ c_m = r_{N,1}a_{N,1} + \cdots + r_{N,t_N}a_{N,t_N} \]
    and so
    \[ g' = r_{N,1}f_{N,1} + \cdots + r_{N,t_n}f_{N,t_N}\in I \]
    and has a leading coefficient of $c_m$, but a degree of $N$.
    Since $m>N$ we can define
    \[ g'' = x^{m-N}g' = r_{N,1}x^{m-N}f_{N,1} + \cdots + r_{N,t_n}x^{m-N}f_{N,t_N} \]
    which has a leading coefficient of $c_m$ and is of degree $N$, since $g'\in I$ we have $g''\in I$.
    We continue as before and define
    \[ h = g - g'' \]
    which has degree strictly less than $m$, and inductively is a linear combination of $f_{k,i}$s and so $g=h+g''$ is a linear combination of $f_{k,i}$s, as required.

    This proves that $I$ is finitely generated, and thus $R[x]$ is noetherian, as required.
    \qed


\end{proof}

Notice that if $R$ is noetherian, then $R[x_1]$ is noetherian, and so $R[x_1,x_2]=(R[x_1])[x_2]$ is noetherian, and so on for $R[x_1,\dots,x_n]$.

Notice that if $F$ is a field, then the only ideals of $F$ are itself and $(0)$, so it is obviously noetherian.
Thus by the above theorem, $F[x_1,\dots,x_n]$ is noetherian as well.

Suppose we define $V\subseteq F^n$ as follows: let $\set{f_\lambda}_{\lambda\in\Lambda}$ be a set of polynomials in $F[x_1,\dots,x_n]$ then we define
\[ V = \set{(a_1,\dots,a_n)\in F^n}[\forall\lambda\in\Lambda\colon f_\lambda(a_1,\dots,a_n)=0] \]
and let
\[ I(V) = \set{f\in F[x_1,\dots,x_n]}[\forall(a_1,\dots,a_n)\in V\colon f(a_1,\dots,a_n)=0] \]
This is an ideal of $F[x_1,\dots,x_n]$ since if you multiply $f\in I(V)$ by some other polynomial, it will still be zero when $f$ is.
Since $F$ is noetherian, $F[x_1,\dots,x_n]$ is as well and so $I(V)$ is finitely generated, suppose by $g_1,\dots,g_k$.
Thus
\[ V = \set{\vec a=(a_1,\dots,a_n)\in F^n}[g_1(\vec a)=\cdots=g_k(\vec a)=0] \]
since $\vec a\in V$ if and only if $f(\vec a)=0$ for all $f\in I(V)$, since if $\vec a\in V$ then for every $f\in I(V)$ by definition $f(\vec a)=0$.
And if $f(\vec a)=0$ for all $f\in I(V)$ then since $f_\lambda\in I(V)$ for all $\lambda$, we have $f_\lambda(\vec a)=0$ and so $\vec a=0$.
Since $I(V)=(g_1,\dots,g_k)$, $f(\vec a)=0$ for $f\in I(V)$ if and only if $g_i(\vec a)=0$ for all $i$.

So every such $V$ can be defined by finitely many polynomials (since recall that $\Lambda$ may not be finite).

\end{document}

