\documentclass[10pt]{article}

\usepackage{amsmath, amssymb, mathtools}
\usepackage[margin=1.5cm]{geometry}

\input pdfmsym
\input prettyprint
\input preamble

\pdfmsymsetscalefactor{10}
\initpps

\def\pmat#1{\begin{pmatrix} #1 \end{pmatrix}}

\let\divides=\mid
\newfunc{metric}\rho({})
\newfunc{metricc}\sigma({})
\newfunc{spa}{{\rm span}}(\vert)
\newfunc{diam}{{\rm diam}}(\vert)
\newfunc{proj}\pi({})
\newfunc{iproj}{\pi^{-1}}({})
\newfunc{cis}{{\rm cis}}({})
\newfunc{Re}{{\rm Re}}({})
\newfunc{Im}{{\rm Im}}({})
\newfunc{sup}{{\rm sup}}\{\vert\}
\newfunc{Res}{{\rm Res}}({})
\newfunc{wind}n({})
\newfunc{pv}{{\rm pv}}({})

\font\bigbf = cmbx12 scaled 2000
\@undervecc@def{underbar}\@linecap\@linecap

\def\mO{{\cal O}}
\def\mU{{\cal U}}
\let\lineseg=\overleftrightvecc
\let\to=\varrightarrow
\let\longto=\longvarrightarrow
\let\ds=\displaystyle

\def\pdv#1#2{\frac{\partial #1}{\partial #2}}

\def\differ#1#2{\left.d#1\strut\right|_{#2}}

\def\qed{%
    \ifmmode%
        \eqno\blacksquare%
    \else%
        \hskip1cm\allowbreak\hbox{}\nobreak\hfill$\blacksquare$%
    \fi%
}

\def\@ppmathcount{\thesection.\thepp@mathcount}

\begin{document}

\c@section=11

\barcolorbox{220, 255, 220}{0, 130, 0}{80, 200, 80}{
    \leftskip=0pt plus 1fill \rightskip=\leftskip
    {\bigbf Complex Functions}

    \medskip
    \textit{Lecture \thesection, Wednesday June 21, 2023}

    \textit{Ari Feiglin}
}

\bigskip

\begin{thrm*}[roucheTheorem,Rouche's\ Theorem]

    Suppose $f$ and $g$ are analytic functions on and in a regular closed curve $\gamma$, such that $\abs{g(z)}<\abs{f(z)}$ for every $z\in\gamma$.
    Then $f+g$ has the same number of zeros as $f$ in $\gamma$.

\end{thrm*}

\begin{proof}

    Note that this means that on $\gamma$, $f(z)\neq0$.
    So
    \[ f+g = f\parens{1+\frac gf} \]
    In general
    \[ \frac{(A(z)B(z))'}{A(z)B(z)} = \frac{A'B+AB'}{AB} = \frac{A'}A + \frac{B'}B \]
    This means that
    \[ \frac{(f+g)'}{f+g} = \frac{f'}f + \frac{\parens{1+\frac gf}'}{1+\frac gf} \]
    Since $f$ and $g$ are analytic and therefore have no poles,
    \[ Z(f+g) = \frac1{2\pi i}\int_\gamma \frac{f'}f + \frac{\parens{1+\frac gf}'}{1+\frac gf}\,dz = Z(f) + \frac1{2\pi i}\int_\gamma\frac{\parens{1+\frac gf}'}{1+\frac gf}\,dz \]

    Let us define
    \[ w(z) = 1 + \frac{g(z)}{f(z)} \]
    is analytic in a neighborhood of $\gamma$ since $f(z)\neq0$ on $\gamma$.
    Thus
    \[ \Gamma\colon[0,1]\longto\bC,\quad t\varmapsto w(\gamma(t)) \]
    is a closed smooth curve.
    So
    \[ \frac1{2\pi i}\int_\gamma\frac{w'(z)}{w(z)}\,dz = \frac1{2\pi i}\int_\Gamma\frac{dz}{z} = \windof{\Gamma,0} \]
    But notice that
    \[ \abs{1-\Gamma(t)} = \abs{1-w(\gamma(t))} = \abs{\frac{g(\gamma(t))}{f(\gamma(t))}} < 1 \]
    And so $\Gamma\subseteq D_1(1)$, meaning $0$ is in $\Gamma$'s exterior, meaning $\windof{\Gamma,0}=0$.

    So we have that
    \[ Z(f+g) = Z(f) + \windof{\Gamma,0} = Z(f) \qed \]

\end{proof}

\subsection{Computing real integrals with complex analysis}

\begin{exam*}

    Let us compute
    \[ I = \int_0^{2\pi}\frac{\sin^2\theta}{5-4\cos\theta}\,d\theta \]

    Let us define
    \[ z\colon[0,2\pi]\longto\bC,\quad \theta\varmapsto e^{i\theta} \]
    Then $z'=ie^{i\theta}=iz$ so $d\theta=\frac{dz}{iz}$.
    Now we know that
    \[ \cos\theta = \frac{e^{i\theta} + e^{-i\theta}}2 = \frac12\parens{z+\frac1z},\quad \sin\theta = \frac1{2i}\parens{z-\frac1z} \]
    So
    \[ \frac{\sin^2(\theta)}{5-4\cos\theta} = -\frac14\frac{\parens{z-\frac1z}^2}{5-2\parens{z+\frac1z}} = -\frac14\frac{(z^2-1)^2}{5z^2-2\parens{z^3+z}}
    = \frac18\frac{(z^2-1)}{z(z-2)\parens{z-\frac12}} \] 
    Since we have parameterized $\abs z=1$, substituting is simply the same as integrating over $\abs z=1$.
    \[ I = \int_{\abs z=1}\frac18\frac{(z^2-1)^2}{z(z-2)\parens{z-\frac12}}\frac{dz}{di} = \frac{-i}8\int_{\abs z=1}\frac{(z^2-1)^2)}{z^2(z-2)\parens{z-\frac12}}\,dz \]
    This is a rational function, and it has poles at $z=0,2,\frac12$.
    $2$ and $\frac12$ are simple poles, and $0$ has degree two.
    Since $2$ is not in $\abs z=1$, we ignore it.
    The residues are
    \[ \Resof0 = \lim_{z\to0}(z^2f(z))' = \lim_{z\to0}\parens{\frac{(z^2-1)^2}{(z-2)\parens{z-\frac12}}}' = \frac52 \]
    and
    \[ \Resof{\frac12} = -\frac32 \]
    So we get that
    \[ I = -\frac i8\cdot2\pi i\parens{\frac52-\frac32} = \frac\pi4 \]

\end{exam*}

\begin{prop*}

    If $p(x)$ and $q(x)$ are real polynomials such that $d=\deg q-\deg p\geq2$ and $q(x)$ has no real zeros, then
    \[ I = \int_{-\infty}^\infty\frac{p(x)}{q(x)}\,dx = 2\pi i\sum\Resof{\frac pq,z_k} \]
    Where the sum is over the singularities of $\frac{p(z)}{q(z)}$ for $0<\Imof z$.

\end{prop*}

\begin{proof}

    Recall that if $\lim_{x\to\infty}\frac{f(x)}{g(x)}=L\in[0,\infty]$ for non-negative $f$ and $g$, if
    \benum
        \item $L\in(0,\infty)$ then $\int_0^\infty f$ converges if and only if $\int_0^\infty g$ does.
        \item If $L=0$ then if $\int_0^\infty g$ converges so does $\int_0^\infty f$.
        \item If $L=\infty$ then if $\int_0^\infty f$ converges so does $\int_0^\infty g$.
    \eenum

    Since
    \[ \lim_{x\to\pm\infty}\frac{\abs{\frac pq}}{\frac1{\abs x^d}} \]
    exists and is finite, and
    \[ \int_{\pm1}^{\pm\infty}\frac1{\abs x^d} \]
    exists and is finite (since $d>1$), then $\int_{\pm1}^{\pm\infty}\abs{\frac pq}$ also converges and so the integral of $\frac pq$ converges absolutely.
    Since
    \[ I = \int_{-\infty}^{-1} \frac pq + \int_{-1}^1 \frac pq + \int_1^\infty \frac pq \]
    $I$ converges absolutely as well (since $q$ has no real zeros, $\frac pq$ has no singularities and is therefore continuous in $(-1,1)$ so its integral exists).

    Let us denote $\Delta_R$ as the top half of the circle $C_R$, and $\Gamma_R$ be the top half of the circle with the line $[-R,R]$.
    Notice that $f(x)=\frac{p(x)}{q(x)}$ is meromorphic in $\bC$.
    We can assume that $\Gamma_R$ does not intersect any of its singularities.
    Notice that $\Gamma_R$ is a closed regular curve and so
    \[ \int_{\Gamma_R} f(z)\,dz = 2\pi i\sum \Resof{f,z_k} \]
    where the sum is over the singularities of $f$ within $\Gamma_R$.

    Now notice that
    \[ \int_{\Gamma_R} f(z) = \int_{-R}^R f(x)\,dx + \int_{\Delta_R}f(z)\,dz \]
    And so
    \[ \abs{\int_{\Delta R}f(z)\,dz} \leq \sup_{z\in\Delta_R}\abs{f(z)}\cdot\pi R \]
    But since $z^2\frac{p(z)}{q(z)}$ is bounded (since it is the division of two polynomials of equal degree), we get that there exists an $A$ such that $\abs{z^2}\cdot\abs{f(z)}<A$.
    And so
    \[ \abs{\int_{\Delta_R}f(z)} \leq \frac A{R^2}\cdot\pi R\xvarrightarrow{}[R\to\infty] 0 \]
    Meaning that
    \[ \lim_{R\to\infty}\int_{\Gamma_R}f(z)\,dz = \lim_{R\to\infty}\int_{-R}^Rf(x)\,dx = \int_{-\infty}^\infty f(x)\,dx \]
    And since
    \[ \int_{\Gamma_R}f(z) = 2\pi i\sum\Resof{f,z_k} \]
    where $z_k$ are singularities in $\Gamma_R$.
    Since $\Gamma_R$ is a half-circle with a positive imaginary part, for any singularity such that $\Imof{z_k}>0$, eventually $z_k$ will be in $\Gamma_R$.
    So
    \[ \int_{-\infty}^\infty f(x)\,dx = 2\pi i\sum\Resof{f,z_k},\qquad z_k\text{ singularity, }\Imof{z_k}>0 \]

\end{proof}

\begin{prop*}

    Let $R(x)=\frac{p(x)}{q(x)}$ be a rational function (assume $p$ and $q$ are coprime in $\bC[z]$; that they have no common root), and $q$ is not zero on the real line.
    Suppose $\deg q>\deg p$ then
    \[ \int_{-\infty}^\infty R(x)\cos(x) = \Reof{2\pi i\sum \Resof{R(z)e^{iz}, z_k}} \]
    and
    \[ \int_{-\infty}^\infty R(x)\sin(x) = \Imof{2\pi i\sum \Resof{R(z)e^{iz}, z_k}} \]
    Where the sum is over singularities of $R(z)e^{iz}$ with positive imaginary parts.

\end{prop*}

\begin{proof}

    Notice that
    \[ R'(x) = \frac{p'q - pq'}{q^2} \]
    and since $\deg(p'q-pq')>0$, and eventually polynomials are either always negative or positive (since otherwise they'd have infinite roots), $R'(x)$ is either always negative or positive in some
    ray $[a,\infty)$ and $(-\infty,b]$ meaning $R$ is eventually monotonic.

    Now note that if we define
    \[ F_+(x) = \int_b^x \cos(t)\,dt,\qquad F_-(x) = \int_x^b \cos(t)\,dt \]
    Then $F_+$ and $F_-$ are bounded in $[b,\infty)$ and $(-\infty,b]$ respectively, and so by Dirichlet's test
    \[ \int_{-\infty}^\infty R(x)\cos(x)\,dx \]
    exists.
    Similar for $\sin(x)$.

    Let us use the same $\Gamma_R$ and $\Delta_R$ as last proof, and for $0<h<R$ let us define
    \[ \Delta_{R,+} = \set{z\in\Delta_R}[\Imof z\geq h],\quad \Gamma_{R,-} = \set{z\in\Delta_R}[\Imof z<h] \]
    Since $\deg q>\deg p$, there exists a $k>0$ such that for $\abs z$ sufficiently large
    \[ \abs{R(z)} < \frac k{\abs z} \]
    Then
    \[ \sup_{z\in\Delta_{R,+}}\abs{R(z)e^{iz}} = \sup_{z\in\Delta_{R,+}} \abs{R(z)}\cdot e^{-\Im(z)} \leq \frac kR\cdot e^{-h} \]
    So
    \[ \abs{\int_{\Delta_{R,+}}R(z)e^{iz}} \leq \frac kRe^{-h}\pi R = k\pi e^{-h} \]
    And since the perimiter of $\Delta_{R,-}\leq4h$ (this involves some trigonometry), we have
    \[ \abs{\int_{\Delta_{R,-}}R(z)e^{iz}} \leq \frac kR4h \]
    So
    \[ \abs{\int_{\Delta_R}R(z)e^{iz}} \leq k\pi e^{-h} + \frac{4k}Rh \]
    Let $h=\sqrt R$ and we get
    \[ \abs{\int_{\Delta_R}R(z)e^{iz}} \leq k\pi e^{-\sqrt{R}} + \frac{4k}{\sqrt R}\xvarrightarrow{}[R\to\infty] 0 \]

    So we get
    \[ \lim\int_{\Gamma_R}R(z)e^{iz} = \lim\int_{-R}^R R(z)e^{iz} = \int_{-\infty}^\infty R(x)e^{ix} \]
    And since
    \[ \int_{\Gamma_R}R(z)e^{iz} = 2\pi i\sum\Resof{R(z)e^{iz},z_k} \]
    where $z_k$ are singularities in $\Gamma_R$ we get
    \[ \int_{-\infty}^\infty R(x)e^{ix}\,dx = 2\pi i\sum\Resof{R(z)e^{iz},z_k} \]
    where $\Imof{z_k}>0$.

    The imaginary and real parts of this integral correspond with the integrals of $R(x)\cos(x)$ and $R(x)\sin(x)$ respectively, as required.
    \qed

\end{proof}

\begin{defn*}

    The \ppemph{principal value} of the integral $\int_{-\infty}^\infty f(x)\,dx$ is defined as
    \[ \pv\int_{-\infty}^\infty f(x)\,dx = \lim_{R\to\infty}\int_{-R}^R f(x)\,dx \]

\end{defn*}

The principal value of an integral may exist without the integral existing, but if the integral exists then they are equal.
For example if $f(x)=x$ then $\pv\int_{-\infty}^\infty x\,dx=0$ but the indefinite integral does not exist.

\end{document}

