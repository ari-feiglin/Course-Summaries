\documentclass[10pt]{article}

\usepackage{amsmath, amssymb, mathtools}
\usepackage{tikz}
\usepackage[margin=1.5cm]{geometry}

\input pdfmsym
\input prettyprint
\input preamble

\pdfmsymsetscalefactor{10}
\initpps

\def\pmat#1{\begin{pmatrix} #1 \end{pmatrix}}

\let\divides=\mid
\newfunc{metric}\rho({})
\newfunc{metricc}\sigma({})
\newfunc{spa}{{\rm span}}(\vert)
\newfunc{diam}{{\rm diam}}(\vert)

\font\bigbf = cmbx12 scaled 2000
\@undervecc@def{underbar}\@linecap\@linecap

\def\openset{{\cal O}}
\let\lineseg=\overleftrightvecc
\let\ds=\displaystyle

\def\pdv#1#2{\frac{\partial #1}{\partial #2}}

\def\differ#1#2{\left.d#1\strut\right|_{#2}}

\newfunc{Re}{{\rm Re}}({})
\newfunc{Im}{{\rm Im}}({})

\def\@ppmathcount{\thesection.\thepp@mathcount}

\begin{document}

\c@section=1

\barcolorbox{220, 255, 220}{0, 130, 0}{80, 200, 80}{
    \leftskip=0pt plus 1fill \rightskip=\leftskip
    {\bigbf Complex Functions}

    \medskip
    \textit{Lecture \thesection, Wednesday March 15, 2023}

    \textit{Ari Feiglin}
}

\bigskip

This lecture was largely a review of the basics of the complex field $\bC$, and so I did not transcribe the first two hours or so, as everyone taking this course should already be familiar with it.

\begin{defn*}

    Suppose $\set{z_n}_{n=1}^\infty$ is a complex sequence, then it converges to $z\in\bC$ if:
    \[ \abs{z_n-z}\xvarrightarrow{}[n\varrightarrow\infty]0 \]
    this is denoted
    \[ z_n\xvarrightarrow{}[n\varrightarrow\infty] z \text{ or } \lim z_n = z \]

\end{defn*}

Note that since $\abs{z_n-z}$ is equal to the norm of $z_n-z$ when viewed as a vector, a sequence converges to at most one value.
And since convergence in $\bR^n$ is equivalent to pointwise convergence, $z_n$ converges to $z$ if and only if $\Re(z_n)$ converges to $\Re(z)$ and $\Im(z_n)$ converges to $\Im(z)$.

The arithmetic of sequences is the same in $\bC$ as it is in $\bR$ since for addition this is simply the addition of two vector sequences, scaling a sequence by $w\in\bC$ has that
\[ \abs{wz_n-wz} = \abs{w}\cdot\abs{z_n-z} \xvarrightarrow{}[n\varrightarrow\infty] 0 \]
and since $\bC$ is a field, we can also multiply two sequences: suppose $\set{z_n}_{n=1}^\infty$ and $\set{w_n}_{n=1}^\infty$ are two complex sequences which converge to $z$ and $w$ respectively.
Then $\set{z_nw_n}_{n=1}^\infty$ converges to $zw$:
\[ \abs{z_nw_n-zw} = \abs{z_n(w_n-w)+w(z_n-z)} \leq \abs{z_n}\abs{w_n-w} + \abs{w}\abs{z_n-z} \]
which converges to $0$ since $\abs{z_n}$ must be bounded (since $\abs{z_n}\leq\abs{z_n-z}+\abs z$).

The definition of a complex series is analogous to a real one, and similarly if $\sum z_n$ converges, then $z_n$ converges to $0$ (the proof is simple using sequence arithmetic).

\end{document}

