\documentclass[10pt]{article}

\usepackage{amsmath, amssymb, mathtools}
\usepackage[margin=1.5cm]{geometry}

\input pdfmsym
\input prettyprint
\input ../preamble

\pdfmsymsetscalefactor{10}
\initpps

\def\pmat#1{\begin{pmatrix} #1 \end{pmatrix}}

\let\divides=\mid
\newfunc{metric}\rho({})
\newfunc{metricc}\sigma({})
\newfunc{ker}{{\rm ker}}({})
\newfunc{spa}{{\rm span}}(\vert)
\newfunc{atan}{{\rm tan}^{-1}}({})
\newfunc{lag}{{\cal L}}({})
\newfunc{sin}{{\rm sin}}({})
\newfunc{sinh}{{\rm sinh}}({})
\newfunc{cosh}{{\rm cosh}}({})
\newfunc{cos}{{\rm cos}}({})
\newfunc{arg}{{\rm arg}}({})
\newfunc{Re}{{\rm Re}}({})
\newfunc{Im}{{\rm Im}}({})
\newfunc{proj}\pi({})
\newfunc{iproj}{\pi^{-1}}({})
\newfunc{cis}{{\rm cis}}({})

\@Arrow@def{varLeftRightarrow}\@Larrow\@Rarrow{1}
\def\iff{\,\longvarLeftRightarrow\,}
\def\implies{\,\longvarRightarrow\,}

\font\bigbf = cmbx12 scaled 2000

\def\mO{{\cal O}}
\def\mU{{\cal U}}
\def\mV{{\cal V}}
\let\lineseg=\overleftrightvecc
\let\ds=\displaystyle

\def\pdv#1#2{\frac{\partial #1}{\partial #2}}

\def\differ#1#2{\left.d#1\strut\right|_{#2}}

\def\@ppmathcount{\thesection.\thepp@mathcount}

\let\ds=\displaystyle
\let\longto=\longvarrightarrow
\let\to=\varrightarrow

\def\bexerc{\begin{exercise*}}
\def\eexerc{\end{exercise*}}
\def\bblank{\begin{blankpp}}
\def\eblank{\end{blankpp}}

\begin{document}

\c@section=8

\barcolorbox{220, 255, 220}{0, 130, 0}{80, 200, 80}{
    \leftskip=0pt plus 1fill \rightskip=\leftskip
    {\bigbf Complex Functions}

    \medskip
    \textit{Assignment \thesection}

    \textit{Ari Feiglin}
}

\bigskip

\bexerc

    Suppose $f$ is analytic and not constant on a compact domain $D$.
    Show that $\Re f$ and $\Im f$ obtain their maxima and minima on the boundary of $D$.

\eexerc

\bblank

    I will prove this using the result of the next exercise.
    Suppose $z_0$ induces a maximum or minimum for $\Re f$ or $\Im f$, then $f(z_0)$ is on the boundary of $f(D)$.
    This is because either $f(z_0)\pm\frac\epsilon2$ or $f(z_0)\pm i\frac\epsilon2$ is not in $f(D)$ for any $\epsilon>0$ (depending on whether $f(z_0)$ is a maximum or minimum, and for which function).
    And since $f(z_0)$ is necessarily in $f(D)$, we have that for every $\epsilon>0$, $D_\epsilon(z_0)$ is not disjoint with $f(D)$ or $f(D)^c$, which is precisely what $f(z_0)$ being a boundary point
    of $f(D)$ means.

    Thus by the result of the next question, $z_0\in\partial D$ as required.

\eblank

\bexerc

    \benum
        \item Show that if $f$ is analytic and not constant on $S$, and $f(S)=T$ then if $f(z)$ is a boundary point of $T$ then $z$ is a boundary point of $S$.
        \item Let $f(z)=z^2$, and let $S$ be the union of $S_1$ and $S_2$ where
        \[ S_1=\set{z}[\abs z\leq2,\,\Re z\leq0],\qquad S_2=\set{z}[\abs z\leq1,\,\Re z\geq0] \]
        Show that there exists a boundary point of $S$, $z$, such that $f(z)$ is an interior point of $f(S)$.
    \eenum

\eexerc

\bblank

    \benum
        \item Since we know non-constant analytic functions are open maps, $f(\interior S)$ is open in $T$, and since the interior of a set is the largest open set contained within said set, we have that
        $f(\interior S)\subseteq\interior T$.
        Since $f(z)\in\boundary T$, this means that $f(z)\notin\interior T$ and thus $z\notin\interior S$.
        So $z\in S\setminus\interior S\subseteq\boundary S$ as required.

        \item Notice that $f(S_1)=\bar D_4(0)$ and $f(S_2)=\bar D_1(0)$.
        This is because if $z=re^{i\theta}\in S_2$ then $r\leq2$ and $\frac\pi2\leq\theta\leq\frac32\pi$.
        Thus $f(re^{i\theta})=r^2e^{2i\theta}$ and since $r^2\leq4$, $f(z)\in\bar D_4(0)$.
        And if $re^{i\theta}\in\bar D_4(0)$, let $z=\sqrt re^{i\alpha}$ where $\alpha=\frac\theta2$ if $\pi\leq\theta$, and $\alpha=\frac\theta2+\pi$ otherwise.
        In any case, we have that $z\in S_1$, and $f(z)=re^{i\theta}$.
        Thus $f(S_1)=\bar D_4(0)$ as required.
        A nearly identical proof holds for $S_2$.

        Thus $f(S)=\bar D_4(0)$.
        So let us take $z=1$, which is on the boundary of $S$ (for any $\epsilon>0$, $z+\frac\epsilon2$ is not in $S$), but $f(z)=1$ which is in the interior of $f(S)=\bar D_4(0)$ ($D_1(1)$ is contained
        within $f(S)$).
    \eenum

\eblank

\bexerc

    Suppose $f$ is an analytic function strictlu bounded by $1$ on the unit disk.
    Further suppose that there exists an $\alpha$ on the unit disk where $f(\alpha)\neq0$.
    Show that there exists an analytic function $g$ which is also strictly bounded by $1$ on the unit disk where $\abs{f'(\alpha)}<\abs{g'(\alpha)}$.

\eexerc

\bblank

    Let us define
    \[ g(z) = \frac{f(z) - f(\alpha)}{1-f(z)\cdot\overline{f(\alpha)}} \]
    We note that this is defined over all of $D_1(0)$, since it is only undefined when
    \[ f(z) = \frac1{\overline{f(\alpha)}} \implies \abs{f(z)} = \frac1{\abs{f(\alpha)}} > 1 \]
    since $\abs f<1$, this is a contradiction.
    And since $g(z)$ is the quotient of two analytic functions, it itself is analytic in $D_1(0)$.

    $g(z)$ is also strictly bounded by $1$ in $D_1(0)$ since
    \[ \abs{g(z)} < 1 \iff \abs{f(z)-f(\alpha)} < \abs{1-f(z)\cdot\overline{f(\alpha)}} \iff \abs{f(z)-f(\alpha)}^2 < \abs{1-f(z)\cdot\overline{f(\alpha)}}^2 \]
    Let us compute both sides with the identity $\abs z^2=z\cdot\overline z$:
    \[ \abs{f(z)-f(\alpha)}^2 = (f(z) - f(\alpha))(\overline{f(z)} - \overline{f(\alpha)}) = \abs{f(z)}^2 + \abs{f(\alpha)}^2 - f(z)\overline{f(\alpha)} - f(\alpha)\overline{f(z)} \]
    and
    \[ \abs{1-f(z)\cdot\overline{f(\alpha)}}^2 = (1-f(z)\cdot\overline{f(\alpha)})(1-\overline{f(z)}f(\alpha)) = 1 + \abs{f(z)}^2\abs{f(\alpha)}^2 - f(z)\overline{f(\alpha)} - f(\alpha)\overline{f(z)} \]
    Thus the inequality holds if and only if
    \[ \abs{f(z)}^2 + \abs{f(\alpha)}^2 < 1 + \abs{f(z)}^2\abs{f(\alpha)}^2 \]
    Which is if and only if
    \[ \abs{f(\alpha)}^2\bigl(1-\abs{f(z)}^2\bigr) < 1 - \abs{f(z)}^2 \]
    and since $\abs{f(z)}<1$, we can divide both sides by $1-\abs{f(z)}^2$ and preserve the inequality, meaning this is if and only if
    \[ \abs{f(\alpha)}^2 < 1 \]
    Thus we have shown that $\abs{g(z)}<1$ in $D_1(0)$ as required.

    Now notice that $g(\alpha)=0$ and so
    \[ g'(\alpha) = \lim_{z\to\alpha}\frac{g(z)}{z-\alpha} = \lim_{z\to\alpha}\frac{f(z)-f(\alpha)}{z-\alpha}\cdot\frac1{1-f(z)\cdot\overline{f(\alpha)}}
    = f'(\alpha)\cdot\frac1{1-f(\alpha)\cdot\overline{f(\alpha)}} = f'(\alpha)\cdot\frac1{1-\abs{f(\alpha)}^2} \]
    Since $0<\abs{f(\alpha)}<1$ we have that $\frac1{1-\abs{f(\alpha)}^2}>1$ and so
    \[ \abs{g'(\alpha)} = \abs{f'(\alpha)}\cdot\frac1{1-\abs{f(\alpha)}^2} > \abs{f'(\alpha)} \]
    as required.

\eblank

\bexerc

    Suppose $f$ is an entire function such that $\abs{f(z)}\leq\frac1{\Reof z^2}$.
    Prove that $f$ is identically zero.

\eexerc

\bblank

    Let $R>0$ and let us define for $\abs z<R$
    \[ g(z) = (z^2 + R^2)^4f(z) \]
    Let $\abs z=R$, and notice that for such a $z$, suppose $z=a+bi$ then $R^2=a^2+b^2$ and so
    \[ z^2+R^2 = a^2 - b^2 + 2abi + a^2 + b^2 = 2a^2 + 2abi = 2a(a+bi) = 2z\Re z \]
    and so
    \[ \abs{g(z)} = \abs{z^2 + R^2}^4\cdot\abs{f(z)} \leq \abs{2z\Re z}^4\cdot\frac1{\Reof z^2} \leq 16R^4\Re z^4\cdot\frac1{\Reof z^2} = 16R^4\Reof z^2 \leq 16R^6 \]
    since $\Reof z\leq R$.

    So for every $z\in\boundary D_R(0)$ we have $\abs{g(z)}\leq 16R^6$.
    But by the maximum modulus principal, for the maximum of $g(z)$ on $D_R(0)$ is obtained on its boundary, ie when $\abs z=R$.
    Thus for every $\abs z\leq R$, $\abs{g(z)}\leq 16R^6$.

    So let $z\in\bC$, then for every $R>0$ such that $\abs z\leq R$, we have
    \[ \abs{f(z)}\cdot\abs{z^2+R^2}^4 \leq 16R^6 \implies \abs{f(z)} \leq \frac{16R^6}{\abs{z^2+R^2}^4} \]
    And by letting $R\to\infty$, we get that $\frac{16R^6}{\abs{z^2+R^2}^4}\to0$ and so $\abs{f(z)}\leq0$ meaning $f(z)=0$ for every $z\in\bC$ as required.

\eblank

\bexerc

    Show that
    \[ f(z) = \int_0^1 \frac{\sin(zt)}t\,dt \]
    is an entire function by
    \benum
        \item Morera's theorem
        \item Finding a powerseries for $f$
    \eenum

\eexerc

\bblank

    \benum
        \item Let $\Gamma$ be the boundary of a complex rectangle, we must show that
        \[ \int_\Gamma f(z)\,dz = 0 \]
        and then by Morera's theorem, $f$ is analytic.

        Notice that
        \[ \int_\Gamma\int_0^1\abs{\frac{\sin(zt)}t}\,dt\,dz \]
        converges since the inner integral converges (we showed this in calculus $2$), and is bounded.

        Thus by Fubini-Tonelli, we have that
        \[ \int_\Gamma f(z)\,dz = \int_\Gamma\int_0^1\frac{\sin(zt)}t\,dtdz = \int_0^1\int_\Gamma\frac{\sin(zt)}t\,dzdt \]
        by Cauchy's theorem, since $z\varmapsto\frac{\sin(zt)}t$ is analytic in $\Gamma$'s interior, the inner integral is $0$, and thus the integral as a whole is zero, as required.

        \item Using $\sin$'s powerseries
        \[ \sin(z) = \sum_{n=0}^\infty (-1)^n\frac{z^{2n+1}}{(2n+1)!} \]
        and thus
        \[ \frac{\sin(zt)}t = \sum_{n=0}^\infty (-1)^n\frac{z^{2n+1}t^{2n}}{(2n+1)!} \]
        This still has a radius of convergence of infinity (both as a powerseries for $z$ and $t$, since we are taking a powerseries defined everywhere and dividing it by $t$, and this still results in a
        powerseries).
        Thus since powerseries converge uniformly
        \[ \int_0^1\frac{\sin(zt}t\,dt = \sum_{n=0}^\infty (-1)^n\frac{z^{2n+1}}{(2n+1)!}\cdot\int_0^1 t^{2n}\,dt = \sum_{n=0}^\infty (-1)^n\frac{z^{2n+1}}{(2n+1)!\cdot(2n+1)} \]

        And this has a radius of convergence of infinity, as is obviously apparent by the ratio test.
        Thus $f(z)$ has a powerseires which is convergent everywhere, meaning it is entire.
    \eenum

\eblank

\bexerc

    Show that the function $f$ from the previous exercise satisfies
    \[ f'(z) = \int_0^1\cos(zt)\,dt \]
    by
    \benum
        \item Using the change of order of integration.
        \item Using the powerseries from the previous exercise.
    \eenum

\eexerc

\bblank

    \benum
        \item Notice that
        \[ \int_0^z \cos(wt)\,dw = \frac{\sin(wt)}w\biggl|_0^z = \frac{\sin(zt)}z \]
        and thus
        \[ f(z) = \int_0^1\int_0^z \cos(wt)\,dwdt \]
        Now for any $z\in\bC$, since
        \[ \int_0^z\int_0^1 \abs{\cos(wt)}\,dt\abs{dw} \leq \int_0^z \abs{dw} \]
        is convergent, by Fubini-Tonelli we have
        \[ f(z) = \int_0^1\int_0^z \cos(wt)\,dwdt = \int_0^z\int_0^1 \cos(wt)\,dtdw \]
        and by the Fundamental theorem of calculus, this means
        \[ f'(z) = \int_0^1 \cos(zt)\,dt \]
        as required.

        \item Using the powerseries of $\cos$,
        \[ \cos(z) = \sum_{n=0}^\infty (-1)^n\frac{z^{2n}}{(2n)!} \]
        we have that
        \[ \int_0^1 \cos(zt)\,dt = \int_0^1 \sum_{n=0}^\infty (-1)^n\frac{z^{2n}t^{2n}}{(2n)!}\,dt = \sum_{n=0}^\infty (-1)^n\frac{z^{2n}}{(2n)!}\int_0^1 t^{2n}\,dt =
        \sum_{n=0}^\infty (-1)^n\frac{z^{2n}}{(2n+1)!} \]
        And we know that by the previous exercise
        \[ f(z) = \sum_{n=0}^\infty (-1)^n\frac{z^{2n+1}}{(2n+1)!\cdot(2n+1)} \]
        So
        \[ f'(z) = \sum_{n=0}^\infty (-1)^n\frac{(2n+1)z^{2n}}{(2n+1)!\cdot(2n+1)} = \sum_{n=0}^\infty (-1)^n\frac{z^{2n}}{(2n+1)!} = \int_0^1 \cos(zt)\,dt \]
        as required.

    \eenum

\eblank

\bexerc

    Show that
    \[ L(z) = \pi i + \int_{-1}^z \frac{dw}w \]
    is a branch of the complex logarithm in $D=\set{z\in\bC}[z\in\bR\implies z<0]$.
    And further show that
    \[ 0 < \Im L(z) = \argof z < 2\pi \]

\eexerc

\bblank

    We showed in lecture that if $D$ is a simply connected domain where $0\notin D$, and $e^{L_0}=z_0$ then
    \[ L(z) = L_0 + \int_{z_0}^z \frac{dw}w \]
    is an analytic branch of the complex logarithm in $D$.

    Since the domain $D$ defined in the question is simply connected, and $e^{i\pi}=-1$, we have that the $L$ defined in the question is indeed an analytic branch of the complex logarithm.

    For $z\in D$ let us define the smooth curve $\Gamma$ as the concatenation of the curve from $-1$ to $-\abs z$ (contained within $\bR$), and then the arc from $-\abs z$ to $z$ (this is part of the
    circle around $0$ of radius $\abs z$).
    Let us denote the first part of this curve by $\Gamma_1$, and the second (the arc) by $\Gamma_2$.
    So we have that
    \[ L(z) = i\pi + \int_\Gamma \frac{dw}w = i\pi + \int_{\Gamma_1} \frac{dw}w + \int_{\Gamma_2} \frac{dw}w \]
    Since $\Gamma_1$ is contained entirely within $\bR$, the integral over $\Gamma_1$ does not contribute to $\Imof{L(z)}$.

    Suppose $z=re^{i\alpha}$, we can parameterize $\Gamma_2$ by
    \[ [\pi,\alpha]\longto\Gamma_2,\quad \theta\varmapsto re^{i\theta} \]
    and thus we have that
    \[ \int_{\Gamma_2} \frac{dw}w = \int_{\Gamma_2} \frac{\overline w}{\abs w^2}\,dw = \int_\pi^\alpha \frac{re^{-i\theta}}{r^2}\cdot rie^{i\theta}\,d\theta = i\int_\pi^\alpha d\theta = i(\alpha-\pi) \]
    So we have
    \[ \Imof{L(z)} = \Imof{i\pi + \int_{\Gamma_1}\frac{dw}w + \int_{\Gamma_2}\frac{dw}w} = \pi + \Imof{i(\alpha-\pi)} = \pi + \alpha - \pi = \alpha = \argof{z} \]
    as required.
    ($\argof z>0$ since if $\argof z=0$ then $z\in\bR$ and $z\geq0$, so $z\notin D$).

\eblank

\end{document}

