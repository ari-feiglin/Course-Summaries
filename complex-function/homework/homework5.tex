\documentclass[10pt]{article}

\usepackage{amsmath, amssymb, mathtools}
\usepackage[margin=1.5cm]{geometry}

\input pdfmsym
\input prettyprint
\input ../preamble

\pdfmsymsetscalefactor{10}
\initpps

\def\pmat#1{\begin{pmatrix} #1 \end{pmatrix}}

\let\divides=\mid
\newfunc{metric}\rho({})
\newfunc{metricc}\sigma({})
\newfunc{ker}{{\rm ker}}({})
\newfunc{spa}{{\rm span}}(\vert)
\newfunc{atan}{{\rm tan}^{-1}}({})
\newfunc{lag}{{\cal L}}({})
\newfunc{sin}{{\rm sin}}({})
\newfunc{sinh}{{\rm sinh}}({})
\newfunc{cosh}{{\rm cosh}}({})
\newfunc{cos}{{\rm cos}}({})
\newfunc{arg}{{\rm arg}}({})
\newfunc{Re}{{\rm Re}}({})
\newfunc{Im}{{\rm Im}}({})
\newfunc{proj}\pi({})
\newfunc{iproj}{\pi^{-1}}({})
\newfunc{cis}{{\rm cis}}({})

\@Arrow@def{varLeftRightarrow}\@Larrow\@Rarrow{1}
\def\iff{\,\longvarLeftRightarrow\,}
\def\implies{\,\longvarRightarrow\,}

\font\bigbf = cmbx12 scaled 2000

\def\mO{{\cal O}}
\def\mU{{\cal U}}
\def\mV{{\cal V}}
\let\lineseg=\overleftrightvecc
\let\ds=\displaystyle

\def\pdv#1#2{\frac{\partial #1}{\partial #2}}

\def\differ#1#2{\left.d#1\strut\right|_{#2}}

\def\@ppmathcount{\thesection.\thepp@mathcount}

\let\ds=\displaystyle
\let\longto=\longvarrightarrow
\let\to=\varrightarrow

\def\bexerc{\begin{exercise*}}
\def\eexerc{\end{exercise*}}
\def\bblank{\begin{blankpp}}
\def\eblank{\end{blankpp}}

\begin{document}

\c@section=2

\barcolorbox{220, 255, 220}{0, 130, 0}{80, 200, 80}{
    \leftskip=0pt plus 1fill \rightskip=\leftskip
    {\bigbf Complex Functions}

    \medskip
    \textit{Assignment \thesection}

    \textit{Ari Feiglin}
}

\bigskip

\bexerc

    Find the Taylor series of $f(z)=z^2$ around $z=2$.

\eexerc

\bblank

    We know that $f'(z)=2z$ and $f''(z)=2$ and $f^{(k)}(z)=0$ for $k\geq3$.
    Thus
    \[ f(z) = \sum_{k=0}^\infty\frac{f^{(k)}(2)}{k!}(z-2)^k = 4 + 4(z-2) + (z-2)^2 \]

\eblank

\bexerc

    Find the Taylor series of $f(z)=e^z$ about $a\in\bC$.

\eexerc

\bblank

    We know that $f'(z)=e^z$ and so inductively $f^{(k)}(z)=e^z$.
    So
    \[ f(z) = \sum_{k=0}^\infty\frac{f^{(k)}(a)}{k!}(z-a)^k = \sum_{k=0}^\infty\frac{e^a}{k!}(z-a)^k = e^a\cdot\sum_{k=0}^\infty\frac1{k!}(z-a)^k \]

\eblank

\bexerc

    An \emph{odd} function is a function such that $f(-z)=-f(z)$, and an \emph{even} function is a function such that $f(-z)=f(z)$.

    Let $f$ be an entire odd function.
    Prove that $f$'s Taylor polynomial has only odd powers.
    Prove a similar result for entire even functions.

\eexerc

\bblank

    If $f$ is an entire odd function then
    \[ f(z) = \frac{f(z) + f(z)}2 = \frac{f(z) - f(-z)}2 \]
    And so
    \[ f'(z) = \frac{f'(z) + f'(-z)}2 \]
    And this is also entire (as the derivative of an entire function is entire), and it is also even since
    \[ f'(-z) = \frac{f'(-z) + f(z)}2 = f'(z) \]

    And if $f$ is an entire even function then
    \[ f(z) = \frac{f(z) + f(z)}2 = \frac{f(z) + f(-z)}2 \]
    And so
    \[ f'(z) = \frac{f'(z) - f'(-z)}2 \]
    And this is an entire odd function.

    So if $f$ is odd, inductively we see that for $k$ even $f^{(k)}$ is odd and if $k$ is odd $f^{(k)}$ is even.
    This is true for $k=0$ and $k=1$ as we showed above.
    And if it is true for $k$ even then $f^{(k)}$ is odd and so $f^{(k+1)}$ is even and $k+1$ is odd, and similar for $k$ odd.
    Notice that if $g$ is an odd function then $g(0)=-g(0)$ so $g(0)=0$.
    Thus for $k$ even, $f^{(k)}(0)=0$ and so all of the coefficients with even indexes in the Taylor series are zero (since $c_k=\frac{f^{(k)}(0)}{k!}$), as required.

    And if $f$ is even then inductively we see that for $k$ even $f^{(k)}$ is even and for $k$ odd $f^{(k)}$ is odd.
    And the odd coefficients in the Taylor series are zero, as required.

\eblank

\bexerc

    Prove that if $f$ is an entire function and $C$ is a circle which contains $a$ then for $k\in\bN$:
    \[ f^{(k)}(a) = \frac{k!}{2\pi i}\int_C \frac{f(w)}{(w-a)^{k+1}}\,dw \]

\eexerc

\bblank

    We prove this inductively on $k$.
    For $k=0$ this reduces to Cauchy's integral theorem which we proved in lecture.
    Let $\tilde C$ be the open circle whose boundary is $C$, thus $a\in C$.
    Now suppose it is true for $k-1$ then for small enough $h$, $a+h\in\tilde C$ as it is open and so by our induction hypothesis
    \[ \frac{f^{(k-1)}(a+h) - f^{(k-1)}(a)}h = \frac{(k-1)!}{2\pi i}\int_C f(w)\cdot\frac1h\parens{\frac1{(w-a-h)^k} - \frac1{(w-a)^k}}\,dw \]
    Our goal is to show that this converges to the target integral (in the statement of the exercise) as $h\longto0$.

    Let $h_n\longto0$, and we can assume without loss of generality that $a+h_n\in\tilde C$ for every $n$.
    Then let $g_n\colon C\longto\bC$ by
    \[ g_n(w) = f(w)\cdot\frac1{h_n}\parens{\frac1{(w-a-h_n)^k} - \frac1{(w-a)^k}}\,dw \]
    Our goal is to show that $g_n$ converges uniformly to $g$ on $C$ where
    \[ g(w) = \frac{k\cdot f(w)}{(w-a)^{k+1}} \]

    We can expand out the fraction in $g_n$:
    \[ \frac1{h_n}\cdot\frac1{(w-a-h_n)^k} - \frac1{(w-a)^k} = \frac{(w-a)^k - (w-a-h_n)^k}{(w-a)^k(w-a-h_n)^k} \]
    Using the binomial theorem, we can see that
    \[ \frac1{h_n}\cdot\Bigl((w-a)^k - (w-a-h_n)^k\Bigr) = \frac1{h_n}\parens{(w-a)^k - \sum_{\ell=0}^k\binom k\ell (-1)^\ell(w-a)^{k-\ell}h_n^\ell} =
    \sum_{\ell=1}^k\binom k\ell (-1)^{\ell-1}(w-a)^{k-\ell}h_n^{\ell-1} \]

    And so
    \[ g_n - g = f(w)\cdot\parens{\frac{\sum_{\ell=1}^k\binom k\ell (-1)^{\ell-1}(w-a)^{k-\ell}h_n^{\ell-1}}{(w-a)^k(w-a-h_n)^k} - \frac k{(w-a)^{k+1}}} = \]
    \[ = f(w)\cdot\frac1{(w-a)^{k+1}(w-a-h_n)^k}\cdot\parens{\sum_{\ell=1}^k\binom k\ell (-1)^{\ell-1}(w-a)^{k-\ell+1}h_n^{\ell-1} - k(w-a-h_n)^k} \]
    The subformula in the parentheses can be rewritten like so
    \[ \sum_{\ell=0}^{k-1}\binom k{\ell+1}(-1)^\ell(w-a)^{k-\ell}h_n^\ell - k\sum_{\ell=0}^k\binom k\ell(-1)^\ell(w-a)^{k-\ell}h_n^\ell =
    \sum_{\ell=0}^{k-1}(-1)^\ell(w-a)^{k-\ell}h_n^\ell\cdot\parens{\binom k{\ell+1}-k\binom k\ell} - (-1)^kh_n^k \]
    Since when $\ell=0$ $\binom k1-k\binom k0=k-k=0$, we can start indexing the sum at $\ell=1$.
    Thus we get that the modulus of this is less than
    \[ \abs{\dots} \leq M\sum_{\ell=1}^{k-1}\abs{w-a}^{k-\ell}\abs{h_n}^\ell + \abs{h_n}^k \]
    where $M$ is the maximum value of $\binom k{\ell+1}-k\binom k\ell$.
    Let $N$ be the maximum value of $\abs{w-a}^{k-\ell}$ iterating over all $w\in C$ and $\ell=1,\dots,k-1$, and we can assume that $\abs{h_n}<1$ so $\abs{h_n}^\ell\leq\abs{h_n}$.
    Thus this in turn is less than
    \[ \leq MN(k-2)\abs{h_n} + \abs{h_n} \]
    Let us simply denote this as $h'_n$, and it is clear that $h'_n\longto0$ since $h_n$ does.

    And so
    \[ \abs{g_n-g} \leq \abs{f(w)}\cdot\frac1{\abs{w-a}^{k+1}\abs{w-a-h_n}^k}\cdot h'_n \]
    We can bound $\abs{f(w)}$ since $\abs f$ is continuous and $C$ is closed and bounded, and therefore compact, so $\abs f$ takes a maximum value on $C$, let it be $B$.
    And since $a\in\tilde C$ which is open and $w\in C$ which is the boundary of $\tilde C$, $\abs{w-a}$ must take a minimum value $\alpha>0$ over all $w\in C$ (take a ball of radius $\alpha>0$ about
    $a$ contained in $C$, which must exist as it is open).
    And since
    \[ \abs{w-a-h_n} \geq \abs{w-a} - \abs{h_n} \geq \alpha - \abs{h_n} \]
    we can assume $\abs{h_n}\leq\frac\alpha2$ and so $\abs{w-a-h_n}\geq\frac\alpha2$.
    And so we have a bound inpdenent of $w$:
    \[ \abs{g_n-g} \leq B\cdot\frac{2^k}{\alpha^{2k+1}}\cdot h'_n \]
    And this bound converges to $0$ as $n\longto\infty$, so $g_n\longvarrightarrows g$ as required.

    This means that
    \[ \frac{(k-1)!}{2\pi i}\int_C g_n\,dz \longto \frac{(k-1)!}{2\pi i}\int_C g\,dz \]
    And we recall that
    \[ \frac{(k-1)!}{2\pi i}\int_C g_n\,dz = \frac{f^{(k-1)}(a+h_n) - f^{(k-1)}(a)}{h_n},\qquad \frac{(k-1)!}{2\pi i}\int_C g\,dz = \frac{k!}{2\pi i}\int_C \frac{f(w)}{(w-a)^{k+1}}\,dz \]
    Since this is true for every $h_n\longto0$, this means that
    \[ f^{(k)}(a) = \lim_{h\to0} \frac{f^{(k-1)}(a+h) - f^{(k-1)}(a)}h = \frac{k!}{2\pi i}\int_C \frac{f(w)}{(w-a)^{k+1}}\,dz \]
    as required.

\eblank

\bexerc

    Let $f$ be an entire function bound by $M$ on the circle $\abs z=R$.
    \benum
        \item Show that the coefficients $c_k$ in $f$'s Taylor series about $0$ satisfy
        \[ \abs{C_k} \leq \frac M{R^k} \]
        \item Show that for the polynomial $p(z)=\sum_{k=0}^n c_kz^k$ bound by $1$ on the open disk $D_1(0)$, every coefficient $c_k$ is bound by $1$.
    \eenum

\eexerc

\bblank

    \benum
        \item We know that by Cauchy's integral theorem:
        \[ f^{(k)}(0) = \frac{k!}{2\pi i}\int_{\abs z=R}\frac{f(z)}{z^{k+1}}\,dz \]
        since the function $\frac{f(z)}{z^{k+1}}$ is bound by $\frac M{R^{k+1}}$ on $\abs z=R$, we get that
        \[ \abs{f^{(k)}(0)} \leq \frac{k!}{2\pi}\cdot\frac M{R^{k+1}}\cdot\int_{\abs z=R}\abs{dz} = k!\cdot\frac M{R^k} \]
        since the length of the curve $\abs z=R$ is $2\pi R$.
        And since
        \[ c_k = \frac{f^{(k)}(0)}{k!} \implies \abs{c_k} \leq \frac M{R^k} \]
        as required.

        \item For every $R<1$, $p$ is bound by $1$ on the circle $\abs z=R$, and since the coefficients of the Taylor series are precisely $c_k$, we get
        \[ \abs{c_k} \leq \frac1{R^k} \]
        for every $R<1$.
        And since $R<1$ is arbitrary, this means that $\abs{c_k}\leq1$ as required.
    \eenum

\eblank

\bexerc

    $f$ is an entire function such that $\abs{f(z)}\leq A+B\abs z^{3/2}$ for every $z$.
    Show that $f$ is a linear polynomial.

\eexerc

\bblank

    Let $z_0\in\bC$ and $R>0$, set $C_R=\set{z\in\bC}[\abs{z-z_0}=R]$.
    Then
    \[ f''(z_0) = \frac1{\pi i}\int_{C_R}\frac{f(z)}{(z-z_0)^3}\,dz \]
    Thus
    \[ \abs{f''(z_0)} \leq \frac1{\pi}\max_{z\in C_R}\frac{\abs{f(z)}}{\abs{z-z_0}^3}\cdot2\pi R \leq \frac2{R^2}\cdot\max_{z\in C_R}A+B\abs{z}^{3/2} \]
    Since $R=\abs{z-z_0}\geq\abs z-\abs{z_0}$, we have $\abs z\leq \abs{z_0}+R$ and so
    \[ \abs{f''(z_0)} \leq \frac{2\bigl(A+B(\abs{z_0}+R)^{3/2}\bigr)}{R^2} \]
    As $R\longto\infty$ this goes to $0$, so $f''=0$ and therefore $f(z)=a+bz$ as required.

\eblank

\bexerc

    Let $f$ be an entire function where $\abs{f'(z)}\leq\abs z$ for every $z$.
    Show that $f(z)=a+bz^2$ where $\abs b\leq\frac12$.

\eexerc

\bblank

    Let $z_0\in\bC$ and let $R>0$ and $C_R=\set{z\in\bC}[\abs{z-z_0}=R]$.
    Then
    \[ f''(z_0) = \frac1{2\pi i}\int_{C_R}\frac{f'(x)}{(z-z_0)^2}\,dz \]
    Thus
    \[ \abs{f''(z_0)} \leq \frac1{2\pi}\max_{z\in C_R}\frac{\abs{f'(x)}}{\abs{z-z_0}^2}\cdot2\pi R \leq \max_{z\in C_R}\frac{\abs z}R \]
    Since $R=\abs{z-z_0}\geq\abs z-\abs{z_0}$, we have $\abs z\leq R+\abs{z_0}$ for every $z\in C_R$ so
    \[ \abs{f''(z_0)} \leq \frac{\abs{z_0} + R}R = 1 + \frac{\abs{z_0}}R \]
    And as $R\longto\infty$ this approaches $1$, so $f''(z_0)$ is bounded and entire and therefore constant.

    So $f''(z)=2b$ so $f'(z)=c+2bx$, and since $\abs{f'(0)}=0$ this means that $c=0$.
    And so $f(z)=a+bx^2$.
    Since $\abs{f'(1)}\leq1$, this means that $\abs{2b}\leq1$ so $\abs{b}\leq\frac b2$ as required.


\eblank

\bexerc

    Prove that no non-constant entire function can satisfy that for all $z\in\bC$, $f(z+1)=f(z)$ and $f(z+i)=f(z)$.

\eexerc

\bblank

    Suppose $f$ is an entire function which satisfies these conditions.

    Let $m,n\in\bZ$ and $z\in\bC$, then inductively we can see that $f(z+m)=f(z)$ for positive integers $m$, and for negative integers $f(z)=f(z+m-m)=f(z+m)$.
    And similarly $f(z+in)=f(z)$.
    Thus $f(z+m+in)=f(z)$.

    This means that $f(\bC)=f\bigl(\set{z\in\bC}[0\leq\Re z,\Im z\leq 1]\bigr)$ since for every $z\in\bC$, $z-\floor{\Re z}-i\floor{\Im z}$ is in the unit square and has the same image as $z$.
    But the unit square is closed and bounded, so it is compact.
    And since $f$ is continuous, it is therefore bounded on the unit square and therefore on all of $\bC$.
    Therefore $f$ is constant.

\eblank

\end{document}

