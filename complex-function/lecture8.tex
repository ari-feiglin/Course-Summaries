\documentclass[10pt]{article}

\usepackage{amsmath, amssymb, mathtools}
\usepackage[margin=1.5cm]{geometry}

\input pdfmsym
\input prettyprint
\input preamble

\pdfmsymsetscalefactor{10}
\initpps

\def\pmat#1{\begin{pmatrix} #1 \end{pmatrix}}

\let\divides=\mid
\newfunc{metric}\rho({})
\newfunc{metricc}\sigma({})
\newfunc{spa}{{\rm span}}(\vert)
\newfunc{diam}{{\rm diam}}(\vert)
\newfunc{proj}\pi({})
\newfunc{iproj}{\pi^{-1}}({})
\newfunc{cis}{{\rm cis}}({})
\newfunc{Re}{{\rm Re}}({})
\newfunc{Im}{{\rm Im}}({})
\newfunc{sup}{{\rm sup}}\{\vert\}
\newfunc{arg}{{\rm arg}}({})

\font\bigbf = cmbx12 scaled 2000
\@undervecc@def{underbar}\@linecap\@linecap

\def\mO{{\cal O}}
\def\mU{{\cal U}}
\let\lineseg=\overleftrightvecc
\let\to=\varrightarrow
\let\longto=\longvarrightarrow
\let\ds=\displaystyle

\def\pdv#1#2{\frac{\partial #1}{\partial #2}}

\def\differ#1#2{\left.d#1\strut\right|_{#2}}

\def\qed{\hskip1cm\hbox{}\hfill$\blacksquare$}

\def\@ppmathcount{\thesection.\thepp@mathcount}

\begin{document}

\c@section=8

\barcolorbox{220, 255, 220}{0, 130, 0}{80, 200, 80}{
    \leftskip=0pt plus 1fill \rightskip=\leftskip
    {\bigbf Complex Functions}

    \medskip
    \textit{Lecture \thesection, Wednesday June 7, 2023}

    \textit{Ari Feiglin}
}

\bigskip

\begin{thrm*}[moreraTheorem,Morera's\ Theorem]

    Let $f$ be a continuous function in an open set $D$, if for every $\Gamma$ which is a boundary of a rectangle in $D$, we have
    \[ \int_\Gamma f\,dz = 0 \]
    then $f$ is analytic in $D$.

\end{thrm*}

\begin{proof}

    Let $z_0\in D$ and take an $\epsilon>0$ small enough such that $D_\epsilon(z_0)\subseteq D$, then for every $z\in D_\epsilon(z_0)$ define
    \[ F(z) = \int_{z_0}^z f(w)\,dw \]
    where the path chosen is $[z_0,\Re(z)+i\Im(z_0)]+[\Re(z)+i\Im(z_0),z]$.
    We have shown already that $F'(z)=f(z)$ for every $z$ in the disk, and thus $F$ is analytic in the disk.
    Since the derivative of an analytic function is analytic, $f$ is analytic.
    \qed

\end{proof}

Recall the following theorem from Calculus $3$:

\begin{thrm*}[fubiniToneliTheorem,Fubini\ Toneli\ Theorem]

    Suppose $f(x,y)$ is a real function continuous in the rectangle $R=[a_1,b_1]\times[a_2,b_2]\subseteq\overline\bR^2$ ($\overline\bR=\bR\dcup\set{\pm\infty}$), then
    \[ \int_{a_1}^{b_1}\parens{\int_{a_2}^{b_2}\abs{f(x,y)}\,dy}\,dx = \int_{a_2}^{b_2}\parens{\int_{a_1}^{b_1}\abs{f(x,y)}\,dx}\,dy = \iint_R\abs{f(x,y)}\,dxdy \]
    and if these integrals are finite then
    \[ \int_{a_1}^{b_1}\parens{\int_{a_2}^{b_2}f(x,y)\,dy}\,dx = \int_{a_2}^{b_2}\parens{\int_{a_1}^{b_1}f(x,y)\,dx}\,dy = \iint_Rf(x,y)\,dxdy \]

\end{thrm*}

So for example let us define
\[ f(z) = \int_0^\infty \frac{e^{zt}}{t+1}\,dt \]
for $z=x+iy$ where $x<0$.
It can be shown that $f$ is continuous.
Note that
\[ \abs{f(z)} \leq \int_0^\infty \frac{\abs{e^{zt}}}{t+1}\,dt = \int_0^\infty\frac{e^{xt}}{t+1}\,dt \leq \int_0^\infty e^{xt}\,dt = \frac{e^{xt}}x\Bigl|_0^\infty = -\frac1x \]
thus the integral converges absolutely, and thus converges.
So $f$ is well-defined.

We will show that $f$ is analytic in its domain $D$ by \ppref{moreraTheorem}.
Let $\Gamma$ be the boundary of a rectangle in $D$, then
\[ \int_\Gamma\abs{f(z)}\,dz = \int_\Gamma\parens{\int_0^\infty\frac{e^{xt}}{t+1}\,dt}\,dz \leq \int_\Gamma-\frac1{\Re z}\,dz \]
which is finite, and thus by \ppref{fubiniToneliTheorem} we can swap the order of integration and have
\[ \int_\Gamma f(z)\,dz = \int_0^\infty\int_\Gamma\frac{e^{zt}}{t+1}\,dzdt = \int_0^\infty0\,dt = 0\]
where the second equality is due to Cauchy's theorem since $\frac{e^{zt}}{t+1}$ is analytic within $\Gamma$.

Thus by \ppref{moreraTheorem}, we have shown that $f$ is analytic in its domain.

\begin{prop*}

    Let $\set{f_n}_{n=1}^\infty$ be a sequence of analytic functions in a domain $D$.
    And for every compact $K\subseteq D$, 
    \[ \sup_{x\in K}\abs{f_n-f}\longto 0 \]
    for some complex function $f$ on $D$ (ie. $f_n\varrightarrows f$ on every compact subspace of $D$), then $f$ is analytic on $D$.

\end{prop*}

\begin{proof}

    Let $z_0\in D$ then there exists a compact set $K_0$ such that $z_0\in K_0\subseteq D$ and in $K_0$ $f$ is the uniform limit of $f_n$, and since $f_n$ are continuous at $z_0$, $f$ is continuous at $z_0$.
    Thus $f$ is continuous in $D$.
    Let $\Gamma$ be the boundary of some rectangle $R$ in $D$ then since every rectangle is contained within some compact subspace of $D$, suppose $R\subseteq K_R$ for $K_R\subseteq D$ compact.
    Then since $f_n\varrightarrows f$ in $K_R$, we have
    \[ \int_\Gamma f(z)\,dz = \int_\Gamma \lim f_n(z)\,dz = \lim\int_\Gamma f_n(z)\,dz = \lim 0 = 0 \]
    where the third equality is due to Cauchy.
    Thus by \ppref{moreraTheorem}, $f$ is analytic in $D$.
    \qed

\end{proof}

\begin{prop*}

    Suppose $f$ is continuous in an open set $D$ then suppose $L$ is a line in $D$.
    If $f$ is analytic in $D\setminus L$ then $f$ is analytic in $D$.

\end{prop*}

\begin{proof}

    We will assume that $L\subseteq\bR$ since $L$ is of the form $z(t)=(x_0,y_0) + t(a,b)$ for $(a,b)=e^{i\theta}$.
    Let us look at $e^{-i\theta}(z(t)-x_0-iy_0) = te^{-i\theta}(a+ib)=t$.
    Thus we can instead look at $\tilde f(z)=f(e^{-i\theta}(z-x_0-iy_0))$ which is just a shift and stretch of $f$, and thus the transform of $L$ is $\tilde L=e^{-i\theta}(L-x_0-iy_0)\subseteq\bR$.

    Let $z_0\in L$.
    We will show that $f$ is analytic in $D_r(z_0)$ for some $r>0$.
    Note that $D_r(z_0)\cap L=(z_0-r,z_0+r)$.
    Let $\Gamma$ be a rectangle $R$'s boundary in $D_r(z_0)$.
    If $\varnothing=\overline R\cap L$ then $\overline R\subseteq D\setminus L$ and so $f$ is analytic in $\overline R$ and so by Cauchy $\int_\Gamma f=0$ by Cauchy.

    Otherwise if one of $R$'s edges in on $L$, let us define a new rectangle $R_t$ with boundary $\Gamma_t$, which shares its edges with $R$ other than the edge touching $L$, which we move away by $t$ units.
    Since $f$ is continuous on $\overline R$, which is compact, $f$ is uniformly continuous on $\overline R$.
    Let
    \[ \Delta = \int_\Gamma f - \int_{\Gamma_t}f \]
    this is the integral of $f$ over the path which goes over the remaining edges of $R$ not removed by $R_t$ plus the reversed edge on $R_t$ parallel to $L$, 

\end{proof}

\subsection{Branches of the logarithm}

We'd like to find a continuous inverse to $e^z$.
But $e^z$ is not surjective, its image is $\bC\setminus\set0$, and it is periodic: $e^{z+2\pi ik}=e^z$.
Thus we'd like to find a function such that
\[ e^{\log z} = z \]
for every $0\neq z\in\bC$.

Suppose $\log=u+iv$ then suppose $re^{i\theta}$
\[ e^{u(z)+iv(z)} = e^{u(z)}e^{iv(z)} = re^{i\theta} \]
if and only if $e^{u(z)}=r$ and $v(z)-\theta=2\pi k$.
Thus we can define $u(z)=\log\abs z=\log r$ and $v(z)=\argof z+2\pi k=\theta+2\pi k$.
If $z\neq0$ then $r>0$ so this is well-defined for any choice of $k$.
So all in all we have
\[ \log(z) = \log(re^{i\theta}) = \log r+i\theta + i2\pi ik \]

Suppose we have a circle $\abs z=R$, parameterized by $\gamma(\theta)=Re^{i\theta}$ then
\[ \log(\gamma(0)) = \log R + 2\pi ik \]
Despite this
\[ \lim_{\theta\to2\pi^-}\log(\gamma(\theta)) = \lim(\log R+i\theta+i2\pi ik) = \log R + i2\pi(k+1) \neq \log(\gamma(0)) \]
Thus the function we've defined is not continuous on any canonical circle.

\begin{defn*}

    We say that $f(z)$ is an \ppemph{analytic branch} of $\log(z)$ on a domain $D$ if $f$ is analytic on $D$ and for every $z\in D$, $e^{f(z)}=z$.

\end{defn*}

\begin{lemm*}

    If $f$ iz an analytic branch of $\log$ on $D$ then for every $k\in\bZ$, so is $g(z)=f(z)+2\pi ik$.

\end{lemm*}

This is trivial.

\begin{thrm*}

    Let $D\subseteq\bC$ be a simple domain where $0\notin D$.
    For every choice of $z_0\in D$ and choice of $\log(z_0)$ define
    \[ f(z) = \int_{z_0}^z\frac1w\,dw + \log(z_0) \]
    where the integral is integrated over a smooth curve in $D$, is an analytic branch of $\log$ in $D$.

\end{thrm*}

\begin{proof}

    Since $0\notin D$, $\frac1w$ is analytic.
    Since the choice of smooth curve does not matter in simple domains $f$ is well-defined.
    We have shown that
    \[ F(z) = \int_{z_0}^z\frac1w\,dw \]
    is an antiderivative of $\frac1z$ in $D$ and thus so is $f(z)$, so $f$ is analytic in $D$.

    Let us define
    \[ g(z) = ze^{-f(z)} \]
    then $g$'s derivative is
    \[ g'(z) = e^{-f(z)}(1+z(-f'(z))) = e^{-f(z)}\parens{1-z\cdot\frac1z} = 0 \]
    and so $g$ is constant, suppose $g\equiv c$.
    For $z=z_0$ we get that
    \[ g(z_0) = z_0e^{-f(z_0)} = z_0e^{-\log(z_0)} = c \]
    if and only if $z_0=ce^{\log(z_0)}$ if and only if $z_0=cz_0$ so $c=1$.
    Thus $g\equiv 1$ and so
    \[ ze^{-f(z)} = 1 \implies e^{f(z)} = z \]
    for every $z\in D$ as required.

\end{proof}

\begin{lemm*}

    If $f_1$ and $f-2$ are two analytic branches of $\log$ in $D$ then $f_1(z)-f_2(z)=2\pi ik$ for every $z\in D$ for some $k\in\bZ$.

\end{lemm*}

\begin{proof}

    Notice that
    \[ e^{f_1(z)-f_2(z)} = \frac{e^{f_1(z)}}{e^{f_2(z)}} = \frac zz = 1 \]
    Recall $e^u=1$ if and only if $u=2\pi ik$ for some $k\in\bZ$.
    \qed

\end{proof}

\begin{defn*}

    For an analytic function $G\colon V\longto V$ surjective, we define an analytic branch of $G^{-1}$ on a domain $V_1\subseteq V$ as an analytic function $g\colon V_1\longto V$ such that for every
    $z\in V_1$, $G(g(z))=z$.

\end{defn*}

For example take $G(z)=z^2$ for $V=\bC$ and then for an analytic branch of $\log$ on a domain $0\notin D$, $f$, we can define
\[ g(z) = \exp\parens{\frac12 f(z)} \]
and this defines a branch of $G^{-1}$:
\[ G(g(z)) = \exp(f(z)) = z \]

\begin{prop*}

    For every simply connected domain $D$ where $0\notin D$, $\sqrt z$ has two branches given by the formula above.

\end{prop*}

\begin{proof}

    Let $D$ be such a domain, and $f$ a branch of $\log$.
    Since every other branch of $\log$ is of the form $f_k=f+2\pi ik$, we have
    \[ g_k(z) = \exp\parens{\frac12 f_k(z)} = \exp\parens{\frac12f(z)+\pi ik} = \exp\parens{\frac12f(z)}\exp(\pi ik) \]
    so all $k$ even give the same $g_k$ and all $k$ odd give the same $g_k$.
    \qed

\end{proof}

\end{document}

