\documentclass[10pt]{article}

\usepackage{amsmath, amssymb, mathtools}
\usepackage[margin=1.5cm]{geometry}

\font\tensy=cmsy10
\newfam\scrfam
\textfont\scrfam=\tensy
\def\mathscr#1{{\fam=\scrfam#1}}

\input pdfmsym
\input prettyprint
\input preamble

\pdfmsymsetscalefactor{10}
\initpps
\newmathpp{conj}{Conjecture}{255,130,130}{150,0,0}{200,0,0}

\@Arrow@def{varLeftRightarrow}\@Larrow\@Rarrow{1}
\def\implies{\,\longvarRightarrow\,}
\def\iff{\,\longvarLeftRightarrow\,}
\let\eiff=\varLeftRightarrow
\let\to=\varrightarrow
\let\longto=\longvarrightarrow
\let\oto=\varleftrightarrow
\let\injection=\longvaruphookrightarrow

\def\PF{\mathbf{PF}}
\def\PC{\mathbf{PC}}
\def\P{\mathbf{P}}
\def\NP{\mathbf{NP}}

\def\bB{\mathbb{B}}
\def\mA{\mathcal{A}}
\def\mL{\mathcal{L}}
\def\mT{\mathcal{T}}
\def\mM{\mathcal{M}}
\def\mN{\mathcal{N}}
\def\mD{\mathcal{D}}
\def\fM{\mathfrak{M}}
\def\fC{\mathfrak{C}}
\def\fm{\mathfrak{m}}
\def\fn{\mathfrak{n}}

\def\qed{\hskip.5cm\hbox{}\hfill$\blacksquare$}

\def\pmat#1{\begin{pmatrix} #1 \end{pmatrix}}

\def\mB{\mathbb{B}}
\def\true{\mathsf{true}}
\def\false{\mathsf{false}}

\let\divides=\mid

\font\bigbf = cmbx12 scaled 2000

\parskip=5pt plus 1pt minus 3pt
\pp@headerskip=\z@
\def\@ppmathcount{\thesection.\thepp@mathcount}

\begin{document}

\c@section=1

\barcolorbox{255, 220, 220}{130, 0, 0}{200, 80, 80}{
    \leftskip=0pt plus 1fill \rightskip=\leftskip
    {\bigbf Computability and Complexity}

    \medskip
    \textit{Recitation \thesection, Tuesday August 1, 2023}

    \textit{Ari Feiglin}
}

\bigskip

\begin{defn}

    If $G$ is a graph, a set of vertices $S$ is \ppemph{independent} if for every $u,v\in S$, $(u,v)\notin S$.

\end{defn}

\def\Rbigis{R_{\text{BigIS}}}
\begin{exercise*}

    We define the following search problem
    \[ \Rbigis = \set{(G,S)}[\text{$S$ is independent and $\abs S\geq\abs V-20$}] \]
    Our goal is to show $\Rbigis\in\PF$.

\end{exercise*}

Firstly, $\Rbigis$ is polynomially bound since if $(G,S)\in\Rbigis$ then $\abs S\leq\abs G$.
And a simple algorithm for this is to iterate over every set of vertices of size $\leq20$, and look at their complement and check if it is independent.
If it is, return it.
Otherwise if we have finished iterating, return $\perp$.

The number of subsets of $G$ of size $20$ is
\[ \binom{\abs G}{20} = \binom n{20} = \frac{n!}{(n-20)!\cdot20!} \in \Theta(n^{20}) \]
And the time it takes to iterate over each set is $O(\abs V^2\cdot\abs E)$ (iterate over every pair $u$ and $v$, and check if they are neighbors).
Thus this algorithm runs in $O(n^{23})$ time, as required.

\def\Ris{R_{\text{IS}}}
\begin{exercise*}

    Let us define
    \[ \Ris = \set{\bigl((G,k),S\bigr)}[\text{$S$ is independent in $G$ and larger than $k$}] \]
    Is $\Ris\in\PF$?
    Is $\Ris\in\PC$?

\end{exercise*}

\benum
    \item Notice that the algorithm from the previous question iterates over $\binom{\abs V}k$ sets.
    But if $k=\frac{\abs V}2$, the time complexity of the algorithm is greater than $2^{\frac{\abs V}2}$, which is exponential.
    Thus the na\"ive algorithm will not work here.
    But this does not prove that $\Ris\notin\PF$, because there may be another algorithm which does solve $\Ris$.
    The question of the existence of such an algorithm is an open problem.

    \item Obviously if we are given $((G,k),S)$ then we can check if $S$ is independent and larger than $k$, which will run in polynomial time, and this verifies the solution.
    Thus $\Ris\in\PC$.
\eenum

\def\comp{\mathsf{Composite}}
\begin{exercise*}

    Let us define the following decision problem:
    \[ \comp = \set{n\in\bN}[n\text{ is reducible (not prime)}] \]
    Show that $\comp\in\NP$.

\end{exercise*}

This is quite simple, let us define $V(x,y)$ to return $1$ if and only if $y$ divides $x$ and $y\neq1$ and $y\neq x$.
Thus there exists a $y$ such that $V(x,y)=1$ if and only if $x$ is not prime, so $V$ is a verifier for $\comp$.
It runs in polynomial time (worst case if you don't want to think about how to define division, just iterate over $i\leq x$ and check if $iy=x$).
And if $V(x,y)=1$ then $y\leq x$, so $V$ is polynomially bound.
Thus $V$ is a polynomial proof system for $\comp$, and therefore $\comp$ is in $\NP$.

Alternatively we could define $V(n,(k_1,k_2))$ which returns $1$ if and only if $1<k_1,k_2<n$ and $n=k_1k_2$.
Then $V$ verifies $\comp$, and it runs in polynomial time, and if $V(n,(k_1,k_2))=1$ then $\abs{(k_1,k_2)}=k_1+k_2\leq 2n$ (storing values as unary), so it is polynomially bound.
Thus we have defined another polynomial proof system for $\comp$.

\begin{exercise*}

    Suppose $S_1,S_2\in\NP$ and $V_1$ and $V_2$ are polynomial proof systems of $S_1$ and $S_2$ respectively.
    Prove or disprove:
    \benum
        \item We define an algorithm $V$ where given an input $(x,y)$, it runs $V_1(x,y)$ and $V_2(x,y)$ and returns $1$ if and only if one returns $1$.
        Then $V$ is a polynomial proof system for $S_1\cup S_2$.
        \item We define an algorithm $V$ where given an input $(x,y)$, it runs $V_1(x,y)$ and $V_2(x,y)$ and returns $1$ if and only if both return $1$.
        Then $V$ is a polynomial proof system for $S_1\cap S_2$.
    \eenum

\end{exercise*}

\benum
    \item This is true.
    $V$ obviously is a verifier for $S_1\cup S_2$ since if $x\in S_1\cup S_2$ then $x$ is in $S_1$ or $S_2$, and so there exists a $y$ such that $V_1(x,y)=1$ or $V_2(x,y)=1$.
    Thus $V(x,y)=1$.
    Otherwise $x\notin S_1\cup S_2$ so $x$ is not in $S_1$ or in $S_2$ and so for every $y$, $V_1(x,y)=0$ and $V_2(x,y)=0$ so $V(x,y)=0$.
    So $V$ is indeed a verifier for $S_1\cup S_2$.

    Now, $V$ runs in polynomial time since suppose the time complexity of $V_1$ is bound by $p_1$ and the time complexity of $V_2$ is bound by $p_2$, then $V$ is bound by $p_1+p_2+c$ (where $c$ is the number
    of transitions to compare the results of $V_1$ and $V_2$ which will be constant).
    This is a polynomial, so $V$ runs in polynomial time.

    $V$ is also polynomially bound, since if $V(x,y)=1$ then $V_1(x,y)=1$ or $V_2(x,y)=1$ and so either $\abs y\leq q_1(\abs x)$ or $\abs y\leq q_2(\abs x)$ for the polynomial bounds $q_1$ and $q_2$ for
    $V_1$ and $V_2$.
    And so $\abs y\leq q_1(\abs x)+q_2(\abs x)$ which is a polynomial.
    Thus $V$ is polynomially bound.

    So $V$ is a polynomial proof system for $S_1\cup S_2$ as required.

    \item This is false, take $V_1$ and $V_2$ to be the two verifiers for $\comp$ that we defined above.
    Then $S_1=S_2=\comp$ so $S_1\cap S_2=\comp$.
    But $V$ would not accept anything, since the types of inputs they accept are different (the first accepts a number, the second accepts a pair of numbers as witnesses).
    So $V$ does not verify $S_1\cap S_2$.

    Another counter example can be constructed as followed: we define a new problem $S_1=S_2=\set{0,1}^*$ (the set of all the binary strings) and $V_1(x,y)=1$ if and only if $y=1$ and $V_2(x,y)=1$ if and
    only if $y=0$.
    Then $V_1$ and $V_2$ are polynomial proof systems for $S_1=S_2$, but $V$ would similarly not accept anything and therefore not be a verifier for $S_1\cap S_2$.
\eenum

\begin{exercise*}

    Show that if $S\in\NP$, then there exist two verifiers (I will use verifier here to mean polynomial proof system) $V_1$ and $V_2$ such that
    \[ V_1(x,y) = 1 \implies V_2(x,y) = 0 \]
    and
    \[ V_2(x,y) = 1 \implies V_1(x,y) = 0 \]

\end{exercise*}

Notice that $V_1(x,y)=1\implies V_2(x,y)=0$ is equivalent to its contrapositive, which is $V_2(x,y)=1\implies V_1(x,y)=0$.
So we need only to prove $V_1(x,y)=1\implies V_2(x,y)=0$.

We know there exists a verifier $V$ for $S$, then let us define $V_1$ which takes as input $(x,(b,y))$ and returns $1$ if and only if $b=1$ and $V(x,y)=1$.
And we defin $V_2$ to take as input $(x,(b,y))$ and returns $1$ if and only if $b=0$ and $V(x,y)=1$.

Then if $x\in S$ then there exists a $y$ such that $V(x,y)=1$ and so $V_1(x,(1,y))=V_2(x,(0,y))=1$.
And if $x\notin S$ then for every $y$, $V(x,y)=0$ so $V_1(x,(b,y))=V_2(x,(b,y))=0$.
So these verify $S$, and are also obviously polynomially bound (since $\abs{(b,y)}=\abs y+1\leq p(\abs x)+1$) and run in polynomial time.

Now, if $V_1(x,(b,y))=1$ then $b=1$ and so $V_2(x,(b,y))=0$ as required.

\end{document}

