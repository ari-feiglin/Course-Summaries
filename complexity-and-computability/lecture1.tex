\documentclass[10pt]{article}

\usepackage{amsmath, amssymb, mathtools}
\usepackage[margin=1.5cm]{geometry}

\font\tensy=cmsy10
\newfam\scrfam
\textfont\scrfam=\tensy
\def\mathscr#1{{\fam=\scrfam#1}}

\input pdfmsym
\input prettyprint
\input preamble

\pdfmsymsetscalefactor{10}
\initpps
\newmathpp{conj}{Conjecture}{255,130,130}{150,0,0}{200,0,0}

\@Arrow@def{varLeftRightarrow}\@Larrow\@Rarrow{1}
\def\implies{\,\longvarRightarrow\,}
\def\iff{\,\longvarLeftRightarrow\,}
\let\eiff=\varLeftRightarrow
\let\to=\varrightarrow
\let\longto=\longvarrightarrow
\let\oto=\varleftrightarrow
\let\injection=\longvaruphookrightarrow

\let\nor=\downarrow
\let\nand=\uparrow

\def\aone#1#2{#2\to(#1\to#2)} \def\Aone{\textbf{A1}}
\def\atwo#1#2#3{\bigl(#1\to(#2\to#3)\bigr)\to\bigl((#1\to#2)\to(#1\to#3)\bigr)} \def\Atwo{\textbf{A2}}
\def\athree#1#2{(\neg#2\to\neg#1)\to\bigl((\neg#2\to#1)\to#2\bigr)} \def\Athree{\textbf{A3}}
\def\afour#1#2#3{(\forall #2#1)\to#1\frac{#3}{#2}}  \def\Afour{\textbf{A4}}
\def\afive#1#2#3{\bigl(\forall #3(#1\to#2)\bigr)\to\bigl(#1\to(\forall#3#2)\bigr)} \def\Afive{\textbf{A5}}
\def\MP{\textit{MP}} \def\Gen{\textit{Gen}}

\def\beqq{\mathrel{\vbox{\hbox{$=$}\kern-\baselineskip\kern.3pt\hbox{$=$}}}}

\def\PF{\mathbf{PF}}
\def\PC{\mathbf{PC}}
\def\P{\mathbf{P}}
\def\NP{\mathbf{NP}}

\def\bB{\mathbb{B}}
\def\mA{\mathcal{A}}
\def\mL{\mathcal{L}}
\def\mT{\mathcal{T}}
\def\mM{\mathcal{M}}
\def\mN{\mathcal{N}}
\def\mD{\mathcal{D}}
\def\fM{\mathfrak{M}}
\def\fC{\mathfrak{C}}
\def\fm{\mathfrak{m}}
\def\fn{\mathfrak{n}}

\def\Var{\mathrm{Var}}

\newfunc{var}{{\rm var}}({})
\newfunc{bound}{{\rm bound}}({})
\newfunc{free}{{\rm free}}({})
\newfunc{theory}{{\rm Th}}({})
\newfunc{Diag}{{\rm Diag}}({})
\newfunc{elDiag}{{\rm Diag}_{\rm el}}({})
\newfunc{depth}{{\rm depth}}({})

\def\qed{\hskip.5cm\hbox{}\hfill$\blacksquare$}

\def\pmat#1{\begin{pmatrix} #1 \end{pmatrix}}

\def\mB{\mathbb{B}}
\def\true{\mathsf{true}}
\def\false{\mathsf{false}}

%\def\Aone#1#2{#1\to(#2\to#1)}
%\def\Atwo#1#2#3{\bigl(#1\to(#2\to#3)\bigr)\to\bigl((#1\to#2)\to(#1\to#3)\bigr)}
%\def\Athree#1#2{(\neg#2\to\neg#1)\to\bigl((\neg#2\to#1)\to#2\bigr)}

\let\divides=\mid

\font\bigbf = cmbx12 scaled 2000

\parskip=5pt plus 1pt minus 3pt
\pp@headerskip=\z@
\def\@ppmathcount{\thesection.\thepp@mathcount}

\begin{document}

\c@section=1

\barcolorbox{255, 220, 220}{130, 0, 0}{200, 80, 80}{
    \leftskip=0pt plus 1fill \rightskip=\leftskip
    {\bigbf Computability and Complexity}

    \medskip
    \textit{Lecture \thesection, Tuesday August 1, 2023}

    \textit{Ari Feiglin}
}

\bigskip

There are two main problems which we will focus on in this course: search problems, and decision problems.
Search problems are problems where our goal is to find a solution or an answer to a question.
Decision problems are problems where our goal is to verify if a given solution or answer to a question is valid (more generally if we should accept or reject a given input).

Examples of search problems are:
\benum
    \item Given a graph $G$ and two vertices $s$ and $t$, find a path from $s$ to $t$ in $G$.
    \item Given a graph $G$, find a three-coloring of the vertices (ie. a function $\sigma\colon V\longto\set{1,2,3}$ such that if $(v,u)\in E$ then $\sigma(v)\neq\sigma(u)$).
    \item Given a boolean formula in conjunctive normal form (of the form $\bigwedge_{i=1}^n\bigvee_{j=1}^m\epsilon_{ij}x_j$ where $\epsilon_{ij}$ is either $\neg$ or nothing), find valuations for the
    variables $x_1,\dots,x_m$ which satisfies the boolean formula.
\eenum
Search problems need not give solutions, if they do not exist.

\begin{defn*}

    A \ppemph{search problem} is specified by a relation $R$, where $(x,y)\in R$ if and only if $y$ is a valid solution to $x$.

\end{defn*}

\begin{defn*}

    If $R$ is a relation, and $x$ is an element of its domain, then we define
    \[ R(x) = \set{y}[(x,y)\in R] \]

\end{defn*}

So in the above examples,
\benum
    \item For the finding paths problem, we have the relation
    \[ R_{st\text{-conn}} = \set{\bigl((G,s,t),P\bigr)}[\text{$G$ is a graph, $s$ and $t$ are vertices, and $P$ is a path from $s$ to $t$ in $G$}] \]
    (conn is short for connectivity.)
    \item For the three coloring problem, we have the relation
    \[ R_{3\text{col}} = \set{(G,\sigma)}[\text{$G$ is a graph, and $\sigma$ is a three-coloring of $G$}] \]
    (col is short for coloring.)
    \item For finding valuations for a CNF, we have the relation
    \[ R_{\mathit{CNF}\text{-sat}} = \set{(\phi,\vec x)}[\text{$\phi$ is a boolean formula in CNF, and $\vec x$ is a boolean vector which satisfies $\phi$}] \] 
    (CNF is short for conjunctive normal form, and sat is short for satisfiability.)
\eenum

\begin{defn*}

    A \ppemph{decision problem} is specified by a set $S$ where $x\in S$ if and only if $x$ is a valid answer to the problem.

\end{defn*}

For example, we can ask the question ``is the graph $G$ three-colorable?''
This is specified by the set
\[ S_{3\text{col}} = \set{G}[\text{$G$ is three-colorable}] \]

\begin{defn*}

    An \ppemph{algorithm} is a finite set of rules which defines a process in a computational model.
    Given an algorithm $A$, and an input $x$, we define $A(x)$ to be result of running $A$ on the input $x$.

    We say that an algorithm $A$ solves a search problem $R$ if for every $x$,
    \benum
        \item If $R(x)\neq\varnothing$, then $A(x)\in R(x)$.
        \item And if $R(x)=\varnothing$, then $A(x)=\perp$ (the symbol for there being no solution).
    \eenum
    And we say that $A$ solves a deciscion problem $S$ if
    \benum
        \item If $x\in S$ then $A(x)=1$.
        \item And if $x\notin S$ then $A(x)=0$.
    \eenum

\end{defn*}

An algorithm may be the definition of a regular automaton, a pushdown automaton, a context-free grammar, a turing machine, etc.
But we can assume that an algorithm is a single-tape turing machine.

\begin{defn*}

    Let $M$ be a turing machine.
    We denote $t_M(x)$ by the number of transitions $M$ performs on the input $x$.

    The \ppemph{time complexity} of $M$ is defined to be the function
    \[ T_M\colon\bN\longto\bN,\quad T_M(n) = \maxof{t_M(x)}[\abs x=n] \]
    (recall that inputs to turing machines are finite strings.)

\end{defn*}

\begin{defn*}

    We say that a turing machine $M$ \ppemph{runs in polynomial time} if there exists a polynomial $p$ such that for every $n$, $T_M(n)\leq p(n)$.
    In other words, the time complexity of $M$ is bound by some polynomial.

    And a search or deciscion problem is \ppemph{solvable in polynomial time} if there exists a single-tape turing machine which solves it and runs in polynomial time.

\end{defn*}

We consider efficient solutions to be solutions which run in polynomial time.
Note that a solution which runs in $n^{100}$ time is still considered efficient for the sake of this course, but in practicality it is of course not.

\begin{note}

    A turing machine runs in polynomial time if and only if $T_M(n)\in O(n^d)$ for some $d\in\bN$.
    This is true since $T_M(n)\in O(n^d)$ if and only if there exists a $c>0$ such that for every $n\geq n_0$, $T_M(n)\leq cn^d$.
    If we let $A=\maxof{T_M(n)}[n<n_0]$, and then $T_M(n)\leq cn^d+A$, which means $M$ runs in polynomial time.

    And if $M$ runs in polynomial time, then $T_M(n)\leq\sum_{k=0}^d a_kx^k$, and so $T_M(n)\in O(n^d)$.

\end{note}

\begin{note}

    Notice that the requirement of the turing machine to be single-tape may be significant.
    For example, given the deciscion problem $\mathsf{Palindrome}=\set{x}[x^R=x]$, with a doube-taped turing machine we can solve it in $\Theta(n)$ time.
    We do this by copying $x$ onto the second tape, then comparing the characters at each head and moving them left and right respectively until they reach the end of the string.

    With a single-taped turing machine we need to move from the beginning to the end of the string over and over, and this takes $\Theta(n^2)$ time.

\end{note}

\begin{conj*}[cobhamEdmondsThesis,Cobham-Edmonds\ Thesis]

    Any problem which can be solved in $T(n)$ time on a ``feasible'' computational model can be solved in $p(T(n))$ time for some polynomial $p$ with a single-taped turing machine.

\end{conj*}

``Feasible'' here is not well-defined, but intuitively it means that the computational model doesn't do an extraordinary amount of computations at each step.

\newpage
So by the \ppref{cobhamEdmondsThesis}, for a problem to be solvable in polynomial time, it is sufficient to provide a solution to it with any (feasible) computational model.

\begin{defn*}

    We say that $R$ is \ppemph{polynomially bound} if there exists a polynomial $p$ such that for every $(x,y)\in R$ then $\abs y\leq p(\abs x)$.

    We define $\PF$ to be the class of all polynomially bound relations $R$ such that $R$ can be solved in polynomial time,
    \[ \PF = \set{R}[\text{$R$ is polynomially bound and $R$ can be solved in polynomial time}] \]
    $\PF$ is short for polynomial-find.

\end{defn*}

Note that $R_{st\text{-con}}$ is not necessarily in $\PF$, as it is not polynomially bound.
Since if there exists a cycle in $G$, then we can find arbitrarily large paths in $G$.
If we require that solutions be simple paths, then the length of the path is less than $\abs E$ and thus is polynomially bound.
And we know we there exist polynomial time algorithms to solve $R_{st\text{-con}}$, so it is in $\PF$.

The question of whether $R_{3\text{col}}\in\PF$ is open.
It is obviously polynomially bound (since solutions have a size of $n^3$), but we do not know if there exists a solution which runs in polynomial time or not.

\begin{defn*}

    We define $\PC$ to be the class of all polynomially bound search problems $R$ such that there exists an algorithm $A$ which runs in polynomial time and verifies solutions to $R$.
    \[ \PC = \set{R}[\vrule width\z@ height15\p@\substack{\hbox{$R$ is polynomially bound, and there exists an algorithm $A$ which runs in polynomial time}\\
    \hbox{such that for every $(x,y)$, $A(x,y)=1$ if and only if $(x,y)\in R$}}] \]
    $\PC$ is short for polynomial-check.

\end{defn*}

Intuitively, we may think that a problem which we can easily find a solution for (in $\PC$) would also be easy to verify a solution for.
But this is not the case, because to verify solutions we must deal with \emph{all} possible solutions, and to find a solution we need only find one.

\begin{prop*}

    $\PF\nsubseteq\PC$

\end{prop*}

\begin{proof}

    Let us define
    \begin{multline*}
        R = \set{\bigl((M,\omega),!\bigr)}[\text{$M$ is a turing machine which halts on the input $\omega$}]\\
        \cup\set{\bigl((M,\omega),?\bigr)}[\text{$M$ is a turing machine, and $\omega$ is any input}]
    \end{multline*}

    $R\in\PF$ since we can simply define the algorithm $A$ which returns $?$ on every input.

    Now, $R\notin\PC$ since if $R\in\PC$ then suppose $A$ is an algorithm where $A((M,\omega),\sigma)=1$ if and only if $((M,\omega),\sigma)\in R$.
    Then we can define the algorithm $B$ whose input is $(M,\omega)$ and returns $A((M,\omega),!)$.
    But then $B$ decides if $M$ halts on $\omega$, which contradicts the halting problem being undecidable.
    \qed

\end{proof}

Now, is $\PC\subseteq\PF$, ie. if we can easily verify a problem, can we solve it?
This is an open question.

\begin{defn*}

    $\PC$ and $\PF$ relate to search problems.
    For decision problems, we define the class $\P$ to be the class of all search problems $S$ which can be solved in polynomial time,
    \[ \P = \set{S}[\text{$S$ can be solved in polynomial time}] \]
    $\P$ is the equivalent of $\PF$ for deciscion problems.

\end{defn*}

But it is not immediately clear how to define the equivalent of $\PC$ for deciscion problems.
After all, how do you verify a solution to a deciscion problem?

\begin{defn*}

    A \ppemph{verifier} for a decision problem $S$ is an algorithm $V$ such that
    \benum
        \item $V$ is entire: for every $x\in S$, there exists a $y$ such that $V(x,y)=1$.
        \item $V$ is reliable: for every $x\notin S$ and for every $y$, $V(x,y)=0$.
    \eenum

    We say that $S$ has a \ppemph{polynomial proof system} if it has a verifier $V$ which runs in polynomial time and there exists a polynomial $p$ such that if $x\in S$ then there exists a $y$ such that
    $\abs y\leq p(\abs x)$ and $V(x,y)=1$.

\end{defn*}

Note that if $V$ is a verifier, then $V(x,y)=0$ does not mean $x\notin S$, it just means that $x$ may not be in $S$.
If we run $V(x,y)$ for every $y$ and we get zero, only then do we know that $x\notin S$.
Conversely, if $V(x,y)=1$ then $x\in S$.

\begin{defn*}

    We define $\NP$ to be the class of all decision problems which have a polynomial proof system,
    \[ \NP = \set{S}[\text{$S$ has a polynomial proof system}] \]
    $\NP$ is the parallel to $\PC$ for decision problems.

\end{defn*}

Note that $S_{3\text{col}}\in\NP$ since we can take the verifier $V$ which accepts $(G,\sigma)$ if and only if $\sigma$ is a three-coloring of $G$ (which is verifiable in linear time).
Since $\sigma$ has a size on the order of $n^3$, $S_{3\text{col}}$ has a polynomial proof system.

\begin{prop*}

    $\P\subseteq\NP$.

\end{prop*}

\begin{proof}

    Let $S\in\P$, suppose $A$ solves $S$.
    Then we define the algorithm $V$ where $V(x,y)=1$ if and only if $A(x)$.
    So if $x\in S$ then $A(x)=1$ and so $V(x,y)=1$ for every $y$.
    And if $x\notin S$, then $A(x)=0$ so for every $y$, $V(x,y)=0$.
    Thus $V$ is a polynomial proof system for $S$, meaning $S\in\NP$ as required.
    \qed

\end{proof}

Now comes the biggie question, does

\centerline{\scalebox{1.5}{$\P = \NP ?$}}
This is a major open question in computer science.

\end{document}

