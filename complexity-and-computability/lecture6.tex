\documentclass[10pt]{article}

\usepackage{amsmath, amssymb, mathtools}
\usepackage{tikz-cd}
\usepackage[margin=1.5cm]{geometry}

\font\tensy=cmsy10
\newfam\scrfam
\textfont\scrfam=\tensy
\def\mathscr#1{{\fam=\scrfam#1}}

\input pdfmsym
\input prettyprint
\input preamble

\pdfmsymsetscalefactor{10}
\initpps
\newmathpp{conj}{Conjecture}{255,130,130}{150,0,0}{200,0,0}

\@Arrow@def{varLeftRightarrow}\@Larrow\@Rarrow{1}
\def\implies{\,\longvarRightarrow\,}
\def\iff{\,\longvarLeftRightarrow\,}
\let\eiff=\varLeftRightarrow
\let\to=\varrightarrow
\let\longto=\longvarrightarrow
\let\oto=\varleftrightarrow
\let\injection=\longvaruphookrightarrow

\let\nor=\downarrow
\let\nand=\uparrow

\def\undersumtext#1{\vtop{\hsize=2cm\relax\centering\small #1}}

\newfunc{oracle}{\hbox{\tencsc oracle}}({})

\def\sps{\texttt{\char32}}
\def\sat{\mathsf{SAT}}
\def\is{\mathsf{IS}}
\def\threesat{\mathsf{3SAT}}
\def\clique{\mathsf{Clique}}
\def\vertcover{\textsf{Vertex-Cover}}
\def\domset{\textsf{Dom-Set}}
\def\dhp{\mathsf{DHP}}
\def\dhc{\mathsf{DHC}}
\def\hc{\mathsf{HC}}
\def\tsp{\mathsf{TSP}}
\def\threecolor{\mathsf{3Color}}
\def\kcolor{k\mathsf{Color}}
\def\Color{\mathsf{Color}}
\def\subsum{\mathsf{SubsetSum}}

\def\PF{\mathbf{PF}}
\def\PC{\mathbf{PC}}
\def\P{\mathbf{P}}
\def\NP{\mathbf{NP}}
\def\NPH{\mathbf{NPH}}
\def\NPC{\mathbf{NPC}}
\def\coNP{\mathbf{coNP}}

\def\bB{\mathbb{B}}
\def\mA{\mathcal{A}}
\def\mL{\mathcal{L}}
\def\mT{\mathcal{T}}
\def\mM{\mathcal{M}}
\def\mN{\mathcal{N}}
\def\mD{\mathcal{D}}
\def\fM{\mathfrak{M}}
\def\fC{\mathfrak{C}}
\def\fm{\mathfrak{m}}
\def\fn{\mathfrak{n}}

\def\qed{%
    \ifmmode%
        \eqno\blacksquare%
    \else%
        \hskip1cm\allowbreak\hbox{}\nobreak\hfill$\blacksquare$%
    \fi%
}

\def\pmat#1{\begin{pmatrix} #1 \end{pmatrix}}

\def\mB{\mathbb{B}}
\let\divides=\mid

\font\bigbf = cmbx12 scaled 2000

\parskip=5pt plus 1pt minus 3pt
\pp@headerskip=\z@
\def\@ppmathcount{\thesection.\thepp@mathcount}

\begin{document}

\c@section=6

\barcolorbox{255, 220, 220}{130, 0, 0}{200, 80, 80}{
    \leftskip=0pt plus 1fill \rightskip=\leftskip
    {\bigbf Computability and Complexity}

    \medskip
    \textit{Lecture \thesection, Thursday August 17, 2023}

    \textit{Ari Feiglin}
}

\bigskip

What is the significance of $\NP$-complete problems?
Well, $\P=\NP$ if and only if there exists an $\NP$-complete problem which is also in $\P$.
Obviously if $\P=\NP$ then every $\NP$ problem, and in particular every $\NP$-complete problem is in $\P$.
And if there exists an $\NP$-complete problem in $\P$, then every problem in $\NP$ can be reduced to this problem in polynomial time and therefore is in $\P$.

But if $\P\neq\NP$, then does there exist an $\NP$ problem which is not $\NP$-complete but is also not in $\P$?
It turns out that there is.

\begin{thrm*}[ladnerTheorem,Ladner's\ Theorem]

    If $\P\neq\NP$, then there exists a probem in $\NP$ which is neither $\NP$-complete nor in $\P$.

\end{thrm*}

\begin{proof}

    The idea is to take an $\NP$-complete problem and remove an infinite number of results, while retaining an infinite number of results.
    So we define a function of the form
    \[ S = \set{x}[x\in\sat,\, f(\abs x)\text{ is even}] \]
    Our goal is to define a function $f\colon\bN\longto\bN$ which has the following properties:
    \benum
        \item $f$ can be computed in polynomial time.
        \item If $S\in\P$ then $f$ is even except for a finite number of inputs.
        This would be a contradiction, as then $S$ would be equal to $\sat$ minus a finite number of results.
        But then $S$ would remain $\NP$-complete (we could define a reduction from $\sat$ to $S$), and thus $S$ is in $\P$ and $\NP$-complete, so $\P=\NP$ in contradiction.
        \item If $S$ is $\NP$-complete, then $f$ is odd except for a finite number of inputs.
        This would imply $S$ is finite, and thus in $\P$, meaning $\P=\NP$ in contradiction.
    \eenum

    So all that remains is to find such a function $f$.
    Let us define the sequence of Turing machines which decide search problems in polynomial time
    \[ M_1^D,M_2^D,\dots,M_n^D,\dots \]
    and let us define the sequence of Turing machines which compute some Karp reduction in polynomial time
    \[ M_1^K,M_2^K,\dots,M_n^K,\dots \]
    Let $M_\sat$ be a Turing machine which decides $\sat$ ($\sat$ is decidable in exponential time).

    Let us define the Turing machine $M_f$ which accepts as input a number in unary, and we define $f(n)=M_f(1^n)$ ($1^n$ means $n$ represented in unary).
    We define it like so:

    \algorithm
        \Function{$M_f$}{$1^n$}
            \lIf{$n=1$} \Return $1$
            \State $k\gets M_f(1^{n-1})$
            \If{$k$ is even}
                \State $i\gets\frac k2$
                \For{$z\in\set{0,1}^*$ where $\abs z\leq\log(n)$}
                    \Comment The idea now is to see if $M_i^D(z)$ returns the correct answer to the question $z\in S$.\cr
                    This is if and only if $M_\sat(n)$ returns one, and $M_f(1^{\abs z})$ returns one (meaning $f(\abs z)$ is odd).
                    \EndComment
                    \lIf{$M_i^D(z)=1$ \textbf{and} ($M_\sat(z)=0$ \textbf{or} $M_f(z)=0$)} \Return $k+1$
                    \lElseIf{$M_i^D(z)=0$ \textbf{and} $M_\sat(z)=1$ \textbf{and} $M_f(z)=1$} \Return $k+1$
                \EndFor
                \State \Return $k$
                \Comment So we return $k+1$ if $M_i^D$ gives the wrong answer, and $k$ if it gives the correct answer.\EndComment
            \ElseIf{$k$ is odd}
                \State $i\gets\frac{k+1}2$
                \For{$z\in\set{0,1}^*$ where $\abs z\leq\log(n)$}
                    \Comment Now we check if $M_i^K$ fails to be a reduction from $\sat$ to $S$.
                    \EndComment
                    \lIf{$M_f(z)\neq M_\sat(M_i^K(z))$} \Return $k+1$
                \EndFor
                \State \Return $k$
            \EndIf
        \EndFunc
    \ealgorithm

    We must show that $M_f$ runs in polynomial time.
    Note that at each step, we iterate over inputs $z$ whose length is at most $\log(n)$, and then it runs some Turing machines whose time is exponential in $\abs z$.
    There are at most $n$ such inputs, and since $\abs z$ is logarithmic in $n$, being exponential in $\abs z$ is polynomial in $n$.
    Thus $M_f$ runs in polynomial time, as required.

    Notice that $f(n)\leq f(n+1)\leq f(n)+1$, as if $M_f(1^{n-1})$ returns $k$, then $M_f(1^n)$ returns either $k$ or $k+1$.
    In other words $f(n+1)\in\set{f(n),f(n)+1}$.

    If $S\in\P$ then there exists a Turing machine $M_i^D$ which decides $S$.
    Let $k=2i$, then if $M_f(1^{n^*})=k$ then for $n=n^*+1$, we get $M_f(1^{n-1})=k$ and so $M_i^D$ still decides $S$ and so $M_f(1^n)$ will return $k$.
    Inductively we see that for every $n^*\geq n$, $f(n)=k$.
    Now suppose that $M_f(1^n)$ is never $k$, then there is some other stable point.
    If this stable point is even, then $f$ is even for all but a finite number of inputs, as required.
    Otherwise we have a stable odd point, and this means we have a Turing machine $M_i^K$ which forms a reduction from $\sat$ to $S$, which would mean $S$ is $\NP$-complete and so $\P=\NP$ in contradiction.

    And if $S$ is $\NP$-complete, then there exists a Turing machine $M_i^K$ which computes a Karp reduction from $\sat$ to $S$.
    Let $k=2i-1$, then similar to above there exists an $n^*$ such that for every $n\geq n^*$, $f(n)=k$ and so $f$ is odd except for a finite number of times.
    \qed

\end{proof}

\begin{prop*}

    The class of $\NP$ is closed under Karp reductions.
    In other words, if $S\in\NP$ and there exists a Karp reduction from $S'$ to $S$, then $S'\in\NP$.

\end{prop*}

In other words, $\NP$ is downward-closed.

\begin{proof}

    Let $S\in\NP$, and so it has a polynomial proof system $V$, and let its polynomial be $p$.
    Let $f$ be the Karp reduction from $S'$ to $S$.
    We will define a verifier $V'(x,y)=V(f(x),y)$.
    Since $f$ can be computed in polynomial time, and $V$ runs in polynomial time, $V'$ also runs in polynomial time.

    And since $f$ can be computed in polynomial time, there exists a polynomial $q$ such that $\abs{f(x)}\leq q(\abs x)$ (since it only has polynomial time to add data).
    So if $x\in S'$ then $f(x)\in S$, and so there exists a $y$ such that $\abs y\leq p(\abs{f(x)})$ and $V(f(x),y)=1$.
    So $V'(x,y)=1$ where $\abs y\leq p(\abs{f(x)})\leq p(q(\abs x))$ which is a polynomial (we can assume that the polynomials are increasing, as we can assume that they are of the form $x^n$).
    And if $x\notin S'$ then $f(x)\notin S$ and so for every $y$, $V'(x,y)=V(f(x),y)=0$ as required.
    \qed

\end{proof}

It is also obvious that $\P$ is closed under Karp reductions.
If $S\in\P$ and $f$ is a Karp reduction from $S'$ to $S$, then if $M$ solves $S$ in polynomial time, then we simply return $M(f(x))$ and this solves $S'$.

\begin{defn*}

    We define the class $\coNP$ to be
    \[ \coNP = \set{S}[S^c\in\NP] \]

\end{defn*}

Note that since $\P$ is closed under complements, $\P\subseteq\NP\cap\coNP$.
And if $\P=\NP$ then $\NP$ is closed under complements and so $\NP=\coNP$.
So it is an open problem if $\NP\neq\coNP$.

\begin{prop*}

    If $\coNP$ contains an $\NP$-hard problem, then $\NP=\coNP$.

\end{prop*}

\begin{proof}

    Notice that $\coNP\subseteq\NP$ if and only if $\NP\subseteq\coNP$.
    Since if $\coNP\subseteq\NP$ then if $S\in\NP$, $S^c\in\coNP\subseteq\NP$ and so $S\in\coNP$ as required.
    Similar for the converse.

    Suppose $S$ is $\NP$-hard and in $\coNP$.
    And since $S^c$ is in $\NP$, there exists a Karp reduction $f$ from $S^c$ to $S$, since $S$ is $\NP$-hard.
    Let $S'$ be in $\coNP$, then $(S')^c\in\NP$ and so there exists a Karp reduction $g$ from $(S')^c$ to $S$, and this is also a Karp reduction from $S'$ to $S^c$.
    So for every $S'\in\coNP$, there exists a Karp reduction from $S'$ to $S^c\in\NP$, and since $\NP$ is downward-closed, $S'\in\NP$.
    Thus $\coNP\subseteq\NP$, and so $\coNP=\NP$.
    \qed

\end{proof}

But $\NP$-hard problems require that there exists a Karp reduction from every $\NP$ problem to them.
Karp reductions are quite restrictive, what if we loosened this constraint to require Cook reductions instead?

\begin{prop*}

    If $\NP\cap\coNP$ contains a problem for which there exists a Cook reduction from every $\NP$ problem to it, then $\NP=\coNP$.

\end{prop*}

\begin{proof}

    We will show $\coNP\subseteq\NP$.
    Let $S$ be such a problem in $\NP\cap\coNP$.
    Let $S'\in\coNP$, and so there exists a Cook reduction from $(S')^c$ to $S$.
    We can take this Cook reduction and invert its output, and this defines a Cook reduction from $S'$ to $S$.

    So for every $S'\in\coNP$, there exists a polynomial-time algorithm $A$ with an oracle for $S$ which decides $S'$.
    Notice that $A$ queries the oracle a polynomial number of times.
    And since $S\in\NP\cap\coNP$ there exist verifiers $V_S$ and $V_{S^c}$ with polynomials $p_S$ and $p_{S^c}$, which form polynomial proof systems for $S$ and S$^c$ respectively.
    Let us define a verifier $V'$ which accepts as a witness a sequence of results of queries and witnesses for the oracle.

    And so the input for $V'$ is of the form $\bigl(x,\bigl((\sigma_1,w_1),(\sigma_2,w_2),\dots,(\sigma_t,w_t)\bigr)\bigr)$ where $\sigma_i$ is the result of a query for the oracle, and $w_i$ is a witness.
    Then $V'$ runs $A(x)$, and whenever $A$ queries the oracle with a query of the form $q_i\in S$, $V'$ will check,
    \benum
        \item If $\sigma_i=1$ then $V'$ checks that $V_S(q_i,w_i)=1$.
        If so, $V'$ continues.
        If not, $V'$ returns zero.
        \item If $\sigma_i=0$ then $V'$ checks that $V_{S^c}(q_i,w_i)=0$.
        If so, $V'$ continues.
        If not, $V'$ returns zero.
    \eenum

    $V'$ runs a polynomial time algorithm, and at each step it may run another polynomial time algorithm ($V_S$ or $V_{S^c}$), which all in all takes polynomial time.

    If $x\in S'$ then let $y$ be a sequence of answers to the oracle calls in $A$, along with the polynomial-length witnesses for $V_S$ or $V_{S^c}$.
    Since $A$ makes a polynomial number of oracle queries, $y$ has a polynomial number of values, and each value is polynomial in $x$ (the algorithm only has time to construct queries which are polynomial in
    $x$, and these have witnesses which are polynomial in their length, which is polynomial in the length of $x$).

    Now, if $x\notin S'$ then for any sequence $y$ if $V'$ does not accept all the witnesses, then it returns zero.
    Otherwise, every oracle call gives a valid result (ie. $\sigma_i=1$ means $q_i\in S$, and $\sigma_i=0$ means $q_i\notin S$), and so $V'$ acts exactly like $A$, and so $V'$ would return zero.

    Thus $V'$ is a polynomial proof system for $S'$, as required.
    \qed

\end{proof}

\end{document}
