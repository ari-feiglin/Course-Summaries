\documentclass[10pt]{article}

\usepackage{amsmath, amssymb, mathtools}
\usepackage{tikz-cd}
\usepackage[margin=1.5cm]{geometry}

\font\tensy=cmsy10
\newfam\scrfam
\textfont\scrfam=\tensy
\def\mathscr#1{{\fam=\scrfam#1}}

\input pdfmsym
\input prettyprint
\input preamble

\pdfmsymsetscalefactor{10}
\initpps
\newmathpp{conj}{Conjecture}{255,130,130}{150,0,0}{200,0,0}

\@Arrow@def{varLeftRightarrow}\@Larrow\@Rarrow{1}
\def\implies{\,\longvarRightarrow\,}
\def\iff{\,\longvarLeftRightarrow\,}
\let\eiff=\varLeftRightarrow
\let\to=\varrightarrow
\let\longto=\longvarrightarrow
\let\oto=\varleftrightarrow
\let\injection=\longvaruphookrightarrow

\let\nor=\downarrow
\let\nand=\uparrow

\def\undersumtext#1{\vtop{\hsize=2cm\relax\centering\small #1}}

\newfunc{oracle}{\hbox{\tencsc oracle}}({})

\def\quant{\mathtt{Q}}

\def\sps{\texttt{\char32}}
\def\sat{\mathsf{SAT}}
\def\csat{\mathsf{CSAT}}
\def\is{\mathsf{IS}}
\def\threesat{\mathsf{3SAT}}
\def\clique{\mathsf{Clique}}
\def\vertcover{\textsf{Vertex-Cover}}
\def\domset{\textsf{Dom-Set}}
\def\dhp{\mathsf{DHP}}
\def\dhc{\mathsf{DHC}}
\def\hc{\mathsf{HC}}
\def\tsp{\mathsf{TSP}}
\def\threecolor{\mathsf{3Color}}
\def\kcolor{k\mathsf{Color}}
\def\Color{\mathsf{Color}}
\def\subsum{\mathsf{SubsetSum}}
\def\maxclique{\mathsf{MaxClique}}
\def\minexpr{\mathsf{MinExpression}}
\def\sparse{\mathsf{Sparse}}

\def\PH{\mathbf{PH}}
\def\PF{\mathbf{PF}}
\def\PC{\mathbf{PC}}
\def\P{\mathbf{P}}
\def\NP{\mathbf{NP}}
\def\NPH{\mathbf{NPH}}
\def\NPC{\mathbf{NPC}}
\def\co{\mathsf{co}}
\def\coNP{\mathbf{coNP}}
\def\poly{\mathsf{poly}}
\def\Ppoly{\slfrac\P\poly}

\def\bB{\mathbb{B}}
\def\mA{\mathcal{A}}
\def\mL{\mathcal{L}}
\def\mT{\mathcal{T}}
\def\mM{\mathcal{M}}
\def\mN{\mathcal{N}}
\def\mD{\mathcal{D}}
\def\fM{\mathfrak{M}}
\def\fC{\mathfrak{C}}
\def\fm{\mathfrak{m}}
\def\fn{\mathfrak{n}}

\def\qed{%
    \ifmmode%
        \eqno\blacksquare%
    \else%
        \hskip1cm\allowbreak\hbox{}\nobreak\hfill$\blacksquare$%
    \fi%
}

\def\pmat#1{\begin{pmatrix} #1 \end{pmatrix}}

\def\mB{\mathbb{B}}
\let\divides=\mid

\font\bigbf = cmbx12 scaled 2000

\parskip=5pt plus 1pt minus 3pt
\pp@headerskip=\z@
\def\@ppmathcount{\thesection.\thepp@mathcount}

\begin{document}

\c@section=8

\barcolorbox{255, 220, 220}{130, 0, 0}{200, 80, 80}{
    \leftskip=0pt plus 1fill \rightskip=\leftskip
    {\bigbf Computability and Complexity}

    \medskip
    \textit{Lecture \thesection, Thursday August 24, 2023}

    \textit{Ari Feiglin}
}

\bigskip

\subsection{Nonuniform Computation}

\begin{defn*}

    We say that a decision problem $S$ is \ppemph{decided} by a sequence of boolean circuits $\set{C_n}_{n=0}^\infty$ where for every $x\in\set{0,1}^*$,
    \[ x\in S \iff C_{\abs x}(x) = 1 \]
    So $C_n$ is the boolean circuit which decides if an input of size $n$ is in $S$.

    The \ppemph{complexity} of a boolean circuit is defined to be the number of edges in it.
    The complexity of a boolean circuit $C$ is denoted by $\abs C$.

    We say that a decision problem $S$ is \ppemph{decidable by a sequence of polynomial circuits} if there exists a sequence of boolean circuits $\set{C_n}_{n=0}^\infty$ which decides $S$, and a polynomial
    $p$ such that for every $n$,
    \[ \abs{C_n} \leq p(n) \]

\end{defn*}

\begin{defn*}

    Given a function $\ell\colon\bN\longto\bN$, we define the class of decision problems $\slfrac\P\ell$.
    A decision problem $S$ is in $\slfrac\P\ell$ if and only if there exists a polynomial-time algorithm $A$ and a sequence of commands $\set{a_n}_{n=0}^\infty$ where
    \benum
        \item For every $n$, $\abs{a_n}\leq\ell(n)$
        \item For every $x$, $x$ is in $S$ if and only if $A(a_{\abs x},x)=1$.
    \eenum

    We define the class
    \[ \Ppoly = \bigcup_{\substack{\ell\text{ is a}\\\text{polynomial}}}\slfrac\P\ell = \bigcup_{k=0}^\infty\slfrac\P{n^k} \]

\end{defn*}

\begin{prop*}

    A decision problem $S$ is in $\Ppoly$ if and only if $S$ can be decided by a sequence of polynomial circuits.

\end{prop*}

\begin{proof}

    Suppose $S$ can be decided by a sequence of polynomial circuits $\set{C_n}_{n=0}^\infty$ whose complexity is bound by a polynomial $p$.
    We define the algorithm $A(C,x)$ which takes as input a boolean circuit $C$ and runs it with the input $x$.
    Then since $\abs{C_n}\leq p(n)$ and for every $x$, $x\in S$ if and only if $C_{\abs x}(x)=1$ which is if and only if $A(C_{\abs x},x)=1$.
    So
    \[ S \in \slfrac\P p \subseteq \Ppoly \]

    Now suppose that $S$ is in $\Ppoly$, so there exists a polynomial $p$ and a sequence of commands $\set{a_n}_{n=0}^\infty$ such that $\abs{a_n}\leq p(n)$ and $x\in S$ if and only if $A(a_{\abs x},x)=1$.
    We showed in our proof that $\csat$ is $\NP$-complete, that for every polynomial algorithm there exists a polynomial circuit which computes the algorithm.
    So there exists a boolean circuit $C_n$ which is equivalent to computing $A$ which has $n+\abs{a_n}$ input nodes, and whose size is polynomial in $n+\abs{a_n}$.
    Let us plug in $a_n$ to the input nodes corresponding to it (the first input nodes) in $C_n$ and this gives us a boolean circuit $C'_n$ where $C'_n(x)=A(a_n,x)$.
    And the complexity of $C_n$ is polynomial in $n$, so $\set{C_n}_{n=0}^\infty$ is a sequence of polynomial circuits which decides $S$.
    \qed

\end{proof}

Notice that $\slfrac\P0$ is equal to $\P$, since for $S\in\slfrac\P0$, the size of each command $a_n$ is zero, so they are all empty.
Thus we can define a polynomial-time algorithm $B(x)=A(\epsilon,x)$ and so $x\in S$ if and only if $A(a_{\abs x},x)=A(\epsilon,x)=B(x)=1$.
So $B$ solves $S$, meaning $S\in\P$.
And if $S\in\P$, it is solved by $B(x)$, so define $A(a,x)=B(x)$ and $a_n=\epsilon$.
So
\[ \slfrac\P0 = \P \]

\begin{prop*}

    $\slfrac\P1$ contains undecidable problems.

\end{prop*}

\begin{proof}

    Let $S$ be an undecidable problem which is encoded in unary, ie. $S\subseteq\set{1}^*$ is undecidable.
    Let us define $S'=\set{x}[1^{\abs x}\in S]$, then $S'$ is also undecidable as otherwise $S$ would be (define an algorithm where given a unary encoding $1^n$, it checks for some $x$ whose length is $n$
    if $x\in S'$).
    Now we will prove $S'\in\slfrac\P1$.
    Let us define
    \[ a_n = \begin{cases} 1 & 1^n\in S \\ 0 & \text{else} \end{cases} \]
    And we define $A(a_{\abs x},x)=a_{\abs x}$ (ie. $A(a,x)=a$).
    Then $A$ is polynomial-time and solves $S'$, as $A(a_{\abs x},x)=1$ if and only if $\abs x\in S$, which is if and only if $x\in S'$.
    So $S'\in\slfrac\P1$ despite $S'$ being undecidable.
    \qed

\end{proof}

So $\slfrac\P1\subseteq\Ppoly$ contains undecidable problems, and therefore

\begin{prop*}

    \[ \Ppoly\nsubseteq\NP \]

\end{prop*}

\begin{prop*}

    For every decision problem $S\in\set{0,1}^*$, $S\in\slfrac\P{2^n}$.
    In other words $\slfrac\P{2^n}=\set{0,1}^*$.

\end{prop*}

\begin{proof}

    For every $n$, there are $2^n$ binary strings of length $n$, and let us enumerate the binary strings of length $n$ lexicographically by $\pi_n$ (ie. $\pi_n(i)$ gives the binary string $\omega$ which is
    the $i$th smallest in the order).
    Notice that all $\pi_n$ does is convert $i$ to binary.
    $\pi_n$ and its inverse $\pi_n^{-1}$ can be computed in polynomial time.
    Then let us define $a_n$ to be a string of length $2^n$ where $[a_n]_i$ is one if and only if $\pi_n(i)$ is in $S$.
    Then let $A(a_{\abs x},x)$ look at $a_{\abs x}$ at index $\pi_{\abs x}^{-1}(x)$ (meaning at index $x$, when we view $x$ as binary) and return the value found.

    Then $A(a_{\abs x},x)$ is one if and only if $\pi_{\abs x}(\pi_{\abs x}^{-1}(x))=x\in S$.
    So $A$ decides $S$, meaning $S\in\slfrac\P{2^n}$.
    \qed

\end{proof}

\begin{defn*}

    A set $S$ is called \ppemph{sparse} if there exists a polynomial $p$ such that for every $n$, the number of binary strings in $S$ of length $n$ is bound by $p(n)$.
    In other words, for every $n$
    \[ \abs{S\cap\set{0,1}^n} \leq p(n) \]
    We will denote $S\cap\set{0,1}^n$ by $S^{=n}$.
    We define
    \[ \sparse = \set{S}[\text{$S$ is a sparse set}] \]

\end{defn*}

Notice that $\sparse\subseteq\Ppoly$.
If $S\in\sparse$ then let $p$ be a polynomial bound of $S^{=n}$.
Then for every $n$, we define $a_n$ to be the list of all binary strings of length $n$ which are in $S$.
Since there are $\leq p(n)$ such strings, and each has a length of $n$, $\abs{a_n}\leq n\cdot p(n)$ so $a_n$'s length is bound by a polynomial.
And we can define $A(a_{\abs x},x)$ to just look at the list $a_{\abs x}$ and check if $x$ is in it.
This solves $S$, and so $S\in\Ppoly$.

\begin{prop*}

    A decision problem $S$ is in $\Ppoly$ if and only if there exists a Cook reduction from $S$ to a sparse set.
    In other words,
    \[ \Ppoly = \P^\sparse \]

\end{prop*}

\begin{proof}

    Suppose $S$ has a Cook reduction to some $S'\in\sparse$, let $A^{S'}$ be this Cook reduction.
    Suppose that $S'^{=n}$'s size is bound by $p(n)$.
    Let $t(n)$ be the polynomial time bound on the runtime of $A^{S'}$.
    Notice then that for inputs of length $n$, the length of $A^{S'}$'s longest possible query is bound by $t(n)$.

    So for every $n$, we define the command $a_n$ to be the list of all the strings in $S'$ whose length is bound by $t(n)$.
    Notice then that
    \[ \abs{a_n} = \sum_{i=0}^{t(n)} \abs{S'^{=i}}\cdot i \leq \sum_{i=0}^{t(n)}i\cdot p(i) \]
    We can assume that $p$ is increasing (we can take $p=n^k$ for example), and so
    \[ \abs{a_n} \leq t(n)\cdot p(t(n)) \]
    so $a_n$'s length is bound by a polynomial.
    Now we define $A(a_{\abs x},x)$ which runs $A^{S'}(x)$ but instead of querying $S'$'s oracle, it instead checks the list $a_{\abs x}$ to see if the query is in there.
    This solves $S$ in polynomial time, since $A^{S'}$ does, and since the commands are polynomially bound, we have that $S\in\Ppoly$.

    For the converse, suppose $S\in\Ppoly$ so there exists a polynomial-time algorithm $A$ and a polynomial $p$ which bounds the length of the commands $\set{a_n}_{n=0}^\infty$.
    We can assume without loss of generality that $\abs{a_n}=p(n)$, and that $p(n)$ is increasing.

    For every $n$ and for every $1\leq i\leq p(n)$, we define $a_n^i$ to be the word of length $p(n)$ where the $i$th index is $1$, and all other indexes are zero.
    Let
    \[ S' = \set{a_n^i}[{n\in\bN,\,a_n[i]=1}] \]
    Notice that for every $n$, $\abs{S'^{=n}}\leq n$ since there are only $n$ choices for where to put the bit which is equal to one in a string of zeroes.
    So $S'\in\sparse$.

    Now we define $B^{S'}$:

    \algorithm
        \Function{$B^{S'}$}{$x$}
            \For{$1\leq i\leq p(\abs x)$}
                \lIf{$a^i_{\abs x}\in S'$} $\sigma_i\gets1$
                \lElse $\sigma_i\gets0$
            \EndFor
            \State $a\gets\sigma_1\sigma_2\cdots\sigma_{p(\abs x)}$
            \State \Return $A(a,x)$
        \EndFunc
    \ealgorithm

    Notice that at line $6$, $a=a_{\abs x}$ as $a[i]$ is one if and only if $a_{\abs x}^i\in S$ which is if and only if $a_{\abs x}[i]$ is one, so $a=a_{\abs x}$.
    So $B^{S'}$ is an oracle machine which solves $S$ in polynomial time, and thus a Cook reduction from $S$ to $S'$.
    And since $S'\in\sparse$, we have that $S\in\P^\sparse$.

    So we have shown inclusion in both direction, meaning $\Ppoly=\P^\sparse$.
    \qed

\end{proof}

\begin{thrm*}

    If $\NP\subseteq\Ppoly$ then $\PH=\Sigma_2$.

\end{thrm*}

\begin{proof}

    We will show that $\Pi_2\subseteq\Sigma_2$, and this would mean $\Sigma_2=\Sigma_3$ which would mean $\PH=\Sigma_2$.
    So let $S\in\Pi_2$, and so there exists a polynomial time verifier $V$ and a polynomial $p$ such that
    \[ x\in S \iff \forall y\exists z\bigl(V(x,y,z)=1\bigr) \]
    where $\abs y,\abs z\leq p(\abs x)$.
    Let us define
    \[ S' = \set{(x,y)}[\abs y\leq p(\abs x),\,\exists z,\abs z\leq p(\abs x)\bigl(V(x,y,z)=1\bigr)] \]
    So
    \[ x\in S \iff \forall y,\abs y\leq p(\abs x)\bigl((x,y)\in S'\bigr) \]
    And $S'$ is in $\NP$ as we can define a verifier $V'\bigl((x,y),z\bigr)=V(x,y,z)$, and this is a polynomial proof system (as there exists a witness $\abs z\leq p(\abs x)$).
    Since $\sat$ is $\NP$-complete, there exists a Karp reduction $f$ from $S'$ to $\sat$.
    And so
    \[ x\in S \iff \forall y\bigl(f(x,y)\in\sat\bigr) \]
    with the usual condition that $\abs y\leq p(\abs x)$.
    Since $\sat\in\NP\subseteq\Ppoly$, and so there exists a polynomial-time algorithm $A$, a polynomial $q$, and a sequence of commands $\set{a_n}_{n=0}^\infty$ whose lengths are bound by $q$ which satisfy
    the condition for $\sat$ to be in $\Ppoly$.

    Let $p_f$ be the time complexity for computing the Karp reduction $f$.
    So $\abs{f(x,y)}\leq p_f(\abs x+\abs y)$.
    Since the Karp reduction $f$ runs in $p_f(\abs x+\abs y)$ time and $\abs y\leq p(\abs x)$, we have that $f$ runs in $p_f(n+p(n))$ time.
    So if $\phi=f(x,y)$ then $\abs\phi\leq p_f(n+p(n))$, thus let us define
    \[ b_n = a_1a_2\cdots a_{p_f(n+p(n))} \]
    And this means that $b_n$ will contain $a_{\abs\phi}$.
    And notice that
    \[ \abs{b_n} = \sum_{i=1}^{p_f(n+p(n))}q(i)i \leq p_f(n+p(n))^2 q(p_f(n+p(n))) \]
    So the length of $b_n$ is polynomial in $n$.

    Let us define $V'(x,b_{\abs x},y)$ which functions as follows:

    \benum
        \item First we compute the image of $x,y$ in the Karp reduction, ie. let $\phi=f(x,y)$.
        \item Since $b_{\abs x}$ contains $a_{\abs\phi}$, let us run $A(a_{\abs\phi},\phi)$.
        If this returns zero, return zero since then $\phi\notin\sat$ so $(x,y)\notin S'$.
        \item Now, using $A$ and $a_1,\dots,a_{\abs\phi}$, we can compute a boolean vector $\tau$ which satisfies $\phi$.
        This can be done by setting $x_1$ to true, and checking if the new boolean formula $\phi'$, whose length is less than $\phi$, is in $\sat$.
        This can be done recursively, and so we construct $\tau$.

        Since $b_{\abs x}$ isn't necessarily actually the list of commands $a_1,\dots$, we must verify that $\tau$ actyually satisfies $\phi$.
        So return if $\phi(\tau)=1$.
    \eenum

    $V'$ runs in polynomial time, since every step runs in polynomial time.
    And if $x\in S$ then for every $\abs y\leq p(\abs x)$, $f(x,y)\in\sat$.
    And so for $b_{\abs x}$ defined as the concatenation of $a_1,\dots,a_{p_f(n+p(n))}$, $V'(x,b_{\abs x},y)=1$.
    Since $b_{\abs x}$'s length is polynomial in $\abs x$, this means that if $x\in S$ then there exists a $b_{\abs x}$ such that for all $y$, $V'(x,b,y)=1$ (with the strings being polynomial in length).

    If $x\notin S$ then there exists a $\abs y\leq p(\abs x)$ such that $f(x,y)\notin\sat$ and so for this $y$, $V'$ will always return zero (since the constructed $\tau$ will not satisfy $\phi$).
    So if $x\notin S$, for every $b$ there exists a $y$ (which is this $y$, it is not dependent on $b$), such that $V'(x,b,y)=0$.
    So
    \[ x\in S \iff \exists b\forall y\bigl(V(x,b,y)=1\bigr) \]
    And this means that $S\in\Sigma_2$, so $\Pi_2\subseteq\Sigma_2$ as required.
    \qed

\end{proof}

Thus many researchers hypothesize that $\NP\nsubseteq\Ppoly$, as otherwise the polynomial hierarchy would collapse to $\Sigma_2$, and many researchers do not believe this is true.

\end{document}

