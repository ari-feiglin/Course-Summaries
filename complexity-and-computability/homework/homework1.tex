\documentclass[10pt]{article}

\usepackage{amsmath, amssymb, mathtools}
\usepackage[margin=1.5cm]{geometry}

\font\tensy=cmsy10
\newfam\scrfam
\textfont\scrfam=\tensy
\def\mathscr#1{{\fam=\scrfam#1}}

\input pdfmsym
\input prettyprint
\input preamble

\pdfmsymsetscalefactor{10}
\initpps
\newmathpp{conj}{Conjecture}{255,130,130}{150,0,0}{200,0,0}

\@Arrow@def{varLeftRightarrow}\@Larrow\@Rarrow{1}
\def\implies{\,\longvarRightarrow\,}
\def\iff{\,\longvarLeftRightarrow\,}
\let\eiff=\varLeftRightarrow
\let\to=\varrightarrow
\let\longto=\longvarrightarrow
\let\oto=\varleftrightarrow
\let\injection=\longvaruphookrightarrow

\let\nor=\downarrow
\let\nand=\uparrow

\newfunc{oracle}{\hbox{\tencsc oracle}}({})

\def\PF{\mathbf{PF}}
\def\PC{\mathbf{PC}}
\def\P{\mathbf{P}}
\def\NP{\mathbf{NP}}

\def\bB{\mathbb{B}}
\def\mA{\mathcal{A}}
\def\mL{\mathcal{L}}
\def\mT{\mathcal{T}}
\def\mM{\mathcal{M}}
\def\mN{\mathcal{N}}
\def\mD{\mathcal{D}}
\def\fM{\mathfrak{M}}
\def\fC{\mathfrak{C}}
\def\fm{\mathfrak{m}}
\def\fn{\mathfrak{n}}

\def\qed{%
    \ifmmode%
        \eqno\blacksquare%
    \else%
        \hskip1cm\allowbreak\hbox{}\nobreak\hfill$\blacksquare$%
    \fi%
}

\def\pmat#1{\begin{pmatrix} #1 \end{pmatrix}}

\def\mB{\mathbb{B}}
\let\divides=\mid

\font\bigbf = cmbx12 scaled 2000

\parskip=5pt plus 1pt minus 3pt
\pp@headerskip=\z@
\def\@ppmathcount{\thesection.\thepp@mathcount}

\def\bexerc{\begin{exercise*}}
\def\eexerc{\end{exercise*}}
\def\bblank{\begin{blankpp}}
\def\eblank{\end{blankpp}}

\begin{document}

\c@section=1

\barcolorbox{255, 220, 220}{130, 0, 0}{200, 80, 80}{
    \leftskip=0pt plus 1fill \rightskip=\leftskip
    {\bigbf Computability and Complexity}

    \medskip
    \textit{Assignment 1}

    \textit{Ari Feiglin}
}

\bigskip

\bexerc

    Prove or disprove the following: for every search problem $R$, if $S_R\in\P$ then $R\in\PF$.

\eexerc

\bblank

    This is false, let us define the following search problem:
    \begin{multline*}
        R = \set{\bigl((M,\omega),1\bigr)}[\text{$M$ is a Turing machine which accepts the input $\omega$}]\\
        \cup\set{\bigl((M,\omega),0\bigr)}[\text{$M$ is a Turing machine which does not accept the input $\omega$}]
    \end{multline*}
    Now, there exists a reduction from $A_{\rm TM}$ ($A_{\rm TM}=\set{(M,\omega)}[\text{$M$ is a Turing machine which accepts $\omega$}]$) to $R$: if $A$ solves $R$ then given an input $(M,\omega)$ we return
    $A(M,\omega)$.
    This returns $1$ if $M$ accepts $\omega$ and $0$ otherwise, so it would decide $A_{\rm TM}$.
    Thus $R$ is not decidable, as that would contradict the undecidability of $A_{\rm TM}$.
    And so in particular $R\notin\PF$.

    But
    \[ S_R = \set{(M,\omega)}[\text{$M$ is a Turing machine}] \]
    which is in $\P$, as all we need to do is determine if given an input $(M,\omega)$, that $M$ is a Turing machine.
    So $S_R\in\P$ but $R\notin\PF$.

\eblank

\bexerc

    We define the class $\NP'$ as follows: given a decision problem $S$, we say that $S\in\NP'$ if there exists a polynomial-time algorithm $V$ such that for every $x$:
    \benum
        \item If $x\in S$ then there exists a polynomial $p$ and an input $y$ whose length is at most $p(\abs x)$ such that $V(x,y)=1$.
        \item If $x\notin S$ then for every $y$, $V(x,y)=0$.
    \eenum
    Prove, disprove, or show an equivalence to an open problem that $\NP=\NP'$.

\eexerc

\bblank

    Notice that the requirement that if $x\in S$ then there exists a polynomial $p$ and $y$ such that $\abs y\leq p(\abs x)$ and $V(x,y)=1$ can be simplified to there simply existing a $y$ such that
    $V(x,y)=1$.
    That is, $S\in\NP'$ if and only if there exists a polynomial-time algorithm $V$ such that
    \benum
        \item If $x\in S$ then there exists a $y$ such that $V(x,y)=1$.
        \item If $x\notin S$ then for every $y$, $V(x,y)=0$.
    \eenum
    Ie. $S\in\NP'$ if and only if $S$ has a polynomial-time verifier.
    Obviously if $S\in\NP'$ then this is true, and if this is true then for every $x\in S$ we can simply define the polynomial $p=\abs y$ (where $y$ is the witness for $x$), then this satisfies the condition
    for $S\in\NP'$.
    This is because the polynomial is dependent on $x$, and not the other way around (ie. for $\NP$ there exists a polynomial which must satisfy a condition for every $x\in S$, and for $\NP'$ for every
    $x\in S$ there must exist a polynomial which satisfies some coniditon).

    Now, notice that $A_{\rm TM}\in\NP'$ as we can define the polynomial-time verifier $V$ where $V\bigl((M,\omega),P\bigr)=1$ if and only if $P$ is a valid path of $\omega$ through $M$'s states from the
    start state to its accept state.
    The verifier works by first verifying the validity of its input (that $M$ is a Turing machine, and $P$ is a path of states starting from $M$'s starting state), and then
    simulate $M$ running on $\omega$ and check that each step is precisely the next state in $P$.
    Both of these steps take polynomial time (checking that $M$ is a Turing machine and $P$ is a path of states takes polynomial time, and then the second step just requires iterating over $P$, which takes
    linear time).
    And $V$ is a verifier for $A_{\rm TM}$ since $(M,\omega)\in A_{\rm TM}$ if and only if there exists some path of states on the input $\omega$ which starts at the start state and ends at the accept state,
    ie if and only if there exists a path $P$ such that $V\bigl((M,\omega),P\bigr)=1$.

    But $A_{\rm TM}\notin\NP$ as if $V(x,y)$ is a polynomial proof system for $V$ then we would be able to decide $A_{\rm TM}$:
    Suppose that $A_{\rm TM}\in\NP$ then there exists $V(x,y)$, a polynomial-time verifier for $A_{\rm TM}$, and $p$ a polynomial such that if $x\in A_{\rm TM}$ then there exists a $y$ where
    $\abs y\leq p(\abs x)$ and $V(x,y)=1$.
    Then we can define the following algorithm:

    \algorithm
        \Function{$A_{\rm TM}$-decider}{$M,\omega$}
            \For{$y\in\set{0,1}^*$ \textbf{and} $\abs y\leq p\bigl(\abs{(M,\omega)}\bigr)$}
                \lIf{$V\bigl((M,\omega),y\bigr)=1$} \Return $1$
            \EndFor
            \State\Return $0$
        \EndFunc
    \ealgorithm

    Then \textcsc{$A_{\rm TM}$-decider} decides $A_{\rm TM}$, as if it accepts $(M,\omega)$ then for some $y$, $V\bigl((M,\omega),y)=1$ and since $V$ is a verifier for $A_{\rm TM}$,
    $(M,\omega)\in A_{\rm TM}$.
    And if it doesn't accept $(M,\omega)$ then for every $y$ whose length is less than $p\bigl(\abs{(M,\omega)}\bigr)$, $V\bigl((M,\omega),y\bigr)=0$.
    Since $V(x,y)$ is a polynomial proof system for $A_{\rm TM}$ whose polynomial bound on $x$ is $p$, this means that $(M,\omega)\notin A_{\rm TM}$.
    Thus \textcsc{$A_{\rm TM}$-decider} indeed decides $A_{\rm TM}$, which contradicts it being undecidable.

    Thus $A_{\rm TM}\notin\NP$ but $A_{\rm TM}\in\NP'$, so they are not equal.

\eblank

\def\TSP{\text{TSP}}
\bexerc

    We define the following decision problem
    \[ \TSP = \set{(X,d,D)}[\vcenter{\advance\hsize by-5cm $X$ is a set of targets and $d\colon X\times X\to\bN$ is a distance (weight) function, and $D$ is a natural number such that there exists a cyclic
    path which visits each target whose total length is at most $D$.}] \]
    We further define the search problem $R_\TSP$ where given $(X,d)$ (as defined in $\TSP$), returns a shortest possible cyclic path which visits every target in $X$ (this is called the
    \ppemph{travelling salesman problem}).

    Define a Cook reduction from $R_\TSP$ to $\TSP$.

\eexerc

\bblank

    Let us observe that in a minimal hamiltonian path (ie. a solution to $R_\TSP$), an edge cannot be reused.
    That is, if we traverse from $x$ to $y$ then we cannot traverse later on from $x$ to $y$ again (we can traverse from $y$ to $x$, though).
    After all, if we did then we'd have a cycle of the form $x\to y\to\cdots\to x\to y\to\cdots\to x$, let us denote the first path $y\to x$ by $P_1$ and then second by $P_2$ so this cycle has the form
    $x\to y\to P_1\to x\to y\to P_2\to x$, but we can then create a new cycle $x\to y\to P_1\to x\to P_2^r\to y\to x$, this reduces the weight of the cycle by $d(x,y)$, and so it is shorter.

    Thus, we have a maximum bound on the length of a minimal hamiltonian path,
    \[ M = \sum_{x\neq y\in X}d(x,y) \]
    Now, we can perform a binary search between $0$ and $M$ to determine the length of the minimal hamiltonian path: we take a lower bound $\ell=0$ and an upper bound $h=M$ and check if
    $\parens{X,d,\frac{h+\ell}2}\in\TSP$ using an oracle for $\TSP$.
    If so, we set $\ell=\frac{h+\ell}2$ and recurse, otherwise we set $h=\frac{h+\ell}2-1$ and recurse.
    We do this until $h\leq\ell$ and then we'd know that $\ell$ is the the length of the minimal hamiltonian path.

    Suppose that $L$ is the length that we computed for the minimal hamiltonian path.
    Now, suppose we have an edge $(u,v)$, we can set $d(u,v)=L+1$.
    If there still exists a hamiltonian cycle of length $L$, then there exists a hamiltonian cycle which does not use $(u,v)$, so it is redundant and we can ignore it.
    If we continue this until we have iterated over every edge, we will get the minimal hamiltonian cycle.
    This is because the remaining edges (the edges whose distance was not set to $L+1$) must contain a hamiltonian cycle, as we only removed redundant edges, and there cannot be a remaining redundant edge
    as we removed all of them.

    All that remains is to convert the remaining edges into the hamiltonian path.
    Since the set of remaining edges constitutes a hamiltonian path, all we must do is start at some node $v$, and follow the remaining edges (and remove them as we go) appending them to the path,
    until we've visited every node and returned to $v$.

    So the steps of the algorithm are as follows:
    \benum
        \item Compute $L$, the length of the minimum hamiltonian path via a binary search from $0$ to $M$ (the sum of all the distances).
        \item Iterate over the edges $(u,v)$ and check if there still exists a hamiltonian cycle of length $L$ even if we set $d(u,v)=L+1$.
            If so, then $(u,v)$ is redundant and we can we can ignore it.
        \item The remaining edges define a minimal hamiltonian path, and we convert this to a path by following the remaining edges.
    \eenum

    Suppose $\oracle_\TSP$ is an oracle for $\TSP$, then the following is an algorithm for computing the minimum hamiltonian path:

    \def\TSPOPT{\text{TSP-OPT}}
    \algorithm
        \Function{$\TSPOPT$}{$X,d$}
            \State $M\gets\sum\limits_{x\neq y\in X}d(x,y)$
            \State $L$ $\gets$ \textbf{binary-search}$\bigl(0,M,i\varmapsto\oracle_\TSP(X,d,i)\bigr)$ \cr
            &\kern1cm$\mathchar"012E$ \it\textbf{binary-search}$(\ell,h,f)$ returns the minimum $\ell\leq i\leq h$ such that $f(i)=1$.\span
%
            \cr\noalign{\kern10pt}
%
            \For{$(u,v)\in X$}
                \State $d\gets d(u,v)$
                \State $d(u,v)\gets L+1$
                \lIf{$\neg\oracle_\TSP(X,d,L)$} $d(u,v)\gets d$
            \EndFor
%
            \cr\noalign{\kern10pt}
%
            \State $u\gets\mathit{start}\in X$
            \State $P\gets\varnothing$
            \While{$u\neq\mathit{start}$ \textbf{and} \textbf{visited}$(X)\neq1$}
                \State $v\in\set{v\in N(u)}[d(u,v)<L+1]$ \cr
                &\kern1cm$\mathchar"012E$ \it Find $u$'s neighbor $v$ such that $(u,v)$ is not redundant.
                \State $P\gets(P\to v)$ \cr
                &\kern1cm$\mathchar"012E$ \it Append $v$ to the path $P$
                \State $d(u,v)\gets L+1$
                \State $\textbf{visited}(u)\gets 1$
                \State $u\gets v$
            \EndWhile
%
            \cr\noalign{\kern10pt}
%
            \State\Return $P$
        \EndFunc
    \ealgorithm

    Lines $2$ and $3$ correspond to the first part of the algorithm, $4$ to $8$ correspond to the second part, and $9$ to $17$ correspond to the final part (converting the remaining edges to the hamiltonian
    path).
    Part $1$ takes $\log(M)$ time, and since $M$'s length is less than the length of $d$ (or at least less than $\abs X\cdot\abs d$), this is polynomial with respect to the input.
    Part $2$ takes $\abs X^2$ time.
    And part $3$ takes $\abs X^2$ time (since we can bound the length of the minimum hamiltonian path by $\abs X^2$).
    Thus this algorithm takes polynomial time, and is therefore a valid Cook reduction, as required.

\eblank

\bexerc

    Prove or disprove the following:
    \benum
        \item For every decision problem $S\in\P$, there exists a decision problem $S'$ such that there exists no Cook reduction from $S$ to $S'$.
        \item For every decision problem $S\in\P$, there exists a decision problem $S'$ such that there exists no Karp reduction from $S$ to $S'$.
    \eenum

\eexerc

\bblank

    \benum
        \item This is false, suppose $S'$ is a decision problem and $\oracle_{S'}$ is an oracle for $S'$.
        And since $S\in\P$, there exists a polynomial-time algorithm $A$ which solves $S$.
        Then we can define the following Cook reduction:
        \algorithm
            \Function{$S,S'$-reduct}{x}
                \State $\oracle_{S'}(x)$
                \State\Return $A(x)$
            \EndFunc
        \ealgorithm

        (Line $2$ is really only to ``use'' $\oracle_{S'}$).
        This is a reduction from $S$ to $S'$ (it obviously solves $S$, as $A$ does).
        And so for every decision problem $S'$, there exists a reduction from $S$ to it.

        \item If $S\neq\varnothing$, let $S'=\varnothing$, then for every function $f$, if $x\in S$ then $f(x)\notin S'$ so there cannot exists a Karp reduction from $S$ to $S'$.
        Similarly if $S\neq\set{0,1}^*$, let $S'=\set{0,1}^*$, then for every function $f$, there exists an $x\notin S$ but $f(x)\in S'$, so there also cannot exists a Karp reduction here.
        This works in particular for when $S=\varnothing$.
        So for every decision problem $S$ there exists a $S'$ such that there is no Karp reduction from $S$ to $S'$, and so this is true.

        But if we require that $S'$ be non-trivial ($S'\neq\varnothing,\set{0,1}^*$) then this is false.
        Let $y_1\in S'$ and $y_2\notin S'$, and let $A$ solve $S$, then we can define the following function
        \algorithm
            \Function{$f$}{$x$}
                \lIf{$A(x)=1$} \Return $y_1$
                \lElse \Return $y_2$
            \EndFunc
        \ealgorithm
        So if $x\in S$ then $A(x)=1$ and so $f(x)=y_1\in S'$.
        And if $x\notin S$ then $A(x)=0$ and so $f(x)=y_2\notin S'$.
        And since $f$ runs in polynomial time, as $A$ does, $f$ is a Karp reduction from $S$ to $S'$.

        So if we take the statement as-is, then it is true.
        But if we require that $S'$ be non-trivial (empty or $\set{0,1}^*$) then it is false.
    \eenum

\eblank

\end{document}

