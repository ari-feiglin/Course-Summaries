\documentclass[10pt]{article}

\usepackage{amsmath, amssymb, mathtools}
\usepackage[margin=1.5cm]{geometry}

\font\tensy=cmsy10
\newfam\scrfam
\textfont\scrfam=\tensy
\def\mathscr#1{{\fam=\scrfam#1}}

\input pdfmsym
\input prettyprint
\input preamble

\pdfmsymsetscalefactor{10}
\initpps
\newmathpp{conj}{Conjecture}{255,130,130}{150,0,0}{200,0,0}

\@Arrow@def{varLeftRightarrow}\@Larrow\@Rarrow{1}
\def\implies{\,\longvarRightarrow\,}
\def\iff{\,\longvarLeftRightarrow\,}
\let\eiff=\varLeftRightarrow
\let\to=\varrightarrow
\let\longto=\longvarrightarrow
\let\oto=\varleftrightarrow
\let\injection=\longvaruphookrightarrow

\let\nor=\downarrow
\let\nand=\uparrow

\def\PF{\mathbf{PF}}
\def\PC{\mathbf{PC}}
\def\P{\mathbf{P}}
\def\NP{\mathbf{NP}}

\def\bB{\mathbb{B}}
\def\mA{\mathcal{A}}
\def\mL{\mathcal{L}}
\def\mT{\mathcal{T}}
\def\mM{\mathcal{M}}
\def\mN{\mathcal{N}}
\def\mD{\mathcal{D}}
\def\fM{\mathfrak{M}}
\def\fC{\mathfrak{C}}
\def\fm{\mathfrak{m}}
\def\fn{\mathfrak{n}}

\def\qed{%
    \ifmmode%
        \eqno\blacksquare%
    \else%
        \hskip1cm\allowbreak\hbox{}\nobreak\hfill$\blacksquare$%
    \fi%
}

\def\pmat#1{\begin{pmatrix} #1 \end{pmatrix}}

\def\mB{\mathbb{B}}
\let\divides=\mid

\font\bigbf = cmbx12 scaled 2000

\parskip=5pt plus 1pt minus 3pt
\pp@headerskip=\z@
\def\@ppmathcount{\thesection.\thepp@mathcount}

\begin{document}

\c@section=2

\barcolorbox{255, 220, 220}{130, 0, 0}{200, 80, 80}{
    \leftskip=0pt plus 1fill \rightskip=\leftskip
    {\bigbf Computability and Complexity}

    \medskip
    \textit{Lecture \thesection, Thursday August 3, 2023}

    \textit{Ari Feiglin}
}

\bigskip

For every search problem $R$, there exists a related decision problem
\[ S_R = \set{x}[\exists y\colon (x,y)\in R] \]
$S_R$ essentially asks ``does this input have a solution?''

\begin{note}

    Recall that in our definition of $\PC$ we similarly related a search problem with a decision problem.
    For a search problem $R$ we instead had the decision problem
    \[ \set{(x,y)}[(x,y)\in R] \]
    (This is equal to $R$, but we are viewing this as a decision problem, not a search problem.
    Meaning that we don't view $(x,y)$ as $x$ being the input and $y$ being the output, rather $(x,y)$ is an element of the decision.)

    This decision problem is more closely related to $R$, but it doesn't really simplify anything; after all the set is literally equal to $R$.

\end{note}

\begin{prop*}

    $\P=\NP$ if and only if $\PC\subseteq\PF$.

\end{prop*}

\begin{proof}

    Suppose $\PC\subseteq\PF$, then let $S\in\NP$.
    So there exists a verifier $V$ which runs in polynomial time and a polynomial $p$ such that $x\in S$ if and only if there exists a $y$ where $\abs y\leq p(\abs x)$ and $V(x,y)=1$.
    Then let us define the search problem $R$
    \[ R = \set{(x,y)}[\abs y\leq p(\abs x),\,V(x,y) = 1] \]
    then $R\in\PC$ since $V$ solves $R$ and runs in polynomial time.
    Thus $R\in\PF$, and so there exists an algorithm $A$ such that for every $x$, $A(x)\in R(x)$, meaning $V(x,A(x))=1$ and $\abs{A(x)}\leq p(\abs x)$ (if such an $A(x)$ exists).
    And so let us define an algorithm $B$ where $B(x)=1$ if and only if $A(x)\neq\perp$.

    Then $B$ runs in polynomial time, and $B(x)=1$ if and only if there exists an $A(x)$ such that $V(x,A(x))=1$ and $\abs{A(x)}\leq p(\abs x)$ which is if and only if $x\in S$.
    So $B$ solves $S$ in polynomial time, and so $S\in\P$.

    Now suppose $\P=\NP$ and let $R\in\PC$.
    Then there exists an algorithm $A$ which runs in polynomial time, and a polynomial $p$, such that $A(x,y)=1$ if and only if $(x,y)\in R$.
    Let us define the \emph{decision} problem
    \[ S'_R = \set{(x,u)}[\exists w\colon (x,uw)\in R] \]
    which asks if there exists a solution to $x$ which starts with $u$.
    Then $S'_R$ is in $\NP$ as $(x,u)\in S'_R$ if and only if there exists a $w$ where $\abs u+\abs w\leq p(\abs x)$ and $A(x,uw)=1$.
    Thus $V((x,u),w)=A(x,uw)$ is a polynomial proof system for $S'_R$.
    And so $S'_R\in\P$, so there exists a polynomial time algorithm $D$ where $D(x,u)=1$ if and only if there exists a $w$ where $(x,uw)\in R$.

    Then let us define an algorithm $B$ where we first run $D(x,\epsilon)$, and if it returns zero then return $\perp$ (as there does not exist a solution).
    Otherwise, let us run $D(x,1)$ and if it returns zero then the solution must start with zero, and so we run $D(x,11)$ and so on.

    \algorithm
        \Function{$C$}{$x,u$}
            \If{$A(x,u)=1$}
                \State\textbf{return} $u$
            \ElseIf{$D(x,u1)=1$}
                \State\textbf{return} $C(x,u1)$
            \Else
                \State\textbf{return} $C(x,u0)$
            \EndIf
        \EndFunc
        \State
        \Function{$B$}{$x$}
            \lIf{$D(x,\epsilon)=0$} \textbf{return} $\perp$
            \lElse \textbf{return} $C(x,\epsilon)$
        \EndFunc
    \ealgorithm

    So our algorithm checks that $D(x,\epsilon)\neq0$, then it runs and returns $C(x,\epsilon)$.
    There are at most $p(\abs x)$ steps, so this runs in polynomial time.
    \qed
    
\end{proof}

Notice that in the proof above, we used a solution to one problem to solve another.
This is called an \emph{reduction}:

\begin{defn*}

    If $A$ and $B$ are both problems, then a \ppemph{reduction} from $A$ to $B$ means that $A$ is no harder than $B$ to solve (as in there exists a solution to $A$ using $B$).
    This is denoted $A\leq B$.

\end{defn*}

So if $A\leq B$, and you do not know how to solve $A$, then you do not know how to solve $B$.

\begin{defn*}

    A \ppemph{Cook reduction} from a problem $A$ to $B$ is a polynomial time algorithm which solves $A$ using an oracle for $B$.
    An \ppemph{oracle} for $B$ takes input and returns a solution for $B$ within one step.

\end{defn*}

So for example, in the above proof we used a Cook reduction from $R$ to $S'_R$.

\begin{defn*}

    If $R$ is a search problem, then a \ppemph{self-reduction} is a Cook reduction from $R$ to $S_R$.

\end{defn*}

\def\Rsat{R_{\text{SAT}}}
\def\sat{\mathsf{SAT}}
\begin{exam*}

    Recall that
    \[ \Rsat = \set{(\phi,\tau)}[\text{$\phi$ is a boolean formula in CNF, and $\phi(\tau)$ is true}] \]
    We can create a self-reduction of $\Rsat$, which is a reduction from $\Rsat$ to $S_{\Rsat}=\sat$.

    Suppose we have an oracle $A$ for $\sat$.
    And suppose $\phi$ is a formula over the set of variables $\set{x_1,\dots,x_n}$.
    What we do is first set $x_1$ to true, and then for each disjunction, it is of the form $\bigvee_{i=1}^n\epsilon_ix_i$.
    If $\epsilon_1$ is empty, then we can get rid of this disjunction, as it is true.
    Otherwise we remove $x_1$ from the disjunction.
    This forms a new formula in CNF $\phi_1$, and we can check if $\phi_1$ has a solution using our oracle for $\sat$.
    If it does, then we can recurse over $\phi_1$ where the beginning of our solution has $x_1=1$.

    Otherwise we form $\phi'_1$ where we do a similar process but when $x_1=0$.

    More explicitly,
    \benum
        \item Ask the oracle if $\phi\in\sat$.
        \item If not, return $\perp$.
        \item Otherwise, set $\tau$ to $\epsilon$.
        \item For $i=1$ to $n$:
        \benum
            \item Define $\tau'$ to be $\tau1$.
            \item Define $\phi'$ by valuating the first $i$ variables as their elements in $\tau$.
            \item Ask the oracle if $\phi'\in\sat$.
            \item If yes, set $\tau$ to be $\tau'$.
            \item Otherwise, set $\tau$ to be $\tau0$.
        \eenum
    \eenum

\end{exam*}

\begin{note}

    If $R$ is a search problem which is self-reducing (as in it has a self reduction), then $R$ and $S_R$ are equivalent up to a polynomial.

    If we can solve $R$, then we can solve $S_R$ by simply checking if the algorithm which solves $R$ does not return $\perp$.
    And since there is a self-reduction, if we can solve $S_R$ then we can solve $R$ using a polynomial time algorithm (as Cook reductions are polynomial time).

\end{note}

\begin{defn*}

    Let $S_1$ and $S_2$ be two decision problems.
    A \ppemph{Karp reduction} from $S_1$ to $S_2$ is a function $f$ which can be computed in polynomial time which satisfies
    \[ x\in S_1 \iff f(x)\in S_2 \]
    This is denoted $S_1\leq_p^m S_2$ ($p$ for polynomial, $m$ for many-to-one, as $f$ need not be injective).

\end{defn*}

\begin{prop*}

    If $S_2\in\P$ and there is a Karp reduction from $S_1$ to $S_2$, then $S_1\in\P$.

\end{prop*}

The proof is not too complicated: define an algorithm $A(x)$ which computes $f(x)$ and then determines if $f(x)\in S_2$.
Since computing $f(x)$ and determining if $f(x)\in S_2$ both take polynomial time (since $f$ is a Karp reduction, and $S_2\in\P$), $A$ is a polynomial time algorithm and therefore $S_2\in\P$.

\end{document}

