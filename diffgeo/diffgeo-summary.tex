\documentclass[10pt]{article}

\usepackage{amsmath, amssymb, mathtools}
\usepackage[margin=1.5cm]{geometry}

\input pdfmsym
\input prettyprint
\input preamble

\pdfmsymsetscalefactor{10}
\initpps

\def\pmat#1{\begin{pmatrix} #1 \end{pmatrix}}

\let\divides=\mid
\newfunc{sff}{{\rm I\mkern-2muI}}({})
\newfunc{sign}{{\rm sgn}}({})
\newfunc{det}{{\rm det}}({})
\newfunc{metric}\rho({})
\newfunc{metricc}\sigma({})
\newfunc{spa}{{\rm span}}(\vert)
\newfunc{diam}{{\rm diam}}(\vert)
\newfunc{proj}\pi({})
\newfunc{iproj}{\pi^{-1}}({})
\newfunc{cis}{{\rm cis}}({})
\newfunc{Re}{{\rm Re}}({})
\newfunc{Im}{{\rm Im}}({})
\newfunc{sup}{{\rm sup}}\{\vert\}
\newfunc{Res}{{\rm Res}}({})
\newfunc{wind}n({})
\newfunc{pv}{{\rm pv}}({})
\newfunc{lspan}{{\rm span}}\{|\}
\newfunc{atan}{{\rm atan}}({})

\def\hort{\vcenter{\hrule width15pt height.3pt}}

\font\bigbf = cmbx12 scaled 2000
\@undervecc@def{underbar}\@linecap\@linecap

\def\pmat#1{\begin{pmatrix}#1\end{pmatrix}}

\def\mO{{\cal O}}
\def\mU{{\cal U}}
\let\lineseg=\overleftrightvecc
\let\to=\varrightarrow
\let\longto=\longvarrightarrow
\let\ds=\displaystyle

\def\pdv#1#2{\frac{\partial #1}{\partial #2}}

\def\differ#1#2{\left.d#1\strut\right|_{#2}}

\def\qed{%
    \ifmmode%
        \eqno\blacksquare%
    \else%
        \hskip1cm\allowbreak\hbox{}\nobreak\hfill$\blacksquare$%
    \fi%
}

\begin{document}

\barcolorbox{220, 220, 220}{0, 0, 0}{80, 80, 80}{
    \leftskip=0pt plus 1fill \rightskip=\leftskip
    {\bigbf Differential and Analytic Geometry}

    \medskip
    \textit{Summer $2023$ Summary}

    \textit{Ari Feiglin}
}

\bigskip

For other summaries, check out \textcolor{blue}{\url{https://github.com/ari-feiglin/Course-Summaries.git}}.

\tableofcontents

\newpage
\section{Conic sections}
\c@subsection=1

\input Conic-Sections/conic-sections

\newpage
\section{Curves}

\subsection{Isometries}

\input Curves/isometries

\subsection{Curves and Reparameterization}

\input Curves/curves-and-reparameterization

\subsection{Curvature}

\input Curves/curvature

\subsection{Total Curvature}

\input Curves/total-curvature

\subsection{Three Dimensional Curves}

\input Curves/three-dimensional-curves

\section{Surfaces}

\subsection{Surfaces}

\input Surfaces/manifolds

\subsection{Geodesics}

\input Surfaces/geodesics

\subsection{Tangent Vector Fields}

\input Surfaces/tangent-vector-fields

\subsection{The Gauss-Bonnet Theorem}

\input Surfaces/gauss-bonnet-theorem

\end{document}
