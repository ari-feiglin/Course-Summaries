\documentclass[10pt]{article}

\usepackage{amsmath, amssymb, mathtools}
\usepackage[margin=1.5cm]{geometry}

\input pdfmsym
\input prettyprint
\input preamble

\pdfmsymsetscalefactor{10}
\initpps

\def\pmat#1{\begin{pmatrix} #1 \end{pmatrix}}

\let\divides=\mid
\newfunc{det}{{\rm det}}({})
\newfunc{metric}\rho({})
\newfunc{metricc}\sigma({})
\newfunc{spa}{{\rm span}}(\vert)
\newfunc{diam}{{\rm diam}}(\vert)
\newfunc{proj}\pi({})
\newfunc{iproj}{\pi^{-1}}({})
\newfunc{cis}{{\rm cis}}({})
\newfunc{Re}{{\rm Re}}({})
\newfunc{Im}{{\rm Im}}({})
\newfunc{sup}{{\rm sup}}\{\vert\}
\newfunc{Res}{{\rm Res}}({})
\newfunc{wind}n({})
\newfunc{pv}{{\rm pv}}({})
\newfunc{lspan}{{\rm span}}({})
\newfunc{atan}{{\rm atan}}({})

\font\bigbf = cmbx12 scaled 2000
\@undervecc@def{underbar}\@linecap\@linecap

\def\pmat#1{\begin{pmatrix}#1\end{pmatrix}}

\def\mO{{\cal O}}
\def\mU{{\cal U}}
\let\lineseg=\overleftrightvecc
\let\to=\varrightarrow
\let\longto=\longvarrightarrow
\let\ds=\displaystyle

\def\pdv#1#2{\frac{\partial #1}{\partial #2}}

\def\differ#1#2{\left.d#1\strut\right|_{#2}}

\def\qed{%
    \ifmmode%
        \eqno\blacksquare%
    \else%
        \hskip1cm\allowbreak\hbox{}\nobreak\hfill$\blacksquare$%
    \fi%
}

\def\@ppmathcount{\thesection.\thepp@mathcount}

\begin{document}

\c@section=5

\barcolorbox{220, 255, 220}{0, 130, 0}{80, 200, 80}{
    \leftskip=0pt plus 1fill \rightskip=\leftskip
    {\bigbf Differential and Analytic Geometry}

    \medskip
    \textit{Lecture \thesection, Monday July 17, 2023}

    \textit{Ari Feiglin}
}

\bigskip

Let $\gamma\colon[0,L]\longto\bR^2$ be a natural parameterization, then $T=\gamma'$ and $k(s)=\iprod{T',N}$.
Suppose $T(0)$ has an angle of $\theta_0$ then let us define
\[ \theta(s) = \int_0^s k(p)\,dp + \theta_0 \]
And we define the curve
\[ \beta(s) = \gamma(0) + \pmat{\int_0^s\cos(\theta(s))\,dp\\\int_0^s\sin(\theta(s))\,dp} \]
Now, notice that
\[ \beta'(s) = \pmat{\cos(\theta(s))\\\sin(\theta(s))} \]
And since $\norm{\beta'}=1$, $\beta$ is a natural parameterization.
And further
\[ \beta''(s) = \theta'(s)\cdot\pmat{-\sin(\theta(s))\\\cos(\theta(s))} = \theta'(s)\cdot R_{\frac\pi2}\beta'(s) \]
Which means that
\[ k_\beta(s) = \iprod{\beta''(s), N_\beta(s)} = \iprod{\theta'(s)\cdot R_{\frac\pi2}\beta'(s), R_{\frac\pi2}\beta'(s)} = \theta'(s)\iprod{\beta'(s),\beta'(s)}=\theta'(s)=k(s) \]
(The third equality is since $R_{\frac\pi2}$ is orthogonal.)
So the curvature of $\beta$ is equal to that of $\gamma$.

Now,
\[ T_\beta(0) = \beta'(0) = \pmat{\cos(\theta(0))\\\sin(\theta(0))} = \pmat{\cos(\theta_0)\\\sin(\theta_0)} = T(0) \]
And $\beta(0)=\gamma(0)$.

So by the fundamental theorem of curves, since $k_\beta=k_\gamma$, $\beta(0)=\gamma(0)$, and $T_\beta(0)=T_\gamma(0)$, we have that $\beta=\gamma$.
This means that
\[ T_\gamma(s) = T_\beta(s) = \beta'(s) = \pmat{\cos(\theta(s))\\\sin(\theta(s))} \]
So $\theta$ is the angle function of $\gamma$ (ie. it gives the angle of $\gamma$).
So we have proven the following proposition:

\begin{prop*}

    If $\gamma\colon[0,L]\longto\bR^2$ is a regular smooth curve, then its angle is given by
    \[ \theta_\gamma(s) = \int_0^s k_\gamma(p)\,dp + \theta_0 \]
    where $\theta_0$ is the angle of $T_\gamma(0)$.

\end{prop*}

\begin{defn*}

    If $\gamma\colon[0,L]\longto\bR^2$ is a natural parameterization, then we define
    \[ K_\gamma = \int_0^L k_\gamma(s)\,ds \]
    to be the \ppemph{total curvature} of $\gamma$.

\end{defn*}

So by the above definitions,
\[ K_\gamma = \theta_\gamma(L) - \theta_\gamma(0) \]
So $K_\gamma$ can also be thought of the total difference in the angle of $\gamma$.

\begin{exam*}

    If $\gamma$ is a circle, then intuitively $K_\gamma=2\pi$ since the total difference in the angle of the curve is $2\pi$.
    And since the natural parameterization is given by a curve from $[0,2\pi R]$ whose curvature is $\frac1R$ and thus
    \[ K_\gamma = \int_0^{2\pi R}\frac1R = 2\pi \]
    as expected.

\end{exam*}

\begin{defn*}

    A smooth curve $\gamma\colon[a,b]\longto\bR^n$ is \ppemph{$n$-closed} if $\gamma^{(k)}(a)=\gamma^{(k)}(b)$ for every $0\leq k\leq n$.
    If $\gamma$ is $n$-closed for every $n$, then $\gamma$ is called \ppemph{closed}.

\end{defn*}

\begin{prop*}

    If $\gamma$ is a $1$-closed regular smooth curve then $K_\gamma=2\pi n$ for some $n\in\bZ$.

\end{prop*}

\begin{proof}

    Since $\gamma$ is $1$-closed, $\gamma'(0)=\gamma'(L)$.
    But recall that
    \[ \gamma'(s) = \pmat{\cos(\theta(s))\\\sin(\theta(s))} \]
    So we have that
    \[ \pmat{\cos(\theta(0))\\\sin(\theta(0))} = \pmat{\cos(\theta(L))\\\sin(\theta(L))} \]
    Which is if and only if $\theta(L)=\theta(0)+2\pi n$ for some $n\in\bZ$, and so $K_\gamma=2\pi n$ as required.
    \qed

\end{proof}

\begin{defn*}

    If $\gamma$ is a $1$-closed regular smooth curve, then $\frac1{2\pi}K_\gamma$ is called $\gamma$'s \ppemph{winding number} (about $0$).

\end{defn*}

\begin{thrm*}[hopfTheorem,Hopf's\ Theorem]

    If $\gamma\colon[0,L]\longto\bR^2$ is a closed natural parameterization, then $\gamma$ is injective (other than at the points $0$ and $L$).

\end{thrm*}

We will not be proving this theorem.

This means that if $\gamma$ is closed, then $K_\gamma=\pm2\pi$.
This is because the winding number is $\pm1$, as otherwise $\gamma$ would have to intersect with itself.
The sign of $K_\gamma$ correlates with its orientation.
We will prove this formally:

\begin{proof}

    Suppose $\gamma(0)=0$, and $T(0)=\pmat{1\\0}$, and $0\leq\gamma_1(s)$ for every $s\neq0,T$ (we can get to this via an isometry).
    Let $B=\set{(x,y)}[0\leq x\leq y\leq T]$ and we define a function $g\colon B\longto[-1,1]$ by
    \[ g(s,t) = \begin{cases} \frac{\gamma(t)-\gamma(s)}{\norm{\gamma(t)-\gamma(s)}} & s\neq t\text{ and } s\neq0,t\neq T \\ \gamma'(s) & s=t \\ -\gamma'(0) & s=0\text{ and }t=T \end{cases} \]
    $g$ is therefore continuous.
    Let us define $\alpha_0(t)$ to be the line which connects $(0,0)$ to $(T,T)$, ie. $\alpha_0(t)=t(T,T)$.
    Thus $\alpha_0$ is contained within $B$.
    Then
    \[ g(\alpha_0(s)) = \gamma'(s) = \pmat{\cos(\theta_0(s)) \\ \sin(\theta_0(s))} \]
    Where $\theta_0$ is $g\circ\alpha_0$'s angle function.
    Thus
    \[ K = \theta_0(T) - \theta_0(0) \]

\end{proof}

\end{document}

