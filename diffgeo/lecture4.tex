\documentclass[10pt]{article}

\usepackage{amsmath, amssymb, mathtools}
\usepackage[margin=1.5cm]{geometry}

\input pdfmsym
\input prettyprint
\input preamble

\pdfmsymsetscalefactor{10}
\initpps

\def\pmat#1{\begin{pmatrix} #1 \end{pmatrix}}

\let\divides=\mid
\newfunc{det}{{\rm det}}({})
\newfunc{metric}\rho({})
\newfunc{metricc}\sigma({})
\newfunc{spa}{{\rm span}}(\vert)
\newfunc{diam}{{\rm diam}}(\vert)
\newfunc{proj}\pi({})
\newfunc{iproj}{\pi^{-1}}({})
\newfunc{cis}{{\rm cis}}({})
\newfunc{Re}{{\rm Re}}({})
\newfunc{Im}{{\rm Im}}({})
\newfunc{sup}{{\rm sup}}\{\vert\}
\newfunc{Res}{{\rm Res}}({})
\newfunc{wind}n({})
\newfunc{pv}{{\rm pv}}({})
\newfunc{lspan}{{\rm span}}({})
\newfunc{atan}{{\rm atan}}({})

\font\bigbf = cmbx12 scaled 2000
\@undervecc@def{underbar}\@linecap\@linecap

\def\pmat#1{\begin{pmatrix}#1\end{pmatrix}}

\def\mO{{\cal O}}
\def\mU{{\cal U}}
\let\lineseg=\overleftrightvecc
\let\to=\varrightarrow
\let\longto=\longvarrightarrow
\let\ds=\displaystyle

\def\pdv#1#2{\frac{\partial #1}{\partial #2}}

\def\differ#1#2{\left.d#1\strut\right|_{#2}}

\def\qed{%
    \ifmmode%
        \eqno\blacksquare%
    \else%
        \hskip1cm\allowbreak\hbox{}\nobreak\hfill$\blacksquare$%
    \fi%
}

\def\@ppmathcount{\thesection.\thepp@mathcount}

\begin{document}

\c@section=4

\barcolorbox{220, 255, 220}{0, 130, 0}{80, 200, 80}{
    \leftskip=0pt plus 1fill \rightskip=\leftskip
    {\bigbf Differential and Analytic Geometry}

    \medskip
    \textit{Lecture \thesection, Monday July 17, 2023}

    \textit{Ari Feiglin}
}

\bigskip

Recall that if $\gamma$ is a natural parameterization, then if $\gamma'=T$ and so $\norm T=1$, thus
\[ T(s) = \pmat{\cos(\alpha(s)) \\ \sin(\alpha(s))} \]
And we showed that the curvature of $\gamma$ is given by
\[ k_\gamma(s) = \alpha'(s) \]
But what if $\gamma$ isn't the natural parameterization?

Let $\beta$ be any regular smooth curve, and $\gamma$ its natural parameterization.
Then recall that $\gamma=\beta\circ s_\beta^{-1}$ and so $\beta=\gamma\circ s_\beta$.
Thus
\[ \beta'(t) = s_\beta'(t)\cdot\gamma'(s_\beta(t)) \]
(This is a bit confusing, since $s_\beta$ is a scalar, and $\gamma$ is a vector).
We know that there exists an $\alpha$ such that
\[ \alpha = \atanof{\frac{\gamma'_2}{\gamma'_2}} \]
And since
\[ \frac{\gamma'_2(s)}{\gamma'_1(s)} = \frac{\beta'_2(t)}{\beta'_1(t)} \]
And thus
\[ \alpha(s) = \atanof{\frac{\beta'_2(t)}{\beta'_1(t)}} \]
Recall that the derivative of $\atanof x=\frac1{1+x^2}$, and since
\[ \frac d{ds}\atanof{\frac{\beta'_2(t)}{\beta'_1(t)}} = \frac d{dt}\atanof{\frac{\beta'_2(t)}{\beta'_1(t)}}\cdot\frac{dt}{ds} \]
We have that
\[ k(s) = \alpha'(s) = \frac1{1+\parens{\frac{\beta'_2(t)}{\beta'_1(t)}}^2}\cdot\frac{\beta''_2(s)\beta'_1(s)-\beta'_2(s)\beta''_1(s)}{\beta'_1(s)^2}\cdot\frac{dt}{ds} =
\frac{\beta''_2(s)\beta'_1(s)-\beta'_2(s)\beta''_1(s)}{\beta'_1(t)^2+\beta'_2(t)^2}\cdot\frac{dt}{ds} \]

By definition,
\[ s(t) = \int_0^t \norm{\beta'(u)}\,du \implies s'(t) = \norm{\beta'(t)} \]
So
\[ \frac{dt}{ds} = \frac1{\norm{\beta'(t)}} = \frac1{\sqrt{\beta'_1(s)^2+\beta_2'(s)^2}} \]
And so all in all we have that
\[ k(s) = \frac{\beta''_2(s)\beta'_1(s)-\beta'_2(s)\beta''_1(s)}{\bigl(\beta'_1(t)^2+\beta'_2(t)^2\bigr)^{1.5}} \]
So we have proven the following proposition:

\begin{prop*}

    If $\beta$ is a regular smooth curve, then its curvature is given by
    \[ k(s) = \frac{\beta''_2(s)\beta'_1(s)-\beta'_2(s)\beta''_1(s)}{\bigl(\beta'_1(t)^2+\beta'_2(t)^2\bigr)^{1.5}} \]

\end{prop*}

\begin{exam*}

    So if $\beta(t)=(t,f(t))$ then
    \[ k_\beta(t) = \frac{f''(t)}{(1+f'(t))^{1.5}} \]

    Thus if we have a function $f\colon\bR\to\bR$, then we can discuss its curvature as the parameterization of its graph.

\end{exam*}

Suppose we have a regular smooth curve $\alpha$ which is a natural parameterization.
Our goal is to find the circle tangent to $\alpha$ at the point $s_0$.
\benum
    \item First, we can write $\alpha$ as a second order Taylor series
    \[ \alpha(s_0+h) = \alpha(s_0) + h\alpha'(s_0) + \frac{h^2}2\alpha''(s_0) + \epsilon(h) \]
    where $\epsilon(h)\in o(h^2)$ (meaning $\frac{\norm{\epsilon(h)}}{h^2}\xvarrightarrow{}[h\to0]0$).

    \item Now, we know that $T=\alpha'$ and $\alpha''=T'=k(s)N$ and thus
    \[ \alpha(s_0+h) - \alpha(s_0) = hT(s_0) + k(s_0)\frac{h^2}2N(s_0) + \epsilon(h) \]
    Let us define
    \[ \Delta(h) = \alpha(s_0+h) - \alpha(s_0),\quad x(h) = \iprod{\Delta(h), T(s_0)},\quad y(h) = \iprod{\Delta(h), N(s_0)} \]
    Thus $\Delta(h)=x(h)T(s_0) + y(h)N(s_0)$, and so
    \[ x(h) = \iprod{hT(s_0) + k(s_0)\frac{h^2}2N(s_0) + \epsilon(h), T(s_0)} = h + \iprod{\epsilon(h), T(s_0)} \]
    Since $\norm T=1$ and $T$ and $N$ are orthogonal.
    Now since by Cauchy-Schwarz, $\abs{\iprod{u,v}}\leq\norm u\norm v$, we have that $\iprod{\epsilon(h), T(s_0)}=\epsilon_1(h)\in o(h^2)$.
    Similarly
    \[ y(h) = k(s_0)\cdot\frac{h^2}2 + \epsilon_2(h) \]
    where $\epsilon_1(h)\in o(h^2)$.

    \item Now, let us define the axis system $(T(s_0),N(s_0))$ centered at $\alpha(s_0)$, then since in this axis system $\alpha(s_0)=0$, we will denote $\alpha(s_0+h)$ by $\alpha(h)$, and $T(s_0)$ and
    $N(s_0)$ by $T$ and $N$, and $k(s_0)$ by $k$.
    Thus
    \[ \alpha(h) = x(h)T + y(h)N = hT + k\cdot\frac{h^2}2N + \bigl(\epsilon_1(h)T + \epsilon_2(h)N\bigr) \]
    So given an $h$, we will define a circle through $(0,0)$, $\parens{\pm h,k\cdot\frac{h^2}2}$.
    Such a circle would have the form $(x-a)^2+(y-b)^2=R^2$.
    Let us assume $a=0$ (the reason for assuming this is by symmetry).
    Thus we must have
    \[ b^2 = R^2,\quad h^2 + \parens{k\cdot\frac{h^2}2-b}^2 = R^2 \]
    So
    \[ h^2 + k^2\cdot\frac{h^4}4 - bkh^2+ b^2 = R^2 \implies k^2\cdot\frac{h^4}4 = bkh^2 - h^2 \implies k^2\cdot\frac{h^2}2 = bk - 1 \]
    So as $h\to0$ we get that
    \[ bk = 1 \implies b=\frac1k \implies R=\abs b=\abs{\frac1k} \]
    And the center of the circle is $\parens{0,\frac1k}$.

    \item Now, we know that $(x,y)$ in this axis system corresponds to $xT(s_0)+yN(s_0)+\alpha(s_0)$ in $\bR^2$, and so the circle we got is the set
    \[ \set{\alpha(s_0)+xT(s_0)+yT(s_0)}[x^2+\parens{y-\frac1{k(s_0)}}^2=\frac1{k(s_0)^2}] \]
    We can also see this because the center of the circle is at $\frac1k$ in the new axis system, which is the point
    \[ c(s_0) = \alpha(s_0) + \frac1{k(s_0)}N \]
    And the radius of the circle is still $\frac1{k(s_0)}$ (since the new axis system is simply an isometry).
\eenum

Newton was originally the person who came up with this formula (for the center of the circle and its radius).
The way he approached it was by taking the points $\alpha(s_0)$ and $\alpha(s_0+h)$ and looking at the intersection of the normal lines at these points, $o(h)$.
Then we will show that $o(h)\varrightarrow c(s_0)$.
Let $\ell_1(t)$ and $\ell_2(t)$ be the normal lines at $\alpha(s_0)$ and $\alpha(s_0-h)$ respectively.
We know that
\[ \ell_1(t) = \alpha(s_0) + tN(s_0),\qquad \ell_2(t) = \alpha(s_0+h) + tN(s_0+h) \]
And since we know that
\[ \alpha(s_0+h) = \alpha(s_0) + h\alpha'(s_0) + o(h) = \alpha(s_0) + hT(s_0) + o(h) \]
And
\[ N(s_0+h) = N(s_0) + hN'(h) + o(h) \]
And since $N'=-k(s_0)T$, we have
\[ N(s_0+h) = N(s_0) - hkT(h) + o(h) \]
Then $\ell_1(t)=\ell_2(p)$ if and only if
\[ \alpha(s_0) + tN(s_0) = \alpha(s_0) + hT(s_0) + o(h) + p(N(s_0) - hkT(s_0) + o(h)) \iff (t-p)N(s_0) = h(1-pk(s_0))T(s_0) + o(h) \]
Meaning that
\[ (p-t)N(s_0) + h(1-pk(s_0))T(s_0) \in o(h) \]
Thus
\[ \frac{p-t}h N(s_0) + (1-pk(s_0))T(s_0) \xvarrightarrow{}[h\to0] 0 \]
Since $N$ and $T$ are orthonormal, this means that $p-t=0$ and $1-pk(s_0)=0$.
So $t=p=\frac1{k(s_0)}$.
And so the center point is
\[ c(s_0) = \alpha(s_0) + \frac1{k(s_0)}N(s_0) \]
as we showed before.

\vskip.5cm
Suppose $\alpha$ is a natural parameterization, and $\phi\colon v\varmapsto Ax+c$ is an isometry (and so $A$ is orthonormal).
Then let $\beta=\phi\circ\alpha$, so
\[ \beta(s) = A\alpha(s) + c \]
Then $\beta'(s)=A\alpha'(s)$, and since $A$ is orthonormal, $\norm{\beta'}=\norm{\alpha'}=1$ since $\alpha$ is natural.
Thus $\beta$ is also a natural parameterization.
And so
\[ k_\beta(s) = \iprod{\beta''(s),R_{\frac\pi2}\beta'(s)} = \iprod{A\alpha''(s),R_{\frac\pi2}A\alpha'(s)} \] 
Now, rotations and $A$ commute up to sign.
If $\detof A=1$ then they commute, and if $\detof A=-1$ then $R_\theta A=-AR_\theta$.
So this is equal to $\detof A\iprod{A\alpha''(s), AR_{\frac\pi2}\alpha'(s)}$, since $A$ is orthogonal this is equal to
\[ = \detof A\iprod{\alpha''(s), R_{\frac\pi2}\alpha'(s)} = \pm k_\alpha(s) \]
So curvature is preserved under isometries, up to the sign of the determinant of the orthogonal matrix.

\end{document}

