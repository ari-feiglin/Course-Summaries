\documentclass[10pt]{article}

\usepackage{amsmath, amssymb, mathtools}
\usepackage{tikz}
\usetikzlibrary{decorations.markings}
\usepackage[margin=1.5cm]{geometry}

\input prettyprint
\input preamble
\input pdfmsym

\pdfmsymsetscalefactor{10}
\initpps

\def\pmat#1{\begin{pmatrix} #1 \end{pmatrix}}

\let\divides=\mid
\newfunc{det}{{\rm det}}({})
\newfunc{metric}\rho({})
\newfunc{metricc}\sigma({})
\newfunc{spa}{{\rm span}}(\vert)
\newfunc{diam}{{\rm diam}}(\vert)
\newfunc{proj}\pi({})
\newfunc{iproj}{\pi^{-1}}({})
\newfunc{cis}{{\rm cis}}({})
\newfunc{Re}{{\rm Re}}({})
\newfunc{Im}{{\rm Im}}({})
\newfunc{sup}{{\rm sup}}\{\vert\}
\newfunc{Res}{{\rm Res}}({})
\newfunc{wind}n({})
\newfunc{pv}{{\rm pv}}({})
\newfunc{lspan}{{\rm span}}\{|\}
\newfunc{spec}{{\rm spec}}({})
\newfunc{trace}{{\rm trace}}({})
\newfunc{sff}{{\rm I\mkern-2muI}}({})

\font\bigbf = cmbx12 scaled 2000
\@undervecc@def{underbar}\@linecap\@linecap

\def\pmat#1{\begin{pmatrix}#1\end{pmatrix}}

\def\mO{{\cal O}}
\def\mU{{\cal U}}
\let\lineseg=\overleftrightvecc
\let\to=\varrightarrow
\let\longto=\longvarrightarrow
\let\ds=\displaystyle

\def\pdv#1#2{\frac{\partial #1}{\partial #2}}

\def\differ#1#2{\left.d#1\strut\right|_{#2}}

\def\qed{%
    \ifmmode%
        \eqno\blacksquare%
    \else%
        \hskip1cm\allowbreak\hbox{}\nobreak\hfill$\blacksquare$%
    \fi%
}

\def\@ppmathcount{\thesection.\thepp@mathcount}

\def\bexerc{\begin{exercise*}}
\def\eexerc{\end{exercise*}}
\def\bblank{\begin{blankpp}}
\def\eblank{\end{blankpp}}

\begin{document}

\c@section=4

\barcolorbox{220, 255, 220}{0, 130, 0}{80, 200, 80}{
    \leftskip=0pt plus 1fill \rightskip=\leftskip
    {\bigbf Differential and Analytic Geometry}

    \medskip
    \textit{Assignment \thesection}

    \textit{Ari Feiglin}
}

\bigskip

\bexerc

    A torus is obtained by rotating the curve
    \[ (x-a)^2 + z^2 = b^2 \]
    about the $y$ axis.
    Suppose that $b<a$, then
    \benum
        \item Find a parameterization for the torus.
        \item Find the first fundamental form of the torus.
        \item Find the Christoffel symbols of the torus.
        \item Find the geodesic equation of the torus.
    \eenum

\eexerc

\bblank

    \benum
        \item We can parameterize the curve by
        \[ \gamma(\theta) = (b\cos\theta+a,0,b\sin\theta) \]
        Thus rotating it about the $y$ axis gives the parameterization
        \[ \sigma(\theta,\phi) = (b\cos\theta\cos\phi+a\cos\phi,b\cos\theta\sin\phi,b\sin\theta) \]

        \item Find the partial derivatives of $\sigma$:
        \begin{align*}
            \sigma_1 &= (-b\sin\theta\cos\phi,\,-b\sin\theta\sin\phi,b\cos\theta) \\
            \sigma_2 &= (-(b\cos\theta+a)\sin\phi,(b\cos\theta+a)\cos\phi,0)
        \end{align*}
        Thus we get, by $g_{ij}=\iprod{\sigma_i,\sigma_j}$,
        \[ g = \pmat{b^2 & 0 \\ 0 & (b\cos\theta+a)^2} \]
        (In general for rotations of the curve $\gamma(\theta)=(r(\theta),0,z(\theta))$, a simple computation yields $g_{11}=r'(\theta)^2+z'(\theta)^2$, $g_{12}=0$, $g_{22}=r(\theta)^2$).

        \item Recall from lecture
        \[ \Gamma_{ij}^k = \frac12g^{km}(g_{mi,j} + g_{jm,i} - g_{ij,m}) \]
        For a diagonal metric (first fundamental form), $g^{-1}$ is also diagonal so we can focus only coordinates which are equal (ie. $g_{ij}=g^{ij}=0$ for $i\neq j$).
        Thus we have
        \[ \Gamma_{ij}^k = \frac12g^{kk}(g_{ki,j} + g_{jk,i} - g_{ij,k}) \]
        So if $i=j=k$ then we have
        \[ \Gamma_{kk}^k = \frac12g^{kk}(g_{kk,k} + g_{kk,k} - g_{kk,k}) = \frac12g^{kk}g_{kk,k} \]
        And if $i=j\neq k$ then $g_{ki}=g_{jk}=0$ and so
        \[ \Gamma_{ij}^k = -\frac12g^{kk}g_{ij,k} \]
        Finally if only one of $i$ and $j$ are equal to $k$, we can assume $i=k$ and $i\neq j$ (since $\Gamma_{ij}^k=\Gamma_{ji}^k$), so
        \[ \Gamma_{ij}^k =\frac12g^{kk}g_{kk,j} \]
        Let us summarize this, using the notation $\neg k=3-k$ (ie. swapping one and two),
        \[ \Gamma_{ij}^k = \begin{cases} \frac12g^{kk}g_{kk,k} \\ -\frac12g^{kk}g_{\neg k\neg k,k} & i=j=\neg k \\ \frac12g^{kk}g_{kk,\neg k} & \text{only one between $i$ and $j$ is equal to $k$}
        \end{cases} \]
        
        Now, since our first fundamental form is diagonal, we can utilize this.
        \begin{alignat*}{3}
            \Gamma_{11}^1 &=& \frac12g^{11}g_{11,1} &= 0 \\
            \Gamma_{12}^1 = \Gamma_{21}^1 &=& \frac12g^{11}g_{11,2} &= 0 \\
            \Gamma_{22}^1 &=& -\frac12g^{11}g_{22,1} &= \frac{\sin\theta\cdot(b\cos\theta+a)}b \\
            \Gamma_{11}^2 &=& -\frac12g^{22}g_{11,2} &= 0 \\
            \Gamma_{12}^2 = \Gamma_{21}^2 &=& \frac12g^{22}g_{22,1} &= -\frac{b\sin\theta}{b\cos\theta+a} \\
            \Gamma_{22}^2 &=& \frac12g^{22}g_{22,2} &= 0
        \end{alignat*}

        \item The geodesic equations of a surface are, for $i=1,2$
        \[ \ddot\beta^i + \dot\beta^k\dot\beta^j\Gamma_{kj}^i = 0 \]
        So for $i=1$, we can focus on only $k=j=2$,
        \[ \ddot\beta^1 + \bigl(\dot\beta^2\bigr)^2\cdot\frac{\sin\theta\cdot(b\cos\theta+a)}b \]
        And for $i=2$, we can focus on only when one of $k$ and $j$ are equal to $2$, then we get
        \[ \ddot\beta^2 - 2\dot\beta^1\dot\beta^2\cdot\frac{b\sin\theta}{b\cos\theta+a} \]
    \eenum

\eblank

\bexerc

    Suppose we have the function $f(x,y)=\frac9x$, and the metric of a surface is given by
    \[ g_{ij} = f(x,y)^2\delta^i_j \]
    find the Christoffel symbols of $\sigma_1$.

\eexerc

\bblank

    Here $g$ is once again a diagonal matrix, so we can utilize the formulas we computed in the previous question.
    \begin{alignat*}{3}
        \Gamma^1_{11} &=& \frac12g^{11}g_{11,1} &= \frac{f_x}f \\
        \Gamma^1_{12} = \Gamma^1_{21} &=& \frac12g^{11}g_{11,2} &= \frac{f_y}f \\
        \Gamma^1_{22} &=& -\frac12g^{11}g_{22,1} &= -\frac{f_x}f
    \end{alignat*}
    Now since $f=\frac9x$ we have
    \begin{align*}
        \Gamma^1_{11} &= -\frac1x \\
        \Gamma^1_{12} = \Gamma^1_{21} &= 0 \\
        \Gamma^1_{22} &= \frac1x
    \end{align*}

\eblank

\bexerc

    Using the coordinates $(r,\phi)$, suppose we have a surface whose metric is
    \[ g = \pmat{r^2 & 0 \\ 0 & r^4} \]
    Prove that every geodesic which isn't constant in $\phi$ (ie. the geodesic, as a curve on the surface, is of the form $\sigma\circ\beta$.
    So we require $\beta^2$ is not constant) is isolated from the origin.
    This means that there is a neighborhood of the origin which is disjoint from the curve.

\eexerc

\bblank

    Let us first find the geodesic equations of this surface.
    To do so we must find the Christoffel symbols, and since the metric is diagonal, we can use the above formulas.
    \begin{alignat*}{3}
        \Gamma_{11}^1 &=& \frac12g^{11}g_{11,1} &= r^{-1} \\
        \Gamma_{12}^1 = \Gamma_{21}^1 &=& \frac12g^{11}g_{11,2} &= 0 \\
        \Gamma_{22}^1 &=& -\frac12g^{11}g_{22,1} &= -2r \\
        \Gamma_{11}^2 &=& -\frac12g^{22}g_{11,2} &= 0 \\
        \Gamma_{12}^2 = \Gamma_{21}^2 &=& \frac12g^{22}g_{22,1} &= 2r^{-1} \\
        \Gamma_{22}^2 &=& \frac12g^{22}g_{22,2} &= 0
    \end{alignat*}
    The geodesic equations are, for $i=1,2$
    \[ \ddot\beta^i + \dot\beta^j\dot\beta^k\Gamma^i_{jk} = 0 \]
    So if $\beta(t)=(r(t),\phi(t))$, we have
    \begin{align*}
        \ddot r + \dot r^2r^{-1} - 2\dot\phi^2r &= 0 \\
        \ddot\phi + 4\dot r\dot\phi r^{-1} &= 0
    \end{align*}
    So we have that $2\dot\phi^2=\ddot rr^{-1}+\dot r^2r^{-2}$ and differentiating this gives
    \[ 4\dot\phi\ddot\phi = \dddot rr^{-1} - \ddot r\dot rr^{-2} + 2\ddot r\dot rr^{-2} - 2\dot r^3r^{-3} = \dddot rr^{-1} + \ddot r\dot rr^{-2} - 2\dot r^3r^{-3} \]
    And multiplying the second equation by $4\dot\phi$ gives
    \[ 4\dot\phi\ddot\phi + 16\dot r\dot\phi^2r^{-1} = \dddot rr^{-1} + \ddot r\dot rr^{-2} - 2\dot r^3r^{-3} + 8\ddot r\dot rr^{-2} + 8\dot r^3r^{-3} =
    \dddot rr^{-1} + 9\ddot r\dot rr^{-2} + 6\dot r^3r^{-3} = 0 \]
    So we have that
    \[ \dddot rr^2 + 9\ddot r\dot rr + 6\dot r^3 = 0 \]

    Suppose now that $\beta$ is not isolated from the origin, then there exists a sequence $t_1<t_2<\dots$ such that $r(t_n),\phi(t_n)\to0$.
    But notice then that since $r$ is positive, and so $r(t_n)$ converges to a minimum/lower bound, we have that $\dot r\leq0$ on this sequence.
    And since this is a minimum, we have $\ddot r\geq0$
    And so we have that
    \[ 9\ddot r\dot rr,6\dot r^3 \leq 0 \]
    which means that by the ODE, $\dddot r\geq0$.
    

\eblank

\bexerc

    Find the Gaussian curvature of the rotation of $t\mapsto(\cosh(t),0,t)$ at every point using
    \benum
        \item Theorema Egregium
        \item The Laplace-Beltrami operator
    \eenum

\eexerc

\bblank

    \benum
        \item We know that the metric of this surface is
        \[ g = \pmat{r'(t)^2+z'(t)^2 & 0 \\ 0 & r(t)^2} = \pmat{\sinh(t)^2+1 & 0 \\ 0 & \cosh(t)^2} = \pmat{\cosh(t)^2 & 0 \\ 0 & \cosh(t)^2} \]
        Theorema Egregium gives us
        \[ K = g^{11}\Bigl(\Gamma_{11,2}^2-\Gamma_{12,1}^2+\Gamma_{11}^2\Gamma_{22}^2+\Gamma_{11}^1\Gamma_{12}^2-\Gamma_{12}^1\Gamma_{11}^2-(\Gamma_{12}^2)^2\Bigr) \]
        Stunning.
        So now lets compute the Christoffel symbols.
        \begin{alignat*}{3}
            \Gamma_{11}^1 &=& \frac12g^{11}g_{11,1} &= \tanh(t) \\
            \Gamma_{12}^1 = \Gamma_{21}^1 &=& \frac12g^{11}g_{11,2} &= 0 \\
            \Gamma_{22}^1 &=& -\frac12g^{11}g_{22,1} &= -\tanh(t) \\
            \Gamma_{11}^2 &=& -\frac12g^{22}g_{11,2} &= 0 \\
            \Gamma_{12}^2 = \Gamma_{21}^2 &=& \frac12g^{22}g_{22,1} &= \tanh(t) \\
            \Gamma_{22}^2 &=& \frac12g^{22}g_{22,2} &= 0
        \end{alignat*}
        So we get
        \[ K = \cosh(t)^{-2}\Bigl(-\cosh(t)^{-2}+\tanh(t)^2-\tanh(t)^2\Bigr) = -\frac1{\cosh(t)^4} \]
        as required.

        \item Since $g=\cosh(t)\delta^i_j$, we have that, by computing the Laplace-Beltrami operator (which we will do in question six):
        \[ K = \frac{f_t^2-f\cdot f_{tt}}{f^4} = \frac{\sinh(t)^2-\cosh(t)^2}{\cosh(t)^4} = -\frac1{\cosh(t)^4} \]
    \eenum

\eblank

\bexerc

    Represent the following in terms $\Gamma_{ij}^k,S^i_j,b_{ij},g_{ij}$:
    \benum
        \item $\iprod{x_{ij},x_\ell}g^{\ell i}$
        \item $\iprod{x_{ij\ell},n}$
        \item $\iprod{x_{ij},n_\ell}\delta^j_m$
        \item $g_{pq}\delta^q_sg^{st}\delta^p_t$
        \item $\iprod{x_{ij},n_k}\delta^k_mg^{m\ell}$
        \item $\iprod{n_i,x_j}g^{i\ell}$
        \item $\iprod{n_i,n_j}$
        \item $\iprod{n,n_{ab}}\delta^a_c$
        \item $\abs{x_{ij}}^2$
        \item $\iprod{x_{ij},x_k}\delta^k_mg^{m\ell}$
    \eenum

\eexerc

\bblank

    \benum
        \item Recall that the definition of the Christoffel symbols (and $b_{ij}$) are
        \[ x_{ij} = \Gamma_{ij}^k x_k + b_{ij}N \]
        so
        \[ \iprod{x_{ij},x_{\ell}} = \Gamma_{ij}^k\iprod{x_k,x_\ell} = \Gamma_{ij}^kg_{k\ell} \]
        Thus
        \[ \iprod{x_{ij},x_\ell}g^{\ell i} = \Gamma_{ij}^kg_{k\ell}g^{\ell i} = \Gamma_{ij}^k\delta^i_k = \Gamma_{ij}^i \]

        \item Since $x_{ij}=\Gamma_{ij}^kx_k+b_{ij}n$, we have
        \[ x_{ij\ell} = \Gamma_{ij,\ell}^kx_k + \Gamma_{ij}^k x_{k\ell} + b_{ij,\ell}n + b_{ij}n_{\ell} \]
        Since $n$ is a unit vector, its derivative is orthogonal to it (and so are $x_k$), so
        \[ \iprod{x_{ij\ell},n} = \Gamma_{ij}^k\iprod{x_{k\ell},n} + b_{ij,\ell} = \Gamma_{ij}^kb_{k\ell} + b_{ij,\ell} \]

        \item Here, this is equal to $\iprod{x_{ij},n_\ell}$.
        Now, we have that
        \[ b_{ij,\ell} = \iprod{x_{ij,\ell},n} + \iprod{x_{ij},n_\ell} \]
        And so we have that from the previous question,
        \[ \iprod{x_{ij},n_\ell} = b_{ij,\ell} - \iprod{x_{ij,\ell},n} = -\Gamma_{ij}^kb_{k\ell} \]

        \item Simplifying gives
        \[ g_{pq}\delta^q_s g^{st}\delta^p_t = g_{ps}g^{sp} = \delta^p_p = 3 \]
        \item We showed already that $\iprod{x_{ij},n_k}=-\Gamma_{ij}^kb_{k\ell}$ so this is equal to
        \[ -\Gamma_{ij}^kb_{k\ell}g^{k\ell} \]
        Recall that $S=g^{-1}B$ and so
        \[ = -\Gamma_{ij}^k S^k_k \]

        \item We know that $\iprod{n,x_j}=0$ and so $\iprod{n_i,x_j}+\iprod{n,x_{ij}}=0$ meaning $\iprod{n_i,x_j}=-b_{ij}$.
        So this is equal to
        \[ -b_{ij}g^{i\ell} = -g^{\ell i}b_{ij} = -S^\ell_j \]
        ($g$ and therefore $g^{-1}$ is symmetric and so $g^{i\ell}=g^{i\ell}$).

        \item Recall that $S(x_i)=-n_i$ and so the $i$th column of $S$ represents $-n_i$ with respect to the basis $x_1,x_2$ meaning
        \[ -n_i = S^j_ix_j \]
        Thus
        \[ \iprod{n_i,n_j} = \iprod{-n_i,-n_j} = \iprod{S^k_ix_k,S^\ell_jx_\ell} = S^k_iS^\ell_j g_{k\ell} \]
        (To satisy Einstein notation perhaps I should've swapped the placement of the indexes on $S$, which is symmetric.)
        Now, we know that $S^\ell_j=g^{\ell t} b_{tj}$ and so this is equal to
        \[ g_{k\ell}g^{\ell t}b_{tj}S^k_i = \delta^k_t b_{tj}S^k_i = b_{kj}S^k_i \]

        \item This is equal to $\iprod{n,n_{ab}}$.
        We know that $\iprod{n,n_a}=0$ as $n$ is a unit vector and thus orthogonal to its derivative.
        Thus $\iprod{n_b,n_a}+\iprod{n,n_{ab}}=0$ and so
        \[ \iprod{n,n_{ab}} = -\iprod{n_a,n_b} = -b_{kb}S^k_a \]

        \item Since $x_{ij}=\Gamma_{ij}^kx_k+b_{ij}n$ we get
        \[ \abs{x_{ij}}^2 = \iprod{x_{ij},x_{ij}} = \iprod{\Gamma_{ij}^kx_k+b_{ij}n,\Gamma_{ij}^\ell x_\ell+b_{ij}n} = \Gamma_{ij}^k\Gamma_{ij}^\ell g_{k\ell} + b_{ij}^2 \]
        Since $n$ is orthogonal to $x_k$ and is a unit vector.

        \item This is equal to
        \[ \iprod{x_{ij},x_k}g^{k\ell} \]
        Now, again we have that $x_{ij}=\Gamma_{ij}^tx_t+b_{ij}n$ and so this is equal to
        \[ \iprod{\Gamma_{ij}^tx_t+b_{ij}n,x_k}g^{k\ell} = \Gamma_{ij}^tg_{tk}g^{k\ell} = \Gamma_{ij}^t\delta^t_\ell = \Gamma_{ij}^\ell \]
    \eenum

\eblank

\def\lapbel{\Delta_{\mathrm{LB}}}
\bexerc

    Suppose we have a surface whose metric is given by
    \[ g_{ij} = \lambda(x,y)\delta^i_j = f(x,y)^2\delta^i_j \]
    Show that the Gaussian curvature of the surface is given by
    \[ K = -\frac12\lapbel(\log\lambda) = -\lapbel(\log f) \]
    Where $\lapbel$ is the Laplace-Beltrami operator
    \[ \lapbel(h) = \frac1\lambda\parens{\frac{\partial^2h}{\partial x^2} + \frac{\partial^2h}{\partial y^2}} \]

\eexerc

\bblank

    Firstly, it is obvious that $\lapbel$ is linear and so
    \[ -\frac12\lapbel(\log\lambda) = -\frac12\lapbel(2\log f) = -\lapbel(\log f) \]
    So we need only to show that this is equal to the Gaussian curvature.
    We know that
    \[ -\lapbel(\log f) = -\frac1{f^2}\parens{\frac{\partial^2}{\partial x^2}\log(f) + \frac{\partial^2}{\partial y^2}\log(f)} =
    -\frac1{f^2}\parens{\frac\partial{\partial x}\frac{f_x}f + \frac\partial{\partial y}\frac{f_y}f} = -\frac{f_{xx}f + f_{yy}f - f_x^2 - f_y^2}{f^4} \]
    So we need to show
    \[ K = \frac{f_x^2 + f_y^2 - f(f_{xx} + f_{yy})}{f^4} \]

    Computing the Christoffel symbols:
    \begin{alignat*}{3}
        \Gamma_{11}^1 &=& \frac12g^{11}g_{11,1} &= \frac{f_x}f \\
        \Gamma_{12}^1 = \Gamma_{21}^1 &=& \frac12g^{11}g_{11,2} &= \frac{f_y}f \\
        \Gamma_{22}^1 &=& -\frac12g^{11}g_{22,1} &= -\frac{f_x}f \\
        \Gamma_{11}^2 &=& -\frac12g^{22}g_{11,2} &= -\frac{f_y}f \\
        \Gamma_{12}^2 = \Gamma_{21}^2 &=& \frac12g^{22}g_{11,2} &= \frac{f_x}f \\
        \Gamma_{22}^2 &=& \frac12g^{22}g_{22,2} &= \frac{f_y}f
    \end{alignat*}
    Now recall that the Gaussin curvature is given by
    \[ K = g^{11}\Bigl(\Gamma_{11,2}^2-\Gamma_{12,1}^2+\Gamma_{11}^2\Gamma_{22}^2+\Gamma_{11}^1\Gamma_{12}^2-\Gamma_{12}^1\Gamma_{11}^2-(\Gamma_{12}^2)^2\Bigr) \]
    Computing the derivatives of the necessary Christoffel symbols:
    \begin{align*}
        \Gamma_{11,2}^2 &= -\frac{f_{yy}f-f_y^2}{f^2} \\
        \Gamma_{12,1}^1 &= \frac{f_{xx}f-f_x^2}{f^2}
    \end{align*}
    And so plugging this in gives
    \[ K = \frac1{f^2}\parens{\frac{f_x^2+f_y^2-f(f_{xx}+f_{yy})}{f^2} - \frac{f_y^2}{f^2} + \frac{f_x^2}{f^2} + \frac{f_y^2}{f^2} - \frac{f_x^2}{f^2}} = \frac{f_x^2+f_y^2-f(f_{xx}+f_{yy})}{f^4} \]
    as required.

\eblank

\bexerc

    Given the metric
    \[ g_{ij} = \frac1y\delta^i_j \]
    for $y>0$, find the Gaussian curvature.

\eexerc

\bblank

    We know, by above,
    \[ K = -\frac12\lapbel(-\log y) = \frac12y\parens{\frac{\partial^2}{\partial x^2}\log(y) + \frac{\partial^2}{\partial y^2}\log(y)} = -\frac1{2y} \]

\eblank

\bexerc

    Given the metric
    \[ g = \pmat{1 & 0 \\ 0 & y} \]
    for $y>0$, find the Gaussian curvature.

\eexerc

\bblank

    This is in polar coordinates, so we can apply the simplified formula from Theorem Egregium:
    \[ K = \frac1G\Bigl(\Gamma^1_{22,1} - \Gamma_{12}^2\Gamma_{22}^1\Bigr) \]
    Here we have
    \[ \Gamma^1_{22} = -\frac12g^{11}g_{22,1} = 0,\quad \Gamma^2_{12} = \frac12g^{22}g_{22,1} = 0 \]
    and so $K=0$.

\eblank

\bexerc

    Find the normal and geodesic curvatures of the curve
    \[ \beta(t) = (\cos t,\sin t,1) \]
    on the parabaloid
    \[ z = x^2 + y^2 \]
    \benum
        \item By taking the inner product of $\beta''$ with $\beta'\times n$ and using the Pythagorean theorem.
        \item Using the second fundamental form, $\kappa_n=\sffof{\beta',\beta'}$ and the Pythagorean theorem.
        %\item By projecting $\beta''$ onto the tangent plane using
        %    \[ \beta'' = (\ddot\alpha^k + \Gamma_{ij}^k\dot\alpha^i\dot\alpha^j)x_k + \dot\alpha^i\dot\alpha^jb_{ij}n \]
        %    Where $\alpha$ is $\beta$'s coordinates.
    \eenum

\eexerc

\bblank

    \benum
        \item Notice that 
            \[ \beta'(t) = (-\sin t,\cos t,0),\quad \beta''(t) = (-\cos t,-\sin t,0) \]
            And so $\norm{\beta'}=1$, so $\beta$ is a natural parameterization.
            Thus recall that
            \[ \kappa_g = \abs{\iprod{\beta'',\beta'\times n}} \]
            So we must find $n$.
            We parameterize the surface by
            \[ \sigma(u,v) = (u,v,u^2+v^2) \]
            Thus it has partial derivatives
            \[ \sigma_1 = (1,0,2u),\quad \sigma_2 = (0,1,2v) \]
            And so we have
            \[ \sigma_1\times\sigma_2 = (-2u,-2v,1) \implies n = \frac1{\sqrt{4u^2+4v^2+1}}(-2u,-2v,1) \]
            Since we are on the curve $\beta$, we have $u=\cos(t)$ and $v=\sin(t)$, thus
            \[ n = \frac1{\sqrt5}(-2\cos t,-2\sin t,1) \]
            And finally
            \[ \beta'\times n = \frac1{\sqrt5}(\cos t,\sin t,2) \]
            Thus we have
            \[ \kappa_g = \abs{\frac1{\sqrt5}(-cos^2t-sin^2t)} = \frac1{\sqrt5} \]
            And since $\kappa=\abs{\beta''}=1$ and $\kappa_n^2+\kappa_g^2=\kappa^2$ we have
            \[ \kappa_n = \frac2{\sqrt5} \]

        \item Here we need to represent $\beta'$ with respect to the basis $\sigma_1$ and $\sigma_2$ and compute the second fundamental form $(b_{ij})$.
            Firstly, we have
            \[ \sigma_1 = (1,0,2\cos t),\quad \sigma_2 = (0,1,2\sin t) \]
            Thus
            \[ \beta' = (-\sin t,\cos t,0) = -\sin(t)\sigma_1 + \cos(t)\sigma_2 \]
            So $\beta'$'s representation is $(-\sin(t),\cos(t))$.
            Now, $b_{ij}=\iprod{\sigma_{ij},n}$ and so
            \[ \sigma_{11} = (0,0,2),\quad \sigma_{12} = (0,0,0),\quad \sigma_{22} = (0,0,2) \]
            And thus
            \[ b_{11} = b_{22} = \frac2{\sqrt5},\qquad b_{12} = b_{21} = 0 \]
            So we have
            \[ \kappa_n = \frac2{\sqrt5}(-\sin(t),\cos(t))\pmat{1 & 0 \\ 0 & 1}\pmat{-\sin(t) \\ \cos(t)} = \frac2{\sqrt5} \]
            And since $\kappa=1$ we get by the Pythagorean theorem, $\kappa_g=\frac1{\sqrt5}$.
    \eenum

\eblank

\bexerc

    Suppose two surfaces have the same second-order Taylor series at a point.
    Show then that they have the same Gaussian curvature.

\eexerc

\bblank

    Suppose our two surfaces are $\sigma$ and $\tau$, and our point is $p=\sigma(a,b)=\tau(c,d)$.
    Then the Taylor series of $\sigma$ at $p$ is
    \[ \sigma(a+h,b+k) = p + (h\sigma_1 + k\sigma_2) + \frac12\bigl(h^2\sigma_{11} + 2hk\sigma_{12} + k^2\sigma_{22}\bigr) \]
    and $\tau$'s is
    \[ \tau(c+h,d+k) = p + (h\tau_1 + k\tau_2) + \frac12\bigl(h^2\tau_{11} + 2hk\tau_{12} + k^2\tau_{22}\bigr) \]
    Since these are equal, this means we can equate the factors and so we have that
    \[ \sigma_i = \tau_i,\qquad \sigma_{ij} = \tau_{ij} \]
    for all $i,j$.
    Since the first fundamental form is determined by the values of $\sigma_i$ (it is $g_{ij}=\iprod{\sigma_i,\sigma_j}$), this means that at $p$, both $f$ and $g$ have the same first fundamental form.
    This in and of itself is not enough to say that they have the same Gaussian curvature, as while the curvature is determined by the first fundamental form and its derivatives, we only have equality at
    a point, not a neighborhood.
    So we cannot take derivatives.
    But, the unit normal $n$ (which is the normalized $\sigma_1\times\sigma_2$) are also the same for $\sigma$ and $\tau$ (as in we can choose an orientation where they are the same, but since Gaussian
    curvature is intrinsic, this choice doesn't affect the result).
    And since the second fundamental form is dependent only on $\sigma_{ij}$ and $n$ (it is $b_{ij}=\iprod{\sigma_{ij},n}$), we have that the second fundamental form of both surfaces at $p$ are equal
    (as in we can choose an orientation for them to be equal).

    Since the Gaussian curvature is given by the ratio between the determinants of the second and first fundamental forms, which are the same at $p$ for both $\sigma$ and $\tau$, the Gaussian curvature of
    $\sigma$ and $\tau$ at $p$ are equal.

\eblank

\bexerc

    \benum
        \item Suppose $\beta$ is the natural parameterization of a curve on the surface $M$.
            Suppose $\beta(0)=p$ and $\beta'(0)=v$.
            Prove that at $p$, $\kappa_n=\sffof{v,v}=\iprod{Sv,v}$.
        \item Suppose $v$ is a unit vector which is an eigenvector of the shape operator with an eigenvalue of $\kappa$ in the tangent space $T_pM$.
            Suppose $\beta(s)$ is a geodesic on $M$ which satisfies $\beta(0)=p$ and $\beta'(0)=v$.
            Prove that $\kappa_\beta(0)=\abs\kappa$, that the curvature of $\beta$ at $0$ is equal to $\abs\kappa$.
        \item Suppose $\kappa_1\leq\kappa_2$ are the eigenvalues of the shape operator.
            Let $\beta(s)$ be a natural parameterization geodesic, where $\beta(0)=p$.
            Prove that $\kappa_1\leq\kappa_n\leq\kappa_2$.
    \eenum

\eexerc

\bblank

    \benum
        \item The normal curvature of $\beta$ is defined to be the coefficient of $n$ in $\beta$'s representation in the orthonormal basis $\beta',n,\beta'\times n$.
            So
            \[ \beta'' = c\beta' + \kappa_n n + \kappa_g\beta'\times n \]
            We know that $\iprod{\beta',n}=0$ and so $\kappa_n=\iprod{\beta'',n}=-\iprod{\beta',n'}$.
            So at $p$, we have $\kappa_n=-\iprod{v,n'(0)}$ which is equal to $\sffof{v,v}$.
        \item Since $\beta$ is a geodesic, $\norm{\beta'}$ is constant.
            Since $\norm{\beta'(0)}=\norm v=1$, $\beta$ is a natural parameterization.
            Therefore $\kappa_\beta(0)^2=\kappa_g^2+\kappa_n^2$.
            Since $\beta$ is a geodesic, its geodesic curvature is zero, and thus we have $\kappa_\beta(0)=\abs{\kappa_n}$.
            Now, by the previous question
            \[ \kappa_n = \sffof{v,v} = \iprod{S(v),v} = \iprod{\kappa v,v} = \kappa\iprod{v,v} = \kappa \]
            which means that
            \[ \kappa_\beta(0) = \abs\kappa \]
            as required.
        \item We showed in lecture that
            \[ \kappa_1 = \min_{\norm v=1}\iprod{Sv,v},\qquad \kappa_2 = \max_{\norm v=1}\iprod{Sv,v} \]
            Since we know that
            \[ \kappa_n = \iprod{Sv,v} \]
            we have that $\kappa_1\leq\kappa_n\leq\kappa_2$ as required.
    \eenum

\eblank

\bexerc

    Let us define the surface $M$ to be the graph of the function
    \[ f(x,y) = 3x^2 + 8xy - 3y^2 \]
    Let $e_1,e_2,e_3$ be the standard basis for $\bR^3$.
    \benum
        \item Find the Hessian of $f$ at the origin.
        \item Let $\lambda_1,\lambda_2$ be $H_f$'s eigenvalues and $v_1,v_2$ corresponding eigenvectors.
            We define $E_i\subseteq\bR^3$ as the hyperplane spanned by $v_i$ and $e_3$.
            We define $\gamma_i=M\cap E_i$.

            Find the curvature of the planar curves $\gamma_i$ at the origin.
        \item Find the shape operator of $M$ at the origin and find the Gaussian curvature of $M$ at the origin.
        \item Find the mean curvature of $M$ at the origin.
    \eenum

\eexerc

\bblank

    \benum
        \item Computing the second derivatives of $f$ yields
            \[ H_f = \pmat{6 & 8 \\ 8 & -6} \]
            this is true for every point, particularly at the origin.
        \item The characteristic polynomial of $H_f$ is
            \[ (x-6)(x+6) - 64 = x^2 - 100 \]
            and so $\lambda_i=\pm10$.
            The eigenvectors are
            \[ v_1 = (1,-2),\qquad v_2 = (1,2) \]
            So we have that
            \[ E_1 = \lspanof{\pmat{1\\-2\\0},\pmat{0\\0\\1}} = \set{\pmat{u\\-2u\\v}} \]
            The intersection of $E_1$ with $M$ requires
            \[ v = 3u^2 - 16u^2 - 12u^2 = -25u^2 \]
            So
            \[ \gamma_1 = \pmat{t\\-2t\\-25t^2} \]
            Since the curvature of a curve is given by
            \[ \kappa = \frac{\norm{\gamma'\times\gamma''}}{\norm{\gamma'}^3} \]
            we compute $\gamma_1$'s derivatives:
            \[ \gamma_1' = (1,-2,-50t),\qquad \gamma_2' = (0,0,-50) \]
            The cross product of these gives $(100,50,0)=50(2,1,0)$ and so 
            \[ \kappa = \frac{50\sqrt5}{(5+2500t^2)^{3/2}} = \frac{10}{(1+500t)^{3/2}} \]
            So when $t=0$, $\gamma_1(0)=0$ and thus the curvature is equal to $10$.

            Now for $E_2$:
            \[ E_2 = \lspanof{\pmat{1\\2\\0},\pmat{0\\0\\1}} = \set{\pmat{u\\2u\\v}} \]
            And so the intersection requires
            \[ v = 3u^2 + 16u^2 - 12u^2 = 7u^2 \]
            So
            \[ \gamma_2(t) = \pmat{t\\2t\\7t^2} \]
            And
            \[ \gamma'_2(t) = (1,2,14t),\qquad \gamma''_2(t) = (0,0,14) \]
            And so the cross product yields $(28,-14,0)=14(2,-1,0)$ and we get
            \[ \kappa = \frac{14\sqrt5}{(5+196t^2)^{3/2}} \]
            So at the origin,
            \[ \kappa = \frac{14\sqrt5}{5\sqrt5} = \frac{14}{\sqrt5} \]

        \item We showed that for the graph of a function,
            \[ g = \pmat{1+f_x^2 & f_xf_y \\ f_xf_y & 1+f_y^2},\qquad B = \frac1{\sqrt{1+f_x^2+f_y^2}}\pmat{f_{xx} & f_{xy} \\ f_{xy} & f_{yy}} \]
            In this case, at the origin $f_x=0$, $f_y=0$, $f_{xx}=6$, $f_{xy}=8$, and $f_{yy}=-6$.
            So we get
            \[ g = \pmat{1 & 0 \\ 0 & 1},\qquad B=\pmat{6 & 8 \\ 8 & -6} \]
            So we get that
            \[ S = g^{-1}B = \pmat{6 & 8 \\ 8 & -6} \]
            The Gaussian curvature is the determinant of this (the product of the eigenvalues)
            \[ K = -36-64 = -100 \]
        \item The mean curvature is half the trace of the shape operator, so
            \[ H = 0 \]
    \eenum

\eblank

\bexerc

    Suppose $C>0$ is a positive constant, and $f(x,y)$ is a function satisfying
    \[ f(x,y) \geq C(x^2+y^2),\quad f(0,0) = 0 \]
    \benum
        \item Find a lower bound on the eigenvalues of $H_f(0,0)$.
        \item Find a lower bound on the Gaussian curvature of the graph of $f$ at the origin.
    \eenum

\eexerc

\bblank

    \benum
        \item Notice that for every $(x,y)\neq(0,0)$,
            \[ f(x,y) \geq C(x^2+y^2) > 0 = f(0,0) \]
            So $(0,0)$ is a minimum of $f$, and in particular it is a critical point.
            We know that at critical points, the second fundamental form and the shape operator of the graph of $f$ is equal to the Hessian of $f$.
            So the eigenvalues of the Hessian at the origin are the principal directions of the graph at the origin.

            Recall that the principal directions are the minimum and maximum curvatures of the intersection of the graph with a normal plane.
            Meaning for every unit vector $w\in T_pM$ (here $p=(0,0,f(0,0))=(0,0,0)$), we define $\gamma_w$ to be the natural parameterization of the curve obtained by intersecting $M$ with the plane
            $p+\lspanof{w,n}$.
            Suppose that $\gamma_w(0)=p$, then the principal directions are
            \[ \kappa_1 = \min_{\substack{w\in T_pM\\\norm w=1}}\kappa_{\gamma_w}(0),\qquad \kappa_2 = \max_{\substack{w\in T_pM\\\norm w=1}}\kappa_{\gamma_w}(0) \]
            We showed this in Euler's theorem.
            So we must find a lower bound of $\kappa_{\gamma_w}(0)$.
            By definition,
            \[ \kappa_{\gamma_w}(0) = \norm{\gamma_w''(0)} \]

            Now, suppose we define $M_1$ to be the graph of $f$, and $M_2$ to be the graph of $C(x^2+y^2)$.
            Both $f$ and $C(x^2+y^2)$ have minimums at $(0,0)$ and so the tangent plane at $(0,0,0)$ for both of them is $T_0M_1=T_0M_2=\lspanof{(1,0,0),\,(0,1,0)}$.
            So for every unit vector $w$ in this plane we have the curve $\gamma_w^1$ on $M_1$ and $\gamma_w^2$ on $M_2$.
            Since $f\geq C(x^2+y^2)$ we have (since $\gamma_w^i$ are planar curves, we can view them as real functions)
            \[ \gamma_w^1(t) \geq \gamma_w^2(t),\qquad \gamma_w^1(0) = \gamma_w^2(0) = (0,0,0) \]
            And so we get that, using this equality and inequality (if $f\geq g$ and $f(0)=g(0)$ then $f'(t)\geq g'(t)$ in a neighbrohood of $0$)
            \[ {\gamma_w^1}'(t) \geq {\gamma_w^2}'(t),\qquad {\gamma_w^1}'(0) = {\gamma_w^2}'(0) \]
            The derivatives at $0$ are equal since $(\gamma_w^i)'(0)=w$.
            So again, we get that
            \[ {\gamma_w^1}''(t) \geq {\gamma_w^2}''(t) \]
            And thus the curvature of $\gamma_w^1$ is greater than that of $\gamma_w^2$.
            And so we have
            \[ \kappa_1^1 = \min_{\norm w=1}\kappa_{\gamma_w^1} \geq \min_{\norm w=1}\kappa_{\gamma_w^2} = \kappa_1^2 \]
            \[ \kappa_2^1 = \max_{\norm w=1}\kappa_{\gamma_w^1} \geq \max_{\norm w=1}\kappa_{\gamma_w^2} = \kappa_2^2 \]
            where $\kappa_j^i$ is the $j$th principal curvature of the surface $M_i$.

            So if we can find the principal curvatures of the graph of $C(x^2+y^2)$, then we can find lower bounds for the principal curvatures of the graph of $f$.
            Since $(0,0,0)$ is a critical point of $C(x^2+y^2)$, its shape operator at the origin is its Hessian:
            \[ S = H_f = C\pmat{2 & 0 \\ 0 & 2} \]
            So the principal curvatures of $C(x^2+y^2)$ are the eigenvalues of $S$, which are both $2C$.
            So we have that the lower bounds for both $\kappa_1$ and $\kappa_2$ (the principal curvatures of the graph of $f$) are $2C$.

        \item The Gaussian curvature of a manifold is equal to the product of its principal curvatures.
            Thus we have
            \[ K = \kappa_1\kappa_2 \geq 2C\cdot2C = 4C^2 \]
    \eenum

\eblank

\end{document}
