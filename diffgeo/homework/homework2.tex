\documentclass[10pt]{article}

\usepackage{amsmath, amssymb, mathtools}
\usepackage[margin=1.5cm]{geometry}

\input prettyprint
\input preamble
\input pdfmsym

\pdfmsymsetscalefactor{10}
\initpps

\def\pmat#1{\begin{pmatrix} #1 \end{pmatrix}}

\let\divides=\mid
\newfunc{det}{{\rm det}}({})
\newfunc{metric}\rho({})
\newfunc{metricc}\sigma({})
\newfunc{spa}{{\rm span}}(\vert)
\newfunc{diam}{{\rm diam}}(\vert)
\newfunc{proj}\pi({})
\newfunc{iproj}{\pi^{-1}}({})
\newfunc{cis}{{\rm cis}}({})
\newfunc{Re}{{\rm Re}}({})
\newfunc{Im}{{\rm Im}}({})
\newfunc{sup}{{\rm sup}}\{\vert\}
\newfunc{Res}{{\rm Res}}({})
\newfunc{wind}n({})
\newfunc{pv}{{\rm pv}}({})
\newfunc{lspan}{{\rm span}}\{|\}
\newfunc{spec}{{\rm spec}}({})

\font\bigbf = cmbx12 scaled 2000
\@undervecc@def{underbar}\@linecap\@linecap

\def\pmat#1{\begin{pmatrix}#1\end{pmatrix}}

\def\mO{{\cal O}}
\def\mU{{\cal U}}
\let\lineseg=\overleftrightvecc
\let\to=\varrightarrow
\let\longto=\longvarrightarrow
\let\ds=\displaystyle

\def\pdv#1#2{\frac{\partial #1}{\partial #2}}

\def\differ#1#2{\left.d#1\strut\right|_{#2}}

\def\qed{%
    \ifmmode%
        \eqno\blacksquare%
    \else%
        \hskip1cm\allowbreak\hbox{}\nobreak\hfill$\blacksquare$%
    \fi%
}

\def\@ppmathcount{\thesection.\thepp@mathcount}

\def\bexerc{\begin{exercise*}}
\def\eexerc{\end{exercise*}}
\def\bblank{\begin{blankpp}}
\def\eblank{\end{blankpp}}

\begin{document}

\c@section=2

\barcolorbox{220, 255, 220}{0, 130, 0}{80, 200, 80}{
    \leftskip=0pt plus 1fill \rightskip=\leftskip
    {\bigbf Differential and Analytic Geometry}

    \medskip
    \textit{Assignment \thesection}

    \textit{Ari Feiglin}
}

\bigskip

\bexerc

    Show that there are no points on the plane which satisfy
    \[ 3x^2 + 2xy + 3y^2 + 2x - 6y + 12.5 = 0 \]

\eexerc

\bblank

    We define the matrix
    \[ A = \pmat{3 & 1 \\ 1 & 3} \]
    which has a characteristic polynomial of
    \[ p_A(x) = (x-3)^2 - 1 = x^2 - 6x + 8 \]
    and therefore its spectrum is $\specof A=\set{4,2}$.
    The eigenspaces are
    \[ V_4 = N\pmat{1 & -1 \\ -1 & 1} = \lspanof{\pmat{1\\1}},\qquad V_2 = N\pmat{1&1\\1&1} = \lspanof{\pmat{1\\-1}} \]
    Thus the orthogonal diagonalizer of $A$ is
    \[ P = \frac1{\sqrt2}\pmat{1&1\\1&-1} \]
    And so we have that
    \[ (d',e') = \frac1{\sqrt2}(2,-6)\pmat{1&1\\1&-1} = \frac1{\sqrt2}(-4,8) \]
    And so the transformed equation is
    \[ 4t^2 + 2s^2 - 2\sqrt2 t + 4\sqrt2 s + 12.5 = 0 \]
    Completing the squares yields
    \[ 4\parens{t-\frac{\sqrt2}4}^2 + 2\parens{s+\sqrt2}^2 + 8 = 0 \]
    This has no solutions (since the sum of two squares cannot be negative), so the original equation has no solutions.

\eblank

\bexerc

    Show that the following equation defines two intersecting lines,
    \[ 4x^2 - 24xy - 6y^2 + 4x - 12y + 1 = 0 \]
    find their point of intersection and the angle between them.

\eexerc

\bblank

    Again, let we define the following matrix as per our algorithm for solving quadratic equations,
    \[ A = \pmat{4&-12\\-12&-6} \]
    This has a characteristic polynomial of
    \[ p_A(x) = (x-4)(x+6) - 144 = x^2 + 2x - 168 \]
    and therefore $\specof A=\set{12,-14}$.
    The eigenspaces are
    \[ V_{12} = N\pmat{8&12\\12&18} = \lspanof{\pmat{3\\-2}},\qquad V_{-14} = N\pmat{18&-12\\-12&8} = \lspanof{\pmat{2\\3}} \]
    Thus the orthogonal diagonalizer of $A$ is
    \[ P = \frac1{\sqrt{13}}\pmat{3&2\\-2&3} \]
    And so we have that
    \[ (d',e') = \frac1{\sqrt{13}}(4,-12)\pmat{3&2\\-2&3} = \frac1{\sqrt{13}}(36,-28) \]
    And so the transformed equation is
    \[ 12t^2 - 14s^2 + \frac{36}{\sqrt{13}}t - \frac{28}{\sqrt{13}}s + 1 = 0 \]
    Where
    \[ \pmat{t\\s} = P^\top\pmat{x\\y} = \frac1{\sqrt{13}}\pmat{3x-2y\\2x+3y} \]
    Completing the squares yields
    \[ 12\parens{t+\frac3{2\sqrt{13}}}^2 - 14\parens{s+\frac1{\sqrt{13}}}^2 = 0 \]
    Which gives the two lines
    \[ \sqrt{12}\parens{t+\frac3{2\sqrt{13}}} = \pm\sqrt{14}\parens{s+\frac1{\sqrt{13}}} \]
    Plugging in the values for $t$ and $s$ gives
    \[ \sqrt{12}(3x-2y+1.5) = \pm\sqrt{14}(2x+3y+1) \iff (2\sqrt{12}\pm3\sqrt{14})y = (3\sqrt{12}\mp2\sqrt{14})x + 1.5\sqrt{12} \mp \sqrt{14} \]
    So our two lines are
    \[ y = \frac{-6+\sqrt{42}}3x + \frac{-6+\sqrt{42}}6,\qquad y = -\frac{6+\sqrt{42}}3x - \frac{6+\sqrt{42}}6 \]

    The point of intersection of these occurs when
    \[ \frac{2\sqrt{42}}3x = -\frac{\sqrt{42}}3 \implies x = -\frac12 \]
    And so $y=0$.
    So the point of intersection is $\parens{-\frac12,0}$.

    And the angle of the first line is
    \[ \tan\theta_1 = \frac{-6+\sqrt{42}}3 \implies \theta_1 = 0.1589 \]
    and the second
    \[ \tan\theta_2 = -\frac{6+\sqrt{42}}3 \implies \theta_2 = -1.3349 \]
    Thus the angle between the two lines is $\theta_1-\theta_2=1.4938$ radians.

\eblank

\bexerc

    Find a parameterization of the line from $(1,2)$ to $(3,4)$, $\gamma(t)$.
    Find $\gamma'(t)$ and its magnitude.

\eexerc

\bblank

    The parameterization is the function
    \[ \gamma\colon[0,1]\longto\bR^2,\qquad \gamma(t) = (1,2) + t\bigl((3,4)-(1,2)\bigr) = (1+2t,\,2+2t) \]
    And so
    \[ \gamma'(t) = (2,2) \]
    And so $\norm{\gamma'(t)}=\norm{(2,2)}=2\sqrt2$.

    This means that the arclength is
    \[ s_\gamma(t) = \int_0^t\norm{\gamma'(u)} = \int_0^t 2\sqrt{2} = 2t\sqrt2 \]

\eblank

\bexerc

    Find a parameterization $\gamma(t)$ of the unit circle.
    Compute $\gamma'(t)$ and show that it is always orthogonal to $\gamma(t)$.

\eexerc

\bblank

    The parameterization is the classic
    \[ \gamma\colon[0,2\pi]\longto\bR^2,\qquad \gamma(t) = \bigl(\cos(t),\sin(t)\bigr) \]
    And thus
    \[ \gamma'(t) = (-\sin(t),\cos(t)) \]
    Notice then that for evert $t$,
    \[ \iprod{\gamma(t),\gamma'(t)} = -\cos(t)\sin(t) + \sin(t)\cos(t) = 0 \]
    So for every $t$, $\gamma(t)\perp\gamma'(t)$ as required.

\eblank

\bexerc

    Find a parameterization for the hyperbola $y^2-x^2=1$ using trigonometric functions.

\eexerc

\bblank

    We know that for $x\neq\frac\pi2 k$ ($k\in\bZ$),
    \[ 1 = \frac{\cos^2(t)}{\cos^2(t)} = \frac{1-\sin^2(t)}{\cos^2(t)} = \parens{\frac1{\cos(t)}}^2 - \parens{\frac{\sin(t)}{\cos(t)}}^2 = \sec(t)^2 - \tan(t)^2 \]
    So let us define
    \[ \gamma\colon\parens{-\frac\pi2,\frac\pi2}\longto\bR^2,\quad \gamma(t) = \bigl(\tan(t),\sec(t)\bigr) \]
    Now, between $-\frac\pi2$ and $\frac\pi2$, $\cos(t)$ is positive, and so $\sec(t)>0$.
    Then by the identity above, $\gamma$'s image is contained within the top half of the hyperbola $y^2-x^2=1$.

    We will show that $\gamma$ is a parameterization for the top half ($y>0$) of the hyperbola.
    Suppose $(x,y)$ is on the hyperbola for $y>0$, then we need to find a $t\in\parens{-\frac\pi2,\frac\pi2}$ such that $\tan(t)=x$ and $\sec(t)=y$.
    Since $\tan(x)$ is surjective when restricted to $\parens{-\frac\pi2,\frac\pi2}$, so there does exist such a $t$.
    Now, since we know that $(x,y)$ is on the hyperbola, $y^2-x^2=1$, meaning
    \[ \sec(t)^2 - \tan(t)^2 = 1 \implies \sec(t)^2 = 1 + x^2 = y^2 \]
    and so $y=\pm\sec(t)$.
    Since $y>0$ and $\sec(t)>0$, we have that $y=\sec(t)$.

    So we have shown that for every $(x,y)$ on the top half of the hyperbola, there exists a $t$ such that $(x,y)=\gamma(t)$.
    And we have also shown that $\gamma$ is contained within the half of the hyperbola, so $\gamma$'s image is the top half of the hyperbola.
    So $\gamma$ is a parameterization of the top half of the hyperbola.

    Since the two sides of the hyperbola are symmetric, $-\gamma$ is a parameterization for the bottom half of the hyperbola (there is no parameterization for the entire hyperbola since intervals are
    connected, and the hyperbola is not, so there cannot exist a surjective continuous function from an interval to the hyperbola).

\eblank

\bexerc

    Let us look at the unit circle
    \[ S^1 = \set{(x,y)\in\bR^2}[x^2+y^2=1] \]
    and let us denote $N=(0,1)$ to be its northern point.
    For every other point $P\in S^1$ we denote $\ell_P$ to be the line which contains both $N$ and $P$.
    \benum
        \item Show that for every $P$, $\ell_P$ intersects the $x$-axis at a single point $P'$.
        Find a formula for $P'$ with respect to $P$.
        \item Let us define a function $\pi\colon S^1\setminus\set N\longto\bR$ which maps $P$ to the $x$ coordinate of the point of intersection of $\ell_P$ with the $x$-axis.
        Ie. $P'=(\pi(P),0)$.
        Show that $\pi$ is a bijection.
        \item Find a formula for $\pi^{-1}\colon\bR\longto S^{-1}\setminus\set N$.
        \item Let us define $\delta=\pi^{-1}$, find $\norm{\delta'(t)}$ and prove that $\delta$ is a smooth regular parameterization for $S^1\setminus\set N$.
        \item The standard parameterization of the unit circle is
        \[ \gamma\colon[0,2\pi]\longto\bR,\quad t\mapsto(\cos(t),\sin(t)) \]
        Show that $\gamma$ and $\delta$ are equivalent, that is there exists a function $h\colon[0,2\pi]\longto\bR$ such that $\delta\circ h=\gamma$ and $h'(t)>0$ for every $t\in[0,2\pi]$.
        Find $h$ explicitly.
        \item Prove that for every $P\in S^1\setminus\set N$, $P$ has rational coordinates if and only if $\pi(P)$ is rational.
        \item Prove that every point with rational coordinates on the unit circle is of the form
        \[ P = \parens{\frac{2mn}{m^2+n^2},\,\frac{m^2-n^2}{m^2+n^2}} \]
        where $n$ and $m$ are coprime integers.
        \item Find all the Pythagorean triplets.
    \eenum

\eexerc

\bblank

    \benum
        \item Suppose $P=(u,v)$, then $\ell_P$ is given by the formula
        \[ \ell_P(t) = N + t(P-N) = (tu,1-t+tv) \]
        And so $\ell_P(t)$ is on the $x$-axis if and only if $1+t(v-1)=0$, and since $P\neq N$, $v\neq1$ and so this is if and only if
        \[ t = \frac1{1-v} \]
        Which means that $P'$ is given by
        \[ P' = \ell_P(t) = \parens{\frac u{1-v},0} \]
        (which is a unique point.)

        \item So we have that
        \[ \pi(u,v) = \frac u{1-v} \]
        We will prove $\pi$ is bijective by finding $\pi^{-1}$, this will solve the next subquestion as well.
        Suppose $x\in\bR$, then in order to determine what point $P\in S^1$ is projected onto $x$ we simply must figure out which line traverses through $N$ and $(x,0)$ and where it intersects with $S^1$.
        This line is
        \[ \ell_P(t) = N + t((x,0) - N) = (tx,1-t) \]
        This is on $S^1$ when
        \[ t^2x^2 + (1-t)^2 = 1 \iff t^2x^2 + t^2 - 2t = 0 \]
        Now, this is if and only if $t=0$, which corresponds to the point $N$, or
        \[ tx^2 + t = 2 \iff t = \frac2{x^2+1} \]
        And so the point of intersection with $S^1$ (which is not $N$), is
        \[ \ell_P\parens{\frac2{x^2+1}} = \parens{\frac{2x}{x^2+1},\frac{x^2-1}{x^2+1}} \]
        Thus we claim that $\pi^{-1}$ is defined by
        \[ \pi^{-1}(x) = \parens{\frac{2x}{x^2+1},\frac{x^2-1}{x^2+1}} \]

        This is well-defined since all the transitions we made above were bidirectional, but we can also verify this directly:
        \[ \parens{\frac{2x}{x^2+1}}^2 + \parens{\frac{x^2-1}{x^2+1}}^2 = \frac{x^4-2x^2+1+4x^2}{(x^2+1)^2} = \frac{(x^2+1)^2}{(x^2+1)^2} = 1 \]
        so $\pi^{-1}(x)\in S^1$ as required (and not equal to $N$ as $\frac{2x}{x^2+1}=0$ means $x=0$, which means that $\frac{x^2-1}{x^2+1}=-1$, and so $\pi^{-1}(0)=(0,-1)$ and no other point can map to
        $N$).

        Now we must verify that $\pi\circ\pi^{-1}=\mathrm{id}_\bR$ and $\pi^{-1}\circ\pi=\mathrm{id}_{S^1\setminus\set N}$.
        We will verify this directly
        \[ \pi\circ\pi^{-1}(x) = \pi\parens{\frac{2x}{x^2+1},\frac{x^2-1}{x^2+1}} = \frac{\frac{2x}{x^2+1}}{1-\frac{x^2-1}{x^2+1}} = \frac{2x}{x^2+1-x^2+1} = x \]
        as required.
        And
        \[ \pi^{-1}\circ\pi(u,v) = \pi^{-1}\parens{\frac u{1-v}} = \parens{\frac{\frac{2u}{1-v}}{1+\frac{u^2}{1-2v+v^2}}, \frac{\frac{u^2}{1-2v+v^2}-1}{\frac{u^2}{1-2v+v^2}+1}}
        = \parens{\frac{2u(1-v)}{1-2v+v^2+u^2}, \frac{u^2-1+2v-v^2}{u^2+1-2v+v^2}} \]
        Now, since $(u,v)\in S^1$ we have that $u^2+v^2=1$ and so $1-2v+v^2+u^2=2-2v$, and so
        \[ \frac{2u(1-v)}{1-2v+v^2+u^2} = \frac{2u(1-v)}{2(1-v)} = u \]
        and similarly
        \[ u^2-v^2-1+2v = 1-2v^2-1+2v = 2v(1-v),\quad u^2+v^2+1-2v = 2-2v \]
        and so
        \[ \frac{u^2-1+2v-v^2}{u^2+1-2v+v^2} = \frac{2v(1-v)}{2(1-v)} = v \]
        Thus
        \[ \pi^{-1}\circ\pi(u,v) = (u,v) \]
        as required.
        Thus we have shown that
        \[ \pi^{-1}(x) = \parens{\frac{2x}{x^2+1},\,\frac{x^2-1}{x^2+1}} \]
        is $\pi$'s inverse.

        \item We found it in the previous subquestion.

        \item So we have
        \[ \delta(x) = \parens{\frac{2x}{x^2+1},\,\frac{x^2-1}{x^2+1}} \]
        So $\delta(x)$ is infinitely differentiable everywhere as rational functions are infinitely differentiable everywhere in their domain, and the domain of the rational functions in $\delta$ are defined
        everywhere.
        And
        \[ \delta'(x) = \parens{\frac{2(x^2+1)-4x^2}{(x^2+1)^2},\,\frac{2x(x^2+1)-2x(x^2-1)}{(x^2+1)^2}} = \frac1{(x^2+1)^2}\bigl(-2x^2+2, 4x\bigr) = \frac2{(x^2+1)^2}\bigl(1-x^2,2x\bigr) \]
        Therefore
        \[ \norm{\delta'(x)}^2 = \frac4{(x^2+1)^4}\bigl((1-x^2)^2+4x^2\bigr) = \frac{4(x^4+2x^2+1)}{(x^4+2x^2+1)^2} = \frac4{(x^2+1)^2} \]
        So $\norm{\delta'(x)}\neq0$, and therefore $\delta$ is indeed regular.

        \item Our goal is to find a function $h$ such that with the substitution $x=h(t)$,
        \[ \delta\circ h(t) = \gamma(t) \iff \parens{\frac{2x}{x^2+1},\,\frac{x^2-1}{x^2+1}} = (\cos(t),\,\sin(t)) \]
        Recall that if we substitute the universal substitution $x=\tan\frac t2$ then we get
        \[ \sin t =\frac{2x}{1+x^2},\quad \cos t=\frac{1-x^2}{x^2+1} \]
        So if we set $x=\tan\parens{\frac{\frac\pi2+t}2}=\tan\frac{\pi+2t}4$, then we get that
        \[ \frac{2x}{1+x^2} = \sin\parens{t+\frac\pi2} = \cos t \]
        and
        \[ \frac{1-x^2}{x^2+1} = \cos\parens{t+\frac\pi2} = -\sin t \implies \sin t = \frac{x^2-1}{x^2+1} \]
        Thus with the function
        \[ h(t) = \tan\parens{\frac{2t+\pi}4} \]
        then we get that
        \[ \delta\circ h = \gamma \]
        And
        \[ h'(t) = \frac12\cdot\frac1{\cos^2\parens{\frac{2t+\pi}4}} > 0 \]
        As required.

        \item If $P=(u,v)$ is a rational point, then $\pi(P)=\frac u{1-v}$ is rational (as the ratio between two rational numbers).
        Now, suppose $P=(u,v)$ and $\pi(P)=\frac u{1-v}$ is rational, suppose $\pi(P)=\frac ab$ for $a,b\in\bZ$.
        Thus we get that
        \[ b^2u^2 = a^2(1-v)^2 \iff b^2(1-v^2) = a^2(1-v)^2 \iff b^2(1+v) = a^2(1-v) \]
        And so we get that
        \[ v(b^2+a^2) = a^2-b^2 \implies v = \frac{a^2-b^2}{a^2+b^2} \]
        which is rational as the ratio of integers.
        And since $u=\frac ab(1-v)$, $u$ is also rational (as the product of rational numbers).

        Thus $P$ has rational coordinates if and only if $\pi(P)$ is rational, as required.

        \item Suppose $P\neq N$ is a rational point, then $\pi(P)=\frac mn$ for $m$ and $n$ coprime integers.
        Then
        \[ P = \pi^{-1}\parens{\frac mn} = \parens{\frac{2\cdot\frac mn}{\frac{m^2}{n^2}+1},\,\frac{\frac{m^2}{n^2}-1}{\frac{m^2}{n^2}+1}} = \parens{\frac{2mn}{m^2+n^2},\,\frac{m^2-n^2}{m^2+n^2}} \]
        as required.
        (For the case that $P=N$, then we take $m=1$ and $n=0$.)

        \item Suppose $a^2+b^2=c^2$ is a Pythagorean triple, then $\parens{\frac ac}^2+\parens{\frac bc}^2=1$, and so $P=\parens{\frac ac,\frac bc}$ is a rational point on $S^1$.
        Thus we have that
        \[ \frac ac = \frac{2mn}{m^2+n^2},\quad \frac bc = \frac{m^2-n^2}{m^2+n^2} \]
        for some coprime integers $n$ and $m$.
        Notice that this means that
        \[ a(m^2+n^2) = 2cmn \]
        This means that $m$ divides $a(m^2+n^2)$, and since it divides $am^2$, it must divide $an^2$.
        Since $m$ and $n$ are coprime, so is $m$ and $n^2$, thus $m$ must divide $a$.
        And similarly, $n$ must divide $a$.
        Since $m$ and $n$ are coprime, this means that $a$ is some multiple of $mn$, ie. $a=mnk$ for some integer $k$.

        Now,
        \[ \frac ba = \frac{m^2-n^2}{2mn} \implies b = \frac{m^2-n^2}2k \]
        And
        \[ c = \frac{m^2+n^2}2k \]
        These must be integers, and so if $2$ doesn't divide either $m$ or $n$, then $m^2\pm n^2$ are even and so we need no restrictions on $k$.
        But otherwise, $2$ can only divide one of $m$ and $n$ as they are coprime, so $m^2\pm n^2$ are odd, and so we require that $k$ be $2$.

        In any case, the $a$, $b$, and $c$ defined by this satisfies $a^2+b^2=c^2$:
        \[ a^2+b^2 = k^2\parens{m^2n^2 + \frac{m^4-2m^2n^2+n^4}4} = \frac{k^2}4\bigl(m^4+2m^2n^2+n^4\bigr) = \frac{k^2}4(m^2+n^2)^2 = c^2 \]

        Thus the set of Pythagorean triplets is
        \[ \set{\parens{mnk,\,\frac{m^2-n^2}2k,\,\frac{m^2+n^2}2k}}[\vcenter{\advance\hsize by-8cm\relax\leftskip=\z@
        $m$ and $n$ are coprime integers, and $k$ is some other integer such that if either $m$ or $n$ is even than so is $k$.}] \]
    \eenum

\eblank

\bexerc

    Find an implicit equation $F(x,y)=0$ for the following parameterizations:
    \benum
        \item $\alpha(t)=(\cos^2(t),\,\sin^2(t))$
        \item $\alpha(t)=(e^t,t^2)$
    \eenum

\eexerc

\bblank

    \benum
        \item Since $\cos^2(t)+\sin^2(t)=1$, we get that $x+y=1$ and so $F(x,y)=x+y-1$.
        \item Notice that
        \[ \log(e^t)^2 = t^2 \]
        So $F(x,y)=\log(x)^2-y$.
    \eenum

\eblank

\bexerc

    Compute the arclength of the curve $\alpha(t)=(e^t,e^{-t},\sqrt2t)$ where $t\in[0,1]$.

\eexerc

\bblank

    The arclength of the curve is given by
    \[ L = \int_0^1 \norm{\alpha'(t)}\,dt \]
    Now,
    \[ \alpha'(t) = (e^t,-e^{-t},\sqrt2) \]
    and so
    \[ \norm{\alpha'(t)} = \sqrt{e^{2t}+e^{-2t}+2} \]
    Thus
    \[ L = \int_0^1 \sqrt{e^{2t}+e^{-2t}+2} \]
    Let us substitute $u=e^{2t}$ and so $t=\frac12\log u$ and so $dt=\frac{du}{2u}$ thus
    \[ L = \frac12\int_1^{e^2} u^{-1}\sqrt{u+u^{-1}+2} = \frac12\int_1^{e^2}u^{-3/2}\sqrt{u^2+2u+1} = \frac12\int_1^{e^2} u^{-3/2}(u+1) = \frac12\int_1^{e^2} u^{-1/2} + u^{-3/2} \]
    Integrating gives
    \[ = \sqrt u - \frac1{\sqrt u}\biggl|_1^{e^2} = e - e^{-1} - 1 + 1 = e - e^{-1} \]
    Thus
    \[ L = e - \frac1e \]

\eblank

\bexerc

    Show that the following curves are natural parameterizations, and find their curvatures
    \benum
        \item $\alpha(t)=\parens{\frac13(1+t)^{3/2},\,\frac13(1-t)^{3/2},\,\frac t{\sqrt2}}$
        \item $\alpha(t)=\parens{\frac45\cos(t),\,1-\sin(t),\,-\frac35\cos(t)}$
    \eenum

\eexerc

\bblank

    \benum
        \item Note that $\alpha$ is only defined when $1+t\geq0$ and $1-t\geq0$, ie $-1\leq t\leq1$.
        So we compute
        \[ \alpha'(t) = \parens{\frac12(1+t)^{1/2},\,-\frac12(1-t)^{1/2},\,\frac1{\sqrt2}} \]
        Now, in order for $\alpha$ to be a natural parameterization we must have that $\norm{\alpha'(t)}=1$, and
        \[ \norm{\alpha'(t)}^2 = \frac14(1+t) + \frac14(1-t) + \frac12 = 1 \]
        as required.
        Now, since $\alpha$ is a natural parameterization its curvature is given by $\kappa(t)=\norm{\alpha''(t)}$.
        And since
        \[ \alpha''(t) = \parens{\frac14(1+t)^{-1/2},\,\frac14(1-t)^{-1/2},\,0} \]
        and therefore
        \[ \kappa(t)^2 = \frac1{16(1+t)} + \frac1{16(1-t)} = \frac{2}{16(1-t^2)} \]
        And so
        \[ \kappa(t) = \frac{\sqrt{2}}{4\sqrt{1-t^2}} \]

        \item Similar to before we compute $\alpha'(t)$:
        \[ \alpha'(t) = \parens{-\frac45\sin(t),\,-\cos(t),\,\frac35\sin(t)} \]
        And so
        \[ \norm{\alpha'(t)}^2 = \frac{16}{25}\sin^2(t) + \cos^2(t) + \frac9{25}\sin^2(t) = \sin^2(t) + \cos^2(t) = 1 \]
        thus $\alpha$ is indeed a natural parameterization.
        And
        \[ \alpha''(t) = \parens{-\frac45\cos(t),\,\sin(t),\,\frac35\cos(t(} \]
        So
        \[ \kappa(t)^2 = \norm{\alpha''(t)}^2 = \frac{16}{25}\cos^2(t) + \sin^2(t) + \frac9{25}\cos^2(t) = \cos^2(t) + \sin^2(t) = 1 \]
    \eenum

\eblank

\bexerc

    Find the natural parameterization of the curve
    \[ \alpha(t) = \bigl(4\cos(t),\,5-5\sin(t),\,-3\cos(t)\bigr) \]
    and find its curvature $\kappa(s)$.

\eexerc

\bblank

    So we must compute $\alpha$'s arclength,
    \[ s_\alpha(t) = \int_0^t \norm{\alpha'(u)}\,du \]
    And so we compute
    \[ \alpha'(t) = \bigl(-4\sin(t),-5\cos(t),3\sin(t)\bigr) \]
    and so
    \[ \norm{\alpha'(t)}^2 = 16\sin^2(t) + 25\cos^2(t) + 9\sin^2(t) = 25 \implies \norm{\alpha'(t)} = 5 \]
    Thus
    \[ s_\alpha(t) = \int_0^t 5 = 5t \]
    And the natural reparameterization of $\alpha$ is given by
    \[ \gamma(s) = \alpha\circ s_\alpha^{-1}(s) \]
    The inverse of $5t$ is $\frac s5$ and so the natural reparameterization of $\alpha$ is
    \[ \gamma(s) = \alpha\parens{\frac s5} \]

    Now,
    \[ \alpha''(t) = \bigl(-4\cos(t),5\sin(t),3\cos(t)\bigr) \]
    Notice that
    \[ \norm{\alpha''(t)} = \sqrt{16\cos^2(t) + 25\sin^2(t) + 9\cos^2(t)} = \sqrt{25} = 5 \]
    And since $\gamma(s)=\alpha\parens{\frac s5}$, we have $\gamma''(s)=\frac1{25}\alpha''\parens{\frac s5}$.
    This means that
    \[ \norm{\gamma''(s)} = \frac1{25}\norm{\alpha''\parens{\frac s5}} = \frac1{25}\cdot5 = \frac15 \]
    And since the curvature of $\gamma$ is equal to $\gamma''(s)$, the curvature is
    \[ \kappa(s) = \frac15 \]

\eblank

\bexerc

    Find the curvature of the curve defined by the intersection of the sphere $x^2+y^2+z^2=4$ with the plane $x+y+z=1$.

\eexerc

\bblank

    The normal to the plane is $(1,1,1)$, ie the plane is defined by
    \[ \set{v}[\iprod{v,(1,1,1)}=1] \]
    So the distance from the center of the circle, $(0,0,0)$, to the plane is the length of the vector $t(1,1,1)$ such that it is on the plane.
    This is if and only if $3t=1$, ie $t=\frac13$.
    So the distance from the center of the circle to the plane is
    \[ \norm{\frac13(1,1,1)} = \frac{\sqrt3}3 \]

    Now, if a point $P$ is on the intersection of the sphere and plane, then $v$ is of the form
    \[ P = \frac13(1,1,1) + u \]
    where $u$ is orthogonal to $(1,1,1)$ ($u$ is a vector on the plane, which is orthogonal to $(1,1,1)$).
    Since $P$ is on the sphere, $\norm{P}^2=4$ and so
    \[ \norm{P}^2 = \norm{\frac13(1,1,1)}^2 + \norm{u}^2 \implies \norm u^2 = 4 - \frac39 = \frac{11}3 \]
    Now, we know that the intersection is a circle, and since the points on the intersection are all equidistant from $\frac13(1,1,1)$ this is the center of the circle.
    And the radius is $\frac{\sqrt{11}}{\sqrt3}$, and we know the curvature of a circle of radius $r$ is $\frac1r$ so the curvature is
    \[ \kappa = \frac{\sqrt{33}}{11} \]

\eblank

\bexerc

    \benum
        \item Suppose $f(x)$ is a smooth function, then show that the curve defined by $y=f(x)$, $\alpha(t)=(t,f(t))$, has a curvature of
        \[ \kappa(t) = \frac{f''(t)}{\bigl(1+f'(t)^2\bigr)^{3/2}} \]
        \item If $F$ is a smooth function, show that the curve defined by the implicit equation $F(x,y)=0$ has an unsigned curvature of
        \[ \abs{\kappa(x,y)} = \frac{\abs{F_y^2F_{xx} - 2F_xF_yF_{xy} + F_x^2F_{yy}}}{(F_x^2+F_y^2)^{3/2}} \]
        when $\nabla F\neq0$.
        \item Can you find a formula for the curvature of $F(x,y)=0$ (the signed curvature)?
    \eenum

\eexerc

\bblank

    \benum
        \item Recall that for a general curve, its curvature is given by
        \[ \kappa_\alpha(t) = \frac{\detof{\alpha'(t),\alpha''(t)}}{\norm{\alpha'(t)}^3} \]
        And so
        \[ \alpha'(t) = (1,f'(t)),\quad \alpha''(t) = (0,f''(t)) \]
        Therefore
        \[ \detof{\alpha'(t),\alpha''(t)} = f''(t),\qquad \norm{\alpha'(t)} = (1+f'(t)^2)^{1/2} \]
        Thus
        \[ \kappa_\alpha(t) = \frac{f''(t)}{(1+f'(t)^2)^{3/2}} \]
        as required.

        \item By the implicit function theorem, if $\nabla f(x_0,y_0)\neq0$ then there exists an open neighborhood of $x_0$, $\mU$, and a smooth function $\phi$ such that $\phi(x_0)=y_0$ and
        $F(x,\phi(x))=0$ for $x\in\mU$.
        So the curve in the neighborhood of $x_0$ is defined by $y=\phi(x)$.
        Thus by the previous subquestion we must find $y'$ and $y'$.
        By the implicit function theorem, we know that
        \[ y' = -F_y(x,y)^{-1}F_x(x,y) \]
        And so
        \[ y'' = \bigl(F_y(x,y)\bigr)'\cdot F_y(x,y)^{-2}F_x(x,y) - F_y(x,y)^{-1}\cdot\bigl(F_x(x,y)\bigr)' \]
        Now, by the chain theorem
        \[ \bigl(F_x(x,y)\bigr)' = F_{xx}(x,y) + y'F_{xy}(x,y) \]
        and similarly $\bigl(F_y(x,y)\bigr)'=F_{xy}(x,y) + y'F_{yy}(x,y)$, thus we get that
        \begin{align*}
            y'' &= (F_{xy} + y'F_{yy})F_xF_y^{-2} - F_y^{-1}(F_{xx} + y'F_{xy}) \\
                &= F_xF_y^{-2}F_{xy} + y'F_xF_y^{-2}F_{yy} - F_y^{-1}F_{xx} - y'F_y^{-1}F_{xy} \\
                &= F_xF_y^{-2}F_{xy} - F_x^2F_y^{-3}F_{yy} - F_y^{-1}F_{xx} + F_y^{-2}F_xF_{xy} \\
                &= F_y^{-3}\Bigl(F_xF_yF_{xy} - F_x^2F_{yy} - F_y^2F_{xx} + F_yF_xF_{xy}\Bigr) \\
                &= -F_y^{-3}\Bigl(F_x^2F_{yy} - 2F_xF_yF_{xy} + F_y^2F_{xx}\Bigr)
        \end{align*}

        And we get that
        \[ (1+(y')^2)^{3/2} = \bigl(1+F_y^{-2}F_x^2\bigr)^{3/2} = F_y^{-3}\bigl(F_x^2+F_y^2\bigr)^{3/2} \]

        And from the previous subquestion we know
        \[ \kappa = \frac{y''}{\bigl(1+(y')^2\bigr)^{3/2}} = -\frac{F_x^2F_{yy} - 2F_xF_yF_{xy} + F_y^2F_{xx}}{\bigl(F_x^2+F_y^2\bigr)^{3/2}} \]
        Thus
        \[ \abs\kappa = \frac{\abs{F_x^2F_{yy} - 2F_xF_yF_{xy} + F_y^2F_{xx}}}{\bigl(F_x^2+F_y^2\bigr)^{3/2}} \]
        as required.

        \item Indeed we computed one for the previous subquestion:
        \[ \kappa = -\frac{F_x^2F_{yy} - 2F_xF_yF_{xy} + F_y^2F_{xx}}{\bigl(F_x^2+F_y^2\bigr)^{3/2}} \]
        But this curvature does not take into account the direction of the curve.
        That is, we could reverse the direction of the curve and this would give us another solution to $F(x,y)=0$, and its curvature would have the opposite sign as above.
    \eenum

\eblank

\bexerc

    Find the curvature of the ellipse $x^2+2y^2=3$ in two ways:
    \benum
        \item Using the formula found above for the curvature of a curve given by an implicit function, and
        \item Using the formula for the curvature of a curve given through a parameterization.
    \eenum

\eexerc

\bblank

    \benum
        \item Here we have $F(x,y)=x^2+2y^2-3$.
        Thus
        \[ F_x = 2x,\quad F_y = 4y,\quad F_{xx} = 2,\quad F_{yy} = 4,\quad F_{xy} = 0 \]
        and so by the formula above
        \[ \kappa(x,y) = -\frac{4(2x)^2 + 2(4y)^2}{((2x)^2+(4y)^2)^{3/2}} = -\frac{16x^2 + 32y^2}{(4x^2+16y^2)^{3/2}} \]
        Now since $(x,y)$ are on the ellipse we get that $2y^2=3-x^2$ and so this is equal to
        \[ \kappa(x,y) = -\frac{16x^2 + 16(3-x^2)}{(4x^2+8(3-x^2))^{3/2}} = -\frac{48}{(24-4x^2)^{3/2}} = -\frac{6}{(6-x^2)^{3/2}} \]

        \item The ellipse can be parameterized by
        \[ \alpha\colon[0,2\pi]\longto\bR^2,\quad \alpha(t) = \sqrt3\parens{\cos t,\frac1{\sqrt2}\sin t} \]
        Thus
        \[ \alpha'(t)=\sqrt3\parens{-\sin t,\,\frac1{\sqrt2}\cos t},\qquad \alpha''(t)=\sqrt3\parens{-\cos t,\,-\frac1{\sqrt2}\sin t} \]
        And we know that
        \[ \kappa(t) = \frac{\detof{\alpha'(t),\alpha''(t)}}{\norm{\alpha'(t)}^3} = \frac{3\parens{\frac{\sin^2(t)}{\sqrt2} + \frac{\cos^2(t)}{\sqrt2}}}{3^{3/2}\parens{\sin^2(t)+\frac{\cos^2(t)}2}^{3/2}} \]
        Now, we know that
        \[ x = \sqrt3\cos(t) \implies \cos^2(t) = \frac{x^2}3,\qquad y=\frac{\sqrt3}{\sqrt2}\sin(t) \implies \sin^2(t) = \frac23y^2 \]
        Thus
        \[ \kappa(x,y) = \frac1{\sqrt6}\cdot\frac1{\parens{\frac23y^2+\frac16x^2}^{3/2}} = \frac{6^{3/2}}{\sqrt6}\cdot\frac1{\parens{x^2+4y^2}^{3/2}} = \frac6{\parens{x^2+4y^2}^{3/2}} \]
        Now since $(x,y)$ is on the ellipse, we have $4y^2=6-2x^2$ and so
        \[ \kappa(x,y) = \frac6{\parens{6-x^2}^{3/2}} \]
        as required.
    \eenum

\eblank

\bexerc

    Find the points with maximum curvature in the following curves:
    \benum
        \item $3x^2+4y^2=1$
        \item $y=e^x$
        \item $y^2-5+xy=0$
    \eenum

\eexerc

\bblank

    \benum
        \item Here $F(x,y)=3x^2+4y^2-1$ and so
        \[ F_x = 6x,\quad F_y=8y,\quad F_{xx}=6,\quad F_{yy}=8,\quad F_{xy} = 0 \]
        and so
        \[ \abs{\kappa(x,y)} = \frac{288x^2+384y^2}{(36x^2+64y^2)^{3/2}} \]
        Since $3x^2+4y^2=1$ we have that
        \[ \abs{\kappa(x,y)} = \frac{96}{(16-12x^2)^{3/2}} \]
        This is maximal when $16-x^2$ is minimal, ie. when $x^2$ is maximal.
        This is when $y=0$ and we get $x^2=\frac13$ and so the maximum value of $\abs\kappa$ is
        \[ \frac{96}{(16-4)^{3/2}} = \frac4{\sqrt3} \]
        at $\parens{\pm\frac1{\sqrt3},0}$.

        \item Here we have that the curvature is given by the formula $\frac{y''}{(1+(y')^2)^{3/2}}$:
        \[ \kappa = \frac{e^x}{(1+e^{2x})^{3/2}} \]
        This is already always positive.
        Let us differentiate:
        \[ \kappa' = \frac{e^x(1+e^{2x})^{3/2} - 3e^{3x}(1+e^{2x})^{1/2}}{(1+e^{2x})^3} \]
        This is equal to zero when
        \[ (1+e^{2x})^{3/2} = 3e^{2x}(1+e^{2x})^{1/2} \iff 1+e^{2x} = 3e^{2x} \iff e^{2x} = \frac12 \iff x = -\frac{\log2}2 \]
        So $x=-\frac{\log2}{2}$ is either a critical point of $\kappa$.
        Plugging in values less than and more than $-\frac{\log2}2$ shows that it is a maximum.

        So the maximum curvature is at $\parens{-\frac{\log2}2,\frac1{\sqrt2}}$ which is
        \[ \kappa = \frac{\frac1{\sqrt2}}{1.5^{3/2}} = \frac{2\sqrt3}9 \]

        \item Here $F(x,y)=y^2-5+xy$ and so
        \[ F_x = y,\quad F_y = 2y+x,\quad F_{xx} = 0,\quad F_{yy} = 2,\quad F_{xy} = 1 \]
        Thus
        \[ \kappa = -\frac{2y^2 - 2y(2y+x)}{(y^2+(2y+x)^2)^{3/2}} = \frac{2y^2+yx}{(5y^2+4xy+x^2)^{3/2}} = \frac{10}{(y^2+x^2+20)^{3/2}} \]
        Thus $\kappa$ is positive, and thus in order to maximize it we must minimize $y^2+x^2+20$.
        Now, since $y\neq0$ (as $F(x,0)=-5$), we have
        \[ x = \frac{5-y^2}y = \frac5y - y \]
        and so we must minimize
        \[ y^2 + \frac{25}{y^2} - 10 + y^2 + 20 = 2y^2 + 25y^{-2} + 10 \]
        Differentiating gives
        \[ 4y - 50y^{-3} \]
        And comparing to zero gives that $4y^4=50$ and so $y=\pm(12.5)^{1/4}$.
        Comparing the derivative at these points (plugging in values less than and more than these $y$s), shows that these are both indeed minimums.

        Since
        \[ \kappa = \frac{10}{(2y^2+25y^{-2}+10)^{3/2}} \]
        we get that the curvature at these points is about $0.0444$.
    \eenum

\eblank

\bexerc

    Let $\beta\colon[0,L]\longto\bR^3$ be a curve in its natural parameterization, such that $\kappa(s),\tau(s)\neq0$ at every point.
    We define
    \[ X(s) = \frac1{\kappa(s)},\quad Y(s) = \frac1{\tau(s)} \]
    Prove that if the curve is contained entirely on the unit sphere then
    \[ X^2 + (X'Y)^2 \]
    is constant.

\eexerc

\bblank

    Since $\beta$ is contained entirely on the unit sphere, $\norm\beta$ and by extension $\iprod{\beta,\beta}$ is constant.
    Thus $\iprod{\beta,\beta}^{(k)}$ is zero for every $k\geq1$.
    But at the same time
    \[ \iprod{\beta,\beta}' = 2\iprod{\beta,\beta'} = 2\iprod{\beta,T} \]
    So $\iprod{\beta,T}=0$.
    And
    \[ \iprod{\beta,\beta}'' = 2(\iprod{\beta,T'} + \iprod{\beta',T}) \]
    By the Frenet-Serret formulas, we have $T'=\kappa N$ and so
    \[ \iprod{\beta,\beta}'' = 2(\kappa\iprod{\beta,N} + \iprod{T,T}) \]
    since $\beta$ is a natural parameterization, $\norm{\beta'}=\norm{T}=1$ and so $\iprod{T,T}=1$.
    And since $\iprod{\beta,\beta}''=0$ we have
    \[ \kappa\iprod{\beta,N} = -1 \implies \iprod{\beta,N} = -X \]
    And differentiating $\iprod{\beta,\beta}''$ gives
    \[ \iprod{\beta,\beta}''' = 2(\kappa'\iprod{\beta,N} + \kappa\iprod{\beta',N} + \kappa\iprod{\beta,N'}) \]
    This is zero, so we have
    \[ \kappa'\iprod{\beta,N} + \kappa\iprod{T,N} + \kappa\iprod{\beta,-\kappa T+\tau B} = 0 \]
    Since $T$ and $N$ are orthogonal, $\iprod{T,N}=0$ and so
    \[ \kappa'\iprod{\beta,N} - \kappa^2\iprod{\beta,T} + \kappa\tau\iprod{\beta,B} = 0 \]
    We showed that $\iprod{\beta,T}=0$ and $\iprod{\beta,N}=-X$ and so
    \[ \kappa\tau\iprod{\beta,B} = -\kappa'X \]
    So
    \[ \iprod{\beta,B} = -\kappa'X^2Y \]
    Notice that since $X=\kappa^{-1}$, $X'=-\kappa'\kappa^{-2}=-\kappa'X^2$, so
    \[ \iprod{\beta,B} = X'Y \]

    Thus we have
    \begin{align*}
        \iprod{\beta,T} &= 0 \\
        \iprod{\beta,N} &= -X \\
        \iprod{\beta,B} &= X'Y
    \end{align*}
    Thus we have
    \[ \beta = \iprod{\beta, T}T + \iprod{\beta, N}N + \iprod{\beta, B}B = -XN + X'YB \]
    Since $\set{T,N,B}$ form an orthonormal basis,
    \[ \norm{\beta} = X^2 + (X'Y)^2 \]
    and since $\beta$ is on the unit sphere, $\norm\beta=1$ and so we get that $X^2+(X'Y)^2$ is constant (it is equal to one), as required.

\eblank

\bexerc

    Show that if $\beta$ is any regular smooth curve in $\bR^3$, then its curvature and torsion are given by
    \[ \kappa(t) = \frac{\norm{\beta'(t)\times\beta''(t)}}{\norm{\beta'(t)}},\quad \tau(t) = \frac{\detof{\beta'(t),\,\beta''(t),\,\beta'''(t)}}{\norm{\beta'(t)\times\beta''(t)}^2}  \] 

\eexerc

\bblank

    Let $\gamma$ be $\beta$'s natural reparameterization.
    \[ \gamma = \beta\circ s^{-1} \implies \beta = \gamma\circ s \]
    then
    \begin{align*}
        \beta'(t) &= s'(t)\gamma'(s(t)),\\
        \beta''(t) &= s''(t)\gamma'(s(t)) + s'(t)^2\gamma''(s(t)),\\
        \beta'''(t) &=s'''(t)\gamma'(s(t))+s''(t)s'(t)\gamma''(s(t))+2s'(t)s''(t)\gamma''(s(t))+s'(t)^3\gamma'''(t)
    \end{align*}
    
    Now, $\gamma'=T$ and $\gamma''=T'=\kappa N$ and
    \[ \gamma'''=\kappa'N+\kappa N'=\kappa'N+\kappa\parens{-\kappa T+\tau B} = \kappa'N - \kappa^2T + \kappa\tau B \]
    
    And since $T\times T=0$ (this is true in general for any vectors), and $T\times N=B$ (by definition),
    \[ \beta'\times\beta'' = (s'T)\times(s''T+(s')^2\kappa N) = (s')^3\kappa B \]
    Now, notice that since $s'=\norm{\beta'}$ by definition, we get
    \[ \norm{\beta'\times\beta''} = \abs{s'}^3\kappa = \norm{\beta'}^3\kappa \implies \kappa = \frac{\norm{\beta'\times\beta''}}{\norm{\beta'}^3} \]
    As required.

    And since the only component not orthogonal to $B$ in $\beta'''$ is $s'(t)^3\gamma'''$, of which the only component not orthogonal to $B$ is $s'(t)^3\kappa\tau B$,
    \[ \iprod{\beta''',B} = \iprod{(s')^3\gamma''',B} = \iprod{(s')^3\kappa\tau B,B} = (s')^3\kappa\tau \]
    Now, $s'=\norm{\beta'}$, and so
    \[ (s')^3\kappa= \norm{\beta'}^3\frac{\norm{\beta'\times\beta''}}{\norm{\beta'}^3}= \norm{\beta'\times\beta''} \]
    And since
    \[ \beta'\times\beta'' = (s')^3\kappa B = \norm{\beta'\times\beta''}B \]
    And so
    \[ \tau = \frac{\iprod{\beta''',B}}{\norm{\beta'\times\beta''}} = \frac{\iprod{\beta''',\beta'\times\beta''}}{\norm{\beta'\times\beta''}^2} \]
    And since
    \[ \iprod{\beta''',\beta'\times\beta''} = \iprod{\beta'\times\beta'',\beta'''} = \detof{\beta',\beta'',\beta'''} \]
    we get
    \[ \tau = \frac{\detof{\beta',\beta'',\beta'''}}{\norm{\beta'\times\beta''}^2} \]
    as required.

\eblank

\bexerc

    \benum
        \item Show that a space curve has zero curvature if and only if it is a line.
        \item Determine all the space curves which have constant curvature and torsion.
    \eenum

\eexerc

\bblank

    \benum
        \item This follows from lines having zero curvature and the fundamental theorem of curves, but I will prove it directly here.
        If a space curve has zero curvature then since $\kappa=\norm{T'}=\norm{\alpha''}$, we get that $\alpha''(t)=0$.
        Thus $\alpha'(t)=v$ and so $\alpha(t)=tv+u$, which defines a line.

        And if $\alpha$ is a line, $\alpha(t)=tv+u$ and so $\alpha'(t)=v$ and so $\alpha''(t)=0$.

        \item If $\alpha$ has constant curvature and torsion, then so does its natural parameterization.
        So we can assume that $\alpha$ is its natural parameterization.
        Since $\kappa$ and $\tau$ are constant, by the Frenet-Serret formulas we get
        \begin{align*}
            \alpha' &= T \\
            \alpha'' &= \kappa N \\
            \alpha''' &= -\kappa^2T + \kappa\tau B \\ 
            \alpha'''' &= -\kappa^3N - \kappa\tau^2N = -\kappa N(\kappa^2+\tau^2)
        \end{align*}
        So we get the ODE 
        \[ \alpha'''' = -\alpha''(\kappa^2+\tau^2) \]
        Let us denote $c=\kappa^2+\tau^2$.
        Then the characteristic polynomial of this linear ODE is
        \[ x^4 + cx^2 \]
        and so the roots are $0$ with multiplicity two, and $\pm\sqrt ci$.
        So the solutions are
        \[ \alpha(t) = A_0 + A_1t + A_2\cos(\sqrt ct) + A_3\sin(\sqrt ct) \]
        where $A_i\in\bR^3$.

        This means that
        \[ \alpha''(t) = -A_2c\cos(\sqrt ct) - A_3c\sin(\sqrt ct) \]
        And since $\kappa=\norm{\alpha''}$ is constant, we get that
        \[ \kappa = c\norm{A_2\cos(\sqrt ct) + A_3\sin(\sqrt ct)} \]
        is constant, and so therefore so is $\frac\kappa c$ and thus
        \[ \frac{\kappa^2}{c^2} = \norm{A_2\cos(\sqrt ct) + A_3\sin(\sqrt ct)}^2 = \norm{A_2}^2\cos^2(\sqrt ct) + \iprod{A_2,A_3}\cos(\sqrt ct)\sin(\sqrt ct) + \norm{A_3}^2\sin^2(\sqrt ct) \]
        is constant.

        This is a function of the form
        \[ A\cos^2(u) + B\sin(2u) + C\sin^2(u) \]
        And for it to be constant, its derivative must be zero.
        So
        \[ 0 = -A\sin(2u) + 2B\cos(2u) + C\sin(2u) \implies (A-C)\sin(2u) = 2B\cos(2u) \]
        Now if $A-C\neq0$ then we get that $\tan(2u)=\frac{2B}{A-C}$ which is constant.
        But the tangent function is not constant, in contradiction.
        So $A=C$ and so we get $2B\cos(2u)=0$ and so $B=0$.

        So we get that
        \[ \norm{A_2}^2 = \norm{A_3}^2,\quad \iprod{A_2,A_3} = 0 \]
        So if we let $\norm{A_2}=\norm{A_3}=A$ then we get
        \[ \frac{\kappa^2}{c^2} = A^2(\cos^2(\sqrt ct)+\sin^2(\sqrt ct)) = A^2 \implies A = \frac\kappa c = \frac\kappa{\kappa^2+\tau^2} \]

        Now, we know that $N'+\kappa T=\tau B$ and so $\norm{N'+\kappa T}=\abs\tau$ and since $\tau$ is continuous if and only if $\abs\tau$ is, and $N=\frac{\alpha''}\kappa$ so we must find when
        \[ \tau = \norm{\frac{\alpha'''}\kappa+\kappa\alpha'} \]
        is constant.
        This is equal to $\frac1\kappa\norm{\alpha'''+\kappa^2\alpha'}$.
        And since
        \[ \alpha'''(t) = A_2c^{1.5}\sin(\sqrt ct) - A_3c^{1.5}\cos(\sqrt ct) \]
        we get that
        \[ \kappa\tau = \norm{A_2\sqrt c\sin(\sqrt ct)(c-\kappa^2) - A_3\sqrt c\cos(\sqrt ct)(c-\kappa^2) + \kappa^2A_1} \] 
        We must have $A_2$ and $A_3$ are orthogonal to $A_1$ in order for this to be constant.
        %And then since $c-\kappa^2=\tau^2$, the normed squared is equal to
        %\[ A^2c\tau^4 + \kappa^4\norm{A_1}^2 = \kappa^2\tau^2 \]
        %since $A=\frac\kappa c$ we get that
        %\[ \frac{\kappa^2}c\tau^4 +\kappa^4\norm{A_1}^2 = \kappa^2\tau^2 \implies \kappa^2\norm{A_1}^2 = \tau^2\parens{1-\frac{\tau^2}c} = \tau^2\frac{\kappa^2}c \]
        %Thus
        %\[ \norm{A_1} = \frac{\tau}{\sqrt{\kappa^2+\tau^2}} \]

        Since $\norm{A_2}=\norm{A_3}$, and $A_1$ is orthogonal to $A_2$ and $A_3$, the curves with constant curvature and torsion are

        {\advance\leftskip by.3cm\relax
        \benum
            \item Circular helixes (if $\kappa,\tau\neq0$)
            \item Circles (if $\tau=0$, as then $A_1=0$)
            \item Lines (if $\kappa=0$)
        \eenum}
    \eenum

\eblank

\end{document}
