\documentclass[10pt]{article}

\usepackage{amsmath, amssymb, mathtools}
\usepackage{tikz}
\usetikzlibrary{decorations.markings}
\usepackage[margin=1.5cm]{geometry}

\input prettyprint
\input preamble
\input pdfmsym

\pdfmsymsetscalefactor{10}
\initpps

\def\pmat#1{\begin{pmatrix} #1 \end{pmatrix}}

\let\divides=\mid
\newfunc{det}{{\rm det}}({})
\newfunc{metric}\rho({})
\newfunc{metricc}\sigma({})
\newfunc{spa}{{\rm span}}(\vert)
\newfunc{diam}{{\rm diam}}(\vert)
\newfunc{proj}\pi({})
\newfunc{iproj}{\pi^{-1}}({})
\newfunc{cis}{{\rm cis}}({})
\newfunc{Re}{{\rm Re}}({})
\newfunc{Im}{{\rm Im}}({})
\newfunc{sup}{{\rm sup}}\{\vert\}
\newfunc{Res}{{\rm Res}}({})
\newfunc{wind}n({})
\newfunc{pv}{{\rm pv}}({})
\newfunc{lspan}{{\rm span}}\{|\}
\newfunc{spec}{{\rm spec}}({})
\newfunc{trace}{{\rm trace}}({})
\newfunc{sff}{{\rm I\mkern-2muI}}({})

\font\bigbf = cmbx12 scaled 2000
\@undervecc@def{underbar}\@linecap\@linecap

\def\pmat#1{\begin{pmatrix}#1\end{pmatrix}}

\def\bS{\mathbb{S}}
\def\mO{{\cal O}}
\def\mU{{\cal U}}
\let\lineseg=\overleftrightvecc
\let\to=\varrightarrow
\let\longto=\longvarrightarrow
\let\ds=\displaystyle

\def\pdv#1#2{\frac{\partial #1}{\partial #2}}

\def\differ#1#2{\left.d#1\strut\right|_{#2}}

\def\qed{%
    \ifmmode%
        \eqno\blacksquare%
    \else%
        \hskip1cm\allowbreak\hbox{}\nobreak\hfill$\blacksquare$%
    \fi%
}

\def\@ppmathcount{\thesection.\thepp@mathcount}

\def\bexerc{\begin{exercise*}}
\def\eexerc{\end{exercise*}}
\def\bblank{\begin{blankpp}}
\def\eblank{\end{blankpp}}

\begin{document}

\c@section=5

\barcolorbox{220, 255, 220}{0, 130, 0}{80, 200, 80}{
    \leftskip=0pt plus 1fill \rightskip=\leftskip
    {\bigbf Differential and Analytic Geometry}

    \medskip
    \textit{Assignment \thesection}

    \textit{Ari Feiglin}
}

\bigskip

\bexerc

    The unit sphere is parameterized by
    \[ x(\theta,\phi) = (\sin\phi\cos\theta,\sin\phi\sin\theta,\cos\phi) \]
    We are given a latitude line, $\gamma$ given by $\phi=\phi_0$.
    Find a parallel unit vector field over $\gamma$.

\eexerc

Let us compute the tangent space to $x$,
\[ x_1 = (-\sin\phi\sin\theta,\sin\phi\cos\theta,0),\qquad x_2 = (\cos\phi\cos\theta,\cos\phi\sin\theta,-\sin\phi) \]
We can form an orthonormal basis out of this,
\[ E_1 = (-\sin\theta,\cos\theta,0),\qquad E_2 = (\cos\phi\cos\theta,\cos\phi\sin\theta,-\sin\phi) \]
Now, since $\gamma(t)=x(t,\phi_0)$, we have that
\[ E_1(t) = (-\sin(t),\cos(t),0),\qquad E_2(t) = (\cos(t)\cos(\phi_0),\sin(t)\cos(\phi_0),-\sin(\phi_0)) \]
Differentiating gives
\[ E'_1(t) = (-\cos(t),-\sin(t),0)=-\cos(\phi_0)E_2(t) - \sin(\phi_0)x(t,\phi_0),\quad E_2'(t) = \cos(\phi_0)\cdot(-\sin(t),\cos(t),0) = \cos(\phi_0)E_1(t) \]
And since $V(t)\in T_{\gamma(t)}\bS^2$, and it is a unit
\[ V(t) = \cos(\alpha(t))E_2(t,\phi_0) + \sin(\alpha(t))E_2(t,\phi_0) \]
This means
\[ V'(t) = (\cos(\phi_0)-\dot\alpha)\sin(\alpha)E_1 + (-\cos(\phi_0)+\dot\alpha)\cos(\alpha)E_2 - \sin(\phi_0)\cos(\alpha)n \]
(Since $x=n$ is the unit normal to $\bS^2$).
Thus we have that in order for $V(t)$ to be parallel, $V'(t)\perp T_{\gamma(t)}\bS^2$ and so
\[ \dot\alpha = \cos(\phi_0) \]
Thus
\[ \alpha(t) = \cos(\phi_0)t + \alpha_0 \]
And so the total change of angle is
\[ \Delta\alpha = \alpha(2\pi) - \alpha(0) = \cos(\phi_0)\cdot2\pi \]
Which is what we got in recitation as well.

\bexerc

    We are given the cylinder $M$ parameterized by
    \[ x(u,v) = (\cos u,\sin u,v) \]
    \benum
        \item For every $p\in M$, find the exponential $\exp_p\colon T_pM\longto M$ explicitly.
        \item For every curve $\gamma\colon I\to M$ and points $a,b\in I$, find the parallel transport $P_\gamma^{ab}\colon T_{\gamma(a)}M\to T_{\gamma(b)}M$
    \eenum

\eexerc

\benum
    \item We know that by definition $\exp_p(v)$ is equal to travelling a unit on the geodesic at $p$ in the direction $v$.
    Since we are on a cylinder, the only geodesics are helixes ($t\mapsto (a(\cos(t)-1),a\sin t,ct)+p$) and vertical lines ($t\mapsto (0,0,ct)+p$).
    And let us recall the definition of the exponential map, for $v\in T_pM$ let $\gamma_v$ be the geodesic where $\gamma_v(0)=p$ and $\gamma'_v(0)=v$ then $\exp_p(v)=\gamma_v(1)$.
    If $v=0$ then $\exp_p(0)=p$.

    So the two geodesics on cylinders have derivatives of the form $(-a\sin(t),a\cos(t),c)$ and $(0,0,c)$.
    So if $v=(0,0,c)$ then $\gamma_v(t)=(0,0,ct)+p$ and so $\exp_p(v)=(0,0,c)+p=v+p$.
    And if $v=(-a\sin(1),a\cos(1),c)$ then $\gamma_v(t)=(a(\cos(t)-1),a\sin t,ct)+p$ and so $\exp_p(v)=(a(\cos(1)-1),a\sin(1),c)+p$, this is equal to $R_{3\pi/2}v+(-a,0,0)+p$.
    And if $v=0$ then $\exp_p(v)=p$.

    \item Let $v\in T_{\gamma(a)}M$ then $P^{ab}_\gamma(v)=W(b)$ where $W(t)$ is the parallel transport on $\gamma$ and $W(a)=v$.
    Then we know that
    \[ W(t) = \cos\parens{\theta_0+\int_a^t\kappa_g}\gamma'(t) + \sin\parens{\theta_0+\int_a^t\kappa_g}n\times\gamma'(t) \]
    Now, since $\gamma$ is smooth, let us focus on the surface whose boundary is given by $\gamma$ (from $a$ to $b$) and the geodesic from $\gamma(a)$ to $\gamma(b)$.
    Since the cylinder is isometric with the plane, both have a Gaussian curvature of zero.
    This means that (since the geodesic curvature of the geodesic is zero),
    \[ \int_a^t\kappa_g = \iint K\,ds = 0 \]
    And so
    \[ W(t) = \cos\parens{\theta_0}\gamma'(t) + \sin\parens{\theta_0}n\times\gamma'(t) \]
    So
    \[ P^{ab}_\gamma(v) = \cos(\theta_0)\gamma'(b) + \sin(\theta_0)n\times\gamma'(b) \]
    
\eenum

\bexerc

    Prove, without using Gauss-Bonnet, that given a geodesic triangle on the unit sphere whose interior angles are $\alpha$, $\beta$, and $\gamma$, we have
    \[ \alpha + \beta + \gamma = \pi + T \]
    where $T$ is the area of the triangle.

\eexerc

Let $S_x$ be the area between two geodesics at an angle $x$ on the sphere.
Since $S_x$ covers $\frac x\pi$ of the total surface area of the sphere, we have that, since the area of the sphere is $4\pi$,
\[ S_x = 4\pi\cdot\frac x\pi = 4x \]
Now, notice that $S_\alpha+S_\beta+S_\gamma$ covers the entire sphere, but each $S_x$ ($x=\alpha,\beta,\gamma$) counts $T$ twice (since the geodesics which define the triangle define a congruent
triangle on the opposite side of the sphere).
So this sum counts $T$ six times, while it should only be counted twice (since the geodesics define two congruent triangles), meaning $T$ is counted four more times than required.
Thus (since the surface area of the sphere is $4\pi$),
\[ S_\alpha + S_\beta + S_\gamma = 4\pi + 4T \]
And so
\[ 4(\alpha + \beta + \gamma) = 4(\pi + T) \implies \alpha + \beta + \gamma = \pi + T \]
as required.

\bexerc

    Let $M$ be a compact orientable surface with a positive genus.
    Show that the surface has ellipitic, hyperbolic, and parabolic points (points where the Gaussian curvature is positive, zero, and negative).

\eexerc

Let $O\in\bR^3$ be a point within $M$, since $M$ is compact there exist radii $R>0$ such that $M$ is contained within the ball $B_R(O)$.
Let $r$ be the infimum of all such radii, and then $M$ must intersect with $B_r(O)$ at at least one point $p$, as otherwise we could reduce the radius $r$.
In fact, they must be tangent at $p$ as otherwise we could increase $r$ and get a ball not containing $M$.
So if we look at a normal section at $p$, we must have that its normal curvature with $M$ is greater than its normal curvature with $B_r(O)$ (since it must curve away from $p$ quicker, as it is contained
within $B_r(O)$).
Since the Gaussian curvature is equal to the product of the maximum and minimum of these values, we conclude that the Gaussian curvature of $M$ at $p$ is greater than that with $B_r(O)$ at $p$.
Since the Gaussian curvature of $B_r(O)$ is $\frac1r>0$, we have that the Gaussian curvature of $M$ at $p$ is positive, and so we have an elliptic point.

Since $M$ has a positive genus, it must have a hole somewhere.
Let $O\in\bR^3$ contained within this hole, and let $r$ be the supremum of all the radii $R$ such that $B_r(O)$ does not intersect $M$.
Then again, $B_r(O)$ must be tangent to $M$ at some point $p$.
If we look at the normal curvatures, not all can curve inward or all outward for every choice of $O$, and so there must be some point $p$ where it has normal curvature both inward and outward.
And so its Gaussian curvature is negative.

And since Gaussian curvature is continuous, there must be a point with zero Gaussian curvature.

\bexerc

    Let us look at the rotation of the curve $\alpha(\psi)=(\cosh\phi,0,\psi)$.
    Find the total Gaussian curvature of the surface.

\eexerc

We have computed that the Gaussian curvature of this surface is $K(t,\phi)=-\frac1{\cosh(t)^4}$.
And the metric has a determinant of $\cosh(t)^4$, meaning that the total curvature is
\[ \iint_M K\,ds = -\int_{-\infty}^\infty\int_0^{2\pi}\frac1{\cosh(t)^2}\,d\phi dt = -2\pi\int_{-\infty}^\infty\frac1{\cosh(t)}^2\,dt = -2\pi\tanh(t)\bigl|^\infty_{-\infty} = -4\pi \]

\bexerc

    Find the total Gaussian curvature of the rotation of $\alpha(\phi)=(\phi,0,\phi^2)$ where $\phi\geq0$.

\eexerc

This forms a parabaloid, which we can parameterize by $x(u,v)=(u,v,u^2+v^2)$ (Gaussian curvature is intrinsic and does not depend on parameterization).
This is the graph of $f(u,v)=u^2+v^2$, and we have shown that the Gaussian curvature of a graph is given by
\[ K = \frac{f_{uu}f_{vv}-f_{uv}^2}{(1+f_u^2+f_v^2)^2} \]
Here that is equal to
\[ \frac4{(1+4u^2+4v^2)^2} \]
And so the total Gaussian curvature is
\[ \iint_M K\,ds = 4\int_{-\infty}^\infty\int_{-\infty}^\infty(1+4u^2+4v^2)^{-2}\,dudv = \pi\int_{-\infty}^\infty(1+4v^2)^{-3/2}\,dv = \pi \]

\bexerc

    Suppose $M_1$ and $M_2$ are two disjoint compact orientable surfaces, show that
    \[ \chi(M_1\dcup M_2) = \chi(M_1) + \chi(M_2) \]

\eexerc

We can show this in two ways.
Firstly, since $M_1\dcup M_2$ is compact and orientable, by Gauss-Bonnet:
\[ \iint_{M_1\dcup M_2}K\,ds = 2\pi\chi(M_1\dcup M_2) \]
But at the same time
\[ \iint_{M_1\dcup M_2}K\,ds = \iint_{M_1}K\,ds + \iint_{M_2}K\,ds = 2\pi\chi(M_1) + 2\pi\chi(M_2) \]
And so
\[ \chi(M_1\dcup M_2) = \chi(M_1) + \chi(M_2) \]

Alternatively, any triangularization of $M_1$ and $M_2$ defines a triangularization of $M_1\dcup M_2$.
So if $V^i$, $F^i$, and $E^i$ are the number of vertices, faces, and edges of the triangularization of $M_i$ then $V^1+V^2$, $F^1+F^2$, and $E^1+E^2$ are the number of vertices, faces, and edges
of a triangularization of $M_1\dcup M_2$.
And so
\[ \chi(M_1\dcup M_2) = V^1+V^2 - (E^1+E^2) + (F^1+F^2) = V^i - E^i + F^i = \chi(M_1) + \chi(M_2) \]

\bexerc

    Given the surface $M$ defined by
    \[ (x^2+y^2+2z^2-1)((x-10)^2+y^2+2z^2-1) = 0 \]
    compute its total curvature.

\eexerc

This surface is the union of $x^2+y^2+2z^2-1=0$ and $(x-10)^2+y^2+2z^2-1=0$.
These are two disjoint ellipsoids, let them be $M_1$ and $M_2$.
Since ellipsoids are compact and orientable surfaces of genus zero, we have
\[ \iint_M K\,ds = \iint_{M_1\dcup M_2}K\,ds = \iint_{M_1}K\,ds + \iint_{M_2}K\,ds = 4\pi + 4\pi = 8\pi \]

\end{document}

