How do we define what a circle is?
Historically, there are two approaches: Descartes defined it as the set of all points $(x,y)$ which satisfy the equation
\[ (x-a)^2 + (y-b)^2 = R^2 \]
for some values $a$ and $b$ and $R>0$.
Euclid defined it as the set of all points whose distance from a specific point is some positive constant $R$.

We know that these two definitions are equivalent (given the standard norm/metric in $\bR^2$), but Descartes's definition was introduced two thousand years after Euclid's.
The idea of translating a visual or intuitive definition to an analytic one, as Descartes did, will be a motif of this course.

Now, recall the definition of an ellipse.
Given two points, called the \emph{foci} of the ellipse, $F_1$ and $F_2$ and a constant $d$, the ellipse defined is the set of all points $A$ such that
\[ \abs{F_1A} + \abs{F_2A} = d \]
We also must have that $\abs{F_1F_2}<d$ as otherwise this just defines some line segment of $F_1F_2$.
This is the Euclidean definition of an ellipse.
Descartes's definition of an ellipse is the set of all points which satisfy
\[ \frac{x^2}{a^2} + \frac{y^2}{b^2} = 1 \]

We must show that the cartesian definition satisfies the euclidean definition (and vice versa).
Let us suppose that $a^2>b^2$ (if we have an equality then this defines a circle), then we define $c=\sqrt{a^2-b^2}$, and $F_1=(-c,0)$ and $F_2=(c,0)$.
Then define $d=2a$.
Now we must show that given $A=(x,y)$, $\abs{F_1A}+\abs{F_2A}=d$ if and only if $\frac{x^2}{a^2} + \frac{y^2}{b^2} = 1$.
Now,
\[ \abs{F_1A} + \abs{F_2A} = \sqrt{(x+c)^2 + y^2},\qquad \abs{F_2A} = \sqrt{(x-c)^2 + y^2} \]
And so we must show that
\[ \sqrt{(x+c)^2 + y^2} + \sqrt{(x-c)^2 + y^2} = 2a \iff \frac{x^2}{a^2} + \frac{y^2}{b^2} = 1 \]
Fortunately, we are not doing boring high school algebra, so we'll just assume that this is true.
Thus the cartesian definition implies the eucliean definition.

Now suppose we have $F_1$, $F_2$, and $d$.
Then we redefine the axes such that the $x$ axis is parallel to $F_1F_2$ and the $y$ axis is equidistant from $F_1$ and $F_2$.
Define $a=\frac d2$, and $c=\abs{F_10}$ (ie. half the distance between $F_1$ and $F_2$), and since $c=\sqrt{a^2-b^2}$, this defines $b$.
Now all that remains is to show that the points which satisfy $\frac{x^2}{a^2}+\frac{y^2}{b^2}=1$ are precisely the points which satisfy the euclidean definition of the ellipse defined by $F_1$, $F_2$, and
$d$.
Again, we won't be doing this.

Now, what about equations of the form
\[ \frac{x^2}{a^2} - \frac{y^2}{b^2} = 1\,? \]
In the language of Euclid, this is defined by
\[ \bigl\lvert\abs{F_1A} - \abs{F_2A}\bigr\rvert = d \]
These are called hyperbolas.

And now for parabolas, Euclid defined them as the set of all points which satisfy
\[ \abs{A\ell} = \abs{AF} \]
where $\ell$ is a line (called the directrix), and $F$ is the focal point.
$\abs{A\ell}$ is defined as the metric between a point and a set is usually defined, by taking the infimum of all the distances between points on $\ell$ and $A$.
This corresponds to the length of the line segment perpendicular to $\ell$ which intersects with $A$.

In cartesian terms, what we can do is define the $x$ axis to be parallel to $\ell$ and halfway between it and $F$, and the $y$ axis to pass through $F$.
Let $F=(0,f)$ and $\ell\colon y=-f$.
Then if $A=(x,y)$,
\[ \abs{AF} = \sqrt{x^2+(y-f)^2},\qquad \abs{A\ell} = \abs{y+f} \]
So
\[ \abs{AF} = \abs{A\ell} \iff x^2+(y-f)^2 = (y+f)^2 \iff x^2 = 4fy \iff y = \frac1{4f}x^2 \]

Notice that all of these shapes are equivalent to the set of solutions of an equation of the form $Q(x,y)=0$ where
\[ Q(x,y) = ax^2 + by^2 + cxy + dx + ey + f \]
and two other forms of solutions are lines, or two lines (of the form $y=\pm\alpha x$).

\begin{prop*}

    The set of solutions to $Q(x,y)=0$ is either a line, two lines, an ellipse, a hyperbola, or a parabola.

\end{prop*}

\begin{proof}

    Notice that $Q(x,y)=0$ if and only if
    \[ \pmat{x&y}\pmat{a&\frac c2\\\frac c2&b}\pmat{x\\y} + \pmat{d&e}\pmat{x\\y}+f = 0 \]
    Let $A$ be the diagonal matrix in the equation above.
    Now recall that if a matrix is symmetric, it can be orthogonally diagonalized.
    Suppose that $P$ is the orthogonal matrix which diagonalizes $A$, so
    \[ P^TAP = \pmat{\lambda_1\\&\lambda_2} \]
    Now suppose
    \[ P^T\pmat{x\\y} = \pmat{t\\s} \iff \pmat{x\\y} = P\pmat{t\\s} \]
    Meaning that
    \[ \pmat{x&y} = \pmat{x\\y}^T = \pmat{t&s}P^T \]

    Thus $Q(x,y)=0$ if and only if
    \[ \pmat{t&s}P^TAP\pmat{t\\s} + \pmat{d&e}P\pmat{t\\s} + f = \pmat{t&s}\pmat{\lambda_1\\&\lambda_2}\pmat{t\\s} + \pmat{d&e}P\pmat{t\\s}+f = 0 \]
    if we denote $\pmat{d&e}P=\pmat{d'&e'}$ we get that this is if and only if
    \[ \lambda_1t^2+\lambda_2s^2 + d't + e's + f = 0 \]
    Now utilizing this new equation, we will split into cases.

    \benum
        \item If $\lambda_1,\lambda_2\neq0$, then we can complete the square, the equation is equivalent to
        \[ \lambda_1\parens{t+\frac{d'}{2\lambda_1}}^2 + \lambda_2\parens{s+\frac{e'}{2\lambda_2}}^2 + f-\frac{{d'}^2}{4\lambda_1}-\frac{{e'}^2}{4\lambda_2} = 0 \]
        This is equivalent to an equation of the form
        \[ \lambda_1u^2 + \lambda_2v^2 + f' = 0 \]
        If $f'=0$ then this is $\lambda_1u^2=-\lambda_2v^2$, which defines two lines (with respect to $u$ and $v$).
        Otherwise this defines an ellipse.

        Note that these define shapes with respect to $u$ and $v$, but since $t$ and $s$ are simply some (orthogonal) linear transformation of $x$ and $y$, and $u$ and $v$ are shifts of $t$ and $s$,
        the shape defined in $x$ and $y$ is some orthogonal linear transformation of this ellipse and a shift, which still defines two lines or an ellipse.
        This will be true of the other cases as well.

        \item If $\lambda_2=0$ and $\lambda_1\neq0$ then we get
        \[ \lambda_1t^2 + d't + e's + f = 0 \]
        which defines a parabola (complete the square).
        Similar for if $\lambda_1=0$ and $\lambda_2\neq0$.

        \item If $\lambda_1=\lambda_2=0$ then we get
        \[ d't + e's + f = 0 \]
        which defines a line.
        \qed
    \eenum

\end{proof}

\begin{coro*}

    The only bound set of the form $A=\set{(x,y)}[Q(x,y)=0]$ is an ellipse.

\end{coro*}


