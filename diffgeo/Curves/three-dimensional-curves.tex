Let $\alpha$ be the natural parameterization of some curve.
In two dimensions, recall that we define $N_\alpha$ by rotating $T_\alpha=\alpha'$ ninety degrees.
But rotation by ninety degrees has less meaning in three dimensions, as there are an infinite number of planes on which we can rotate ninety degrees.
But recall by \ppref[proposition]{constVecField} that if $\norm{T_\alpha}$ is constant, then $T'_\alpha$ is orthogonal to $T_\alpha$.
Thus we can define $N_\alpha$ to be the unit vector in the direction of $T'_\alpha$, ie $N_\alpha=\frac{T_\alpha'}{\norm{T_\alpha'}}$.
And recall that we defined curvature as the scalar function $\kappa_\alpha$ such that
\[ T'_\alpha = \kappa_\alpha N_\alpha \]
in three dimensions this becomes
\[ T'_\alpha = \kappa_\alpha\frac{T_\alpha'}{\norm{T_\alpha'}} \implies \kappa_\alpha = \norm{T'_\alpha} \]
So in three dimensions, curvature is always positive, while in two dimensions it may be signed.

But $\set{T_\alpha,N_\alpha}$ is not yet an orthonormal basis, we require one more vector.
We can obtain it by simply defining
\[ B_\alpha = T_\alpha\times N_\alpha \]
this is orthogonal to $T_\alpha$ and $N_\alpha$ and since $\norm{T_\alpha}=\norm{N_\alpha}=1$, $\norm{B_\alpha}=1$.
So $\set{T_\alpha,N_\alpha,B_\alpha}$ is an orthonormal basis.

Let us summarize the definitions:

\begin{defn*}

    Let $\alpha$ be a natural parameterization, then we define
    \benum
        \item $T_\alpha(s)$ as $\alpha'(s)$.
        \item $N_\alpha(s)=\frac{T'_\alpha(s)}{\norm{T'_\alpha(s)}}$.
        \item $B_\alpha(s)=T_\alpha(s)\times N_\alpha(s)$.
    \eenum

    And the \ppemph{curvature} of $\alpha$ is defined to be $\kappa_\alpha(s)=\norm{T'_\alpha(s)}$.
    Or alternatively
    \[ \kappa_\alpha(s) = \iprod{T'_\alpha(s), N_\alpha(s)} \]

\end{defn*}

Now, since $\norm{B}=1$, $B'$ is orthogonal to $B$ and so
\[ B' = \iprod{B',T}T + \iprod{B',N}N + \iprod{B',B}B = \iprod{B',T}T + \iprod{B',N}N \]
And since we know that $\iprod{B,T}=\iprod{B,N}=0$, we get that by differentiating
\[ \iprod{B',T} = -\iprod{B,T'},\qquad \iprod{B',N} = -\iprod{B,N'} \]
And since
\[ \iprod{B,T'} = \iprod{B,\kappa N} = 0 \]
and so $\iprod{B',T}=0$.
Therefore
\[ B' = \iprod{B',N}N \]

\begin{defn*}

    Let $\alpha$ be the natural parameterization of a curve, we define the \ppemph{torsion} of $\alpha$ to be
    \[ \tau_\alpha(s) = -\iprod{B'_\alpha(s),N_\alpha(s)} = \iprod{B_\alpha(s),N'_\alpha(s)} \]

\end{defn*}

Now, we know that since $T$'s norm is constant, $T'\perp T$ and so
\[ N' = \iprod{N',T}T + \iprod{N',N}N + \iprod{N',B}B = \iprod{N',T}T + \iprod{N',B}B = -\iprod{N,T'}T - \iprod{N,B'}B = -\kappa T + \tau B \]
Thus we have the system of ODEs:
\begin{align*}
    T'_\alpha(s) &= \kappa_\alpha(s)N_\alpha(s)\\
    N'_\alpha(s) &= -\kappa(s)T_\alpha(s) + \tau(s)B_\alpha(s) \\
    B'_\alpha(s) &= -\tau_\alpha(s)N_\alpha(s)
\end{align*}
Or using matrices,
\[ \pmat{T'\\N'\\B'} = \pmat{0 & \kappa & 0 \\ -\kappa & 0 & \tau \\ 0 & -\tau & 0}\pmat{T\\N\\B} \]
Thus if we are given $\kappa_\alpha(s)$ and $\tau_\alpha(s)$, and $T_\alpha(0)$, $N_\alpha(0)$, and $B_\alpha(0)$ then we can solve the ODE for $T_\alpha$ and integrate to get $\alpha$.
In fact, we need only two out of $T_\alpha(0)$, $N_\alpha(0)$, and $B_\alpha(0)$, since that will determine the third.
This proves the fundamental theorem of curves for three dimensions,

\begin{thrm*}[ftoc3d,The\ Fundamental\ Theorem\ of\ Curves]

    Every natural parameterization is determined uniquely by its curvature, torsion, and initial conditions for $T$, $N$, and $B$.

\end{thrm*}

\begin{exam*}

    Suppose we have a natural parameterization
    \[ \gamma(s) = \pmat{\gamma_1(s) \\ \gamma_2(s) \\ 0} \]
    which is a two dimensional curve embedded onto the $[xy]$ plane in $\bR^3$.
    Then
    \[ T(s) = \pmat{\gamma'_1(s) \\ \gamma'_2(s) \\ 0},\qquad T'(s) = \pmat{\gamma''_1(s) \\ \gamma''_2(s) \\ 0} \]
    If $T'(s)=0$ then $\kappa(s)=\norm{T'(s)}=0$ and so $N(s)$ is undefined, and therefore so is $B(s)$ and $\tau(s)$.
    Otherwise since $T'$ is on the $[xy]$ plane, so is $N$.
    Thus since $B=T\times N$, $B=(0,0,\pm1)$.
    Using the right-hand rule, we can see that $B$'s sign is $+1$ when the curve is turning left, and $-1$ when turning right, and so $B$'s sign encodes the sign of the curvature of the curve when viewed
    as a planar curve.

    And since for any neighborhood in which $B$ is defined, $B$ is constant, and so $B'=0$ and thus $\tau(s)=0$ when defined.

\end{exam*}

\begin{lemm*}

    Suppose $f_{ij}$ are all differential at $x_0$ then let us define
    \[ D = \det\pmat{f_{11} & \cdots & f_{1n} \\ \vdots & \ddots & \vdots \\ f_{n1} & \cdots & f_{nn}} \]
    then
    \[ D'(x_0) = \sum_{i=1}^n \det\pmat{f_{11} & \cdots & f_{1n} \\ \vdots & \ddots & \vdots \\ f'_{i1} & \cdots & f'_{in} \\ \vdots & \ddots & \vdots \\ f_{n1} & \cdots & f_{nn}} \]

\end{lemm*}

\begin{proof}

    By definition
    \[ D = \sum_{\sigma\in S_n}\signof\sigma\cdot\prod_{i=1}^n f_{i\sigma(i)} \]
    and thus
    \[ D' = \sum_{\sigma\in S_n}\signof\sigma\cdot\parens{\prod_{i=1}^n f_{i\sigma(i)}}' = \sum_{\sigma\in S_n}\signof\sigma\cdot\sum_{i=1}^n f'_{i\sigma(i)}\cdot\prod_{i\neq j=1}^n f_{j\sigma(j)} 
    = \sum_{i=1}^n\sum_{\sigma\in S_n} f'_{i\sigma(i)}\cdot\prod_{i\neq j=1}^n f_{j\sigma(j)} \]
    Each component in this sum is the determinant of $D$ if instead we swapped the $i$th row with $f'_{i1},\dots,f'_{in}$, thus we get the equality required by the lemma.

\end{proof}

This means that by using the determinant formula for cross products, we have that
\[ (f_1\times f_2)' = f'_1\times f_2 + f_1\times f'_2 \]

\begin{prop*}

    Curvature and torsion are invariant under isometries.

\end{prop*}

\begin{proof}

    Suppose $\phi$ is an isometry, then it is of the form $v\mapsto Av+c$ for some orthogonal matrix $A$ and vector $c$.
    Then if $\alpha$ is a natural parameterization, so is $\phi\circ\alpha$ and
    \[ (\phi\circ\alpha(t))' = J_\phi(\alpha(t))\cdot\alpha'(t) = A\alpha'(t) \]
    And thus $T_{\phi\circ\alpha}=AT_\alpha$ and so
    \[ \kappa_{\phi\circ\alpha} = \norm{T'_{\phi\circ\alpha}} = \norm{AT'_\alpha} = \norm{T'_\alpha} = \kappa_\alpha \]
    so curvature is indeed invariant.

    And
    \[ B_{\phi\circ\alpha} = T_{\phi\circ\alpha}\times N_{\phi\circ\alpha} = \frac1{\norm{T'_{\phi\circ\alpha}}}\cdot (AT_\alpha)\times(AT'_\alpha) = \frac1{\kappa_\alpha}(AT_\alpha)\times(AT'_\alpha) \]
    And so
    \[ B'_{\phi\circ\alpha} = \frac1\kappa\parens{AT'_\alpha\times AT'_\alpha + AT_\alpha\times AT''_\alpha} = \frac1\kappa(AT_\alpha\times AT''_\alpha) \]
    And since $N_{\phi\circ\alpha}=\frac{T'_{\phi\circ\alpha}}{\kappa}=\frac{AT'_\alpha}\kappa$,
    \[ \tau_{\phi\circ\alpha} = \frac1{\kappa^2}\iprod{AT_\alpha\times AT''_\alpha, AT'_\alpha} = \frac1{\kappa^2}\iprod{T_\alpha\times T''_\alpha,T'_\alpha} = \tau_\alpha \qed \]

\end{proof}

\begin{prop*}

    Let $\gamma$ be a regular smooth curve such that $\kappa\neq0$ in the entire domain.
    Then the image of $\gamma$ is contained within a plane if and only if $\tau=0$ everywhere.

\end{prop*}

\begin{proof}

    If $\gamma$ is contained within a plane, we can compose it with an isometry to move the plane to $[xy]$.
    Then we showed in the above example that the torsion of the transformed curve is zero.
    And since torsion is preserved under isometries, $\gamma$'s torsion is zero.

    Now, if $\tau=0$ then $B'=0$ so $B=w$ is constant.
    Since $T$ is orthogonal to $B$, $\iprod{T,w}=0$ and so
    \[ (\iprod{\gamma,w}') = \iprod{\gamma',w} + \iprod{\gamma,0} = \iprod{T,w} = 0 \]
    and thus $\iprod{\gamma,w}=c$ is constant.
    Thus $\gamma$ is contained within the plane $\set{x}[\iprod{x,w}=c]$.
    \qed

\end{proof}

In two dimensions, the osculating circle is the circle centered at
\[ c(s_0) = \alpha(s_0) + \frac1{\kappa(s_0)}N(s_0) \]
whose radius is $\frac1{\kappa(s_0)}$.
Using the same derivation for the two dimensional case, we get the same result in three dimensions.

The plane spanned by $T(s_0)$ and $N(s_0)$ which contains $\alpha(s_0)$ is the \emph{tangent plane} to the curve at $s_0$.

How do we compute $\kappa$ and $\tau$ for arbitrary regular curves?
Let $\beta$ be an arbitrary regular curve and $\gamma$ its natural parameterization:
\[ \gamma = \beta\circ s^{-1} \implies \beta = \gamma\circ s \]
then
\begin{align*}
    \beta'(t) &= s'(t)\gamma'(s(t)),\\
    \beta''(t) &= s''(t)\gamma'(s(t)) + s'(t)^2\gamma''(s(t)),\\
    \beta'''(t) &=s'''(t)\gamma'(s(t))+s''(t)s'(t)\gamma''(s(t))+2s'(t)s''(t)\gamma''(s(t))+s'(t)^3\gamma'''(t)
\end{align*}

Now, $\gamma'=T$ and $\gamma''=T'=\kappa N$ and
\[ \gamma'''=\kappa'N+\kappa N'=\kappa'N+\kappa\parens{-\kappa T+\tau B} = \kappa'N - \kappa^2T + \kappa\tau B \]

And so
\[ \beta'\times\beta'' = (s'T)\times(s''T+(s')^2\kappa N) = (s')^3\kappa B \]
Thus
\[ \norm{\beta'\times\beta''} = \norm{\beta'}^3\kappa \implies \kappa = \frac{\norm{\beta'\times\beta''}}{\norm{\beta'}^3} \]
And
\[ \iprod{\beta''',B} = \iprod{(s')^3\gamma''',B} = (s')^3\kappa\tau \]
Now, $s'=\norm{\beta'}$, and so
\[ (s')^3\kappa= \norm{\beta'}^3\frac{\norm{\beta'\times\beta''}}{\norm{\beta'}^3}= \norm{\beta'\times\beta''} \]
And since
\[ \beta'\times\beta'' = (s')^3\kappa B = \norm{\beta'\times\beta''}B \]
And so
\[ \tau = \frac{\iprod{\beta''',B}}{\norm{\beta'\times\beta''}} = \frac{\iprod{\beta''',\beta'\times\beta''}}{\norm{\beta'\times\beta''}^2} \]
And since
\[ \iprod{\beta''',\beta'\times\beta''} = \iprod{\beta'\times\beta'',\beta'''} = \detof{\beta',\beta'',\beta'''} \]
we get
\[ \tau = \frac{\detof{\beta',\beta'',\beta'''}}{\norm{\beta'\times\beta''}^2} \]
Let us summarize this in the following proposition:

\begin{prop*}

    If $\beta$ is an arbitrary regular smooth curve in $\bR^3$, then its curvature and torsion are given by
    \[ \kappa_\beta(s) = \frac{\norm{\beta'(s)\times\beta''(s)}}{\norm{\beta'(s)}^3},\qquad \tau_\beta(s) = \frac{\det\bigl(\beta'(s),\beta''(s),\beta'''(s)\bigr)}{\norm{\beta'(s)\times\beta''(s)}^2} \]

\end{prop*}

