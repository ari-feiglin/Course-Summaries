\begin{defn*}

    Let $\alpha$ be a natural parameterization.
    We define $T_\alpha(s)=\alpha'(s)$, and in the case that we are in $2$ dimensions, we define $N_\alpha(s)=R_{\frac\pi2}\cdot T(s)$.
    $R_\theta$ is the rotation matrix
    \[ R_\theta = \pmat{\cos\theta & -\sin\theta \\ \sin\theta & \cos\theta} \]

    Since $\alpha$ is a natural parameterization and $R_\theta$ is orthogonal, $\norm{T_\alpha}=\norm{N_\alpha}=1$ and thus $\set{T(s),N(s)}$ forms an orthonormal basis, called the
    \ppemph{Frenet-Serret Frame}.

\end{defn*}

We can think of $T_\alpha$ as the direction of motion, or the velocity, of $\alpha$, and $T'_\alpha$ as its acceleration.
Since $T_\alpha$ is constant, its derivative is perpendicular to itself, meaning the acceleration of $\alpha$ is orthogonal to its velocity.
We will prove this formally:

\begin{prop*}[constVecField]

    Suppose $V\colon\bR\longto\bR^n$ (ie. $V$ is a vector field over $\bR$), if $\norm V=c$ then $V'\perp V$ whenever $V$ is differentiable.

\end{prop*}

\begin{proof}

    Since $\iprod{V,V}=c^2$ is constant, we have that the function
    \[ f(t) = \iprod{V(t),V(t)} = \sum_{k=1}^n V_i(t)V_i(t) \]
    Is constant and therefore if $V$ is differentiable at $t$, then so must $V_i$ be, and therefore $f(t)$ is.
    And since $f$ is constant, $f'(t)=0$.
    Therefore
    \[ f'(t) = \sum_{k=1}^n V_i'(t)V_i(t) + V_i(t)V_i'(t) = \iprod{V'(t),V(t)} + \iprod{V(t),V'(t)} = 0 \]
    And since this inner product is over $\bR$, this means $\iprod{V,V'}=0$ so $V'\perp V$ as required.
    \qed

\end{proof}

So when $n=2$, this means that $T'_\alpha$ is parallel with $N_\alpha$ and so
\[ T'_\alpha(s) = \kappa(s)\cdot N_\alpha(s) \]
For some function $k\colon\bR\longto\bR$.
In fact, since $\set{T_\alpha,N_\alpha}$ is an orthonormal basis,
\[ T' = \iprod{T',T}T + \iprod{T',N}N = \iprod{T',N}N \]
So $\kappa(s)=\iprod{T'(s),N(s)}$.

Let us look at this function $\kappa$.
\benum
    \item When $\kappa(s)=0$, then $T'(s)=0$ and so there is no acceleration, and we are moving in a straight line.
    \item When $\kappa(s)>0$, then the curve $\alpha$ is accelerating away from $T$ ``upward'' (toward $N$), and this creates a steep curve.
    \item When $\kappa(s)<0$, the curve is accelerating away from $T$ ``downward'', also creating a steep curve.
\eenum

Thus $\kappa$ can be seen as a measure of curvature.

\begin{defn*}

    The \ppemph{curvature} of a regular two-dimensional curve $\alpha$ at point $s$ is defined to be
    \[ \kappa(s) = \iprod{T'_\alpha(s), N_\alpha(s)} \]
    Where $T_\alpha$ and $N_\alpha$ are taken as their values for the natural reparameterization of $\alpha$.

\end{defn*}

Notice that
\[ N' = \parens{\pmat{0 & -1 \\ 1 & 0}T}' = \pmat{0 & -1 \\ 1 & 0}T' = \kappa\pmat{0 & -1 \\ 1 & 0}N = \kappa\pmat{0 & -1 \\ 1 & 0}^2T = \kappa\pmat{-1 & 0 \\ 0 & -1}T = -\kappa T \]
Therefore $T$ and $N$ are solutions to the ODE,
\[ T' = kappa N,\quad N'=-\kappa T \]
Thus by the uniqueness theorem for ODEs, if we are given the function $\kappa(s)$, and $N(0)$ and $T(0)$, then we can solve for $N$ and $T$.
Since $N$ is determined by $T$, we need only $T(0)$ and $\kappa(s)$.
And since $T=\alpha'$,
\[ \alpha(s) - \alpha(0) = \int_0^s T \]
for all $s$, so if we are given $T$ and $\alpha(0)$, we can find $\alpha(s)$.
Thus given $\kappa(s)$, $\alpha(0)$, and $T(0)$ we can determine $\alpha$.

\begin{thrm*}[ftoc,The\ Fundamental\ Theorem\ of\ Curves]

    Every regular curve is uniquely determined by its curvature, initial position, and $T(0)$.

\end{thrm*}

Now, recall that
\[ \kappa(s) = \iprod{T'(s), N(s)} = \iprod{\alpha''(s), \pmat{0 & -1 \\ 1 & 0}\alpha'(s)} = \iprod{\pmat{\alpha''_1(s) \\ \alpha''_2(s)}, \pmat{-\alpha'_2(s) \\ \alpha'_1(s)}} =
\alpha''_2(s)\alpha'_1(s) - \alpha'_2(s)\alpha''_1(s) \]
And so
\[ \kappa(s) = \alpha''_2\alpha'_1 - \alpha'_2\alpha''_1 \]
Where $\alpha$ is the natural parameterization.

\begin{exam*}

    Suppose $\alpha$ is the curve in $\bR^2$ connecting $x$ and $y$, ie.
    \[ \alpha\colon[0,1]\longto\bR^2,\quad s\mapsto x\cdot\frac sL+y\cdot\frac{1-s}L \]
    where $L=\norm{x-y}$.
    Thus
    \[ \alpha'(s) = \frac xL - \frac yL \]
    And so $\alpha''(s)=0$, meaning $\kappa(s)=0$.

\end{exam*}

\begin{exam*}

    Suppose $\alpha$ is the curve which parameterizes the circle of radius $R$,
    \[ \alpha\colon[0,2\pi R]\longto\bR^2,\quad s\mapsto R\parens{\cos\frac sR, \sin\frac sR} \]
    Thus
    \[ \alpha'(s) = \parens{-\sin\frac sR, \cos\frac sR},\qquad \alpha''(s) = -\frac1R\parens{\cos\frac sR,\sin\frac sR} \]
    $\norm\alpha=1$, so $\alpha$ is the natural parameterization.
    And thus
    \[ \kappa(s) = -\frac1R\Bigl(-\sin^2\frac sR + \cos^2\frac sR\Bigr) = \frac1R \]
    So the curvature of a circle of radius $R$ is $\frac1R$.

\end{exam*}

Since the curves are determined by $\alpha(0)$, $T(0)$, and their curvature, by the above two examples, if
\benum
    \item $\kappa(s)=c\neq0$ then $\alpha$ is a circle.
    If $\kappa(s)>0$ then the curve is drawn counterclockwise, and if $\kappa(s)<0$ the curve is parameterized clockwise (the proof above means that $\alpha(-s)$ is a circle of radius $-R$).
    \item $\kappa=0$ then $\alpha$ is a line.
\eenum

Notice that if $\gamma$ is a natural parameterization then
\[ \gamma'(s) = T(s) = \pmat{\cos(\alpha(s)) \\ \sin(\alpha(s))} \]
This means that
\[ \alpha(s) = \mathrm{atan2}(\cos\alpha(s),\sin\alpha(s)) \]

Now we claim that $\kappa(s)=\alpha'(s)$.
Since
\[ T(s) = \pmat{\cos(\alpha(s)) \\ \sin(\alpha(s))} \implies T'(s) = \pmat{-\sin(\alpha(s)) \\ \cos(\alpha(s))}\cdot\alpha'(s) = \pmat{0 & -1 \\ 1 & 0}T\cdot\alpha'(s) = \alpha'(s)N \]
And since $T'(s)=\kappa(s)N$ this means that $\alpha'(s)=\kappa(s)$ as required.

So if we are given $\gamma'=T$, then we can compute $\alpha$ based on $T$ and then taking its derivative gives $\kappa(s)$.

But what if we aren't given the natural parameterization of the curve?
Let $\beta$ be any regular smooth curve, and $\gamma$ its natural parameterization.
Then recall that $\gamma=\beta\circ s_\beta^{-1}$ and so $\beta=\gamma\circ s_\beta$.
Thus
\[ \beta'(t) = s_\beta'(t)\cdot\gamma'(s_\beta(t)) \]
(This is a bit confusing, since $s_\beta$ is a scalar, and $\gamma$ is a vector).
We know that there exists an $\alpha$ such that
\[ \alpha = \atanof{\frac{\gamma'_2}{\gamma'_2}} \]
And since
\[ \frac{\gamma'_2(s)}{\gamma'_1(s)} = \frac{\beta'_2(t)}{\beta'_1(t)} \]
And thus
\[ \alpha(s) = \atanof{\frac{\beta'_2(t)}{\beta'_1(t)}} \]
Recall that the derivative of $\atanof x=\frac1{1+x^2}$, and since
\[ \frac d{ds}\atanof{\frac{\beta'_2(t)}{\beta'_1(t)}} = \frac d{dt}\atanof{\frac{\beta'_2(t)}{\beta'_1(t)}}\cdot\frac{dt}{ds} \]
We have that
\[ \kappa(s) = \alpha'(s) = \frac1{1+\parens{\frac{\beta'_2(t)}{\beta'_1(t)}}^2}\cdot\frac{\beta''_2(s)\beta'_1(s)-\beta'_2(s)\beta''_1(s)}{\beta'_1(s)^2}\cdot\frac{dt}{ds} =
\frac{\beta''_2(s)\beta'_1(s)-\beta'_2(s)\beta''_1(s)}{\beta'_1(t)^2+\beta'_2(t)^2}\cdot\frac{dt}{ds} \]

By definition,
\[ s(t) = \int_0^t \norm{\beta'(u)}\,du \implies s'(t) = \norm{\beta'(t)} \]
So
\[ \frac{dt}{ds} = \frac1{\norm{\beta'(t)}} = \frac1{\sqrt{\beta'_1(s)^2+\beta_2'(s)^2}} \]
And so all in all we have that
\[ \kappa(s) = \frac{\beta''_2(s)\beta'_1(s)-\beta'_2(s)\beta''_1(s)}{\bigl(\beta'_1(t)^2+\beta'_2(t)^2\bigr)^{1.5}} \]
So we have proven the following proposition:

\begin{prop*}

    If $\beta$ is a regular smooth curve, then its curvature is given by
    \[ \kappa(s) = \frac{\beta''_2(s)\beta'_1(s)-\beta'_2(s)\beta''_1(s)}{\bigl(\beta'_1(t)^2+\beta'_2(t)^2\bigr)^{1.5}} \]

\end{prop*}

\begin{exam*}

    So if $\beta(t)=(t,f(t))$ then
    \[ \kappa_\beta(t) = \frac{f''(t)}{(1+f'(t))^{1.5}} \]

    Thus if we have a function $f\colon\bR\to\bR$, then we can discuss its curvature as the parameterization of its graph.

\end{exam*}

Suppose we have a regular smooth curve $\alpha$ which is a natural parameterization.
Our goal is to find the circle tangent to $\alpha$ at the point $s_0$.
\benum
    \item First, we can write $\alpha$ as a second order Taylor series
    \[ \alpha(s_0+h) = \alpha(s_0) + h\alpha'(s_0) + \frac{h^2}2\alpha''(s_0) + \epsilon(h) \]
    where $\epsilon(h)\in o(h^2)$ (meaning $\frac{\norm{\epsilon(h)}}{h^2}\xvarrightarrow{}[h\to0]0$).

    \item Now, we know that $T=\alpha'$ and $\alpha''=T'=\kappa(s)N$ and thus
    \[ \alpha(s_0+h) - \alpha(s_0) = hT(s_0) + \kappa(s_0)\frac{h^2}2N(s_0) + \epsilon(h) \]
    Let us define
    \[ \Delta(h) = \alpha(s_0+h) - \alpha(s_0),\quad x(h) = \iprod{\Delta(h), T(s_0)},\quad y(h) = \iprod{\Delta(h), N(s_0)} \]
    Thus $\Delta(h)=x(h)T(s_0) + y(h)N(s_0)$, and so
    \[ x(h) = \iprod{hT(s_0) + \kappa(s_0)\frac{h^2}2N(s_0) + \epsilon(h), T(s_0)} = h + \iprod{\epsilon(h), T(s_0)} \]
    Since $\norm T=1$ and $T$ and $N$ are orthogonal.
    Now since by Cauchy-Schwarz, $\abs{\iprod{u,v}}\leq\norm u\norm v$, we have that $\iprod{\epsilon(h), T(s_0)}=\epsilon_1(h)\in o(h^2)$.
    Similarly
    \[ y(h) = \kappa(s_0)\cdot\frac{h^2}2 + \epsilon_2(h) \]
    where $\epsilon_1(h)\in o(h^2)$.

    \item Now, let us define the axis system $(T(s_0),N(s_0))$ centered at $\alpha(s_0)$, then since in this axis system $\alpha(s_0)=0$, we will denote $\alpha(s_0+h)$ by $\alpha(h)$, and $T(s_0)$ and
    $N(s_0)$ by $T$ and $N$, and $\kappa(s_0)$ by $k$.
    Thus
    \[ \alpha(h) = x(h)T + y(h)N = hT + k\cdot\frac{h^2}2N + \bigl(\epsilon_1(h)T + \epsilon_2(h)N\bigr) \]
    So given an $h$, we will define a circle through $(0,0)$, $\parens{\pm h,k\cdot\frac{h^2}2}$.
    Such a circle would have the form $(x-a)^2+(y-b)^2=R^2$.
    Let us assume $a=0$ (the reason for assuming this is by symmetry).
    Thus we must have
    \[ b^2 = R^2,\quad h^2 + \parens{k\cdot\frac{h^2}2-b}^2 = R^2 \]
    So
    \[ h^2 + \kappa^2\cdot\frac{h^4}4 - b\kappa h^2+ b^2 = R^2 \implies \kappa^2\cdot\frac{h^4}4 = b\kappa h^2 - h^2 \implies \kappa^2\cdot\frac{h^2}2 = b\kappa - 1 \]
    So as $h\to0$ we get that
    \[ bk = 1 \implies b=\frac1\kappa \implies R=\abs b=\abs{\frac1\kappa} \]
    And the center of the circle is $\parens{0,\frac1\kappa}$.

    \item Now, we know that $(x,y)$ in this axis system corresponds to $xT(s_0)+yN(s_0)+\alpha(s_0)$ in $\bR^2$, and so the circle we got is the set
    \[ \set{\alpha(s_0)+xT(s_0)+yT(s_0)}[x^2+\parens{y-\frac1{\kappa(s_0)}}^2=\frac1{\kappa(s_0)^2}] \]
    We can also see this because the center of the circle is at $\frac1k$ in the new axis system, which is the point
    \[ c(s_0) = \alpha(s_0) + \frac1{\kappa(s_0)}N \]
    And the radius of the circle is still $\frac1{\kappa(s_0)}$ (since the new axis system is simply an isometry).
\eenum

Newton was originally the person who came up with this formula (for the center of the circle and its radius).
The way he approached it was by taking the points $\alpha(s_0)$ and $\alpha(s_0+h)$ and looking at the intersection of the normal lines at these points, $o(h)$.
Then we will show that $o(h)\varrightarrow c(s_0)$.
Let $\ell_1(t)$ and $\ell_2(t)$ be the normal lines at $\alpha(s_0)$ and $\alpha(s_0-h)$ respectively.
We know that
\[ \ell_1(t) = \alpha(s_0) + tN(s_0),\qquad \ell_2(t) = \alpha(s_0+h) + tN(s_0+h) \]
And since we know that
\[ \alpha(s_0+h) = \alpha(s_0) + h\alpha'(s_0) + o(h) = \alpha(s_0) + hT(s_0) + o(h) \]
And
\[ N(s_0+h) = N(s_0) + hN'(h) + o(h) \]
And since $N'=-\kappa(s_0)T$, we have
\[ N(s_0+h) = N(s_0) - h\kappa(s_0)T(h) + o(h) \]
Then $\ell_1(t)=\ell_2(p)$ if and only if
\[ \alpha(s_0) + tN(s_0) = \alpha(s_0) + hT(s_0) + o(h) + p(N(s_0) - h\kappa(s_0)T(s_0) + o(h)) \iff (t-p)N(s_0) = h(1-p\kappa(s_0))T(s_0) + o(h) \]
Meaning that
\[ (p-t)N(s_0) + h(1-p\kappa(s_0))T(s_0) \in o(h) \]
Thus
\[ \frac{p-t}h N(s_0) + (1-p\kappa(s_0))T(s_0) \xvarrightarrow{}[h\to0] 0 \]
Since $N$ and $T$ are orthonormal, this means that $p-t=0$ and $1-p\kappa(s_0)=0$.
So $t=p=\frac1{\kappa(s_0)}$.
And so the center point is
\[ c(s_0) = \alpha(s_0) + \frac1{\kappa(s_0)}N(s_0) \]
as we showed before.
Let us summarize this in the following definition:

\begin{defn*}

    If $\alpha$ is a regular smooth planar curve, then the \ppemph{osculating circle} of $\alpha$ at the point $s_0$ (this is the input, we could also think of it as the point $\alpha(s_0)$) is the circle
    centered at
    \[ c_\alpha(s_0) = \alpha(s_0) + \frac1{\kappa_\alpha(s_0)}N_\alpha(s_0) \]
    and whose radius is $\frac1{\kappa_\alpha(s_0)}$.
    The curve $c_\alpha$ is called the \ppemph{evolute} of $\alpha$.

\end{defn*}

Suppose $\alpha$ is a natural parameterization, and $\phi\colon v\varmapsto Ax+c$ is an isometry (and so $A$ is orthonormal).
Then let $\beta=\phi\circ\alpha$, so
\[ \beta(s) = A\alpha(s) + c \]
Then $\beta'(s)=A\alpha'(s)$, and since $A$ is orthonormal, $\norm{\beta'}=\norm{\alpha'}=1$ since $\alpha$ is natural.
Thus $\beta$ is also a natural parameterization.
And so
\[ \kappa_\beta(s) = \iprod{\beta''(s),R_{\frac\pi2}\beta'(s)} = \iprod{A\alpha''(s),R_{\frac\pi2}A\alpha'(s)} \] 
Now, rotations and $A$ commute up to sign.
If $\detof A=1$ then they commute, and if $\detof A=-1$ then $R_\theta A=-AR_\theta$.
So this is equal to $\detof A\iprod{A\alpha''(s), AR_{\frac\pi2}\alpha'(s)}$, since $A$ is orthogonal this is equal to
\[ = \detof A\iprod{\alpha''(s), R_{\frac\pi2}\alpha'(s)} = \pm \kappa_\alpha(s) \]
So we have proven the following:

\begin{prop*}

    If $A$ is an orthogonal matrix, and $c$ a vector then $\phi\colon x\mapsto Ax+c$ is an isometry, and if $\alpha$ is a natural parameterization, then so is $\beta=\phi\circ\alpha$, and
    $\kappa_\alpha=\kappa_\beta$.

\end{prop*}

