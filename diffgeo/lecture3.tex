\documentclass[10pt]{article}

\usepackage{amsmath, amssymb, mathtools}
\usepackage[margin=1.5cm]{geometry}

\input pdfmsym
\input prettyprint
\input preamble

\pdfmsymsetscalefactor{10}
\initpps

\def\pmat#1{\begin{pmatrix} #1 \end{pmatrix}}

\let\divides=\mid
\newfunc{metric}\rho({})
\newfunc{metricc}\sigma({})
\newfunc{spa}{{\rm span}}(\vert)
\newfunc{diam}{{\rm diam}}(\vert)
\newfunc{proj}\pi({})
\newfunc{iproj}{\pi^{-1}}({})
\newfunc{cis}{{\rm cis}}({})
\newfunc{Re}{{\rm Re}}({})
\newfunc{Im}{{\rm Im}}({})
\newfunc{sup}{{\rm sup}}\{\vert\}
\newfunc{Res}{{\rm Res}}({})
\newfunc{wind}n({})
\newfunc{pv}{{\rm pv}}({})
\newfunc{lspan}{{\rm span}}({})

\font\bigbf = cmbx12 scaled 2000
\@undervecc@def{underbar}\@linecap\@linecap

\def\pmat#1{\begin{pmatrix}#1\end{pmatrix}}

\def\mO{{\cal O}}
\def\mU{{\cal U}}
\let\lineseg=\overleftrightvecc
\let\to=\varrightarrow
\let\longto=\longvarrightarrow
\let\ds=\displaystyle

\def\pdv#1#2{\frac{\partial #1}{\partial #2}}

\def\differ#1#2{\left.d#1\strut\right|_{#2}}

\def\qed{%
    \ifmmode%
        \eqno\blacksquare%
    \else%
        \hskip1cm\allowbreak\hbox{}\nobreak\hfill$\blacksquare$%
    \fi%
}

\def\@ppmathcount{\thesection.\thepp@mathcount}

\begin{document}

\c@section=3

\barcolorbox{220, 255, 220}{0, 130, 0}{80, 200, 80}{
    \leftskip=0pt plus 1fill \rightskip=\leftskip
    {\bigbf Differential and Analytic Geometry}

    \medskip
    \textit{Lecture \thesection, Monday July 17, 2023}

    \textit{Ari Feiglin}
}

\bigskip

Recall that if $\alpha\colon[a,b]\longto\bR^n$ is a regular smooth curve, then we define its \emph{natural parameterization} as the curve
\[ \beta\colon[0,L]\longto\bR^n \]
where $L=s_\alpha(b)$ is the arclength of $\alpha$ by
\[ \beta(u) = \alpha\circ s_\alpha^{-1}(u) \]
And this is unique (up to reparameterization).
A curve from $[0,L]$ is a natural parameterization if and only if $\norm{\alpha'}=1$.

\begin{defn*}

    Let $\alpha$ be a natural parameterization.
    We define $T_\alpha(s)=\alpha'(s)$, and in the case that we are in $2$ dimensions, we define $N_\alpha(s)=R_{\frac\pi2}\cdot T(s)$.
    $R_\theta$ is the rotation matrix
    \[ R_\theta = \pmat{\cos\theta & -\sin\theta \\ \sin\theta & \cos\theta} \]

    Since $\alpha$ is a natural parameterization and $R_\theta$ is orthogonal, $\norm{T_\alpha}=\norm{N_\alpha}=1$ and thus $\set{T(s),N(s)}$ forms an orthonormal basis, called the
    \ppemph{Frenet-Serret Frame}.

\end{defn*}

We can think of $T_\alpha$ as the direction of motion, or the velocity, of $\alpha$, and $T'_\alpha$ as its acceleration.
Since $T_\alpha$ is constant, its derivative is perpendicular to itself, meaning the acceleration of $\alpha$ is orthogonal to its velocity.
We will prove this formally:

\begin{prop*}

    Suppose $V\colon\bR\longto\bR^n$ (ie. $V$ is a vector field over $\bR$), if $\norm V=c$ then $V'\perp V$ whenever $V$ is differentiable.

\end{prop*}

\begin{proof}

    Since $\iprod{V,V}=c^2$ is constant, we have that the function
    \[ f(t) = \iprod{V(t),V(t)} = \sum_{k=1}^n V_i(t)V_i(t) \]
    Is constant and therefore if $V$ is differentiable at $t$, then so must $V_i$ be, and therefore $f(t)$ is.
    And since $f$ is constant, $f'(t)=0$.
    Therefore
    \[ f'(t) = \sum_{k=1}^n V_i'(t)V_i(t) + V_i(t)V_i'(t) = \iprod{V'(t),V(t)} + \iprod{V(t),V'(t)} = 0 \]
    And since this inner product is over $\bR$, this means $\iprod{V,V'}=0$ so $V'\perp V$ as required.
    \qed

\end{proof}

So when $n=2$, this means that $T'_\alpha$ is parallel with $N_\alpha$ and so
\[ T'_\alpha(s) = k(s)\cdot N_\alpha(s) \]
For some function $k\colon\bR\longto\bR$.
In fact, since $\set{T_\alpha,N_\alpha}$ is an orthonormal basis,
\[ T' = \iprod{T',T}T + \iprod{T',N}N = \iprod{T',N}N \]
So $k(s)=\iprod{T'(s),N(s)}$.

Let us look at this function $k$.
\benum
    \item When $k(s)=0$, then $T'(s)=0$ and so there is no acceleration, and we are moving in a straight line.
    \item When $k(s)>0$, then the curve $\alpha$ is accelerating away from $T$ ``upward'' (toward $N$), and this creates a steep curve.
    \item When $k(s)<0$, the curve is accelerating away from $T$ ``downward'', also creating a steep curve.
\eenum

Thus $k$ can be seen as a measure of curvature.

\begin{defn*}

    The \ppemph{curvature} of a regular two-dimensional curve $\alpha$ at point $s$ is defined to be
    \[ k(s) = \iprod{T'_\alpha(s), N_\alpha(s)} \]
    Where $T_\alpha$ and $N_\alpha$ are taken as their values for the natural reparameterization of $\alpha$.

\end{defn*}

Notice that
\[ N' = \parens{\pmat{0 & -1 \\ 1 & 0}T}' = \pmat{0 & -1 \\ 1 & 0}T' = k\pmat{0 & -1 \\ 1 & 0}N = k\pmat{0 & -1 \\ 1 & 0}^2T = k\pmat{-1 & 0 \\ 0 & -1}T = -kT \]
Therefore $T$ and $N$ are solutions to the ODE,
\[ T' = kN,\quad N'=-kT \]
Thus by the uniqueness theorem for ODEs, if we are given the function $k(s)$, and $N(0)$ and $T(0)$, then we can solve for $N$ and $T$.
Since $N$ is determined by $T$, we need only $T(0)$ and $k(s)$.
And since $T=\alpha'$,
\[ \alpha(s) - \alpha(0) = \int_0^s T \]
for all $s$, so if we are given $T$ and $\alpha(0)$, we can find $\alpha(s)$.
Thus given $k(s)$, $\alpha(0)$, and $T(0)$ we can determine $\alpha$.

\begin{thrm*}[ftoc,The\ Fundamenta\ Theorem\ of\ Curves]

    Every regular curve is uniquely determined by its curvature, initial position, and $T(0)$.

\end{thrm*}

Now, recall that
\[ k(s) = \iprod{T'(s), N(s)} = \iprod{\alpha''(s), \pmat{0 & -1 \\ 1 & 0}\alpha'(s)} = \iprod{\pmat{\alpha''_1(s) \\ \alpha''_2(s)}, \pmat{-\alpha'_2(s) \\ \alpha'_1(s)}} =
\alpha''_2(s)\alpha'_1(s) - \alpha'_2(s)\alpha''_1(s) \]
And so
\[ k(s) = \alpha''_2\alpha'_1 - \alpha'_2\alpha''_1 \]
Where $\alpha$ is the natural parameterization.

\begin{exam*}

    Suppose $\alpha$ is the curve in $\bR^2$ connecting $x$ and $y$, ie.
    \[ \alpha\colon[0,1]\longto\bR^2,\quad s\mapsto x\cdot\frac sL+y\cdot\frac{1-s}L \]
    where $L=\norm{x-y}$.
    Thus
    \[ \alpha'(s) = \frac xL - \frac yL \]
    And so $\alpha''(s)=0$, meaning $k(s)=0$.

\end{exam*}

\begin{exam*}

    Suppose $\alpha$ is the curve which parameterizes the circle of radius $R$,
    \[ \alpha\colon[0,2\pi R]\longto\bR^2,\quad s\mapsto R\parens{\cos\frac sR, \sin\frac sR} \]
    Thus
    \[ \alpha'(s) = \parens{-\sin\frac sR, \cos\frac sR},\qquad \alpha''(s) = -\frac1R\parens{\cos\frac sR,\sin\frac sR} \]
    $\norm\alpha=1$, so $\alpha$ is the natural parameterization.
    And thus
    \[ k(s) = -\frac1R\Bigl(-\sin^2\frac sR + \cos^2\frac sR\Bigr) = \frac1R \]
    So the curvature of a circle of radius $R$ is $\frac1R$.

\end{exam*}

Since the curves are determined by $\alpha(0)$, $T(0)$, and their curvature, by the above two examples, if
\benum
    \item $k(s)=c\neq0$ then $\alpha$ is a circle.
    If $k(s)>0$ then the curve is drawn counterclockwise, and if $k(s)<0$ the curve is parameterized clockwise (the proof above means that $\alpha(-s)$ is a circle of radius $-R$).
    \item $k=0$ then $\alpha$ is a line.
\eenum

Notice that if $\gamma$ is a natural parameterization then
\[ \gamma'(s) = T(s) = \pmat{\cos(\alpha(s)) \\ \sin(\alpha(s))} \]
This means that
\[ \alpha(s) = \mathrm{atan2}(\cos\alpha(s),\sin\alpha(s)) \]

Now we claim that $k(s)=\alpha'(s)$.
Since
\[ T(s) = \pmat{\cos(\alpha(s)) \\ \sin(\alpha(s))} \implies T'(s) = \pmat{-\sin(\alpha(s)) \\ \cos(\alpha(s))}\cdot\alpha'(s) = \pmat{0 & -1 \\ 1 & 0}T\cdot\alpha'(s) = \alpha'(s)N \]
And since $T'(s)=k(s)N$ this means that $\alpha'(s)=k(s)$ as required.

So if we are given $\gamma'=T$, then we can compute $\alpha$ based on $T$ and then taking its derivative gives $k(s)$.

\end{document}
