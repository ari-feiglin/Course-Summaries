\input pdfToolbox

\setlayout{horizontal margin=2cm, vertical margin=2cm}
\parindent=0pt
\parskip=3pt plus 2pt minus 2pt

\input preamble

\def\printmcount{\the\counter{section}.\the\counter{math counter}}

\footline={}

%%%%%%%%%%%%%%%%%%%%%%%%%%%%%%%%%%%%%%%%%%%%%%%%%%%%%%%%%%%%%%%%

\headline={\pageborder{rgb{1 .5 .5}}{rgb{.4 0 0}}{5}}

\color rgb{.1 0 .8}

{\def\boxshadowcolor{rgb{.8 .3 .3}}
\bppbox{rgb{1 .5 .5}}{rgb{.4 0 0}}{rgb{.4 0 .1}}

    \centerline{\setfontandscale{bf}{20pt}Automorphisms and Logical Equivalences}
    \centerline{\setfontandscale{bf}{20pt}of Linear Groups}
    \smallskip
    \centerline{\setfont{it}Lectures by Elena Bunina}
    \centerline{\setfont{it}Summary by Ari Feiglin \setfont{rm}({\tt ari.feiglin@gmail.com})}

\eppbox

\bigskip

\bppbox{rgb{1 .5 .5}}{rgb{.4 0 0}}{rgb{.4 0 .1}}
    \section*{Contents}
    
    \tableofcontents
\eppbox

}

\vfill\break

\color{black}

\pageno=1
\newif\ifpageodd
\pageoddtrue
\headline={%
    \hbox to \hsize{\color{black}%
        \ifpageodd\hfil{\it\currsubsection\quad\bf\folio}\global\pageoddfalse%
        \else{\bf\folio\quad\it\currsubsection}\hfil\global\pageoddtrue\fi%
    }%
}

\section{Linear Groups}

Let $R$ be a ring and $V$ a free $R$-module with rank $n$ (meaning it has a basis, and all its basis have cardinality $n$).
We define the {\it general linear group} of $V$ to be $\GLof V$, the group of invertible $R$-linear endomorphisms over $V$.
This is contained within the group $\Endof V$, the group of all $R$-linear endomorphisms over $V$.

By fixing a basis of $V$, we can identify $\GLof V$ with $\GLof[n]R$, the group of invertible $n\times n$ $R$-matrices.
We define the {\it special linear group} $\SLof[n]R$ to be the group of all invertible $n\times n$ $R$-matrices with a determinant of $1$.

\bdefn

    Let $I=I_n$ be the $n\times n$ identity matrix, $E_ij$ to be the standard unit matrix (i.e. $(E_{ij})_{\ell k}=1$ iff $i=\ell,j=k$ otherwise zero).
    Then define the {\emphcolor elementary transvection matrix} to be $t_{ij}(\lambda)=I+\lambda E_{ij}$ for $i\neq j$ and $\lambda\in R$.

\edefn

Notice that
$$ t_{ij}(\lambda)t_{ij}(\delta) = I + \lambda E_{ij} + \delta E_{ij} + \lambda\delta E_{ij}^2 = I + (\lambda+\delta)E_{ij} = t_{ij}(\lambda+\delta) $$
If we define $X_{ij}=\set{t_{ij}(\lambda)}[\lambda\in R]$ then $X_{ij}$ is an Abelian subgroup of $\SLof[n]R$.

Define $E_n(R)$ to be the group generated by all elementary transvection matrices, called the {\it elementary linear group}.
$E_n(R)$ contains the following set of automorphisms which are called {\it standard}:

\benum
    \item Let $S/R$ be a (suitable; i.e. the following definition is well-defined) ring extension, and $g\in\GLof[n]S$, then define $\iota_g$ to be the inner automorphism generated by $g$: $a\mapsto g^{-1}a$
    This is of course an {\it inner automorphism}, if $g\in\GLof[n]R$ then this is a {\it strict inner automorphism}.
    \item If $\delta\colon R\longto R$ is an $R$-automorphism, then $\bar\delta$ defined by
    $$ \bar\delta\colon (a_{ij})\mapsto(\delta(a_{ij})) $$
    is a an automorphism in $E_n(R)$.
    In the case that $A=t_{ij}(\lambda)$ notice that $\bar\delta(t_{ij}(\lambda))=t_{ij}(\delta\lambda)$.
    \item If $e\in R$ is idempotent, meaning $e^2=e$, then
    $$ \Lambda_e\colon A\mapsto (A^\top)^{-1}e+A(1-e) $$
    is also in $E_n(R)$.
    In the case that $R$ has no idempotents other than the identity, then we simply write $\Lambda\colon A\mapsto (A^\top)^{-1}$.
    \middletext Compositions of automorphisms of the above forms $(1)-(3)$ are called {\it standard} in $E_n(R)$.
    Beyone these automorphisms, $\GLof[n]R$ and $\SLof[n]R$ have another form of automorphism:
    \item If $\gamma$ is some homomorphism from $\SLof[n]R$ or $\GLof[n]R$ to the center of the group, then
    $$ \Gamma_\gamma\colon A\mapsto\gamma(A)A $$
    is an automorphism.
    \middletext A composition of automorphisms of the form $(1)-(4)$ is called {\it standard} in $\GLof[n]R$ or $\SLof[n]R$.
\eenum

\bthrm

    All automorphisms of $E_n(R)$ for $n\geq4$ and $R$ commutative are standard.
    If $2\in R^\times$ then all the automorphisms of $E_3(R)$ are also standard.

\ethrm

\bthrm

    All automorphisms of $\GLof[n]R$ and $\SLof[n]R$ for $n\geq4$ and $R$ commutative are standard.
    If $2\in R^\times$ then all automorphisms of $\GLof[3]R$ and $\SLof[3]R$ are also standard.

\ethrm

\bye

