Recall the following methods of constructing fields:
\benum
    \item If $R$ is a commutative ring and $M\triangleleft R$ is a maximal ideal that $\slfrac RM$ is a field.
    In particular if $F$ is a field, $R=F[x]$, and $p$ is an irreducible polynomial then $(p)$ is maximal and so $\slfrac{F[x]}{(p)}$ is a field.
    \item If $F$ is a field, so is the field of rational functions:
    $$ F \subseteq F(x) \coloneqq \set{\frac{f(x)}{g(x)}}[{f,g\in F[x],g\neq0}] $$
    \item If $C$ is a chain of fields (meaning that for every $F,F'\in C$ either $F\subseteq F'$ or $F'\subseteq F$), then $\bigcup_{F\in C}F$ is also a field (the theory of fields is {\it inductive}).
    So for example $F(\lambda_1,\lambda_2,\dots)$ is a field, the union of the chain $F_n=F(\lambda_1,\dots,\lambda_n)$, the field of rational functions over $F_{n-1}$.
    \item If $C$ is a chain of fields, then $\bigcap_{F\in C}F$ is also a field.
\eenum

\bdefn

    Let $K/F$ be a field extension and $a\in K$, then denote $F(a)$ the smallest subfield of $K$ containing both $F$ and $a$.

\edefn

It is not hard to see that
$$ F(a) = \set{\frac{f(a)}{g(a)}}[{f,g\in F[x],g(a)\neq0}] $$
Though we can actually get a simpler structure for $F(a)$.

\bdefn

    Let $K/F$ be a field extension with $a\in K$, then define the {\emphcolor evaluation homomorphism} at $a$ to be the homomorphism $\psi_a\colon F[x]\longto K$ defined by $\psi_a(s)=s$ for $s\in F$ and
    $\psi_a(x)=a$.
    This uniquely defines
    $$ \psi_a\parens{\sum\alpha_ix^i} = \sum\alpha_ia^i $$

\edefn

\bdefn

    Let $K/F$ be a field extension, then $a\in K$ is {\emphcolor transcendental} if the kernel of the evaluation homomorphism is trivial: $\ker\psi_a=1$.
    Otherwise $a$ is {\emphcolor algebraic}.

\edefn

If $a$ is transcendental then $\ker\psi_a=1$ and so by the isomorphism theorem
$$ {\rm Im}\psi_a = \set{f(a)}[{f\in F[x]}] = F[a] \cong \slfrac{F[x]}{\ker\psi_a} \cong F[x] $$
In fact we can extend $\psi_a$ to a homomorphism $F(x)\longto F(a)$, and we similarly get an isomorphism $F(x)\cong F(a)$.
Thus in the case that $a$ is transcendental, we get

\medskip
\centerline{\vbox{\tabskip=3pt\halign{&\hfil${}#{}$\hfil\cr
F &\subseteq& F[a] &\subseteq& F(a) &\subseteq& K\cr
  &         &\cong &         &\cong\cr
  &         & F[x] &         & F(x)\cr
}}}
\medskip

Otherwise, suppose $a$ is algebraic.
Since $F[x]$ is a Euclidean domain, it is a PID, and therefore every ideal is a prime ideal.
In particular $\ker\psi_a$ must be generated by some polynomial $h_a$.
This means that $\ker\psi_a=(h_a)=h\cdot F[x]$, and so $h_a(a)=0$ and if $f(a)=0$ as well then $h_a$ divides $f$.
$h_a$ is therefore called the {\it minimal polynomial} of $a$.

Now if $n=\deg h$ then $F[a]=\lspanof{1,a,\dots,a^{n-1}}$ since if $f\in F[x]$ then $f=h_aq+r$ for $\deg r<n$ by Euclidean division, and so $f(a)=r(a)$.
And $r(x)$ is in $\lspanof{1,\dots,a^{n-1}}$ due to its dimension being at most $n-1$.
Thus $\set{1,\dots,a^{n-1}}$ spans $F[a]$, and it is a basis since any linear combination cannot be zero as $h_a(x)$ is minimal and has degree $n$.
Therefore $F[a]$ is a $F$-linear space of dimension $n$.

Notice that
$$ \slfrac{F[x]}{(h_a)} = \slfrac{F[x]}{\ker\psi_a} \cong {\rm Im}\psi_a = \set{f(a)}[{f\in F[x]}] = F[a] = \lspanof{1,\dots,a^{n-1}} \subseteq K $$
Since $K$ is an integral domain, so is $F[a]$.
Therefore $(h_a)$ is a prime ideal, since a quotient ring is an integral domain iff the ideal is prime.
Since $F[x]$ is a PID, prime and maximal ideals are the same, so $(h_a)$ is maximal and therefore $F[a]$ is a field.

So we have proven

\bprop

    Let $K/F$ be a field extension and $a\in K$ algebraic in $F$.
    Let $h_a$ be $a$'s minimal polynomial over $F$, then
    \benum
        \item $h_a$ is irreducible,
        \item $F[a]$ is a field,
        \item $[F[a]:F]=n=\deg h_a$ and has a basis $\set{1,a,\dots,a^{n-1}}$.
    \eenum

\eprop

In particular we have shown that when $a$ is algebraic, $F(a)=F[a]$.

\bprop

    Suppose $F\subseteq K$ where $F$ is a field and $K$ is an integral domain.
    Further suppose $[K:F]$ is finite.
    Then every element of $K$ is algebraic and $K$ is a field.

\eprop

\Proof let $a\in K$, then
$$ [K:F] = [K:F[a]]\cdot[F[a]:F] $$
meaning $[F[a]:F]$ must be finite and so $a$ must be algebraic (as otherwise $F[a]\cong F[x]$ which has infinite degree).
Since $F[a]$ is a field, it must have a multiplicative inverse for $a$, meaning $K$ is a field.
\qed

Notice that $[F[a,b]:F[a]]\leq[F[b]:F]$, since the minimal polynomial of $b$ relative to $F$, $h_b$, is also a zeroing a polynomial of $b$ over $F[a]$.
And so $[F[a,b]:F[a]]\leq\deg h_b=[F[b]:F]$.
Thus we have that by multiplicity
$$ [F[a,b]:F] = [F[a,b]:F[a]]\cdot[F[a]:F] \leq [F[b]:F]\cdot[F[a]:F] $$
And inductively we can show

\bprop

    Suppose $K/F$ is a field extension and $a_1,\dots,a_n$ then
    $$ \bigl[F[a_1,\dots,a_n]:F\bigr] \leq \prod_{i=1}^n\bigl[F[a_i]:F\bigr] $$

\eprop

\bdefn

    Call a field extension $K/F$ {\emphcolor algebraic} if every $a\in K$ is algebraic over $F$.

\edefn

\blemm

    Suppose $F_3/F_2/F_1$ are field extensions such that $F_2/F_1$ is algebraic and $a\in F_3$ is algebraic over $F_2$.
    Then it is also algebraic over $F_1$.

\elemm

\Proof there exists an $f\in F_2[x]$ such that $f(a)=0$.
Suppose $f=\sum b_ix^i$, then $a$ is algebraic over $F_1[b_0,\dots,b_n]$.
Then
$$ [F_1[b_0,\dots,b_n,a]:F_1] = [F_1[b_0,\dots,b_n,a]:F_1[b_0,\dots,b_n]]\cdot[F_1[b_0,\dots,b_n]:F_1] $$
and since $a$ is algebraic over $F_1[b_0,\dots,b_n]$ and $b_i\in F_2$ are algebraic over $F_1$, the right-hand side is finite.
Thus $a$ is algebraic over $F_1$ by the left-hand side, as required.
\qed

\bthrm

    Let $K/F$ be a field extension, then
    $$ {\rm Alg}_F(K) \coloneqq \set{a\in K}[\hbox{$a$ is algebraic over $F$}] $$
    is a field.
    Furthermore, every element in $K\setminus{\rm Alg}_F(K)$ is transcendental over ${\rm Alg}_F(K)$.

\ethrm

\Proof notice that $F[a\cdot b],F[a+b]\subseteq F[a,b]$ and $[F[a,b]:F]\leq[F[a]:F]\cdot[F[b]:F]<\infty$ for $a,b\in{\rm Alg}_F(K)$.
So ${\rm Alg}_F(K)$ is closed under addition and multiplication.
It is also obviously closed under additive inverses since $F[-a]=F[a]$.
And since $F[a]$ is a field, $a^{-1}\in F[a]$ so $F[a^{-1}]\subseteq F[a]$ and thus $[F[a^{-1}]:F]\leq[F[a]:F]<\infty$, so $a^{-1}$ is algebraic over $F$.
So ${\rm Alg}_F(K)$ is indeed a field.

Now suppose $a\in K\setminus{\rm Alg}_F(K)$ is algebraic over ${\rm Alg}_F(K)$.
Then by the above lemma, it is algebraic over $F$ since ${\rm Alg}_F(K)/F$ is trivially algebraic.
But then $a\in{\rm Alg}_F(K)$ by definition, in contradiction.
\qed

