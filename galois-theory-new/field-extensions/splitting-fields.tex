\bprop

    Let $F$ be a field and $f\in F[x]$ be an irreducible polynomial.
    Then there exists a field extension $K/F$ such that $f$ has a root in $K$ and $[K:F]=\deg f$.

\eprop

\Proof since $f$ is irreducible, $(f)$ is maximal (since $F[x]$ is a PID so prime ideals are maximal).
Thus $K=\slfrac{F[x]}{(f)}$ is a field.
The dimension of $K$ is $\deg f$ since it has a basis $\set{1,x,\dots,x^{\deg f-1}}$.

By the second isomorphism theorem,
$$ \slfrac F{F\cap(f)} \cong \slfrac{F+(f)}{(f)} \subseteq \slfrac{F[x]}{(f)} = K $$
But $F\cap(f)=0$, and so $\slfrac F{F\cap(f)}=\slfrac F{0}\cong F$.
Thus we can embed $F$ into $K$, so we can view $K/F$ as a field extension.

Now, define $\alpha=x+(f)$, and suppose $f(x)=\sum_{i=0}^na_ix^i$ for $a_i\in F$.
Then
$$ f(\alpha) = \sum_{i=0}^na_i\bigl(x+(f)\bigr)^i = \sum_{i=0}^na_i(x^i+(f)) = \sum_{i=0}^na_ix^i + \sum_{i=0}^na_i(f) = f + (f) = 0 $$
Thus $\alpha$ is a root of $f$ in $K$.
\qed

\bcoro[name=polyhasroot]

    Let $F$ be a field and $f\in F[x]$ a polynomial.
    Then there exists a field extension $K/F$ such that $f$ has a root in $K$ and $[K:F]\leq\deg f$.

\ecoro

\Proof take an irreducible factorization of $f$ and apply the above result to one of its factors.
\qed

\bdefn

    Suppose $F$ is a field and $f\in F[x]$.
    Then $f$ {\emphcolor splits} in $F$ if there exist $\alpha_1,\dots,\alpha_n\in F$ such that $f(x)=(x-\alpha_1)\cdots(x-\alpha_n)$.

\edefn

\bprop

    Let $f\in F[x]$ then there exists a field extension $K/F$ such that $f$ splits in $K$ and $[K:F]\leq(\deg f)!$.

\eprop

\Proof by induction on $n=\deg f$.
For $n=1$, $f$ is linear and thus has a root so we can take $K=F$.
Now suppose $\deg f=n+1$, then by \refmath[corollary]{polyhasroot} there exists a field extension $K_0/F$ such that $f$ has a root in $K_0$ and $[K_0:F]\leq n+1$.
Now suppose $\alpha\in K_0$ is a root of $f$, then there exists a $g(x)\in K_0[x]$ such that $(x-\alpha)g(x)=f(x)$ and so $\deg g\leq n$.
Therefore inductively there is a field extension $K/K_0$ which splits $g(x)$ and thus $f(x)$ and
$$ [K:F] = [K:K_0]\cdot[K_0:F] \leq n!\cdot(n+1) = (n+1)! $$
as required.
\qed

\bdefn

    Suppose $K/F$ is a field extension and $\phi\colon F\longembeds E$ is an embedding into some other field $E$.
    Then an {\emphcolor extension} of $\phi$ to $K$ is an embedding $\overline\phi\colon K\longembeds E$ such that $\overline\phi\bigl|_F=\phi$ ($\overline\phi$ is equal to $\phi$ on $F$).
    Let us then define
    $$ \eta^\phi_{K/F} \coloneqq \#\set{\overline\phi}[\hbox{$\overline\phi$ is an extension of $\phi$}] $$

\edefn

In other words, an extension is an embedding $\overline\phi$ such that the following diagram commutes:

\medskip
\centerline{\drawdiagram{
    &$E$\cr
    $F$&&$K$\cr
}{
    \diagarrow{from={2,1}, to={2,3}, text=$\iota$, y distance=.25cm}
    \diagarrow{from={2,1}, to={1,2}, text=$\phi$, x distance=-.35cm, slide=.6}
    \diagarrow{from={2,3}, to={1,2}, text=$\overline\phi$, x distance=.35cm, slide=.6}
}}
\medskip

Where $\iota\colon F\longto K$ is the inclusion embedding.

Suppose $f,g$ are two field homomorphisms $F(a_1,\dots,a_n)\longto K$ such that $f(x)=g(x)$ for all $x\in F$ and $f(a_i)=g(a_i)$ for $1\leq i\leq n$.
Then $f(x)=g(x)$ on all of $F(a_1,\dots,a_n)$.
This is since $\set{r\in F(a_1,\dots,a_n)}[f(r)=g(r)]$ is a field containing $F$ and $a_1,\dots,a_n$ and thus $F(a_1,\dots,a_n)$.

In particular if $\phi\colon F\longembeds E$ is an embedding, then an extension $\overline\phi\colon F(a_1,\dots,a_n)\longto E$ is defined entirely by its image on $a_1,\dots,a_n$.

\bprop

    Suppose $K=F[\alpha]$, then $\eta_{K/F}^\phi$ is equal to the number of distinct roots the minimal polynomial of $\alpha$ has in $E$.
    Formally, if $h(x)=\sum_{i=0}^na_ix^i$ then define $\hat h(x)=\sum_{i=0}^n\phi(a_i)x^i$, and $\eta_{K/F}^\phi$ is equal to the number of roots $\hat h(x)$ has in $E$.

    In particular $\eta_{K/F}^\phi$ is independent of the choice of $\phi$.

\eprop

\Proof let $h(x)\in F[x]$ be the minimal polynomial of $\alpha$, and $\overline\phi$ be an extension of $\phi$ to $K$, then
$$ \hat h(\overline\phi(\alpha)) = \sum_{i=0}^n\phi(a_i)\overline\phi(\alpha)^i = \sum_{i=0}^n\overline\phi(a_i)\overline\phi(\alpha^i) = \overline\phi\parens{\sum_{i=0}^na_i\alpha^i} =
\overline\phi(h(\alpha)) = \overline\phi(0) = 0 $$
Thus $\overline\phi(\alpha)$ must be a root of $\hat h(x)$, and as explained above extensions of embeddings to $K=F[\alpha]$ are dependent only on their image of $\alpha$. 
So there are at most as many extensions as there are distinct roots of $\hat h$.

Now suppose $\beta\in E$ is a root of $\hat h$, then we claim that there exists an extension with $\overline\phi(\alpha)=\beta$.
Indeed, $\alpha\notin F$ and $\beta$ is not in the image of $\phi$ (as then $0=\hat h(\phi(a))=\phi(\hat h(a))$ so $a$ is a root of $\hat h(x)$ but $\hat h$ is irreducible), so this is well-defined.
\qed

\bdefn

    A polynomial $f$ which splits over $E$ is called {\emphcolor separable} over $E$ if its linear factors are all distinct (i.e. it has $n=\deg f$ distinct roots in $E$).

\edefn

When we have an embedding $\phi\colon F\longembeds E$ and a polynomial $f\in F[x]$ and we say that $f$ has some property in $E$ (e.g. splits over $E$, separable over $E$), then we mean that its image under
$\phi$ has that property.
Meaning if $f(x)=\sum_{i=0}^na_ix^i$ then $\sum_{i=0}^n\phi(a_i)x^i$ has said property.

\bthrm[name=etacount]

    Let $K/F$ be a finite extension, and let $\phi\colon F\longembeds E$ be an embedding.
    Then
    \benum
        \item $\eta^E_{K/F}\leq[K:F]$;
        \item if $K=F[\alpha_1,\dots,\alpha_n]$ where $\alpha_i$ are roots of some $f\in F[x]$ which splits over $E$, then $1\leq\eta^\phi_{K/F}$.
        Meaning there exists at least one extension of $\phi$ to $K$;
        \item if $f$ is also separable over $E$, then $\eta^\phi_{K/F}=[K:F]$.
    \eenum

\ethrm

\Proof since $K/F$ is finite, we have that $K=F[\alpha_1,\dots,\alpha_n]$ (we can take $\set{\alpha_1,\dots,\alpha_n}$ to be a basis for $K$ as a $F$-linear space).
\bgroup\def\enumindent{0pt}
\benum
    \item We proceed inductively on $n$.
    For $n=1$, by the previous proposition $\eta_{K/F}^\phi$ is equal to the number of roots $h_{\alpha_1}$ (the minimal polynomial of $\alpha_1$) has in $E$.
    
    For the inductive step, define $F_1=F[\alpha_1]$, and so
    $$ \eqalign{
        \eta^\phi_{K/F} &= \#\set{\phi''\colon K\longto E\hbox{ is an extension of $\phi$}}\cr
            &= \#\bigcup\set{\phi''\colon K\longto E\hbox{ is an extension of $\phi'$}}[\phi'\colon F_1\longto E\hbox{ is an extension of $\phi$}]\cr
            &= \sum_{\phi'}\eta_{K/F_1}^{\phi'}
    } $$
    By our inductive hypothesis, $\eta_{K/F_1}^{\phi'}\leq[K:F_1]$ and $\eta_{F_1/F}^\phi\leq[F_1:F]$ so
    $$ \leq \sum_{\phi'}[K:F_1] = [F_1:F]\cdot[K:F] = [K:F] $$
    as required.

    \item Again, we proceed inductively on $n$.
    For $n=1$, $K=F[\alpha]$ and $\eta_{K/F}^\phi$ is equal to the number of roots $h_\alpha$ has in $E$.
    But since $f(\alpha)=0$ and $h_\alpha$ is minimal, $h_\alpha$ must divide $f$ and therefore split in $E$, meaning it has at least one root in $E$.
    So $1\leq\eta^\phi_{K/F}$ as required.

    Inductively, set $F_1=F[\alpha_1]$ and so there exists an extension of $\phi$ to $\phi'\colon F_1\longembeds E$ by our base case.
    And there then exists an extension of $\phi'$ to $\phi''\colon K\longembeds E$, so there exists at least one extension as required.

    \item If we review the proof of $(2)$, for the base case we must have that $f$ is separable and splits in $E$, which means that $h_\alpha$ does as well.
    Then $h_\alpha$ has precisely $\deg h_\alpha$ distinct roots in $E$, so $\eta^\phi_{K/F}=\deg h_\alpha=[K:F]$ as required.
    The rest of the proof proceeds similarly.
    \qed
\eenum
\egroup

\bdefn

    Let $f\in F[x]$ be any polynomial over $F$.
    Then a field $F\subseteq K$ is called a {\emphcolor splitting field} if $f$ splits over $K$ and it contains no other field over which $f$ splits (meaning it is the smallest field which splits $f$).

\edefn

Notice that if $K$ is a splitting field of $f\in F[x]$, then $K$ is of the form $K=F[\alpha_1,\dots,\alpha_n]$ where $\alpha_i$ are roots of $f$ in $K$.
Then
$$ [K:F] \leq \prod_{i=1}^n[F[\alpha_i]:F] < \infty $$
so $K/F$ is a finite extension.
And such a finite field exists: we know there exists a field extension $F_1$ such that $f$ has a root $\alpha_1$ in $F_1$, so there must be an extension $F_2/F_1$ such that $f/(x-\alpha)$ has a root
$\alpha_2$ in $F_2$, and we continue inductively.
This gives us a field $F_n$ with roots $\alpha_1,\dots,\alpha_n$ and so defining $K=F[\alpha_1,\dots,\alpha_n]$ gives us a splitting field.

\bthrm

    Any two splitting fields of a polynomial $f\in F[x]$ are isomorphic.

\ethrm

\Proof let $K$ be a splitting field of $f$, and suppose $f$ splits in $E$, where $F\subseteq E$.
By the above theorem, there must exist an extension of the inclusion embedding $F\longembeds E$ to an embedding $K\longembeds E$.
This embedding gives rise to an embedding of $F$-linear spaces, meaning $[K:F]\leq[E:F]$.
In particular, if $E$ is another splitting field of $f$ then $[E:F]\leq[K:F]$ as well, so that $K$ and $E$ are isomorphic $F$-linear spaces, and thus are isomorphic as fields.
\qed

\blemm

    Let $K/F$ be a field extension and $f\in F[x]$ irreducible where $a,b\in K$ are two roots of $f$.
    Then there exists an isomorphism
    $$ \phi\colon F[a]\longto F[b] $$
    such that $\phi(a)=b$ and $\phi(\alpha)=\alpha$ for all $\alpha\in F$.

\elemm

\Proof let us look at the inclusion map $\iota\colon F\longembeds F[b]$ which can then be extended to $\iota\colon F[x]\longembeds F[b]$ by $\iota(x)=b$.
This is a surjective map whose kernel is $(f)$ since it is the minimal polynomial of $f$, and so there exists an isomorphism $\psi\colon F[x]/(f)\longto F[b]$.
Similarly there exists an isomorphism $\sigma\colon F[x]/(f)\longto F[a]$ and so $\psi\sigma^{-1}$ is an isomorphism which keeps $F$ constant and maps $a$ to $b$.
\qed

\bprop[name=rootmap]

    Let $f\in F[x]$ be an irreducible polynomial with two roots $a,b$ and its splitting field $E$.
    Then there exists an automorphism $\phi\colon E\longto E$ such that $\phi(a)=\phi(b)$.

\eprop

\Proof the isomorphism $\phi\colon F[a]\longto F[b]$ from the above lemma can be extended to an automorphism of $E$.
This is because $\eta_{E/F[a]}^E\geq1$ by \refmath[theorem]{etacount}.
\qed

\bdefn

    Let $f(x)=\sum_{k=0}^na_kx^k\in F[x]$ be a polynomial.
    We define its {\emphcolor formal derivative} to be the polynomial
    $$ f'(x) = \sum_{k=1}^nka_kx^{k-1} $$

\edefn

It is not hard to prove that $(f+g)'=f'+g'$ and $(f\cdot g)'=f'g+fg'$.

\blemm

    Let $f,g\in F[x]$ and define $r(x)=\gcd(f,g)$.
    Then $r(x)$ is the gcd of $f$ and $g$ over {\it every} field extension $K/F$.

\elemm

\Proof let $r_K(x)$ be the gcd of $f,g$ over $K$.
Since $r(x)$ still divides $f,g$ we have that $r(x)\divides r_K(x)$.
And by Euclid's algorithm there exist $a(x),b(x)\in F[x]$ such that
$$ r(x) = a(x)f(x) + b(x)g(x) $$
But $r_K(x)$ divides $f,g$ so it divides $r(x)$.
Thus $r_K(x)=r(x)$ as required.
\qed

\bthrm[name=derivsep]

    Let $f\in F[x]$ be a polynomial, then $f$ is separable if and only if $\gcd(f,f')=1$.

\ethrm

\Proof let $K$ be a splitting field of $f$.
Suppose $f$ is not separable, then it has the form $f(x)=(x-\alpha)^mg(x)$ for $g(x)\in K[x]$ and $m>1$.
But then $f'(x)=m(x-\alpha)^{m-1}g(x)+(x-\alpha)^mg'(x)$ and so $x-\alpha$ is a common factor of both $f$ and $f'$ so $\gcd(f,f')\neq1$ in $K[x]$, but the gcd of $f,f'$ in $F$ is equal to its gcd in $K$ by
the above lemma.

Alternatively if $f$ is separable, then $f(x)=\prod_{i=1}^n(x-\alpha_i)$ and so
$$ f'(x) = \sum_{j=1}^n\prod_{\stackmath{1\leq i\leq n\cr i\neq j}}(x-\alpha_i) $$
But the irreducible factors of $f$, which are $x-\alpha_i$, do not divide $f'(x)$ since no two roots are equal.
Thus $\gcd(f,f')=1$.
\qed

Recall that for any ring $R$, there is a unique homomorphism $\phi\colon{\bb Z}\longto R$.
In particular if $F$ is a field then $\slfrac{\bb Z}{\ker\phi}\cong{\rm Im}\phi\subseteq F$.
Since $F$ is a field, ${\rm Im}\phi$ is an integral domain and so $\ker\phi$ is a prime ideal of ${\bb Z}$, meaning $\ker\phi=(p)$ for some prime $p$ or $0$.
This is called the {\it characteristic} of $F$.

Since $\phi(n)=1+\cdots+1$, the characteristic of $F$ is simply the prime $p$ such that $\phi(p)=0$, i.e. $1+\cdots+1=0$ ($p$ times), or $0$ if no such primes exist.

\bdefn

    The {\emphcolor characteristic} of a field $F$ is the unique positive generator of the kernel of $\phi\colon{\bb Z}\longto F$.
    Equivalently it is the minimum number $p$ such that $1+\cdots+1=0$ ($p$ times), or $0$ if no such $p$ exists.

\edefn

If $F$ has characteristic $0$, then $\phi$ is an embedding so we can view ${\bb Z}$ as a subfield of $F$.
But then the field generated by ${\bb Z}$ must also be a subfield of (embeddable into) $F$, meaning ${\bb Q}\subseteq F$.
Similarly for fields of characteristic $p>0$, $\slfrac{\bb Z}{p{\bb Z}}={\bb F}_p\subseteq F$.

Notice that for fields of characteristic $p$, $\binom pk=\frac{p!}{k!(p-k)!}$ is zero for $k\neq0,p$.
Thus:
$$ (a+b)^p = \sum_{k=0}^p\binom pka^kb^{n-k} = a^p + b^p $$
So $x\mapsto x^p$ is a homomorphism, called the {\it Frobenius homomorphism}.
It can be viewed as a homomorphism to $F^p=\set{x^p}[x\in F]$ (which is a field precisely because the Frobenius homomorphism is a homomorphism).
The homomorphism has a trivial kernel, so $F\cong F^p$.
In particular every element of $F$ is of the form $x^p$.

\bthrm

    Let $f\in F[x]$ be an irreducible polynomial, then the following are equivalent:
    \benum
        \item $f$ is not separable (has a multiple root),
        \item $F$ has a characteristic $p>0$, and $f(x)=g(x^p)$ for some $g\in F[x]$,
        \item every root of $f$ is a multiple root.
    \eenum

\ethrm

\Proof $(1)\implies(2)$: by \refmath[theorem]{derivsep} we have that $\gcd(f,f')\neq1$.
But $f$ is irreducible and thus has no nontrivial divisors, so $f'=0$.
But since $f$ is nonconstant, we must have that $F$ is of characteristic $p$ (since in characteristic $0$ a nonconstant polynomial cannot have a zero derivative).

Now, if $f(x)=\sum_{k=0}^na_kx^k$ then $ka_k=0$ for all $k$ since $f'(x)=0$.
So for $k$ not divisible by $p$, this means that $k\neq0$ and so $a_k=0$.
Thus
$$ f(x) = \sum_{p\divides k}a_kx^k = \sum_ja_{pj}x^{pj} $$
so define $g(x)=\sum_ja_{pj}x^j$ and we have the desired result.

$(2)\implies(3)$: take a splitting field of $g(x)$, then write $g(x) = a\prod_i(x-a_i)^{m_i}$.
Then we have that $f(x)=a\prod_i(x^p-a_i)^{m_i}$.
We can extend this to a field with $p$-roots of $a_i$ (which are roots of $x^p-a_i$), $\alpha_i$, and so over this field $f(x)=a\prod_i(x-\alpha_i)^{pm_i}$.
So all the roots of $f$ have a multiplicity greater than $1$.

$(3)\implies(1)$ is trivial.
\qed

