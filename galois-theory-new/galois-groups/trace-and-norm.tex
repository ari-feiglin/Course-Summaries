\bdefn

    Let $K/F$ be finite Galois, and $G=\Galof{K/F}$.
    Then we define the {\emphcolor trace} to be
    $$ T\colon K\longto F,\qquad T(a) = \sum_{\sigma\in G}\sigma(a) $$
    and {\emphcolor norm} to be to be
    $$ N\colon K^\times\longto F^\times,\qquad N(a) = \prod_{\sigma\in G}\sigma(a) $$

\edefn

Notice that the trace and norm are well defined.
Suppose $a\in K$, take $\tau\in G$ then
$$ \tau T(a) = \sum_{\sigma\in G}\tau\sigma(a) = \sum_{\sigma\in G}\sigma(a) = T(a) $$
so $T(a)\in K^G=F$ as required.
Similar for the norm.

Notice that the trace is also a surjective group homomorphism (relative to $+$): $a/\abs G$ is an element of the fiber of $a$.
And the norm is a group homomorphism between $K^\times$ and $F^\times$.

\bexam

    We know that ${\bb C}/{\bb R}$ is finite Galois as it is the splitting field of $x^2+1$ which is separable.
    The Galois group is $G=\set{1,z\mapsto\overline z}$ and so the norm and trace are
    $$ T(x+iy) = (x+iy) + (x-iy) = 2x,\qquad N(x+iy) = (x+iy)(x-iy) = \abs{x+iy}^2 $$
    So $\ker T=i{\bb R}$ and $\ker N=S^1$.

\eexam

\bdefn

    A Galois extension $K/F$ is {\emphcolor cyclic} if $\Galof{K/F}$ is cyclic.

\edefn

If $K/F$ is cyclic and its Galois group is generated by $\sigma$, then define $D\colon K\longto K$ by $D(a)=a-\sigma(a)$.
Its kernel is $\ker D=K^G=F$ and
$$ T\circ D(a) = \sum_{i=0}^n\sigma^i(a) - \sum_{i=0}^n\sigma^{i+1}(a) = 0 $$
so ${\rm Im}D\subseteq\ker T$.
Now, $D$ is an $F$-linear transformation, so by the rank-nullity theorem
$$ \dim\ker D + \dim{\rm Im}D = [K:F] $$
meaning $\dim{\rm Im}D=[K:F]-1$.
Since $\ker T$ is not all of $K$, we must have that ${\rm Im}D=\ker T$.

Recall that a Galois, and thus cyclic, extension is generated by a single element.

\blemm[title=Dedekind's Lemma, name=dedekindlemm]

    Let $F,K$ be fields and $\phi_1,\dots,\phi_n\colon F\longto K$ be distinct field homomorphisms.
    Then $\phi_1,\dots,\phi_n$ are $F$-linearly independent.

\elemm

\Proof suppose that $\phi_2,\dots,\phi_n$ is linearly independent (i.e. we induct on $n$).
Now suppose
$$ \sum_{i=1}^na_i\phi_i = 0 $$
for $a_i\in F$.
Then since $\phi_1\neq\phi_2$ there exists some $\alpha\in F$ such that $\phi_1(\alpha)\neq\phi_2(\alpha)$.
So for all $x\in F$,
$$ \sum_{i=1}^na_i\phi_1(\alpha)\phi_i(x) = 0 $$
and on the other hand,
$$ 0 = \sum_{i=1}^na_i\phi_i(\alpha x) = \sum_{i=1}^na_i\phi_i(\alpha)\phi_i(x) $$
So subtracting the two gives that
$$ \sum_{i=2}^na_i(\phi_1(\alpha)-\phi_i(\alpha))\phi_i = 0 $$
Since $\phi_2,\dots,\phi_n$ is linearly independent, we have that $a_i=0$ or $\phi_1(\alpha)=\phi_i(\alpha)$.
Thus we have that $a_2=0$.
Continuing with multiplying by $\phi_1(\alpha)\neq\phi_i(\alpha)$ we see that $a_i=0$ for all $i>1$.
And so $a_i=0$ for all $i$.
\qed

Notice that this means there can only be at most $[E:F]$ distinct $F$-automorphisms.
This is since every $F$-automorphism is an element of ${\rm Hom}(E,F)$ which has dimension $[E:F]$.
This of course also follows from \refmath[theorem]{etacount}.

\bthrm[title=Hilbert's Theorem 90, name=hilbert90]

    Suppose $K/F$ is cyclic whose Galois group is generated by $\sigma$, then
    $$ \ker N = \set{\frac a{\sigma(a)}}[a\in K] $$

\ethrm

\Proof for the first inclusion $\supseteq$,
$$ N\parens{\frac a{\sigma(a)}} = \frac{N(a)}{N(\sigma(a))} = \frac{N(a)}{\sigma(N(a))} = \frac{N(a)}{N(a)} = 1 $$
where $N(\sigma(a))=\sigma(N(a))$ since $\sigma$ is a homomorphism and $\sigma(N(a))=N(a)$ since $\sigma$ holds $F$ constant.

For the other direction, $\subseteq$, suppose $N(\alpha)=1$.
Define $\tau_\alpha\colon K\longto K$ by $\tau_\alpha(e)=\alpha\sigma(e)$.
As is easily verified, $\tau_\alpha$ is an $F$-linear map.
Furthermore, $\tau_\alpha^2(e)=\alpha\sigma(\alpha)\sigma^2(e)$, and inductively $\tau_\alpha^k(e)=\alpha\sigma(\alpha)\cdots\sigma^{k-1}(\alpha)\sigma^k(e)$.
In particular, if $[K:F]=n$ then
$$ \tau_\alpha^n = (\alpha\sigma(\alpha)\cdots\sigma^{n-1}(\alpha))\sigma^n = N(\alpha){\rm id} = {\rm id} $$
And so $\tau_\alpha$'s minimal polynomial divides $x^n-1$.
On the other hand, $1,\sigma,\dots,\sigma^{n-1}$ are $E$-linearly independent by \refmath{dedekindlemm}.
Therefore ${\rm i},\tau_\alpha,\dots,\tau_\alpha^{n-1}$ are $F$-linearly independent and so the minimal polynomial of $\tau_\alpha$ must be $x^n-1$.

In particular, $1$ is an eigenvalue of $\tau_\alpha$.
So there exists a $\beta$ such that $\tau_\alpha(\beta)=\beta$, and so $\alpha\sigma(\beta)=\beta$ meaning $\alpha=\frac\beta{\sigma(\beta)}$ as required.
\qed

Notice then that we have an exact sequence

\bigskip
\centerline{\drawdiagram{
    $1$&$F^\times$&$K^\times$&$K^\times$&$F^\times$\cr
}{
    \diagarrow{from={1,1}, to={1,2}}
    \diagarrow{from={1,2}, to={1,3}}
    \diagarrow{from={1,3}, to={1,4}, text=$\frac a{\sigma(a)}$, y distance=.25cm}
    \diagarrow{from={1,4}, to={1,5}, text=$N$, y distance=.25cm}
}}
\medskip

