\bprop

    Every finite separable extension $K/F$ is contained in some finite Galois extension.

\eprop

\Proof suppose $K=F[\alpha_1,\dots,\alpha_n]$, and let $h_i$ be the minimal polynomial of $\alpha_i$.
Since $K/F$ is separable, $h_i$ only has simple roots (roots of multiplicity $1$) in $K$.
Let $f(x)=\prod_ih_i(x)$ where repetitions are removed, so that $f(x)$ is still separable.
Let $E$ be $f$'s splitting field, so it is the splitting field of a separable polynomial, so by \refmath[theorem]{galoisextension}, $E/F$ is Galois.
\qed

\bprop

    Let $K/L/F$ be finite extensions such that $K/F$ is Galois.
    Let $G=\Galof{K/F}$ and $H=\Galof{K/L}$.
    Define $N={\rm core}_G(H)=\bigcap_{\sigma\in G}\sigma H\sigma^{-1}$.
    Then $K^N/F$ is Galois, and moreso it is the smallest Galois extension in $K/F$ that contains $L$.

\eprop

\Proof we know that the core of a subgroup is normal, meaning $N\normaleq G$ and so by \refmath[corollary]{galoisproperties} $E^N/F$ is Galois.
Since $N\leq H$, $E^N\supseteq E^H=M$ by the fundamental theorem.
Furthermore, if $M=K^{N_0}\supseteq L$ such that $M/F$ is Galois, then by \refmath[corollary]{galoisproperties} again, $N_0$ is normal in $G$.
And by the correspondence, $N_0\leq H$.
So $N_0$ is a normal subgroup of $G$ contained in $H$, but $N$ is the core which is the largest such normal group, so $N_0\leq N$.
And so $K^N\subseteq E^{N_0}=M$.
So $K^N$ is minimal.
\qed

\bdefn

    Given finite extensions $K/L/F$ such that $K/F$ is Galois, and for every $L\subseteq M\subset K$, $M/F$ is not Galois, then $K$ is called the {\emphcolor Galois closure} of $L/F$.

\edefn

\bprop

    The Galois closure of a separable extension $L/F$ is unique up to isomorphism.

\eprop

\Proof suppose $L=F[\alpha_1,\dots,\alpha_n]$ and let $h_i$ be the minimal polynomial of $\alpha_i$ which is separable.
Then define $f=\prod_ih_i$ without repetitions, and this is still separable.
We claim that $E^N$ (where $N$ is defined in the above proposition) is the splitting field of $f$.
Since $E^N/F$ is Galois, $f$ splits into distinct linear factors over $E^N$.
Let $K$ be the splitting field of $f$, so $K\subseteq E^N$ and since $K$ is the splitting field of a separable polynomial, $K/F$ is Galois.
But $E^N$ is minimal so $E^N\subseteq K$, meaning $E^N=K$.
\qed

\bprop

    Let $K/F$ be separable, then there exist only finitely many intermediate fields.

\eprop

\Proof let $E$ be the Galois closure of $K/F$.
Then $E/F$ is Galois and thus has finitely many intermediate fields, and therefore so does $K/F$ (every intermediate field of $K/F$ is an intermediate field of $E/F$).
\qed

\bthrm[title=Steinitz's Theorem, name=steinitz]

    Every finite dimension separable field extension $K/F$ is generated by a single element.

\ethrm

\Proof we assume for the sake of this proof that the fields are infinite, and we induct on the number of generators of $K$.
It is sufficient to prove this for the case of two generators, $K=F[x,y]$, as we can then go from $F[x_1,\dots,x_n]=F[x_1,\dots,x_{n-2}][x_{n-1},x_n]$ to $F[x_1,\dots,x_{n-1}]$ and continue inductively.

Let us focus on elements of the form $x+\alpha y$ for $\alpha\in F$.
And so we have infinitely many intermediate fields $F[x+\alpha y]$ (counting repetitions).
By the above proposition, there are finitely many intermediate fields of $K/F$, and so there must be $\alpha\neq\beta\in F$ such that $L=F[x+\alpha y]=F[x+\beta y]$.
But then
$$ (x+\alpha y) - (x+\beta y) = (\alpha-\beta)y \in L \implies y\in L $$
and similarly we can show that $x\in L$.
Thus we have that $L=F[x,y]=K$, meaning we can generate $K$ using a single element.
\qed

\bdefn

    Suppose $F,L$ are fields contained in some larger field $K$.
    The {\emphcolor compositum} of $F$ and $L$ is defined to be the smallest field containing both $L$ and $F$.
    This can be shown to be
    $$ FL = \set{\sum_{i=1}^n\alpha_i\beta_i}[\alpha_i\in F,\beta_i\in L] $$
    the compositum is also denoted $F\vee L$.

\edefn

\bprop[name=res]

    If $K/F$ is Galois and $L/F$ is a finite extension, then $KL/F$ is also Galois, and
    $$ \res\colon\Galof{KL/L}\longto\Galof{K/K\cap L},\qquad \sigma\mapsto\sigma\bigl|_K $$
    is a well-defined isomorphism.

\eprop

\Proof since $K/F$ is Galois, $K$ is the splitting field of some separable polynomial $f\in F[x]$.
This means that $KL$ is the splitting field of $f\in L[x]$ since it is the smallest field containing both $L$ and the roots of $f$, which is by definition the splitting field of $f$ over $L$.
Since $f$ is separable, this means $KL/L$ is Galois.

Now, $\res$ is well-defined since if $\sigma\in\Galof{KL/L}$ then $\sigma$ permutes the roots of $f$, which generates $K$, and so $\sigma(K)=K$.
And since it also fixes $L$, we must have that it fixes $K\cap L$.
So $\sigma\bigl|_K$ is a $K\cap L$-automorphism.
$\res$ is clearly a homomorphism.

Now we prove that $\res$ is injective: if $\sigma\bigl|_K=1$, then $\sigma$ is the identity on $K$ and $L$ (since it is a $L$-automorphism), so it is the identity on $KL$.
Thus the kernel of $\res$ is trivial, meaning it is injective.

Finally, we prove that $\res$ is surjective.
If $\alpha\in K^{{\rm Im}\res}$ then $\sigma(\alpha)=\alpha$ for every $\sigma\in\Galof{KL/L}$, then since $KL^{\Galof{KL/L}}=L$, we have that $\alpha\in L$.
So $\alpha\in K\cap L$, meaning $K^{{\rm Im}\res}\subseteq K\cap L$.
Conversely, ${\rm Im}\res\subseteq\Galof{K/K\cap L}$ so $K^{{\rm Im}\res}\subseteq K\cap L$.
Thus we have the equality, $K\cap L=K^{{\rm Im}\res}$.
But then by taking $\Galof{K/\bullet}$, we have that $\Galof{K/K\cap L}={\rm Im}\res$ as required.
\qed

Notice then that we get, by the Galois correspondence,
$$ [K:F] = [K:K\cap L][K\cap L:F],\qquad [KL:F] = [KL:L][L:F] = [K:K\cap L][L:F] $$
So $[K:K\cap L]=\frac{[K:F]}{[K\cap L:F]}$ and thus
$$ [KL:F] = \frac{[K:F][L:F]}{[K:K\cap L]} $$
when $K/F$ is Galois and $L/F$ is finite.

