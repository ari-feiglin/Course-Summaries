\input pdfToolbox

\setlayout{horizontal margin=2cm, vertical margin=2cm}
\parindent=0pt
\parskip=3pt plus 2pt minus 2pt

\input preamble

\footline={}

%%%%%%%%%%%%%%%%%%%%%%%%%%%%%%%%%%%%%%%%%%%%%%%%%%%%%%%%%%%%%%%%

\headline={\pageborder{rgb{.5 1 .5}}{rgb{0 .4 0}}{5}}

\color rgb{.1 .8 0}

{\def\boxshadowcolor{rgb{.3 .8 .3}}
\bppbox{rgb{.5 1 .5}}{rgb{0 .4 0}}{rgb{.1 .4 0}}

    \centerline{\setfontandscale{bf}{20pt}Modern Analysis }
    \smallskip
    \centerline{\setfont{it}Lectures by Shimon Brooks \setfont{rm}({\tt brookss@math.biu.ac.il})}
    \centerline{\setfont{it}Summary by Ari Feiglin \setfont{rm}({\tt ari.feiglin@gmail.com})}

\eppbox

\bigskip

\bppbox{rgb{.5 1 .5}}{rgb{0 .4 0}}{rgb{.1 .4 0}}
    \section*{Contents}
    
    \tableofcontents
\eppbox

}

\vfill\break

\color{black}

\pageno=1
\newif\ifpageodd
\pageoddtrue
\headline={%
    \hbox to \hsize{\color{black}%
        \ifpageodd\hfil{\it\currsubsection\quad\bf\folio}\global\pageoddfalse%
        \else{\bf\folio\quad\it\currsubsection}\hfil\global\pageoddtrue\fi%
    }%
}

\section{Introduction}

\subsection*{The Riemann Integral and its Faults}

Recall the definition of the Riemann integral: given a function $f(x)$ on an interval $[a,b]$ we take a partition of this interval and representatives from each partition, $x_i$.
Then the Riemann sum of this function over this partition is $\sum f(x_i)\Delta x_i$.
Then if this sum converges to a value as $\sup\Delta x_i$ converges to $0$, this function is {\it Riemann integrable} and has an integral represented by $\int_a^b f(x)\,dx$.
Previously we have shown that a function is Riemann integrable if and only if it is almost always continuous, in particular all continuous functions are continuous.

But take for example ${\bb Q}$'s indicator function
$$ \chi_{{\bb Q}}(x) \coloneqq \cases{1 & $x\in{\bb Q}$\cr 0 & $x\notin{\bb Q}$} . $$
This function is nowhere continuous and thus is not Riemann integrable.
But we argue that it {\it should} be integrable.

Any theory if an integral should have the following two basic properties:
\benum
    \item {\it monotonicity\/}: if $f(x)\leq g(x)$ for every $x$ in the domain then $\int f(x)\,dx\leq\int g(x)\,dx$.
    \item {\it linearity\/}: $\int(\alpha f(x)+\beta g(x))\,dx=\alpha\int f(x)\,dx+\beta\int g(x)\,dx$.
\eenum

Let $\set{q_n}_{n=1}^\infty$ be an enumeration of the rationals and $\epsilon>0$.
Let us define $E_n\coloneqq\parens{q_n-\frac\epsilon{2^{n+1}},q_n+\frac\epsilon{2^{n+1}}}$, then we should think that
$$ \int\sum_{n=1}^\infty\chi_{E_n}(x)\,dx = \sum_{n=1}^\infty\int\chi_{E_n}(x)\,dx = \sum_{n=1}^\infty\frac\epsilon{2^n} = \epsilon \eqnum $$
Now $\chi_{\bb Q}(x)\leq\sum_{n=1}^\infty\chi_{E_n}(x)$ for all $x$ and so $\int\chi_{\bb Q}(x)\,dx\leq\epsilon$ for every $\epsilon>0$ and so we should think that $\int\chi_{\bb Q}(x)\,dx=0$.

Now obviously $\chi_{\bb Q}$ is not Riemann-integrable, and so there is an issue with the above argument.
In fact there are two: firstly in $(1)$ we assumed that $\int\sum_{n=1}^\infty\chi_{E_n}=\sum_{n=1}^\infty\int\chi_{E_n}$, which only holds if the sum converges uniformly.
Secondly, $\sum_{n=1}^\infty\chi_{E_n}(x)$ takes on infinite values (for every rational number, the sum is infinite) and so it is not even Riemann integrable.

So we want a theory of integration which allows for two things: 1) the ability to deal with convergence of an integral without necessarily needing uniform convergence, 2) the ability to deal with functions
which are not Riemann-integrable.
Lebesgue's theory of integration is based on the following observation: partitioning the domain of the function will necessarily require some form of continuity, so instead try partitioning the {\it range}
of the function.
So given the partition $y_0<\cdots<y_n$ we can imagine some Lebesgue sum of this partition to be
$$ \sum_{i=0}^n y_i\cdot\abs{E_i},\qquad E_i=\set{x\in[a,b]}[y_i\leq f(x)\leq y_{i+1}] $$
where $\abs S$ is some notion of the ``width'' of the set $S$, {\it which we have not yet defined} (it is not the cardinality of the set).
For an interval this should be the length of the interval, but for arbitrary sets it becomes harder to understand how we should approach defining it.
And in order to define the integral of $\chi_{\bb Q}$, it is necessary to define this width for arbitrary sets, or at least for a larger family of sets than just intervals, since computing the Lebesgue
sums of $\chi_{\bb Q}$ will involve terms containing $\abs{{\bb Q}}$.

This ``Lebesgue sum'' will not be precisely how we define Lebesgue integration, but it does give us a starting point: how do we define the ``width'' of a set $E\subseteq{\bb R}$.

\vfill\break

\section{Lebesgue Integration}

\subsection{The Lebesgue Measure}

We would like a function $m$ which measures the width of arbitrary sets $E\subseteq{\bb R}$.
Such a function would ideally satisfy the following properties:
{\def\enumstyle#1{$\m@th\bf(m#1)$}
\benum
    \item $m$ is a function $m\colon\powsetof{\bb R}\longto[0,\infty]$,
    \item for every interval $I$, $m(I)$ is the length of $I$,
    \item the measure of a set is preserved under movement, ie. $m(E+\alpha)=m(E)$ for every $\alpha\in{\bb R}$ where $E+\alpha=\set{x+\alpha}[x\in E]$,
    \item if $\set{E_n}_{n=1}^\infty$ is a sequence of disjoint sets then $m\parens{\bigdcup_{n=1}^\infty E_n}=\sum_{n=1}^\infty m(E_n)$.
        This is called {\it $\sigma$-additivity}.
\eenum}

\bdefn[title=The Outer Measure]

    Let $E\subseteq{\bb R}$, then we define $E$'s {\emphcolor outer measure} to be
    $$ m^*(E) \coloneqq \infof{\sum_{n=1}^\infty\abs{I_n}}[E\subseteq\bigcup_{n=1}^\infty I_n\hbox{ where $I_n$ are all open intervals}] $$

\edefn

This set is nonempty since ${\bb R}$ can be covered by open intervals (eg. $\set{(n,n+2)}_{n\in{\bb Q}}$) and therefore so can every subset.
Obviously we have that $m^*(\varnothing)=0$ since we can take arbitrarily small arbitrary intervals.
Notice that for every $\epsilon>0$, we showed in the previous section there exists open intervals $\set{E_n}_{n=1}^\infty$ such that ${\bb Q}\subseteq\bigcup_{n=1}^\infty E_n$ and
$\sum_{n=1}^\infty\abs{E_n}=\epsilon$.
Thus we have that $m^*({\bb Q})\leq\epsilon$ for every $\epsilon>0$ and so $m^*({\bb Q})=0$ as well.

Notice that we can also take a finite set of $I_n$s, as we can add infinitely many $I_n$s of arbitrarily small width (eg. add $I_n$s of length $\frac\epsilon{2^n}$) and this will add $\epsilon$ to the
sum, and so the infimum remains the same.
Thus we have that
$$ m^*(E) = \infof{\sum_{n=1}^\infty\abs{I_n}}[E\subseteq\bigcup_{j\in J}I_j\hbox{ where $I_n$ are all open intervals and $J$ is countable}] $$

Now, does the outer measure satisfy the conditions $\rm m1$ through $\rm m4$?
\benum
    \item $m^*$ is indeed a function from $\powsetof{\bb R}$ to $[0,\infty]$,
    \item $m^*(I)=\abs I$ for all intervals $I$ (proven below).
    \item $m^*(E+\alpha)=m^*(E)$ as there is a width-preserving bijection between collections of intervals covering $E$ and $E+\alpha$ (in particular $I\mapsto I+\alpha$).
\eenum

But $m^*$ is not $\sigma$-additive, and we can only ensure $\sigma$-subadditivity:

\bthrm

    Let $\set{E_n}_{n=1}^\infty$ be a sequence of (not necessarily disjoint) subsets of ${\bb R}$, then
    $$ m^*\parens{\bigcup_{n=1}^\infty E_n} \leq \sum_{n=1}^\infty m^*(E_n) $$

\ethrm

\Proof for every $E_n$, let us take a cover for $E_n$ of open intervals $\set{I_k^{(n)}}_{k=1}^\infty$ such that $m^*(E_n)\geq\sum_{k=1}^\infty\abs{I_k^{(n)}}+\frac\epsilon{2^n}$.
Then we have that
$$ \sum_{n=1}^\infty m^*(E_n) \geq \sum_{n=1}^\infty\parens{\sum_{k=1}^\infty\abs{I_k^{(n)}}+\frac\epsilon{2^n}} = \sum_{n,k=1}^\infty\abs{I_k^{(n)}}+\epsilon $$
Since $\set{I_k^{(n)}}_{n,k=1}^\infty$ is a cover of $\bigcup_{n=1}^\infty E_n$, we have that
$$ \sum_{n=1}^\infty m^*(E_n) \geq \sum_{n,k=1}^\infty\abs{I_k^{(n)}}+\epsilon \geq m^*\parens{\bigcup_{n=1}^\infty E_n}+\epsilon $$
Since $\epsilon>0$ is arbitrary, we get the desired inequality (by taking $\epsilon\to0$, the above inequality is preserved).
\qed

Of course this does not prove that $m^*$ is not $\sigma$-additive, we have simply proven a weaker condition.

This helps us show that $m^*(I)=\abs I$ for all intervals $I$.
For open intervals this is trivial, and we have that $m^*(E\cup\set a)\leq m^*(E)+m^*(\set a)=m^*(E)$ (the measure of a singleton is obviously zero) and $m^*(E\cup\set a)\geq m^*(E)$ by monotonicity.
All intervals are obtained by adding a finite number of points to an open interval, so for example $m^*([a,b])=m^*\bigl((a,b)\cup\set{a,b}\bigr)=m^*((a,b))=b-a$ as required.

\bprop[title=The Vitali Set]

    There exists no function which satisfies properties $\rm m1$ through $\rm m4$.

\eprop

\Proof let us define an equivalence relation on ${\bb R}$ as follows: $x\sim y\iff x-y\in{\bb Q}$.
Notice that the equivalence classes of this relation are of the form $x+{\bb Q}$ for some $x\in{\bb R}$ and thus they are all countable and dense.
For every equivalence class choose a single representative in $[0,1]$ to form the set $E\subseteq[0,1]$.
Notice then that
$$ {\bb R} = \bigdcup_{q\in{\bb Q}}(E+q) = E+{\bb Q} = \set{x+q}[x\in E,q\in{\bb Q}] $$
this union is disjoint since if $x\in E+q\cap E+p$ then $x=x_1+q$ and $x=x_2+p$ so $x\sim x_1\sim x_2$ but we put only a single representative of each equivalence class into $E$.
For every $y\in[0,1]$, there exists an $x\in E$ such that $x\sim y$ so $x-y\in{\bb Q}\cap[-1,1]$, and so if $y\in E+q$ for some $q\in[-1,1]$.
Thus
$$ [0,1] \subseteq \bigdcup_{q\in{\bb Q}\cap[-1,1]}E+q = E + {\bb Q}\cap[-1,1] \subseteq [-1,2] $$
If $m$ is a function which satisfies the four properties above, it is also monotonic (which can be derived from $\sigma$-additivity: if $A\subseteq B$ then $m(B)=m(A)+m(B\setminus A)\geq m(A)$), and so
$$ m([0,1]) = 1 \leq \sum_{q\in{\bb Q}\cap[-1,1]}m(E) \leq 3 $$
Since ${\bb Q}\cap[-1,1]$ is infinite, we must have that $m(E)=0$ as otherwise the sum is infinite.
But then the sum is $0$ and not greater than $1$, which is a contradiction.
So the {\it Vitali set\/} $\bigcup_{q\in{\bb Q}\cap[-1,1]}E+q$ cannot be measurable.
\qed

So we must weaken one of the conditions.
$m^*$ is already an example of a function which satisfies $\rm m1$ through $\rm m3$ so they are not contradictory.
But $\sigma$-additivity is extremely important to Lebesgue's theory of integration, so we will instead weaken $\rm m1$ to be
$$ \hbox{$({\bf m1})$\kern.25cm $m$ is a function ${\cal L}({\bb R})\longto[0,\infty]$} $$
where ${\cal L}({\bb R})$ is the set of all Lebesgue-measurable sets of ${\bb R}$:

\bdefn[title=Carath\'eodory]

    A set $E\subseteq{\bb R}$ is called {\emphcolor Lebesgue measurable} (or just {\it measurable}) if for every $A\subseteq{\bb R}$, $m^*(A)=m^*(A\cap E)+m^*(A\cap E^c)$.

\edefn

Notice that by subadditivity $m*(A)\leq m^*(A\cap E)+m^*(A\cap E^c)$, to show that $E$ is Lebesgue measurable it is sufficient to show the other direction ($\geq$).
This implies that all sets $E$ with zero outer measure are Lebesgue measurable: $m^*(A\cap E)+m^*(A\cap E^c)=m^*(A\cap E^c)\leq m^*(A)$ as required ($m^*(A\cap E)\leq m^*(E)=0$).
Further notice that $E$ is measurable if and only if $E^c$ is by symmetry of the definition.

\bthrm

    Every open interval of the form $(a,\infty)$ is measurable.

\ethrm

\Proof let $A$ be a subset of ${\bb R}$.
Let us define
$$ A^+ \coloneqq A\cap(a,\infty),\qquad A^- \coloneqq A\cap(a,\infty)^c = A\cap(-\infty,a] $$
We must show that $m^*(A^+)+m^*(A^-)\leq m^*(A)$.
Let $\epsilon>0$ and let $\set{I_n}_{n=1}^\infty$ be a cover of $A$ consisting of open intervals where $\sum_{n=1}^\infty\abs{I_n}\leq m^*(A)+\epsilon$.
Then for every $n$, define $I_n^+=I_n\cap(a,\infty)$ and $I_n^-=I_n\cap(a,\infty)^c$, so $\set{I_n^+}$ is a cover for $A^+$ and $\set{I_n^-}$ is a cover for $A^-$.
And since these are intervals we have $\abs{I_n}=\abs{I_n^+}+\abs{I_n^-}$.
So we have
$$ m^*(A) \geq \sum_{n=1}^\infty\abs{I_n}-\epsilon = \sum_{n=1}^\infty\abs{I_n^+} + \sum_{n=1}^\infty\abs{I_n^-} + \epsilon \geq m^*(A^+) + m^*(A^-) + \epsilon $$
And since $\epsilon>0$ is arbitrary, we get $m^*(A)\geq m^*(A^+)+m^*(A^-)$ as required.
\qed

\bye

