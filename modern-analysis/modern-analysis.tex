\input pdfToolbox

\setlayout{horizontal margin=2cm, vertical margin=2cm}
\parindent=0pt
\parskip=3pt plus 2pt minus 2pt

\input preamble

\footline={}

%%%%%%%%%%%%%%%%%%%%%%%%%%%%%%%%%%%%%%%%%%%%%%%%%%%%%%%%%%%%%%%%

\headline={\pageborder{rgb{.5 1 .5}}{rgb{0 .4 0}}{5}}

\color rgb{.1 .8 0}

{\def\boxshadowcolor{rgb{.3 .8 .3}}
\bppbox{rgb{.5 1 .5}}{rgb{0 .4 0}}{rgb{.1 .4 0}}

    \centerline{\setfontandscale{bf}{20pt}Modern Analysis }
    \smallskip
    \centerline{\setfont{it}Lectures by Shimon Brooks \setfont{rm}({\tt brookss@math.biu.ac.il})}
    \centerline{\setfont{it}Summary by Ari Feiglin \setfont{rm}({\tt ari.feiglin@gmail.com})}

\eppbox

\bigskip

\bppbox{rgb{.5 1 .5}}{rgb{0 .4 0}}{rgb{.1 .4 0}}
    \section*{Contents}
    
    \tableofcontents
\eppbox

}

\vfill\break

\color{black}

\pageno=1
\newif\ifpageodd
\pageoddtrue
\headline={%
    \hbox to \hsize{\color{black}%
        \ifpageodd\hfil{\it\currsubsection\quad\bf\folio}\global\pageoddfalse%
        \else{\bf\folio\quad\it\currsubsection}\hfil\global\pageoddtrue\fi%
    }%
}

\section{Introduction}

\subsection*{The Riemann Integral and its Faults}

Recall the definition of the Riemann integral: given a function $f(x)$ on an interval $[a,b]$ we take a partition of this interval and representatives from each partition, $x_i$.
Then the Riemann sum of this function over this partition is $\sum f(x_i)\Delta x_i$.
Then if this sum converges to a value as $\sup\Delta x_i$ converges to $0$, this function is {\it Riemann integrable} and has an integral represented by $\int_a^b f(x)\,dx$.
Previously we have shown that a function is Riemann integrable if and only if it is almost always continuous, in particular all continuous functions are continuous.

But take for example ${\bb Q}$'s indicator function
$$ \chi_{{\bb Q}}(x) \coloneqq \cases{1 & $x\in{\bb Q}$\cr 0 & $x\notin{\bb Q}$} . $$
This function is nowhere continuous and thus is not Riemann integrable.
But we argue that it {\it should} be integrable.

Any theory if an integral should have the following two basic properties:
\benum
    \item {\it monotonicity\/}: if $f(x)\leq g(x)$ for every $x$ in the domain then $\int f(x)\,dx\leq\int g(x)\,dx$.
    \item {\it linearity\/}: $\int(\alpha f(x)+\beta g(x))\,dx=\alpha\int f(x)\,dx+\beta\int g(x)\,dx$.
\eenum

Let $\set{q_n}_{n=1}^\infty$ be an enumeration of the rationals and $\epsilon>0$.
Let us define $E_n\coloneqq\parens{q_n-\frac\epsilon{2^{n+1}},q_n+\frac\epsilon{2^{n+1}}}$, then we should think that
$$ \int\sum_{n=1}^\infty\chi_{E_n}(x)\,dx = \sum_{n=1}^\infty\int\chi_{E_n}(x)\,dx = \sum_{n=1}^\infty\frac\epsilon{2^n} = \epsilon \eqnum $$
Now $\chi_{\bb Q}(x)\leq\sum_{n=1}^\infty\chi_{E_n}(x)$ for all $x$ and so $\int\chi_{\bb Q}(x)\,dx\leq\epsilon$ for every $\epsilon>0$ and so we should think that $\int\chi_{\bb Q}(x)\,dx=0$.

Now obviously $\chi_{\bb Q}$ is not Riemann-integrable, and so there is an issue with the above argument.
In fact there are two: firstly in $(1)$ we assumed that $\int\sum_{n=1}^\infty\chi_{E_n}=\sum_{n=1}^\infty\int\chi_{E_n}$, which only holds if the sum converges uniformly.
Secondly, $\sum_{n=1}^\infty\chi_{E_n}(x)$ takes on infinite values (for every rational number, the sum is infinite) and so it is not even Riemann integrable.

So we want a theory of integration which allows for two things: 1) the ability to deal with convergence of an integral without necessarily needing uniform convergence, 2) the ability to deal with functions
which are not Riemann-integrable.
Lebesgue's theory of integration is based on the following observation: partitioning the domain of the function will necessarily require some form of continuity, so instead try partitioning the {\it range}
of the function.
So given the partition $y_0<\cdots<y_n$ we can imagine some Lebesgue sum of this partition to be
$$ \sum_{i=0}^n y_i\cdot\abs{E_i},\qquad E_i=\set{x\in[a,b]}[y_i\leq f(x)\leq y_{i+1}] $$
where $\abs S$ is some notion of the ``width'' of the set $S$, {\it which we have not yet defined} (it is not the cardinality of the set).
For an interval this should be the length of the interval, but for arbitrary sets it becomes harder to understand how we should approach defining it.
And in order to define the integral of $\chi_{\bb Q}$, it is necessary to define this width for arbitrary sets, or at least for a larger family of sets than just intervals, since computing the Lebesgue
sums of $\chi_{\bb Q}$ will involve terms containing $\abs{{\bb Q}}$.

This ``Lebesgue sum'' will not be precisely how we define Lebesgue integration, but it does give us a starting point: how do we define the ``width'' of a set $E\subseteq{\bb R}$.

\vfill\break

\section{Lebesgue Integration}

\subsection{The Lebesgue Measure}

We would like a function $m$ which measures the width of arbitrary sets $E\subseteq{\bb R}$.
Such a function would ideally satisfy the following properties:
{\def\enumstyle#1{$\m@th\bf(m#1)$}
\benum
    \item $m$ is a function $m\colon\powsetof{\bb R}\longto[0,\infty]$,
    \item for every interval $I$, $m(I)$ is the length of $I$,
    \item the measure of a set is preserved under movement, ie. $m(E+\alpha)=m(E)$ for every $\alpha\in{\bb R}$ where $E+\alpha=\set{x+\alpha}[x\in E]$,
    \item if $\set{E_n}_{n=1}^\infty$ is a sequence of disjoint sets then $m\parens{\bigdcup_{n=1}^\infty E_n}=\sum_{n=1}^\infty m(E_n)$.
        This is called {\it $\sigma$-additivity}.
\eenum}

\bdefn[title=The Outer Measure]

    Let $E\subseteq{\bb R}$, then we define $E$'s {\emphcolor outer measure} to be
    $$ m^*(E) \coloneqq \infof{\sum_{n=1}^\infty\abs{I_n}}[E\subseteq\bigcup_{n=1}^\infty I_n\hbox{ where $I_n$ are all open intervals}] $$

\edefn

This set is nonempty since ${\bb R}$ can be covered by open intervals (eg. $\set{(n,n+2)}_{n\in{\bb Q}}$) and therefore so can every subset.
Obviously we have that $m^*(\varnothing)=0$ since we can take arbitrarily small arbitrary intervals.
Notice that for every $\epsilon>0$, we showed in the previous section there exists open intervals $\set{E_n}_{n=1}^\infty$ such that ${\bb Q}\subseteq\bigcup_{n=1}^\infty E_n$ and
$\sum_{n=1}^\infty\abs{E_n}=\epsilon$.
Thus we have that $m^*({\bb Q})\leq\epsilon$ for every $\epsilon>0$ and so $m^*({\bb Q})=0$ as well.

Notice that we can also take a finite set of $I_n$s, as we can add infinitely many $I_n$s of arbitrarily small width (eg. add $I_n$s of length $\frac\epsilon{2^n}$) and this will add $\epsilon$ to the
sum, and so the infimum remains the same.
Thus we have that
$$ m^*(E) = \infof{\sum_{n=1}^\infty\abs{I_n}}[E\subseteq\bigcup_{j\in J}I_j\hbox{ where $I_n$ are all open intervals and $J$ is countable}] $$

Now, does the outer measure satisfy the conditions $\rm m1$ through $\rm m4$?
\benum
    \item $m^*$ is indeed a function from $\powsetof{\bb R}$ to $[0,\infty]$,
    \item $m^*(I)=\abs I$ for all intervals $I$ (proven below).
    \item $m^*(E+\alpha)=m^*(E)$ as there is a width-preserving bijection between collections of intervals covering $E$ and $E+\alpha$ (in particular $I\mapsto I+\alpha$).
\eenum

But $m^*$ is not $\sigma$-additive, and we can only ensure $\sigma$-subadditivity:

\bthrm

    Let $\set{E_n}_{n=1}^\infty$ be a sequence of (not necessarily disjoint) subsets of ${\bb R}$, then
    $$ m^*\parens{\bigcup_{n=1}^\infty E_n} \leq \sum_{n=1}^\infty m^*(E_n) $$

\ethrm

\Proof for every $E_n$, let us take a cover for $E_n$ of open intervals $\set{I_k^{(n)}}_{k=1}^\infty$ such that $m^*(E_n)\geq\sum_{k=1}^\infty\abs{I_k^{(n)}}+\frac\epsilon{2^n}$.
Then we have that
$$ \sum_{n=1}^\infty m^*(E_n) \geq \sum_{n=1}^\infty\parens{\sum_{k=1}^\infty\abs{I_k^{(n)}}+\frac\epsilon{2^n}} = \sum_{n,k=1}^\infty\abs{I_k^{(n)}}+\epsilon $$
Since $\set{I_k^{(n)}}_{n,k=1}^\infty$ is a cover of $\bigcup_{n=1}^\infty E_n$, we have that
$$ \sum_{n=1}^\infty m^*(E_n) \geq \sum_{n,k=1}^\infty\abs{I_k^{(n)}}+\epsilon \geq m^*\parens{\bigcup_{n=1}^\infty E_n}+\epsilon $$
Since $\epsilon>0$ is arbitrary, we get the desired inequality (by taking $\epsilon\to0$, the above inequality is preserved).
\qed

Of course this does not prove that $m^*$ is not $\sigma$-additive, we have simply proven a weaker condition.

This helps us show that $m^*(I)=\abs I$ for all intervals $I$.
For open intervals this is trivial, and we have that $m^*(E\cup\set a)\leq m^*(E)+m^*(\set a)=m^*(E)$ (the measure of a singleton is obviously zero) and $m^*(E\cup\set a)\geq m^*(E)$ by monotonicity.
All intervals are obtained by adding a finite number of points to an open interval, so for example $m^*([a,b])=m^*\bigl((a,b)\cup\set{a,b}\bigr)=m^*((a,b))=b-a$ as required.

\bprop[title=The Vitali Set]

    There exists no function which satisfies properties $\rm m1$ through $\rm m4$.

\eprop

\Proof let us define an equivalence relation on ${\bb R}$ as follows: $x\sim y\iff x-y\in{\bb Q}$.
Notice that the equivalence classes of this relation are of the form $x+{\bb Q}$ for some $x\in{\bb R}$ and thus they are all countable and dense.
For every equivalence class choose a single representative in $[0,1]$ to form the set $E\subseteq[0,1]$.
Notice then that
$$ {\bb R} = \bigdcup_{q\in{\bb Q}}(E+q) = E+{\bb Q} = \set{x+q}[x\in E,q\in{\bb Q}] $$
this union is disjoint since if $x\in E+q\cap E+p$ then $x=x_1+q$ and $x=x_2+p$ so $x\sim x_1\sim x_2$ but we put only a single representative of each equivalence class into $E$.
For every $y\in[0,1]$, there exists an $x\in E$ such that $x\sim y$ so $x-y\in{\bb Q}\cap[-1,1]$, and so if $y\in E+q$ for some $q\in[-1,1]$.
Thus
$$ [0,1] \subseteq \bigdcup_{q\in{\bb Q}\cap[-1,1]}E+q = E + {\bb Q}\cap[-1,1] \subseteq [-1,2] $$
If $m$ is a function which satisfies the four properties above, it is also monotonic (which can be derived from $\sigma$-additivity: if $A\subseteq B$ then $m(B)=m(A)+m(B\setminus A)\geq m(A)$), and so
$$ m([0,1]) = 1 \leq \sum_{q\in{\bb Q}\cap[-1,1]}m(E) \leq 3 $$
Since ${\bb Q}\cap[-1,1]$ is infinite, we must have that $m(E)=0$ as otherwise the sum is infinite.
But then the sum is $0$ and not greater than $1$, which is a contradiction.
So the {\it Vitali set\/} $\bigcup_{q\in{\bb Q}\cap[-1,1]}E+q$ cannot be measurable.
\qed

So we must weaken one of the conditions.
$m^*$ is already an example of a function which satisfies $\rm m1$ through $\rm m3$ so they are not contradictory (also notice that this means $m^*$ cannot be $\sigma$-additive).
But $\sigma$-additivity is extremely important to Lebesgue's theory of integration, so we will instead weaken $\rm m1$ to be
$$ \hbox{$({\bf m1})$\kern.25cm $m$ is a function ${\cal L}({\bb R})\longto[0,\infty]$} $$
where ${\cal L}({\bb R})$ is the set of all Lebesgue-measurable sets of ${\bb R}$:

\bdefn[title=Carath\'eodory]

    A set $E\subseteq{\bb R}$ is called {\emphcolor Lebesgue measurable} (or just {\it measurable}) if for every $A\subseteq{\bb R}$,
    $$ m^*(A)=m^*(A\cap E)+m^*(A\cap E^c) $$

\edefn

Notice that by subadditivity $m^*(A)\leq m^*(A\cap E)+m^*(A\cap E^c)$, to show that $E$ is Lebesgue measurable it is sufficient to show the other direction ($\geq$).
This implies that all sets $E$ with zero outer measure are Lebesgue measurable: $m^*(A\cap E)+m^*(A\cap E^c)=m^*(A\cap E^c)\leq m^*(A)$ as required ($m^*(A\cap E)\leq m^*(E)=0$).
Further notice that $E$ is measurable if and only if $E^c$ is by symmetry of the definition.

\bthrm

    Every open interval of the form $(a,\infty)$ is measurable.

\ethrm

\Proof let $A$ be a subset of ${\bb R}$.
Let us define
$$ A^+ \coloneqq A\cap(a,\infty),\qquad A^- \coloneqq A\cap(a,\infty)^c = A\cap(-\infty,a] $$
We must show that $m^*(A^+)+m^*(A^-)\leq m^*(A)$.
Let $\epsilon>0$ and let $\set{I_n}_{n=1}^\infty$ be a cover of $A$ consisting of open intervals where $\sum_{n=1}^\infty\abs{I_n}\leq m^*(A)+\epsilon$.
Then for every $n$, define $I_n^+=I_n\cap(a,\infty)$ and $I_n^-=I_n\cap(a,\infty)^c$, so $\set{I_n^+}$ is a cover for $A^+$ and $\set{I_n^-}$ is a cover for $A^-$.
And since these are intervals we have $\abs{I_n}=\abs{I_n^+}+\abs{I_n^-}$.
So we have
$$ m^*(A) \geq \sum_{n=1}^\infty\abs{I_n}-\epsilon = \sum_{n=1}^\infty\abs{I_n^+} + \sum_{n=1}^\infty\abs{I_n^-} + \epsilon \geq m^*(A^+) + m^*(A^-) + \epsilon $$
And since $\epsilon>0$ is arbitrary, we get $m^*(A)\geq m^*(A^+)+m^*(A^-)$ as required.
\qed

\bprop

    The set of all measurable sets is invariant under movement, meaning if $E$ is measurable so too is $E+\alpha$.

\eprop

\Proof let $A\subseteq{\bb R}$ then notice that $x-\alpha\in A\cap(E+\alpha)-\alpha\iff x\in A\cap(E+\alpha)\iff x\in A,E+\alpha\iff x-\alpha\in A-\alpha,E$.
So $A\cap(E+\alpha)-\alpha=(A-\alpha)\cap E$, and so we get
\multlines{
    m^*(A\cap(E+\alpha)) + m^*(A\cap(E+\alpha)^c) = m^*(A\cap(E+\alpha)-\alpha) + m^*(A\cap(E+\alpha)^c-\alpha)\cr
    &= m^*((A-\alpha)\cap E) + m^*((A-\alpha)\cap E^c) = m^*(A-\alpha) = m^*(A)
}
as required.
\qed

\blemm

    The finite union of measurable sets is measurable.
    And if $\set{E_k}_{k=1}^n$ are disjoint and measurable, then
    $$ m^*\parens{\bigdcup_{k=1}^nE_k} = \sum_{k=1}^n m^*(E_k) $$

\elemm

\Proof it is sufficient to prove this for two measurable sets and then to proceed inductively for arbitrary $n$.
So suppose $E,F$ are measurable, we must show that $E\cup F$ is.
Let $A\subseteq{\bb R}$, then
\multlines{
    m^*(A\cap(E\cup F)) + m^*(A\cap(E\cup F)^c) = m^*((A\cap E)\cup(A\cap F)) + m^*(A\cap E^c\cap F^c)\cr
    &= m^*((A\cap E)\dcup(A\cap F\cap E^c)) + m^*(A\cap E^c\cap F^c)\leq m^*(A\cap E) + m^*(A\cap E^c\cap F) + m^*(A\cap E^c\cap F)\cr
    &= m^*(A\cap E) + m^*(A\cap E^c) = m^*(A)
}
The final equality is due to $F$ being measurable, and so it satisfies Carath\'eodory's definition for $A\cap E^c$.

And we similarly prove additivity by induction on $n$.
For two disjoint measurable sets $E,F$, we have by Carath\'eodory on $E$:
$$ m^*(E\dcup F) = m^*((E\dcup F)\cap E) + m^*((E\dcup F)\cap E^c) = m^*(E) + m^*(F) $$
\qed

Since the complement of a set is measurable, and the empty set is measurable, this proposition tells us that ${\cal L}({\bb R})$, the set of all Lebesgue measurable sets, is an algebra of sets
(non-empty, closed under complements and unions/intersections).
Intersections are of course measurable since $\bigcap_{k=1}^nE_k=\parens{\bigcap_{k=1}^nE_k^c}^c$, and so are differences since $E\setminus F=E\cap F^c$.
Notice that since $m^*(E)=m^*(E\cap F)+m^*(E\cap F^c)$, we have
$$ m^*(E\setminus F) = m^*(E\cap F^c) = m^*(E) - m^*(E\cap F) $$
But stronger than this, ${\cal L}({\bb R})$ is in fact a $\sigma$-algebra (to be defined later).

\blemm

    Suppose $E_1,\dots,E_n$ are measurable and disjoint, then for every $A\subseteq{\bb R}$,
    $$ m^*\parens{A\cap\bigdcup_{k=1}^n E_k} = \sum_{k=1}^n m^*(A\cap E_k) $$

\elemm

\Proof We prove this by induction on $n$; for $n=1$ this is trivial.
If this holds for $n$, then using the inductive assumption for $E_1,\dots,E_{n-1},E_n\dcup E_{n+1}$
$$ m^*\parens{A\cap\bigdcup_{k=1}^{n+1}E_k} = \sum_{k=1}^{n-1}m^*(A\cap E_k) + m^*(A\cap(E_n\dcup E_{n+1})) $$
Now,
\multlines{
    m^*(A\cap(E_n\dcup E_{n+1})) = m^*(A\cap(E_n\dcup E_{n+1})\cap E_n) = m^*(A\cap(E_n\dcup E_{n+1})\cap E_n^c)\cr
    &= m^*(A\cap E_n) + m^*(A\cap E_{n+1})
}
and so we get that indeed
$$ m^*\parens{A\cap\bigdcup_{k=1}^{n+1}E_k} = \sum_{k=1}^{n+1}m^*(A\cap E_k) \qed $$

\bthrm

    Let $\set{E_n}_{n=1}^\infty$ be Lebesgue measurable, then $\bigcup_{n=1}^\infty E_n$ is also Lebesgue measurable (equivalently ${\cal L}({\bb R})$ is a $\sigma$-algebra).
    Furthermore if the sets are disjoint, then we have $\sigma$-additivity:
    $$ m^*\parens{\bigdcup_{n=1}^\infty E_n} = \sum_{n=1}^\infty m^*(E_n) $$

\ethrm

\Proof we can assume from the outset that $\set{E_n}$ are disjoint as we can define $F_n\coloneqq E_n\setminus\bigcup_{k=1}^{n-1}E_k$.
This is measurable by the previous lemmas, and $\bigcup_{n=1}^\infty E_n=\bigdcup_{n=1}^\infty F_n$.
Now define
$$ E \coloneqq \bigdcup_{n=1}^\infty E_n,\qquad G_n\coloneqq\bigdcup_{k=1}^nE_k $$
Let $A\subseteq{\bb R}$, and by $G_n$'s measurability we have that
$$ m^*(A) = m^*(A\cap G_n) + m^*(A\cap G_n^c) \geq m^*(A\cap G_n) + m^*(A\cap E^c) $$
By the above lemma this means $m^*(A) \geq \sum_{k=1}^nm^*(A\cap E_k) + m^*(A\cap E^c)$, and this inequality is preserved by taking the limit $n\to\infty$ so
$$ m^*(A) \geq \sum_{k=1}^\infty m^*(A\cap E_k) + m^*(A\cap E^c) \eqnum $$
By subadditivity, we have that $m^*(A\cap E)=m^*\parens{\bigdcup_{k=1}^\infty(A\cap E_k)}\leq\sum_{k=1}^\infty m^*(A\cap E_k)$, so
$$ m^*(A) \geq m^*(A\cap E) + m^*(A\cap E^c) $$
so $E$ is indeed measurable, as required.
If we let $A=E=\bigdcup_{n=1}^\infty E_n$, then we get that $A\cap E_k=E_k$ and so by ({\bf1}) we get that
$$ m^*(E) \geq \sum_{k=1}^\infty m^*(E_k) $$
and $\leq$ is given by subadditivity so we have equality.
\qed

Notice that the Lebesgue outer measure defines a pseudometric on $\powsetof{\bb R}$ by $d(A,B)\coloneqq m^*(A\symdiff B)$.

\bthrm

    $E\subseteq{\bb R}$ is Lebesgue measurable if and only if forall $\epsilon>0$ there exists a countable union of open intervals $\bigcup_{n=1}^\infty I_n$ such that
    $d\parens{\bigcup_{n=1}^\infty I_n,E}<\epsilon$.

\ethrm

\Proof suppose $E$ is measurable, then for every $\epsilon>0$ there exists a cover $E\subseteq\bigcup I_n$ of open intervals such that $m^*(E)+\epsilon>m^*\parens{\bigcup I_n}$.
Now
$$ d\parens{\bigcup I_n,E} = m^*\parens{\bigcup I_n\setminus E} = m^*\parens{\bigcup I_n} - m^*(E) < \epsilon $$
since all the sets in play are measurable, as required.
Now suppose that the condition holds, we must prove that $E$ is measurable.
We can assume without loss of generality that $E\subseteq\bigcup I_n$ (as we can cover the symmetric difference with open intervals).
Now let us take a sequence of such covers $U_k$ such that $m^*(U_k\setminus E)<\frac1k$, and define $U=\bigcap U_k$.
And $m^*(U\setminus E)\leq m^*(U_k\setminus E)<\frac1k$ so $m^*(U\setminus E)=0$, and this means that $U\setminus E$ is measurable.
Since $U_k$ are all measurable, so is $U$ and since $E=U\setminus(U\setminus E)$, $E$ is measurable.
\qed

In fact what we have shown is that every measurable set $E$ is of the form $G\setminus N$, where $G$ is a countable intersection of open sets (a $G_\delta$ set), and $N$ is a set of zero measure.
This theorem implies that every open interval is Lebesgue measurable, as just define $I_n=I$.
Recall that ${\bb R}$ is a second-countable topological space, meaning it has a countable basis, namely the basis $\set{(p,q)}[p<q\in{\bb Q}]$.
This means that every open set in ${\bb R}$ is the countable union of open intervals, and therefore every open set is contained in ${\cal L}({\bb R})$.
And so closed sets, which are complements of open sets, are also measurable.
And so too are $G_\delta$ sets as the countable intersection of measurable sets, and so sets of the form $G\setminus N$ where $G$ is $G_\delta$ and $m^*(N)=0$ are measurable.
So we have proven

\bprop

    $E$ is measurable if and only if there exists a $G_\delta$ set $G$ and a zero-measure set $N$ such that $E=G\setminus N$.

\eprop

\subsection{General Measure Spaces, Briefly}

\bdefn

    Let $X$ be a set, then a non-empty collection $\Sigma\subseteq\powsetof X$ is called a {\emphcolor $\sigma$-algebra} if it satisfies
    \benum
        \item $S\in\Sigma\iff S^c\in\Sigma$
        \item if $\set{S_n}_{n=1}^\infty\subseteq\Sigma$ then $\bigcup_{n=1}^\infty S_n\in\Sigma$
    \eenum
    By $(1)$ and $(2)$ we get that $\sigma$-algebras are also closed under intersections, and so $\varnothing=S\cap S^c\in\Sigma$ and $X=\varnothing^c\in\Sigma$.

    If $X$ is a set and $\Sigma$ a $\sigma$-algebra over $X$, then $(X,\Sigma)$ is called a {\emphcolor measurable space}.
    And a {\emphcolor measure space} is a triplet $(X,\Sigma,\mu)$ where $\mu$ is a $\sigma$-algebra over $X$, and $\mu$ is a {\emphcolor measure} over $\Sigma$:
    \benum
        \item $\mu\colon\Sigma\longto[0,\infty]$,
        \item $\mu$ is $\sigma$-additive: if $\set{S_n}_{n=1}^\infty\subseteq\Sigma$ are disjoint $\mu\parens{\bigdcup_{n=1}^\infty S_n}=\sum_{n=1}^\infty\mu(S_n)$
        \item $\mu(\varnothing)=0$.
    \eenum

\edefn

\bexam

    Let $m$ be the {\emphcolor Lebesgue measure} over ${\bb R}$, the restriction of $m^*$ to the collection of Lebesgue measurable sets: $m\coloneqq m^*\bigr|_{{\cal L}({\bb R})}$.
    Then $({\bb R},{\cal L}({\bb R}),m)$ is a measure space.

\eexam

Measures are obviously monotonic: if $E,F\in\Sigma$ and $E\subseteq F$ then $\mu(F)=\mu((F\setminus E)\dcup E)=\mu(F\setminus E)+\mu(E)\geq\mu(E)$ (since recall that $\sigma$-algebras are closed under
differences, as they are just the intersections of the complement).
And measures are subadditive:

\bthrm

    Let $(X,\Sigma,\mu)$ be a measure space, then for any $\set{A_n}_{n=1}^\infty\subseteq\Sigma$ (not necessarily disjoint):
    $$ \mu\parens{\bigcup_{n=1}^\infty A_n}\leq\sum_{n=1}^\infty\mu(A_n) $$

\ethrm

\Proof define $B_n\coloneqq A_n\setminus\bigcup_{k=1}^{n-1}A_k$, these are all disjoint and $\bigdcup_{n=1}^\infty B_n=\bigcup_{n=1}^\infty A_n$.
And since $B_n\subseteq A_n$, $\mu(B_n)\leq\mu(A_n)$.
Thus
$$ \mu\parens{\bigcup_{n=1}^\infty A_n} = \mu\parens{\bigdcup_{n=1}^\infty B_n} = \sum_{n=1}^\infty \mu(B_n) \leq \sum_{n=1}^\infty\mu(A_n) \qed $$

One of the most important properties of measure spaces is {\it continuity from below} and {\it continuity from above}:

\bthrm[title=Continuity of Measures, name=contofmeasures]

    Let $(X,\Sigma,\mu)$ be a measure space, then
    \benum
        \item {\emphcolor Continuity from below}: if $E_1\subseteq E_2\subseteq\cdots$ is an increasing sequence in $\Sigma$, then
            $$ \mu\parens{\bigcup_{n=1}^\infty E_n} = \lim_{n\to\infty}\mu(E_n) $$
        \item {\emphcolor Continuity from above}: if $E_1\supseteq E_2\supseteq\cdots$ is a decreasing sequence in $\Sigma$ such that $\mu(E_1)<\infty$, then
            $$ \mu\parens{\bigcap_{n=1}^\infty E_n} = \lim_{n\to\infty}\mu(E_n) $$
    \eenum

\ethrm

\Proof notice that the limits in both cases exist since the sequences are monotonic.
\benum
    \item Let $E\coloneqq\bigcup_{n=1}^\infty E_n$, then $E\supseteq E_n$ so $\mu(E)\geq\mu(E_n)$ and so $\mu(E)\geq\lim_{n\to\infty}\mu(E_n)$.
        Define $F_n\coloneqq E_n\setminus E_{n-1}$, so that $F_n$ are disjoint and $\bigdcup_{k=1}^nF_k=E_n$ for all $n$, and so $\bigdcup_{k=1}^\infty F_k=E$, so
        $$ \mu(E) = \sum_{k=1}^\infty \mu(F_k) = \lim_{n\to\infty}\sum_{k=1}^n\mu(F_k) = \lim_{n\to\infty}\mu\parens{\bigdcup_{k=1}^nF_k}=\lim_{n\to\infty}\mu(E_n) $$
        as required.
    \item Define $E\coloneqq\bigcap_{n=1}^\infty E_n$, $F_n\coloneqq E_1\setminus E_n$, $F\coloneqq\bigcup_{n=1}^\infty F_n$.
        Thus $E=E_1\setminus F$, and so by above
        $$ \mu(E) = \mu(E_1) - \mu(F) = \mu(E_1) - \lim_{n\to\infty}\mu(F_n) = \mu(E_1) - \lim_{n\to\infty}\bigl(\mu(E_1)-\mu(E_n)\bigr) = \lim_{n\to\infty}\mu(E_n) $$
        \qed
\eenum

\subsection{Measurable Functions}

\bthrm

    Let $(X,\Sigma)$ be a measureable space, and $f\colon X\longto\overline{\bb R}\ (={\bb R}\cup\set{\pm\infty})$ be an extended real function.
    Then the following are equivalent:
    \benum
        \item for every $\alpha\in{\bb R}$, $\set{x\in X}[f(x)>\alpha]\in\Sigma$,
        \item for every $\alpha\in{\bb R}$, $\set{x\in X}[f(x)\geq\alpha]\in\Sigma$,
        \item for every $\alpha\in{\bb R}$, $\set{x\in X}[f(x)<\alpha]\in\Sigma$,
        \item for every $\alpha\in{\bb R}$, $\set{x\in X}[f(x)\leq\alpha]\in\Sigma$.
    \eenum

\ethrm

\Proof notice that $(1)\iff(4)$ and $(2)\iff(3)$ are trivial since $\set{x\in X}[f(x)>\alpha]^c=\set{x\in X}[f(x)\leq\alpha]$ and similar, and $S$ is measurable if and only if $S^c$ is.
Now we prove $(1)\iff(2)$, notice that
$$ \set{x\in X}[f(x)\geq\alpha] = \bigcap_{n=1}^\infty\set{x\in X}[f(x)>\alpha-\frac1n] $$
so if $(1)$ then the right side is the countable intersection of measurable sets and is therefore measurable, so $(1)\implies(2)$.
And
$$ \set{x\in X}[f(x)>\alpha] = \bigcup_{n=1}^\infty\set{x\in X}[f(x)\geq\alpha+\frac1n] $$
so anlogously $(2)\implies(1)$.
Similarly for $(3)\iff(4)$.
\qed

\bdefn

    Let $(X,\Sigma)$ be a measurable space and $f\colon X\longto\overline{\bb R}$ an extended real function.
    If any of the above equivalent conditions hold, then $f$ is a {\emphcolor $\Sigma$-measurable function} (if $\Sigma$ is understood, just measurable).

\edefn

Notice that constant functions $f\colon x\mapsto c$ are measurable: for $\alpha\geq c$, $\set{x\in X}[f(x)>\alpha]=\varnothing$ which is measurable.
And for $\alpha<c$, $\set{x\in X}[f(x)>\alpha]=X$ which is measurable.

\bcoro

    Let $f$ be an extended real function, then
    \benum
        \item if $f$ is measurable, for every $x_0\in\overline{\bb R}$, $f^{-1}(x_0)\in\Sigma$,
        \item if $f$ is measurable, for every interval $I\subseteq\overline{\bb R}$, $f^{-1}(I)\in\Sigma$,
        \item if $f\colon{\bb R}\longto{\bb R}$ is continuous, then $f$ is Lebesgue measurable (ie. measurable in $({\bb R},{\cal L}({\bb R}))$).
    \eenum

\ecoro

\Proof
\benum
    \item Notice that
        $$ f^{-1}(x_0) = \set{x\in X}[f(x)\geq x_0]\cap\set{x\in X}[f(x)\leq x_0] $$
        which is the intersection of measurable sets.
    \item For an interval of the form $I=[a,b)$,
        $$ f^{-1}(I) = \set{x\in X}[f(x)\geq a]\cap\set{x\in X}[f(x)<b] $$
        which is the intersection of measurable sets.
        All other intervals are done similarly.
    \item Let $\alpha\in{\bb R}$, then $E=(\alpha,\infty)$ is open and so $f^{-1}(E)$ is open and thus Lebesgue measurable.
        And $f^{-1}(E)=\set{x\in{\bb R}}[f(x)>\alpha]\in{\cal L}({\bb R})$ as required.
        \qed
\eenum

\bthrm

    Let $(X,\Sigma)$ be a measurable space, $f,g\colon X\longto{\bb R}$ and $c\in{\bb R}$.
    Then $f\pm g,c\cdot f,f\cdot g$ are measurable.

\ethrm

\benum
    \item Let $\alpha\in{\bb R}$, then $f(x)+g(x)<\alpha$ if and only if $f(x)<\alpha-g(x)$ if and only if there exists a $\beta\in{\bb Q}$ such that $f(x)<\beta<\alpha-g(x)$.
        So
        $$ \set{x\in X}[f(x)+g(x)<\alpha] = \bigcup_{\beta\in{\bb Q}}\bigl(\set{x\in X}[f(x)<\beta]\cap\set{x\in X}[g(x)<\alpha-\beta]\bigr) $$
        which is the countable union of the intersection of measurable sets, which is measurable.
        For $f-g$, this is due to $(2)$ which states that $-g$ is measurable.
    \item Notice that for $\alpha\in{\bb R}$, if $c>0$ then $\set{x\in X}[c\cdot f(x)<\alpha]=\set{x\in X}[f(x)<\frac\alpha c]$ which is measurable.
        If $c<0$ then $\set{x\in X}[c\cdot f(x)<\alpha]=\set{x\in X}[f(x)>\frac\alpha c]$.
        If $c=0$, then $c\cdot f=0$, and constant functions are measurable always.
        This combined with $(1)$ show us that all linear combinations of measurable functions are measurable.
    \item First we will prove that $f^2$ is measurable.
        Notice that for $\alpha\geq0$, the set $\set{x\in X}[f(x)^2\geq\alpha]=\set{x\in X}[f(x)\geq\sqrt\alpha]$ is measurable.
        And for $\alpha<0$, $\set{x\in X}[f(x)\geq\alpha]=X$ is measurable.
        And
        $$ f\cdot g = \frac14\parens{(f+g)^2 - (f-g)^2} $$
        We just showed that $f\pm g$ are measurable, and so too are $(f\pm g)^2$ and any of their linear combinations, so $f\cdot g$ is measurable.
        \qed
\eenum

Notice that for $f,g\colon X\longto\overline{\bb R}$, $f(x)+g(x)$ is undefined when $f(x)=\infty$ and $g(x)=-\infty$ or vice versa.
We generally take that $\pm\infty\cdot0\coloneqq0$, but $f(x)\cdot g(x)$ is not necessarily defined when $f(x)=\pm\infty$ and $g(x)=0$ or vice versa.
But if we define $f\pm g$ and $f\cdot g$ to be any arbitrary constant (eg. $0$) at these problematic points, we can similarly show that $f\pm g$ and $f\cdot g$ are measurable.

\bthrm

    Let $(X,\Sigma)$ be a measurable space and $\set{f_n}_{n=1}^\infty$ a sequence of extended real measurable functions.
    Then $\sup_nf_n(x),\inf_nf_n(x),\limsup f_n(x),\liminf f_n(x)$, and $\lim f_n(x)$ if it exists are measurable.

\ethrm

\Proof first we prove it for $f(x)=\sup_nf_n(x)$.
Let $\alpha\in{\bb R}$, then $f(x)\leq\alpha$ if and only if $f_n(x)\leq\alpha$ for all $\alpha$, so
$$ \set{x\in X}[f(x)\leq\alpha] = \bigcap_{n=1}^\infty\set{x\in X}[f_n(x)\leq\alpha] $$
which is the countable intersection of measurable sets.
Similar for $g(x)=\inf_nf_n(x)$.

Now define $h(x)=\limsup_nf_n(x)$, then $h(x)=\inf_k\sup_{n\geq k}f_n(x)$.
Let us define $g_k(x)=\sup_{n\geq k}f_n(x)$, so $h(x)=\inf_kg_k(x)$.
Since $g_k$ is the supremum of a sequence of measurable functions, it is measurable.
And so $h$ is the infimum of measurable functions, meaning it too is measurable.
We use an analogous proof for the case $\liminf_nf_n(x)=\sup_k\inf_{n\geq k}f_n(x)$.
If $\lim_nf_n(x)$ exists, then it is just equal to $\limsup_nf_n(x)=\liminf_nf_n(x)$.
\qed

\bye

