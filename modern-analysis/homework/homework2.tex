\input pdfToolbox

\setlayout{horizontal margin=2cm, vertical margin=2cm}
\parindent=0pt
\parskip=3pt plus 2pt minus 2pt

\input ../preamble

\footline={}

\setcounter{section}{2}

%%%%%%%%%%%%%%%%%%%%%%%%%%%%%%%%%%%%%%%%%%%%%%%%%%%%%%%%%%%%%%%%

\def\printmcount{\the\counter{section}.\the\counter{math counter}}

{\bppbox{rgb{.5 1 .5}}{rgb{0 .4 0}}{rgb{.1 .4 0}}

    \centerline{\setfontandscale{bf}{20pt}Modern Analysis}
    \smallskip
    \centerline{\setfont{it}Homework \the\counter{section}}
    \centerline{\setfont{it}Ari Feiglin}

\eppbox}

\bexerc

    Determine if the following collections are alegbras or $\sigma$-algebras:
    \benum
        \item $X=\set{1,2,3,4},\ S=\set{\varnothing,\set1,\set2,\set{1,2},\set{1,2,3,4}}$
        \item $X=[0,1],\ S=\set\varnothing\cup\set{A}[{[0,1/2]}\subseteq A]$
        \item $X=[0,1],\ S=\set{A}[\set{0,1}\subseteq A\hbox{ or }\set{0,1}\cap A=\varnothing]$
        \item $X=\set{0,1}^{\bb N}$ and $A\in S$ if and only if every $(x_n)\in A$ satisfies that $\set{(y_n)}[y_1=x_1]\subseteq A$.
    \eenum

\eexerc

\benum
    \item This is not an algebra since $\set1^c=\set{2,3,4}$ is not in $S$.
    \item This is not an algebra since $[0,1/2]\in S$ but $[0,1/2]^c\notin S$.
    \item This is a $\sigma$-algebra: $\varnothing\in S$ since it has an empty intersection with $\set{0,1}$, $A\cap\set{0,1}=\varnothing$ if and only if $\set{0,1}\subseteq A^c$ so $S$ is closed under
    complements.
    Finally if $\set{A_n}_{n=1}^\infty\subseteq S$, if there exists any $n$ such that $\set{0,1}\subseteq A_n$ then $\set{0,1}\subseteq\bigcup_nA_n$ so $\bigcup_nA_n\in S$.
    Otherwise $A_n\cap\set{0,1}=\varnothing$ for every $n$, so $\parens{\bigcup_nA_n}\cap\set{0,1}=\bigcup_n(A_n\cap\set{0,1})=\varnothing$, so $\bigcup_nA_n\in S$ as well.
    \item This is a $\sigma$-algebra since $\varnothing\in S$ vaccuously.
    If $A\in S$, then let $(x_n)\in A^c$, if $y_1=x_1$ then if $(y_n)\in A$, since $y_1=x_1$ we'd have $(x_n)\in A$ in contradiction, so $(y_n)\in A^c$, meaning $A^c\in S$.
    Now suppose $\set{A_n}_{n=1}^\infty\subseteq S$, let $(x_k)_k\in\bigcup A_n$, then $(x_k)_k\in A_n$ for some $n$, and if $y_1=x_1$ then $(y_k)_k\in A_n\subseteq\bigcup_nA_n$, meaning
    $\set{(y_k)_k}[y_1=x_1]\subseteq\bigcup_nA_n$, so $\bigcup_nA_n\in S$ as required.
\eenum

\bexerc

    Let $\set{f_n}_{n\in{\bb N}}$ be a sequence of continuous functions on $[0,1]$.
    Prove that
    $$ A = \set{x\in[0,1]}[f_n(x)\to0] $$
    is a Borel set.

\eexerc

By definition $f_n(x)\to0$ if and only if for every $\epsilon>0$ there exists an $n_0\in{\bb N}$ such that for every $n\geq n_0$, $\abs{f_n(x)}<\epsilon$, meaning $x\in f_n^{-1}(-\epsilon,\epsilon)$.
Since the rationals are dense, we can take only rational $\epsilon$s.
Therefore
$$ A = \bigcap_{\epsilon\in{\bb Q}_+}\bigcup_{n_0\in{\bb N}}\bigcap_{n=n_0}^\infty f_n^{-1}(-\epsilon,\epsilon) $$
Since $f_n$ is continuous, $f_n^{-1}(-\epsilon,\epsilon)$ is an open set for every $\epsilon>0$ and $n$.
So this is a composition of intersections and unions of open sets and is therefore a Borel set.

\bexerc

    Let $m$ be the Lebesgue measure and $\set{A_n}_{n\in{\bb N}}$ be a sequence of measurable sets in $[0,1]$.
    Define $F=\set{x}[\forall n\in{\bb N}\exists k>n\colon x\in A_k]$.
    Prove
    \benum
        \item $F=\bigcap_{n=1}^\infty\bigcup_{k=n}^\infty A_k$,
        \item if $m(A_n)\geq\delta>0$ for all $n$ then $m(F)\geq\delta$,
        \item if $\sum_{n=1}^\infty m(A_n)<\infty$ then $m(F)=0$,
        \item there exists a sequence $\set{A_n}$ such that $\sum_{n=1}^\infty m(A_n)=\infty$ and $m(F)=0$.
    \eenum

\eexerc

\benum
    \item $x\in F$ if and only if for every $n$ there exists a $k>n$ such that $x\in A_k$.
    This is obviously contained in $\bigcap_n\bigcup_{k\geq n}A_k$, and if $x\in\bigcap_n\bigcup_{k\geq n}A_k$ then for every $n$ there exists a $k\geq n$ such that $x\in A_k$.
    So for any $n$, there exists a $k\geq n+1>n$ such that $x\in A_k$, so $x\in F$.
    \item By continuity from above, since $\set{\bigcup_{k\geq n}A_k}_n$ is a decreasing sequence,
    $$ m(F) = \lim_{n\to\infty}m\parens{\bigcup_{k\geq n}A_k} \geq \lim_{n\to\infty}m(A_n) \geq \lim_{n\to\infty}\delta = \delta $$
    \item Again by continuity from above,
    $$ m(F) = \lim_{n\to\infty}m\parens{\bigcup_{k\geq n}A_k} \leq \lim_{n\to\infty}\sum_{k\geq n}m(A_k) $$
    since $\sum_{k=1}^\infty m(A_k)$ converges, its tail must converge to zero, meaning $\lim_{n\to\infty}\sum_{k\geq n}m(A_k)=0$, so $m(F)=0$.
    \item Define $A_n=(0,1/n)$, then $m(A_n)=\frac1n$ so its sum diverges.
    But $F=\varnothing$ since if $x\in F$, then eventually $\frac1n<x$ so for every $k\geq n$, $x\notin A_k$.
    So $m(F)=0$ as required.
\eenum

\def\C{{\cal C}}
\bexerc

    Let $\C$ be the Cantor set, prove it
    \benum
        \item is compact,
        \item does not contain any interval of positive measure,
        \item is not the countable union of closed intervals.
    \eenum

\eexerc

\benum
    \item Recall the definition of the Cantor set, defining $C_0\coloneqq[0,1]$ and $C_{n+1}\coloneqq\frac13C_n\bigcup\parens{\frac23+\frac13C_n}$,
    $$ \C \coloneqq \bigcap_{n=1}^\infty C_n $$
    Since all $C_n$ are closed (by induction: $C_0$ is closed, then $C_{n+1}$ is the union of two closed sets which is also closed), $\C$ is the countable intersection of closed sets, and is therefore also
    closed.
    Therefore $\C$ is closed and bound, meaning it is compact.
    \item By induction, $m(C_n)=\parens{\frac23}^n$, since
    $$ m(C_{n+1}) = \frac13m(C_n) + \frac13m(2+C_n) = \frac23m(C_n) = \parens{\frac23}^{n+1} $$
    And so by continuity from above,
    $$ m(\C) = \lim_{n\to\infty}m\parens{\bigcap_{k=1}^n C_k} \leq \lim_{n\to\infty}m(C_n) = \lim_{n\to\infty}\parens{\frac23}^n = 0 $$
    so $\C$ has zero measure, and therefore cannot contain any set of positive measure.
    \item Since $\C$ contains no set of positive measure, it cannot contain any interval and in particular cannot be the union of intervals.
\eenum

\def\E{{\cal E}}
\bexerc

    Let $X$ be a set and $\E\subseteq\powsetof X$.
    Show that for every $F\in\sigma(\E)$, there exists a countable subset $\E'\subseteq\E$ such that $F\in\sigma(\E')$.

\eexerc

This is equivalent to showing that
$$ \sigma(\E) = \bigcup_{\E'\subseteq\E\hbox{\setscale{7pt} countable}}\sigma(\E') $$
Since $\sigma(\E')\subseteq\sigma(\E)$ for every $\E'\subseteq\E$, $\supseteq$ is trivial.
Now we claim that the right hand side (denote it \def\U{{\cal U}} $\U$) is a $\sigma$-algebra: it obviously contains $\varnothing$ and if $A\in\U$ then $A\in\sigma(\E')$ for some countable $\E'\subseteq\E$
and so $A^c\in\sigma(\E')\subseteq\U$.
And if $\set{A_n}_n\in\U$, then for every $n$ there exists a countable $\E_n\subseteq\E$ such that $A_n\in\sigma(\E_n)$.
Then define $\E'\coloneqq\bigcup_n\E_n$ which is the countable union of countable sets, so $\E'$ is countable.
And for every $n$, $A_n\in\sigma(\E_n)\subseteq\sigma(\E')$, so $\bigcup_nA_n\in\sigma(\E')\subseteq\U$.
Thus $\U$ is a $\sigma$-algebra containing $\E$, since for every $A\in\E$, $A\in\sigma(\set A)$, thus $\sigma(\E)\subseteq\U$ and so $\sigma(\E)=\U$ as required.

\bye

