\input pdfToolbox

\setlayout{horizontal margin=2cm, vertical margin=2cm}
\parindent=0pt
\parskip=3pt plus 2pt minus 2pt

\input ../preamble

\footline={}

\setcounter{section}{1}

%%%%%%%%%%%%%%%%%%%%%%%%%%%%%%%%%%%%%%%%%%%%%%%%%%%%%%%%%%%%%%%%

\def\printmcount{\the\counter{section}.\the\counter{math counter}}

{\bppbox{rgb{.5 1 .5}}{rgb{0 .4 0}}{rgb{.1 .4 0}}

    \centerline{\setfontandscale{bf}{20pt}Modern Analysis}
    \smallskip
    \centerline{\setfont{it}Homework \the\counter{section}}
    \centerline{\setfont{it}Ari Feiglin}

\eppbox}

\bexerc

    Let $E\subseteq{\bb R}$ and $a,b\in{\bb R}$.
    Show that $m^*(aE+b)=\abs a m^*(E)$.

\eexerc

Note that there is a one-to-one correspondence between open covers of $aE+b$ and $E$: if $\set{I_n}$ is a cover of open intervals of $E$, $\set{aI_n+b}$ is a cover of $aE+b$, and every cover of $aE+b$ is
of this form (since if $\set{J_n}$ is a cover of $aE+b$, $\set{\frac{J_n-b}a}$ is a cover of $E$).
And since $\abs{aI+b}=\abs a\abs I$,
$$ m^*(aE+b) = \infof{\sum\abs{aI_n+b}}[E\subseteq\bigcup I_n]  = \abs a\infof{\sum\abs{I_n}}[E\subseteq\bigcup I_n] = \abs a m^*(E) $$

\bexerc

    Prove or disprove:
    \benum
        \item if $A\subseteq{\bb R}$ is bound, then $m^*(A)<\infty$,
        \item if $A\subseteq{\bb R}$ has finite measure, $A$ is bound.
    \eenum

\eexerc

\benum
    \item This is true: if $A$ is bound then $A\subseteq[a,b]$ for some $a,b\in{\bb R}$ and by monotonicity $m^*(A)\leq m^*([a,b])=b-a<\infty$.
    \item This is false: $m^*({\bb Q})=0$ but ${\bb Q}$ is unbound.
\eenum

\bexerc

    We say that a set $S$ is $G_\delta$ if it is the countable intersection of open sets.
    Show that for every $E\subseteq{\bb R}$, there exists a $G_\delta$ set $S$ such that $m^*(E)=m^*(S)$.

\eexerc

For every $n$, let $\set{I_{n,k}}_k$ be an open cover of $E$ such that $\sum_k\abs{I_{n,k}}\leq m^*(E)+\frac1n$.
Then define
$$ S\coloneqq\bigcap_n\bigcup_k I_{n,k} $$
which is a $G_\delta$ set since $\bigcup_k I_{n,k}$ are open as the unions of open sets.
Furthermore, $S\subseteq\bigcup_k I_{n,k}$ for every $n$ and so by subadditivity for every $n$,
$$ m^*(S) \leq \sum_k\abs{I_{n,k}} \leq m^*(E) + \frac1n $$
thus $m^*(S)\subseteq m^*(E)$.
And since $E\subseteq\bigcup_k I_{n,k}$ for every $n$, $E\subseteq S$ so $m^*(E)\leq m^*(S)$ and therefore $m^*(E)=m^*(S)$ as required.

\bexerc

    Suppose $f\colon{\bb R}\longto{\bb R}$ is continuous, show that $f^{-1}\set0$ is $G_\delta$.

\eexerc

We claim that
$$ f^{-1}\set0 = \bigcap_{n=1}^\infty f^{-1}\parens{-\frac1n,\frac1n} $$
since $f^{-1}\set0\subseteq f^{-1}\parens{-\frac1n,\frac1n}$ for every $n$ we have $\subseteq$.
And if $-\frac1n<f(x)<\frac1n$ for all $n$ then $f(x)=0$ so we have $\supseteq$ and therefore equality.
Since $f$ is continuous and $\parens{-\frac1n,\frac1n}$ is open, $f^{-1}\parens{-\frac1n,\frac1n}$ is open and therefore $f^{-1}\set0$ is $G_\delta$.

\bye

