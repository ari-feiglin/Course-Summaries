\input pdfToolbox

\setlayout{horizontal margin=2cm, vertical margin=2cm}
\parindent=0pt
\parskip=3pt plus 2pt minus 2pt

\input preamble

\footline={}

%%%%%%%%%%%%%%%%%%%%%%%%%%%%%%%%%%%%%%%%%%%%%%%%%%%%%%%%%%%%%%%%

\headline={\pageborder{rgb{1 .7 .1}}{rgb{.6 .4 0}}{5}}

\color rgb{.8 .5 .1}

{\def\boxshadowcolor{rgb{.8 .6 0}}
\bppbox{rgb{1 .8 .1}}{rgb{.6 .4 0}}{rgb{.4 .2 0}}

    \centerline{\setfontandscale{bf}{20pt}Formal Verification Methods}
    \smallskip
    \centerline{\setfont{it}Lectures by Doron Peled}
    \centerline{\setfont{it}Summary by Ari Feiglin \setfont{rm}({\tt ari.feiglin@gmail.com})}

\eppbox

\bigskip

\bppbox{rgb{1 .8 .1}}{rgb{.6 .4 0}}{rgb{.4 .2 0}}
    \section*{Contents}
    
    \tableofcontents
\eppbox

}

\vfill\break

\color{black}

\pageno=1
\newif\ifpageodd
\pageoddtrue
\headline={%
    \hbox to \hsize{\color{black}%
        \ifpageodd\hfil{\it\currsubsection\quad\bf\folio}\global\pageoddfalse%
        \else{\bf\folio\quad\it\currsubsection}\hfil\global\pageoddtrue\fi%
    }%
}

\section{Transition Systems}

When modelling systems, one must take into consideration a variety of factors: for example, is the system sequential or concurrent? When investigating the transitions between states, how granular should they
be?
These questions are common questions in computer science, the terms may not be though.
A sequential system is a system with only one thread of execution, while a concurrent system may be multithreaded/multiprogrammed/multiprocessed.
The granularity of a transition refers to how detailed we view the transition: is the command $\tt x\coloneqq y$ atomic? Or do the variables first need to be loaded into memory?

We now begin to discuss how we model systems.

\bdefn

    A {\emphcolor transition system} over a first-order language ${\cal L}$ is a triplet $({\cal S},T,\Theta)$, where
    \benum
        \item ${\cal S}$ is a (potentially many-sorted) ${\cal L}$-structure.
            The symbols of ${\cal L}$ correspond to the symbols utilized within the program in question.
            For example, ${\cal L}$ may contain the $+$ operator, $<$ relation, etc.
            As opposed to general first-order logic, the set of variables $V$ is taken to be finite here.
            This set of variables correspond to precisely what you'd expect: the set of all variables in the program.
            This includes internal registers utilized by the program, called the {\it program counters}, for which there is one for each concurrent process, and they point to the location of the next
            instruction to be executed.
        \item $T$ is a {\it finite} set of {\emphcolor transitions}.
            Each transition $t\in T$ has the form (${\cal T}_{\cal L}$ is the set of ${\cal L}$-terms)
            $$ p\longto (v_1,\dots,v_n)\coloneqq(e_1,\dots,e_n) \qquad (v_1,\dots,v_n\in V,e_1,\dots,e_n\in{\cal T}_{\cal L}) $$
            $p$ is a quantifier-free formula in ${\cal L}$.
            Notice that even in concurrent systems, there is a single set of transitions, meaning all the transitions are grouped together.
        \item $\Theta$ is the {\it initial condition}, a quantifier-free formula in ${\cal L}$.
    \eenum

    In this model, a {\emphcolor state} is an assignment of the variables in $V$ to elements of the domain of ${\cal S}$.
    In other words, a state is a valuation $s\colon V\longto S$ ($S={\sl dom}{\cal S}$), so ${\cal S}$ together with a state form an ${\cal L}$-model.
    The {\emphcolor state space} is the set of all possible states, which can be taken to be $S^V$ or a subset of this (if for example, ${\cal S}$ contains all the naturals, but our computer's memory is
    bound in size).
\edefn

A transition of the form $p\longto(v_1,\dots,v_n)\coloneqq(e_1,\dots,e_n)$ intuitively can execute from any state which satisfies the condition $p$.
The condition $p$ is called the {\it enabledness condition} of the transition $t$, and if $p$ is satisfied by the state $s$, ie. ${\cal S},s\vDash p$ (recall that ${\cal S},s$ is simply an ${\cal L}$-model),
then $t$ is said to be {\it enabled} at $s$.
$t$ transitions from a state $s$ in which it is enabled to a state where the value of each $v_i$ is set to $e_i^{\cal S}$ for $1\leq i\leq n$, denoted $s'=t(s)=s[e_1/v_1,\dots,e_n/v_n]$.

Note that the assignment is simultaneous: $(x,y)\coloneqq(y,x)$ has the effect of swapping the values of $x$ and $y$.
Allowing for simultaneous assignments may seem contrary to the idea of having transitions be atomic.
But this again goes back to the notion of granularity: we decide what transitions are atomic, and it can be useful to view assignments, even simultaneous ones, as atomic.

\bdefn

    Given a system $({\cal S},T,\Theta)$, an {\emphcolor execution} is an infinite sequence of states $s_0,s_1,s_2,\dots$ such that ${\cal S},s_0\vDash\Theta$ (we will also use the notation
    $s_0\vDash^{\cal S}\Theta$), meaning the first state satisfies the initial condition, and for every $i\geq0$ one of the following holds:
    \benum
        \item There exists some transition $p\longto(v_1,\dots,v_n)\coloneqq(e_1,\dots,e_n)\in T$ that is enabled at $s_i$, ie. $s_i\vDash^{\cal S}p$, and $s_{i+1}$ is obtained by this assignment, meaning
        $s_{i+1}=s_i[e_1^{\cal S}/v_1,\dots,e_n^{\cal S}/v_n]$.
        \item There is no transition enabled at $s_i$, meaning for every transition $t\in T$ whose enabledness condition is $p$, $s_i\nvDash^{\cal S}p$.
            In this case, for every $j\geq i$ we set $s_j=s_i$.
            So in such a case, we manually extend the sequence if it can no longer be extended.
    \eenum

\edefn

Instead of the second condition, we could add a new transition to $T$ of the form $\neg(p_1\lor\cdots\lor p_n)\to(v\coloneqq v)$ where $p_1,\dots,p_n$ exhaust all the enabledness conditions of
transitions in $T$, and $v\in V$ is arbitrary.
Alternatively we could allow for finite sequences of states, provided the final state enables no transition.

A state which appears in some execution of a program (system) is called {\it reachable}.
Not every state needs to be reachable: consider a program that can hold (bounded) natural numbers with variables $y_1,y_2$ and the program is written in such a way that $y_1\geq y_2$ always.
But the state $s[y_1]=1$ and $s[y_2]=2$ is a valid, yet unreachable, state.

We can view the execution of a system as a {\it scheduler} which can generate interleaved sequences (sequences where a single transition is executed at a time)

\medskip
\algorithm
    \Function{Scheduler}{${\cal S},T,\Theta$}
        \State\textbf{choose} some initial state $s$ such that $s\vDash^{\cal S}\Theta$
        \While{$s$ has an enabled transition}
            \State\textbf{choose} a transition $t$ enabled by $s$
            \State $s\gets t(s)$
        \EndWhile
        \Comment Extend the sequence infinitely if the final state has no enabled transition\EndComment
        \State \textbf{repeat} $s$ forever
    \EndFunc
\ealgorithm
\medskip

This scheduler is non-deterministic as the choice for the initial state and the choices between transitions enabled at each state along the execution are made non-deterministically.

\bexam

    Let us give an example of {\it mutual exclusion}: we have two programs sharing a shared {\it critical section} (here the variable {\tt turn}):
    
    \medskip
    \hbox to \hsize{
        \hfil\vbox{\hsize=.4\hsize
        \algorithm
            \InnerText{\textbf{routine} {\tencsc Program1}}
                \While{{\sf true}}
                    \Comment wait until ${\tt turn}$ is zero\EndComment
                    \State {\sf wait}(${\tt turn}=0$)
                    \State ${\tt turn}\gets1$
                \nonum\EndWhile
            \InnerText{\textbf{end routine}}
        \ealgorithm
        }
        \hfil\vbox{\hsize=.4\hsize
        \algorithm
            \InnerText{\textbf{routine} {\tencsc Program2}}
                \While{{\sf true}}
                    \Comment wait until ${\tt turn}$ is one\EndComment
                    \State {\sf wait}(${\tt turn}=1$)
                    \State ${\tt turn}\gets0$
                \nonum\EndWhile
            \InnerText{\textbf{end routine}}
        \ealgorithm
        }
        \hfil
    }
    \medskip
    
    \def\turn{{\tt turn}}
    \def\pc{{\tt pc}}
    In this example, we have three variables: $\turn$, the first program counter $\pc_1$, and the second program counter $\pc_2$.
    The transitions are as follows:
    $$ \displaylines{
        t_0\colon \pc_1=1\longto\pc_1\coloneqq2,\ t_1\colon(\pc_1=2\land\turn=0)\longto \pc_1\coloneqq3,\ t_2\colon(\pc_1=3)\longto(\pc_1,\turn)\coloneqq(1,1)\cr
        t_3\colon \pc_2=1\longto\pc_2\coloneqq2,\ t_4\colon(\pc_2=2\land\turn=1)\longto \pc_2\coloneqq3,\ t_5\colon(\pc_2=3)\longto(\pc_2,\turn)\coloneqq(1,0)
    } $$
    Then the initial condition is
    $$ \Theta = \pc_1=1\land\pc_2=1 $$
    
    Viewing states as $(\turn,\pc_1,\pc_2)$, then we can draw the following diagram for the transition system, initial states are bold:
    
    \centerline{\drawdiagram{%
                 &$\bf(0,1,1)$&&$\bf(1,1,1)$\cr
        $(0,1,2)$&$(0,2,1)$&&$(1,1,2)$&$(1,2,1)$\cr
        $(0,2,2)$&$(0,3,1)$&&$(1,1,3)$&$(1,2,2)$\cr
                 &$(0,3,2)$&&$(1,2,3)$\cr
    }{%
        \diagarrow{from={1,2}, to={2,1}, color=rgb{.1 .65 .1}, text=$t_3$, x distance=-.4cm, slide=.4}%t3
        \diagarrow{from={1,2}, to={2,2}, color=rgb{.1 .65 .1}, text=$t_0$, x distance=-.25cm}%t0
        \diagarrow{from={2,1}, to={3,1}, color=rgb{.1 .65 .1}, text=$t_0$, x distance=-.25cm}%t0
        \diagarrow{from={2,2}, to={3,1}, color=rgb{.1 .65 .1}, text=$t_3$, x distance=-.4cm, slide=.4}%t3
        \diagarrow{from={2,2}, to={3,2}, color=rgb{.1 .65 .1}, text=$t_1$, x distance=-.25cm}%t1
        \diagarrow{from={3,1}, to={4,2}, color=rgb{.1 .65 .1}, text=$t_1$, x distance=-.4cm, slide=.6}%t1
        \diagarrow{from={3,2}, to={4,2}, color=rgb{.1 .65 .1}, text=$t_3$, x distance=-.25cm}%t3
        %                              , color=rgb{.1 .65 .1}}
        \diagarrow{from={1,4}, to={2,4}, color=rgb{.1 .65 .1}, text=$t_3$, x distance=.25cm}%t3
        \diagarrow{from={1,4}, to={2,5}, color=rgb{.1 .65 .1}, text=$t_0$, x distance=.4cm, slide=.4}%t0
        \diagarrow{from={2,5}, to={3,5}, color=rgb{.1 .65 .1}, text=$t_3$, x distance=.25cm}%t3
        \diagarrow{from={2,4}, to={3,5}, color=rgb{.1 .65 .1}, text=$t_0$, x distance=.4cm, slide=.4}%t0
        \diagarrow{from={2,4}, to={3,4}, color=rgb{.1 .65 .1}, text=$t_4$, x distance=.25cm}%t4
        \diagarrow{from={3,5}, to={4,4}, color=rgb{.1 .65 .1}, text=$t_4$, x distance=.5cm, slide=.6}%t4
        \diagarrow{from={3,4}, to={4,4}, color=rgb{.1 .65 .1}, text=$t_0$, x distance=.25cm}%t0
        %                              , color=rgb{.1 .65 .1}}
        \diagarrow{from={3,2}, to={1,4}, color=rgb{.1 .65 .1}, text=$t_2$, x distance=.4cm, slide=.6}%t2
        \diagarrow{from={3,4}, to={1,2}, color=rgb{.1 .65 .1}, text=$t_5$, x distance=-.4cm, slide=.6}%t2
        \diagarrow{from={4,2}, to={2,4}, color=rgb{.1 .65 .1}, text=$t_2$, x distance=-.4cm, slide=.4}%t2
        \diagarrow{from={4,4}, to={2,2}, color=rgb{.1 .65 .1}, text=$t_5$, x distance=.4cm, slide=.4}%t5
    }}
    
    Now notice that we do indeed have mutual exclusion, where formally this means always $\neg(\pc_1=3\land\pc_2=3)$.
    Furthermore we have that if $\turn=0$ then eventually $\turn=1$, to prove this we must go through every possible execution which starts with $\turn=0$ and to show that eventually $\turn=1$.

    Say instead of implementing {\sf wait} via a lock (eg. mutex), we utilize busy waiting, adding the following two transitions:
    $$ t_1'\colon(\pc_1=2\land\turn=1)\longto\pc_1\coloneqq2,\qquad t_4'\colon(\pc_2=2\land\turn=0)\longto\pc_2\coloneqq2 $$
    then we no longer have that if $\turn=0$ then eventually $\turn=1$.
    For example $(0,1,1)\to(0,1,2)$ and then $(0,1,2)$ is extended forever via $t_4'$.

\eexam

In the above example, the focus was more on the states and the possible transitions between them rather than the explicit content of each transition.
We can generalize this idea to the concept of {\it state spaces}:

\bdefn

    A {\emphcolor state space} is a triplet $(S,\Delta,I)$, where $S$ is a set of states, $\Delta\subseteq S\times S$ is the transition relations, and $I\subseteq S$ are the initial transitions.
    This defines a graph, called an {\emphcolor automaton}.
    A {\emphcolor run} of the automaton is a sequence $s_0s_1s_2\dots$ such that $s_0\in I$ is an initial state and for every $i\geq0$, $(s_i,s_{i+1})\in\Delta$.
    Such a run must be maximal, meaning it is either infinite or it reaches a state with no successor.

    Sometimes we give names to the transitions in $\Delta$ in which case our state space becomes $(S,\Delta,\Sigma,I)$ where $\Delta$ now is a subset of $S\times\Sigma\times S$.
    Every transition gets its own name, so if $(s,\alpha,s'),(r,\alpha,r')\in\Delta$ then $s=r$ and $s'=r'$.

\edefn

In particular, a transition system defines a state space where $S$ is the set of all states, which are valuations $V\longto{\cal S}$.
Then $(s,s')\in\Delta$ if and only if there is a transition $t$ enabled at $s$ such that $s'=t(s)$.
And $I$ is the set of states which satisfy the initial condition, $I=\set{s\in S}[s\vDash\Theta]$.

Suppose we have $n$ concurrent processes, each with a variable $v_i$ and the transitions
$$ t^i_1\colon v_i=1\longto v_i\coloneqq2,\quad t^i_2\colon v_i=2\longto v_i\coloneqq3,\quad t^i_3\colon v_i=3\longto v_i\coloneqq1 $$
in other words, if $v_i$ is $1$, then it is $2$, then it is $3$, then it is $1$.
Since this is a concurrent system, we must combine these states together, and then we get that the number of global states becomes $3^n$ (each state is $(v_1,\dots,v_n)$ and each $v_i$ can take on three
values).
This is called {\it combinatorial explosion}: a relatively simple transition system becomes exponentially larger with the growth of concurrent processes.

Let us examine this above example more closely: notice how we took multiple transition systems and combined them into one.
We will define this notion formally: suppose we have a transition system whose transitions $T$ is constructed from {\it local} components $T_1,\dots,T_n$.
Here, each $T_i$ refers to a local component of the system, be it a concurrent process, a variable, or whatever.
For each $T_i$ we also have a set of {\it transition names} $\Sigma_i$ and a {\it labelling function} which is a bijection $L_i\colon T_i\longto\Sigma_i$.
Importantly while the $T_i$s are disjoint, $\Sigma_i$ need not be.

If two transitions have the same name, then we execute them together.
Formally, from each global state $s$, we can execute all the transitions with the name $d$ (meaning $L_i(t)=d$) provided {\it all of them} are enabled at $s$.
So suppose we have the transitions $t_i\colon p_i\longto(v_1^i,\dots,v_{n_i}^i)\coloneqq(e_1^i,\dots,e_{n_i}^i)$ for $1\leq i\leq k$ such that $L_i(t_i)=d$ for all $1\leq i\leq k$.
Then the resulting transition is
$$ \bigl(p_1\land\cdots\land p_k\bigr)\longto(v_1^1,\dots,v_{n_1}^1,\dots,v_1^k,\dots,v_{n_k}^k)\coloneqq(e_1^1,\dots,e_{n_1}^1,\dots,e_1^k,\dots,e_{n_k}^k) $$

Now suppose that each local component can be represented as a local state space $G_i=(S_i,\Sigma_i,\Delta_i,I_i)$ which corresponds to the local component $T_i$.
We assume that the set of local states $S_i$ are disjoint, but $\Sigma_i$ need not be.
If the label $\alpha$ appears in both $\Sigma_i$ and $\Sigma_j$, then $G_i$ and $G_j$ must be synchronized to perform $\alpha$ at the same time.

We define the operator $\circ$ to combine two state spaces $G_1$ and $G_2$ as follows: let $G_1\circ G_2=(S,\Sigma,\Delta,I)$ as follows:
\benum
    \item $S=S_1\times S_2$, each state is a pair of a state from $G_1$ and $G_2$,
    \item $\Sigma=\Sigma_1\cup\Sigma_2$, the transition names include all the names in both $G_1$ and $G_2$,
    \item The set of transitions $\Delta$ is the union of the following three sets:
        \benum
            \item $\set{\bigl((s,r),\alpha,(s',r)\bigr)}[(s,\alpha,s')\in\Delta_1,\,\alpha\in\Sigma_1\setminus\Sigma_2,\,r\in S_2]$.
                In this case, we have a transition $(s,\alpha,s')$ in $G_1$ with no transition of the same name in $\Sigma_2$, so we transition from $(s,r)$ to $(s',r)$, leaving the state in $G_2$
                unchanged.
            \item $\set{\bigl((s,r),\beta,(s,r')\bigr)}[(r,\beta,r')\in\Delta_2,\,\beta\in\Sigma_2\setminus\Sigma_1,\,s\in S_1]$.
                This is similar to the previous set, but for $G_2$ instead of $G_1$.
            \item $\set{\bigl((s,r),\gamma,(s',r')\bigr)}[(s,\gamma,s')\in\Delta_1,\,\gamma\in\Sigma_1\cap\Sigma_2,\,(r,r')\in\Delta_2]$.
                Here, we have a transition in both $G_1$ and $G_2$, so the transition is done simultaneously.
        \eenum
\eenum

So for example, the following two state spaces combine together to give

\medskip
\hbox to\hsize{\hfil
    \drawdiagram{%
        $\bf s_1$\cr
        \cr
        $s_2$\cr
    }{%
        \diagarrow{from={1,1}, to={3,1}, x off=-.25cm, text=$\alpha$, x distance=-.5cm}
        \diagarrow{from={3,1}, to={1,1}, x off=.25cm, text=$\beta$, x distance=.5cm}
    }\kern.75cm$\circ$\kern.75cm
    \drawdiagram{%
        $\bf r_1$\cr
        \cr
        $r_2$\cr
    }{%
        \diagarrow{from={1,1}, to={3,1}, x off=-.25cm, text=$\beta$, x distance=-.5cm}
        \diagarrow{from={3,1}, to={1,1}, x off=.25cm, text=$\gamma$, x distance=.5cm}
    }\kern1cm$=$\kern.5cm
    \drawdiagram{%
        &$\bf(s_1,r_1)$\cr
        $(s_1,r_2)$&&$(s_2,r_1)$\cr
        &$(s_2,r_2)$\cr
    }{
        \diagarrow{from={1,2}, to={2,3}, text=$\alpha$, slide=.4, x distance=.4cm}
        \diagarrow{from={2,1}, to={1,2}, text=$\gamma$, slide=.6, x distance=-.4cm}
        \diagarrow{from={2,3}, to={2,1}, text=$\beta$, y distance=.25cm}
        \diagarrow{from={2,1}, to={3,2}, text=$\alpha$, slide=.6, x distance=-.4cm}
        \diagarrow{from={3,2}, to={2,3}, text=$\gamma$, slide=.3, x distance=.5cm}
    }\hfil
}

\vfill\break

\section{Specification Formalisms}

We now introduce language which allows us to formally discuss properties of systems and their executions.
By doing so, we can prove these properties formally and without room for interpretative error.

Let ${\cal L}$ be a set logic (either propositional or first-order),
${\cal S}$ will be an ${\cal L}$-structure, but in general we will refrain from mentioning it instead; we will write $\vDash$ in place of $\vDash^{\cal S}$.

\bdefn

    {\emphcolor Linear temporal logic} (abbreviated LTL) is an instance of modal logic.
    It is defined over ${\cal L}$ recursively as follows:
    \benum
        \item Every formula of ${\cal L}$ is also a formula of LTL,
        \item if $\phi$ and $\psi$ are LTL formulas, so too are $\neg\phi,(\phi\land\psi),\bigcirc\phi,\diamond\phi,\square\phi,\phi\U\psi,\phi\V\psi$.
    \eenum
    An LTL formula is interpreted over an infinite sequence of states $\xi=x_0x_1x_2\dots$.
    Let us write $\xi^k$ for the suffix $\xi^k\coloneqq x_kx_{k+1}\dots$, then we define
    \benum
        \item if $\phi\in{\cal L}$ then $\xi^k\vDash\phi$ if $x_k\vDash\phi$ in ${\cal L}$,
        \item $\xi^k\vDash\neg\phi$ if $\xi^k\nvDash\phi$,
        \item $\xi^k\vDash\phi\land\psi$ if $\xi^k\vDash\phi$ and $\xi^k\vDash\psi$,
        \item $\xi^k\vDash\bigcirc\phi$ if $\xi^{k+1}\vDash\phi$,
        \item $\xi^k\vDash\diamond\phi$ if there is an $i\geq k$ such that $\xi^i\vDash\psi$,
        \item $\xi^k\vDash\square\phi$ if $\xi^i\vDash\psi$ for every $i\geq k$,
        \item $\xi^k\vDash\phi\U\psi$ if there is an $i\geq k$ such that $\xi^i\vDash\psi$ and for all $k\leq j<i$, $\xi^j\vDash\psi$,
        \item $\xi^k\vDash\phi\V\psi$ if for every $i\geq k$, $\xi^i\vDash\psi$; or for some $i\geq k$, $\xi^i\vDash\phi$ and for every $k\leq j\leq i$, $\xi^j\vDash\psi$.
    \eenum

\edefn

\def\sqwd{.5cm}
\def\sqht{.5cm}
\def\sqstroke{.3pt}
\def\sq{\hbox{$\vcenter to\sqht{\hrule height\sqstroke\hbox to\sqwd{\vrule width\sqstroke height\dimexpr\sqht-\sqstroke * 2\relax\hfil\vrule width\sqstroke}\hrule height\sqstroke}$}}
\def\tosq#1{\hbox to\sqwd{\hss#1\hss}}

Intuitively we can explain the new operators as follows:
\benum
    \global\enumcount=3
    \item $\bigcirc$ is the {\it nexttime} operator: $\bigcirc\phi$ holds in the sequence $x_kx_{k+1}\dots$ if $\phi$ holds starting from the next state $x_{k+1}$.
        Visually we can view it like so:

        \medskip
        \moveright\leftskip\hbox to\dimexpr\hsize-\leftskip\relax{\rlap{\kern\sqwd\tosq{$\phi$}}\leaders\sq\hfill\null}
    \item $\diamond$ is the {\it eventually} operator: $\diamond\phi$ holds in the sequence $\xi$ provided there exists a suffix in which $\phi$ holds.
        Visually:

        \medskip
        \moveright\leftskip\hbox to\dimexpr\hsize-\leftskip\relax{\rlap{\kern\dimexpr\sqwd * 10\relax\tosq{$\phi$}}\leaders\sq\hfill\null}
    \item $\square$ is the {\it always} operator: $\square\phi$ holds in the sequence $\xi$ provided it holds in every suffix of $\xi$.
        Visually:

        \medskip
        \moveright\leftskip\hbox to\dimexpr\hsize-\leftskip\relax{\leaders\tosq{$\phi$}\hfill\kern-\dimexpr\hsize-\leftskip\relax\leaders\sq\hfill\null}
    \item $\U$ is the {\it until} operator: $\phi\U\psi$ holds in the sequence $\xi$ if $\psi$ holds eventually and $\phi$ holds up until then:

        \medskip
        \moveright\leftskip\hbox to\dimexpr\hsize-\leftskip\relax{\leaders\tosq{$\phi$}\hskip\dimexpr\sqwd * 20\relax\tosq{$\psi$}\kern-\dimexpr\sqwd * 21\relax\leaders\sq\hfill\null}
    \item $\V$ is the {\it release} operator: $\phi\V\psi$ holds if $\psi$ either holds forever, or up until some point when both $\phi$ and $\psi$ hold.
        The reasoning for the name is that $\phi$ ``releases'' $\psi$ from having to hold for forever.

        \medskip
        \moveright\leftskip\hbox to\dimexpr\hsize-\leftskip\relax{\leaders\tosq{$\psi$}\hfill\kern-\dimexpr\hsize-\leftskip\relax\leaders\sq\hfill\null}
        \medskip

        \centerline{Or:}
        \medskip
        \moveright\leftskip\hbox to\dimexpr\hsize-\leftskip\relax{%
            \def\sqht{1cm}\leaders\tosq{$\psi$}\hskip\dimexpr\sqwd * 20\relax\tosq{$\stackmath{\psi\cr\noalign{\kern.2cm}\phi}$}\kern-\dimexpr\sqwd * 21\relax\leaders\sq\hfill\null}
\eenum

Notice that $\diamond$ is a special case of $\U$: $\diamond\phi\equiv\true\U\phi$.
And $\square$ is a special case of $\V$: $\square\phi\equiv\false\V\phi$ (since $\false$ can never relase $\phi$).
$\square$ and diamond are also related through $\neg\square\phi\equiv\diamond\neg\phi$.

We can also relate $\U$ and $\V$ by $\neg(\phi\V\psi)\equiv(\neg\phi)\U(\neg\psi)$.
We will prove this directly from definition:
$$ \xi^k\vDash\phi\V\psi \iff \bigl((\forall i\geq k)\xi^i\vDash\psi\bigr)\lor\bigl((\exists j\geq k)(\forall k\leq i<j)\xi^i\vDash\psi\land\psi^j\vDash\phi\bigr) $$
and so
$$ \xi^k\vDash\neg(\phi\V\psi) \iff \bigl((\exists i\geq k)\xi^i\vDash\neg\psi\bigr)\land\bigl((\forall j\geq k)(\exists k\leq i<j)\xi^i\vDash\neg\psi\lor\psi^j\vDash\neg\phi\bigr) $$
So at every $j\geq k$, either $\neg\phi$ or $\neg\psi$ holds, and eventually $\neg\psi$ holds.
This just means that $\neg\phi$ holds until $\neg\psi$ holds, ie. $(\neg\phi)\U(\neg\psi)$.

Thus, we could've defined LTL with only the operators $\neg,\land,\bigcirc,\U$ (in other words, these form a {\it complete bundle}).

We can combine operators: for example $\square\diamond\phi$ means that always, $\phi$ eventually happens; or equivalently $\phi$ happens infinitely many times.
$\diamond\square\phi$ means that at some point, $\phi$ will hold forever.
$\bigcirc\bigcirc\phi$ means that $\phi$ holds after two steps.
Notice that $\xi\vDash\diamond\phi$ if and only if there exists some $n$ such that $\xi\vDash\bigcirc^n\phi$ ($\bigcirc^n$ meaning $\bigcirc\cdots\bigcirc$ $n$ times).

Let $P$ be a system which has multiple executions, then we write $P\vDash\phi$ if $\xi\vDash\phi$ for all executions $\xi$ of $P$.
Importantlu $P\nvDash\phi$ does not imply $P\vDash\neg\phi$, since one execution not satisfying $\phi$ does not mean all executions don't satisfy $\phi$.

\bexam

    \def\mal{{\it malfunctioned}}\def\ext{{\it extended}}
    Let us consider a simple model of a spring.
    The spring can be in one of the following three states: $\set{{\it initial},{\it extended},{\it extended\ and\ malfunctioned}}$ which we denote $s_1,s_2,s_3$ respectively.
    So our propositional variables are $\PV=\set{\ext,\mal}$.
    Since $s_1$ is neither extended nor malfunctioned, $s_1\vDash\neg\ext\land\neg\mal$, $s_2\vDash\ext\land\neg\mal$, $s_3\vDash\ext\land\mal$.

    We can transition from $s_1$ to $s_2$ via pulling the spring, and releasing the spring can either transition to $s_1$ or to $s_3$.
    From $s_3$ we transition only to $s_3$.

    This system has an infinite number of executions, for example
    $$ \eqalign{
        \xi_0 &= s_1s_2s_1s_2s_3s_3s_3s_3\cdots\cr
        \xi_1 &= s_1s_2s_3s_3s_3s_3s_3s_3\cdots\cr
        \xi_2 &= s_1s_2s_1s_2s_1s_2s_1s_2\cdots\cr
    } $$
    Let us investigate $\xi_0$:
    \benum
        \item $\xi_0\nvDash\ext$ since $\ext$ is a formula of the underlying logic of the LTL and so $\xi_0$ satisfies $\ext$ if and only if its first state, $s_1$, does.
            It does not.
        \item $\xi_0\vDash\bigcirc\ext$ (``nexttime extended'') since $\xi_0\vDash\bigcirc\ext\iff\xi_0^1\vDash\ext\iff s_2\vDash\ext$ which it does.
        \item $\xi_0\nvDash\bigcirc\bigcirc\ext$ (``nexttime nexttime extended'') since $\xi_0^2$ begins with $s_1$ which does not satisfy $\ext$.
        \item $\xi_0\vDash\diamond\ext$ (``eventually extended'') since eventually the spring is extended (this is since $\xi_0\vDash\bigcirc\ext$).
        \item $\xi_0\nvDash\square\ext$ (``always extended'') since the spring is not always extended.
        \item $\xi_0\vDash\diamond\square\ext$ (``eventually always extended'') since eventually the spring remains in $s_3$ where it is extended.
        \item $\xi_0\nvDash(\neg\ext)\U\mal$ (``not extended until malfunctioned'') since the spring is not extended, then extended and not malfunctioned.
    \eenum

    Let us now investigate the system $P$ as a whole:
    \benum
        \item $P\vDash\diamond\ext$ since for the spring to not extend, it would need to forever remain in $s_1$, which is impossible.
        \item $P\vDash\square(\neg\ext\to\bigcirc\ext)$ which means that always, if the spring is not extended then the next time it is.
            This is since in order for the spring to not be extended, it must be in $s_1$, which means that the next time it is in $s_2$, extended.
        \item $P\nvDash\diamond\square\ext$, since $\xi_2$ is a counterexample: here we have that we never are only extended, in other words $\xi_2\vDash\square\diamond\neg\ext$.
        \item $P\nvDash\neg\diamond\square\ext$, since $\xi_0$ is a counterexample: here we have that eventually we are only extended.
        \item $P\nvDash\square(\ext\to\bigcirc\neg\ext)$ since it is possible to go from extended to extended ($s_2$ to $s_3$).
            The only sequence in which this is true is $\xi_2$.
    \eenum

\eexam

We can form a Hilbert calculus to axiomatize LTL with respect to a system $P$.
To form it we adjoin to the Hilbert calculus of ${\cal L}$ (which is either first-order or propositional, usually propositional. But importantly these axioms now range over all LTL formulas, not just
formulas in ${\cal L}$) the following eight axioms:

\medskip
{\tabskip=0pt plus 1fil
\halign to\hsize{(#)\hfil\tabskip=.25cm&$#$\hfil\tabskip=.5cm&(#)\hfil\tabskip=.25cm&$#$\hfil\tabskip=.5cm&(#)\hfil\tabskip=.25cm&$#$\hfil\tabskip=0pt plus 1fil\cr
    A1&\neg\diamond\phi\oto\square\neg\phi&A2&\square(\phi\to\psi)\to(\square\phi\to\square\psi)&A3&\square\phi\to(\phi\land\bigcirc\square\phi)\cr
    A4&\bigcirc\neg\phi\oto\neg\bigcirc\phi&A5&\bigcirc(\phi\to\psi)\to(\bigcirc\phi\to\bigcirc\psi)&A6&\square(\phi\to\bigcirc\phi)\to(\phi\to\square\phi)\cr
    A7&(\phi\U\psi)\oto\bigl(\psi\lor(\phi\land\bigcirc(\phi\U\psi))\bigr)&A8&(\phi\U\psi)\to\diamond\psi\cr
}}
\medskip

Here we take $\V$ as defined by $(\phi\V\psi)\coloneqq\neg((\neg\phi)\V(\neg\psi))$.
We use the following rule of reference (temporal generalization) as well as MP
$$ \vbox{\halign{\hfil$#$\hfil\cr\phi\cr\noalign{\kern2pt\hrule\kern2pt}\ \square\phi\ \cr}} $$
meaning that if $\phi$ then $\square\phi$ (notice that here we do not have an initial state, and hence we obtain the soundness of generalization).

\bexam

    Let us look at a model of a traffic light, which can transition between colors as follows:

    \def\gr{{\it green}}\def\ye{{\it yellow}}\def\re{{\it red}}
    \medskip
    \centerline{\drawdiagram{
        $\gr$&$\ye$&$\re$&\cr
    }{%
        \diagarrow{from={1,1}, to={1,2}, color=rgb{.1 .65 .1}}
        \diagarrow{from={1,2}, to={1,3}, color=rgb{.1 .65 .1}}
        \diagarrow{from={1,3}, to={1,1}, curve=1cm, origin orient={xcenter, bottom}, dest orient={xcenter, bottom}, color=rgb{.1 .65 .1}}
    }}
    \kern.5cm

    The traffic light is only ever in one color, which can be expressed in LTL as:
    $$ \square\bigl(\neg(\gr\land\ye)\land\neg(\ye\land\re)\land\neg(\re\land\gr)\land(\gr\lor\ye\lor\re)\bigr) $$
    Specifying the transition of colors can be done via
    $$ \square\bigl((\gr\U\ye)\lor(\ye\U\re)\lor(\re\U\gr)\bigr) $$
    Now suppose we alter the state graph to be

    \medskip
    \centerline{\drawdiagram{
        $\gr$&$\ye$&$\re$&$\ye$&\cr
    }{%
        \diagarrow{from={1,1}, to={1,2}, color=rgb{.1 .65 .1}}
        \diagarrow{from={1,2}, to={1,3}, color=rgb{.1 .65 .1}}
        \diagarrow{from={1,3}, to={1,4}, color=rgb{.1 .65 .1}}
        \diagarrow{from={1,4}, to={1,1}, curve=1cm, origin orient={xcenter, bottom}, dest orient={xcenter, bottom}, color=rgb{.1 .65 .1}}
    }}
    \kern.5cm

    Specifying the colors is now harder, since $\square(\ye\to\ye\U\re)$ is no longer necessarily true.
    We could attemp $\square\bigl(((\gr\lor\re)\U\ye)\lor(\ye\U(\gr\lor\re))\bigr)$, but $(\gr\lor\re)\U\ye$ allows the light to switch between $\gr$ and $\re$ before turning $\ye$.
    A correct specification would be

    \medskip
    \centerline{\vbox{\halign{$#$\hfil&\hfil$#$&$#$\hfil\cr
        \square\bigl(&(\gr\to&(\gr\U(\ye\land(\ye\U\re))))\cr
        \land        &(\re\to&(\re\U(\ye\land(\ye\U\gr))))\cr
        \land        &(\ye\to&(\ye\U(\re\U\gr)))\bigr)\cr
    }}}
    \medskip

    The first line allows $\gr\to\ye\to\re$, the second allows $\re\to\ye\to\gr$.
    These two lines deal only with the case that the light begins on $\gr$ or $\re$, the third line deals with the case when it starts on $\ye$.

\eexam

\bye

