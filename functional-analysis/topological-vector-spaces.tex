\section{Topological Vector Spaces}

\subsection{Separation Properties}

Recall the definitions of vector, normed, and inner product spaces (or refer to the reference of your choice).
Also recall the definitions of topological and metric spaces.
A {\it complete\/} metric space is one where all Cauchy sequences converge.

\bdefn

    A {\emphcolor topological field} is a field endowed with a topology which agrees with its operations.
    Explicitly, the maps $\alpha,\beta\mapsto\alpha+\beta$; $\alpha,\beta\mapsto\alpha\beta$; $\alpha\mapsto-\alpha$; and $\alpha\mapsto\alpha^{-1}$ are continuous.

\edefn

We will focus mostly on the topological fields $\bb R$ and $\bb C$.

\bdefn

    A {\emphcolor topological vector space} is a vector space endowed with a topology which behaves well with its existing vector space structure.
    Formally, it is a vector space $\bV$ with a topology $\tau$ such that
    \benum
        \item every singleton in $\bV$ is closed;
        \item the vector space operations (vector addition and scalar multiplication) are continuous w.r.t.\ the topology.
    \eenum

\edefn

We can explicitly state what it means for addition (an operation $\bV\times \bV\to \bV$) and scalar multiplication (a mapping $F\times \bV\to \bV$) to be continuous.
Addition is continuous if and only if for every neighborhood $\U\subseteq \bV$ the set $\set{(x_1,x_2)}[x_1+x_2\in\U]$ is open.
This is equivalent to saying that there exists neighborhoods $\U_1,\U_2$ of $x_1,x_2$ such that $\U_1+\U_2\subseteq\U$, since then
$$ \U = \bigcup_{x_1+x_2\in\U}\U_{x_1}+\U_{x_2} \implies \set{(x_1,x_2)}[x_1+x_2\in\U] = \bigcup\U_{x_1}\times\bigcup\U_{x_2} $$
Similarly scalar multiplication $\cdot\colon F\times \bV\to \bV$ is continuous if for every neighborhood $\U$ of $\alpha x$, then there is some neighborhood $O$ of $\alpha$ and $\V$ of $x$ such that
$$ O\cdot\V\subseteq\c U $$

Let us define the mappings $T_a\colon x\mapsto a+x$ for $a\in X$ and $M_\lambda\colon x\mapsto\lambda x$ for $\lambda\in F$ nonzero.
Then $T_a$ and $M_\lambda$ are homeomorphisms over $X$.
They are both obviously bijective, with inverses $T_{-a}$ and $M_{1/\lambda}$ respectively.
Since $T_a$ is just the restriction of addition to $a$, it is continuous (similar for $M_\lambda$).
Thus they are homeomorphisms (since their inverses are also of the given form).

\bprop

    The following are equivalent:
    \benum
        \item $\U$ is open;
        \item $a+\U$ is open for some (all) $a$;
        \item $\lambda\U$ is open for some (all) $\lambda$.
    \eenum

\eprop

This is due to $T_a$ and $M_\lambda$ being homeomorphisms.

Thus the topology $\tau$ is entirely determined by any local base (a local base of point $x$ is a collection of neighborhoods of $x$ such that any neighborhood of $x$ contains some set in the base).
In particular the local base around $0$ (from now on, when we say local base we will mean around $0$).

\bdefn

    A metric on a vector space $\bV$ is {\emphcolor invariant} if $d(x+z,y+z)=d(x,y)$ for all $x,y,z\in \bV$.

\edefn

\blemm

    If $\U$ is a neighborhood of $0$, then there exists a symmetric neighborhood of $0$ $\V$ (i.e.\ $-\V=\V$) where $\V+\V\subseteq\U$.

\elemm

\bproof

    Since addition is continuous, the preimage of $\U$ is open and therefore contains $\V_1\times\V_2$; i.e.\ $\V_1+\V_2\subseteq\U$ for $\V_i$ open.
    Let
    $$ \V = \V_1\cap\V_2\cap(-\V_1)\cap(-\V_2), $$
    it is easy to see that $\V$ has the desired properties.
    \qed

\eproof

Note that there then exists a symmetric neighborhood $\V'$ of $\V$ such that $\V'+\V'\subseteq\V$ and thus $\sum_4\V'\subseteq\U$.
We can induct, and we get that for any $n$ there exists a symmetric neighborhood $\V$ of $0$ such that
$$ \sum_n\V\subseteq\U . $$

\bthrm

    Suppose $K,C\subseteq \bV$ where $K$ compact and $C$ closed and both sets are disjoint.
    Then there exists a neighborhood $\U$ of $0$ such that $K+\U$ and $C+\U$ are disjoint.

\ethrm

\bproof

    If $K$ is empty then this is trivial (since $K+\U$ is empty).
    Otherwise let $x\in K$, since $x\notin C$ this means there exists an open set $\V'_x$ such that $\V'_x$ is disjoint from $C$.
    Then $\V'_x-x$ is a neighborhood of $0$ and thus contains $\V_x+\V_x+\V_x$ for $\V_x$ symmetric.
    And so $x+\V_x+\V_x+\V_x$ is disjoint from $C$ (as a subset of $\V'_x$).
    Thus $x+\V_x+\V_x$ is disjoint from $C-\V_x=C+\V_x$.

    Since $K$ is compact, we can find finitely many points such that $K\subseteq\bigcup x_i+\V_{x_i}$.
    Let $\U=\bigcap\V_{x_i}$, then
    $$ K+\U \subseteq \bigcup (x_i+\V_{x_i}+\U) \subseteq \bigcup (x_i+\V_{x_i}+\V_{x_i}) , $$
    each of which is disjoint from $C+\V_{x_i}$ and thus $C+\U$.
    \qed

\eproof

Notice that
$$ K+\U = \bigcup_{k\in K}k+\U , $$
and is therefore open.
This means that compact and closed sets can be separated by open sets.
Since singletons are closed and compact, this means that topological vector spaces are Hausdorff.

\bthrm

    A topological vector space is Hausdorff.

\ethrm

\bthrm

    If $\b B$ is a local base for $\bV$, then every member of $\b B$ contains the closure of some member of $\b B$.

\ethrm

\bproof

    Let $\U\in\b B$; by the theorem above, there exists a neighborhood $\V$ (we can assume to be in $\b B$) of $0$ such that $\U^c+\V$ and $\set0+\V=\V$ are disjoint.
    Recall that if $X,Y$ are open and disjoint then $\overline X,Y$ are disjoint.
    Thus $\U^c\subseteq\U^c+\V$ and $\overline\V$ are disjoint, meaning $\overline\V\subseteq\U$.
    \qed

\eproof

\bprop

    Let $\bV$ be a topological vector space.
    \benum
        \item if $A\subseteq \bV$ then $\overline A=\bigcap(A+\U)$ (where $\U$ runs over all neighborhoods of $0$);
        \item if $A,B\subseteq \bV$ then $\overline A+\overline B\subseteq\overline{A+B}$
        \item if $Y\leq \bV$ (i.e.\ is a subspace) then $\overline Y\leq\bV$;
        \item if $C$ is a convex subset of $\bV$ then so are $\overline C$ and $\interior C$;
    \eenum

\eprop

\bproof

    \benum
        \item $x\in\overline A$ iff $x+\U\cap A\neq\emptyset$ for all $\U$ neighborhoods of $0$.
        This is iff $x\in A-\U$ for all such $\U$, and $-\U$ is a neighborhood of $0$ iff $\U$ is one, completing the proof.
        \item Let $x\in\overline A,y\in\overline B$ and $\U$ a neighborhood of $x+y$.
        Then there exists $x\in\U_1,y\in\U_2$ such that $\U_1+\U_2\subseteq\U$.
        Since $x\in\overline A$ and $y\in\overline B$, there are $a\in\U_1\cap A,b\in\U_2\cap B$, and so $a+b\in\U\cap(A+B)$.
        Thus $\U\cap(A+B)\neq\emptyset$.
        This means that $x+y\in\overline{A+B}$.
        \item Since $M_a$ is a homeomorphism for $a\neq0$ we have that $a\overline Y=\overline{aY}$ (and for $a=0$ this is true trivially).
        We then have by the previous point
        $$ a\overline Y + b\overline Y = \overline{aY} + \overline{bY} \subseteq \overline{aY+bY} = \overline Y $$
        as required.
        \item Notice that
        $$ t\overline C + (1-t)\overline C \subseteq \overline{tC + (1-t)C} \subseteq \overline C $$
        and since $\interior C\subseteq C$ and $C$ is convex, we have that $t\interior C+(1-t)\interior C\subseteq C$.
        The sum of two open sets is open, we have that $t\interior C+(1-t)\interior C\subseteq\interior C$ as required.
        \qed
    \eenum

\eproof

\bdefn

    When $F=\bb R$ or $\bb C$ (with the usual topology) and $\bV$ a $F$-topological vector space, a {\emphcolor balanced subset} of $\bV$ is a set $B$ such that $\alpha B\subseteq B$ for every $\abs\alpha\leq1$.
    And $E$ is {\emphcolor bounded} if for every neighborhood of $0$ $\U$ there is a number $s_{\U}$ such that $E\subseteq t\U$ for every $t>s_{\U}$.

\edefn

\bprop

    Let $\bV$ be a real or complex topological vector space.
    \benum
        \item If $B$ is balanced, so is $\overline B$.
        If $0\in\interior B$ then $\interior B$ is also balanced.
        \item If $E$ is bounded, so is $\overline E$.
    \eenum

\eprop

\bproof

    \benum
        \item If $0<\abs a<1$ then $M_a$ is a homeomorphism so $a\interior B=\interior{(aB)}$.
        And so $a\interior B=\interior{(aB)}\subseteq\interior B$ since $aB\subseteq B$.
        And if $0\in\interior B$ then $0\interior B=\set0\subseteq\interior B$.
        \item Let $\U$ be a neighborhood of $0$; by the previous theorem, $\overline\V\subseteq\U$ for some neighborhood $\V$ of $0$.
        Then $E\subseteq t\V$ for all $t>s_{\V}$ and then $\overline E\subseteq t\overline\V\subseteq t\U$.
        \qed
    \eenum

\eproof

\bnote

    From now on, all vector spaces under consideration are either real or complex.

\enote

\bthrm

    In a topological vector space $\bV$,
    \benum
        \item every neighborhood of $0$ contains a balanced neighborhood of $0$;
        \item every convex neighborhood of $0$ contains a balanced convex neighborhood of $0$.
    \eenum

\ethrm

\bproof

    \benum
        \item Let $\U$ be a neighborhood of $0$, since scalar multiplication is continuous the preimage of $\U$ under multiplication contains $B_r(0)\times\V$ where $\V$ is a neighborhood of $0$.
        This means that if $\abs a<r$ then $a\V\subseteq\U$.
        Let $\c W$ be the union of all such $a\V$, then $\c W$ is a balanced neighborhood of $0$ contained in $\U$.
        \item Let $A=\bigcap_{\abs a=1}a\U$ and $\c W$ be as in the previous point.
        Since $\c W$ is balanced, when $\abs a=1$ we have $a^{-1}\c W\subseteq\c W$ and so $\c W\subseteq a\c U$.
        Therefore $\c W\subseteq A$.
        Since $\c W$ is a neighborhood of $0$, this means that $\interior A$ is a neighborhood of $0$ as well, clearly contained in $\U$.
        Since $a\U$ is convex, $A$ is an intersection of convex sets and is therefore convex, and therefore so is $\interior A$.

        All that remains is to show that $\interior A$ is balanced.
        It is sufficient to show that $A$ is balanced.
        Let $0\leq r\leq 1$ and $\abs b=1$ then,
        $$ rb A = \bigcap_{\abs a=1}rba\U = \bigcap_{\abs a=1}ra\U . $$
        Since $a\U$ is a convex set containing $0$, we have $ra\U\subseteq aU$ (since $ra\U\subseteq ra\U+(1-r)a\U$).
        Thus $rb A\subseteq A$ as required.
        \qed
    \eenum

\eproof

Let us say that a local base $\b B$ is {\it balanced\/} if its members are balanced, and {\it convex\/} if its members are convex.

\bcoro

    \benum
        \item Every topological vector space has a balanced local base.
        \item Every topological vector space has a convex local base.
    \eenum

\ecoro

\bthrm

    Let $\U$ be a neighborhood of $0$ in $\bV$.
    \benum
        \item If $\set{r_i}_i$ is an increasing sequence to $\infty$, then
        $$ V = \bigcup_i r_i\U . $$
        \item Every compact subset of $\bV$ is bounded.
        \item If $\set{\delta_i}_i$ is a decreasing sequence to $0$ and $\U$ is bounded, then $\set{\delta_i\U}_i$ is a local base for $\bV$.
    \eenum

\ethrm

\bproof

    \benum
        \item Let $x\in\bV$, then $\V_x=\set{a\in F}[ax\in\U]$ is open (as the projection of the preimage of $\U$ under scalar multiplication).
        Furthermore, it contains $0$ and therefore $1/r_n$ for large enough $n$ (as it is open).
        Therefore $1/r_nx\in\U$ and so $x\in r_n\U$ as required.
        \item Let $\U$ be a balanced neighborhood of $0$.
        By the above point, $\set{n\U}$ is an open cover of $K$, and thus there exist $n_1,\dots,n_k$ such that $K\subseteq\bigcup_1^k n_i\U$.
        Since $\U$ is balanced, for $a<b$ we have $a\U\subseteq b\U$ since $a/b\U\subseteq\U$.
        Thus $K\subseteq n_k\U$, and we can take $s=n_k$.
        And for $\U$ arbitrary, it contains a balanced neighborhood.
        \item Let $\c W$ be a neighborhood of $0$.
        Since $\U$ is bounded $\U\subseteq t\c W$ for $t>s$ and therefore $1/t\U\subseteq\c W$.
        Since $\delta_i$ is decreasing, there is a $\delta_n$ such that $\delta_n\U\subseteq\c W$ and therefore $\set{\delta_i\U}_i$ is a local base.
        \qed
    \eenum

\eproof

\subsection{Linear Mappings}

Note that if $\Lambda$ is a linear map, then it and its preimage preserve subspaces, convex sets, and balanced sets.

\bthrm

    The following are equivalent of a linear map $\Lambda\colon X\to Y$:
    \benum
        \item $\Lambda$ is continuous at $0$;
        \item $\Lambda$ is continuous everywhere;
        \item $\Lambda$ is uniformly continuous: for every $0\in\V\subseteq Y$ open there is a $0\in\U\subseteq X$ open where
        $$ x-y\in\U \iff \Lambda x-\Lambda y\in\V $$
    \eenum

\ethrm

\bproof

    Obviously $(2)\implies(1)$, so we show that $(1)\implies(2),(3)$ and $(3)\implies(1)$.

    Let $\U$ be a neighborhood of $\Lambda(x)\in Y$, then $\U-\Lambda(x)$ is a neighborhood of $0$ and so $\Lambda^{-1}(\U-\Lambda(x))$ is open.
    But this is equal to $\set{a\in X}[\Lambda a\in\U-\Lambda(x)]=\set{a\in X}[\Lambda(a+x)\in\U]=\Lambda^{-1}\U-x$.
    So $\Lambda^{-1}\U$ is open, and therefore $\Lambda$ is continuous.

    For $(1)\implies(3)$, let $\U=\Lambda^{-1}\V$ which is open as $\Lambda$ is continuous at $0$ (and $\Lambda0=0$).
    Now,
    $$ x-y\in\U \iff \Lambda x-\Lambda y\in\V $$
    This also shows $(3)\implies(1)$.
    \qed

\eproof

\bthrm

    Let $\Lambda$ be a non-constant linear functional $X\to F$.
    Then the following are equivalent:
    \benum
        \item $\Lambda$ is continuous;
        \item The kernel $\ker\Lambda$ is closed;
        \item $\ker\Lambda$ is not dense in $X$;
        \item $\Lambda$ is bounded in some neighborhood $\U$ of $0$.
    \eenum

\ethrm

\bproof

    $(1)\implies(2)$ follows since $\ker\Lambda=\Lambda^{-1}\set0$ and $\set0$ is closed.

    $(2)\implies(3)$: since $\Lambda$ is non-constant, $\ker\Lambda\neq X$ and is closed.
    Since it is closed and non-trivial, it cannot be dense.

    $(3)\implies(4)$: so $\ker\Lambda^c$ has non-empty interior.
    We know then that there exists a balanced neighborhood $\U$ of $0$ and $x\in X$ such that $x+\U$ and $\ker\Lambda$ are disjoint.
    (This is because we can have a local basis of balanced neighborhoods.)
    Now, $\Lambda\U$ is a balanced subset of $F$, which means that if it is unbounded $\Lambda\U=F$.
    But then there exists a $y\in\U$ such that $\Lambda y=-\Lambda x$ and so $x+y\in x+\U$ and $\ker\Lambda$, in contradiction.
    \qed

\eproof

\subsection{Finite-Dimension Vector Spaces}

In this chapter we focus on finite-dimension vector spaces.
We endow upon $\bb R^n$ and $\bb C^n$ the topologies induced by their euclidean norms (which is equivalent to the product topology).

\blemm

    If $X$ is a complex topological vector space, and $f\colon\bb C^n\to X$ is linear, then it is continuous.

\elemm

\bproof

    Let $\set{e_1,\dots,e_n}$ form the standard basis for $\bb C^n$.
    Note that
    $$ f(z) = \sum_{i=1}^n\pi_i(z)f(e_i) , $$
    where $\pi_i$ are the projections $\bb C^n\to\bb C$.
    Projections are continuous, and thus $f$ is the linear combination of continuous functions and is therefore continuous.
    \qed

\eproof

\bthrm

    If $Y$ is an $n$-dimensional subspace of a complex topological vector space $X$, then
    \benum
        \item every vector space isomorphism of $\bb C^n$ and $Y$ is a homeomorphism;
        \item $Y$ is closed.
    \eenum

\ethrm

Note that $(1)$ is not a trivial consequence of the above lemma: an isomorphism $\bb C^n\to Y$ is continuous, but it is not obvious that its inverse $Y\to\bb C^n$ is!

\bproof

    Let $S=\partial B_1(0)\subseteq\bb C^n$ (i.e.\ $z\in S\iff\sum_i\abs z_i^2=1$).
    Now suppose $f\colon\bb C^n\to Y$ is an isomorphism.
    By the above lemma is continuous, and therefore since $S$ is compact so is $K=fS$.
    Since $f$ is injective and $0\notin S$, this means that $0\notin K$, and so there is a balanced neighborhood of $0$ which does not intersect $K$, let this be $\U$.
    Define
    $$ E = f^{-1}(\U\cap Y) , $$
    this is disjoint from $S$ (since $x\in E\cap S$ would imply $fx\in\U$ and $fx\in K$).
    Since $f$ is linear, and $\U\cap Y$ is balanced, so is $E$.
    Since balanced subsets of $\bb C^n$ are connected, $E$ is connected.
    Since $0\in E$, $E\subseteq B_1(0)$ (since otherwise $fE$ would have an element of $K$).

    Now, $f^{-1}$ is a tuple of linear functionals, each of which is bounded around $0$ and therefore are continuous.
    Therefore $f$ is a homeomorphism.

    For $(2)$, let $p\in\overline Y$, and have $f$ and $\U$ has above.
    Then for some $t>0$, $p\in t\U$, so $p$ is in the closure of $Y\cap t\U$.
    Since $\U\subseteq fB$ we have that $\overline{Y\cap t\U}\subseteq\overline{f(tB)}\subseteq\overline{f(t\overline B)}$.
    But $\overline B$ is compact and therefore so is $f(t\overline B)$.
    Thus $p\in f(t\overline B)$.
    And $f(t\overline B)\subseteq Y$, as required.
    \qed

\eproof

\bthrm

    Every locally compact topological vector space has finite dimension.

\ethrm

\bproof

    Since $X$ is locally compact, $0\in X$ has a neighborhood $\U$ whose closure is compact.
    Thus $\U$ is bounded and therefore $2^{-n}\U$ form a local basis for $X$.
    Compactness of $\overline\U$ shows that there exist $x_1,\dots,x_n\in X$ such that $\overline\U\subseteq\bigcup_ix_i+1/2\U$.
    Let $Y=\gen{x_1,\dots,x_n}$, and so $Y$ has dimension $\leq n$ and by the previous theorem is closed.

    So we have $\overline\U\subseteq\bigcup_ix_i+1/2\U\subseteq Y+1/2\U$.
    This is open so $\U\subseteq Y+1/2\U$.
    Since $\lambda Y=Y$ for $\lambda\neq0$ we see that $\frac12\U\subseteq Y+1/4\U$ and so $\U\subseteq Y+1/4\U$.
    Continuing we get
    $$ \U \subseteq \bigcap_{n=1}^\infty(Y+2^{-n}\U) . $$
    Now, $\set{2^{-n}\U}$ is a local base, and so $\U\subseteq\overline Y=Y$ (since $Y$ is closed).
    But then $n\U\subseteq Y$ for all $n>0$, and since $\bigcup_nn\U=X$ we see that $X=Y$, and so $X$ is finite-dimensional.
    \qed

\eproof

\bdefn

    A topological vector space has the {\emphcolor Heine-Borel property} if compactness is equivalent to being closed and bounded.

\edefn

\bthrm

    If $X$ is a locally bounded topological vector space with the Heine-Borel property, then $X$ is finite-dimensional.

\ethrm

\bproof

    By assumption, $0\in X$ has a bounded neighborhood $\U$.
    Therefore $\overline\U$ is bounded too, and closed, and therefore compact.
    So $0$ has a compact neighborhood, and therefore $X$ is locally compact so we can apply the above theorem.
    \qed

\eproof

