\bdefn

    An {\emphcolor $n$-dimensional compact manifold} is a compact Hausdorff space $M$ such that every point has a neighborhood homeomorphic to an open ball in ${\bb R}^n$.
    A $2$-dimensional manifold is also called a {\emphcolor surface}.

    An {\emphcolor $n$-dimensional compact manifold with boundary} is a compact Hausdroff space $M$ such that every point has a neighborhood homeomorphic to either an open ball or half-open ball in
    ${\bb R}^n$.
    The half-open unit ball (to which all half-open balls are homeomorphic to) is $H^n\coloneqq\set{x\in{\bb R}^n}[\abs x<1,x_n\geq0]$.
    A manifold without boundary is called {\emphcolor closed}.

\edefn

For example the torus $T=S_1\times S_1$ is a surface.

For two manifolds $M_1$ and $M_2$ we define (in a hand-wavy fashion) their connected sum.
Simply choose embeddings $i_1\colon D^n\longto M_1$ ($D^n$ is the closed $n$-ball) and $i_2\colon D^n\longto M_2$.
Then define the connected sum to be
$$ M_1\#M_2\coloneqq\bigl(M_1\setminus i_1(S^{n-1})\amalg(M_2\setminus i_2(S^{n-1})\bigr)/{\sim} $$
where $i_1(x)\sim i_2(x)$.
Visually, remove an open ball $B^n$ from $M_1$ and $M_2$ and then glue the boundaries together.
A non-trivial fact is that this is indeed a manifold and is unique up to isomorphism (in choice of embeddings, if the spaces are path-connected).

We write $nM$ in place of $M\#\cdots\#M$ ($n$ times).

\subsection{The Real Projective Plane}

\bdefn

    Define the {\emphcolor real projective plane} to be $\slfrac{S^2}{v\sim-v}$ (the quotient space of the $2$-sphere obtained by identifying antipodal points).
    This is denoted ${\bb R}P^2$.
    In general ${\bb R}P^n=\slfrac{S^n}{v\sim-v}$.

\edefn

The real projective plane is equivalently defined by $\slfrac{H^n}{v\sim-v}$ where $v_n=0$.

\subsection{CW Complexes}

\bdefn

    Suppose $X$ and $Y$ are two topological spaces and $f\colon A\longto Y$ with $A\subseteq X$.
    We can {\emphcolor attach} $X$ to $Y$ along $A$, denoted $X\amalg_fY$ by taking $\slfrac{X\amalg Y}{x\sim f(x)}$.
    We can generalize this to attaching $\set{X_i}_{i\in I}$ to $Y$ naturally: $Y\coprod_{i\in I,f_i}X_i$ by $\slfrac{Y\coprod_{i\in I}X_i}{x\sim f_i(x)}$ where $x\sim f_i(x)$ means $x\sim f_i(x)$ for all
    $i\in I$.

\edefn

\bdefn

    A {\emphcolor CW Complex} is a space defined in $n$ steps as follows: suppose on by step $k$ we have constructed the skeleton $X^{k-1}$.
    Then we choose $f_k$ $k$-discs $D^k_1,\dots,D^k_{f_k}$ and attaching maps $\phi_i\colon S^{k-1}\longto X^{k-1}$ and define $X^k$ by attaching each of these discs to $X^{k-1}$ using the map $\phi_i$.
    Recall that $\partial D^k=S^{k-1}$ so we are attaching closed balls to $X^{k-1}$ along their boundary.
    Ie.
    $$ X^k = X^{k-1}\coprod_{1\leq i\leq k,\phi_i}D^k_i $$

\edefn

So for example, $S^n$ is a CW complex: take $X^0=D^0$ to be a single point and attach $D^n$ by identifying its boundary with a single point.
So the CW complex is $X^n=D^n/\partial D^n=D^n/S^{n-1}\cong S^n$ ($X/A$ means $X/{\sim}$ where $a\sim b$ for $a,b\in A$).

