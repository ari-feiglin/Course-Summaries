Suppose we have two groups $G$ and $H$, then we'd like some universal group $U$ along with two morphisms $g\colon U\longto G$ and $f\colon U\longto H$ such that for every $K$ and $\phi\colon K\longto G$,
$\psi\colon K\longto H$, there exists a unique $L\colon K\longto U$ such that the following diagram commutes

\centerline{\def\diagcolwidth{1cm}\def\diagrowheight{.75cm}\drawdiagram{
    &&$G$\cr
    $K$&$U$\cr
    &&$H$\cr
}{
    \diagarrow{from={2,1}, to={1,3}, curve=.6cm, text=$\phi$, y distance=.5cm, dest orient={left,ycenter}}
    \diagarrow{from={2,1}, to={3,3}, curve=-.6cm, text=$\psi$, y distance=-.6cm, dest orient={left,ycenter}}
    \diagarrow{from={2,1}, to={2,2}, text=$\exists!L$, y distance=.25cm}
    \diagarrow{from={2,2}, to={1,3}, text=$g$, y distance=.25cm, slide=.4}
    \diagarrow{from={2,2}, to={3,3}, text=$f$, y distance=-.25cm, slide=.4}
}}

Such a group would be unique up to isomorphism: if $U'$ is another, then our diagram becomes

\centerline{\def\diagcolwidth{1cm}\def\diagrowheight{.75cm}\drawdiagram{
    &&$G$\cr
    $U'$&$U$\cr
    &&$H$\cr
}{
    \diagarrow{from={2,1}, to={1,3}, curve=.6cm, text=$f'$, y distance=.5cm, dest orient={left,ycenter}}
    \diagarrow{from={2,1}, to={3,3}, curve=-.6cm, text=$g'$, y distance=-.6cm, dest orient={left,ycenter}}
    \diagarrow{from={2,1}, to={2,2}, text=$\exists!L$, y off=.1cm, y distance=.35cm}
    \diagarrow{from={2,2}, to={2,1}, text=$\exists!L'$, y off=-.1cm, y distance=-.35cm}
    \diagarrow{from={2,2}, to={1,3}, text=$g$, y distance=.25cm, slide=.4}
    \diagarrow{from={2,2}, to={3,3}, text=$f$, y distance=-.25cm, slide=.4}
}}

Taking $U=U'$ then we see that $L={\rm id}$ satisfies the conditions for $f,g$, so it must be unique.
And so now when $U\neq U'$, $L\circ L'$ also satisfies the conditions, so $L\circ L'={\rm id}$.
And similarly $L'\circ L={\rm id}$, meaning $L$ is an isomorphism between $U$ and $U'$.

One such universal construction is $U=G\times H$ with the projection maps $p_G$ and $p_H$, since if $\phi\colon K\longto G$ and $\psi\colon K\longto H$, then $p_G\circ L=\phi$ and $p_H\circ L=\psi$ if and
only if $L=(\phi,\psi)$.
So such a map exists, and it is unique, meaning $G\times H$ indeed has this universal property.
Any universal construction satisfying this is thus called a {\it product} of $G$ and $H$.

\medskip
Similarly let $G$ be a group, then we want an abelian group $U$ and a morphism $\rho\colon G\longto U$ such that for every other abelian group $K$ and $\phi\colon G\longto K$, there exists a unique
$L\colon U\longto K$ such that $\phi=L\circ\rho$.
In other words, the following diagram commutes

\centerline{\def\diagcolwidth{1cm}\def\diagrowheight{.75cm}\drawdiagram{
    $G$&&$K$\cr
    &$U$\cr
}{
    \diagarrow{from={1,1}, to={1,3}, text=$\phi$, y distance=.25cm}
    \diagarrow{from={1,1}, to={2,2}, text=$\rho$, y distance=-.25cm, slide=.4}
    \diagarrow{from={2,2}, to={1,3}, text=$\exists!L$, x distance=.5cm, slide=.4}
}}

Such a $U$ is once again unique, since by setting $K=U$ and $\phi=\rho$ we get that ${\rm id}$ is the unique morphism which makes the diagram commute.
And if we add another universal construction $U'$, we get

\medskip
\centerline{\def\diagcolwidth{1cm}\def\diagrowheight{.75cm}\drawdiagram{
    $G$&&$U'$\cr
    &$U$\cr
}{
    \diagarrow{from={1,1}, to={1,3}, text=$\rho'$, y distance=.25cm}
    \diagarrow{from={1,1}, to={2,2}, text=$\rho$, y distance=-.25cm, slide=.4}
    \diagarrow{from={2,2}, to={1,3}, text=$\exists!L$, x distance=.65cm, slide=.4, x off=.2cm}
    \diagarrow{from={1,3}, to={2,2}, text=$\exists!L'$, x distance=-.65cm, slide=.4, x off=-.2cm}
}}

So $L'\circ L$ makes the previous diagram commute, since $L'\circ L\circ \rho=L'\circ \rho'=\rho$.
But since this is unique, that means $L'\circ L={\rm id}$, and similarly $L\circ L'={\rm id}$, meaning $L$ is an isomorphism between $U$ and $U'$.
Such a $U$ is called the {\it abelianization} of $G$.

To construct this, define the commutator of $g,h\in G$ by $[g,h]=g^{-1}h^{-1}gh$.
Notice that $[g,h]^{-1}=[h,g]$ and $s[g,h]s^{-1}=[sgs^{-1},sh^{-1}s^{-1}]$.
Then define the {\it commutator subgroup} $[G,G]$ to be the subgroup of $G$ generated by the set of its commutators.
Since commutators are closed under conjugation, it follows that $[G,G]$ is normal.
Then this means that elements of $[G,G]$ are of the form $[g_1,h_1]\cdots[g_n,h_n]$.
Notice that $\slfrac G{[G,G]}$ is abelian: $ghg^{-1}h^{-1}\in[G,G]$ so $gh[G,G]=hg[G,G]$.
In fact, this is what we define to be the abelianization of $G$: $\Abof G\coloneqq\slfrac G{[G,G]}$.

Define $\rho$ naturally, $\rho(g)=g[G,G]$.
Then if $\phi\colon G\longto K$ is a homomorphism to an abelian group, then $L\circ\rho=\phi$ if $L(g[G,G])=\phi(g)$.
This is well-defined since if $g_1[G,G]=g_2[G,G]$ then $g_1g_2^{-1}\in[G,G]$ and every commutator is mapped to $1$ by $\phi$ (since $K$ is abelian), we get that $\phi(g_1)=\phi(g_2)$.
And $L$ is of course a homomorphism.

\medskip
We now define another universal construct, where $G,H$ are groups.
Then we want another group $U$ and $g\colon G\longto U$ and $h\colon H\longto U$ such that for every other group $K$ and $\phi\colon G\longto K$ and $\psi\colon H\longto K$, there exists a unique
$L\colon U\longto K$ such that the following diagram commutes:

\medskip
\centerline{\def\diagcolwidth{1cm}\def\diagrowheight{.75cm}\drawdiagram{
    $G$&\cr
    &$U$&$K$\cr
    $H$&&\cr
}{
    \diagarrow{to={2,3}, from={1,1}, curve=.6cm, text=$\phi$, y distance=.5cm, origin orient={right,ycenter}}
    \diagarrow{to={2,3}, from={3,1}, curve=-.6cm, text=$\psi$, y distance=-.6cm, origin orient={right,ycenter}}
    \diagarrow{to={2,3}, from={2,2}, text=$\exists!L$, y distance=.25cm}
    \diagarrow{to={2,2}, from={1,1}, text=$g$, y distance=.25cm, slide=.4}
    \diagarrow{to={2,2}, from={3,1}, text=$f$, y distance=-.25cm, slide=.4}
}}

Again, $U$ is unique.
Let us define $M$ to be the set of all words which utilize letters in $G$ and $H$ (which we assume to be disjoint).
We define an equivalence relation on $M$ where consecutive letters of the same group in a word are merged together, meaning for example
$$ (\cdots,g,g',\cdots) \sim (\cdots,gg',\cdots) $$
where $g,g'\in G$.
And we can remove any identity from any word,
$$ (\cdots,1,\cdots) \sim (\cdots,\cdots) $$
Define the partition of $M$ by $\sim$ as $G*H$ (called the {\it free product} of $G$ and $H$).
$G*H$ is a group under concatenation: $[\omega_1][\omega_2]=[\omega_1\omega_2]$ where $\omega_1\omega_2$ is the concatenation of the two words in $M$.
So for example $[(g_1,h_1)][(h_2,g_3,h_3)]=[(g_1,h_1,h_2,g_3,h_3)]=[(g_1,h_1h_2,g_3,h_3)]$.
This is well-defined, since if $\omega_1\sim\omega_1'$ and $\omega_2\sim\omega_2'$ then $\omega_1\omega_1'\sim\omega_2\omega_2'$.
This is since if $\omega_1'$ results in $\omega_1$ by combining consecutive elements or removing an identity and similar for $\omega'_2$, then $\omega_2\omega_2'$ results in $\omega_1\omega_1'$ by the
combined results of these operations.
Then continue inductively.
Since concatenation in $M$ is associative, so is $G*H$'s operation.
The identity is $[\epsilon]$ (the equivalence class of the empty word).
The inverse of $[(x_1,\dots,x_n)]$ is $[(x_n^{-1},\dots,x_1^{-1})]$.
So indeed $G*H$ is a group.

We claim that the free product satisfies this universal construction.
We can define $\iota_G\colon G\longto G*H$ by $\iota_G(g)=[(g)]$ and $\iota_H\colon H\longto G*H$ by $\iota_H(h)=[(h)]$ which are obviously embeddings (so we can view $G$ and $H$ as subgroups of their free
product).
Now if $\phi\colon G\longto K$ and $\psi\colon H\longto K$ then $L\circ\iota_G=\phi$ and $L\circ\iota_H=\psi$ if and only if $L[g]=\phi(g)$ and $L[h]=\psi(h)$ which uniquely determines the homomorphism $L$.
Namely, $L[x_1,\dots,x_n]=y_1\cdots y_n$ where $y_i=\phi(x_i)$ if $x_i\in G$ and $y_i=\psi(x_i)$ if $x_i\in H$.
