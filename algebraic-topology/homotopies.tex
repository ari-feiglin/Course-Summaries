\bdefn

    Let $X$ and $Y$ are topological spaces and $f,g\colon X\longto Y$ (meaning they are morphisms, thus continuous).
    We say that $f$ is homotopic to $g$, denoted $f\sim g$, if there exists an $H\colon X\times I\longto Y$ ($I=[0,1]$, $X\times I$ is the product topology) such that $H(x,0)=f(x)$ and
    $H(x,1)=g(x)$ for all $x\in X$.
    We denote $h_t(x)\coloneqq H(x,t)$, and $H$ is called a {\emphcolor homotopy} from $f$ to $g$.

\edefn

A homotopy is essentially a smooth mapping from one morphism $f$ to another $g$.
Homotopy is indeed an equivalence relation: firstly $f\sim f$ as we can define $H(x,t)=f(x)$ which is continuous as the composition of continuous functions ($H=f\circ\pi_1$), if $f\sim g$ then
define $H'(x,t)=H(x,1-t)$ which is also continuous (since $(x,t)\mapsto(x,1-t)$ is continuous since its components are) and $H'(x,0)=g(x)$ and $H'(x,1)=f(x)$ so $g\sim f$, and if $H_1$ is a homotopy from $f$
to $g$ and $H_2$ is a homotopy from $g$ to $h$, define
$$ H(x,t) = \cases{H_1(x,2t) & $0\leq t\leq\frac12$\cr H_2(x,2t-1) & $\frac12\leq t\leq1$} $$
$X\times[0,1/2]$ and $X\times[1/2,1]$ are closed (since $X\times[0,1/2]$ is the preimage of $[0,1/2]$ in the mapping $(x,t)\mapsto t$) and $H(x,t)$ is continuous on both of these (since $H_1(x,2t)$ and
$H_2(x,2t-1)$ are continuous), so $H(x,t)$ is continuous.

\bprop

    For every topological space $X$ and every two morphisms $f,g\colon X\longto{\bb R}^n$, $f$ and $g$ are homotopic.

\eprop

\Proof define $H(x,t)=(1-t)f(x)+tg(x)$ (addition and scalar multiplication are continuous).
\qed

\bdefn

    A topological space $X$ is {\emphcolor contractible} if the identity map ${\rm id}_X$ is homotopic to some constant map.

\edefn

Notice that all two constant maps are homotopic if and only if the space is path connected.
If all two constant maps are homotopic, for $x_1,x_2\in X$ let $H(x,t)$ be a homotopy from $x_1$ to $x_2$ and define $\gamma(t)=H(x_0,t)$ for any $x_0\in X$, this is a continuous path from $x_1$ to $x_2$.
And if $X$ is path connected, for $x_1$ and $x_2$ and $\gamma$ connecting them, define $H(x,t)=\gamma(t)$.

\bprop

    Let $X,Y,Z$ be topological spaces, $f,f'\colon X\longto Y$ and $g,g'\colon Y\to Z$ such that $f\sim f'$ and $g\sim g'$, then $g\circ f\sim g'\circ f'$.

\eprop

\Proof let $H$ be a homotopy from $f$ to $f'$ and $K$ a homotopy from $g$ to $g'$.
Then define $J(x,t)=K(H(x,t),t)$ which is a composition of continuous functions (map $(x,t)$ to $((x,t),t)$ to $(H(x,t),t)$ to $K(H(x,t),t)$).
\qed

We call the equivalence classes of morphisms under $\sim$ {\it homotopy classes}, and the homotopy class of a morphism $f$ is denoted $[f]$.
So by above, $[f]\circ[g]\coloneqq[f\circ g]$ is a well-defined operation.
This gives us a new category whose objects are topological spaces and morphisms are homotopy classes.
What are the isomorphisms in this category?
Well the identities are obviously $[1_X]$ since $[f]\circ[1_X]=[f\circ1_X]=[f]$ and $[1_X]\circ[g]=[1_X\circ g]=[g]$.
So an isomorphism $X\xvarrightarrow{\,[f]\,}Y$ is a homotopy class such that there exists a $Y\xvarrightarrow{\,[g]\,}X$ such that $[f]\circ[g]=[f\circ g]=[1_X]$ and $[g\circ f]=[1_Y]$.
We give these isomorphisms a different name:

\bdefn

    Let $X$ and $Y$ be topological spaces, then $f\colon X\longto Y$ is a {\emphcolor homotopic equivalence} if there exists a $g\colon Y\longto X$ such that $g\circ f\sim{\rm id}_X$ and
    $f\circ g\sim{\rm id}_Y$.
    If a homotopy equivalence exists between $X$ and $Y$, then $X$ and $Y$ are said to be {\emphcolor homotopy equivalent}, denoted $X\simeq Y$.

\edefn

Notice that homeomorphisms are homotopic equivalences, since $f^{-1}\circ f={\rm id}_X$ and $f\circ f^{-1}={\rm id}_Y$.

\bdefn

    Let $X$ and $Y$ be topological spaces, $A\subseteq X$, and $f,g\colon X\longto Y$.
    We say that $f$ and $g$ are homotopic relative to $A$, denoted $f\buildrel A\over\sim g$, if there exists a homotopy $H$ from $f$ to $g$ such that $H(a,t)=f(a)$ for all $a\in A$ and $t\in I$.
    In such a case we must have $f\bigr|_A=g\bigr|_A$.

\edefn

It is not enough for $f\sim g$ and $f\bigr|_A=g\bigr|_A$ for $f$ and $g$ to be homotopic relative to $A$.
For example take $I$ and $S^1$ and the points $0$ and $1$ on $I$.
Then we can continuously deform $I$ so that it maps onto the bottom or top of the circle.
These are two continuous mappings which are homotopic, but no homotopy between them which keeps the image of $0$ and $1$ constant.

Notice that $\buildrel A\over\sim$ is an equivalence relation, the proof of this is analogous to the proof that homotpy is an equivalence relation.
It also preserves composition, if $f,f'\colon(X,A)\longto(Y,B)$ (meaning they are morphisms from $X$ to $Y$ and $f(A),f'(A)\subseteq B$) and $g,g'\colon(Y,B)\longto(Z,C)$ such that $f\buildrel A\over\sim f'$
and $g\buildrel B\over\sim g'$, then $g\circ f\buildrel A\over\sim g'\circ f'$.

\bdefn

    Let $X$ be a topological space.
    $A\subseteq X$ is called a {\emphcolor retract} if there exists an $r\colon X\longto A$ such that $r\circ\iota={\rm id}_A$ where $\iota\colon A\longto X$ is the inclusion map.
    In other words $r(a)=a$ for all $a\in A$.
    $r$ is called a {\emphcolor retraction}.

\edefn

For example $\partial I=\set{0,1}$ is not a retraction of $I$ since every continuous image of $I$ must be connected, and $\partial I$ is not.
But if we take $X$ to be an eight shape, and $A$ its bottom circle, then we can map the top circle to the middle point and $A$ to itself and this is a retraction.

\bdefn

    $A\subseteq X$ is called a {\emphcolor deformation retract} if there exists a retraction $r$ such that $\iota\circ r\buildrel A\over\sim{\rm id}_X$.

\edefn

Instead of requiring $r$ be a retraction, we can require only that $r(X)\subseteq A$.
Since then if $\iota\circ r\buildrel A\over\sim{\rm id}_X$, this means that $r(a)={\rm id}_X(a)=a$ for all $a\in A$ so it is already a retraction.
Explicitly, this is equivalent to saying that there exists a homotopy $H\colon X\times I\longto X$ such that $H(x,0)=x$ for all $x\in X$, $H(a,t)=a$ for all $a\in A,t\in I$, $H(x,1)\in A$ for all $x\in X$.

Notice that if $A\subseteq X$ is a deformation retract then $\iota\colon A\longto X$ is a homotopy equivalence, since $r\circ\iota={\rm id}_A$ and $\iota\circ r\sim{\rm id}_X$.

\bexam

    Let $X={\bb R}^n\setminus\set0$ and $A=S^{n-1}\subseteq{\bb R}^n\setminus\set0$.
    Then $r(x)\coloneqq\frac x{\norm x}$ is a retraction with the homotopy $H(x,t)=(1-t)x+t\frac x{\norm x}$.
    This is the homotopy we used to show that all morphisms to ${\bb R}^n$ are homotopic.

\eexam

A morphism $f$ is called {\it null-homotopic} if it is homotopic to a constant morphism.

\bprop

    Let $X$ be a topological space and $f\colon S^1\longto X$, then the following are equivalent
    \benum
        \item $f$ is null-homotopic,
        \item $f$ is null-homotopic relative to any point on $S^1$,
        \item $f$ can be expanded to a morphism on $D^2$ (the disk in ${\bb R}^2$), meaning there exists an $F\colon D^2\longto X$ such that $F\bigr|_{S'}=f$.
    \eenum

\eprop

$(2)\implies(1)$ is trivial since a null-homotopy relative to a point is still a null-homotopy.
$(3)\implies(2)$: let $\iota\colon S^1\longto D^2$ be the inclusion map, and let $a\in S^1$, define the homotopy $H\colon S^1\longto I\longto D^2$ by $H(x,t)=(1-t)\iota(x)+ta$, which is a homotopy from
$\iota$ to the constant map $K_a$.
Then $F\circ H$ is a null-homotopy between $f$ and $K_{f(a)}$ (since $F\circ H(x,0)=F(x)=f(x)$ and $F\circ H(x,1)=F(a)$) relative to $a$ since $F\circ H(a,t)=F(a)$.
$(1)\implies(3)$: so there exists a homotopy $H\colon S^1\times I\longto X$ such that $H(x,0)=f(x)$ for every $x\in S^1$ and there exists a $p\in X$ such that $H(x,1)=p$ for all $x\in S^1$.
Let us define $\rho\colon S^1\times I\longto D^2$ by $\rho(x,t)=(1-t)x$, this is a continuous map from a compact (since $S^1$ and $I$ are compact and therefore so is their product) to a Hausdorff space, and
so it is closed.
And it is surjective, so it is a quotient map.
So $D^2$ is the quotient space of $S^1\times I$ with respect to $\rho$, and $H$ respects $\rho$, since $\rho(x,t)=\rho(y,s)$ implies $(1-t)x=(1-s)y$ and this means that either $(x,t)=(y,s)$ or $t=s=1$.
But in both cases $H(x,t)=H(y,s)$, and so there exists an $F\colon D^2\longto X$ which is continuous such that $H=F\circ\rho$, meaning $F(x)=H(x,0)=f(x)$ as required.
\qed

This proof uses the fact that if $\rho$ is a quotient map, and $f\colon X\longto Y$ is continuous then there exists a $F\colon\overline X\longto Y$ such that $f=F\circ\rho$ if and only if $\rho(a)=\rho(b)$
implies $f(a)=f(b)$.


