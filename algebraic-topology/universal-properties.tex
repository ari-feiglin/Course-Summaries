{\bf Direct products}:
Suppose we have two groups $G$ and $H$, then we'd like some universal group $U$ along with two morphisms $g\colon U\longto G$ and $f\colon U\longto H$ such that for every $K$ and $\phi\colon K\longto G$,
$\psi\colon K\longto H$, there exists a unique $L\colon K\longto U$ such that the following diagram commutes

\centerline{\def\diagcolwidth{1cm}\def\diagrowheight{.75cm}\drawdiagram{
    &&$G$\cr
    $K$&$U$\cr
    &&$H$\cr
}{
    \diagarrow{from={2,1}, to={1,3}, curve=.6cm, text=$\phi$, y distance=.5cm, dest orient={left,ycenter}}
    \diagarrow{from={2,1}, to={3,3}, curve=-.6cm, text=$\psi$, y distance=-.6cm, dest orient={left,ycenter}}
    \diagarrow{from={2,1}, to={2,2}, text=$\exists!L$, y distance=.25cm}
    \diagarrow{from={2,2}, to={1,3}, text=$g$, y distance=.25cm, slide=.4}
    \diagarrow{from={2,2}, to={3,3}, text=$f$, y distance=-.25cm, slide=.4}
}}

Such a group would be unique up to isomorphism: if $U'$ is another, then our diagram becomes

\centerline{\def\diagcolwidth{1cm}\def\diagrowheight{.75cm}\drawdiagram{
    &&$G$\cr
    $U'$&$U$\cr
    &&$H$\cr
}{
    \diagarrow{from={2,1}, to={1,3}, curve=.6cm, text=$f'$, y distance=.5cm, dest orient={left,ycenter}}
    \diagarrow{from={2,1}, to={3,3}, curve=-.6cm, text=$g'$, y distance=-.6cm, dest orient={left,ycenter}}
    \diagarrow{from={2,1}, to={2,2}, text=$\exists!L$, y off=.1cm, y distance=.35cm}
    \diagarrow{from={2,2}, to={2,1}, text=$\exists!L'$, y off=-.1cm, y distance=-.35cm}
    \diagarrow{from={2,2}, to={1,3}, text=$g$, y distance=.25cm, slide=.4}
    \diagarrow{from={2,2}, to={3,3}, text=$f$, y distance=-.25cm, slide=.4}
}}

Taking $U=U'$ then we see that $L={\rm id}$ satisfies the conditions for $f,g$, so it must be unique.
And so now when $U\neq U'$, $L\circ L'$ also satisfies the conditions, so $L\circ L'={\rm id}$.
And similarly $L'\circ L={\rm id}$, meaning $L$ is an isomorphism between $U$ and $U'$.

One such universal construction is $U=G\times H$ with the projection maps $p_G$ and $p_H$, since if $\phi\colon K\longto G$ and $\psi\colon K\longto H$, then $p_G\circ L=\phi$ and $p_H\circ L=\psi$ if and
only if $L=(\phi,\psi)$.
So such a map exists, and it is unique, meaning $G\times H$ indeed has this universal property.
Any universal construction satisfying this is thus called a {\it product} of $G$ and $H$.

\medskip
{\bf Abelianization}:
Similarly let $G$ be a group, then we want an abelian group $U$ and a morphism $\rho\colon G\longto U$ such that for every other abelian group $K$ and $\phi\colon G\longto K$, there exists a unique
$L\colon U\longto K$ such that $\phi=L\circ\rho$.
In other words, the following diagram commutes

\centerline{\def\diagcolwidth{1cm}\def\diagrowheight{.75cm}\drawdiagram{
    $G$&&$K$\cr
    &$U$\cr
}{
    \diagarrow{from={1,1}, to={1,3}, text=$\phi$, y distance=.25cm}
    \diagarrow{from={1,1}, to={2,2}, text=$\rho$, y distance=-.25cm, slide=.4}
    \diagarrow{from={2,2}, to={1,3}, text=$\exists!L$, x distance=.5cm, slide=.4}
}}

Such a $U$ is once again unique, since by setting $K=U$ and $\phi=\rho$ we get that ${\rm id}$ is the unique morphism which makes the diagram commute.
And if we add another universal construction $U'$, we get

\medskip
\centerline{\def\diagcolwidth{1cm}\def\diagrowheight{.75cm}\drawdiagram{
    $G$&&$U'$\cr
    &$U$\cr
}{
    \diagarrow{from={1,1}, to={1,3}, text=$\rho'$, y distance=.25cm}
    \diagarrow{from={1,1}, to={2,2}, text=$\rho$, y distance=-.25cm, slide=.4}
    \diagarrow{from={2,2}, to={1,3}, text=$\exists!L$, x distance=.65cm, slide=.4, x off=.2cm}
    \diagarrow{from={1,3}, to={2,2}, text=$\exists!L'$, x distance=-.65cm, slide=.4, x off=-.2cm}
}}

So $L'\circ L$ makes the previous diagram commute, since $L'\circ L\circ \rho=L'\circ \rho'=\rho$.
But since this is unique, that means $L'\circ L={\rm id}$, and similarly $L\circ L'={\rm id}$, meaning $L$ is an isomorphism between $U$ and $U'$.
Such a $U$ is called the {\it abelianization} of $G$.

To construct this, define the commutator of $g,h\in G$ by $[g,h]=g^{-1}h^{-1}gh$.
Notice that $[g,h]^{-1}=[h,g]$ and $s[g,h]s^{-1}=[sgs^{-1},sh^{-1}s^{-1}]$.
Then define the {\it commutator subgroup} $[G,G]$ to be the subgroup of $G$ generated by the set of its commutators.
Since commutators are closed under conjugation, it follows that $[G,G]$ is normal.
Then this means that elements of $[G,G]$ are of the form $[g_1,h_1]\cdots[g_n,h_n]$.
Notice that $\slfrac G{[G,G]}$ is abelian: $ghg^{-1}h^{-1}\in[G,G]$ so $gh[G,G]=hg[G,G]$.
In fact, this is what we define to be the abelianization of $G$: $\Abof G\coloneqq\slfrac G{[G,G]}$.

Define $\rho$ naturally, $\rho(g)=g[G,G]$.
Then if $\phi\colon G\longto K$ is a homomorphism to an abelian group, then $L\circ\rho=\phi$ if $L(g[G,G])=\phi(g)$.
This is well-defined since if $g_1[G,G]=g_2[G,G]$ then $g_1g_2^{-1}\in[G,G]$ and every commutator is mapped to $1$ by $\phi$ (since $K$ is abelian), we get that $\phi(g_1)=\phi(g_2)$.
And $L$ is of course a homomorphism.

\medskip
{\bf Coproduct (free product)}:
We now define another universal construct, where $G,H$ are groups.
Then we want another group $U$ and $g\colon G\longto U$ and $h\colon H\longto U$ such that for every other group $K$ and $\phi\colon G\longto K$ and $\psi\colon H\longto K$, there exists a unique
$L\colon U\longto K$ such that the following diagram commutes:

\medskip
\centerline{\def\diagcolwidth{1cm}\def\diagrowheight{.75cm}\drawdiagram{
    $G$&\cr
    &$U$&$K$\cr
    $H$&&\cr
}{
    \diagarrow{to={2,3}, from={1,1}, curve=.6cm, text=$\phi$, y distance=.5cm, origin orient={right,ycenter}}
    \diagarrow{to={2,3}, from={3,1}, curve=-.6cm, text=$\psi$, y distance=-.6cm, origin orient={right,ycenter}}
    \diagarrow{to={2,3}, from={2,2}, text=$\exists!L$, y distance=.25cm}
    \diagarrow{to={2,2}, from={1,1}, text=$g$, y distance=.25cm, slide=.4}
    \diagarrow{to={2,2}, from={3,1}, text=$f$, y distance=-.25cm, slide=.4}
}}

Again, $U$ is unique.
Let us define $M$ to be the set of all words which utilize letters in $G$ and $H$ (which we assume to be disjoint).
We define an equivalence relation on $M$ where consecutive letters of the same group in a word are merged together, meaning for example
$$ (\cdots,g,g',\cdots) \sim (\cdots,gg',\cdots) $$
where $g,g'\in G$.
And we can remove any identity from any word,
$$ (\cdots,1,\cdots) \sim (\cdots,\cdots) $$
Define the partition of $M$ by $\sim$ as $G*H$ (called the {\it free product} of $G$ and $H$).
$G*H$ is a group under concatenation: $[\omega_1][\omega_2]=[\omega_1\omega_2]$ where $\omega_1\omega_2$ is the concatenation of the two words in $M$.
So for example $[(g_1,h_1)][(h_2,g_3,h_3)]=[(g_1,h_1,h_2,g_3,h_3)]=[(g_1,h_1h_2,g_3,h_3)]$.
This is well-defined, since if $\omega_1\sim\omega_1'$ and $\omega_2\sim\omega_2'$ then $\omega_1\omega_1'\sim\omega_2\omega_2'$.
This is since if $\omega_1'$ results in $\omega_1$ by combining consecutive elements or removing an identity and similar for $\omega'_2$, then $\omega_2\omega_2'$ results in $\omega_1\omega_1'$ by the
combined results of these operations.
Then continue inductively.
Since concatenation in $M$ is associative, so is $G*H$'s operation.
The identity is $[\epsilon]$ (the equivalence class of the empty word).
The inverse of $[(x_1,\dots,x_n)]$ is $[(x_n^{-1},\dots,x_1^{-1})]$.
So indeed $G*H$ is a group.

We claim that the free product satisfies this universal construction.
We can define $\iota_G\colon G\longto G*H$ by $\iota_G(g)=[(g)]$ and $\iota_H\colon H\longto G*H$ by $\iota_H(h)=[(h)]$ which are obviously embeddings (so we can view $G$ and $H$ as subgroups of their free
product).
Now if $\phi\colon G\longto K$ and $\psi\colon H\longto K$ then $L\circ\iota_G=\phi$ and $L\circ\iota_H=\psi$ if and only if $L[g]=\phi(g)$ and $L[h]=\psi(h)$ which uniquely determines the homomorphism $L$.
Namely, $L[x_1,\dots,x_n]=y_1\cdots y_n$ where $y_i=\phi(x_i)$ if $x_i\in G$ and $y_i=\psi(x_i)$ if $x_i\in H$.

\medskip
{\bf Amalgation}:
Now let us work upon this construction.
Suppose we have the following commutative diagram:

\medskip
\centerline{\def\diagcolwidth{1cm}\def\diagrowheight{.75cm}\drawdiagram{
    &$G$\cr
    $P$&&&$K$\cr
    &$H$\cr
}{
    \diagarrow{from={2,1}, to={1,2}, text=$\eta_G$, x distance=-.4cm}
    \diagarrow{from={2,1}, to={3,2}, text=$\eta_H$, x distance=-.4cm}
    \diagarrow{from={1,2}, to={2,4}, text=$\phi$, y distance=.25cm}
    \diagarrow{from={3,2}, to={2,4}, text=$\psi$, y distance=-.25cm}
}}

Meaning $\phi\circ\eta_G=\psi\circ\eta_H$.
Then we can add in $G*H$,

\medskip
\centerline{\def\diagcolwidth{1cm}\def\diagrowheight{.75cm}\drawdiagram{
    &$G$\cr
    $P$&$G*H$&&$K$\cr
    &$H$\cr
}{
    \diagarrow{from={2,1}, to={1,2}, text=$\eta_G$, x distance=-.4cm}
    \diagarrow{from={2,1}, to={3,2}, text=$\eta_H$, x distance=-.4cm}
    \diagarrow{from={1,2}, to={2,4}, text=$\phi$, y distance=.25cm}
    \diagarrow{from={3,2}, to={2,4}, text=$\psi$, y distance=-.25cm}
    \diagarrow{from={1,2}, to={2,2}, text=$\delta_G$, x distance=.25cm}
    \diagarrow{from={3,2}, to={2,2}, text=$\delta_H$, x distance=.25cm}
    \diagarrow{from={2,2}, to={2,4}, text=$\exists!L'$, y distance=.25cm}
}}

But we no longer have that this diagram commutes, since $\delta_G\circ\eta_G$ and $\delta_H\circ\eta_H$ are not necessarily equal.
So let us define $N$ to be the normal subgroup of $G*H$ generated by $\set{\bigl(\delta_G\eta_G(p)\bigr)\bigl(\delta_H\eta_H(p)\bigr)^{-1}}[p\in P]$.
Now, $N\subseteq\kerof{L'}$ since
$$ L'\bigl(\delta_G\eta_G(p)\bigr)L'\bigl(\delta_H\eta_H(p)\bigr)^{-1} = \phi\eta_G(p)\psi\eta_H(p)^{-1} $$
since $\phi\eta_G=\psi\eta_H$, this is just $1$.
Thus $L'$ generates a homomorphism $L\colon\slfrac{G*H}N\longto K$ by $\omega N\mapsto L'(\omega)$, so we get the following diagram

\medskip
\centerline{\def\diagcolwidth{1.5cm}\def\diagrowheight{1.5cm}\drawdiagram{
    &$G$\cr
    $P$&$G*H$&$\slfrac{G*H}N$&$K$\cr
    &$H$\cr
}{
    \diagarrow{from={2,1}, to={1,2}, text=$\eta_G$, x distance=-.4cm}
    \diagarrow{from={2,1}, to={3,2}, text=$\eta_H$, x distance=-.4cm}
    \diagarrow{from={1,2}, to={2,4}, text=$\phi$, y distance=.25cm}
    \diagarrow{from={3,2}, to={2,4}, text=$\psi$, y distance=-.25cm}
    \diagarrow{from={1,2}, to={2,2}, text=$\delta_G$, x distance=.25cm}
    \diagarrow{from={3,2}, to={2,2}, text=$\delta_H$, x distance=.25cm}
    \diagarrow{from={2,2}, to={2,4}, text=$L'$, curve=1cm, y distance=.8cm, y off=.15cm}
    \diagarrow{from={2,2}, to={2,3}, text=$\rho$, y distance=.15cm}
    \diagarrow{from={2,3}, to={2,4}, text=$L$, y distance=-.25cm}
    \diagarrow{from={1,2}, to={2,3}, text=$\epsilon_G$, x distance=.25cm}
    \diagarrow{from={3,2}, to={2,3}, text=$\epsilon_H$, x distance=.25cm}
}}

Now we claim that the following diagram, obtained by removing $G*H$, commutes:

\medskip
\centerline{\def\diagcolwidth{1cm}\def\diagrowheight{.75cm}\drawdiagram{
    &$G$\cr
    $P$&$\slfrac{G*H}N$&&$K$\cr
    &$H$\cr
}{
    \diagarrow{from={2,1}, to={1,2}, text=$\eta_G$, x distance=-.4cm}
    \diagarrow{from={2,1}, to={3,2}, text=$\eta_H$, x distance=-.4cm}
    \diagarrow{from={1,2}, to={2,4}, text=$\phi$, y distance=.25cm}
    \diagarrow{from={3,2}, to={2,4}, text=$\psi$, y distance=-.25cm}
    \diagarrow{from={2,2}, to={2,4}, text=$L$, y distance=-.25cm}
    \diagarrow{from={1,2}, to={2,2}, text=$\epsilon_G$, x distance=.25cm}
    \diagarrow{from={3,2}, to={2,2}, text=$\epsilon_H$, x distance=.25cm}
}}

Indeed, let $p\in P$, then $\epsilon_G\circ\eta_G(p)=\rho\circ\delta_G\circ\eta_G(p)=\delta_G\circ\eta_G(p)N$ and $\epsilon_H\circ\eta_H(p)=\delta_H\circ\eta_H(p)N$.
By definition of $N$, these are equal.
And $L\circ\epsilon_G=L\circ\rho\circ\delta_G=L'\circ\delta_G=\phi$ and similarly $L\circ\epsilon_H=\psi$.
So this diagram does indeed commute.
$\slfrac{G*H}N$ is called the {\it amalgated product} of $G$ and $H$ relative to $P$, denoted $G*_PH$.

\medskip
{\bf Free groups}:
Let $X$ be a set, then the {\it free group} over $X$ is a group $U$ with $i\colon X\longto U$ set function, such that for every group $K$ and set function $f\colon X\longto K$, there exists a unique
homomorphism $L\colon U\longto K$ such that the following diagram commutes

\medskip
\centerline{\def\diagcolwidth{1cm}\def\diagrowheight{.75cm}\drawdiagram{
    $X$&&$K$\cr
    &$U$\cr
}{
    \diagarrow{from={1,1}, to={1,3}, text=$i$, y distance=.25cm}
    \diagarrow{from={1,1}, to={2,2}, text=$f$, y distance=-.25cm, slide=.4}
    \diagarrow{from={2,2}, to={1,3}, text=$\exists!L$, x distance=.5cm, slide=.4}
}}

Let $W=W(X)$ be the set of all words over $X\coprod X$.
For elements $a\in X$, let us denote $(a,0)$ by $a$ and $(a,1)$ by $a^{-1}$, meaning we denote elements $a\in X$ in the left $X$ by $a$ and in the right $X$ by $a^{-1}$.
We define an equivalence relation over $W$ as follows: a word $U$ is equivalent to a word $V$ if $U$ and $V$ can be brought to a similar word using a finite number of the following operations:
\benum
    \item A word of the form $(\underline{\ A\ },a,a^{-1},\underline{\ B\ })$ is brought to $(\underline{\ A\ },\underline{\ B\ })$.
    \item Similarly a word of the form $(\underline{\ A\ },a^{-1},a,\underline{\ B\ })$ is also brought to $(\underline{\ A\ },\underline{\ B\ })$.
\eenum
We define the group operation over this set by $[U][V]=[UV]$, meaning the product of (the equivalence classes) of two words is the (equivalence class) of the concatenation of the two words.
This is indeed a group operation as it is associative, the identity is the empty word, and the inverse of $U$ is $U^{-1}$ (the word obtained by reversing the order of $U$ and swapping $a$ with $a^{-1}$ and
$a^{-1}$ with $a$).
The set $W$ equipped with this operation is called the {\it free group} over $X$, denoted $F(X)$.

The free group satisfies the above universal property, with $i\colon a\mapsto[a]$.
Let $f\colon X\longto K$ be a set function, then $f=L\circ i$ if and only if $f(a)=L[a]$, so define $L[a]=f(a)$ for $a\in X$.
And similarly define $L[a^{-1}]=f(a)^{-1}$, so that this extends to a homomorphism:
$$ L[a_1^{x_1}\cdots a_n^{x_n}] = f(a_1)^{x_1}\cdots f(a_n)^{x_n},\qquad x_i=1,-1 $$
Call a word in $W$ {\it reduced} if there are no adjacent occurrences of $a$ and $a^{-1}$ in it.

\bprop

    For every $g\in F(X)$, there exists a unique representative of $g$ which is a reduced word.

\eprop

\Proof let $\omega\in W$ be a word, and define $\#\omega$ be the number of times an element $a\in X$ occurs adjacent to $a^{-1}$.
We claim that $\omega$ is equivalent to a reduced word.
If $\#\omega=0$ then $\omega$ is reduced.
Otherwise we induct on $\omega$: if $\#\omega=n+1$ then we can apply one of the operations to get an $\omega'$ equivalent to $\omega$ such that $\#\omega'=n$.
But then by induction, $\omega'$ is equivalent to a reduced word so by transitivity so is $\omega$.
The reduced word is unique, since a reduced word cannot have one of the operations applied to it, so two reduced words cannot be equivalent.
\qed

Let $R\subseteq W(X)$ be a set of words over $X\coprod X$.
Then let $N$ be the normal subgroup of $F(X)$ generated by the equivalence classes of words in $R$ (ie. $N$ is generated by $\set{[\omega]}[\omega\in R]$).
Let $f\colon X\longto K$ be a function to a group $K$, notice that it can be extended to $W(X)\longto K$ which maps $a_1^{x_1}\cdots a_n^{x_n}$ to $f(a_1)^{x_1}\cdots f(a_n)^{x_n}$.
If $f(\omega)=1$ for all $\omega\in R$.
Since $F(X)$ is the free group, by its construction it induces a unique homomorphism $L'\colon F(X)\longto K$ such that $f=L'\circ i'$ (where $i'$ is the inclusion $X\longto F(X)$).
Now, $N\subseteq\kerof{L'}$ since $L'[\omega]=f(\omega)=1$ for $\omega\in R$.
Thus $L'$ induces a homomorphism $L\colon\slfrac{F(X)}N\longto K$ such that the following diagram commutes

\medskip
\centerline{\def\diagcolwidth{1cm}\def\diagrowheight{.75cm}\drawdiagram{
    $X$&&$K$\cr
    &$F(X)$\cr
    &$\slfrac{F(X)}N$\cr
}{
    \diagarrow{from={1,1}, to={1,3}, text=$f$, y distance=.25cm}
    \diagarrow{from={1,1}, to={2,2}, text=$i'$, y distance=.25cm}
    \diagarrow{from={1,1}, to={3,2}, text=$i$, y distance=-.65cm, slide=.3}
    \diagarrow{from={2,2}, to={1,3}, text=$L'$, y distance=.25cm}
    \diagarrow{from={3,2}, to={1,3}, text=$L$, y distance=-.65cm, slide=.7}
    \diagarrow{from={2,2}, to={3,2}, text=$\rho$, x distance=.25cm}
}}

We denote $\slfrac{F(X)}N$ by $\gen{X}[R]$, called the free group generated by $X$ modulo the relations $R$.
Notice that $F(X)=\gen{X}[\varnothing]$, also denoted $\gen X$.

Every group is isomorphic to such a free group.
Let $G$ be a group with generators $X$ (which could just be $X=G$), then there exists a homomorphism $\phi\colon F(X)\longto G$ given by the set function $X\longto G$ which is simply the inclusion.
$\phi$ is surjective since $X$ generates $G$, so $G\cong\slfrac{F(X)}{\ker\phi}$.
So if we let $R$ be the set of generators of $\ker\phi$ as a normal subgroup (in the worst case, $R=\ker\phi$, or more correctly representatives of elements in $\ker\phi$), then
$G\cong\slfrac{F(X)}{\ker\phi}=\gen X[R]$ as required.

