First, let us review a few concepts from point-set topology.
Namely, we will review product and quotient topologies, since they are the most complicated constructions we had, the rest should not be too hard to recall.

Recall that if $B$ is a set of subsets of $X$, then we define $\tau_B$ to be the set of all unions of sets from $B$:
$$ \tau_B = \set{\bigcup_{A\in L}A}[L\subseteq B] $$
then $\tau_B$ is a topology if and only if $X\in\tau_B$ and for every $\U,\V\in B$, $\U\cap\V\in\tau_B$.
If $\tau_B$ is a topology, these hold because $X$ and the intersection of open sets are open.
Conversely, $\varnothing\in\tau_B$ vacuously, if $\set{\U_\alpha}_{\alpha\in I}\in\tau_B$ then suppose $\U_\alpha=\bigcup_{i\in J_\alpha}\V_i^\alpha$ then
$\bigcup\U_\alpha=\bigcup_I\bigcup_{J_\alpha}\V_i^\alpha\in\tau_B$ so the union of open sets is open.
And similarly if $\U=\bigcup_I\U_\alpha$ and $\V=\bigcup_I\V_\beta$ then $\U\cap\V=\bigcup_{I,J}\U_\alpha\cap\V_\beta\in\tau_B$, so the intersection of open sets is open.

\bdefn

    Now if $\set{(X_\alpha,\tau_\alpha)}_{\alpha\in I}$ is a collection of topological spaces then we define $X=\prod_{\alpha\in I}X_\alpha$
    $$ B = \set{\prod_{\alpha\in I}\U_\alpha}[\hbox{$\U_\alpha\in\tau_\alpha$ and all but a finite number of $\alpha$ satisfy $\U_\alpha=X_\alpha$}] $$
    then we define the {\emphcolor product topology} on $X$ by $\tau=\tau_B$.

\edefn

Firstly $X\in B$, and if $\U=\prod_{\alpha\in I}\U_\alpha\in B$ and $\V=\prod_{\alpha\in I}\V_\alpha\in B$ then $\U\cap\V=\prod_{\alpha\in I}(\U_\alpha\cap\V_\alpha)$ and all but a finite number of these
are $X_\alpha$, so $\U\cap\V\in B$.
So $\tau_B$ is indeed a topology.

Let $X=\prod_{\alpha\in I}X_\alpha$ be equipped with the product topology.
We now state a few facts of product topologies without proving them (proofs can be found in my topology summary):

\benum
    \item Let $\pi_\beta\colon X\longto X_\beta$ be the projection map $(x_\alpha)_{\alpha\in I}\mapsto x_\beta$.
    Then $\pi_\beta$ is an open and continuous map.
    \item Let $Y$ be an arbitrary topological space.
    Then $f\colon Y\longto X$ is continuous if and only if $\pi_\alpha\circ f$ is for every $\alpha\in I$.
    \item If $X_\alpha$ has a basis $B_\alpha$ then $B=\set{\prod_{\alpha\in I}\U_\alpha}[\hbox{$\U_\alpha\in B_\alpha$ and all but a finite number are $X_\alpha$}]$ is a basis for $X$.
    \item If $Y=\prod_{\alpha\in I}Y_\alpha$ and $f_\alpha\colon X_\alpha\longto Y_\alpha$ are functions, define $f\colon X\longto Y$ by $(x_\alpha)_{\alpha\in I}\mapsto(f(x_\alpha))_{\alpha\in I}$.
    Then
    \benum
        \item $f$ is continuous if and only if every $f_\alpha$ is,
        \item if $f$ is open then each $f_\alpha$ is, and this becomes if and only if if $I$ is finite or $f_\alpha$ are all surjective,
        \item $f$ is a homeomorphism if and only if each $f_\alpha$ is.
    \eenum
    \item If $f_\alpha\colon Y\longto X_\alpha$ is continuous, so is $f\colon Y\longto X$ defined by $x\mapsto(f_\alpha(x))_{\alpha\in I}$.
    \item If every $X_\alpha$ is connected (respectively, path connected), so is $X$.
    \item $X$ is compact if and only if every $X_\alpha$ is compact.
\eenum

Now let us discuss quotient spaces, and in more depth than products.

\bdefn

    If $(X,\tau)$ is a topological space and $\rho\colon X\longto Y$ is a surjection, then $\sigma$ is a {\emphcolor quotient topology} of $Y$ if
    \benum
        \item $\rho\colon(X,\tau)\longto(Y,\sigma)$ is continuous,
        \item if $\gamma$ is another topology on $Y$ such that $\rho$ is continuous with respect to it, then $\gamma\subseteq\sigma$.
    \eenum

\edefn

In such a case, $\sigma=\set{\U\subseteq Y}[\rho^{-1}(\U)\in\tau]$.
To prove this, set $\sigma'=\set{\U\subseteq Y}[\rho^{-1}(\U)\in\tau]$, our goal is to of course show that $\sigma=\sigma'$.
First notice that $\sigma'$ is a topology: $\rho^{-1}(Y)=X,\rho^{-1}(\varnothing)=\varnothing$ so both $Y$ and the empty set are in $\sigma'$.
If $\rho^{-1}(\U),\rho^{-1}(\V)\in\tau$ then $\rho^{-1}(\U)\cap\rho^{-1}(\V)=\rho^{-1}(\U\cap\V)\in\tau$.
$\sigma'$ is also closed under unions since $\bigcup_{i\in I}\rho^{-1}(\U_i)=\rho^{-1}\parens{\bigcup_{i\in I}\U_i}$.

$\sigma'$ makes $\rho$ continuous by definition so $\sigma'\subseteq\sigma$.
And if $\U\in\sigma$ then $\rho^{-1}(\U)\in\tau$ since $\rho$ is continuous with respect to $\sigma$, so $\U\in\sigma'$.
Thus $\sigma=\sigma'$ as required.

And so we have shown that $\rho$ is a quotient map if and only if it is a surjection and $\U\subseteq Y$ is open if and only if $\rho^{-1}(\U)$ is.

\bprop

    Suppose $\rho\colon X\longto Y$ is a quotient map, and $q\colon Y\longto Z$ is continuous.
    Then $q$ is a quotient map if and only if $q\circ\rho$ is.

\eprop

\Proof if $q$ is a quotient map then $q\circ\rho$ is a surjection.
And $\U\subseteq Z$ is open if and only if $q^{-1}(\U)$ is, if and only if $\rho^{-1}\circ q^{-1}(\U)=(q\circ\rho)^{-1}(\U)$ is.
So $q\circ\rho$ is a quotient map.
Conversely, if $q\circ\rho$ is a quotient map then it is a surjection and therefore so is $q$.
Now if $\U\subseteq Y$ is open then so is $q^{-1}(\U)$ since it is continuous.
And if $q^{-1}(\U)$ is open, then so is $\rho^{-1}\circ q^{-1}(\U)=(\rho\circ q)^{-1}(\U)$ and therefore so is $\U$.
\qed

\bprop

    If $\rho\colon X\longto Y$ is a quotient map, and $f\colon Y\longto Z$ is a function, then $f$ is continuous if and only if $f\circ\rho$ is.

\eprop

\Proof if $f$ is continuous, then so is the composition of continuous functions $f\circ\rho$.
If $f\circ\rho$ is continuous, then let $\U\subseteq Z$ open, then $(f\circ\rho)^{-1}(\U)=\rho^{-1}\circ f^{-1}(\U)$ is open, and therefore so is $f^{-1}(\U)$ since $\rho$ is a quotient map.
\qed

This is an important result, as it allows us to prove that a function from a quotient space $Y$ is continuous by proving that a function from a simpler space $X$ is.

\bprop

    If $\rho\colon X\longto Y$ is surjective, continuous, and open (or closed), then it is a quotient map.

\eprop

\Proof if $\U\subseteq Y$ is open then $\rho^{-1}(\U)$ is open since $\rho$ is continuous.
If $\rho^{-1}(\U)$ is open then $\rho\circ\rho^{-1}(\U)=\U$ is open since $\rho$ is.
If $\rho$ is closed, then $\rho^{-1}(\U)$ is open if $\rho^{-1}(\U)^c=\rho^{-1}(\U^c)$ is closed, which is implies $\rho\circ\rho^{-1}(\U^c)=\U^c$ is closed, and therefore $\U$ is open.
\qed

This shows us that the projection maps $\pi_\alpha$ are quotient maps.

\bprop

    If $\rho\colon X\longto Y$ is a continuous bijection, it is a quotient map if and only if it is a homeomorphism.

\eprop

\Proof if $\rho$ is a homeomorphism then by above it is a quotient map.
If $\rho$ is a quotient map then $\rho(\U)$ is open if and only if $\rho^{-1}\circ\rho(\U)=\U$ is, so $\rho$ is an open map and thus a homeomorphism.
\qed

\bprop

    If $\rho\colon X\longto Y$ is continuous and there exists an $A\subseteq X$ such that $\rho\bigl|_A\colon A\longto Y$ is a quotient map, then $\rho$ itself is a quotient map.

\eprop

\Proof $\rho$ is surjective since its restriction is.
Now suppose $\U\subseteq Y$ such that $\rho^{-1}(\U)$ is open.
Then $\rho\bigl|_A^{-1}(\U)=\rho^{-1}(\U)\cap A$ is open with respect to $A$, and so $\U$ is open since $\rho\bigl|_A$ is a quotient map.
\qed

\bdefn

    Let $X$ be a topological space and $\sim$ an equivalence relation on $X$.
    Then we define $\rho\colon X\longto\slfrac X\sim$ by $\rho(x)=[x]_{\sim}$, then this is a quotient map with respect to the topology
    $\set{\U\subseteq\slfrac X\sim}[\hbox{$\rho^{-1}(\U)$ is open in $X$}]$.
    $\slfrac X\sim$ equipped with this topology is called the {\emphcolor quotient topology} of $X$ with respect to $\sim$.

\edefn

Since $\rho$ is a quotient map, everything we proved above about quotient maps holds true for it.

Let us say that $f\colon X\longto Y$ {\it preserves} $\sim$ if $a\sim b$ implies $f(a)=f(b)$.
And $f$ {\it strongly preserves} $\sim$ if $a\sim b$ is equivalent to $f(a)=f(b)$.
Then we have

\bthrm

    Let $f\colon X\longto Y$ be continuous.
    Then there exists a continuous $F\colon\slfrac X\sim\longto Y$ such that $f=F\circ\rho$ if and only if $f$ preserves $\sim$.
    $F$ is injective if and only if $f$ strongly preserves $\sim$.

\ethrm

\Proof if $f=F\circ\rho$ then $a\sim b$ implies $\rho(a)=\rho(b)$ and so $f(a)=F\rho(a)=F\rho(b)=f(b)$, so $f$ preserves $\sim$.
And if $f$ preserves $\sim$, then define $F[a]=f(a)$.
This is well-defined since if $[a]=[b]$ then $a\sim b$ so $f(a)=f(b)$.
And we showed that since $\rho$ is a quotient map, $F$ is continuous if and only if $F\circ\rho$ is already.

If $F$ is injective, suppose $f(a)=f(b)$ then this means $F[a]=F[b]$, so $[a]=[b]$, meaning $a\sim b$, by injectivity.
So $f$ strongly preserves $\sim$.
And if $f$ strongly preserves $\sim$, then $F[a]=F[b]$ implies $f(a)=f(b)$ so $a\sim b$, and so $[a]=[b]$.
Meaning $F$ is injective.
\qed

This theorem gives us the following commutative diagram:

\medskip
\centerline{
\def\diagrowheight{1cm}
\def\diagcolwidth{1cm}
\drawdiagram{
    $X$\cr
    $\slfrac X\sim$&$Y$\cr
}{%
    \diagarrow{from={1,1}, to={2,1}, text=$\rho$, x distance=-.25cm}
    \diagarrow{from={1,1}, to={2,2}, text=$f$, x distance=.4cm, slide=.4}
    \diagarrow{from={2,1}, to={2,2}, text=$F$, y distance=-.25cm}
}}
\medskip

\bprop

    If $f\colon X\longto Y$ is a quotient map that strongly preserves $\sim$ if and only if $F$ is a homeomorphism.

\eprop

\Proof if $f$ is a quotient map and strongly preserves $\sim$, then $F$ is injective.
And since $f=F\circ\rho$ and $\rho$ and $f$ are quotient maps, so is $F$.
So $F$ is an injective quotient map, which is a homeomorphism.

And if $F$ is a homeomorphism, then since $f=F\circ\rho$, $f$ is a quotient map as well since homeomorphisms are quotient maps.
Since $F$ is injective, it trivially strongly preserves $\sim$ as well.
\qed

