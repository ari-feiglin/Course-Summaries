\bdefn

    Let $X$ be a topological space, and for every $a,b\in X$ define $\Gamma_{ab}$ to be the set of all paths from $a$ to $b$, which are continuous maps $I\longto X$.
    On $\Gamma_{ab}$ we take the equivalence relation of homotopy relative to $\partial I=\set{0,1}$.
    Take $\hat\Gamma_{ab}$ to be the partition defined by this relation, ie. $\hat\Gamma_{ab}=\slfrac{\Gamma_{ab}}{\buildrel\partial I\over\sim}$.

    If $[\gamma]\in\hat\Gamma_{ab}$ and $[\delta]\in\hat\Gamma_{bc}$ then we define $[\gamma][\delta]\coloneqq[\gamma*\delta]$ (their concatenation).

\edefn

We must show that this is well-defined, meaning we must show that if $\gamma\hdi\gamma'$ and $\delta\hdi\delta'$ then $\gamma*\delta\hdi\gamma'*\delta'$.
So let $H\colon I\times I\longto X$ be a homotopy relative to $\partial I$ between $\gamma$ and $\gamma'$, and $G\colon I\times I\longto X$ between $\delta$ and $\delta'$.
Then define
$$ K(s,t) \coloneqq \cases{H(2s,t) & $0\leq s\leq\frac12$\cr G(2s-1,t) & $\frac12\leq s\leq1$} $$
this is continuous, $K(0,t)=H(0,t)=0$ and $K(1,t)=G(1,t)=1$ so it is a homotopy between the concatenations relative to $\partial I$.

Notice that concatenation is not necessarily associative, since in $(\gamma*\delta)*\epsilon$, the speed of $\gamma$ and $\delta$ is quadrupled while in $\gamma*(\delta*\epsilon)$, $\gamma$'s speed is only
doubled.
But it is the case that $[\gamma]\bigl([\delta][\epsilon]\bigr)=\bigl([\gamma][\delta]\bigr)[\epsilon]$, so in homotopy concatenation is associative.
So we need to prove $\gamma(\delta\epsilon)\hdi(\gamma\delta)\epsilon$, the idea behind this is that for every $x$ and $y$ where $\gamma(\delta\epsilon)(x)=(\gamma\delta)\epsilon(y)$, define in $I\times I$
the line between $(x,0)$ and $(y,1)$.
These lines cover $I\times I$ and for every point $(t,s)$ which is on the line from $(x,0)$ map it to $\gamma(\delta\epsilon)(x)$.

We can prove in a similar manner that for $\gamma\in\Gamma_{ab}$, $[K_a][\gamma]=[\gamma][K_b]=[\gamma]$.

And so we have defined a category.
The objects of this category are the points $a\in X$ and the morphisms between $a$ and $b$ are $\hat\Gamma_{ab}$ (notice that $[\gamma]\in\hat\Gamma_{ab}$ can be composed with elements from
$\hat\Gamma_{bc}$, so the order of composition is reversed).
Here the identity morphisms are $[K_a]$.

Notice that every morphism in this category is an isomorphism.
This is since for every $\gamma\in\Gamma_{ab}$ we defined its reverse $\overline\gamma\in\Gamma_{ba}$ by $\overline\gamma(t)\coloneqq\gamma(1-t)$.

\bprop

    $[\gamma][\overline\gamma]=[K_a]$ and $[\overline\gamma][\gamma]=[K_b]$.

\eprop

The idea is that at time $t$ we take the path $\gamma$ but not all the way, then wait, then take the reverse path $\overline\gamma$.
So
$$ H(x,t) = \cases{\gamma(2x) & $0\leq x\leq\frac{1-t}2$\cr \gamma(1-t) & $\frac{1-t}2\leq x\leq\frac{1+t}2$\cr \gamma(2-2x) & $\frac{1+t}2\leq x\leq1$} $$
is a homotopy from $\gamma*\overline\gamma$ to $K_a$.
\qed

\bdefn

    A {\emphcolor groupoid} is a small category (a category whose objects form a set, not a pure class) such that every morphism is an isomorphism.
    If ${\cal C}$ is a groupoid, then $\Mor(A,A)$ is then a group for every $A\in{\cal C}$.

\edefn

\bdefn

    Given a pointed topological space $(X,a)$ (call $a$ the basis point), define the {\emphcolor first homotopy group} (or the {\emphcolor fundamental group}) $\pi_1(X,a)\coloneqq\hat\Gamma_{aa}$.
    And given a morphism $f\colon(X,a)\longto(Y,b)$ (meaning $f$ is continuous and $f(a)=b$), then we define a group homomorphism $f_*\colon\pi_1(X,a)\longto\pi_1(Y,b)$ by $f_*([\gamma])=[f\circ\gamma]$.
    The correspondence $(X,a)\mapsto\pi_1(X,a)$ and $f\mapsto f_*$ is a functor.

\edefn

We need to show that $f_*$ is well-defined and also a group homomorphism.
To show that it is well-defined, suppose $\gamma\hdi\delta$, then we must show $f\circ\gamma\hdi f\circ\delta$.
Now we showed that if $f\htopic Af'$ and $g\htopic Bg'$ such that $f(A),f'(A)\subseteq B$ then $g\circ f\htopic Ag'\circ f'$.
And we have that $\gamma\hdi\delta$ and $f\htopic{\set a}f$ so $f\circ\gamma\hdi f\circ\delta$ as required.
Now we must show that $f_*$ is a homomorphism, ie.
$$ f_*([\gamma][\delta]) = [f\circ(\gamma*\delta)] = [(f\circ\gamma)*(f\circ\delta)] = f_*([\gamma])f_*([\delta]) $$
Actually a stronger result holds, $f\circ*(\gamma*\delta)=(f\circ\gamma)*(f\circ\delta)$, as both are given by
$$ \cases{f\circ\gamma(2t) & $0\leq t\leq\frac12$\cr f\circ\delta(2t-1) & $\frac12\leq t\leq1$} $$

To finish the proof that the correspondence is a functor, we need to show that $(g\circ f)_*=g_*\circ f_*$ and $({\rm id}_X)_*={\rm id}_{\pi_1(X,a)}$.
We do so directly:
$$ (g\circ f)_*([\gamma]) = [(g\circ f)\circ\gamma] = [g\circ(f\circ\gamma)] = g_*([f\circ\gamma]) = g_*\circ f_*([\gamma]) $$
and
$$ ({\rm id}_X)_*([\phi]) = [{\rm id}_X\phi] = [\phi] $$
so $({\rm id}_X)_*={\rm id}_{\pi_1(X,a)}$ as required.
Thus we have defined a functor from the category of pointed topological spaces to the category of groups.

\bprop

    Let $X$ be a topological space, $a\in X$, and $A$ be $a$'s connected component.
    Let $\iota\colon A\longto X$ be the inclusion map, then $\iota_*\colon\pi_1(A,a)\longto\pi_1(X,a)$ is an isomorphism.

\eprop

\Proof $\iota_*$ is injective: $\iota_*([\gamma])=[K_a]$ if and only if $[\iota\circ\gamma]=[K_a]$, which means $\iota\circ\gamma\hdi K_a$, let $H$ be such a homotopy.
Then $H\colon I\times I\longto X$, but we claim that $H$'s image is contained in $A$ so that it is also a homotopy $\gamma\hdi K_a$, meaning $[\gamma]=[K_a]$.
Suppose not, that $H(t_0,s_0)\notin A$, then define $\delta(t)=H(t_0\cdot t,s_0)$.
Then $\delta(0)=H(0,s_0)=a$ since $H$ is a homotopy relative to $\partial I$, and $\delta(1)=H(t_0,s_0)\notin A$, so $a$ is connected to a value not in $A$, in contradiction.
So $\iota_*$ is indeed injective.

Now suppose $[\gamma]\in\pi_1(X,a)$, meaning $\gamma$ is a path connecting $a$ to itself in $X$.
But every point in $\gamma$'s image must also be connected to $a$, meaning $\gamma$ is a path connecting $a$ to itself contained within $A$.
So there exists a $\gamma'\in\Gamma_{aa}^A$ such that $\gamma=\iota\circ\gamma'$ and in particular $\iota_*([\gamma'])=[\gamma]$ as required.
So $\iota_*$ is a bijective homomorphism, an isomorphism.
\qed

Suppose $a,b\in X$ such that there exists a path between them, $\gamma\colon I\longto X,\gamma(0)=a,\gamma(1)=b$.
Let us define
$$ F_\gamma\colon\pi_1(X,a)\longto\pi_1(X,b),\qquad F_\gamma[\phi] = [\overline\gamma*\phi*\gamma] =[\overline\gamma][\phi][\gamma] $$
so $F_\gamma[\phi]$ is the class of curves homotopic to the curve obtained by walking from $b$ along $\gamma$ backward to $a$, then going back along $\gamma$ to $b$.
Notice that $F_\gamma[\phi]=[\gamma]^{-1}[\phi][\gamma]$, so $F_\gamma$ is simply conjugation by $[\gamma]$.

In general, suppose ${\cal G}$ is a groupoid and let $A,B\in{\cal G}$ such that $\Mor(A,B)\neq\varnothing$.
Then $\Mor(A,A)$ and $\Mor(B,B)$ are isomorphic groups.
Let $\phi\in\Mor(A,B)$ and define $F_\phi\colon\Mor(A,A)\longto\Mor(B,B)$ by $F_\phi(\varkappa)=\phi\circ\varkappa\circ\phi^{-1}$.
This is a group homomorphism:
$$ F_\phi(\varkappa_1)\circ F_\phi(\varkappa_2) = \phi\circ\varkappa_1\circ\phi^{-1}\circ\phi\circ\varkappa_2\circ\phi^{-1} = \phi\circ(\varkappa_1\circ\varkappa_2)\circ\phi^{-1} = F_\phi(\varkappa_1\circ\varkappa_2) $$
and it is injective:
$$ F_\phi(\varkappa) = 1_B \iff \phi\circ\varkappa\circ\phi^{-1} = 1_B \iff \phi\circ\varkappa=\phi \iff \varkappa = \phi^{-1}\circ\phi = 1_A $$
and it is surjective: let $\varkappa'\in\Mor(B,B)$ and define $\varkappa\coloneqq\phi^{-1}\circ\varkappa'\circ\phi$, it is clear $F_\phi(\varkappa)=\varkappa'$.
It is clear that $F_\phi^{-1}=F_{\phi^{-1}}$ by this.

Our $F_\gamma$ above is precisely this $F_{[\gamma]}$ defined in the groupoid of first homotopy groups above $X$, meaning it is an isomorphism between $\pi_1(X,a)$ and $\pi_1(X,b)$.
And so $F_\gamma^{-1}=F_{\overline\gamma}$.
Let us summarize this:

\bprop

    Let $a,b\in X$ be two points in $X$ connected by a path $\gamma$.
    Then $\pi_1(X,a)$ and $\pi_1(X,b)$ are isomorphic.

\eprop

\bprop

    Suppose $H\colon I\times I\longto X$ is a homotopy from the closed loop $\phi$ to the closed loop $\psi$ such that for all $t\in I$, $H(0,t)=H(1,t)$.
    Define the path $\gamma(t)=H(0,t)=H(1,t)$, then
    $$ [\psi] = [\overline\gamma][\phi][\gamma] $$

\eprop

\Proof this is equivalent to saying
$$ [\overline\psi][\overline\gamma][\phi][\gamma] = [K_p] \iff \overline\psi*\overline\gamma*\phi*\gamma \hdi K_p $$
Now, $\overline\psi*\overline\gamma*\phi*\gamma$ is a curve $I\longto X$ whose endpoints are the same, so it can be viewed as a curve $S^1\longto X$.
And it can be extended to the curve $H$ on $D^2$ (since the unit disc and unit square are homeomorphic), which we showed above is equivalent to $\overline\psi*\overline\gamma*\phi*\gamma$ being
null-homotopic relative to any point on $S^1$, so we can choose the point which corresponds to $0$ and $1$.
\qed

\bthrm

    Let $f,g\colon X\longto Y$ be homotopic with homotopy $H$.
    Define $\gamma(t)\coloneqq H(a,t)$, so $\gamma(0)=f(a)$ and $\gamma(1)=g(a)$.
    Then $g_*=F_\gamma\circ f_*$ (recall that $f_*\colon\pi_1(X,a)\longto\pi_1(Y,f(a))$ and $g_*\colon\pi_1(X,a)\longto\pi_1(Y,g(a))$).

\ethrm

\Proof let $[\phi]\in\pi_1(X,a)$ then
$$ F_\gamma(f_*[\phi]) = [\overline\gamma][f\circ\gamma][\gamma],\qquad g_*[\phi] = [g\circ\phi] $$
Define $K(s,t)\coloneqq H(\phi(s),t)$ which is continuous and
$$ K(s,0) = H(\phi(s),0) = f\circ\phi(s),\quad K(s,1) = H(\phi(s),1) = g\circ\phi(s),\quad K(0,t) = H(\phi(0),t) = H(a,t) = K(1,t) $$
so by the above proposition, since $K$ is a homotopy from the closed loop $f\circ\phi$ to the closed loop $g\circ\phi$,
$$ [g\circ\phi] = [\overline\gamma][f\circ\phi][\gamma] $$
as required.
\qed

Notice then that $g_*=f_*$ if and only if $F_\gamma={\rm id}$ (requiring $f(a)=g(a)=b$).
This happens when $[\gamma]=[K_b]$ for example, which can happen when $\gamma=K_b$.
Ie. if $H(a,t)=b$ for all $t\in I$, then $f_*=g_*$.
But notice that this is simply the condition for $H$ to be a homotopy relative to $\set a$, so

\bprop

    If $f\htopic{\set a}g$ then $f_*=g_*$.

\eprop

\bthrm

    If $f\colon X\longto Y$ is a homotopy equivalence, then $f_*\colon\pi_1(X,a)\longto\pi_1(Y,f(a))$ is an isomorphism of groups.

\ethrm

\Proof there exists a $g\colon Y\longto X$ such that $g\circ f\sim{\rm id}_X$ and $g\circ f\sim{\rm id}_Y$.
Now since $g\circ f\sim{\rm id}_X$, then by above $(g\circ f)_*=F_\gamma\circ({\rm id}_X)_*=F_\gamma$, since ${(\cdot)}_*$ is a functor, $g_*\circ f_*=F_\gamma$.
Now recall that $F_\gamma=g_*\circ f_*\colon\pi_1(X,a)\longto\pi_1(X,g(f(a)))$ is an isomorphism.
And similarly $f\circ g\sim{\rm id}_Y$, so there exists an $F_\delta$ such that $(f\circ g)_*=f_*\circ g_*=F_\delta\colon\pi_1(Y,f(a))\longto\pi_1(X,f(g(f(a))))$ is an isomorphism.

Recall that if $f\circ g$ is injective, then $g$ is injective, and if $f\circ g$ is surjective, $f$ is surjective.
Thus $f_*$ is injective and surjective, meaning it is a bijective homomorphism, an isomorphism.
\qed

Thus if $X$ is contractible, then it is homotopic to the singleton space (an exercise in homework), $\set b$.
Thus $\pi_1(X,a)\cong\pi_1(\set b,b)$, and $\pi_1(\set b,b)$ has a single point, meaning $\pi_1(X,a)$ is the trivial group.

\bcoro

    If $X$ is contractible, then $\pi_1(X,a)$ is trivial.

\ecoro

\bdefn

    Let $E,B$ be topological spaces, then a map $p\colon E\longto B$ is a {\emphcolor covering map} (or just a {\it covering}) if
    \benum
        \item $p$ is surjective,
        \item There exists an open cover $\set{\U_\alpha}_{\alpha\in I}$ of $B$ such that for every $\alpha\in I$, $p^{-1}\U_\alpha\subseteq E$ is equal to the disjoint union of open sets
        $\bigdcup_{\beta\in J}\tilde\U_\beta$ such that $p$ forms a homeomorphism from $\tilde\U_\beta$ to $\U_\alpha$.
        For the sake of conciseness, such an open cover will be called an {\emphcolor open cover for $p$}.
    \eenum

\edefn

For example, take $B=S^1$ and $E={\bb R}$.
Define $p(t)\coloneqq(\cos(2\pi t),\sin(2\pi t))$, or if we identify $S^1$ with $S^1=\set{z\in{\bb C}}[\abs z=1]$, then $p(t)=e^{2\pi it}$.
This is like spiraling the real line so that all the integers are all on a vertical line (they all map to $1\in S^1$), and projecting this spiral onto a plane to form the circle $S^1$.
$p$ is obviously a surjective continuous map.
For every $e^{2\pi it}$, let $\epsilon<2\pi$ then the preimage of the open set ${\cal U}_t\coloneqq\set{e^{2\pi is}}[t-\epsilon<s<t+\epsilon]$ is
$\bigdcup_{n\in{\bb Z}}\bigl(n+(t-\epsilon,t+\epsilon)\bigr)$ and every $n+(t-\epsilon,t+\epsilon)$ is homeomorphic (via $p$) to this ${\cal U}_t$.

\bdefn

    If $p\colon E\longto B$ is a covering, $F\colon X\longto B$ a map, a {\emphcolor lift} of $F$ is a function $\tilde F\colon X\longto E$ such that $F=p\circ\tilde F$.

\edefn

\blemm

    Let $p\colon E\longto B$ be a covering, $F\colon X\times I\longto B$ be a map such that there exists an initial lift of $F\bigl|_{X\times\set0}$, $\tilde F\colon X\times\set0\longto E$, then there is a
    unique extension of $\tilde F$ to a lift $X\times I\longto E$.

\elemm

\Proof let $\set{\U_\alpha}_{\alpha\in J}$ be a covering of $B$ for $p$, and $x_0\in X$ and $t\in I$.
Then $(x_0,t)\in F^{-1}(\U_\alpha)$ for some $\alpha\in J$, and since $X\times I$ is a product space, there exists an open neighborhood $N_t$ of $x_0$ and $(a_t,b_t)$ a neighborhood of $t$ such that
$N_t\times(a_t,b_t)\subseteq F^{-1}(\U_\alpha)$ so $F(N_t\times(a_t,b_t))\subseteq\U_\alpha$.
Now $\set{N_t\times(a_t,b_t)}_{t\in I}$ forms an open cover of $\set{x_0}\times I$, and therefore it has a finite subcover and $\set{(a_t,b_t)}_{t\in I}$ has a Lebesgue number.
Let $N$ be the (finite) intersection of the $N_t$ in the subcover, and $0=t_0<\cdots<t_n=1$ be a partition of $I$ where the difference between subsequent $t_i$s is less than the Lebesgue number.
This means that $N\times[t_i,t_{i+1}]\subseteq N_t\times(a_t,b_t)$ and thus is mapped into a single $\U_\alpha$, let it be denoted by $\U_i$.

Let us assume inductively that $\tilde F$ has been constructed on $X\times[0,t_i]$, we will extend it to $X\times[0,t_{i+1}]$.
Since $p$ is a covering, there exists a $\tilde\U_i$ homeomorphic by $p$ to $\U_i$ which contains $\tilde F(x_0,t_i)$ (which is in $p^{-1}F(x_0,t_i)\in p^{-1}\U_i$).
We can assume that $\tilde F(N\times\set{t_i})$ is contained within $\tilde\U_i$ by replacing $N\times\set{t_i}$ with its intersection with $\tilde F\bigl|_{N\times\set{t_i}}^{-1}(\tilde\U_i)$.
Then since $p\circ\tilde F=F$, we must have that on $N\times[t_i,t_{i+1}]$, $\tilde F=p^{-1}\circ F$ for the restriction $p\colon\tilde\U_i\longto\U_i$.
So define $\tilde F\bigl|_{N\times[t_i,t_{i+1}]}=p^{-1}\circ F\bigl|_{N\times[t_i,t_{i+1}]}$ where $p$ is the restriction.
Since $\tilde F(N\times\set{t_i})$ is contained within $\tilde\U_i$, its definition in $[t_{i-1},t_i]$ must agree with this definition, so $\tilde F$ is well-defined and continuous.
\qed

\bthrm

    Suppose $p\colon E\longto B$ is a covering, $b\in B$ and $e\in E$ such that $p(e)=b$.
    Now suppose $\gamma\colon I\longto B$ is a curve starting at $b$, $\gamma(0)=b$.
    Then there exists a unique lift of $\gamma$ which starts at $e$, ie. a curve $\hat\gamma^e\colon I\longto E$ such that $\hat\gamma^e(0)=e$.

\ethrm

\Proof using the lemma above, take $X$ to be the trivial space of a single point so we can ignore it in the proof and result of the lemma above.
So the initial $\hat\gamma^e$ ($\tilde F$ in the proof) is a single point, which we can define to be $e$ (which is an initial lift since $p(e)=b$).
And by the lemma, $\gamma$ has a unique lift which extends this.
\qed

\bprop

    Let $p\colon E\longto B$ be a covering, $\gamma,\delta\colon I\longto B$ are curves such that $\gamma\hdi\delta$.
    Let $a=\gamma(0)=\delta(0)$, and let $e\in E$ such that $p(e)=a$.
    Then $\hat\gamma^e\hdi\hat\delta^e$, and in particular $\hat\gamma^e(1)=\hat\delta^e(1)$; both of the curves finish on the same point.

\eprop

\Proof let $H\colon I\times I\longto B$ be a homotopy relative to $\partial I$ from $\gamma$ to $\delta$.
We can then lift $H$ to a homotopy $\hat H\colon I\times I\longto E$, using the lemma above with $X=I$ we must have that the initial lift satisfies $p\circ\hat H(t,0)=H(t,0)=\gamma(t)$ so
$\hat H(t,0)=\hat\gamma^e(t)$ is the initial lift.
And similarly $p\circ\hat H(t,1)=H(t,1)=\delta(t)$ so $\hat H(t,1)=\hat\delta^e(t)$ by the uniqueness of lifts (since $t\mapsto\hat H(t,1)$ forms a lift of $\delta$).
And $p\circ\hat H(0,s)=H(0,s)=a$, we can view this as the curve $K_a$ from $a$ to $a$, and so by uniqueness we have again that $\hat H(0,s)=\hat K_a^e(s)=e$.
Similar for $\hat H(1,s)$.
Thus $\hat H$ is a homotopy from $\hat\gamma^e$ to $\hat\delta^e$ relative to $\partial I$.
\qed

Let $p\colon E\longto B$ be a covering, $a\in B$, and $p(e)=a$.
Then we define a function $F\colon\pi_1(B,a)\longto p^{-1}(a)$ by $F([\gamma])\coloneqq\hat\gamma^e(1)$.
This is well-defined by the previous proposition (it is in $p^{-1}(a)$ since $p\circ\hat\gamma^e=\gamma$, so $p\circ\hat\gamma^e(1)=a$).

\bprop

    \benum
        \item If $E$ is path connected, then $F$ is surjective.
        \item If $E$ is simply connected (see homework $2+3$, for every two $a,b\in E$ every two paths between them are homotopic relative to $\partial I$), then it is also injective (it is bijective).
    \eenum

\eprop

\Proof
\benum
    \item Let $p(x)=a$ then $x$ and $e$ are connected since $E$ is path connected, so let $\delta$ be a path $e$ to $x$, then $\widehat{p\circ\delta}^e=\delta$ by uniqueness.
        And so $F[p\circ\delta]=\delta(1)=x$.
        So $F$ is surjective.
    \item Suppose $F[\gamma]=F[\delta]$, then $\hat\gamma^e(1)=\hat\delta^e(1)$.
        Of course $\hat\gamma^e(0)=\hat\delta^e(0)=e$, and so by simple connectivity, $\hat\gamma^e\hdi\hat\delta^e$ since they begin at the same point and end at the same point.
        Now recall that $\gamma=p\circ\hat\gamma^e$ and since homotopy respects the composition of homotopic functions, $\gamma\hdi\delta$.
        \qed
\eenum

So for example, the covering $p\colon{\bb R}\longto S^1$ by $t\mapsto e^{2\pi it}$ is a covering from a simply connected space (since contractible implies simply connected, homework $2$) to $S^1$, this
means that $F\colon\pi_1(S^1,e^{2\pi it})\longto\set{t+n}[n\in{\bb Z}]\cong{\bb Z}$ is a bijection.
Since the fundamental group of a simply connected space is trivial, $S^1$ is not simply connected.

Notice that $F^{-1}$ can be viewed as mapping a curve $\gamma$ which starts at $e$ and ends at $b\in p^{-1}(a)$ to $[p\circ\gamma]$.
So for example, $\pi_1(S^1,1)$ is $\set{[\hbox{the curve which winds around the circle $n$ times}]}[n\in{\bb Z}]$.
This is since a curve from $0$ to $n$ in ${\bb R}$ is mapped to a curve which winds around the circle $n$ times (for negative $n$ this winds in the opposite direction).
Notice that $F$ is a group isomorphism here, since concatenating two curves which wind around the circle $n$ and $m$ times gives a curve which winds around the circle $n+m$ times.
So $\pi_1(S^1)\cong{\bb Z}$.

Now, since $S^1$ is a deformation retract of ${\bb R}^2\setminus\set0$, so the inclusion map is a homotopic equivalence and thus defines an isomorphism of their fundamental groups.
So $\pi_1({\bb R}^2\setminus\set0)\cong\pi_1(S^1)\cong{\bb Z}$ as well.

Now suppose $A\subseteq X$ is a retract, meaning there exists a retraction $r\colon X\longto A$ such that $r\circ\iota={\rm id}_A$.
So $\iota_*\colon\pi_1(A,a)\longto\pi_1(X,a)$ and $r_*\colon\pi_1(X,a)\longto\pi_1(A,a)$ and since this is a functor, $r_*\circ\iota_*=(r\circ\iota)_*=({\rm id}_A)_*={\rm id}_{\pi_1(A,a)}$.
So $\iota_*$ is injective.
Let us summarize this:

\bprop

    If $A\subseteq X$ is a retract, then $\iota_*\colon\pi_1(A,a)\longto\pi_1(X,a)$ is a monomorphism (injective).

\eprop

Now, for example $S^1=\partial D^2\subseteq D^2$ is not a retract: if it were then $\iota_*$ would be a monomorphism $\pi_1(S^1,1)\longto\pi_1(D^2,1)$.
But $\pi_1(S^1,1)\cong{\bb Z}$ and $\pi_1(D^2,1)=1$ since $D^2$ is contractible ($H(x,t)=(1-t)x$).
And there is no monomorphism ${\bb Z}\longto1$.

\bthrm[title=Brouwer Fixed-Point Theorem (for $D^2$), name=fpthrm]

    If $f\colon D^2\longto D^2$ is continuous, then it has a fixed point: a point $x\in D^2$ such that $f(x)=x$.

\ethrm

\Proof suppose not, then $f(x)\neq x$ for all $x$.
Using this $f$ we will construct an $r\colon D^2\longto S^1$ such that $r\circ\iota={\rm id}_{S^1}$, meaning then $S^1$ would be a retract of $D^2$, which we just showed it is not.
For $x\in D^2$, look at the segment which begins at $f(x)$ to $x$, and set $r(x)$ to be the intersection of this line with the segment.
The segment is $f(x)+t(x-f(x))$ and so we want to solve (setting $f(x)=(f_1,f_2)$ and $x=(x_1,x_2)$)
$$ \bigl(f_1+t(x_1-f_1)\bigr)^2 + \bigl(f_2+t(x_2-f_2)\bigr)^2 = 1 $$
which can be solved, and gives a $t$ which is continuous in $f$ and $x$.
Notice that if $x\in S^1$ then by definition $r(x)=x$ since the intersection of the segment with the boundary of the circle is $x$.
So $r$ is indeed a retract, in contradiction.
\qed

\blemm

    Let $f,g\colon X\longto{\bb C}\setminus\set0$ such that $\abs{f(x)-g(x)}<\abs{g(x)}$ for all $x\in X$.
    Then $f$ and $g$ are homotopic.

\elemm

\Proof define $H(x,t)=(1-t)g(x)+tf(x)$, then $\abs{H(x,t)}=\abs{(1-t)g(x)+tf(x)}\geq\abs{g(x)}-t\abs{f(x)-g(x)}>0$ so $H(x,t)$ is indeed a homotopy to ${\bb C}\setminus\set0$.
\qed

\bthrm[title=The Fundamental Theorem of Algebra]

    If $p(z)$ is a non-constant polynomial over ${\bb C}$, then it has a root.

\ethrm

\Proof we can assume $p(z)=z^n+a_{n-1}z^{n-1}+\cdots+a_1z+a_0$ for $n>0$.
Suppose that $p(z)\neq0$ for all $z\in{\bb C}$, so $p$ is a map ${\bb C}\longto{\bb C}\setminus\set0$.
For $r>0$ define $S_r=\set{z}[\abs z=r]$ and $D_r=\set{z}[\abs z\leq r]$ so that $\partial D_r=S_r$.
Then $p\bigl|_{S_r}\colon S_r\longto{\bb C}\setminus\set0$, and $p\bigl|_{D_r}$ is an extension of $p\bigl|_{S_r}$ to $D_r$, so $p\bigl|_{S_r}$ is null-homotopic.
Define $g(z)=z^n$, then $p(z)-g(z)=a_{n-1}z^{n-1}+\cdots+a_0$, then
$$ \frac{\abs{p(z)-g(z)}}{\abs{g(z)}} \leq \frac{\abs{a_{n-1}}\abs{z^{n-1}}+\cdots+\abs{a_0}}{\abs{z^n}} = \abs{a_{n-1}}\frac1{\abs z}+\cdots+\abs{a_0}\frac1{\abs{z^n}} $$
let $r=\abs z$, then for $r$ large enough this becomes less than $1$, so $\abs{p(z)-g(z)}<\abs{g(z)}$ on $S_r$, so $p\bigl|_{S_r}\sim g\bigl|_{S_r}$.

Now recall that for $a\in X$, if $f\sim g$ then $f_*=0$ if and only if $g_*=0$, so since $p\bigl|_{S_r}$ is null-homotopic this means $\parens{g\bigl|_{S_r}}_*$ is trivial.
But notice that $g_*\colon\pi_1(S_r)\longto\pi_1(S_{r^n})$ maps the generator of $S_r$ (which is the loop which wraps around $S_r$ once) to the loop which wraps around $S_{r^n}$ $n$ times.
So $g_*$ (the restriction) maps $1$ to $n$, which is not a trivial homomorphism.
\qed

