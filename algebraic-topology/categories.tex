\bdefn

    A {\emphcolor category} ${\cal C}$ is a mathematical object which contains the following
    \benum
        \item a class of objects $\obof{\cal C}$ (the objects need not be sets),
        \item for every two objects $A,B\in\obof{\cal C}$ a class of {\emphcolor morphisms} $\Morof{A,B}$,
        \item an operation on morphisms $\circ$, where for every $f\in\Morof{A,B}$ and $g\in\Morof{B,C}$, $g\circ f\in\Morof{A,C}$,
        \item for every object $A\in\obof{\cal C}$ there exists an identity morphism $1_A\in\Morof{A,A}$ where for every $A,B\in\obof{\cal C}$ and $f\in\Morof{A,B}$, $f\circ 1_A=1_B\circ f=f$,
        \item for every $A,B,C,D\in\obof{\cal C}$ and $f\in\Morof{A,B},g\in\Morof{B,C},h\in\Morof{C,D}$, there is associativity: $(h\circ g)\circ f=h\circ (g\circ f)$.
    \eenum

\edefn

Although morphisms are not necessarily functions, we use similar notation: both $f\colon A\longto B$ and $A\xvarrightarrow{\,f\,}B$ are to be understood to mean $f\in\Morof{A,B}$.
And we write $A\in{\cal C}$ to mean $A\in\obof{\cal C}$.

Notice that for every $A\in{\cal C}$, $1_A$ is unique: suppose $1_A$ and $1_A'$ are both identity morphisms then $1_A\circ1_A'=1_A$ since $1_A'$ is an identity, but $1_A\circ1_A'=1_A'$ since $1_A$ is an
identity so $1_A=1_A'$.

\bdefn

    Suppose ${\cal C}$ and ${\cal D}$ are two categories, a {\emphcolor functor} $F$ from ${\cal C}$ to ${\cal D}$ is a correspondence where for every $A\in{\cal C}$ there is defined a single
    $F(A)\in{\cal D}$, and for every $f\in\Morof{A,B}$ there exists a unique $F(f)\in\Morof{F(A),F(B)}$ such that for all $A,B,C\in{\cal C}$ and $f\in\Morof{A,B}$ and $g\in\Morof{B,C}$ we have that
    $F(g\circ f)=F(g)\circ F(f)$ and $F(1_A)=1_{F(A)}$.

\edefn

\bexam

    The following are examples of categories:
    \benum
        \item The category of all groups, morphisms are taken to be homomorphisms between groups;
        \item The category of all topological spaces, morphisms are taken to be homeomorphisms;
        \item The category of all sets, the morphisms are taken to be set functions;
        \item The category of pairs of topological spaces: the objects are of the form $(X,A)$ where $X$ is a topological space and $A\subseteq X$.
            Morphisms between $(X,A)$ and $(Y,B)$ of this category are continuous functions $f$ between $X$ and $Y$ such that $f(A)\subseteq B$.
        \item The category of pointed topological spaces: the objects are $(X,a)$ where $X$ is a topological space and $a\in X$ and the morphisms between $(X,a)$ and $(Y,b)$ are continuous functions between
            $X$ and $Y$ such that $a\mapsto b$.
    \eenum
    An example of a functor is the so-called {\it forgetful functor} from the category of topological spaces to the category of sets: map a topological to itself as a pure set.

\eexam

This course will focus on a specific functor between the category of pointed topological spaces to the category of groups.

\bdefn

    Let ${\cal C}$ be a category, and $A,B\in{\cal C}$.
    A morphism $f\colon A\longto B$ is an {\emphcolor isomorphism} if there exists a morphism $g\colon B\longto A$ such that $g\circ f=1_A$ and $f\circ g=1_B$.
    Such a $g$ is called the {\emphcolor inverse} of $f$ and is denoted $f^{-1}$ (notice that by symmetry the inverse is also an isomorphism).
    If there exists an isomorphism between $A$ and $B$, we denote this by $A\cong B$ and $A$ and $B$ are called {\emphcolor isomorphic}.

\edefn

Inverses are unique: if $g_1$ and $g_2$ are inverses of $f$ then $(g_1\circ f)\circ g_2=1_A\circ g_2=g_2$ but $g_1\circ(f\circ g_2)=g_1\circ 1_B=g_1$ and by associativity these are equal.
Furthermore the composition of isomorphisms is an isomorphism: it is easily verified that $(f\circ g)^{-1}=g^{-1}\circ f^{-1}$.
Notice that $1_A$ is an isomorphism and it is its own inverse.

\bprop

    A functor maps isomorphisms to isomorphisms, in particular $F(f^{-1})=F(f)^{-1}$ if $f\colon A\longto B$ is an isomorphism.

\eprop

\Proof notice that $F(f)\circ F(f^{-1})=F(f\circ f^{-1})=F(1_B)=1_{F(B)}$ and $F(f^{-1})\circ F(f)=F(f^{-1}\circ f)=F(1_A)=1_{F(A)}$.
So $F(f^{-1})$ is indeed the inverse of $F(f)$.
\qed


