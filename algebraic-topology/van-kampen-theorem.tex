\bthrm[title=Van Kampen, name=vankampen]

    Let $X$ be a topological space, and $\U,\V$ be open such that $X=\U\cup\V$ and $\U\cap\V$ is non-empty and path connected.
    Choose a basepoint $a\in\U\cap\V$, we have the diagram of topological spaces

    \medskip
    \centerline{
    \drawdiagram{
        &$(\U,a)$\cr
        $(\U\cap\V,a)$&&$(X,a)$\cr
        &$(\V,a)$\cr
    }{
        \diagarrow{from={2,1}, to={1,2}, y distance=.25cm, text=$i$, color=rgb{.1 .1 .8}}
        \diagarrow{from={1,2}, to={2,3}, y distance=.25cm, text=$k$, color=rgb{.1 .1 .8}}
        \diagarrow{from={2,1}, to={3,2}, y distance=-.25cm, text=$j$, color=rgb{.1 .1 .8}}
        \diagarrow{from={3,2}, to={2,3}, y distance=-.25cm, text=$\ell$, color=rgb{.1 .1 .8}}
    }}

    Where $i,j,k,\ell$ are inclusion maps.
    This then induces the following commutative diagram

    \medskip
    \centerline{\def\diagcolwidth{1cm}\def\diagrowheight{.75cm}\drawdiagram{
        &$\pi_1(\U)$\cr
        $\pi_1(\U\cap\V)$&$\pi_1(\U)*_{\pi_1(\U\cap\V)}\pi_1(\V)$&&$\pi_1(X)$\cr
        &$\pi_1(\V)$\cr
    }{
        \diagarrow{from={2,1}, to={1,2}, color=rgb{.1 .1 .8}, text=$i_*$, x distance=-.4cm}
        \diagarrow{from={2,1}, to={3,2}, color=rgb{.1 .1 .8}, text=$j_*$, x distance=-.4cm}
        \diagarrow{from={1,2}, to={2,4}, color=rgb{.1 .1 .8}, text=$k_*$, y distance=.25cm}
        \diagarrow{from={3,2}, to={2,4}, color=rgb{.1 .1 .8}, text=$\ell_*$, y distance=-.25cm}
        \diagarrow{from={2,2}, to={2,4}, color=rgb{.1 .1 .8}, text=$L$, y distance=-.25cm}
        \diagarrow{from={1,2}, to={2,2}, color=rgb{.1 .1 .8}}
        \diagarrow{from={3,2}, to={2,2}, color=rgb{.1 .1 .8}}
    }}

    Van Kampen's theorem then states that $L$ is an isomorphism.

\ethrm

\Proof in our proof we will assume $\U$ and $\V$ are path connected.
Let $x\in X$, then take $\gamma_x$ to be a path from $a$ to $x$ such that
\benum
    \item $\gamma_a$ is constant,
    \item if $x\in\U$ then $\gamma_x$ is contained within $\U$,
    \item if $x\in\V$ then $\gamma_x$ is contained within $\V$,
\eenum
Thus if $x\in\U\cap\V$, $\gamma_x$ is contained within $\U\cap\V$.
If $\phi$ is a path in $X$ from $x$ to $y$, define $\hat\phi\coloneqq\gamma_x*\phi*\overline\gamma_y$ a loop on $a$.
Notice that if $\phi$ is contained within $\U$, so is $\hat\phi$ and similar for $\V$.
Then if $[\phi]\in\pi_1(X,a)$ then $\hat\phi=\gamma_a*\phi*\overline\gamma_a=K_a*\phi*K_a$ so $[\phi]=[\hat\phi]$.

If $y_0,\dots,y_n\in X$ and $\phi_1,\dots,\phi_n$ are paths where $\phi_i$ is a path from $y_{i-1}$ to $y_i$.
Thus $\phi_1*\cdots*\phi_n$ is defined and a path from $y_0$ to $y_n$.
Then
$$ \varwidehat{\phi_1*\cdots*\phi_n} = \gamma_{y_0}*\phi_1*\overline\gamma_{y_1}*\cdots*\gamma_{y_{n-1}}*\phi_n*\overline\gamma_{y_n} $$
and
$$ \hat\phi_1*\cdots*\hat\phi_n = \gamma_{y_0}*\phi_1*\overline\gamma_{y_1}*\gamma_{y_1}*\phi_2*\cdots*\gamma_{y_{n-1}}*\phi_n*\overline\gamma_{y_n} $$
thus we have that adjacent $\overline\gamma_{y_i}$ and $\gamma_{y_i}$s cancel out modulo homotopy, so
$$ \bigl[\varwidehat{\phi_1*\cdots*\phi_n}\bigr] = [\hat\phi_1]\cdots[\hat\phi_n] $$

Now, we defined $L'\colon\pi_1(\U)*\pi_1(\V)\longto\pi_1(X)$ by $L'[x_1\cdots x_n]=y_1\cdots y_n$ where $y_i=k_*(x_i)$ if $x_i\in\pi_1(\U)$ and otherwise $y_i=\ell_*(x_i)$.
Since $k_*$ just makes loops in $\pi_1(\U)$ be viewed as loops in $\pi_1(X)$, $L'$ maps like so:
$$ \bigl([\phi_1],\dots,[\phi_n]\bigr) \xvarmapsto{\quad L'\quad} [\phi_1]\cdots[\phi_n] $$

We now claim that $L$ is surjective.
Let $[\phi]\in\pi_1(X,a)$, so $\phi^{-1}\U,\phi^{-1}\V$ is an open cover of $I$, so it has a Lebesgue number.
Meaning we can take a partition of $I$ into closed intervals $I_i$ whose lengths are less than that of the Lebesgue number.
So $I_i$ is contained in either $\phi^{-1}\U$ or $\phi^{-1}\V$ for every $i$, so $\phi(I_i)$ is contained in either $\U$ or $\V$.
Let us denote $\phi_i$ the restriction of $\phi$ to the loop on $I_i$, so $\phi=\phi_1*\cdots*\phi_n$.
Thus
$$ [\phi] = [\hat\phi] = [\hat\phi_1]\cdots[\hat\phi_n] $$
And $\hat\phi_i$ are contained entirely in $\U$ or $\V$: since if $\phi_i$ is in $\U$ then so is $\gamma_{\phi_i(0)}$ and $\gamma_{\phi(1)}$ and thus so is $\hat\phi_i$.

So let us focus on the word $([\hat\phi_1],\dots,[\hat\phi_n])$ in $\pi_1(\U)*\pi_1(\V)$ and its image under $L'$ is $[\hat\phi]=[\phi]$.
Thus $L'$ is surjective, and therefore so is the induced homomorphism $L$.

We now claim that $L$ is injective.
Suppose that $L'$ maps $([\phi_1],\dots,[\phi_n])$ to $[\phi_1]\cdots[\phi_n]=1$.
Thus there exists a loop homotopy from $\phi_1*\cdots*\phi_n$ to $K_a$, $H$.
Since $H^{-1}\U,H^{-1}\V$ forms an open cover of the compact metric space $I\times I$, it has a Lebesgue number.
So we can take a partition of $I\times I$ into squares $I_{i,j}$.
We can assume that the lengths of all $I_{i,j}$ are equal and the total number of squares on each edge is divisible by $n$.
We can do this since if the number of squares is not divisible by $n$, we can refine the squares.
Thus each $\phi_i$ can be written as $\phi_i^1*\cdots*\phi_i^k=\phi_i$ where $\phi_i^j$ is the restriction of $\phi_i$ to $I_{1,j}$.
$k$ is the number of squares on each edge.
Every $\phi_i^j$ is contained within either $\U$ or $\V$ since $I_{i,j}$ is.
\qed

So for example, take the figure $8$ space.
We can take $\U$ to be the top circle plus a small neighborhood of the middle point and $\V$ to be the bottom circle with a small part of the top circle.
Then $\U\cap\V$ looks like an {\sf x} meaning it deformation contracts to a point, so $\pi_1(\U\cap\V)=1$.
And $\U$ deformation retracts onto the top circle and similarly $\V$ deformation retracts to the bottom circle, so $\pi_1(\U)=\pi_1(\V)=F_1$ the free group generated by a single element (which is isomorphic
to ${\bb Z}$, but we don't want to refer to its Abelianness).
Thus by van Kampen, $\pi_1(8)=\pi_1(\U)*_1\pi_2(\V)=F_1*F_1=F_2$.
Thus the figure $8$ space's fundamental group is isomorphic to $F_2$.

\bdefn

    If $\set{(X_i,x_i)}_{i\in I}$ are pointed topological spaces, we define their {\emphcolor wedge sum} to be the quotient space of $\coprod_{i\in I}X_i$ under the equivalence $x_i\sim x_j$.
    This is denoted $\bigvee_{i\in I}X_i$ (the basepoints $x_i$ are implicit).

\edefn

Notice that $8$ is just $S^1\vee S^1$, so we have shown $\pi_1(S^1\vee S^1)=F_2$.

