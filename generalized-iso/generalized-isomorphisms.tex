\input pdfToolbox
\input preamble

\parindent=\z@
\parskip=3pt plus 1pt

\setlayout{horizontal margin=2cm, vertical margin=2cm}

{\vbox{\leftskip=0pt plus 1fill\relax\rightskip=\leftskip\setfontandscale{bf}{25pt}%
Generalizations of the Isomorphism Theorems\par}
\smallskip
\centerline{\it Ari Feiglin}
}

\bigskip
\hbox to\hsize{\hfil\vbox{\hsize=.7\hsize
\leftskip=0pt plus 1fil \rightskip=\leftskip \parfillskip=\z@
\hrule
\kern5pt

In this lecture we will discuss certain generalizations of the three isomorphism theorems from algebra, and the relation of these generalizations to their algebraic counterparts.
The lecture will discuss certain basic notions of universal algebra, without delving too deep.

\kern5pt
\hrule
}\hfil}

\section{The Isomorphism Theorems}

Recall the following three group isomorphism theorems:

\bthrm[title=The First Group Isomorphism Theorm]

    Let $f\colon G\longto H$ be a group homomorphism, then $G/\ker f\cong{\rm im}f$.

\ethrm

\bthrm[title=The Second Group Isomorphism Theorm]

    Let $G$ be a group, $H\leq G$ a subgroup, and $N\normaleq G$ a normal subgroup.
    Then $N$ is normal in $HN$ and $H\cap N$ is normal in $H$ and
    $$ \slfrac{HN}N \cong \slfrac H{H\cap N} $$

\ethrm

\bthrm[title=The Third Group Isomorphism Theorm]

    Let $G$ be a group, $N\normaleq G$ a normal subgroup.
    Then subgroups of $G/N$ are precisely $H/N$ for $N\leq H\leq G$, and normal subgroups are $K/N$ for $N\leq K\normaleq G$, furthermore
    $$ \slfrac{G/N}{K/N} \cong \slfrac GK $$

\ethrm

This family of three theorems have corresponding results in ring theory:

\bthrm[title=The First Ring Isomorphism Theorm]

    Let $f\colon R\longto S$ be a ring homomorphism, then $R/\ker f\cong{\rm im}f$.

\ethrm

\bthrm[title=The Second Ring Isomorphism Theorm]

    Let $R$ be a group, $S\leq R$ a subring, and $I\normaleq R$ an ideal.
    Then $I$ is an ideal in $S+I$ and $S\cap I$ is an ideal in $S$ and
    $$ \slfrac{S+I}I \cong \slfrac S{S\cap I} $$

\ethrm

\bthrm[title=The Third Ring Isomorphism Theorm]

    Let $R$ be a ring, $I\normaleq R$ an ideal.
    Then subrings of $R/I$ are precisely $S/I$ for $I\leq S\leq I$, and ideals are $J/I$ for $I\leq J\normaleq R$, furthermore
    $$ \slfrac{R/I}{K/I} \cong \slfrac RJ $$

\ethrm

The statements and proofs of the ring isomorphism theorems are very similar to those of the group isomorphism theorems.
This begs the question, can we generalize these theorems enough to encompass groups, rings, modules, and more?
We turn to universal algebra for this.

\section{Signatures and Structures}

We begin by defining what signatures and structures are:

\bdefn

    A {\emphcolor signature} is an object $\sigma$ consisting of two disjoint sets: a set $\F$ of function symbols, and $\R$ of relation symbols.
    In addition, it has a function $\ar\colon(\F\cup\R)\longto{\bb N}_{\geq0}$, called the {\emphcolor arity function}.
    Functions with zero arity are called {\emphcolor constants}, and relations with zero arity are called {\emphcolor propositions}.

\edefn

\bdefn

    Let $\sigma$ be a signature, then a {\emphcolor $\sigma$-structure} is an object $\c A$ consisting of the following three items:
    \benum
        \item A non-empty set $A$, called the {\emphcolor domain} of $\c A$.
        \item For every function symbol $f\in\sigma$, a function $f^{\c A}\colon A^{\ar f}\longto A$.
        \item For every relation symbol $r\in\sigma$, a relation $r^{\c A}\subseteq A^{\ar r}$.
    \eenum

\edefn

For example, let $\sigma=\set\circ$ where $\circ$ is a binary function symbol.
Then $\sigma$-structures are {\it magmas}, this includes semigroups, monoids, and groups.
We can also have a signature $\sigma=\set{\circ,{}^{-1},1}$, where ${}^{-1}$ is a unary function symbol, and $1$ is a constant.

So for example we can define a model $\c Z=({\bb Z},+,-,0)$ as such a $\sigma$-model.

We can also take a signature of ordered groups $\sigma=\set{\circ,<}$ where $<$ is a binary relation.
$\c Z$ with the standard order is a $\sigma$-structure.

All these examples are pretty well structured, they satisfy some axioms regarding the signature (e.g. $\forall x,y,z. (x\circ y)\circ z=x\circ(y\circ z)$).
But we need not have a $\sigma$-structure satisfy anything, all we require is that it associates to each a symbol a function or relation.
What the structure satisfies is the subject of mathematical logic and beyond the scope of this lecture.

\bdefn

    Let $\c A$ be a $\sigma$-structure and $B\subseteq\c B$ such that for every $f\in\sigma$ of arity $n$ and $\vec b\in B^n$, $f\vec b\in B$, then we can make $\c B$ into a $\sigma$-structure where
    $f^{\c B}=f^{\c A}\bigl|_{B^n}$ and $r^{\c B}=r^{\c A}\cap B^n$.
    Such a structure is called a {\emphcolor substructure} of $\c A$.

\edefn

So for example, the only substructures of $\c Z$ are subgroups $n\c Z$.
But if we take $\c Z$ over the reduced signature $\c Z=({\bb Z},+)$ then ${\bb N}$ is also a substructure.
So the signature of a structure is very significant.

To state the isomorphism theorems, we need some sense of a homomorphism:

\bdefn

    Let $\c A$ and $\c B$ be $\sigma$-structures, then a {\emphcolor homomorphism} from $\c A$ to $\c B$ is a function $h\colon A\longto B$ such that
    \benum
        \item For every $f\in\sigma$ and $\vec a\in A^n$, $h(f^{\c A}\vec a)=f^{\c B}(h\vec a)$ where $h\vec a=(ha_1,\dots,ha_n)$.
        For constant symbols this means $hc^{\c A}=c^{\c B}$.
        \item For every $r\in\sigma$ then $r^{\c B}h\vec a$ if and only if there exists a $\vec c\in A^n$ such that $h\vec a=h\vec c$ and $r^{\c A}\vec c$.
        Ignore the definition for propositional symbols.
    \eenum
    Bijective homomorphisms are isomorphisms, etc.

\edefn

Notice that if $h$ is injective, then the condition on relations just says $r^{\c B}h\vec a\iff r^{\c A}\vec a$.

\blemm

    The image of a homomorphism is a substructure of the codomain.

\elemm

We also need some notion of quotienting:

\bdefn

    Let $\c A$ be a $\sigma$-structure, then a {\emphcolor congruence} on $\c A$ is an equivalence relation $\sim$ on $A$ such that for every $f\in\sigma$, if $\vec a\sim\vec b$ (meaning $a_i\sim b_i$
    for every $i$), then $f^{\c A}\vec a\sim f^{\c A}\vec b$.

    If $\sim$ is a congruence on $\c A$ then define the {\emphcolor quotient structure} $\c A/{\sim}$ as follows:
    \benum
        \item The domain is the partition $A/{\sim}$.
        \item For $f\in\sigma$ and $a\in A$, $f^{\c A/{\sim}}\vec a_\sim = (f^{\c A}\vec a)_\sim$, this is well defined by congruence.
        \item For $r\in\sigma$, $r^{\c A/{\sim}}\vec a_\sim$ if and only if there exists a $\vec b\sim\vec a$ such that $r^{\c A}\vec b$.
    \eenum

\edefn

Notice that if $N\normaleq G$ is a normal subgroup, then $a\sim b\iff ab^{-1}\in N$ is a congruence.
This is because if $(a_1,a_2)\sim(b_1,b_2)$ then $a_1a_2(b_1b_2)^{-1}=a_1a_2b_2^{-1}b_1^{-1}\in a_1Nb_1^{-1}=a_1b^{-1}N=N$.
So $a_1a_2\sim b_1b_2$, as required.
Similar for ${}^{-1}$ (for those who know logic, this is because ${}^{-1}$ is definable by $\circ$).

Then the quotient group $G/N$ is precisely the quotient structure $G/{\sim}$.
Notice: $a_\sim b_\sim = (ab)_\sim$ which is the same as saying $aN\,bN=abN$, since $aN=a_\sim$.

\section{The Isomorphism Theorems}

Notice that if $h\colon\c A\longto\c B$ is a homomorphism, then we can define a congruence $\sim_h$ on $\c A$ by $x\sim_hy$ if and only if $hx=hy$.
This is indeed a congruence: if $\vec x\sim_h\vec y$ then $hf^{\c A}\vec x=f^{\c B}h\vec x=f^{\c B}h\vec y=hf^{\c A}\vec y$ so $f^{\c A}\vec x\sim_hf^{\c A}\vec y$.
This congruence is called the {\it kernel} of the homomorphism.

Notice that if $h\colon G\longto H$ is a homomorphism of groups, then $a\sim_hb$ if and only if $ha=hb$ if and only if $h(ab^{-1})=1$, which is if and only if $ab^{-1}\in\ker h$.
So $G/\ker h=G/{\sim_h}$ by our above discussion.

\bthrm[title=The First Isomorphism Theorem]

    \benum
        \item Let $\sim$ be a congruence on $\c A$, then $\rho\colon\c A\longto\c A/{\sim}$ defined by $a\mapsto a/{\sim}$ is an epimorphism.
        \item Let $h\colon\c A\longto\c B$ be a homomorphism of $\sigma$-structures, then $\iota\colon\c A/{\sim_h}\longto h\c A$ defined by $a/{\sim_h}\mapsto ha$ is an isomorphism.
    \eenum

\ethrm

\Proof
\benum
    \item This is obviously surjective.
    Now to prove it is a homomorphism:
    $$ \rho f\vec a = (f\vec a)/{\sim} = f(\vec a/{\sim}) = f\rho\vec a $$
    and $r\rho\vec a$ if and only if $r(\vec a/{\sim})$ if and only if there exists a $\vec b\sim\vec a$ such that $r\vec b$, and this is if and only if $\rho\vec b=\rho\vec a$.
    So $\rho$ is a homomorphism.

    \item Notice that if $a/{\sim_h}=b/{\sim_h}$ then $ha=hb$ by definition, so $\iota$ is well-defined.
    $\iota$ is also trivially a bijection.
    It is a homomorphism since
    $$ \iota f(\vec a/{\sim_h}) = \iota(f\vec a/{\sim_h}) = hf\vec a = fh\vec a = f\iota(\vec a/{\sim_h}) $$
    and
    $$ r\iota(\vec a/{\sim_h}) \iff rh\vec a \iff \exists\vec b\sim\vec a.r\vec b \iff r(\vec a/{\sim_h}) \qed $$
\eenum

This gives us the classic first isomorphism theorem diagram:

\bigskip
\centerline{\def\diagcolbuf{1cm}\def\diagrowbuf{1cm}\drawdiagram{
    $\c A$ & $\c B$\cr
           & $\c A/{\sim_h}$\cr
}{
    \diagarrow{from={1,1}, to={1,2}, text=$h$, y distance=.25cm}
    \diagarrow{from={1,1}, to={2,2}, text=$\rho$, y distance=.25cm}
    \diagarrow{from={2,2}, to={1,2}, text=$\iota$, x distance=.25cm}
}}

This was pretty straightforward.
Generalizing the next two isomorphism theorems requires a few more definitions.

\medskip
Recall the second isomorphism theorem: if $H\leq G$ is a subgroup and $N\normaleq G$ is a normal subgroup then $N\normaleq HN\leq G$, $H\cap N\normaleq H$, and $(HN)/N\cong H/(H\cap N)$.
Our counterpart to the concept of normal subgroups in structures are congruences, which are relations on the underlying domain.
So how do we define the product of a substructure and a congruence?

Well, notice that $a\in HN$ if and only if there exists an $h\in H$ such that $a\sim_Nh$ (where $\sim_N$ is the congruence induced by $N$).
So we can also define $HN=\set{a\in G}[\exists h\in H.\,a\sim_Nh]$.
We generalize this

\bdefn

    Let $\c A$ be a $\sigma$-structure, $\c B\leq\c A$ a substructure, and $\sim$ a congruence on $\c A$.
    Define the {\emphcolor $\sim$-closure} of $\c B$ to be $\c B_\sim=\set{a\in A}[\exists b\in B.\,a\sim b]$.

\edefn

And of course the induced congruence on $H$ by $H\cap N$ is just $\sim_N\cap H^2$, i.e. $\sim_N\bigr|_H$.
So we would expect

\bthrm[title=The Second Isomorphism Theorem]

    Let $\sigma$ be an algebraic signature, $\c A$ be a $\sigma$-structure, $\c B\leq\c A$ a substructure, and $\sim$ a congruence on $\c A$.
    Then
    \benum
        \item The $\sim$-closure $\c B_\sim$ is a substructure of $\c A$,
        \item $\c B_\sim/{\sim}\cong\c B/{\sim}\bigr|_{\c B}$.
        (The $\sim$ under $\c B_\sim$ should be restricted to $\c B_\sim$, but since $b\sim a\iff b\sim\bigr|_{\c B}a$ for $b\in\c B_\sim$, we omit the subscript.)
    \eenum

\ethrm

\Proof
\benum
    \item Let $f\in\sigma$ and $\vec a\in\c B_\sim$, then $\vec a\sim\vec b\in\c B$, so $f\vec a\sim f\vec b\in\c B$ so $f\vec a\in B_\sim$.
    \item Define
    $$ h\colon\c B\longto\slfrac{\ds\c B_\sim}{\ds\sim},\qquad hb = b/{\sim} $$
    This is a homomorphism:
    $$ hf\vec b = (f\vec b)/\sim = f(\vec b/\sim) = f(h\vec b) $$
    And its kernel is $a\sim_hb$ if and only if $a\sim b$, so the kernel is $\sim_h\bigr|_{\c B}$ as required.
    \qed
\eenum

Why must the signature be algebraic?
Well, otherwise $h$ may not have been a homomorphism:
$$ rh\vec b \iff r(\vec b/{\sim}) \iff \exists\vec a\sim\vec b.\,r\vec a $$
But suppose no $\vec b\in\c B$ satisfies $r\vec b$, but there exists a $\vec a\sim\vec b$ such that $r\vec a$.
Then we don't have that there exists a $\vec b'$ such that $h\vec b'=h\vec b$ and $r\vec b'$, yet we do have $rh\vec b$.

For example, here is a simple counterexample: let our signature contain a single unary relation $r$ and define $\c A=\set{a,b}$, $\c B=\set b$ where only $r^{\c A}a$.
Define $a\sim b$, so we have that $\c B_\sim=\c A$, so $\c B_\sim/{\sim}=\set{\set{a,b}}$ and $\c B/{\sim\bigr|_B}=\set{\set b}$.
But $r^{\c A/{\sim}}\set{a,b}$ since $r^{\c A}a$, but we don't have that $\c B_\sim/{\sim}\cong\c B/{\sim\bigr|_B}$.

This shouldn't be surprising: quotienting should have weird effects on relations.
For example, let us look at $({\bb Z},<)$: if we take the congruence $\sim_n$ corresponding to $n{\bb Z}$ over ${\bb Z}$ then $a/\sim_n<b/\sim_n$ if and only if there exists $a'\sim_na$ and $b'\sim_nb$ such
that $a'<b'$.
But this is always true since the congruence classes are all infinite.
So $<^{{\bb Z}/n{\bb Z}}$ is just the trivial relation.

This may not be all that surprising, after all how do you define an order on ${\bb Z}/n{\bb Z}$?
But it does mean that while a structure may satisfy some theory, its quotient need not.
Both of these examples have demonstrated this.

The issue is congruences don't take relations into account.
Let us fix this.

\bdefn

    A {\emphcolor total congruence} is a congruence $\sim$ on a $\sigma$-structure $\c A$ such that for all $\vec a,\vec b$, if $\vec a\sim\vec b$ then $r\vec a\iff r\vec b$ for all $r\in\sigma$.

\edefn

Notice that when quotienting by a total congruence, $r(\vec a/{\sim})\iff r\vec a$ since if $\vec a\sim\vec b$ and $r\vec b$ then $r\vec a$ by definition.

\bthrm[title=The Second Isomorphism Theorem{,} Relational-Style]

    Let $\sim$ be a total congruence on $\c A$ (which need not be algebraic), and $\c B\leq\c A$ a substructure.
    Then
    $$ \c B_\sim/{\sim}\cong\c B/{\sim}\bigr|_{\c B} $$

\ethrm

\Proof all we need to show is that the $h$ defined before is a homomorphism.
$rh\vec b$ if and only if $r(\vec b/{\sim})$ if and only if $r\vec b$.
\qed

\medskip
For the third isomorphism theorem, we have the following: let $N\normaleq G$, then
\benum
    \item Every subgroup of $G/N$ is of the form $K/N$ for $N\leq K\leq G$ (and vice versa).
    \item Every normal subgroup of $G/N$ is of the form $K/N$ for $N\leq K\normaleq G$ (and vice versa).
    \item If $N\leq K\normaleq G$ then $(G/N)/(K/N)\cong G/K$.
\eenum

But here $(G/N)/(K/N)$ is a quotient structure of a quotient structure, so we must somehow deal with this.
Notice that
$$ aN\cdot\slfrac KN = bN\cdot\slfrac KN \iff ab^{-1}N \in \slfrac KN \iff ab^{-1} \in K $$
so we have that $a/{\sim_N}\sim_{K/N}b/{\sim_N}$ if and only if $a\sim_Kb$.

\bdefn

    Let $\c A$ be a $\sigma$-structure, $\sim_N$ and $\sim_K$ be congruences such that $\sim_N\subseteq\sim_K$.
    Then define their {\emphcolor quotient congruence} to be $\sim_{K/N}={\sim_K}/{\sim_N}$ over $\c A/{\sim_N}$ by
    $$ \slfrac{\ds a}{\ds\sim_N} \sim_{K/N} \slfrac{\ds b}{\ds \sim_N} \iff a\sim_Kb $$

\edefn

This is well-defined since if $a/{\sim_N}=b/{\sim_N}$ then $a\sim_Kb$ as well.
And this is obviously a congruence.

This helps us show the following:

\bthrm[title=The Third Isomorphism Theorem]

    Let $\c A$ be a $\sigma$-structure, and $\sim_N$ a congruence on $\c A$.
    \benum
        \item Every substructure of $\c A/{\sim_N}$ is of the form $\c B/{\sim_N}$ for some substructure satisfying $\c B_{\sim_N}=\c B$ and vice versa.
        \item Every congruence on $\c A/{\sim_N}$ is of the form $\sim_{K/N}$ for some congruence on $\c A$, $\sim_K\supseteq\sim_N$, and vice versa.
        \item $$ \slfrac{\slfrac{\ds\c A}{\ds\sim_N}}{\ds\sim_{K/N}} \cong \slfrac{\ds\c A}{\ds\sim_K} $$
    \eenum

\ethrm

\Proof
\benum
    \item Let $\c C$ be a substructure of $\c A/{\sim_N}$, then define $B=\set{a\in A}[a/{\sim_N}\in C]$.
    This defines a substructure of $\c A$, since for $f\in\sigma$ and $\vec a\in B^n$, $(f\vec a)/{\sim_N}=f(\vec a)/{\sim_N}\in C$ so $f\vec a\in B$.
    And we can see that if $a\in B$ and $a\sim_Nb$ then $a/{\sim_N}=b/{\sim_N}$, so $b/{\sim_N}\in C$ and so $b\in B$, thus $\c B_\sim=\c B$ as required.
    It is obvious that $\c B/{\sim_N}=\c C$.

    And conversely, if $\c B_\sim=\c B$ then $\c B/{\sim_N}$ is well-defined (since $\c B$ contains all its equivalence classes), and so from definition it is a substructure.

    \item Let $\sim$ be a congruence on $\c A/{\sim_N}$ then define $a\sim_Kb\iff (a/\sim_N)\sim(b/\sim_N)$, this obviously contains $\sim_N$.
    This a congruence: if $\vec a\sim_K\vec b$ then
    $$ (\vec a/\sim_N)\sim(\vec b/\sim_N) \implies f(\vec a/\sim_N) \sim f(\vec b/\sim_N) \implies (f\vec a/\sim_N)\sim(f\vec b/\sim_N) \implies f\vec a\sim_Kf\vec b $$
    And by definition $\sim_{K/N}=\sim$.

    \item Let us define
    $$ h\colon\c A\longto\slfrac{\ds\slfrac{\ds\c A}{\ds\sim_N}}{\ds\sim_{K/N}},\qquad a\mapsto(a/{\sim_N})/{\sim_{K/N}} $$
    This is indeed a homomorphism:
    $$ hf\vec a = (f\vec a/{\sim_N})/{\sim_{K/N}} = f\bigl((a/{\sim_N})/{\sim_{K/N}}\bigr) = fh\vec a $$
    And $rh\vec a$ is equivalent to $r(\vec a/{\sim_N})/{\sim_{K/N}}$ which is equivalent to there existing a $\vec b/{\sim_N}\sim_{K/N}\vec a/{\sim_N}$ such that $r(\vec b/{\sim_N})$.
    This is equivalent to $\vec b\sim_K\vec a$ such that $r(\vec b/{\sim_N})$.
    This is equivalent to there existing a $\vec c\sim_N\vec b$ such that $r\vec c$.
    Now, since $\vec c\sim_N\vec b\sim_K\vec a$, we have $\vec c\sim_K\vec a$ by transitivity, so $h\vec c=(\vec c/{\sim_N})/{\sim_{K/N}}=(\vec a/{\sim_N})/{\sim_{K/N}}=h\vec a$.
    So if $rh\vec a$ then there exists a $\vec c$ with $h\vec c=h\vec a$ and $r\vec c$.

    The converse is trivial (in general, since $h\vec a=h\vec b$ and $r\vec b$ means $rh\vec b$ so $rh\vec a$).
    \qed
\eenum

\section{Applications}

\subsection{Rings}

In this section, we will prove the well-known special cases of the isomorphism theorems for rings and modules.
All we need to show is that congruences, closures, and quotient congruences can be associated with their corresponding notions in these objects.

So for rings, we need to show that congruences can be associated with ideals.
Let $R$ be a ring, $\sim$ a congruence.
Then we claim that $I=0/{\sim}$ is an ideal: indeed $0\in I$, and if $a,b\sim0$ then $a+b\sim0+0=0$.
Finally if $r\in R$ and $a\in I$ then $r\sim r$ and $a\sim0$ so $ra\sim r0=0$ so $ra\in I$, as required.
Similar for $ar$.

Now if $I$ is an ideal, then $a\sim b\iff a-b\in I$ is a congruence.
This is because if $f$ is a homomorphism such that $\ker f=I$, then
$$ a\sim_fb \iff f(a)=f(b) \iff f(a-b) = 0 \iff a-b\in I $$
So $\sim_f=\sim$ and in particular $\sim$ is a congruence.

Now suppose $R$ is a ring, $S\leq R$ a subring, and $I$ an ideal.
Let $\sim$ be the congruence associated with $I$, then we claim $S+I=S_\sim$.
Indeed, if $s+i\in S+I$ then $s+i\sim s$ so $s+i\in S_\sim$, thus $S+I\subseteq S_\sim$.
And if $s\sim r$ then $r=s+i$.

And finally if $I\normaleq R$ is an ideal and $I\leq J$ is another ideal, then the congruence associated with $J/I$ is the quotient congruence of $J$ by the congruence of $I$.
I.e. we have that $\sim_{J/I}=\slfrac{\sim_J}{\sim_I}$.
So we must show that $a\sim_Jb\iff(a/{\sim_I})\sim_{J/I}(b/{\sim_I})$.
This is $a-b\in J\iff (a-b+I)\in J/I$ which is just $a-b\in J$ as required.

\subsection{Boolean Algebras}

\bdefn

    A {\emphcolor Boolean algebra} is a $\set{\wedge,\vee,\neg,0,1}$-structure satisfying the following axioms:
    \benum
        \item $\vee,\wedge$ are associative and commutative
        \item $a\vee(a\wedge b)=a$ and $a\wedge(a\vee b)=a$
        \item $\vee,\wedge$ distribute: $a\vee(b\wedge c)=(a\vee b)\wedge(a\vee c)$ and $a\wedge(b\vee c)=(a\wedge b)\vee(a\wedge c)$
        \item $a\vee0=a$ and $a\wedge1=a$
        \item $a\vee\neg a=1$ and $a\wedge\neg a=0$
    \eenum

\edefn

Suppose $\s B$ is a boolean algebra and $\sim$ a congruence on it.
Then let $I=0/{\sim}$ then it has the following properties:
\benum
    \item If $a,b\in I$ then $a\vee b\sim0\vee0=$ so $a\vee b\in I$
    \item If $a\in I$ and $x\in\s B$ then $a\wedge x\sim0\wedge x=0$ so $a\wedge x\in I$
\eenum

These two properties define what is called a {\it Boolean Ideal}, and every boolean ideal defines a congruence: let $I$ be a boolean ideal, then define $a\sim b\iff a\symdiff b\in I$.
It can be seen by computation that this is indeed a congruence.
So there is a correspondence between boolean ideals and congruences.
A {\it Prime Ideal} is a filter such that for every $a\in\s B$ either $a$ or $\neg a\in I$, equivalently it is a maximal ideal.

If we look at $F=1/{\sim}$, this has dual properties:
\benum
    \item If $a,b\in F$ then $a\wedge b\sim1\wedge1=1$ so $a\wedge b\in F$
    \item If $a\in F$ and $x\in\s B$ then $a\vee x\sim1\vee x=1$ so $a\vee x\in F$.
\eenum

A subset of $\s B$ satisfying these properties is called a {\it Filter}.
A filter $F$ defines a congruence by $a\sim b\iff a\oto b\in F$.
An {\it Ultrafilter} is a filter such that for every $a\in\s B$, either $a$ or $\neg a\in F$, equivalently it is a maximal filter.

Now, it can be shown that a substructure of a boolean algebra is itself a boolean algebra (for those who know logic, this is because the theory of boolean algebras is a $\forall$-theory).
So let $\s C\leq\s B$ be a subboolean algebra, $I$ a boolean ideal, and $\sim$ its congruence.
Then $\s C_\sim=\s C\symdiff I=\set{c\symdiff i}[c\in\s C,i\in I]$ since
$$ x\in\s C_\sim \iff \exists c\in\s C.\, x\symdiff c\in I \iff \exists c\in\s C.i\in I.\, x=c\symdiff i \iff x\in C\symdiff I $$
So we have that by the second isomorphism theorem,
$$ \slfrac{\s C}{\s C\cap I} \cong \slfrac{\s C\symdiff I}{I} $$

And by the third isomorphism theorem, every subboolean algebra of $\slfrac{\s B}I$ is of the form $\slfrac{\s C}I$ for $I\subseteq\s C$, and an ideal is of the form $\slfrac JI$ for $I\subseteq J$.
This is because if $\sim$ is a congruence on $\slfrac{\s B}I$ then there exists some $J\supseteq I$ ideal such that $a/I\sim b/I\iff a\sim_Jb$.
So $a/I\sim0/I$ if and only if $a\sim_J0$ if and only if $a\in J$ so the induced ideal is $J/I$.
And then the third isomorphism says
$$ \slfrac{\slfrac{\s B}I}{\slfrac JI} \cong \slfrac{\s B}J $$

\section{Ideals}

Notice that in all the objects discussed thus far there is some notion of an ideal.
In each instance, an ideal is just a congruence class $e/{\sim}$ which satisfies the following properties:
\benum
    \item $e/{\sim}$ uniquely defines the congruence, i.e. if $e/{\sim}=e/{\approx}$ then ${\sim}={\approx}$.
    \item For any substructure $\c B$, $e/{\sim}\subseteq\c B$ if and only if $\c B_\sim=\c B$.
\eenum
How can we get these properties in a general structure?
Notice that for all of these structures, and any congruence which defines an ideal $I$, $a\sim b$ if and only if $t(a,b)\in I$ for some function $t$ which can be defined using the function symbols in the
signature.
Such a function is called a {\it term} (or more specifically, a {\it term function}).

\bdefn

    Let $\sigma$ be a signature and $V$ some collection of variables (we assume such a collection is global, we don't really care too much what it is).
    Then a {\emphcolor $\sigma$-term} is defined recursively as follows:
    \benum
        \item Every constant $c\in\sigma$ is a term.
        \item Every variable $x\in V$ is a term.
        \item If $f\in\sigma$ has arity $n$, and $t_1,\dots,t_n$ are terms then $ft_1\cdots t_n$ is a term.
    \eenum
    We denote the set of all $\sigma$-terms by $\c T_\sigma$, and we omit the $\sigma$ when the signature is understood.
    For a term $t\in\c T_\sigma$, we write $t=t(\bar x)$ to mean that the variables in $t$ are contained in $\bar x$.

    If $\c A$ is a $\sigma$-structure, $t(\bar x)$ a $\sigma$-term, and $\bar a\in\c A^n$, we define $t(\bar a)\in\c A$ to be the result of substituting $\bar a$ for $\bar x$ in $t$.
    Recursively this is defined as follows:
    \benum
        \item If $t=c$ then $t(\bar a)=c^{\c A}$.
        \item If $t=x_i$ then $t(\bar a)=a_i$.
        \item Otherwise $t=ft_1\cdots t_n$ and so $t(\bar a)=f^{\c A}t_1(\bar a)\cdots t_n(\bar a)$.
    \eenum

\edefn

\blemm

    If $\c B\leq\c A$ are $\sigma$-structures, $t(\bar x)$ a $\sigma$-term, if $\bar b\in\c B$ then $t(\bar b)\in\c B$.

\elemm

\Proof this is done by term induction, it's not complicated.\qed

\blemm

    If $\sim$ is a congruence on $\c A$, $t(\bar x)$ a term, and $\bar a\sim\bar b\in\c A$ then $t(\bar a)\sim t(\bar b)$.

\elemm

\Proof again by term induction.\qed

\bdefn

    Call a $\sigma$-structure $\c A$ {\emphcolor idealized} if there exists a constant $e\in\sigma$, and two $\sigma$-terms ${\frak t}$ and $\bar{\frak t}$ such that:
    \benum
        \item For every congruence $\sim$: $a\sim b \iff \f t(a,b) \in \slfrac e{\sim}$.
        \item For every $a,b$, $\bar{\f t}(\f t(a,b),b)=a$.
            This means that $\bar{\f t}(\f t(\bullet,b),b)={\rm id}$, so $\bar{\f t}(\bullet,b)$ is surjective and $\f t(\bullet,b)$ is injective for every $b\in\c A$.
        \item $\f t(\bullet,b)$ is surjective (so it is a bijection), this is equivalent to $\bar{\f t}(\bullet,b)$ being a bijection since their composition is the identity.
    \eenum
    Call $e/{\sim}$ an {\emphcolor ideal} of $\sim$.

\edefn

\blist
    \item Groups are idealized: $\f t(a,b)=ab^{-1}$ and $\bar{\f t}(a,b)=ab$.
    \item Rings are idealized: $\f t(a,b)=a-b$ and $\bar{\f t}(a,b)=a+b$.
    \item Boolean algebras are idealized: $\f t(a,b)=\bar{\f t}(a,b)=a\symdiff b$ and $e=0$, or $\f t(a,b)=\bar{\f t}(a,b)=a\oto b$ and $e=1$.
\elist

\bthrm

    Let $\c A$ be idealized, then there is a bijective correspondence between congruences and ideals.

\ethrm

\Proof map $\sim$ to $e/{\sim}$.
This is obviously surjective by the definition of ideals, and it is injective since if $e/{\sim}=e/{\approx}$ then
$$ a\sim b \iff \f t(a,b)\in e/{\sim} = e/{\approx} \iff a\approx b $$
so this correspondence is bijective.
\qed

This means that if $I$ is an ideal of $\c A$, we can write $\c A/I$ for $\c A/{\sim}$ where $I=e/{\sim}$ and there can be no confusion.

\bprop

    If $\c A$ is idealized, then for any congruence $\sim$, all of its cosets (congruence classes) have the same cardinality.

\eprop

\Proof notice that for the identity congruence, $\f t(b,b)\in e/{=}$ means $\f t(b,b)=e$.
So $\f t(\bullet,b)$ maps $b/{\sim}\longto e/{\sim}$ since $\f t(b',b)\sim\f t(b,b)=e$.
It is a bijection, so $\abs{b/{\sim}}=\abs{e/{\sim}}$.
\qed

\bthrm

    Let $\c A$ be idealized, then $\c B_\sim=\bar{\f t}(e/{\sim},\c B)$.
    Therefore $e/{\sim}\subseteq\c B$ if and only if $\c B_\sim=\c B$.

\ethrm

\Proof let $a\in\c B_\sim$, so there exists a $b\in\c B$ such that $a\sim b$.
So $\f t(a,b)\in e/{\sim}$, then $\bar{\f t}(\f t(a,b),b)=a$ and this is in $\bar{\f t}(e/{\sim},\c B)$.
And for $\bar{\f t}(e',b)\in\bar{\f t}(e/{\sim},\c B)$, notice that $\f t(\bar{\f t}(e',b),b)=e'\in e/{\sim}$.
Thus $\bar{\f t}(e',b)\sim b\in\c B$.
So $\bar{\f t}(e',b)\in\c B_\sim$.

So if $e/{\sim}\subseteq\c B$ then $\c B_\sim=\bar{\f t}(e/{\sim},\c B)\subseteq\bar{\f t}(\c B,\c B)\subseteq\c B$.
And since $\c B\subseteq\c B_\sim$ we have the required result.
Conversely, since $e\in\c B$ as it is a constant, we always have that $e/{\sim}\subseteq\c B_\sim=\c B$.
\qed

Let us write $I\c B$ for $\bar{\f t}(I,\c B)$.
Now, we'd like to state the following:

\bthrm[title=The Idealized Second Isomorphism Theorem]

    Let $\c A$ be idealized, $\c B\leq\c A$ a substructure, and $I\normaleq\c A$ an ideal.
    Then $I$ is an ideal of $I\c B$ and $I\cap\c B$ is an ideal of $\c B$ and
    $$ \slfrac{I\c B}I \cong \slfrac{\c B}{I\cap\c B} $$

\ethrm

Unfortunately, just because $\c A$ is idealized doesn't mean that its substructures are.
So writing $\c B/(I\cap\c B)$ has no meaning.
In order to get this result, we turn to model theory.
The following is quite tedious and may have to be left for another lecture.
But first let us generalize the third isomorphism theorem for ideals

\bthrm[title=The Idealized Third Isomorphism Theorem]

    Let $\c A$ be idealized, $I\normaleq\c A$ an ideal.
    Then
    \benum
        \item $\c A/I$ is idealized.
        \item Substructures of $\c A/I$ are of the form $\c B/I$ for $I\subseteq\c B$.
        \item Ideals of $\c A/I$ are of the form $J/I$ for $I\subseteq J\normaleq\c A$.
        \item $\slfrac{\c A/I}{J/I}\cong\slfrac{\c A}J$
    \eenum

\ethrm

\Proof Let $\sim_N$ be the congruence associated with $I$.
Let $\sim_{K/N}$ be a congruence of $\c A/I$, then
Now suppose $\f t(a/I,b/I)\sim_{K/N} e/I$, that means that $\f t(a,b)/I\sim_{K/N} e/I$, so $\f t(a,b)\sim_Ke$ meaning $a\sim_Kb$.
And so $a/I\sim_{K/N} b/I$.
So $\c A/I$ is idealized, furthermore its ideals are $(e/I)/{\sim_{K/N}}$ and $e/I\sim_{K/N}e'/I$ if and only if $e\sim_Ke'$, so the ideal is of the form $J/I$ where $J$ is the ideal associated with
$\sim_K$.
$(2)$ and $(4)$ follow immediately.
\qed

\section{Theories Allowing For Ideals}

Let us quickly define the groundwork for first order logic.
We will gloss over the details due to lack of time.

\bdefn

    Given a signature $\sigma$, we define $\c L$ the set of all formulas recursively as follows:
    \benum
        \item If $t,s$ are $\sigma$-terms then $t\eq s\in\c L$.
        \item If $r\in\sigma$ is an $n$-ary relation and $t_1,\dots,t_n$ are $\sigma$-terms then $rt_1\cdots t_n\in\c L$.
        \item If $x\in V$ is a variable and $\phi\in\c L$ then $\forall x\phi\in\c L$.
        \item If $\phi,\psi\in\c L$ then $(\phi\land\psi),\neg\phi\in\c L$.
    \eenum

\edefn

\bdefn

    Let $\phi\in\c L$ be a formula, and suppose $x\in V$ has an occurrence in $\phi$.
    This occurrence is {\emphcolor bound} if it occurs within the scope of the quantifier $\forall x$, and otherwise {\emphcolor free}.
    Let $\free\phi$ be all the variables which occur free in $\phi$.
    Write $\phi=\phi(\bar x)$ if $\free\phi\subseteq\bar x$.

\edefn

\bdefn

    Let $\phi(\bar x)$ be a formula, $\c A$ a structure, and $\bar a$ a sequence of elements from $\c A$.
    Then we write $\c A\vDash\phi(\bar a)$ to mean that $\phi$ holds in $\c A$ when we valuate $\bar x$ as $\bar a$.
    Formally:
    \benum
        \item If $\phi=t(\bar x)\eq s(\bar x)$, then $\c A\vDash\phi(\bar a)$ if and only if $t(\bar a)=s(\bar a)$.
        \item If $\phi=rt_1(\bar x)\cdots t_n(\bar x)$, then $\c A\vDash\phi(\bar a)$ if and only if $r^{\c A}t_1(\bar a)\cdots t_n(\bar a)$.
        \item If $\phi=\forall y\psi(\bar x,y)$, then $\c A\vDash\phi(\bar a)$ if and only if for every $b\in\c B$, $\c A\vDash\phi(\bar a,b)$.
        \item $\c A\vDash(\phi\land\psi)(\bar a)$ if and only if $\c A\vDash\phi(\bar a),\psi(\bar a)$.
        \item $\c A\vdash\neg\phi(\bar a)\iff\c A\nvDash\phi(\bar a)$.
    \eenum
    $\vDash$ is called the {\emphcolor satisfaction relation}.

\edefn

\bdefn

    Let $X$ be a set of $\c L$-formulas, and $\phi\in\c L$.
    Then we say $X\vDash\phi$ if and only if for every $\c A\vDash X$, $\c A\vDash\phi$.
    This is called the {\emphcolor consequence relation}.

\edefn

\bdefn

    An $\c L$-theory $T$ is a set of $\c L$ sentences (formulas with no free variables) such that $T\vDash\phi\implies\phi\in T$.
    I.e. it is deductively closed.
    Let $X$ be a set of sentences, then there exists a smallest theory which contains it (its deductive close: $T_X=\set{\phi}[X\vDash\phi]$).
    $T_X$ is said to be {\emphcolor axiomatized} by $X$.

\edefn

Now for some model-theoretic results which we will not prove:

\bthrm[title=The Compactness Theorem]

    Let $X$ be a set of $\c L$-formulas, and $\phi\in\c L$.
    Then $X\vDash\phi$ if and only if there exists a finite $X_0\subseteq X$ such that $X_0\vDash\phi$.

\ethrm

\bdefn

    A {\emphcolor $\forall$ formula} is a formula of the form $\forall\vec x\phi$ where $\phi$ is quantifier-free.
    A theory is a $\forall$-theory if it can be axiomatized by a set of $\forall$-sentences.

\edefn

\bthrm

    A theory is invariant under substructures (i.e. $\c B\subseteq\c A\vDash T\implies\c B\vDash T$) if and only if it is a $\forall$-theory.

\ethrm

\bdefn

    A {\emphcolor positive formula} is one constructed only by $\land,\lor,\forall,\exists$ (i.e. without $\neg$).

\edefn

\bthrm

    A theory is invariant under homomorphic images if and only if it can be axiomatized by positive sentences.

\ethrm

\bdefn

    Say that a positive theory $T$ {\emphcolor allows for weak ideals} if there exists a constant $e$ and a term $\f t$ such that
    $$ T\vDash\forall a,b(\f t(a,b)\eq e\oto a\eq b) $$
    And $T$ {\emphcolor allows for ideals} if it is also universal and there exists a constant $e$ and terms $\f t,\bar{\f t}$ such that
    $$ T\vDash\forall a,b(\bar{\f t}(\f t(a,b),b)=a),\quad T\vDash\forall a,b(\f t(\bar{\f t}(a,b),b)=a),\quad T\vDash\forall a(\f t(a,a)=e) $$

\edefn

If $T$ allows for ideals then $\bar{\f t}(e,b)=\bar{\f t}(\f t(b,b),b)=b$.
And so if $T$ allows for ideals then it allows for weak ideals: if $\f t(a,b) = e$ then $\bar{\f t}(\f t(a,b),b) = \bar{\f t}(e,b) = b$.
And $\bar{\f t}(\f t(a,b),b)=a$, so $a=b$.

\bdefn

    Call $\c A$ {\emphcolor weakly idealized} by $e,\f t$ if for every congruence, $a\sim b\iff\f t(a,b)\sim e$.

\edefn

\bthrm

    If $T$ allows for weak ideals, then all of its models are weakly idealized.
    Similarly if $T$ allows for ideals, then all of its models are idealized.

\ethrm

\Proof let $\c A\vDash T$ be a model of $T$ and let $\sim$ be a congruence on $\c A$.
Then the canonical homomorphism $f\colon\c A\longto\c A/{\sim}$ is surjective and thus since $T$ is positive, $\c A/{\sim}\vDash T$.
Now,
$$ a\sim b \iff fa = fb \iff \f t(fa,fb) = e \iff f\f t(a,b) = e = fe \iff \f t(a,b) \sim e $$
as required.
And so if $T$ allows for ideals, its models are obviously all idealized.
\qed

If $T$ allows for ideals then it is closed under substructures, so substructures of its models are also idealized.
So we can now, in good conscience, restate the second isomorphism theorem:

\bthrm[title=The Idealized Second Isomorphism Theorem]

    Let $T$ allow for ideals, $\c A\vDash T$, $\c B\leq\c A$ a substructure, and $I\normaleq\c A$ an ideal.
    Then $I$ is an ideal of $I\c B$ and $I\cap\c B$ is an ideal of $\c B$ and
    $$ \slfrac{I\c B}I \cong \slfrac{\c B}{I\cap\c B} $$

\ethrm

Since the theory of groups, rings, modules, and boolean algebras all allow for ideals, we have successfully generalized all the isomorphism theorems in a way which takes ideals into account.

\bye

