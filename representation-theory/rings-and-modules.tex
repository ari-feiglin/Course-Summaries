\section{Rings and Modules}

\subsection{Finiteness properties of modules and rings}

\bdefn

    Let $M$ be an $R$-module.
    It is {\emphcolor Noetherian} (resp.\ {\emphcolor Artinian}) if it has no infinite strictly increasing (resp.\ decreasing) sequence of $R$-submodules.

\edefn

\bexam

    If $R$ is a $\bF$-algebra for some field $\bF$, then every $R$-module can be regarded as a $\bF$-vector space.
    In particular, if $M$ is finite-dimensional as a $\bF$-vector space, then it is both Noetherian and Artinian.
    Indeed, submodules are subspaces, and a chain of subspaces must have distinct dimensions.

\eexam

\bthrm

    An $R$-module $M$ is Noetherian if and only if every submodule of $M$ is finitely generated.

\ethrm

\bproof

    Suppose that $M$ is Noetherian, and let $N\subseteq M$ be a submodule.
    Assume that $N$ is not finitely generated, then for every finite set of elements $v_1,\dots,v_n\in N$, $Rv_1+\cdots+Rv_n\neq N$.
    We will construct an infinite increasing sequence of finitely generated submodules of $N$.
    Define $N_0=0$, and if $N_n=Rv_1+\cdots+Rv_n$, since this cannot be equal to $N$ there is a $v_{n+1}\in N-N_n$.
    Define $N_{n+1}=N_n+Rv_{n+1}$.
    This is an infinite sequence of strictly increasing submodules.

    Now conversely suppose that every submodule of $M$ is finitely generated.
    Let $M_1\subseteq M_2\subseteq\cdots$ be an increasing sequence of submodules in $M$.
    Let $N=\bigcup_n M_i$ be their union, which is itself a submodule of $M$.
    By assumption $N$ is finitely generated, $N=Rv_1+\cdots+Rv_n$.
    Since the $M_i$ form a chain, we must have that all $v_i\in M_N$ for some $N$.
    But then $N=M_N$, so the chain is finite.
    \qed

\eproof

\blemm

    Let $M$ be an $R$-module and $N\subseteq M$ a submodule.
    Then $M$ is Noetherian (resp.\ Artinian) iff $N$ and $M/N$ are.

\elemm

\bproof

    We prove only the case for Noetherian modules, the Artinian case is handled similarly.
    If $M$ is Noetherian, then clearly every submodule is too (since submodules of $N$ are submodules of $M$).
    And submodules of $M/N$ are of the form $L/N$ for $N\subseteq L\subseteq M$, so a chain of $M/N$ submodules is of the form
    $$ L_1/N \subseteq L_2/N \subseteq \cdots $$
    and so $L_1\subseteq L_2\subseteq\cdots$ forms a chain of $M$ submodules, and thus must be finite as well.

    Conversely, if $N$ and $M/N$ are Noetherian then let $M_1\subseteq M_2\subseteq\cdots$ be a chain of $M$ submodules.
    Consider then $M_1\cap N\subseteq M_2\cap N\subseteq\cdots$, a chain of $N$ submodules.
    Since $N$ is Noetherian this must be finite, i.e.\ $M_i\cap N=M_k\cap N$ for $i\geq k$ for some $k$.
    Similarly consider $(M_1+N)/N\subseteq(M_2+N)/N\subseteq\cdots$, a chain of $M/N$ submodules.
    Once again this must be finite, so $(M_i+N)/N=(M_k+N)/N$ for some $i\geq k$ (we can assume wlog that this is the same $k$).
    This means $M_i+N=M_k+N$ and $M_i\cap N=M_k\cap N$, which means that if $x\in M_i$ then
    $$ x\in(M_i+N)\cap M_i = (M_k+N)\cap M_i = M_k\cap M_i + N\cap M_i = M_k\cap M_i + N\cap M_k \subseteq M_k $$
    So $M_i=M_k$, as required.
    \qed

\eproof

\bcoro

    The direct sum of finitely many Noetherian (resp.\ Artinian) modules is Noetherian (resp.\ Artinian).

\ecoro

\bproof

    Indeed, suppose $M,N$ are Noetherian (or Artinian).
    Then $M\oplus0\subseteq M\oplus N$ is Noetherian, and $(M\oplus N)/(M\oplus0)\cong N$ is also Noetherian.
    By our above lemma, this means $M\oplus N$ is Noetherian.
    \qed

\eproof

Recall that $R$ is itself a left $R$-module, and submodules correspond to left ideals.
$R$ is then Noetherian if it has no infinite strictly increasing sequence of left ideals, and it is Artinian if it has no infinite decreasing sequence of left ideals.
There is of course the dual notion of a right Noetherian (Artinian) ring, where we consider right ideals.

\bexam

    $R$ is left Noetherian (Artinian) if and only if all finitely generated $R$-modules are Noetherian (Artinian).

    Indeed, if $R$ is left Noetherian and $M$ is finitely generated, then suppose $M=Rv_1+\cdots+Rv_n$.
    Then let $f\colon R^n\to M$ be the $R$-module morphism defined by $f(r_1,\dots,r_n)=r_1v_1+\cdots+r_nv_n$.
    This is surjective, and as such $M\cong R^n/\ker f$.
    Since $R^n$ is the direct sum of finitely many copies of $R$, it is Noetherian.
    As such, all quotients of it are Noetherian, $M$ included.

    And conversely if all finitely generated $R$-modules are Noetherian, then $R$ is too (as a finitely generated $R$-module).

\eexam

\bdefn

    An $R$-module $M$ is {\emphcolor simple} (or {\emphcolor irreducible}) if $M$ is nonzero and has no non-trivial proper submodules (i.e.\ all submodules of $M$ are either $M$ or zero).

\edefn

\bprop

    If $I\subseteq R$ is a maximal left ideal, then $R/I$ is a simple $R$-module.
    And all simple $R$-modules are isomorphic to some quotient of this form.

    In other words, if we define $\f M_R$ to be the set of maximal left ideals of $R$, and $\f S_R$ to be the set of isomorphism classes of simple $R$-modules,
    we obtain a surjection
    $$ \f M_R\to\f S_r,\qquad I\mapsto R/I $$

\eprop

\bproof

    $R/I$ being simple is, simple.
    A submodule of $R/I$ is a quotient of the form $J/I$ where $I\subseteq J\subseteq R$ is a left ideal.
    Since $I$ is maximal, this means that $J/I$ must be trivial.

    Let $M$ be a simple $R$-module, and define $f\colon R\to M$ by $f(r)=ra$ for some nonzero $a\in M$.
    Now, $\im f$ is a nonzero submodule in $M$, and as such it must be all of $M$.
    By the isomorphism theorem, this means $R/\ker f\cong M$.
    Now $\ker f$ must be maximal: if $\ker f\subseteq I$ then $R/I\subseteq R/\ker f$ (by the injection $r+I\mapsto r+\ker f$, which is injective since if $r\in\ker f$ then $r\in I$).
    Since $M$ is simple, then $R/I$ is either $0$ (so $I=R$) or $M$ (so $I=\ker f$), thus $\ker f$ is maximal.
    \qed

\eproof

\bthrm[title=Schur]

    A morphism between simple $R$-modules is either zero or an isomorphism.

\ethrm

\bdefn

    An $R$-module $M$ is {\emphcolor semisimple} if every submodule $N\subseteq M$ has a complementary submodule $L\subseteq M$ such that $M=N\oplus L$.

\edefn


\blemm

    \benum
        \item The direct sum of finitely many semisimple modules is semisimple.
        \item A submodule of a semisimple module is semisimple.
        \item A quotient of a semisimple module is semisimple.
    \eenum

\elemm

\bproof

    \benum
        \item Let $M_1,M_2$ be semisimple and let $N\subseteq M_1\oplus M_2$.
        Let $p_1\colon M_1\oplus M_2\to M_1$ be the projection operator, and view $M_2\subseteq M_1\oplus M_2$ by inclusion.
        Since $M_1,M_2$ are semisimple there exist $N_i\subseteq M_i$ such that
        $$ M_1 = p_1(N)\oplus N_1,\qquad M_2 = (N\cap M_2)\oplus N_2 $$
        Now we claim that $M_1\oplus M_2=N\oplus(N_1\oplus N_2)$.

        Clearly their intersection is zero: $n\in N\cap(N_1\oplus N_2)$ means that $p_1(n)\in p_1(N)\cap N_1=0$.
        So $p_1(n)=0$.
        Thus $n\in M_2$ and so $n\in N\cap M_2\cap N_2=0$, so $n=0$.

        Let $m=(m_1,m_2)\in M_1\oplus M_2$, then $m_1=p_1(m)=p_1(n)+n_1$ for $n\in N$ and $n_1\in N_1$.
        Then $p_1(m-(n+n_1))=0$ so $m-(n+n_1)\in M_2$.
        This means there exists $n'\in M_2\cap N$ and $n_2\in N_2$ such that $m-(n+n_1)=n'+n_2$ and so $m=(n+n')+n_1+n_2\in N\oplus(N_1\oplus N_2)$.

        \item Let $N\subseteq M$ be a submodule, we want to show it too is semisimple.
        Let $L\subseteq N$ be a submodule, then there exists $L'\subseteq M$ such that $M=L\oplus L'$.
        But then $N=L\oplus(L'\cap N)$.

        \item Let $N\subseteq M$ be a submodule, we want to show that $M/N$ is semisimple.
        Let $L\subseteq M/N$ then there exists an $L'\subseteq M$ such that $M=L'\oplus p^{-1}(L)$ where $p\colon M\to M/N$ is the canonical projection.
        Now $M/N=p(L')\oplus L$: for $m\in M$, $m=\ell'+\ell$, then $p(m)=p(\ell')+p(\ell)\in p(L')+L$.
        Note that $pp^{-1}={\rm id}$ since $p$ is surjective, and so $p(L')\cap L=p(L'\cap p^{-1}L)=p(0)=N$.
        \qed
    \eenum

\eproof

\bprop

    Let $M$ be a finitely generated $R$-module, then it is semisimple if and only if it can be written as a finite direct sum of simple modules.

\eprop

\bproof

    If $M$ can be written as a finite direct sum of simple modules, then since simple modules are semisimple, it too is semisimple.

    Conversely suppose $M$ is semisimple, and let $N\subseteq M$.
    Then $N$ has a complement: $M=N\oplus N'$.
    The projection operator $M\to N$ is surjective with kernel $N'$, so $M/N'\cong N$.
    This shows us that $N$ must be finitely generated: $M$ is, so $M/N'$ is.
    Thus $M$ is Noetherian, as all its submodules are finitely generated.

    We now claim that every nonzero submodule of $M$ contains a maximal proper submodule.
    Indeed, let $N\subseteq M$ be a nonzero submodule.
    Suppose that $N$ has no maximal proper submodule: then by choosing an initial proper submodule we can create an increasing chain of
    submodules of $N$: $N_1\subseteq N_2\subseteq\cdots$, a contradiction since $N$ is Noetherian as a submodule of $M$.

    If $M$ is zero, the claim is trivial, so suppose $M\neq0$.
    Otherwise, let $N_1\subseteq M$ be a maximal proper submodule and let $E_1\subseteq M$ be a complement.
    Since $N_1$ is maximal, $E_1$ must be simple.
    If $N_1$ is zero, we are done.
    Otherwise we take $N_2\subseteq N_1$ maximal and $E_2$ its simple complement, and continue.
    Thus we have simple $E_1,E_2,\dots$ with an increasing sequence
    $$ E_1,E_1\oplus E_2,\dots $$
    Since $M$ is Noetherian, this must be finite.
    But then $M=E_1\oplus\cdots\oplus E_n$.
    \qed

\eproof

\bprop

    Let $R$ be left Artinian.
    Then the set of isomorphism classes of simple $R$-modules is finite.

\eprop

\bproof

    Suppose $E_1,E_2,\dots$ is an infinite sequence of non-isomorphic simple $R$-modules.
    Choose nonzero $e_i\in E_i$, and let $I_i$ be the annihilator of $e_i$:
    $$ I_i = \set{x\in R}[xe_i=0] $$
    This is a left ideal of $R$, and we have an isomorphism $R/I_i\cong E_i$ (indeed, $r\mapsto re_i$ has kernel $I_i$).

    We claim that $I_{n+1}$ does not contain $I_1\cap\cdots\cap I_n$.
    If we can demonstrate this, then $\bigcap_{i=1}^nI_i$ forms an infinite descending chain, contradicting $R$ being left Artinian.

    Now, suppose that $I_1\cap\cdots\cap I_n\subseteq I_{n+1}$.
    Notice that the module morphism $R\to E_1\oplus\cdots\oplus E_n$ given by $x\mapsto(xe_1,\dots,xe_n)$ has kernel $I_1\cap\cdots\cap I_n$.
    Thus it induces an injective morphism $R/(I_1\cap\cdots\cap I_n)\to E_1\oplus\cdots\oplus E_n$.
    Since $E_1\oplus\cdots\oplus E_n$ is semisimple as a direct sum of semisimple modules, and so $R/(I_1\cap\cdots\cap I_n)$ has a complementary
    submodule, and thus the projection operator provides a surjection $E_1\oplus\cdots\oplus E_n\to R/(I_1\cap\cdots\cap I_n)$.
    Since $I_1\cap\cdots\cap I_n\subseteq I_{n+1}$, there is a surjection $R/(I_1\cap\cdots\cap I_n)\to R/I_{n+1}\cong E_{n+1}$.
    And all in all, we have a surjection $E_1\oplus\cdots\oplus E_n\to E_{n+1}$.

    In particular this is a non-zero map, and as such one of the factor $E_i\to E_{n+1}$ must be nonzero.
    By Schur this means that $E_i\cong E_{n+1}$, a contradiction.
    \qed

\eproof

\bcoro

    If $G$ is a finite group, then the set of isomorphism classes of irreducible $G$-representations over $\bF$ is finite.

\ecoro

Note that we have dropped the condition that $\abs G\in\bF^\times$.

\bproof

    This follows immediately, since irreducible $G$-representations are simple $\bF[G]$-submodules, and $\bF[G]$, as a finite-dimensional algebra over a field, is Artinian.
    \qed

\eproof

\subsection{Semisimple rings}

\bdefn

    A ring $R$ is {\emphcolor (left) semisimple} if it is semisimple as an $R$-module.

\edefn

\blemm

    $R$ is semisimple iff all finitely generated $R$-modules are semisimple.

\elemm

\bproof

    Since $R$ is a finitely generated $R$-module, one direction is trivial.
    Now suppose that $R$ is semisimple, and let $M$ be a finitely generated $R$-module.
    Then there exists a surjection $R^n\to M$, and as such $M$ is isomorphic to a quotient of $R^n$.
    $R^n$ is semisimple as the direct sum of semisimple modules, and quotients of semisimple modules are semisimple.
    \qed

\eproof

\bexam

    Let $G$ be a group and $\bF$ a field with $\abs G\in\bF^\times$.
    Then Maschke's theorem says all $\bF[G]$-modules which are finite-dimensional $\bF$-vector spaces are semisimple.
    In particular, $\bF[G]$ is a finite-dimensional $\bF$-vector space, and as such $\bF[G]$ is a semisimple $\bF$-algebra.

\eexam

\bexam

    Division rings are simple, and thus semisimple.
    Indeed, $D$ has no nontrivial submodules: for $0\neq e\in D$, $De=D$.

\eexam

\bexam

    Let $D$ be a division ring and $V$ a finite-dimensional $D$-module.
    Then $\endo_D(V)$ is a semisimple ring.
    Indeed, let $e_1,\dots,e_n$ form a basis for $V$.
    Then let $L_i\subseteq\endo_D(V)$ be the submodule of operators which are zero on $e_j$ for $j\neq i$.
    Then $L_i$ is a left ideal in $\endo_D(V)$, and clearly $\endo_D(V)=L_1\oplus\cdots\oplus L_n$.

    Finally, each $L_i$ is simple.
    Indeed, given $T_1,T_2\in L_i$ there is an $S\in\endo_D(V)$ such that $S(T_1(e_i))=T_2(e_2)$ (like in linear algebra).
    Then $ST_1=T_2$ (since $ST_1(e_j)=0$ for $i\neq j$), and as such $L_i$ is simple.

    Further note, with the $D^\op$-module $V=(D^\op)^n$, we have $\endo_{D^\op}(V)\cong M_n(D)$.
    So we could restate this by saying that $M_n(D)$ are semisimple for $D$ a division ring.

\eexam

\bprop

    A semisimple ring is both Noetherian and Artinian.

\eprop

\bproof

    Since $R$ is semisimple and a finitely-generated $R$-module, it can be written as a finite direct sum of simple $R$-modules.
    Simple $R$-modules are clearly Noetherian and Artinian, and their finite direct sums are as well.
    \qed

\eproof

\bcoro

    The set of isomorphism classes of simple $R$-modules, for a semisimple $R$, is finite.

\ecoro

\bproof

    $R$ is Artinian, and the claim follows.
    \qed

\eproof

\subsection{The Artin-Wedderburn Theorem}

If $M$ is an $R$-module, let $S=\endo_R(M)$.
Then $M$ can be viewed as an $S$-module: for $T\in S$ and $m\in M$, $T\cdot m=Tm$.

\bprop[title=Jacobson's Density Theorem]

    Let $M$ be a semisimple $R$-module, and let $S=\endo_R(M)$.
    Given $T\in\endo_S(M)$ and $v_1,\dots,v_n\in M$ then there exists an $r\in R$ such that $Tv_i=rv_i$ for each $i$.

\eprop

\bproof

    We first deal with $n=1$, so $v\in M$.
    Since $M$ is semisimple, write $M=Rv\oplus N$ and let $P\in S$ be the projection of $M$ into $N$ along $Rv$.
    Notice that $PTv=TPv=Tv$ (since $T\in\endo_S(M)$, $T$ commutes with elements of $S$).
    Thus $Tv\in Rv$, and as such $Tv=rv$ for some $r\in R$.

    We now reduce the general case to the case of $n=1$.
    Consider the semisimple $R$-module $M^n$, and $(v_1,\dots,v_n)\in M^n$.
    And consider $T^{\oplus n}\colon(m_1,\dots,m_n)\mapsto(Tm_1,\dots,Tm_n)$.
    We want to show that $T^{\oplus n}$ is in $\endo_{\endo_R(M^n)}(M^n)$.

    Notice that there is an isomorphism $\endo_R(M^n)\cong M_n(S)$.
    Indeed, to $F\in\endo_R(M^n)$ we can define the matrix $[F]_{ij}=\iota_j\circ F\circ\pi_i$ ($\pi_i\colon M^n\to M$ the $i$th projection, and $\iota_i\colon M\to M^n$ the $i$th inclusion).
    Then since $T^{\oplus n}$ is scalar multiplication by an element of $S=\endo_R(M)$, it commutes with matrices.
    That is to say, $T^{\oplus n}\in\endo_{\endo_R(M^n)}(M^n)$.

    So by the $n=1$ case, there is an $r\in R$ such that $T^{\oplus n}(v_1,\dots,v_n)=r(v_1,\dots,v_n)$ which is precisely what we want.
    \qed

\eproof

\bcoro

    Let $M$ be a semisimple $R$-module, and let $S=\endo_R(M)$.
    If $M$ is finitely generated as an $S$-module, then $R\to\endo_S(M)$ (defined by mapping $r$ to scalar multiplication by $r$) is surjective.

\ecoro

\bproof

    Let $T_r\colon M\to M$ be scalar multiplication by $r$.
    Clearly $T_r\in\endo_S(M)$ since for $F\in S$, $T_r(Fm)=rFm=Frm=FT_rm$.

    Now suppose that $v_1,\dots,v_n$ generate $M$.
    Then given $T\in\endo_S(M)$, we know that there exists an $r\in R$ such that $Tv_i=rv_i$.
    But since $v_1,\dots,v_n$ generate $M$, this extends to all of $M$: $T=T_r$, as required.
    \qed

\eproof

Let $E$ be a simple $R$-module, and let $D_E=\endo_R(E)$.
Recall that $D_E$ is a division ring by Schur.
Then define
$$ \F_E\colon R\to\endo_{D_E}(E) $$
the natural map which maps $r\in R$ to multiplication by $r$ on $E$.
This is clearly a ring morphism.

\bthrm[title=Artin-Wedderburn]

    Let $R$ be semisimple, then $R$ has finitely many simple $R$-modules up to isomorphism.
    Furthermore, every simple $R$-module $E$ is finite dimensional over $D_E$.
    Let $E_1,\dots,E_n$ list all non-isomorphic simple $R$-modules, then the ring morphism
    $$ \F\colon R\to\prod_{i=1}^n\endo_{D_{E_i}}(E_i) $$
    given by the product $\F_{E_1}\times\cdots\times\F_{E_n}$ is an isomorphism.

\ethrm

\bproof

    As already discussed, a semisimple ring is Artinian and thus has finitely many simple modules up to isomorphism.
    $\F$ is injective: if $\F(r)=0$, then $r$ acts as zero on every simple $R$-module, and thus as zero on every finitely generated semisimple $R$-module (as they are the
    finite sums of simple $R$-modules).
    In particular, it is zero in $R$.

    We now claim $\F$ is surjective.
    Let us define $E=E_1\oplus\cdots\oplus E_n$, and let $S=\endo_R(E)$.
    We know that
    $$ S = \endo_R(E) \cong \prod_{i,j=1}^n\hom_R(E_i,E_j) $$
    by mapping $(\phi_{ij})_{ij}$ on the right to $\phi(e_1,\dots,e_n)=\parens{\sum_i\phi_{i1}(e_i),\dots,\sum_i\phi_{in}(e_i)}$.
    By Schur we then have
    $$ S \cong \prod_{i=1}^n\hom_R(E_i,E_i) = \prod_{i=1}^nD_{E_i} $$
    Now, notice that we have a ring morphism
    $$ \endo_{D_{E_1}}(E_1)\times\cdots\times\endo_{D_{E_n}}(E_n) \to \endo_S(E) $$
    which maps $(T_1,\dots,T_n)$ to $T(e_1,\dots,e_n)=(T_1e_1,\dots,T_ne_n)$.
    We claim that this is an isomorphism.
    Indeed it is clearly injective, we now show it is surjective.
    Let $T\in\endo_S(E)$, and define $T_i=\pi_i\circ T\circ\iota_i$ where $\iota_i\colon E_i\to E$ is the inclusion morphism and $\pi_i\colon E\to E_i$ is the projection morphism.
    Then $(T_i)_i$ are mapped to the morphism $\bar T$:
    $$ (e_1,\dots,e_n) \mapsto (T_1e_1,\dots,Te_n) = (\pi_iT\iota_1e_1,\dots,\pi_nT\iota_ne_n) $$
    Now consider $\Phi_i\in S$ which is the identity on the $i$th component and $0$ everywhere else.
    We know that $\Phi_iT=T\Phi_i$ and $\pi_i\Phi_i=\pi_i$ and $\iota_ie_i=\Phi_i\bar e$.
    Putting this together we have
    $$ \pi_i\bar T\bar e = \pi_iT\iota_ie_i = \pi_iT\Phi_i\bar e = \pi_i\Phi_iT\bar e = \pi_iT\bar e $$
    That is, $\pi_i\bar T=\pi_iT$ for all $i$, so $\bar T=T$ as required.

    So we have that
    $$ \endo_S(E) \cong \prod_{i=1}^n\endo_{D_{E_i}}(E_i) $$
    Now, our map $\F$ can be identified with the natural map $R\to\endo_S(E)$.
    We know that this map is surjective for $E$ semisimple and finitely generated as an $S$-module.
    $E$ is clearly semisimple since it is the finite direct product of simple modules.
    To show that $E$ is finitely generated, it is sufficient to show that $E_i$ is finitely generated as a $D_{E_i}$-module.

    Indeed, let $E$ be a simple $R$-module.
    We want to show that $E$ is a finitely generated $D_E$-module.
    Notice that $E\cong\hom_R(R,E)$ (which maps $e$ to the map $\phi(1)=e$).
    So a $D_E$ structure on $E$ corresponds to a $D_E$ structure on $\hom_R(R,E)$ where $d\in D_E$ and $\phi\in\hom_R(R,E)$, $d$ acts on $\phi$ by $d\circ\phi$.
    $R$ is semisimple and as such it is isomorphic to a finite direct sum of simple $R$-modules, let $R=R_1\oplus\cdots\oplus R_k$.
    Then $\hom_R(R,E)\cong\prod_{i=1}^k\hom_R(R_i,E_i)$ as $D_E$ modules.
    It is therefore sufficient to show that $\hom_R(R_i,E)$ is finitely generated.
    By Schur it is either zero or isomorphic to $\hom_R(E,E)=D_E$ which is finitely generated as a $D_E$ module obviously.
    \qed

\eproof

\bcoro[title=Artin-Wedderburn]

    A ring is semisimple if and only if it is isomorphic to a finite product of matrix rings over division algebras.

\ecoro

\bproof

    Indeed, $\endo_{D_E}(E)$ can be viewed as a matrix ring, since $E$ is finitely generated over $D_E$, it is isomorphic to $M_n(D_E^\op)$ where $n=\dim_{D_E}E$.
    Conversely a matrix ring over a division ring is semisimple ($M_n(D)\cong\endo_D(D^n)$ which we saw was semisimple), and the finite product of semisimple rings is semisimple.
    \qed

\eproof

\bprop

    Let $R$ be a semisimple ring, and $R\cong M_{n_1}(D_1)\times\cdots\times M_{n_k}(D_k)$.
    Then the division rings $D_i$ and dimensions $n_i$ are unique up to isomorphism.

\eprop

\subsection{The Jacobson radical}

Note that our Fourier transform can be extended to all rings $R$.
Indeed, we know that the set of isomorphism classes of simple modules over $R$ is indeed a set (in bijection with the class of maximal ideals of $R$).
So let $\f S_R$ be the set of simple $R$-modules, and we can consider
$$ \F\colon R\to\prod_{E\in\f S_R}\endo_{D_E}(E) $$
defined as before.

If $R$ is not semisimple, then $\F$ may have a nontrivial kernel.
That is, there may be a nonzero $r\in R$ which acts trivially on all simple $R$-modules.

\bdefn

    The {\emphcolor (left) Jacobson radical} of $R$ is the subset $J(R)\subseteq R$ consisting of all $r\in R$ for which $rE=0$ for all simple $R$-modules $E$.

\edefn

\blemm

    \benum
        \item The Jacobson radical is a two-sided ideal in $R$.
        \item The Jacobson radical is equal to the intersection of all maximal left ideals in $R$.
        \item Let $x\in R$, then $x\in J(R)$ iff $1-yx$ is left-invertible for all $y\in R$.
    \eenum

\elemm

\bproof

    \benum
        \item Clearly $J(R)$ is a left ideal.
        It is a right ideal because if $xE=0$ then $xrE=0$ as well ($rE\subseteq E$).
        \item Let $\f M_R$ be the set of all maximal left ideals in $R$.
        We know that every maximal left ideal in $R$ is of the form $\ann_R(e)=\set{r\in R}[re=0]$ for nonzero $e\in E$ where $E$ is simple.
        (This was proven a while back.)
        Now, $x\in J(R)$ if and only if it is in the annihlator of every simple $R$-module $E$.
        That is, $x\in J(R)$ iff it is in $\ann_R(e)$ for every $e\in E$ for $e$ nonzero and $E$ simple.
        Equivalently, $x\in J(R)$ iff it is every maximal left ideal, as required.
        \item Note that $y\in R$ is left-invertible if and only if the left ideal generated by $y$ is all of $R$.
        Equivalently, $y\in R$ is left-invertible iff it is not contained in any maximal left ideal.
        So suppose $x\in J(R)$ and $y\in R$, then $1-yx$ cannot be in any maximal left ideal (let $I$ be a maximal ideal, then $x\in I$ so $yx\in I$, and
        so if $1-yx\in I$ we get $1\in I$, a contradiction).
        Thus $1-yx$ is left invertible.

        Conversely suppose $x\notin J(R)$, then there exists a maximal ideal $I$ such that $x\notin I$.
        Then $Rx+I=R$, so there exists a $y\in R,z\in I$ such that $yx+z=1$, i.e.\ $z=1-yx$.
        But $z$ is not left-invertible, so $1-yx$ isn't.
        \qed
    \eenum

\eproof

\bprop

    Let $R$ be a ring, then the following are equivalent:
    \benum
        \item $R$ is semisimple,
        \item $R$ is left Artinian and $J(R)=0$.
    \eenum

\eprop

\bproof

    If $R$ is semisimple, we already know it is Artinian.
    Furthermore, $J(R)=\ker\F$ which is an isomorphism so $J(R)=0$.

    So now assume that $R$ is Artinian and $J(R)=0$.
    Since $R$ is Artinian, it has a minimal left ideal $I_1\subseteq R$, which is thus a simple submodule.
    Since $J(R)=0$, there is a maximal left ideal $\f m\subseteq R$ which does not include $I_1$.
    Since $I_1$ is simple, $I_1\cap\f m=0$ and thus $R=I_1\oplus\f m$.
    If $\f m=0$ then we are finished, otherwise find a simple submodule $I_2\subseteq\f m$ and a maximal left ideal $\f n\subseteq R$ such that
    $R=I_2\oplus\f n$.
    Then $\f m=I_2\oplus(\f n\cap\f m)$, and so $R=I_1\oplus I_2\oplus(\f m\cap\f n)$.
    We can proceed, with $R=I_1\oplus\cdots\oplus I_n\oplus J_n$ with $J_{n+1}\subseteq J_n$.
    Since $R$ is Artinian, eventually $J_n=0$ and we will have finished.
    \qed

\eproof

\bexam

    $\bZ$ and $\bF[X]$ are rings whose Jacobson radical is zero, but aren't semisimple (equivalently, not left Artinian).

\eexam

Note that a simple $R$-module $E$ can be seen as a $R/J(R)$-module.
Indeed, $J(R)\subseteq\ann_R(E)$ so we can define $(x+J(R))e=xe$, so every simple $R$-module can also be seen as a $R/J(R)$-module.
Since $R/J(R)$ is a quotient of $R$, every $R/J(R)$-module is also an $R$-module.
Now let $E$ be a simple $R$-module, and as such it is a $R/J(R)$-module.
It is simple as an $R/J(R)$-module: if $F\subseteq E$ is an $R/J(R)$-submodule, then it is also an $R$-module and thus is trivial.
Now if $E$ is a simple $R/J(R)$-module, then let $F\subseteq E$ be an $R$-submodule.
Since $\ann_R(E)\subseteq\ann_R(F)$, $J(R)\subseteq\ann_R(F)$ and so $F$ is also a $R/J(R)$-module, and as such is trivial.
So $E$ is a simple $R$-module.

So we have shown

\bprop

    $R$ and $R/J(R)$ have the same simple modules.

\eprop

\bcoro

    Let $R$ be a ring, then $J(R/J(R))=0$.

\ecoro

\bproof

    We know that $J(R/J(R))$ is the intersection of all the annihlators of $R/J(R)$-simple modules.
    Since an $R/J(R)$-simple module is just an $R$-simple module, we get
    $$ J(R/J(R)) = \bigcap_E\ann_{R/J(R)}(E) $$
    where the intersection is over all the simple $R$-modules.
    We know $(r+J(R))e=re$, and as such $\ann_{R/J(R)}(E)$ is equal to $\set{r+J(R)}[r\in\ann_R(E)]$.
    This means that the intersection of all these annihlators is precisely $J(R)$.
    So $J(R/J(R))=0$ (since $J(R)=0$ in $R/J(R)$).
    \qed

\eproof

\bcoro

    If $R$ is left Artinian, then $R/J(R)$ is semisimple.

\ecoro

\bproof

    $R/J(R)$ is left Artinian, and we showed its Jacobson radical is zero.
    \qed

\eproof

\bcoro

    Let $R$ be a left Artinian ring and $E_1,\dots,E_n$ list all non-isomorphic simple $R$-modules.
    Then the ring morphism
    $$ \F\colon R\to\prod_{i=1}^n\endo_{D_{E_i}}(E_i) $$
    is surjective.

\ecoro

\bproof

    We know that $\F$ quotients over its kernel, which is $J(R)$.
    This quotient is precisely the Fourier transform of $R/J(R)$, which is semisimple.
    (This is because $E_1,\dots,E_n$ list all the simple $R/J(R)$-modules too.)
    Since the Fourier transform of a semisimple ring is an isomorphism, the quotient is surjective, and thus so too is $\F$.
    \qed

\eproof

\blemm[title=Nakayama]

    Let $M$ be a finitely generated $R$-module.
    If $J(R)M=M$ then $M=0$.

\elemm

\bproof

    Let $v_1,\dots,v_n$ generate $M$ as an $R$-module, so $M=Rv_1+\cdots+Rv_n$.
    We know that $J(R)M=M$, and so $M=J(R)Rv_1+\cdots+J(R)Rv_n$.
    $J(R)$ is a two-sided ideal, so $M=J(R)v_1+\cdots+J(R)v_n$.
    In particular $v_1=x_1v_1+\cdots+x_nv_n$ for $x_i\in J(R)$, and so $(1-x_1)v_1=x_2v_2+\cdots+x_nv_n$.
    But since $1-x_1$ is left invertible as $x_1\in J(R)$, we get $v_1\in Rv_2+\cdots+Rv_n$.
    And so $v_2,\dots,v_n$ generates $M$.
    We can then continue and iteratively remove vectors from this generating set to see that $M$ is generated by the
    empty set, $M=0$.
    \qed

\eproof

\bprop

    \benum
        \item Every nilpotent left ideal in $R$ is contained in $J(R)$.
        \item If $R$ is left Artinian, then $J(R)$ is nilpotent.
    \eenum

\eprop

\bproof

    \benum
        \item Let $I\subseteq R$ be a nilpotent ideal, i.e.\ $I^n=0$.
        Let $E$ be a simple $R$-module, then if $IE\neq0$, since $E$ is simple we obtain $IE=E$.
        (Note that $IE=\set{\sum_{j=1}^ni_je_j}[i_j\in I,e_j\in E]$.)
        But then inductively $I^kE=E$, and in particular $0=I^nE=E$, a contradiction.
        So $IE=0$ and thus $I\subseteq\ann_R(E)$, and so $I\subseteq J(R)$ as required.

        \item Consider $J(R)^n$ for $n\geq1$.
        They form a decreasing sequence, and since $R$ is left Artinian, must be finite.
        That is $J(R)^{n+k}=J(R)^n$ for some $n$ and all $k\geq0$.
        Let $I=J(R)^n$, we claim that it is zero.
        Notice that we have $J(R)^kI=J(R)^{n+k}=I$ for all $k\geq0$, in particular for $k=1$: $J(R)I=I$ and $k=n$: $II=I$.

        So let us assume that $I\neq0$.
        Let $\c J$ be the set of left ideals $J\subseteq R$ for which $IJ\neq0$.
        $\c J$ is nonempty (since it contains $R$), and since $R$ is Artinian, contains minimal elements.
        Let $J_0\in\c J$ be minimal.
        Notice that $IJ_0\in\c J$ since $I(IJ_0)=IIJ_0=IJ_0\neq0$, and since $IJ_0\subseteq J_0$ we must have $J_0=IJ_0$.
        Then we have $J(R)J_0=J(R)(IJ_0)=IJ_0=J_0$.
        Now if we can show that $J_0$ is finitely generated, by Nakayama we have $J_0=0$, a contradiction.

        Since $IJ_0\neq0$, let $x\in J_0$ such that $Ix\neq0$, and thus $Rx\in\c J$.
        By $J_0$'s minimality, $Rx=J_0$, so $J_0$ is indeed finitely generated and we are finished.
        \qed
    \eenum

\eproof

\bcoro

    If $R$ is commutative then $J(R)$ contains all nilpotent elements.
    If $R$ is Artinian, then every element in $J(R)$ is nilpotent.
    Therefore if $R$ is commutative and Artinian, $J(R)$ contains precisely all nilpotent elements.

\ecoro

\bproof

    Let $x\in R$ be nilpotent.
    Then $Rx\subseteq R$ is a nilpotent ideal: if $x^n=0$ then $(rx)^n=r^nx^n=0$ for all $rx\in Rx$.
    Thus by above $x\in Rx\subseteq J(R)$ as required.

    If $R$ is Artinian, then $J(R)$ is itself nilpotent, and as such for every $x\in J(R)$, $x^n\in J(R)=0$.
    So $x$ is nilpotent.
    \qed

\eproof

Now, let $R$ be a finite-dimensional $\bF$-alegbra.
Given $x\in R$ consider the $\bF$-linear transformation $m_x\colon R\to R$ given by $y\mapsto xy$.
Let $\tr_R(x)$ be the trace of this linear transformation (which is well-defined as in linear algebra).
Then define the function
$$ (\bullet,\bullet)\colon R\times R\to R,\qquad (x,y)=\tr_R(xy) $$
This is a symmetric bilinear form: this is because $m_{xy}=m_xm_y,m_{x+y}=m_x+m_y,m_{\alpha x}=\alpha m_x$ (and $\tr_R(m_xm_y)=\tr(m_ym_x)$).
The {\bf radical} (or {\bf kernel}) of a bilinear form is given by
$$ \set{x\in R}[(x,y)=0\hbox{ for all $y\in R$}] $$

\bprop

    Suppose $R$ is a finite-dimensional $\bF$-algebra.
    Then $J(R)$ is contained in the radical of the symmetric bilinear form $(x,y)=\tr_R(xy)$.
    In particular, if this bilinear form is nondegenerate then $R$ is semisimple.

\eprop

\bproof

    Since $R$ is a finite-dimensional $\bF$-algebra it is Artinian (submodules are $\bF$-spaces), and so $J(R)$ is nilpotent, and thus contains only
    nilpotent elements.
    Let $x\in J(R)$ then for any $y\in R$ we have $xy\in J(R)$, and so $xy$ is nilpotent.
    This means that $m_{xy}$ is nilpotent, and so $\tr_R(xy)=\tr(m_{xy})=0$.
    So $x$ is in the radical of the bilinear form.

    If the bilinear form is nondegenerate, then its radical is zero, and thus $J(R)=0$.
    Since $R$ is Artinian, this means $R$ is semisimple.
    \qed

\eproof

Now we can prove Maschke once more!

\bproof[Maschke]

    We recall that $G$ is a finite group and $\abs G\in\bF^\times$.
    We want to show finite dimensional representations over $G$ are semisimple.
    Since $G$-representations can be thought of $\bF[G]$-modules, it is sufficient to show that finite-dimensional $\bF[G]$-modules are semisimple.
    A finite-dimensional $\bF[G]$-module is a quotient of $\bF[G]^n$, which is semisimple if $\bF[G]$ is.
    Thus it is sufficient to prove that $\bF[G]$ is semisimple.

    By the above proposition, it is sufficient to show that the bilinear form $(x,y)=\tr_{\bF[G]}(xy)$ is nondegenerate.
    Given $g\in G$, $m_{\delta_g}(\delta_h)=\delta_{gh}$, which means that the matrix representation of $m_{\delta_g}$ is a permutation matrix.
    This permutation has no fixed points if $g\neq1$ and is the identity if $g=1$.
    As such $\tr_{\bF[G]}(\delta_g)=0$ if $g\neq1$ and $\tr_{\bF[G]}(\delta_1)=\abs G$.
    All in all, we have
    $$ \parens{\sum_gc_g\delta_g,\delta_h} = \sum_gc_g\tr_{\bF[G]}(\delta_{gh}) = \abs Gc_{h^{-1}} $$
    So if $x=\sum_gc_g\delta_g\neq0$, let $c_g\neq0$, and then $(x,c_{g^{-1}})\neq0$.
    So the bilinear form is indeed non-degenerate, as required.
    \qed

\eproof

