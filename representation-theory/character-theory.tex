\section{Character Theory}

We fix a finite group $G$ and a field $\bF$ whose characteristic does not divide the order of $G$.

\subsection{Definition and orthogonality}

Given a $G$-representation $V$, we will sometimes write $g_V$ for the image of $g\in G$ in this representation (e.g.\ $g_V=\rho(g)$).

\bdefn

    Let $V$ be a finite-dimensional $G$-representation.
    The {\emphcolor character} of this representation is the function $\ch_V\in\St(G,\bF)$ given by
    $$ \ch_V(g) = \tr(g_V) $$
    i.e.\ $\ch_V(g)$ is the trace of $g$ in $V$.

\edefn

\bdefn

    The space of {\emphcolor class functions} on $G$ is the set $\St(G,\bF)^\cl\subseteq\St(G,\bF)$ consisting of all functions constant on conjugacy classes of $G$,
    that is functions $f\in\St(G,\bF)$ such that $f(hgh^{-1})=f(g)$ (equivalently $f(hg)=f(gh)$) for all $g,h\in G$.

\edefn

Clearly every character of $G$ is a class function: $\ch_V(hgh^{-1})=\tr(h_Vg_Vh^{-1}_V)=\tr(g_V)$, since similar transformations have equal traces.

\bexam

    Let $\chi\colon G\to\bF^\times$ be a character (in the previous sense, a group homomorphism).
    Then $\ch_{\bF_\chi}=\chi$.

\eexam

\bexam

    Let $X$ be a finite $G$-set, and recall the standard $G$-representation on $\bF[X]$.
    Using $X$ as a basis, the representation matrix of $g\in G$ is its associated permutation matrix on $X$, a $1$ on the diagonal corresponds to a fixed point of $X$ under $g$.
    Thus, $\ch_{\bF[X]}(g)$ is equal to the number of fixed points of $g$.

\eexam

\bdefn

    Notice that function multiplication on $\St(G,\bF)$ can be defined pointwise: $(f_1\cdot f_2)(g)=f_1(g)\cdot f_2(g)$.
    This restricts to multiplication in $\St(G,\bF)^\cl$ (i.e.\ the pointwise product of two class functions is another class function).

\edefn

\blemm

    Let $V,W$ be finite $G$-representations, then
    $$ \ch_{V\otimes W}=\ch_V\cdot\ch_W $$
    We write $V\otimes W$ for $V\otimes_\bF W$.

\elemm

\bproof

    Let $(e_i)_{i\in I}$ be a basis for $V$ and $(f_j)_{j\in J}$ a basis for $W$.
    Write $ge_i=\sum_{i'}c_{ii'}e_{i'}$ and $gf_j=\sum_{j'}d_{jj'}f_{j'}$.
    Then $(e_i\otimes f_j)_{i\in I,j\in J}$ is a basis for $V\otimes W$, and
    $$ g(e_i\otimes f_j) = ge_i\otimes gf_j = \parens{\sum_{i'}c_{ii'}e_{i'}}\otimes\parens{\sum_{j'}d_{jj'}f_{j'}} = \sum_{i',j'}c_{ii'}d_{jj'}e_{i'}\otimes f_{j'} $$
    Thus
    $$ \ch_{V\otimes W}(g) = \tr_{V\otimes W}(g) = \sum_{i\in I,j\in J}c_{ii}d_{jj} = \sum_ic_{ii}\cdot\sum_jd_{jj} = \ch_V(g)\cdot\ch_W(g) $$
    as required.
    \qed

\eproof

\bdefn

    For a function $f\in\St(G,\bF)$, define its {\emphcolor dual} $f^*\in\St(G,\bF)$ by $f^*(g)=f(g^{-1})$.
    Note that this again restricts to class functions as well.

\edefn

\blemm

    Let $V$ be a finite-dimensional $G$-representation, and let $V^*$ be its dual space.
    Then $\ch_{V^*}=\ch_V^*$.

\elemm

\bproof

    Let $(e_i)_{i\in I}$ be a basis for $V$ and let $(e_i^*)_{i\in I}$ be its dual basis in $V^*$.
    Write $g^{-1}e_j=\sum_ic_{ji}e_i$.
    Then $ge_i^*(e_j)=e_i^*(g^{-1}e_j)=c_{ji}$, meaning $ge_i^*=\sum_jc_{ji}e_j^*$, so
    $$ \ch_{V^*}(g) = \sum_ic_{ii} = \ch_V(g^{-1}) $$
    as required.
    \qed

\eproof

\bcoro

    Let $V,W$ be finite-dimensional $G$-representations.
    Then
    $$ \ch_{\hom(V,W)} = \ch_V^*\cdot\ch_W $$

\ecoro

\bproof

    This follows from the previous two lemmas, as $\hom(V,W)\cong V^*\otimes W$.
    \qed

\eproof

\bdefn

    For $f\in\St(G,\bF)$, define $\av(f)\in\bF$ by
    $$ \av(f) = \frac1{\abs G}\sum_{g\in G}f(g) $$

\edefn

\blemm

    For a finite-dimensional $G$-representation $V$: $\av(\ch_V)=\dim V^G$.

\elemm

\bproof

    Recall the averaging operator $\av_V^G\colon V\to V$,
    $$ \av_V^G(v) = \frac1{\abs G}\sum_{g\in G}gv $$
    As it is a projection operator on $V^G$, we have $\tr_V(\av_V^G)=\dim V^G$.
    But
    $$ \tr_V(\av_V^G) = \frac1{\abs G}\sum_{g\in G}\tr_V(g) = \frac1{\abs G}\sum_{g\in G}\ch_V(g) = \av(\ch_V) $$
    as required.
    \qed

\eproof

\bdefn

    Define the symmetric bilinear form $(\bul,\bul)$ on $\St(G,\bF)$ by
    $$ (f_1,f_2) = \av(f_1^*\cdot f_2) = \frac1{\abs G}\sum_{g\in G}f_1(g^{-1})f_2(g) $$

\edefn

This being symmetric and bilinear are obvious (for symmetry, note that $\av(f)=\av(f^*)$ and $(f_1^*\cdot f_2)^*=f_2^*\cdot f_1$).

\bcoro

    Let $V,W$ be finite-dimensional $G$-representations.
    Then
    $$ \dim\hom_G(V,W) = (\ch_V,\ch_W) $$

\ecoro

\bproof

    Note that $\hom_G(V,W)=\hom(V,W)^G$ and as such
    $$ \dim\hom_G(V,W) = \av(\ch_{\hom(V,W)}) = \av(\ch_V^*\cdot\ch_W) = (\ch_V,\ch_W) \qed $$

\eproof

As such, we have almost immediately:

\bcoro

    Let $E,F$ be irreducible $G$-representations.
    Then
    $$ (\ch_E,\ch_F) = \cases{d_E&$E\cong F$\cr0&else} $$
    where $d_E\in\bZ_{\geq1}$ is the dimension of the division algebra $\endo_G(E)$.

\ecoro

\bproof

    We know $(\ch_E,\ch_F)=\dim\hom_G(E,F)$.
    Since $E,F$ are irreducible, by Schur $\hom_G(E,F)\cong\endo_G(E)$ iff $E\cong F$ and $\hom_G(E,F)=0$ otherwise.
    \qed

\eproof

Note that, depending on the characteristic of our field $\bF$, we may have that $(\ch_E,\ch_F)=0$ even if $E\cong F$ (since $d_E$ may be $0$ in $\bF$).
But if $\bF$ is algebraically closed then we know $\endo_G(E)\cong\bF$, as such:

\bcoro

    Let $E,F$ be irreducible $G$-representations over $\bF$ algebraically closed.
    Then
    $$ (\ch_E,\ch_F) = \cases{1&$E\cong F$\cr0&else} $$

\ecoro

\bcoro

    If $\bF$ is algebraically closed, and $E_1,\dots,E_n$ be an exhaustive, non-isomorphic, list of all irreducible $G$-representations.
    Then $\ch_{E_1},\dots,\ch_{E_n}\in\St(G,\bF)^\cl$ are linearly independent.

\ecoro

\bproof

    Suppose $\sum_ic_i\ch_{E_i}=0$, then by orthogonality we have
    $$ c_j = \parens{\sum_ic_i\ch_{E_i},\ch_{E_j}} = (0,\ch_{E_j}) = 0 $$
    so $\ch_{E_1},\dots,\ch_{E_n}$ are indeed linearly independent.
    \qed

\eproof

\bthrm

    If $\bF$ is algebraically closed and $E_1,\dots,E_n$ an exhaustive list of irreducible $G$-representations.
    Then $\ch_{E_1},\dots,\ch_{E_n}$ form a basis for $\St(G,\bF)^\cl$.

\ethrm

\bproof

    We showed that the characters of $E_i$ are linearly independent.
    Furthermore, we showed earlier that $\dim\St(G,\bF)^\cl=n$ (which is equal to the number of conjugacy classes of $G$).
    Thus these characters must also span $\St(G,\bF)^\cl$, and as such they form a basis.
    \qed

\eproof

For functions $f_1,f_2\in\St(G,\bF)$ we define $f_1+f_2$ pointwise as well.
This again restricts to class functions.
It is not hard to see that
$$ \ch_{V\oplus W}=\ch_V+\ch_W $$

\bprop

    Suppose $\bF$ has characteristic $0$.
    Let $V,W$ be finite-dimensional $G$-representations.
    Then $V$ is isomorphic to $W$ if and only if $\ch_V=\ch_W$.

\eprop

\bproof

    Clearly isomorphic $G$-representations have the same character.
    Conversely, suppose $\ch_V=\ch_W$.
    To show that $V\cong W$ it is sufficient to show that for every irreducible $E$, $[V:E]=[W:E]$ (since they are the sum of irreducibles).
    Note that we have
    $$ [V:E] = \dim\hom_G(V,E) = (\ch_V,\ch_E) = (\ch_W,\ch_E) = \dim\hom_G(W,E) = [W:E] $$
    This is equality in $\bF$, but since it has characteristic $0$, it is equality in $\bZ$ as well.
    \qed

\eproof

\bprop

    Let $E$ be an irreducible $G$-representation over algebraically closed $\bF$.
    Then the characteristic of $\bF$ does not divide $\dim_{\bF}E$.

\eprop

\bproof

    Let $f\in\St(G,\bF)^\cl$ and define $d=\sum_{g\in G}f(g)\delta_g\in Z(\bF[G])$.
    The action of $d$ on $E$ defines $T_d\in\endo_G(E)$, and so by Schur $T_d$ is scalar multiplication.
    Thus the trace of $T_d$ is a multiple of $\dim_{\bF}E$.
    If the characteristic of $\bF$ divides this dimension, this would mean $\tr(T_d)=0$.
    But
    $$ \tr(T_d) = \sum_{g\in G}f(g)\ch_E(g) = (f^*,\ch_E) $$
    So let $f=\ch_E^*$, then we get $\tr(T_d)=(\ch_E,\ch_E)=1$.
    But this contradicts $\tr(T_d)=0$.
    \qed

\eproof

Recall that for an irreducible $G$-representation $E$ over an algebraically closed field $\bF$, we define $e_E\in Z(\bF[G])$ which acts by the identity on $E$ and $0$ on all other non-isomorphic
irreducible representations.
Equivalently, $e_E$ is the unique element whose action on every $G$-representation $V$ is projection onto $V_E$.
Recall that we had a formula for $e_E$ in the commutative case.

\bprop

    $$ e_E = \frac{\dim_\bF E}{\abs G}\sum_{g\in G}\ch_E(g^{-1})\delta_g $$

\eprop

\bproof

    We will take $e_E$ to be this formula, and show that it has the desired property.
    Let $F$ be an irreducible $G$-representation, given by the homomorphism $\rho\colon\bF[G]\to\endo_\bF(F)$.
    Now, we have that $\rho(e_E)\in\endo_G(F)$: (we show this for $e'_E=\abs G/\dim_\bF Ee_E$)
    $$ \rho(e'_E)\rho(\delta_h) = \sum_{g\in G}\ch_E(g^{-1})\rho(\delta_{gh}) = \sum_g\ch_E(hg^{-1})\rho(\delta_g) = \sum_g\ch_E(g^{-1}h)\rho(\delta_g) $$
    which is $\rho(\delta_h)\rho(e'_E)$.
    Thus by Schur, $\rho(e_E)$ is scalar multiplication.

    We showed that $\dim_\bF F\neq0$ in $\bF$.
    As such, if $\tr(\rho(e_E))=0$ in $\bF$, then $\rho(e_E)=0$, and if $\tr(\rho(e_E))=\dim_\bF F$ then $\rho(e_E)={\rm id}$.
    So we want to show that if $F$ is not isomorphic to $E$ then $\tr(\rho(e_E))$, and if $E\cong F$ then $\tr(\rho(e_E))=\dim_\bF E$.
    Indeed:
    $$ \tr(\rho(e_E)) = \frac{\dim_\bF E}{\abs G}\sum_{g\in G}\ch_E(g^{-1})\tr(\rho(g)) = \frac{\dim_\bF E}{\abs G}\sum_{g\in G}\ch_E(g^{-1})\ch_F(g) = \dim_\bF E\cdot(\ch_E,\ch_F) $$
    So we have the desired result.
    \qed

\eproof

Note that in the case $\bF=\bC$, we can also define the Hermitian form (inner product):
$$ \gen{f_1,f_2} = \frac1{\abs G}\sum_{g\in G}f_1(g)\overline{f_2(g)} $$
We claim that we have $(\ch_V,\ch_W)=\gen{\ch_V,\ch_W}$.
Indeed, this is because $\ch_W(g^{-1})=\overline{\ch_W(g)}$; consider the eigenvalues $\set{\lambda_i}_i$ (with multiplicity) of $g_W$, since $g_W^{\abs W}={\rm id}$, these are all roots of unity.
As such $\lambda_i^{-1}=\overline\lambda_i$.
The eigenvalues of $g^{-1}_W$ are $\set{\lambda_i^{-1}}_i$, and thus
$$ \ch_W(g^{-1}) = \sum_i\lambda_i^{-1} = \sum_i\overline\lambda_i = \overline{\sum_i\lambda_i} = \overline{\ch_W(g)} $$

The benefit of using the inner product is that it can generalize to topological groups on Hilbert spaces.
The bilinear form has the benefit of generalizing to other fields.

\bprop

    The bilinear form $(\bul,\bul)$ on $\St(G,\bF)$, and its restriction to the set of class functions, is nondegenerate.

\eprop

\bproof

    Let $f\in\St(G,\bF)$.
    If $(f,f')=0$ for all $f'\in\St(G,\bF)$ then in particular we have $(f,\delta_g^*)=0$ for all $g\in G$.
    But by definition
    $$ (f,\delta_g^*) = \frac1{\abs G}f(g) $$
    and so $f(g)=0$ for all $g\in G$, so $f=0$.
    So the bilinear form is non-degenerate on $\St(G,\bF)$.

    Now we consider its restriction to $\St(G,\bF)^\cl$.
    Consider
    $$ \av\colon\St(G,\bF)\to\St(G,\bF)^\cl,\qquad \av(f)(g) = \frac1{\abs G}\sum_{h\in G}f(hgh^{-1}) $$
    We quickly check that this is a projection operator onto $\St(G,\bF)^\cl$.
    Furthermore, we have
    $$ (\av(f),f') = (f,\av(f')) $$
    Now suppose $f\in\St(G,\bF)^\cl$ with $(f,f')=0$ for all $f'\in\St(G,\bF)^\cl$.
    Then for every $f'\in\St(G,\bF)$ we have
    $$ (f,f') = (\av(f),f') = (f,\av(f')) = 0 $$
    And so $f=0$ by the non-degeneracy of the bilinear form on $\St(G,\bF)$.
    \qed

\eproof

\subsection{Integral elements}

We will assume that $\bF$ is algebraically closed.

\bdefn

    An element $a$ in a ring $R$ is {\emphcolor integral} if it is the root of some monic polynomial $f\in\bZ[x]$.

\edefn

If $a\in S$ is integral and $S\subseteq R$, clearly $a\in R$ is integral.

\blemm

    Let $R$ be a ring and $a\in R$, then $a$ is integral iff $\bZ[a]\subseteq R$, defined to be the $\bZ$-span of $\set{1,a,a^2,\dots}$, is a finitely generated $\bZ$-module.

\elemm

\bproof

    If $a$ is integral, then suppose $f\in\bZ[x]$ is monic with $f(a)=0$.
    Suppose its degree is $n$, then $a^n\in\lspan_\bZ\set{1,\dots,a^{n-1}}$, and so we see that the span of $\set{1,a,\dots}$ is equal to the span of $\set{1,\dots,a^{n-1}}$.

    Conversely, suppose $\lspan_\bZ\set{a^k}_k$ is finitely generated.
    Notice that $\bZ[a]$ can be equivalently written as $\set{f(a)}[{f\in\bZ[x]}]$, and so if it is finitely generated, it can be generated by a set of the form $\set{1,\dots,a^{n-1}}$ (take $n-1$ to be the maximum
    degree of polynomials in the generating set).
    So $a^n$ can be expressed as a $\bZ$-linear combination of elements $1,\dots,a^{n-1}$, and so $a$ is integral.
    \qed

\eproof

\bcoro

    Let $R$ be a ring, finitely-generated as a $\bZ$-module.
    Then all elements of $R$ are integral.

\ecoro

\bproof

    Let $a\in R$, then $\bZ[a]\subseteq R$ is a $\bZ$-submodule.
    Since $\bZ$ is Noetherian, submodules of finitely generated modules are also finitely generated.
    As such $\bZ[a]$ is finitely generated; the result follows from the previous lemma.
    \qed

\eproof

\bprop

    The subset of integral elements of a commutative ring is a subring.

\eprop

\bproof

    Clearly $0,1\in R$ are integral.
    Now if $a,b\in R$ are integral, say $\bZ[a]=\lspan_\bZ\set{1,\dots,a^n}$ and $\bZ[b]=\lspan_\bZ\set{1,\dots,b^m}$.
    Then by commutativity, $\bZ[a,b]=\lspan_\bZ\set{a^kb^\ell}_{k\leq n,\ell\leq m}$ is finitely-generated.
    Now we clearly have $\bZ[a+b],\bZ[ab]\subseteq\bZ[a,b]$, and since $\bZ$ is Noetherian, these must be finitely-generated too.
    \qed

\eproof

\bexam

    The subring of integral elements of $\bQ$ is $\bZ$.

\eexam

A clear property of integral elements is that they are preserved under ring morphisms: if $a\in R$ is integral and $\phi\colon R\to S$ is a morphism, then $\phi(a)\in S$ is integral.
In particular what this means is that every integer in a ring is integral.
Explicitly, let $n\in\bZ$, then $n$ is clearly integral in $\bZ$.
Let $\phi\colon\bZ\to R$ be the canonical morphism, then $\phi(n)\in R$ is integral.

\blemm

    \benum
        \item Let $d=\sum_{g\in G}c_g\delta_g\in\bF[G]$, and suppose that $c_g$ are all integers.
        Then $d$ is integral.
        \item Let $d=\sum_{g\in G}c_g\delta_g\in Z(\bF[G])$, and suppose that $c_g$ are all integral in $\bF$.
        Then $d$ is integral.
    \eenum

\elemm

\bproof

    \benum
        \item $d$ clearly lies in the obvious morphism $\bZ[G]\to\bF[G]$, and every element of $\bZ[G]$ is integral since $\bZ[G]$ is a finitely-generated $\bZ$-module.
        \item Note that each $\delta_g$ is integral by the previous point, and $c_g\in\bF\subseteq\bF[G]$ are integral.
        $Z(\bF[G])$ is commutative, and so its subring of integral elements forms a subring.
        Thus $d$ must be integral, as the sum and product of integral elements.
        \qed
    \eenum

\eproof

\blemm

    Let $V$ be a finite-dimensional $G$-representation.
    Then for every $g\in G$, $\ch_V(g)\in\bF$ is integral.

\elemm

\bproof

    Since $g^n=1$ in $G$, $g^n_V={\rm id}_V$ for some $n$.
    Thus the eigenvalues of $g_V$ are all roots of unity, which are clearly integral.
    Since $\ch_V(g)$ is the sum of these integral elements, and $\bF$ is commutative, $\ch_V(g)$ must be integral.
    \qed

\eproof

\bprop

    Let $E$ be an irreducible $G$-representation, then $\dim E$ divides $\abs G$.

\eprop

\bproof

    We will prove this for the case that ${\rm char}\bF=0$.

    Let $\rho\colon\bF[G]\to\endo(E)$ be the $\bF$-algebra morphism corresponding to $G$'s action on $E$.
    Recall the central idempotent
    $$ e_E = \frac{\dim E}{\abs G}\sum_{g\in G}\ch_E(g^{-1})\delta_g \in Z(\bF[G]) $$
    This acts by identity on $E$, and so
    $$ d = \sum_{g\in G}\ch_E(g^{-1})\delta_g \in Z(\bF[G]) $$
    acts as multiplication by $\lambda=\abs G/\dim E$ on $E$.
    Thus $\rho(d)=\lambda{\rm id}_E$ is integral in $\bF\cdot{\rm id}_E\subseteq\endo(E)$.
    Thus $\lambda\in\bF$ (identified with $\rho(d)\in\bF\cdot{\rm id}_E\cong\bF$) is integral.
    So $\lambda=\abs G/\dim E\in\bQ\subseteq\bF$ is integral, which means it must be an integrer, so $\dim E$ divides $\abs G$.
    \qed

\eproof

We can strengthen this result.

\bdefn

    Let $V$ be a $G$-representation and $W$ an $H$-representation.
    Their {\emphcolor outer tensor product} is the $G\times H$-representation $V\boxtimes_\bF H$, whose base vector space is $V\otimes_\bF H$, but with an action given by
    $$ (g,h)(v\otimes w) = (gv)\otimes(hw) $$

\edefn

Note that if $V,W$ are both $G$-representations then composing the diagonal morphism $G\to G\times G$ with the representation $G\times G\to\endo(V\boxtimes_\bF W)$ gives the
canonical $G$-representation $V\otimes_\bF W$.

\bprop

    Let $E_1,E_2$ be irreducible representations of the finite groups $G_1,G_2$ respectively.
    Then $E_1\boxtimes E_2$ is an irreducible representation of $G_1\times G_2$.

\eprop

Using this we can show

\bprop

    Let $E$ be an irreducible $G$-representation.
    Then $\dim E$ divides $[G:Z(G)]$.

\eprop

\bproof

    Let us write $A=Z(G)$.
    For $m\geq1$ consider the irreducible representation $E^{\boxtimes m}=E\boxtimes_\bF\cdots\boxtimes_\bF E$ of $G^m=G\times\cdots\times G$.
    Let us consider the subgroup $A_m$ of $A^m$ given by all tuples $(a_1,\dots,a_m)\in A^m$ for which $a_1\cdots a_m=1$ (this is clearly a subgroup).

    $A$ acts on $E$ by scalars (Schur), i.e.\ there exists a character $\chi\colon A\times\bF^\times$ such that $a\cdot v=\chi(a)v$ for $a\in A,v\in E$.
    Then $A^m$ acts on $E^{\boxtimes m}$ by $(a_1,\dots,a_m)\cdot(v_1\otimes\cdots\otimes v_m)=\chi(a_1\cdots a_m)(v_1\otimes\cdots\otimes v_m)$ (by linearity and $\chi$ being a morphism).
    This means that $A_m$ acts trivially on $E^{\boxtimes m}$.

    In general if $H$ is a normal subgroup of $G$ which acts trivially on $V$, then $V$ can be given a natural $G/H$-representation, which preserves things like irreducibility.
    $A_m$, as a subgroup of $A^m$, in turn a subgroup of $Z(G^m)$, is normal in $G^m$.
    Thus $E^{\boxtimes m}$ has a natural $G/A_m$-representation, and is irreducible.
    Thus we have that $\dim E^{\boxtimes m}=(\dim E)^m$ divides $\abs{G^m/A_m}=\abs G^m/\abs{A_m}=\abs G^m/\abs A^{m-1}$ (clearly by construction $\abs{A_m}=\abs A^{m-1}$).

    Given a prime $p$ and an integer $n$, let $v_p(n)$ be the number of times $p$ divides $n$.
    Clearly we have $v_p(n^k)=kv_p(n)$ and if $n$ divides $m$ then $v_p(n)\leq v_p(m)$ (in general $v_p(nm)=v_p(n)+v_p(m)$), so
    $$ mv_p(\dim E) \leq v_p(\abs G^m / \abs A^{m-1}) = mv_p(\abs G) - (m-1)v_p(\abs A) $$
    Thus
    $$ v_p(\dim E) \leq v_p(\abs G/\abs A) + \frac1mv_p(\abs A) = v_p([G:Z(G)]) + \frac1mv_p(\abs A) $$
    Letting $m\to\infty$ we get $v_p(\dim E)\leq v_p([G:Z(G)])$ for all prime $p$.
    Thus we must have that $\dim E$ divides $[G:Z(G)]$.
    \qed

\eproof

\subsection{Burnside's Theorem}

We will prove the following theorem:

\bthrm[title=Burnside]

    If $G$ is a finite group whose order has at most two prime divisors, then $G$ is solvable.

\ethrm

This theorem's statement does not mention representation at all, yet its proof will utilize it.

We will call integral elements of $\bC$ {\it algebraic integers}.

\blemm

    Let $\zeta_1,\dots,\zeta_n\in\bC^\times$ be roots of unity, then
    \benum
        \item The average $(\zeta_1+\cdots+\zeta_n)/n$ has modulus in $[0,1]$, and $1$ is attained iff $\zeta_1=\cdots=\zeta_n$.
        \item The average $(\zeta_1+\cdots+\zeta_n)/n$ is an algebraic integer iff it is zero or $\zeta_1=\cdots=\zeta_n$.
    \eenum

\elemm

\bproof

    \benum
        \item Clearly we have $\abs{(\zeta_1+\cdots+\zeta_n)/n}\leq(\abs{\zeta_1}+\cdots+\abs{\zeta_n})/n=1$ with equality occurring meaning that all $\zeta_i$ are real scalar multiples of one another by Cauchy-Schwarz.
        Since these are roots of unity, they are real scalar multiples iff they differ only by sign.
        The result follows.

        \item By the previous point, it is sufficient to show that the average is an algebraic integer iff its modulus is $0$ or $1$.
        Let $a$ be the average, suppose it is not zero, then consider the Galois extension $K=\bQ[\zeta_1,\dots,\zeta_n]/\bQ$ (it is Galois as it is the compositum of Galois extensions), which defines the norm
        $$ N\colon K^\times\to\bQ^\times,\qquad N(x) = \prod_{\sigma\in{\rm Gal}(K/\bQ)}\sigma(x) $$
        In particular, $N(a)\in\bQ$.
        Furthermore, if $a$ is an algebraic integer, so is every $\sigma(a)$, and thus so is $N(a)$ as their product.
        Therefore $N(a)\in\bZ$, and is nonzero.
        Furthermore, $\sigma(a)$ must also be an average of roots of unity (since $\sigma(\zeta_i)$ is a root of unity), thus its modulus is in $[0,1]$.
        Therefore the modulus (now absolute value) of $N(a)$ is in $[0,1]$; since it is a nonzero integer it must be $1$.
        Thus the modulus of every $\sigma(a)$ must be $1$; in particular $\abs a=1$ as required.
        \qed
    \eenum

\eproof

We denote by $C_g$ the conjugacy class of an element $g\in G$.

\blemm

    Let $E$ be an irreducible $G$-representation over $\bC$.
    Let $g\in G$ and suppose that $\abs{C_g}$ and $\dim E$ are relatively prime.
    Then either $\ch_E(g)=0$ or $g$ acts on $E$ by scalar.

\elemm

\bproof

    Let $\rho\colon\bC[G]\to\endo(E)$ be the $\bC$-algebra morphism corresponding to $G$'s action on $E$.
    Define
    $$ d = \sum_{h\in C_g}\delta_h \in Z(\bC[G]) $$
    By Schur, $\rho(d)$ is scalar multiplication on $E$.
    Furthermore, $d$ is an integral element, and so its image under $\rho$ in $\bC\cong\bC\cdot{\rm id}_E\subset\endo(E)$ is also an integral element.
    Its image under this correspondence (i.e.\ the scalar it multiplies by) is
    $$ \frac{\tr(\rho(d))}{\dim E} = \frac{\abs{C_g}\ch_E(g)}{\dim E} $$

    Now, since $\abs{C_g}$ and $\dim E$ are relatively prime, this means that $\ch_E(g)/\dim E$ is integral.
    Indeed, $1$ can be expressed as a $\bZ$-linear combination of $\abs{C_g}$ and $\dim E$, suppose $1=\alpha\abs{C_g}+\beta\dim E$ then
    $$ \frac{\alpha\abs{C_g}\ch_E(g)}{\dim E} = \frac{(1-\beta\dim E)\ch_E(g)}{\dim E} = \frac{\ch_E(g)}{\dim E}-\beta\ch_E(g) $$
    So $\ch_E(g)/\dim E\in\bZ[\abs{C_g}\ch_E(g)/\dim E]$, meaning $\bZ[\ch_E(g)/\dim E]=\bZ[\abs{C_g}\ch_E(g)/\dim E]$, which is finitely-generated and as such $\ch_E(g)/\dim E$ is integral.

    Now, $\ch_E(g)/\dim E$ is the average of the eigenvalues of $g_E$; an average of roots of unity.
    This is integral iff $\ch_E(g)=0$ or all of $g_E$'s eigenvalues are equal.
    In the latter case, this means that $g_E$ is multiplication by scalar.
    \qed

\eproof

\bprop

    If $G$ contains a conjugacy class in which the number of elements is a positive power of a prime number, then $G$ is not simple.

\eprop

\bproof

    Let $C_g\subseteq G$ be a conjugacy class whose order is a positive power of a prime $p$.
    We will show there exists a non-trivial irreducible $G$-representation $E$ over $\bC$ over which $C_g$'s elements act by scalar multiplication (they all act by the same scalar, since they are conjugate).
    Then for $g\neq h\in C_g$ we have that $gh^{-1}$ acts by identity, so the kernel of $\rho\colon G\to\gl(E)$ is a non-trivial proper normal subgroup of $G$; $G$ is not simple.

    By the previous lemma, it is sufficient to find a non-trivial irreducible $G$-representation $E$ such that $p$ does not divide $\dim E$ and $\ch_E(g)\neq0$.
    Recall that (letting the sum run over irreducible $G$-representations):
    $$ \sum_E(\dim E)\cdot\ch_E(g) = 0 $$
    We split this sum up
    $$ 1 + \sum_{p\mid\dim E}(\dim E)\ch_E(g) + \sum_{\neg p\mid\dim E}(\dim E)\ch_E(g) = 0 $$
    (The final sum also supposes that $E$ is nontrivial.)
    This is an equation in the complex subring of algebraic integers.
    Since $p$ is not a unit in this ring (since $1/p$ is not an integer; not integral in $\bQ$), there must be an $E$ such that $p$ does not divide $(\dim E)\cdot\ch_E(g)$.
    In particular, $p$ does not divide $\dim E$ and $(\dim E)\cdot\ch_E(g)\neq0$ so $\ch_E(g)\neq0$ as desired.
    \qed

\eproof

We can now prove Burnside's theorem:

\bproof[Burnside]

    We know that finite $p$-groups (groups of order $p^n$) are solvable.
    So we will show that if $G$'s order is divisible by precisely two primes, then $G$ is not simple.
    This is sufficient: inducting over the order of the group, a normal subgroup in $G$ is either a $p$-group (and thus solvable) or has order divisible by precisely two primes (solvable by induction),
    similar for $G/N$, and so $G$ is solvable.

    So let $G$ be a group whose order is divisible by two primes, $p$ and $q$.
    If $Z(G)$ is nontrivial, then $G$ is not simple.
    Otherwise $Z(G)=1$, then the sum of the orders of the non-trivial conjugacy classes is not divisible by $pq$ (let $C_1,\dots,C_n$ be thee non-trivial conjugacy classes, then
    $\abs G\equiv\abs{C_1}+\cdots+\abs{C_n}+1\equiv0\pmod{pq}$, since $pq$ doesn't divide $1$, it doesn't divide the sums of $\abs{C_i}$).
    Thus there must be a conjugacy class whose order is not divisible by $p$ or $q$.
    Since the order of each conjugacy class divides the order of $\abs G$, this means such a conjugacy class's order is a positive power of $p$ or $q$, and the result follows from the previous proposition.
    \qed

\eproof

