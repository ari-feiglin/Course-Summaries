\section{Tensor Products}

\subsection{The basic definition}

Recall that what we have until now called a $R$-module is actually a {\it left\/} $R$-module.
A right $R$-module is defined in much the same way, but multiplication is defined as a function $M\times R\to M$.
While a left $R$-module is equivalent to a ring morphism $R\to\endo(M)$, a right $R$-module is equivalent to a ring morphism $R^\op\to\endo(M)$ (where $\endo(M)$ is the
ring of endomorphisms on the Abelian group $M$).
So a right $R$-module is equivalent to a left $R^\op$-module.

We also write ${}_RM$ to mean $M$ is a left $R$-module and $N_R$ to mean $N$ is a right $R$-module.

\bdefn

    Let $M$ be a left $R$-module, and $N$ a right $R$-module, and $A$ an Abelian group.
    A biadditive map $\Phi\colon N\times M\to A$ (meaning it is additive in both of its components) is {\emphcolor balanced} (or {\emphcolor $R$-balanced})
    if for all $n\in N,m\in M,r\in R$:
    $$ \Phi(nr,m) = \Phi(n,rm) $$

\edefn

Let ${}_RM$ and $N_R$ be $R$-modules.
We define their {\it tensor product} to be an Abelian group $N\otimes_RM$ as well as a balanced biadditive map $\Phi\colon N\times M\to N\otimes_R M$
which satisfies the following the following {\it universal property}: for any Abelian group $A$ and a balanced biadditive map $\psi\colon N\times M\to A$,
there exists a unique Abelian group morphism $f\colon N\otimes_RM\to A$ such that $\psi=f\circ\Phi$.
In a diagram:

\centerline{\drawdiagram{
    $N\times M$&$N\otimes_RM$\cr
    &$A$
}{
    \diagarrow{from={1,1}, to={1,2}, text=$\Phi$, y distance=.25cm}
    \diagarrow{from={1,1}, to={2,2}, text=$\psi$, y distance=-.2cm}
    \diagarrow{from={1,2}, to={2,2}, text=$\exists!$, x distance=.2cm, dashed}
}}

$\Phi$ should be thought of part of the structure of the tensor product, but it is usually left implicit and instead of writing $\Phi(n,m)$ one writes $n\otimes m$.

Now we can ask two questions: does such a construction exist, and if so is it unique?
The answer to both is positive (for the latter, it is true of course up to isomorphism).
We will show that the tensor product is unique.

First, notice that by taking $A=N\otimes_RM$ and $\psi=\Phi$ we see that $f={\rm id}$ makes the diagram above commute.
But $f$ is unique; it is the {\it only\/} function which can make the diagram commute.
Which means that if $\Phi=f\circ\Phi$ then $f={\rm id}$.

Now suppose $C_1,C_2$ both satisfy the universal property for tensor products.
Then the following diagram commutes (i.e.\ all compositions from the same source to the same destination are equal):

\bigbreak
\centerline{\drawdiagram{
    $N\times M$&$C_1$\cr
    &$C_2$
}{
    \diagarrow{from={1,1}, to={1,2}, text=$\Phi_1$, y distance=.25cm}
    \diagarrow{from={1,1}, to={2,2}, text=$\Phi_2$, y distance=-.25cm}
    \diagarrow{from={1,2}, to={2,2}, text=$f$, x distance=-.3cm, dashed, x off=-.1cm}
    \diagarrow{from={2,2}, to={1,2}, text=$g$, x distance=.3cm, dashed, x off=.1cm}
}}

In particular, we have $\Phi_1=(g\circ f)\circ\Phi_1$ and $\Phi_2=(f\circ g)\circ\Phi_2$.
But as said earlier, this means $g\circ f$ and $f\circ g$ are their respective identities, so $f,g$ are isomorphisms: $C_1\cong C_2$.

So all that remains is to show that an Abelian group with the universal property exists.
To do so we will explicitly construct an Abelian group with the universal property.
Given $N,M$ we consider the free Abelian group over their product: $\bZ[N\times M]$.
Let us write $\delta_{(n,m)}$ for the element of the basis corresponding to $(n,m)\in N\times M$.
We will construct $N\otimes_RM$ to be a quotient of this free group.
In particular, we will have $\Phi(n,m)=[\delta_{(n,m)}]$.

In order for $\Phi$ to be a biadditive balanced morphism, this means our quotient must satisfy:
$$ \eqalign{
    \Phi(n_1+n_2,m) = \Phi(n_1,m) + \Phi(n_2,m) &\implies [\delta_{(n_1+n_2,m)}] = [\delta_{(n_1,m)}] + [\delta_{(n_2,m)}]\cr
    \Phi(n,m_1+m_2) = \Phi(n,m_1) + \Phi(n,m_2) &\implies [\delta_{(n,m_1+m_2)}] = [\delta_{(n,m_1)}] + [\delta_{(n,m_2)}]\cr
    \Phi(nr,m) = \Phi(n,rm) &\implies [\delta_{(nr,m)}] = [\delta_{(n,rm)}]
} $$
This means that our quotient must contain all elements of the form
$$ \delta_{(n_1+n_2,m)} - \delta_{(n_1,m)} - \delta_{(n_2,m)},\qquad \delta_{(n,m_1+m_2)} - \delta_{(n,m_1)} - \delta_{(n,m_2)},\qquad \delta_{(nr,m)} - \delta_{(n,rm)} $$
It turns out that taking these relations is sufficient, that is let us define $K$ to be the subgroup generated by these elements, then we define
$$ N\otimes_RM = \bZ[N\times M]/K $$
and as discussed, $\Phi(n,m)=[\delta_{(n,m)}]=\delta_{(n,m)}+K$.

The relations given were sufficient for making $\Phi$ biadditive and balanced, so this construction is a valid contender for the tensor product.
And it satisfies the universal property.

\subsection{Basic cases}

Now let us consider the tensor product of direct sums.
That is, given $(N_i)_{i\in I}$ right $R$-modules and $(M_j)_{j\in J}$ left $R$-modules, consider the tensor product of
$$ M\otimes_RN = \parens{\bigoplus_{i\in I}N_i}\otimes_R\parens{\bigoplus_{j\in J}M_j} $$
We claim that it is isomorphic to
$$ \bigoplus_{i\in I,j\in J}N_i\otimes_R M_j $$
All we need to do is show that this has the universal property.

Take $\Phi\colon N\times M\to\bigoplus_{i,j}N_i\otimes_RM_j$ as $\Phi(n_i,m_j)=n_i\otimes m_j$ where $n_i\in N_i,m_j\in M_j$.
This extends linearly: for $r_i,s_j\in R$ (all but finitely many being zero):
$$ \Phi\parens{\sum_{i\in I}n_ir_i,\sum_{j\in J}s_jm_j} = \sum_{i\in I,j\in J}(n_ir_i)\otimes(s_jm_j) $$
This is balanced: clearly this needs only be checked on elements of $N_i,M_j$: $\Phi(nr,m)=(nr)\otimes m=n\otimes(rm)=\Phi(n,rm)$.
So $\Phi$ is biadditive and balanced.
Notice that $\Phi_{ij}=\pi_{ij}\circ\Phi\circ\iota_{ij}$.

Now, suppose $\psi\colon N\times M\to A$ is also biadditive and balanced.
Then we want an $f$ such that $\psi=f\circ\Phi$.
Note that we have a unique $f_{ij}$ such that $\psi\circ\iota_{ij}=f_{ij}\circ\Phi_{ij}=f_{ij}\circ\pi_{ij}\circ\Phi\circ\iota_{ij}$.
Further notice that if such an $f$ exists,
$$ f\circ\iota^\otimes_{ij}\circ\pi_{ij}\circ\Phi\circ\iota_{ij} = f\circ\Phi\circ\iota_{ij} = \psi\circ\iota_{ij} $$
So $f\circ\iota^\otimes_{ij}$ must be equal to $f_{ij}$.
This uniquely determines $f$, and sastisfies the condition.

So we have shown:

\bprop

    If $(N_i)_{i\in I}$ is a family of right $R$-modules, and $(M_j)_{j\in J}$ a family of left $R$-modules, then
    $$ \parens{\bigoplus_{i\in I}N_i}\otimes_R\parens{\bigoplus_{j\in J}M_j} \cong \bigoplus_{i\in I,j\in J}N_i\otimes_RM_j $$

\eprop

\bdefn

    Let $R,S$ be rings.
    Then an {\emphcolor $(R,S)$-bimodule} is an Abelian group $M$ which is both a left $R$-module and a right $S$-module, such that
    for all $r\in R,s\in S,m\in M$: $(rn)s=r(ns)$.
    We write that ${}_RM_S$ is a module.

\edefn

In general a $(R,S)$-bimodule is equivalent to a ring morphism $R\times S^\op\to\endo(M)$.
Note that all left $R$-modules are $(R,\bZ)$-bimodules, and all right $R$-modules are $(\bZ,R)$-bimodules.

So now suppose that ${}_SN_R$ and ${}_RM_T$ are modules.
Then we claim that $N\otimes_RM$ is an $(S,T)$-bimodule.

Indeed, let $s\in S,t\in T$ and consider the map $N\times M\to N\otimes_RM$ by $(n,m)\mapsto (sn)\otimes(mt)$.
This is clearly a biadditive balanced map.
Thus we get a unique group morphism $N\otimes_RM\to N\otimes_RM$ such that $n\otimes m\mapsto(sn)\otimes(mt)$.
So we have defined a function
$$ S\times T^\op\to\endo(N\otimes_RM) $$
and we readily check that this is a ring morphism.

That is to say, given modules ${}_SN_R,{}_RM_T$, their tensor product $N\otimes_RM$ is an $(S,T)$-bimodule.
For special cases, if $N$ or $M$ is a one-sided module, then $S$ or $T$ can be considered to be $\bZ$.
For example, if ${}_SN_R$ is a bimodule but ${}_RM$ is a one-sided module, then considering it as ${}_RM_{\bZ}$, we get that
$N\otimes_RM$ is a $(S,\bZ)$-bimodule, or simply a left $S$-module.

In particular, if $R$ is a commutative ring, there is no difference between left and right $R$-modules.
So if $N,M$ are $R$-modules (considered as $(R,R)$-bimodules), then $N\otimes_RM$ is an $R$-module.

Now, consider $R$ a right $R$-module, then we claim that $R\otimes_RM\cong M$.
Indeed, define $\Phi\colon R\times M\to M$ to be multiplication: $\Phi(r,m)=rm$.
This is biadditive and balanced.
Now for any Abelian group $A$ and biadditive $\psi\colon R\times M\to A$, we want an $f\colon M\to A$ such that $\psi=f\circ\Phi$.
I.e.\ $\psi(r,m)=f(rm)$.
Since $\psi$ is biadditive and balanced, we have $\psi(r,m)=\psi(1r,m)=\psi(1,rm)$.
So define $f(m)=\psi(1,m)$, then $\psi(r,m)=\psi(1,rm)=f(rm)$ as required.
Similarly we see that $M\otimes_RR\cong M$.

In particular, let $R=\bF$ be a field.
Then if $N,M$ are $\bF$-vector spaces, let $(n_i)_{i\in I}$ and $(m_j)_{j\in J}$ be bases.
Then we have $N=\bigoplus_{i\in I}\bF n_i$ and $M=\bigoplus_{j\in J}\bF m_j$ and so by our above theorem:
$$ N\otimes_\bF M \cong \bigoplus_{(i,j)\in I\times J}(\bF n_i)\otimes_\bF(\bF m_j) $$
Now, we know that $(\bF n_i)\otimes_\bF(\bF m_j)\cong\bF\otimes_\bF\bF\cong\bF$.
So this is isomorphic to $\bigoplus_{I\times J}\bF$.
Furthermore $\set{n_i\otimes m_j}_{i,j}$ forms a basis for $N\otimes_\bF M$.

\bprop

    Let $N,M$ be vector spaces with bases $(n_i)_i$ and $(m_j)_j$.
    Then $(n_i\otimes m_j)$ forms a basis of $N\otimes M$, and in particular one has
    $$ \dim(N\otimes M) = (\dim N)(\dim M) $$

\eprop

Now, if $R$ is commutative, then an $R$-bilinear map $N\times M\to U$ (where $U$ is an $R$-module), is in particular balanced.
In the case of a commutative $R$, the universal property for tensor products is equivalent to the following universal property.
Given an $R$-module $U$ and an $R$-bilinear map $\psi\colon N\times M\to U$, there exists a unique $R$-module morphism $T\colon N\otimes_RM\to U$ such that
$T(n\otimes m)=\psi(n,m)$.
That is, when our ring is commutative, we can consider the universal property as a property of only $R$-modules (and not Abelian groups in general).

\subsection{Basic properties}

Let $\bF$ be a field, and $V,W$ be $\bF$-vector spaces.
Consider the dual space of $V$, $V^\vee=\hom_\bF(V,\bF)$.
Now, we have a $\bF$-bilinear map $V^\vee\times W\to\hom_\bF(V,W)$ defined by $(\phi,w)\mapsto[v\mapsto\phi(v)w]$.
So by the universal property, there exists a unique $\bF$-morphism $T\colon V^\vee\otimes_\bF W\to\hom_\bF(V,W)$ such that $T(\phi\otimes w)=[v\mapsto\phi(v)w]$.

\bprop

    If $V$ is finite-dimensional then the above $\bF$-linear map $T\colon V^\vee\otimes_\bF W\to\hom_\bF(V,W)$ is an isomorphism.

\eprop

\bproof

    Let $(e_i)_{i\in I}$ be a basis for $V$, and let $(e_i^*)_{i\in I}$ be its dual basis in $V^\vee$.
    Let $(f_j)_{j\in J}$ be a basis for $W$, then $(e_i^*\otimes f_j)_{i,j}$ is a basis for $V^\vee\otimes_\bF W$.
    The image of $e_i^*\otimes f_j$ under $T$ is $T_{ij}\colon v\mapsto e_i^*(v)f_j$.
    It is uniquely defined as $T_{ij}\colon e_j\mapsto \delta_{ij}f_j$.
    This obviously is a basis for $\hom_\bF(V,W)$.
    \qed

\eproof

\subsection{Tensor products of representations}

Let $G$ be a group and $\bF$ a field.
We will consider $G$-representations over $\bF$.

Let $V,W$ be two $G$-representations over $\bF$.
Then consider the map $V\times W\to V\otimes_\bF W$ by $(v,w)\mapsto(gv)\otimes(gw)$ for $g\in G$.
This is bilinear and balanced.
As such, there exists an endomorphism $f_g\in\endo_\bF(V\otimes_\bF W)$ such that $(gv)\otimes(gw)=f_g(v\otimes w)$.
This forms a group morphism: $f_1(v\otimes w)=v\otimes w$ so $f_1={\rm id}$ and $f_{gg'}=(gg'v)\otimes(gg'w)=f_g(f_{g'}(v\otimes w))$, so $f_{gg'}=f_g\circ f_{g'}$.
Thus $g\mapsto f_g$ forms a representation in $V\otimes_\bF W$.

\bdefn

    Let $V$ be a $G$-representation.
    We define its {\emphcolor dual representation} as $V^\vee$ ($V$'s dual space), where for $g\in G$ and $\phi\in V^\vee$ we define
    $g\phi\colon v\mapsto\phi(g^{-1}v)$.

\edefn

Notice that this corresponds to our previous representation of $\hom_\bF(V,W)$ (where $g\star T(v)=gT(g^{-1}v)$), where $W=\bF$, since $\bF$ has a trivial representation.

In particular, consider the $\bF$-linear map $T\colon V^\vee\otimes_\bF W\to\hom_\bF(V,W)$.
The domain and codomain of this map are all $G$-representations.
We see that
$$ T(g(\phi\otimes w)) = T((g\phi)\otimes(gw))\colon v\mapsto (g\phi)(v)gw = \phi(g^{-1}v)gw = g\phi(g^{-1}v)w $$
While
$$ g\star T(\phi\otimes w)(v) = gT(\phi\otimes w)(g^{-1}v) = g\phi(g^{-1}v)w $$
Thus $T$ is a morphism of $G$-representations.

\bcoro

    If $V$ is a finite-dimensional $G$-representation, then the $G$-representations $V^\vee\otimes_\bF W$ and $\hom_\bF(V,W)$ are isomorphic.

\ecoro

This isomorphism is actually natural.


