\section{Basic Representation Theory}

\subsection{Group actions}

We recall that we can view a group action of a group $G$ on a set $X$ equivalently as either a group homomorphism $\rho\colon G\to S_X$, or as a map $\cdot\colon G\times X\to X$ (written as juxtaposition) where
\benum
    \item $ex=x$,
    \item $g_1(g_2x)=(g_1g_2)x$
\eenum
The relation between these two equivalent definitions is $\rho(g)(x)=gx$.
A group action is also called a {\bf $G$-set}.

\bexam

    Consider a finite-dimensional vector space $V$.
    Then the group of automorphisms over $V$ (denoted $\gl(V)$) acts on $V$ in the obvious way.

    If we further assume that $V$ is an inner product space, then let $S=\set{v\in V}[\norm v=1]$ and let $O(V)$ be the group of orthonormal automorphisms (those which preserve the inner product).
    Then $O(V)$ acts on $S$ again in the obvious way.
    Note that $\gl(V)$ acts on $V$ the same way that $O(V)$ acts on $S$.

\eexam

\bexam

    Let $H\subseteq G$ be a subgroup (not necessarily normal).
    Then $G/H$ (the set of cosets) can be made into a $G$-set by defining $g(g'H)=(gg')H$.

\eexam

\bdefn

    Let $X$ and $Y$ be $G$-sets.
    Then a {\emphcolor morphism of $G$-sets} $X\to Y$ is a function $f\colon X\to Y$ satisfying $f(gx)=gf(x)$ for all $g\in G$, $x\in X$.

\edefn

This definition of morphisms of $G$-sets, along with the usual composition, makes the class of $G$-sets into a category.
We denote this category by $\St G$ (or $\St_G$), and similarly denote the hom-set of morphisms as $\St_G(X,Y)$.

Recall that a transitive group action is one where for every $x_1,x_2\in X$ there exists a $g\in G$ such that $gx_1=x_2$.

\bexam

    Let $G$ act on $X$ transitively.
    Let $x_0\in X$ and define
    $$ \stab_G(x_0) = \set{g\in G}[gx_0=x_0] $$
    the {\emphcolor stabilizer} of $x_0$.
    The stabilizer is clearly a subgroup of $G$, and we have a natural isomorphism of $G$-sets $G/\stab_G(x_0)\cong X$.

\eexam

\bexam

    Notice that $O(V)$ acts transitively on $S$ (this is a simple result of linear algebra).
    Let $s_0\in S$, and define $W$ to be the orthogonal complement of $s_0$.
    Then notice that $\stab_{O(V)}(s_0)$ is naturally isomorphic to $O(W)$ (since an orthonormal automorphism of $W$ can be uniquely extended to an orthonormal automorphism of $V$ with $s_0$ as a fixed point).

    By the above example, we have that
    $$ O(V)/\stab_{O(V)}(s_0) \cong S $$
    We can thus abuse notation and write $O(V)/O(W)\cong S$ (viewing $O(W)$ as a subgroup of $O(V)$).
    Writing $O(n)$ for $O(V)$ when $n=\dim V$, we thus have $O(n)/O(n-1)\cong S^n$.

\eexam

\subsection{Representations}

\bdefn

    Given a group $G$, a {\emphcolor representation of $G$} (or a {\emphcolor $G$-representation}), is a group homomorphism $\rho\colon G\to\gl(V)$ (where $V$ is a vector space over some given field).
    $\rho$ is usually kept implicit, so instead of writing $\rho(g)v$ for example, we write $gv$.

    Given two $G$-representations $rho_1\colon G\to\gl(V_1)$ and $\rho_2\colon G\to\gl(V_2)$, a morphism $\rho_1\to\rho_2$ is a linear morphism $T\colon V_1\to V_2$ such that
    $T(\rho_1(g)v)=\rho_2(g)(Tv)$ (i.e.\ $Tgv=gTv$).

\edefn

We denote the space of linear morphisms $V\to W$ by $\hom(V,W)$, and of $G$-representations by $\hom_G(V,W)$ (which is a subspace of $\hom(V,W)$).

\bexam

    $\bF^n$ can be made into a representation of $S_n$ by defining
    $$ \sigma\pmatrix{x_1\cr\vdots\cr x_n} = \pmatrix{x_{\sigma^{-1}1}\cr\vdots\cr x_{\sigma^{-1}n}} $$

\eexam

\bexam

    In general, consider the space of (pure) functions $X\to\bF$ $\St(X,\bF)$ where $X$ is a $G$-set.
    This can be made into a representation of $G$ by setting
    $$ (gf)(x) = f(g^{-1}x) $$

    Consider then the $\bF$-vector space $\bF[X]$ (which is the space obtained by taking formal linear combinations of symbols $x\in X$).
    Then $\bF[X]$ can be made into a $G$-representation where $x$ is mapped to $gx$ (this uniquely extends to all of $\bF[X]$).

    Now, when $X$ is finite, we have that $\St(X,\bF)\cong\bF[X]$ naturally (as vector spaces).
    If $X$ is a $G$-set, this is an isomorphism of $G$-representations.

\eexam

Let $V$ be an $\bF$-vector space, and $G$ a group.
Then the {\bf trivial representation} of $G$ is the representation which maps each $g\in G$ to the identity morphism.

\bexam

    A {\emphcolor character} is a group morphism $\chi\colon G\to\bF^\times$.
    Given a character, we can make $\bF$ a $G$-representation by defining
    $$ gc = \chi(g)c $$
    We denote this representation by $\bF_\chi$.
    This is indeed a representation: $\bF_\chi(g)$ is clearly in $\gl(\bF)$ for each $g\in G$, and $\bF_\chi(gh)c=\chi(g)\chi(h)c=\bF_\chi(g)\chi(h)c=\bF_\chi(g)\bF_\chi(h)c$.

\eexam

\bdefn

    Let $V$ be a $G$-representation, and $W$ a subspace of $V$.
    Then $W$ is a {\emphcolor $G$-subrepresentation} of $V$ if it is invariant: $gw\in W$ for all $w\in W$.
    We can then naturally view $W$ as a $G$-representation in and of itself.

\edefn

\bdefn

    Let $V$ be a $G$-representation and $W$ a $G$-subrepresentation.
    Then $V/W$ forms a $G$-representation, called the {\emphcolor quotient $G$-representation} defined by $g(v+W)=gv+W$.
    This is well-defined precisely because $W$ is a $G$-subrepresentation.

\edefn

\bdefn

    A $G$-representation $V$ is {\emphcolor irreducible} (or {\emphcolor simple}) if $V\neq0$ and it has no non-trivial $G$-subrepresentations (meaning that every $G$-subrepresentation is either $V$ or $0$).
    $V$ is {\emphcolor indecomposable} if $V$ is non-trivial and when $V=W_1\oplus W_2$ then the direct summands are trivial (i.e.\ $W_1=V$ or $0$).

\edefn

Clearly an irreducible representation is indecomposable.
We will later see that when the characteristic of the underlying field is zero, the converse is true.

\bexam[title=An indecomposable representation which is not irreducible]

    Let us consider $S_2$ acting on $\bF^2$ as above, where $\bF=\bZ/2\bZ$.
    Then $\bF^2$ is not an irreducible $S_2$-representation, since $W=\set{(x_1,x_2)}[x_1+x_2=0]$ is a $S_2$-subrepresentation of $\bF^2$ which is non-trivial.
    However, $\bF^2$ is an indecomposable $S_2$-representation.

    For let $\bF^2=W_1\oplus W_2$, then $W_1,W_2$ must have dimension $1$, let $L$ have dimension $1$.
    Since linear automorphisms of $L$ must be scalar multiplications, $(1\ 2)\in S_2$ acts as scalar multiplication.
    But since the only non-zero scalar in the field is $1$, $S_2$ acts trivialy on $L$.

    Then $S_2$ acts trivially on $W_1,W_2$, and thus trivially on all of $\bF^2$.
    But $(1\ 2)$ does not act trivially on all of $\bF^2$, in contradiction.

\eexam

\subsection{Semisimple representations}

The {\bf direct sum} of two $G$-representations $V_1$ and $V_2$ is constructed over their direct sum as vector spaces $V_1\oplus V_2$ where $g(v_1,v_2)=(gv_1,gv_2)$.

\bprop

    Let $V_1,V_2$ be $G$-representations, and $T\colon V_1\to V_2$ be a morphism of $G$-representations.
    Then the kernel of $T$ is a $G$-subrepresentation of $V_1$, and the image of $T$ is a $G$-subrepresentation of $V_2$.
    Furthermore, the {\emphcolor cokernel} $\coker T$, defined to be $V_2/\im T$, is a $G$-representation.

\eprop

\bdefn

    A $G$-representation $V$ is {\emphcolor semisimple} if for every $G$-subrepresentation $W\subseteq V$, there exists another $G$-subrepresentation $W'\subseteq V$ such that $V=W\oplus W'$.

\edefn

\bthrm[title=Maschke's Theorem]

    Let $G$ be a finite group, and $\bF$ a field whose characteristic does not divide the order of $G$.
    Then every finite-dimensional $G$-representation is semisimple.

\ethrm

We give two distinct proofs of Maschke's theorem.

\bproof[first proof]

    Let $V$ be a finite-dimensional $G$-representation, where $\rho$ is the representation, and let $W\subseteq V$ be a $G$-subrepresentation.
    Since $V$ is finite-dimensional, there exists a complementary subspace $W'\subseteq V$ where $V=W\oplus W'$.
    We now consider the projection operator of $W$ along $W'$: i.e.\ $P(w+w')=w$.
    We define the endomorphism $Q\colon V\to V$:
    $$ Q = \frac1{\abs G}\sum_{g\in G}\rho(g)\circ P\circ\rho(g)^{-1} $$

    We claim that $Q$ is a projection operator of $W$.
    First we note that $Q$ is the identity on $W$: since $\rho(g)^{-1}(w)=\rho(g^{-1})(w)\in W$ we have that $\rho(g)P\rho(g^{-1})(w)=\rho(g)\rho(g^{-1})(w)=w$.
    And so
    $$ Qw = \frac1{\abs G}\sum_{g\in G}w = w $$
    as required.
    Now we note that the image of $Q$ is contained within $W$.
    This is simply because the image of $P$ is $W$, and $\rho$ preserves $W$.
    Thus $Q$ is a projection operator of $W$ (where $V=W\oplus\ker Q$).

    Now we claim that $Q$ is a morphism of $G$-representations: for $h\in G$ we must show that $Q\circ\rho(h)=\rho(h)\circ Q$.
    Indeed:
    $$ Q\circ\rho(h) = \frac1{\abs G}\sum_{g\in G}\rho(g)\circ P\circ\rho(g)^{-1}\rho(h) = \frac1{\abs G}\sum_{g\in G}\rho(g)\circ P\circ\rho((h^{-1}g)^{-1}) $$
    substituting $h^{-1}g$ for $g$ in the sum gives
    $$ \frac1{\abs G}\sum_{g\in G}\rho(hg)\circ P\circ\rho(g^{-1}) = \frac1{\abs G}\rho(h)\sum_{g\in G}\rho(g)\circ P\circ\rho(g^{-1}) = \rho(h)\circ Q $$
    as required.

    So $Q\colon V\to V$ is a morphism of $G$-representations and a projection operator of $W$.
    Since it is a morphism of $G$-representations, $\ker Q$ is a $G$-subrepresentation.
    And as noted, this is a complementary subspace of $W$, as required.
    \qed

\eproof

Let us take a moment to internalize a few parts of this proof.

\bdefn

    Let $V$ and $W$ be $G$-representations.
    On $\hom(V,W)$ (not just $\hom_G(V,W)$) we define the structure of a $G$-representation by defining
    $$ (g\phi)(v) = g\phi(g^{-1}v) $$
    (The representation maps $g$ to an automorphism of $\hom(V,W)$, i.e.\ we must map $\phi$ to another morphism in $\hom(V,W)$.)
    We also denote $g\phi$ by $g\star\phi$, to note confuse it with $\rho(g)\circ\phi$.
    So $g\star\phi=\rho(g)\circ\phi\circ\rho(g)^{-1}$.

\edefn

This is indeed a representation: $g\star\phi$ is clearly linear so $g\star\phi\in\hom(V,W)$.
Now, $\phi\mapsto g\star\phi$ is itself an automorphism of $\hom(V,W)$: it is clearly a bijection, and it is similarly clearly linear.
Now let us consider the representation itself $R\colon G\to\gl(\hom(V,W))$, $R(g)(\phi)=g\star\phi$.
This must be a group homomorphism: $R(gh)=R(g)\circ R(h)$.
Indeed: $R(g)\circ R(h)(\phi)=R(g)(h\star\phi)=g\star(h\star\phi)=(gh)\star\phi=R(gh)(\phi)$ as required.

\bdefn

    Let $V$ be a $G$-representation, define the $G$-subrepresentation of {\emphcolor invariants} to be
    $$ V^G = \set{v\in V}[\hbox{for all $g\in G$, $gv=v$}] $$

\edefn

This is indeed a subspace, since $\rho(g)$ is linear, and it is a subrepresentation since $g(hv)=(gh)v=v$.

\bdefn

    Let $V$ be a $G$-representation, and suppose that the characteristic of $\bF$ does not divide $\abs G$ (so that $\abs G\neq0$ and is therefore invertible).
    Define the {\emphcolor averaging operator} $\av^G_V\colon V\to V$ to be
    $$ \av^G_V(v) = \frac1{\abs G}\sum_{g\in G}gv $$

\edefn

Note that $\av^G_V$ is a projection operator on $V^G$: clearly it is the identity on $V^G$, and its image is contained in $V^G$.

Now, notice that $T\in\hom(V,W)$ is a morphism of $G$-representations if and only if $g\star T=T$ for all $g\in G$.
Indeed, $g\star T=\rho(g)\circ T\circ\rho(g)^{-1}$, and this is always $T$ if and only if $T$ commutes with all $\rho(g)$, i.e.\ is a morphism.
Thus
$$ \hom_G(V,W) = \hom(V,W)^G $$

So in our above proof, we consider the $G$-representation $\hom(V,V)$ and an element $P\in\hom(V,V)$.
We then define $Q=\av^G_{\hom(V,V)}(P)$, and so $Q\in\hom(V,V)^G=\hom_G(V,V)$.
And we further showed that $P$ being a projection implies $Q$ being a projection.

\bproof[second proof]

    Consider the projection $\pi\colon V\to V/W$.
    We want to show the existence of a $G$-morphism $\iota\colon V/W\to V$ such that $\pi\circ\iota={\rm id}_{V/W}$.
    Then the image of $\iota$ would be complementary to $W$, and would be a $G$-subrepresentation.

    More generally, given a surjective morphism of $G$-representations $\pi\colon V\to Z$, and a $G$-representation $U$ we would like to show that $\pi_*\colon\hom_G(U,V)\to\hom_G(U,Z)$ (post composition with $\pi$) is
    surjective.
    This map is the restriction of the more general $\pi_*\colon\hom(U,V)\to\hom(U,Z)$.
    This map is clearly surjective and a morphism of $G$-representations.

    So now we recast the problem as follows: given $G$-representations $V$ and $W$ and a sujective $G$-morphism $p\colon V\to W$, the restricted morphism $p\colon V^G\to W^G$ is surjective as well.
    Indeed: given $w\in W^G$ let $v\in V$ such that $pv=w$, then $p(\av^G_V(v))=\av^G_V(p(v))=\av^G_V(w)=w$, so $\av^G_V(v)$ is in the preimage of $w$ under the restricted $p$.
    \qed

\eproof

\subsection{Decomposition into irreducibles}

We will assume in this section that $G$ is a finite group and $\bF$'s characteristic does not divide $G$'s order.

\blemm

    Let $V$ be a finite-dimensional $G$-representation.
    Then there exist irreducible $G$-representations $E_1,\dots,E_n$ such that $V$ is isomorphic to $E_1\oplus\cdots\oplus E_n$ as $G$-representations.

\elemm

\bproof

    We induct on the dimension of $V$.
    If $\dim V=0$, then an empty sum suffices.
    If $V$ is itself irreducible, then $E_1=V$ works.
    Otherwise, let $W$ be a non-trivial subrepresentation of $V$.
    By Maschke's theorem, $V$ is semisimple and therefore there exists a subrepresentation $W'$ such that $V=W\oplus W'$.
    By induction, both of these subrepresentations are isomorphic to the direct sum of irreducible $G$-representations.
    Then $V$ is isomorphic to the direct sum of these direct sums, itself a direct sum.
    \qed

\eproof

\blemm[title=Schur's Lemma]

    Let $E$ and $F$ be irreducible $G$-representations.
    Then a morphism between them is either trivial or an isomorphism.

\elemm

\bproof

    Let $T\colon E\to F$ be a non-trivial morphism of $G$-representations.
    Since $T$ is non-trivial, $\ker T$ mustn't be all of $E$, and since $E$ is irreducible this means that $\ker T$ must be trivial.
    So $T$ is injective.
    Similarly, consider $\im T$, which cannot be trivial and therefore (since $F$ is irreducible) must be all of $F$.
    So $T$ is surjective.
    Therefore, $T$ is an isomorphism.
    \qed

\eproof

\bprop

    An irreducible $G$-representation (when $G$ is finite), is finite-dimensional.

\eprop

\bproof

    Let $V$ be an infinite-dimensional $G$-representation.
    Take $v\in V$ non-zero, and define $W=\lspan\set{gv}_{g\in G}$.
    Since $v\in W$, it is non-zero, and because it is spanned by a finite set $W$ is not all of $V$.
    And furthermore $W$ is clearly a $G$-representation.
    \qed

\eproof

\blemm

    Let $V$ be finite-dimensional, and $V\cong E_1\oplus\cdots\oplus E_n\cong F_1\oplus\cdots\oplus F_m$ be irreducible factorizations of $V$.
    Then for every irreducible $G$-representation $E$, the number of $E_i$ isomorphic to $E$ is equal the number of $F_i$ isomorphic to $E$, both being
    $\dim\hom_G(V,E)/\dim\hom_G(E,E)$.

\elemm

\bproof

    Let $d=\hom_G(E,E)$, and $d\geq1$ as $\hom_G(E,E)$ is non-trivial (the identity).
    By Schur's lemma, for an irreducible $G$-representation $F$, $\dim_G(E,F)=0$ if $F$ is not isomorphic to $E$.
    Thus:
    $$ \dim\hom_G(V,E) = \dim\hom_G(E_1\oplus\cdots\oplus E_n,E) = \dim\hom_G(E_1,E)+\cdots+\dim\hom_G(E_n,E) $$
    For all $E_i$ not isomorphic to $E$, the summands are zero, and so we are left with $d$ times the number of $E_i$ isomorphic to $E$:
    $$ \dim\hom_G(V,E) = d\cdot\#\set{\hbox{$E_i$ isomorphic to $E$}} $$
    And so we obtain that the number of $E_i$ isomorphic to $E$ is the aforementioned number.
    \qed

\eproof

\bdefn

    Let $V$ be a finite-dimensional $G$-representation, and $E$ be an irreducible $G$-representation.
    Then the {\emphcolor multiplicity} of $E$ in $V$, denoted $[V:E]$, is the number of irreducible components in a factorization of $V$ isomorphic to $V$.
    That is,
    $$ [V:E] = \frac{\dim\hom_G(V,E)}{\dim\hom_G(E,E)} $$

\edefn

Let $V$ be a finite-dimensional $G$-representation, and $E$ an irreducible one.
Consider the $G$-subrepresentation of $V$ obtained by taking the sum of all $G$-subrepresentations of $V$ isomorphic to $E$.
This is called the {\bf isotypical component} $V_E$.


\blemm

    Let $V=E_1\oplus\cdots\oplus E_n$ be a factorization of $V$ into a direct sum of irreducible $G$-subrepresentations.
    Then $V_E$ is equal to the sum of the $E_i$s isomorphic to $E$.

\elemm

\bproof

    Clearly the sum of the $E_i$s isomorphic to $E$ is contained in $V_E$.
    Now, suppose that $F\subseteq V$ is isomorphic to $E$.
    We will show that given an $E_i$ not isomorphic to $E$, composing the inclusion $F\to V$ with the projection $V\to E_i$ is $0$.
    From this it follows that $F$ must be contained in the sum of $E_i$s isomorphic to $E$, giving us our desired result.

    Indeed, since $F$ is irreducible $F\to V\to E_i$ must be either zero or an isomorphism (by Schur).
    Since $F$ is isomorphic to $E$ which is not isomorphic to $E_i$, we get that this morphism must be zero, as desired.
    \qed

\eproof

Although the decomposition of $V$ into a direct sum of irreducible subrepresentations is not necessarily unique, if we group the subrepresentations
by isomorphism class, we get a result independent of the representation, the isotypical component of that isomorphism class.

Note that $V_E$ has a unique complementary subrepresentation.
Firstly it has one by Maschke.
Let $W$ be a complementary subrepresentation of $V_E$, then it decomposes into irreducible subrepresentations.
That is, $W=F_1\oplus\cdots\oplus F_n$, and so $V=V_E\oplus F_1\oplus\cdots\oplus F_n$.
In particular this means that if we group $F_i$ by isomorphism class, we get $W=V_{F_1}\oplus\cdots\oplus V_{F_n}$, as required.

Let $\bF$ be the trivial representation of $G$, i.e.\ $gx=x$ for all $x\in\bF$.
This is clearly irreducible, as it has dimension one.
Notice that $V^G$ is equal to the isotypical component $V_{\bF}$.
Let $E\subseteq V$ be isomorphic to $\bF$, then since $\bF$ is $G$-invariant so too must be $E$.
And so $V_{\bF}\subseteq V^G$.
Let $E\subseteq V^G$ be irreducible, then it must be isomorphic to $\bF$ (since every subvector-space of $V^G$ is a subrepresentation, so the only
irreducible subrepresentations are one-dimensional, and $V^G$'s representation is trivial).
Thus $V^G\subseteq V_{\bF}$.

The kernel of the projection operator on $V^G$, $\av_V^G$, is thus the sum of all irreducible components of $V$ not isomorphic to $\bF$.

\blemm[name=Schur]

    Let $\bF$ be algebraically closed, and $E$ an irreducible $G$-representation.
    Then $\endo_G(E)=\hom_G(E,E)$ is equal to $\bF\cdot{\rm id}_E$.

\elemm

\bproof

    Let $f\colon E\to E$ be an endomorphism of $E$.
    Since $\bF$ is algebraically closed, $f$ has an eigenvalue $\lambda$.
    Then $f-\lambda{\rm id}_E$ is also an endomorphism of $E$ which is not invertible, and thus by Schur's lemma is zero.
    Therefore $f=\lambda{\rm id}_E$ as required.
    \qed

\eproof

Since $\hom_G(E,E)$ is one-dimensional for algebraically-closed $\bF$ we have that
$$ [V:E] = \frac{\dim\hom_G(V,E)}{\dim\hom_G(E,E)} = \dim\hom_G(V,E) $$

\subsection{The regular representation}

For a set $X$ and a field $\bF$, we define the vector space $\bF[X]$ to be the space of all formal linear combinations of elements of $X$.
For clarity, we will denote elements of $X$ in $\bF[X]$ by $\delta_x$ for $x\in X$.
That is, $\bF[X]=\set{\sum_i a_i\delta_{x_i}}[x_i\in X]$.
This is the free vector space over $X$.

\bdefn

    The {\emphcolor regular $G$-set} is simply $G$, with the action given by left-multiplication by $G$: $\rho(g)(h)=gh$.
    The {\emphcolor regular $G$-representation} is $\bF[G]$ with the action given by $\rho(g)(\delta_h)=\delta_{gh}$ (this defines a unique
    automorphism).

\edefn

Recall that morphisms out of $\bF[X]$ are determined uniquely by their image of $X$; $\hom(\bF[X],V)\cong\St(X,V)$.
This is an isomorphism of vector spaces.

Let $\St_G$ be the category of $G$-sets.

\blemm

    Let $X$ be a $G$-set and $V$ a $G$-representation.
    Then the isomorphism of vector spaces $\hom(\bF[X],V)\cong\St(X,V)$ restricts to an isomorphism of vector spaces
    $\hom_G(\bF[X],V)\cong\St_G(X,V)$.

\elemm

This is simple.
Recall that $\hom_G(\bF[X],V)\cong\hom(\bF[X],V)^G$ has a trivial representation structure.

Let $E$ be an irreducible $G$-representation, then
$$ [\bF[G]:E] = \frac{\dim\hom_G(\bF[G],E)}{\dim\hom_G(E,E)} = \frac{\dim\St_G(G,E)}{\dim\endo_G(E)} $$
$\St_G(G,E)$ is one-dimensional: for $f\in\St_G(G,E)$ we have that $f(g)=gf(1)$ so $\St_G(G,E)\cong E$ (by mapping $f$ to $f(1)$).
Thus
$$ [\bF[G]:E] = \frac{\dim E}{\dim\endo_G(E)} $$
In particular, if $\bF$ is algebraically closed then $\dim\endo_G(E)=1$ and so $[\bF[G]:E]=\dim E$.

\bcoro

    There are finitely many isomorphism classes of irreducible $G$-representations.

\ecoro

\bproof

    By above, every irreducible $G$-representation occurs in $\bF[G]$ $[\bF[G]:E]>0$ times.
    Since $\bF[G]$ is finite-dimensional, it has finitely many non-isomorphic subspaces, and thus there can only be finitely many irreducible
    $G$-representations.
    \qed

\eproof

Notice that in general, if $E_1,\dots,E_n$ are all irreducible subrepresentations of $V$ up to isomorphism.
Now, $V=E_1^{\oplus[V:E_1]}\oplus\cdots\oplus E_n^{[V:E_n]}$ and so $\dim V=\sum_i[V:E_i]\dim E_i$.
In particular if $E_1,\dots,E_n$ list all irreducible $G$-representations up to isomorphism,
$$ \dim\bF[G] = \sum_i[\bF[G]:E_i]\dim E_i $$
and if $\bF$ is algebraically closed, $[\bF[G]:E_i]=\dim E_i$, and so we get:

\bcoro

    Suppose that $\bF$ is algebraically closed.
    Let $E_1,\dots,E_n$ be all irreducible $G$-representations up to isomorphism.
    Then
    $$ \abs G = \sum_i(\dim E_i)^2 $$

\ecoro

\bexam

    Consider the group $G=S_3$ (permutations on $\set{0,1,2}$).
    Let $\bF$ be an algebraically closed field whose characteristic does not divide $\abs G=3!$, i.e.\ its characteristic is not $2$ or $3$.
    We have two irreducible one-dimensional representations of $G$: the trivial, and the other given by the sign character
    ${\rm sgn}\colon S_3\to\set{\pm1}$.
    We have already shown that the usual representation of $S_3$ on $\bF^3$ has an irreducible subrepresentation of vectors whose entries sum to zero.
    This subrepresentation has dimension two, and we see that
    $$ \abs{S_3} = 6 = 1 + 1 + 2^2 $$
    So these are all the irreducible $S_3$-representations, up to isomorphism

\eexam

\subsection{The group algebra}

A ring here has an identity, but may be non-commutative.

\bdefn

    A $\bF$-algebra is a ring $A$ which is also a $\bF$-vector space, such that multiplication $A\times A\to A$ is bilinear.

\edefn

Notice that if $A$ is a $\bF$-algebra, then we have a map $\bF\to A$ given by $c\mapsto c\cdot1$ ($1$ is the unit in $A$).
This map is injective, so we can embed $\bF$ in $A$.
In fact, an equivalent formulation of a $\bF$-algebra is a ring $A$ together with a ring homomorphism $\bF\to Z(A)$.
Indeed, this map ($c\mapsto c\cdot1$) is a ring homomorphism $\bF\to Z(A)$.
And given a ring homomorphism $\sigma\colon\bF\to Z(A)$, we can define $c\cdot a=\sigma(c)a$.
This clearly defines a vector space over $A$, and multiplication is bilinear.

We recall the definition of an $R$-module, and $R$-module-morphisms.

Note that if $A$ is a $\bF$-algebra and $M$ an $A$-module, then $M$ can be naturally given the structure of a $\bF$-vector space.
This is since $\bF$ embeds in $A$.

\bdefn

    The {\emphcolor group algebra} of $G$, denoted $\bF[G]$, is the vector space denoted as above with multiplication given by
    $\delta_g\delta_h=\delta_{gh}$.

\edefn

Notice that if $A$ is a $\bF$-algebra, then there is a natural bijection
$$ \alg_{\bF}(\bF[G],A) \cong \grp(G,A^\times) $$
(The left is morphisms of $\bF$-algebras, and the right is morphisms of groups.
$A^\times$ is the group of invertible elements of $A$.)
This is given by sending $f\colon\bF[G]\to A$ to the map $g\to f(\delta_g)$.
This is well-defined since $f(\delta_g)f(\delta_{g^{-1}})=f(\delta_1)=1$ and so $f(\delta_g)$ are all invertible.
This is also clearly a homomorphism: $f(\delta_{gh})=f(\delta_g\delta_h)=f(\delta_g)f(\delta_h)$.
Being a bijection and natural is also clear.

Note that $\bF$-algebra homomorphisms $A\to\endo(V)$ give rise to $A$-modules on $V$, and vice versa.
Indeed: given $\sigma\colon A\to\endo(V)$ define $a\cdot v=\sigma(a)(v)$.

\bcoro

    Let $V$ be an $\bF$-vector space, then there is a natural bijection between $G$-representations on $V$ and $\bF[G]$-modules
    on $V$.

\ecoro

\bproof

    $G$-representations on $V$ are homomorphisms $G\to\endo(V)^\times=\gl(V)$.
    $\bF[G]$-modules on $V$ are $\bF$-algebra homomorphisms $\bF[G]\to\endo(V)$.
    Then apply the previous remark.
    \qed

\eproof

\bprop

    Let $V,W$ be two $G$-representations, so also $\bF[G]$-modules.
    Then a linear morphism $T\colon V\to W$ is a morphism of $G$-representations if and only if it is a morphism of
    $\bF[G]$-modules.

\eprop

This is just definition chasing.

\subsection{The non-commutative Fourier transform}

As before, we assume that $G$ is a finite group whose size is divisible in $\bF$.

A division ring is one for which every nonzero element in the ring is invertible.
A division algebra is an algebra which is a division ring as a ring.

\blemm[title=Schur]

    Let $E$ be an irreducible $G$-representation.
    Then $\endo_G(E)$ is a division algebra.

\elemm

This is immediate since every non-zero endomorphism is an isomorphism.

\blemm

    Let $A$ be a finite-dimensional division $\bF$-algebra, then every subalgebra is also a division algebra.

\elemm

\bproof

    Let $B$ be a subalgebra of $A$, and let $b\in B$ be nonzero.
    Let $m\in\bF[x]$ be the minimal polynomial of the linear map $\ell_b\colon A\to A$ given by left-multiplication by $b$.
    Note then that $m(\ell_b)=0$ and so $m(\ell_b1)=m(b)=0$.
    We write $m(x)=xn(x)+c$, and we note that $c$ must be nonzero as since $\ell_b$ is invertible (because $b$ has an inverse),
    $\ell_b n(\ell_b)=0$ would imply $n(\ell_b)=0$, contradicting $m$'s minimality.
    Since $c$ is nonzero, it has an inverse in $A$.
    So $m(b)=0$ means $bn(b)+c=0$, so $bn(b)=-c$ and so $b^{-1}=-c^{-1}n(b)$.
    This is in $B$, as required.
    \qed

\eproof

\bprop

    Suppose $\bF$ is algebraically closed, and let $A$ be a finite-dimensional division $\bF$-algebra.
    Then $A=\bF$.

\eprop

\bproof

    Let $a\in A$, and take $B$ be the subalgebra spanned by $a$: $B=\lspan\set{a^n}_{n\in\bN}$.
    By the above lemma, $B$ is itself a division algebra.
    But $B$ is commutative, and thus is a field, moreso a finite field extension of $\bF$ (as it has finite dimension).
    But $\bF$ is algebraically closed, so $B=\bF$ (since all extensions of an algebraically closed field are transcendental).
    Thus $a\in\bF$, so $A=\bF$.
    \qed

\eproof

Note that we have stumbled upon another proof of Schur: $\endo_G(E)$ is a division algebra, and since it is finite-dimensional, if $\bF$ is
algebraically closed it must be $\bF$.

\bnote

    Modules over division rings behave similarly to modules over fields (vector spaces).
    For example, the proof that vector spaces have unique (up to size) bases is the same for modules over division rings.
    Moreso, linearly independent sets can be extended to bases.

    If $D$ is a $\bF$-algebra, then we have that $\dim_{\bF}V=\dim_D V\cdot\dim_{\bF}D$ (the proof is similar as that for field extensions).

\enote

Let $E$ be an irreducible $G$-representation, and let $D_E=\endo_G(E)$ (it is a division algebra by Schur).
Then we have a natural $\bF$-algebra morphism
$$ \F_E\colon\bF[G]\to\endo_{D_E}(E) $$
given by $\F_E(g)(x)=gx$.
This is well-defined: $\F_E(g)$ is an endomorphism of $E$ over $D_E$ since $\F_E(g)(f\cdot e)=\F_E(g)(fe)=gfe$, and since $f$ is a $G$-morphism,
this is equal to $fge=f\cdot\F_E(g)(e)$.

Let $E_1,\dots,E_n$ list all the non-isomorphic irreducible representations of $G$.
For each $E_i$ we have $\F_{E_i}\colon\bF[G]\to\endo_{D_{E_i}}(E_i)$, and thus we can gather them into one large $\bF$-algebra morphism
$$ \F\colon\bF[G] \to \prod_{i=1}^n\endo_{D_{E_i}}(E_i) $$
This is called the {\bf non-commutative Fourier transform} of $G$.

Let $E$ and $F$ be finitely-generated modules over a division $\bF$-algebra $D$ also of finite dimension.
Then
$$ \dim_{\bF}\hom_D(E,F) = \dim_DE\cdot\dim_{\bF}F = \frac{\dim_{\bF}E\cdot\dim_{\bF}F}{\dim_{\bF}D} $$
Indeed, let $v_1,\dots,v_n$ be a basis of $E$ over $D$, then $\hom_D(E,F)$ is isomorphic to $F^r$ as $\bF$-vector spaces: send $T$ to
$(Tv_1,\dots,Tv_n)$.
The rest of the equality follows from towering dimensions.

\bprop[title={Artin-Wedderburn, special case}]

    $\F$ is an isomorphism of $\bF$-algebras.

\eprop

\bproof

    We show that $\F$ is an injection and that domain and codomain of $\F$ have equal dimension.
    To show that $\F$ is injective, suppose $\F(a)=0$, so $a$ acts trivially on each $E_i$.
    So $a$ acts trivially on every finite-dimensional representation (as the sum of $E_i$s), in particular it must act trivially on $\bF[G]$.
    This is only possible if $a=0$ (since $a=a\cdot1=0$).

    The dimension of the domain is $\abs G$.
    Let $e_i=\dim_{\bF}E_i$ and $d_i=\dim_{\bF}D_{E_i}$.
    By above, we have that
    $$ \dim_{\bF}\endo_{D_{E_i}}E_i=\frac{(\dim_{\bF}E_i)^2}{\dim_{\bF}D_{E_i}}=e_i^2/d_i $$
    And so we need to show that
    $$ \abs G = \sum_{i=1}^n\frac{e_i^2}{d_i} $$
    And indeed, recall that
    $$ \abs G = \sum_{i=1}^n[\bF[G]:E_i]\cdot\dim_{\bF}E_i = \sum_{i=1}^n\frac{e_i^2}{d_i} $$
    Since $[\bF[G]:E_i]=e_i/d_i$ since we showed
    $$ [\bF[G]:E] = \frac{\dim\hom_G(\bF[G],E)}{\dim\endo_G(E)} = \frac{\dim_{\bF}E}{\dim_{\bF}\endo_G(E)} \qed $$

\eproof

In the case that $\bF$ is algebraically closed, we have that $D_{E_i}=\bF$ and so $\F$ forms an isomorphism
$$ \F\colon\bF[G]\to\prod_{i=1}^n\endo_{\bF}(E_i) $$
So we can consider the group algebra to be the product of matrix alegbras.

\bexam

    We can identify $\bF[G]$ with $\St(G,\bF)$ (map $f\colon G\to\bF$ to $\sum_{g\in G}f(g)\delta_g$).
    Now notice that the center $Z(\bF[G])$ are all functions $f\colon G\to\bF$ for which for all $h\colon G\to\bF$:
    $$ \parens{\sum_{g\in G}f(g)\delta_g}\parens{\sum_{g\in G}h(g)\delta_g} = \sum_{g\in G}\parens{\sum_{ab=g}f(a)h(b)}\delta_g $$
    is equal to
    $$ \sum_{g\in G}\parens{\sum_{ab=g}h(a)f(b)}\delta_g $$
    That is, $\sum_{ab=g}f(a)h(b)=\sum_{ab=g}f(b)h(a)$.

    In particular, if we let $h_x$ be the indicator of $x\in G$: $h_x(x)=1$ and $h_x(y)=0$, then we see that for $g\in G$,
    $$ \sum_{ab=g}f(a)h_x(b) = f(ax^{-1}) = f(x^{-1}a) = \sum_{ab=g}f(b)h_x(a) $$
    So $f(gh)=f(hg)$ for all $g,h\in G$.
    Equivalently $f(hgh^{-1})=f(g)$, i.e.\ $f$ is a {\emphcolor class function}: it is identical on conjugacy classes.

    This is also sufficient:
    $$ \sum_{ab=g}f(a)h(b) = \sum_{b\in G}f(gb^{-1})h(b) = \sum_{a\in G}f(a^{-1}g)h(a) = \sum_{ab=g}f(b)h(a) $$
    So $Z(\bF[G])$ can be identified with the set of class functions: $\St(G,\bF)^\cl$.

\eexam

\bexam

    Let $D$ be a division ring and $V$ a finite-dimensional $D$-module.
    Then the morphism of rings $Z(D)\to Z(\endo_D(V))$, which maps $z$ to $v\mapsto zv$, is an isomorphism.
    Let $n=\dim V$, then $\endo_D(V)$ is isomorphic to ${\rm Mat}_n(D^{\sf op})$ ($D^{\sf op}$ being the opposite ring of $D$).
    Indeed, let $B=\set{v_1,\dots,v_n}$ form a basis for $V$, then mapping $T\in\endo_D(v)$ to the representation matrix $[T]_B$ is an isomorphism.
    Note that this indeed must be in the opposite ring since
    $$ [T]_B[\sum_id_iv_i]_B = \sum_i [Tv_i]_B\cdot^{\sf op}d_i = \sum_i d_i[Tv_i]_B = [T(\sum_i d_iv_i)]_B $$

    A matrix in the center of ${\rm Mat}_n(D^{\sf op})$ must be a scalar matrix, whose entries are in $Z(D^{\sf op})=Z(D)$.
    So the center of $Z(\endo_DV)$ is isomorphic to $Z(D)$.

\eexam

\bcoro[title=Basic formula]

    Let $\bF$ be algebraically closed, then the cardinality of the set of isomorphism classes of irreducible $G$-representations is equal to
    the cardinality of the set of conjugacy classes of $G$.

\ecoro

\bproof

    By the above exercise, the center of $Z(\bF[G])$ is isomorphic to $\St(G,\bF)^\cl$, the set of class functions.
    The dimension of the codomain of $\F$ is $\prod_{i=1}^n Z(\endo_{D_{E_i}}(E_i))$, which by the above exercise is isomorphic to
    $\prod_{i=1}^n Z(D_{E_i})$.
    Since $\bF$ is algebraically closed, $D_{E_i}\cong\bF$, and so this is isomorphic to $\bF^n$.
    Thus we have that $n=\dim\St(G,\bF)^\cl$, where $n$ is the number of irreducible $G$-representations.

    Let the conjugacy classes of $G$ be $[g_1],\dots,[g_m]$.
    Define $f_i\colon G\to\bF$ to be the indicator of $[g_i]$.
    This forms a basis of $\St(G,\bF)^\cl$ and as such $\dim\St(G,\bF)^\cl$ is equal to the number of conjugacy classes of $G$, thus completing our
    proof.
    \qed

\eproof

In total, let $G$ be a finite group and $\bF$ an algebraically closed field whose characteristic does not divide $\abs G$.
Let $n$ be the number of conjugacy classes of $G$, and $d_1,\dots,d_n$ be the dimensions of the irreducible representations of $G$.
Then
$$ \abs G = \sum_{i=1}^n d_i^2 $$

\bexam

    The following example will not be used, but is an example of a result of the above investigation.
    Define the {\emphcolor zeta function} of the group $G$ to be
    $$ \zeta_G(s) = \sum_{i=1}^n d_i^{-s} $$
    Note that $\zeta_G(0)$ is equal to the number of conjugacy classes in $G$, and $\zeta_G(-2)$ is equal to the number of elements in $G$.
    In general one has
    $$ \zeta_G(-2+2n) = \frac1{\abs G^{2n-1}}\abs{c_n^{-1}(1)} $$
    where $c_n\colon G^{2n}\to G$ is given by $c_n(x_1,y_1,\dots,x_n,y_n)=[x_1,y_1]\cdots[x_n,y_n]$.
    (Note that $\abs{c_n^{-1}(1)}$ is the cardinality of the fiber of $1$.)

\eexam

Let $E$ be an irreducible $G$-representation.
Then there is a unique element $e_E\in\bF[G]$ which acts as the identity on irreducible subrepresentations isomorphic to $E$ and as $0$ on
irreducible subrepresentations not isomorphic to $E$.
Indeed, such an element must satisfy $\F(e_E)=(T_i)_i$ where $T_i={\rm id}_{E_i}$ when $E_i\cong E$ and $0$ otherwise.
Due to $\F$'s bijectivity a unique element must exist.

Notice that $e_E\in Z(\bF[G])$ and $e_E^2=e_E$ (since $(T_i)_i$ satisfies this).
And the action of $e_E$ on any finite-dimensional $G$-representation $V$ is the unique $G$-morphic projection onto the isotypic component $V_E$.
So if we want a formula for this projection, we can equivalently find a formula for $e_E$: scalars $c_g$ such that $e_E=\sum_gc_g\delta_g$.
We shall return to this.

\subsection{The commutative Fourier transform}

In this section we will take $G$ to be a finite Abelian group, and $\bF$ to be algebraically closed where $\abs G\in\bF^\times$.

\bprop

    All irreducible representations of $G$ are one-dimensional.

\eprop

\bproof

    Let $\rho\colon G\to\gl(E)$ be a representation.
    Then notice that $\rho(g)\colon E\to E$ is a $G$-morphism: $\rho(g)(hv)=\rho(gh)(v)=\rho(hg)(v)=h\rho(g)v$ (we use both juxtaposition and $\rho$ here
    to denote the same representation).
    So $\rho(g)\in\endo(E)$, and by Schur (since $\bF$ is algebraically closed), $\rho(g)$ is scalar multiplication.
    This means that any $1$-dimensional space is a representation, and as such all irreducible representations must be $1$-dimensional.

\eproof

Define $\ch_\bF(G)$ to be the set of characters of $G$, i.e.\ the set of group morphisms $G\to\bF^\times$.
Recall that every character $\chi\in\ch_\bF(G)$ induces a one-dimensional representation (well, it simply {\it is\/} a representation since $\gl(\bF)=\bF^\times$, but I digress).
We denote the representation induced by $\chi$ as $\bF_\chi$.

\bcoro

    The family $(\bF_\chi)_{\chi\in\ch_\bF(G)}$ lists all the irreducible representations of $G$.
    (That is to say every irreducible representation of $G$ is isomorphic to some $\bF_\chi$, and every character induces a unique representation.)

\ecoro

Clearly $(\bF_\chi)_\chi$ list all the one-dimensional (and thus irreducible) representations of $G$.
And distinct characters induce non-isomorphic representations: if $f\colon\bF_\chi\to\bF_\mu$ is an isomorphism then $f(\chi(g)1)=\chi(g)f(1)$ by linearity, while
$f(\chi(g)1)=\mu(g)f(1)$ by equivariance ($G$-morphism).
Thus $\chi(g)=\mu(g)$.

Note that
$$ \abs G = \sum_{\chi\in\ch_\bF(G)}(\dim\bF_\chi)^2 = \abs{\ch_\bF(G)} $$

\bexam

    If $\bF$ is not algebraically closed, this is not true.
    For instance, let $G=\mu_3$ be the group of third roots of unity in $\bC$: $\mu_3=\set{1,\omega_3,\omega_3^2}=\gen{\omega_3}$, and let $\bF=\bR$.
    Then $\bC$ is a two-dimensional irreducible representation of $\mu_3$ over $\bR$, where $\mu_3$ acts on $\bC$ by multiplication.

\eexam

Now note that the Fourier transform in the Abelian, algebraically-closed case reduces to
$$ \F\colon\bF[G] \to \prod_{\chi\in\ch_\bF(G)}\endo_\bF(\bF_\chi) = \prod_{\chi\in\ch_\bF(G)}\bF = \St(\ch_\bF(G),\bF) $$
The algebra structure of the right side is given by pointwise multiplication.
Originally, $\F$ sent $\delta_g$ to the action of multiplication by $g$ on each $\bF_\chi$.
Which means that $\F$ sent $\delta_g$ to $c\mapsto\chi(g)c$.
In the Abelian case, this means that $\F$ sends $\delta_g$ to the function $\F(g)\colon\ch_\bF(G)\to\bF$ which maps $\chi$ to $\chi(g)$.
So
$$ \F\parens{\sum_{g\in G}c_gg}(\chi) = \sum_{g\in G}c_g\chi(g) $$

Naturally, we can ask ourselves to find an inverse for $\F$.
Since $\St(\ch_\bF(G),\bF)$ is generated by $\set{\delta_\chi}_{\chi\in\ch_\bF(G)}$, where $\delta_\chi(\chi)=1$ and $\delta_\chi(\mu)=0$ for $\chi\neq\mu$,
it is sufficient to find the inverse image of $\delta_\chi$.
So we want to find $e_\chi=\sum_{g\in G}c_g\delta_g$ such that $\F(e_\chi)=\delta_\chi$.
That is,
$$ \F\parens{\sum_{g\in G}c_g\delta_g} = \delta_\chi $$
So for $\mu\in\ch_\bF(G)$, we need
$$ \F(e_\chi)(\mu) = \sum_{g\in G}c_g\mu(g) = \delta_\chi(\mu) $$
In particular, we need $\sum_{g\in G}c_g\chi(g)=1$.
So an initial guess (and the correct one) will be $c_g=\chi(g)^{-1}/\abs G$, i.e.\ $e_\chi=\frac1{\abs G}\sum_{g\in G}\chi(g)^{-1}\delta_g$.

And we see for $\mu\neq\chi$:
$$ \F(e_\chi)(\mu) = \frac1{\abs G}\sum_{g\in G}\chi(g)^{-1}\mu(g) = \frac1{\abs G}\sum_{g\in G}(\mu\chi^{-1})(g) $$
Setting $\theta=\mu\chi^{-1}$, since $\mu\neq\chi$ there is some $g_0\in G$ such that $\theta(g_0)\neq1$.
Then
$$ \sum_{g\in G}\theta(g) = \sum_{g\in G}\theta(g_0g) = \theta(g_0)\sum_{g\in G}\theta(g) $$
Solving for this gives $\sum_{g\in G}\theta(g)=0$, as required.

Now notice that
$$ \F(\delta_g) = (\chi(g))_{\chi\in\ch_\bF(G)} = \sum_{\chi\in\ch_\bF(G)}\chi(g)\delta_\chi $$
Applying the inverse Fourier transform, we get
$$ \delta_g = \sum_{\chi\in\ch_\bF(G)}\chi(g)e_\chi $$

Now, $\set{e_\chi}_\chi$ forms a basis for $\bF[G]$ (since $\F(e_\chi)=\delta_\chi$ forms a basis for $\St(\ch_\bF(G),\bF)$).
This gives us two bases for $\bF[G]$:
\benum
    \item The {\it geometric basis} $\set{\delta_g}_{g\in G}$, and
    \item The {\it spectral basis} $\set{e_\chi}_{\chi\in\ch_\bF(G)}$.
\eenum
The change-of-basis matrices are
$$ \eqalign{
    e_\chi &= \sum_{g\in G}\frac1{\abs G}\chi(g)^{-1}\delta_g\cr
    \delta_g &= \sum_{\chi\in\ch_\bF(G)}\chi(g)e_\chi
} $$

Notice that characters are elements of $\bF[G]$: $\chi=\sum_{g\in G}\chi(g)\delta_g$.
Quickly, this gives us $\chi=\abs Ge_{\chi^{-1}}$ (where $\chi^{-1}$ is the multiplicative inverse of $\chi$).
So
$$ \delta_g = \sum_{\chi\in\ch_\bF(G)}\chi(g)e_\chi = \frac1{\abs G}\sum_{\chi\in\ch_\bF(G)}\chi(g)e_{\chi^{-1}} = \frac1{\abs G}\sum_{\chi\in\ch_\bF(G)}\chi(g)^{-1}e_\chi $$

We now provide an application of the Fourier transform for finite Abelian groups:

\bthrm[title=Dirichlet]

    Let $d\in\bZ_{\geq1}$ and $a\in\bZ$ be relatively prime.
    Then there exist infinitely many primes $p$ such that $p\equiv a\pmod d$.

\ethrm

\bproof

    Consider the group $(\bZ/d\bZ)^\times$ (the Euler group of numbers invertible modulo $d$).
    Given a function $f\in\St((\bZ/d\bZ)^\times,\bC)$, we can extend it to a function on all of $\bZ/d\bZ$ by setting it to be $0$ on non-invertible elements.
    Then consider
    $$ M_f(s) = \sum_{p\ \mathrm{prime}}\frac{f([p]_d)}{p^s} $$
    where $[\bullet]_d\colon\bZ\to\bZ/d\bZ$ is the canonical projection.
    We assume that $s\in\bR$.
    Note that when $s>1$, the series $M_f(s)$ is bound by $\sum_n n^{-s}$ and therefore converges.

    Let $\delta_a\colon(\bZ/d\bZ)^\times\to\bC$ be the Kronecker delta for $[a]_d$ (i.e.\ $\delta_a([a]_d)=1$ and $0$ everywhere else), then notice that
    $$ M_f(s) = \sum_{p\ \mathrm{prime}}\frac{\delta_a([p]_d)}{p^s} $$
    Now if $M_f$ is unbounded from $1$ on the right, then there must be infinitely many non-zero terms in the series, meaning infinitely many primes where $\delta_a([p]_d)=1$,
    i.e.\ $p\equiv a\pmod d$.
    
    We will prove this with help from the following proposition.

\eproof

\bprop

    Let $\chi\in\ch_\bC((\bZ/d\bZ)^\times)$ be a character not equal to $1$.
    Then $\abs{M_\chi(s)}$ is bounded as $s$ tends to $1$ from the right.
    If $\chi=1$, then $M_\chi(s)$ is unbounded as $s$ tends to $1$ from the right.

\eprop

If we prove this proposition, then we have proven our theorem.
Recall that using $\ch_\bC((\bZ/d\bZ)^\times)$ as a basis for our group algebra, we have as before
$$ \delta_a = \frac1{\abs{(\bZ/d\bZ)^\times}}\sum_{\chi\in\ch_\bC((\bZ/d\bZ)^\times)}\chi(a)^{-1}\chi $$
Thus in this sum, the trivial character $1$ appears with non-zero coefficient.
Now,
$$ M_{\delta_a} = \frac1{\abs{(\bZ/d\bZ)^\times}}\sum_{\chi\in\ch_\bC((\bZ/d\bZ)^\times)}\chi(a)^{-1}M_\chi $$
so $M_{\delta_a}$ is the sum of finitely many bounded functions ($M_\chi$ for $\chi\neq1$), and one unbounded function ($M_1$).
Thus $M_{\delta_a}$ is unbounded, as required.

\bproof

    For $x\in\set{z\in\bC}[\abs z<1]$, we have that $-\log(1-x)=\sum_mx^m/m$.
    Now let us assume that $\abs f\leq1$, and let us define $a_p(s)=\frac{f([p]_d)}{p^s}$, so that $M_f(s)=\sum_pa_p(s)$.
    Let us consider
    $$ \sum_{p\ \mathrm{prime}}\abs{-\log(1-a_p(s))-a_p(s)} = \sum_{p\ \mathrm{prime}}\abs{\sum_{m=2}^\infty\frac{a_p(s)^m}m} \leq \sum_p\sum_m\frac1{mp^{sm}} $$
    We consider $s>1$, and as such
    $$ \leq \sum_p\sum_m\frac1{p^m} = \sum_p\frac1{p^2}\frac1{1-1/p} \leq 2\sum_p\frac1{p^2} \leq 2\sum_n\frac1{n^2} $$

    So we can deduce that if $\abs f\leq1$, then $M_f(s)$ is (un)bounded as $s$ tends to $1$ from the right iff
    $$ \ell_f(s) = \sum_{p\ \mathrm{prime}}-\log\parens{1-\frac{f([p]_d)}{p^s}} $$
    is (un)bounded.
    Exponentiating, we can equivalently see if
    $$ L_f(s) = \prod_{p\ \mathrm{prime}}\frac1{1-\frac{f([p]_d)}{p^s}} $$
    is bounded (or unbounded or zero).

    Notice that for a finite $N$, let $X_N$ be the set of integers whose prime factors are $\leq N$.
    Then notice that
    $$ \prod_{p\leq N}\frac1{1-\chi([p]_d)p^{-s}} = \sum_{n\in X_N}\frac{\chi([n]_d)}{n^s} $$
    We prove this by induction.

    For brevity, let us define
    $$ L_f(s,N) = \prod_{p\leq N}\frac1{1-f([p]_d)p^{-s}} $$
    So we claim that
    $$ L_\chi(s,N) = \sum_{n\in X_N}\frac{\chi([n]_d)}{n^s} $$

    For the case $N=1$, we have that $X_N=\set1$ and the product is empty ($1$).
    We will write $\chi(\bullet)$ for $\chi([\bullet]_d)$ for brevity; we get $1=\chi(1)$ which is indeed true (since $\chi$ is a character).
    Now notice that for $N+1$ not prime, $X_{N+1}=X_N$ and the product is the same as well.
    So the only interesting case is when $N=P-1$, in which case $X_P=\bigdcup_{k=0}^\infty P^kX_{P-1}$.
    We get
    $$ \sum_{n\in X_P}\frac{\chi(n)}{n^s} = \sum_{k=0}^\infty\sum_{n\in X_{P-1}}\frac{\chi(P^kn)}{(P^kn)^s} $$
    by $\chi$'s multiplicity, this is equal to
    $$ \sum_{k=0}^\infty\frac{\chi(P)^k}{P^{sk}}\sum_{n\in X_{P-1}}\frac{\chi(n)}{n^s} = \sum_{k=0}^\infty\frac{\chi(P)^k}{P^{sk}}L_\chi(s,P-1) $$
    This is a geometric series, whose sum is
    $$ L_\chi(s,P-1) \cdot \frac1{1-\chi(P)P^{-s}} = \prod_{p\leq P-1}\frac1{1-\chi(p)p^{-s}}\cdot\frac1{1-\chi(P)P^{-s}} = L_\chi(s,P) $$
    as required.

    As $N\to\infty$, we notice that $X_N\to\bZ_{\geq1}$ and the set of primes $\leq N$ is all of the primes, so we have our desired result.

    Note that this hinges on $\abs{\chi(n)}\leq1$.
    This is true since $n$ (for $n\in(\bZ/d\bZ)^\times$) has finite order, so $\chi(n)$ must too.
    The only elements of $\bC^\times$ with finite order are the roots of unity, so $\abs{\chi(n)}=1$.

    So we have shown
    $$ L_\chi(s) = \sum_{n=1}^\infty\frac{\chi([n]_d)}{n^s} $$
    Now notice that $L_1(s)=\sum_n\frac1{n^s}$.
    This clearly tends to $+\infty$ as $s\to1^+$, so we have our result for $\chi=1$.

    For $\chi\neq1$, we note that $L_\chi(s)\to L_\chi(1)$ for $s\to1^+$.
    $L_\chi(1)\neq0$ (this is a technical point, we will not show it here).
    And thus we have completed the proof (modulo some technicalities arising from analysis).
    \qed

\eproof

